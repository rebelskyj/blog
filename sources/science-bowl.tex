\documentclass[12pt]{article}[titlepage]
\newcommand{\say}[1]{``#1''}
\newcommand{\nsay}[1]{`#1'}
\usepackage{endnotes}
\newcommand{\B}{\backslash{}}
\renewcommand{\,}{\textsuperscript{,}}
\usepackage{setspace}
\usepackage{tipa}
\usepackage{hyperref}
\begin{document}
\doublespacing
\section{\href{science-bowl.html}{Science Bowl}}
First Published: 2022 February 5

\section{Draft 1}
Today I had the wonderful opportunity to volunteer with the Wisconsin Regional Middle School Science Bowl\footnote{Who knows if that's actually what it is called, but that's what I remember it as}.
I really love volunteering for this event, and loved doing so last year.
In part, it's fascinating for me to realize how much science\footnote{almost exclusively biology based} I've forgotten since I was younger and took biology adjacent classes.
In part, it reminds me of when I was in high school, and the incredibly fun times I had participating in the science bowl and preparing for it with friends.
In part, I really love the opportunity to watch young scientists develop their love for science.
In part I just love competitions and trivia, so competitive trivia is incredible.
And finally, in part I just like volunteering.

All this leads to the perfect storm of a morning and afternoon spent reading between 18 and 36 questions a round to each of the participating teams.
My fellow roommates\footnote{volunteers in the same room as me, a term I have invented here} were both members of my research group, which is really exciting.
I found, in mentioning that I blog, that if you are somewhat careful about your link clicking, you still can find my blog from my home site.

You just have to go to scores, click the \say{maintained by rebelskyj} link, and then it takes you to my personal GitHub account, which is where all of these posts are stored.\footnote{It is a little interesting to me that I've shifted from calling them musings, like my inspiration, into posts, like the standard usage is.}
Anyways, that's a side note.

All this to say, I loved volunteering for the science bowl, and I'm very excited to do so again next year!
\end{document}