\documentclass[12pt]{article}[titlepage]
\newcommand{\say}[1]{``#1''}
\newcommand{\nsay}[1]{`#1'}
\usepackage{endnotes}
\newcommand{\1}{\={a}}
\newcommand{\2}{\={e}}
\newcommand{\3}{\={\i}}
\newcommand{\4}{\=o}
\newcommand{\5}{\=u}
\newcommand{\6}{\={A}}
\newcommand{\B}{\backslash{}}
\renewcommand{\,}{\textsuperscript{,}}
\usepackage{setspace}
\usepackage{tipa}
\usepackage{hyperref}
\begin{document}
\doublespacing
\section{\href{christian-writing.html}{On Christian Writing}}
First Published: 2023 November 2
\section{Draft 1}
\href{nanowrimo-5.html}{Yesterday} I talked about how my goal for this NaNoWriMo is writing something more explicitly Christian.
I said that I did not have the time or energy to discuss what that meant in that post, and I do stand by that.
However, as I tossed and turned in my bed last night, I thought about what it means to have Christian writing.

Of course, like anything else, the concept exists in a spectrum.
At one end is something like the Golden Compass, which is a fiction book meant to mock and denigrate the Church.\footnote{I think? I'll be honest, I heard that as a child and saw a few things which corroborated that story and was never interested in researching further.
I'm fully willing to believe that it is not true.}
At the other end of the spectrum\footnote{in my own world view, and compressing any orthogonal things into one single projected axis} is something like Joseph Cardinal Ratzinger's \textit{Jesus of Nazareth}, which is wholly a reflection on the Lord, more a work of theology than of creative writing.\footnote{Ok so I'm absolutely realizing that I should have two axes (hmmmm, maybe not. I don't actually know if that's the case}

Closer to the areligious side would be the works of Ernest Hemmingway, a devout\footnote{maybe faithful? I'm not sure which word is better for describing someone who was self professed a poor Catholic but who still believed strongly and deeply.} Catholic who actively tried to avoid being labeled as such in his writings.
My scale may not be absolute, because I don't know how to rank Tolkein, who created a wold which was completely in line with the Catholic faith, though not meant to be seen allegorically, against Lewis, who wrote didactic tracts on faith with the slimmest metaphor.

Of course, there are plenty of non Christian fictions written by non Christians.
As someone who is, however, I wonder about how it should be reflected in my writing.
I don't think that, at this phase of my life, at least, I'm called to be writing pure theology.
Certainly I'm not well enough read in the theology or formed in my faith\footnote{despite framing like this, I'll be the first to say that the two are, though not mutually inclusive, at least fairly well linked. It's not two distinct axes, but the two don't overlap exactly} to feel as though I have things that need to be said to the world of theologians.

However, I do also have a platform.
As of this morning, I have almost twelve hundred followers on my webnovel, and my chapters are read an average of almost 4000 times.
With an audience like that, I have to wonder if there's something I can do to spread the Gospel, if only covertly.
Certainly I don't want to suddenly shift the tone of my book to something explicitly religious, for two reasons.

First, I don't really love didactic Christian writing as a genre.
There are books I've read that I really enjoy, but I find that I tend to get too caught up in where the metaphor falls apart.
I don't feel confident enough in my ability to portray nuance to where I think that I could write something without leading people astray.

I also feel like didactic Christian writing tends to only stand when read with that intention.\footnote{Hmm that's not well phrased, let's see if I can say that better}
That is,\footnote{the two words that every author afraid of not making sense loves} while I feel like a work of art tends to stand on its own, even when divorced of its context, I find that a lot of didactic writing requires people to be seeking out a book that will lecture them in order to be enjoyable.

That leads to the second concern I have, which is that a blatantly Catholic piece of writing is not what the readers of my book have sought out.
Now, there is an argument I respect that people do not always know what they want, but I don't know if it really holds here.
After all, there is almost no cost to people dropping my story if they do not like it, and I don't know if the morals I'm espousing in the book are particularly non-Christian.

I guess that question is certainly one that I have.
To what extent does a work I write become Christian and what are the important parts of a Christian fiction to me?\footnote{if you couldn't tell, I'm using this posting as a way for me to consider these questions, not because I already have the answers and I'm leading you to them as readers.
It would be funny if I, the person who just said that I dislike didactic writing, would have done so, and I won't say that I never do that.
Here, though, I'm really exploring the thought as I type. (One side effect of learning to type without self editing is that thoughts become much less filtered as you type them.
In the book I'm writing, this often manifests in new plot threads being picked up, often in ways that I never could or would have considered.
Here it means that I'm approaching the question through the lens of each word I type, which is a weird thing to realize)}
There's something to be said for the idea that a Christian book can be as much about the worldview it espouses as any particular theological point.

In that regard, I think that my web novel is at least somewhat Christian in its view of the world.
Violence is not something that the main character idolizes, and his family is filled with people who care deeply for each other.
Man made institutions are ultimately fallible, but people by and large are good and motivated by a desire to see the world improve.
Without adding a messianic character, I suppose that it's hard to get too much more Christian, especially given that Christ is\footnote{shockingly enough,} central to the Christian faith.

I don't know if this musing did what I wanted it to.
I think that I wanted to find a way to write in a way that I considered more Christian, despite never really having considered what that would look like.
Instead, I think that I'm walking away from this posting with the idea that I should just be more intentional about the morals I'm espousing implicitly and explicitly in my writing.
I don't think that I'm good enough at writing or thinking to effectively weave in allegory yet, though that is something I should consider.\footnote{I suppose I'm trying some allegory in the NaNo book I'm doing.}

Daily Reflection:
\begin{itemize}
\item Did I write 1700 words for NaNoWriMo? Yes! I did a writing session with a friend, and it was really fun. I think I ended up writing close to 2000 words in that hour, which is near what my maximum WPH has ever been.\footnote{which is funny when you consider that my sprint writing pace can often meet or exceed 60 wpm, implying that I'm writing for less than half of the hour.}1726/1700. I continue\footnote{maybe the wrong word for day two, but c'est la vie (la vie)} to grow ahead of schedule.
\item Did I write a chapter of Jeb? I did! I finished it and sent it to my beta reader before finishing this post.
Interestingly, the writing of the chapter was far easier than it felt yesterday.
I guess there's something to be said in consistency, especially in terms of making something which is not inherently difficult seem easier.
\item Did I blog? I did! I thought about the post over the day, which was good, and wow we're now on a three day streak! That's super cool and I'm excited to continue it further.
\item Did I stretch? Unlike last night where I forgot, today I was too rushed before leaving home\footnote{see the whole doing a writing session with a friend before work}, and I wore dress clothes\footnote{because I was singing at a mass tonight}, so couldn't really stretch during the day.
As with last night, I am again going to stretch after posting this.
\item Am I doing better at prayer than a rushed and thoughtless rosary? I did an Angelus again today around noon, and I went to mass.
I also went to mass last night, which I forget if I mentioned or not.
I also tried to focus on the rosary last night, but did not do a great job.
\item Am I doing a good job writing letters to friends?
I thought about one of the letters I want to write.\footnote{A year ago today, a friend invited me to go pray with her.
She's one of the people to whom I want to write a letter, and so it seemed somehow relevant}
I did not, however, write the letter or in any other demonstrable way make progress on my goal of writing more letters.
I didn't\footnote{It will never not be strange to me that didn't and did not have such different connotations} know if I would be able to today yesterday, though, so I'm ok with this.
\end{itemize}

I\end{document}