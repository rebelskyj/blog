\documentclass[12pt]{article}[titlepage]
\newcommand{\say}[1]{``#1''}
\newcommand{\nsay}[1]{`#1'}
\usepackage{endnotes}
\newcommand{\1}{\={a}}
\newcommand{\2}{\={e}}
\newcommand{\3}{\={\i}}
\newcommand{\4}{\=o}
\newcommand{\5}{\=u}
\newcommand{\6}{\={A}}
\newcommand{\B}{\backslash{}}
\renewcommand{\,}{\textsuperscript{,}}
\usepackage{setspace}
\usepackage{tipa}
\usepackage{hyperref}
\begin{document}
\doublespacing
\section{\href{losing-marbles.html}{Losing My Marbles}}
First Published: 2022 January 21

\section{Draft 1}
Sometimes there are perfect storms in my life, where multiple, seemingly unrelated events conspire to cause something to occur to my mind.
Today I got to experience one again!
As always, I'm very excited by the concept of metacognition.

First, the background.
I'm reading a book called \say{Don't Teach Coding (Until You Read This Book)}, a book about teaching computer science targeted to secondary and below teachers, many of whom do not have formal training in computer science themselves.
Rather than being a manual of how to learn Java\footnote{or C or Lisp or any other language}, the book\footnote{so far, I'm on page 166} instead takes an incredibly high-level approach to computer science.
It begins by defining computer science less as bits on the page and more as a formal language.
Without continuing this digression, one important topic it mentions is the concept of extended memory.

Extended memory is a concept which more or less says that when you need to look at your phone to remember someone's phone number, it's not that you've forgotten it, it's that your brain is using your phone as an external storage device for memory.
I really like it as a concept, both because it appears consistent with scientific findings\footnote{I think, I refuse to find citations}, but mostly because it is anti-Ludditical.
One extended memory I've been using since I started this blog\footnote{in addition to this blog} is a series of journals.

If I remember correctly, I'm now on journal four, which was started November 24, 2019.\footnote{which you may notice is after I abandoned this blog but not by much relative to now}
Since switching many of my note-taking and artistic concepts to other media, I've struggled to find a use for the journal, which is a large reason why it's lasted so long.
As I've mentioned \href{living-scheduled.html}{a few} \href{thinking-about-life.html}{other times}, I've recently started bullet-journaling lite.
It's been really helpful when I decide to do it.

But, what happens when\footnote{as happened today} I lose the journal itself.
In searching for it, I realized that I would be out more than the pages of writings and art that I've made, but also my ability to keep organized.
I don't really have a good solution to that, but it made me think about my reliance on this medium.
Hopefully over time I will need it less, but I'm not really thinking that's likely.
\end{document}