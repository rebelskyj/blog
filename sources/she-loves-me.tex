\documentclass[12pt]{article}[titlepage]
\newcommand{\say}[1]{``#1''}
\newcommand{\nsay}[1]{`#1'}
\usepackage{endnotes}
\newcommand{\1}{\={a}}
\newcommand{\2}{\={e}}
\newcommand{\3}{\={\i}}
\newcommand{\4}{\=o}
\newcommand{\5}{\=u}
\newcommand{\6}{\={A}}
\newcommand{\B}{\backslash{}}
\renewcommand{\,}{\textsuperscript{,}}
\usepackage{setspace}
\usepackage{tipa}
\usepackage{hyperref}
\begin{document}
\doublespacing
\section{\href{she-loves-me.html}{She Loves Me Review}}
First Published: 2022 February 21

\section{Draft 1}
Yesterday I had the fantastic opportunity to watch the musical \say{She Loves Me} performed by the Madison Opera Company.\footnote{side note: this does add credence to my position that musicals and opera are the same genre}
It's a fun 1960's story about love and perfume and miscommunication.
Since it's not as much in the cultural zeitgeist as a lot of the shows I've reviewed before\footnote{meaning I hadn't heard of it} I'm not really sure how much detail I can give without spoiling major elements.

There were some really shocking pacing choices.
Most notably, almost immediately after the darkest moment of the play, the scene hard cuts to a romantic restaurant where they perform a klezmer number.
I liked it, but it was certainly a bit of a tonal whiplash.

I also had the opportunity to attend a pre-show talk about the show.
It was written by the two creators of \say{Fiddler on the Roof}, which certainly helped the klezmer scene seem more reasonable.
It was also based on a movie which is based on a play.
I just thought that was amusing.
The Tom Hanks and Meg Ryan movie \say{You've Got Mail} is also based on \say{She Loves Me}, which is pretty nifty.

All in all, it was a lovely show, made lovelier by my company.
\end{document}