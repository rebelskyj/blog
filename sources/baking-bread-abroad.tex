\documentclass[12pt]{article}[titlepage]
\newcommand{\say}[1]{``#1''}
\newcommand{\nsay}[1]{`#1'}
\usepackage{endnotes}
\newcommand{\1}{\={a}}
\newcommand{\2}{\={e}}
\newcommand{\3}{\={\i}}
\newcommand{\4}{\=o}
\newcommand{\5}{\=u}
\newcommand{\6}{\={A}}
\newcommand{\B}{\backslash{}}
\renewcommand{\,}{\textsuperscript{,}}
\usepackage{setspace}
\usepackage{tipa}
\usepackage{hyperref}
\begin{document}
\doublespacing
\section{\href{baking-bread-abroad.html}{Baking Bread Abroad}}
First Published: 2018 November 5
\section{Draft 2}
As I mentioned in \href{basic-bread-recipe.html}{an earlier post},\footnote{I still don't know how to do this} I used to make bread a lot.
But, since coming abroad, I haven't made bread once.
It's a travesty, I know.
To rectify this horrible occasion, on one of the worst anniversaries in British history, I made some bread.

Of course, I had to make adjustments to my general recipe.
First, I tend to use five pounds of flour at a recipe.
For a variety of reasons, I wasn't going to do that.

So, I went with my normal way of making things, eyeballing.
I poured in maybe too much yeast, but it's yeast so it doesn't matter.
I then poured in most of the flour we had, poured in some water, and realized that it was far too much water.
I realized that, while I had almost no \say{bread} flour, I had plenty of self-raising flour.

For those of us who don't know what self raising flour is, it's flour with a leavening agent mixed in.\footnote{apparently baking powder, also salt for some reason}
Since I didn't really need one, having yeast and all, I was tempted to throw in some vinegar, to get rid of the leavening.
But, since I couldn't find vinegar, and didn't feel like putting in the effort, I just went for it.

And, about an hour and a half into rising, I realized that I didn't really want to wait until tomorrow to eat it, so I made it a semi-knead bread.
I just kneaded it, and gave up on rising, so threw it in the oven.
As I remember from this summer, I can't let it sit for 90 minutes, so I'll come back in the future, but since I'm not editing the draft any other way, it'll be in this draft.

After cooking for around 20, it looked risen but still too pale.
I set another timer for 10 minutes, and threw in more water.

Looked cooked, so out it came.
After letting it cool, I opened it, and it smelled bready.
It tasted like bread, and that's the general goal
\section{Draft 2}
As I mentioned in \href{basic-bread-recipe.html}{an earlier post},\footnote{I still don't know how to do this} I used to make bread a lot.
But, since coming abroad, I haven't made bread once.
It's a travesty, I know.
So, tonight I decided that I would make a loaf of bread, if only to assure myself that England is part of the real world.

Of course, I had to make adjustments.
As I mentioned in the recipe, I tend to use five pounds of flour at a recipe.
For a variety of reasons, I wasn't going to do that.

So, I went with my normal way of making things, eyeballing.
I poured in maybe too much yeast, but it's yeast so it doesn't matter.
I then poured in most of the flour we had, poured in some water, and realized that it was far too much water.
I realized that, while I had almost no \say{bread} flour, I had plenty of self-raising flour.

For those of us who don't know what self raising flour is, it's flour with a leavening agent mixed in.\footnote{apparently baking powder, also salt for some reason}
Since I didn't really need one, having yeast and all, I was tempted to throw in some vinegar, to get rid of the leavening.
But, since I couldn't find vinegar, and didn't feel like putting in the effort, I just went for it.

And, about an hour and a half into rising, I realized that I didn't really want to wait until tomorrow to eat it, so I made it a semi-knead bread.
I just kneaded it, and now I'll wait again for it.
\section{Draft 1}
As I mentioned in \href{basic-bread-recipe.html}{an earlier post},\footnote{I still don't know how to do this} I used to make bread a lot.
But, since coming abroad, I haven't made bread once.
It's a travesty.
So, tonight I decided that I would make a loaf of bread, if only to assure myself that England is part of the real world.

Of course, I had to make adjustments.
As I mentioned in the recipe, I tend to use five pounds of flour at a recipe.
For a variety of reasons, I wasn't going to do that.

So, I went with my normal way of making things, eyeballing.
I poured in maybe too much yeast, but it's yeast so it doesn't matter.
I then poured in most of the flour we had, poured in some water, and realized that it was far too much water.
I realized that, while I had almost no \say{bread} flour, I had plenty of self-raising flour.

For those of us who don't know what self raising flour is, it's flour with a leavening agent mixed in.\footnote{apparently baking powder, also salt for some reason}
Since I didn't really need one, having yeast and all, I was tempted to throw in some vinegar, to get rid of the leavening.
But, since I couldn't find vinegar, and didn't feel like putting in the effort, I just went for it.

It's raising right now, so we'll see how it goes.

\end{document}