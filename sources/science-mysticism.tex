\documentclass[12pt]{article}[titlepage]
\newcommand{\say}[1]{``#1''}
\newcommand{\nsay}[1]{`#1'}
\usepackage{endnotes}
\newcommand{\B}{\backslash{}}
\renewcommand{\,}{\textsuperscript{,}}
\usepackage{setspace}
\usepackage{tipa}
\usepackage{hyperref}
\begin{document}
\doublespacing
\section{\href{science-mysticism.html}{On Science and Mysticism}}
First Published: 2023 December 5
\section{Draft 2}
What is science?
The easy answer is what we were all taught in the early days of our schooling: science is the scientific method.
That is, science is the process of following a rote set of instructions which begins in observation and ends in reporting a disproved\footnote{I still want this to be disproven} hypothesis.
Like most easy answers, however, this answer is wrong.

Science is fundamentally a method of learning truths.
More than that, science is fundamentally a method of conveying truth.
An individual's experience can be noted and generalized to the universe at large.

Why does this redefinition matter?
Hopefully, this redefinition frames science not as something one does but as an experience one has.

That is, truths are not created, which I think most scientists would agree with.
After all, while we may not know the exact frequency of any given molecular transition, we know that it is not dependent on observation.\footnote{ignoring, of course, rest frames and the quantum uncertainty of anything molecular.}

More than that, truth is not an assembly of facts, but is something broader.
That is, reductionism does not work as a philosophical concept.
Although we can often make successful\footnote{read: predictive} models which are reductive, these always lack some fundamental element of reality.

From here, I would go so far as to claim that truth should, in fact, be capitalized as Truth.
That is, science's goal is uncovering part of Truth.

Although every philosophy and religion has its own experience of Mysticism, all definitions tend to agree on one point: a mystic is one who sees Truth.
Especially in Catholic Tradition, where I am most versed, Mysticism is an experiential reality.
Mystics encounter the Divine.

Now, at this point,\footnote{assuming that my argument about science even convinced you} you may be thinking that I've pulled some clever slights of hand to make science into a religion.
That is not my goal.\footnote{even though many people do treat science as a religion. I personally am very in favor of science and the scientific method of discovering (almost said creating, but that's sort of my point, isn't it) knowledge}
Instead, my goal is to draw a connection between two fundamentally different methods of understanding reality.

Many philosophies have a metaphysics: a way of viewing reality.
Scientific metaphysics can only ask, and therefore answer, questions about material reality.
Christian Mystical Metaphysics can only uncover spiritual reality.\footnote{don't quote me on this if that's actually a heresy.
Given that the Church tends to believe that Faith and Reason cannot contradict each other, it seems reasonable to assume that Mystics only reveal spiritual truths}

Crucially, both domains believe that reality is both universal and describable.
That is, the speed of light\footnote{in a vacuum, at rest frame velocity, for pedants} is not only constant throughout the entirety of the universe, but anyone can learn this to be true.
In a slightly more abstract analogy, differentiation, once discovered by Newton and Liebniz\footnote{if it wasn't clear, I do, in fact, fall into the camp which believes math is discovered not created}, can now be taught to students around the world.
Similarly, mystics in the Church have taught us spiritual truths that we can know, even without having experienced them ourselves.

I know that there's more to say on this topic, but at five thousand words, I think that I've exhausted my well on this topic for the day.
I'll hopefully revisit this again.\footnote{along with every other post I need to revisit}

Daily Reflection:
\begin{itemize}
\item Hobbies:
\begin{itemize}
\item Did I embroider today? I again forgot this in my car, which is fine.
\item Did I play guitar today? I did! I tuned it down to DADGAD last night, and it turns out that my voice is approximately 4 frets\footnote{a minor third} lower than I normally think of it. (Read: I started singing Oh No Not I and my voice kept breaking so I kept moving the capo lower until it felt nice)
\item Did I practice touch typing today? Oh gosh, I didn't realize that C was supposed to be typed with my middle finger, so it's very hard for me to start learning the letter. Oh well.
\end{itemize}
\item Reading
\begin{itemize}
\item Have I made progress on my Currently Reading Shelf? I finished one of the books on it! It was a weekly newsletter going through Screwtape Letters, but it finished this week. I should really get through Dracula Daily, because I'm about 2 months behind on it.
\item Did I read the book on craft? It's in my backpack, so I should try to read it at least a little before bed.
\item Have I read the library books? I returned one of them, because I found the music in it that I cared about. So that counts kind of, for all that I do plan to read the other book sometime soon. Maybe tomorrow morning? Probably not, but who can say?
\end{itemize}
\item Writing
\begin{itemize}
\item Did I write a sonnet? I will have by the time this posts.
\item Did I revise a sonnet? I learned how to revise one, at least. Apparently I should \say{see if the sonnet says what I want it to say, and if it doesn't, then I make it say that}, which is relatively helpful.
\item Did I blog? Look at this wow. I even thought about it.
\item Did I write ahead on Jeb? I finished tomorrow's chapter today, which is fun. I'd like to write a little of Friday's chapter as well, but we'll see.
\item Letter to friends? Nope!
\item Paper? I replotted the better data, and it does look better, which is really nice. Now I'm just confused about how distortion constants work.\footnote{or, more accurately, how they don't work}
\end{itemize}
\item Wellness
\begin{itemize}
\item How well did I pray? Terribly.
\item Did I clean my space? For a little bit!
\item Did I spend my time well? Eh, or so.
\item Did I stretch? Not a ton, but a little.
\item Did I exercise? I did! I did the minimal amount I'm doing on days with no other exercise.\footnote{one pushup, body weight squat, and 5 second plank per day of december, so today it was 5 squats then a 25 second plank then 5 pushups. If I can keep it up, by the end of the month I'll be doing a 2 minute plank which would be cool}
\item Water? I tried! Water didn't taste so good
\end{itemize}
\end{itemize}

\section{Draft 1.9, things that I cut from the essay as I wrote it, without context, because I think it's a fun kind of blackout poem like this}
Modern philosophy\footnote{post modern, technically} describes truth as fundamentally a linguistic phenomenon.
That is, we create truth by setting definitions.


More than that, though, truths do not need to be assembled, which may be a slightly larger claim.
That is, while much of science, especially the fundamental work I do, relies mostly on reducing the error between prediction and observation, that is not the science that non-scientists mean.
What they think of, and what I think every scientist truly wishes they did\footnote{maybe this is too broad of a claim, but I think it's at least true for me} is the science that we read about, 

As someone who has spent years studying quantum mechanics, one thing that I have learned is just how 

These two domains intersect most cleanly in the human person.
There are a number of ways to describe the way that we have souls, like angels, and bodies, like apes.

That is, just as the speed of light\footnote{in a vacuum, for the pedants} is not dependent on where in the universe we are, Salvation is not.\footnote{ok but salvation does look different to different people. What's a universal spiritual truth?}

In both domains, however, reality is assumed to be universal.
That is, just as the speed of light\footnote{in a vacuum, for the pedants} is not dependent on where in the universe we are, the nature of G-d is not dependent on where or when we are.
Crucially, as well, truth is transmissible in both domains.

\section{Draft 1.7, Find a way to start the post}
Knowledge, like a virus, spreads.

No, too evocative.\footnote{I know that's not the right word, but it feels like it}

To study, one must first accept starting presuppositions.

Too pretentious.

Prometheus brought fire to man.
Once gifted, man was able to reproduce this flame endlessly, and today nearly everyone over a young age can create fire of their own.

Maybe? Feels a little too anecdotal for me.

To be first is to break new ground and pave a new path.
To be second is to follow.

Eh.

What is science?
The easy answer, and the one that I, like so many others, learned in my schooling, is a six part process.
It begins with an observation\footnote{apparently. It's been long enough since I learned it that I thought it began with hypothesis. Makes sense to begin with observation}, which prompts a question: the hypothesis.\footnote{wild! They make this two steps! Ope, seems as though there are a lot of different versions of the scientific method. I'll use the one that I like, which is an abomination of a combination}
Once a literature review shows that the question has not been answered, an experiment can be designed which can disprove the hypothesis.
Should a hypothesis fail to be broken, it can be presumed true, and then reported to the world.

Ehhh too bogged down, especially since I don't really care about the scientific method.

What is science?
The easy answer, of course, is what we were all taught in science class: science is the scientific method.
That is, science is the process of learning knowledge via rigid and rote steps, where observations lead to questions lead to disproven hypotheses.

That definition, of course, does not hold for a lot of what scientists do.
What I do, for instance, does not have a hypothesis, except at the most basic level.
My goal is simply measurement, which allows us to ask other questions.

That misses the thread, let's try to go \say{you might think science is scientific method, in fact, science is about transmission of reality}
ope there we go

that's what I need

What is science?
Like many, my first answer to that question came from science class: science is the scientific method.
That is, science is following a rigid set of steps to go from an observation to a disproved hypothesis.
However, like most first answers, this answer doesn't work.

Science, at its core, is about transferring knowledge from the few to the many.
It is often remarked that the difference between messing around and science is how well you keep notes.

Nope that still lost the thread.

What is science?
The easy answer is what we were all taught in our science classes: science is the scientific method.
That is, science is following rote steps which begin with an observation about reality and end in a disproved hypothesis.
Like most easy answers, however, that is wrong.

Science, at its core, is a method of describing reality.\footnote{hmm is it describing reality, uncovering reality, finding truth? I think finding truth is a good enough starting name}

Science, at its core, is a method of uncovering truths.

Hmm, is it truths, truth, or Truth?
Each of those has a slightly different flavor.
I think that I'll go with \say{truth}

Is it uncover, discover, or create?
I think I like uncover, because it has implications of illumination which is always fun.

Science is a method to uncover truth.
At its core, science\footnote{capitalize it? Yeah probably. Do that in the final draft} is a fundamentally social discipline.\footnote{no}

Science is fundamentally a method to uncover truth.
More than that, however, it is a method to convey truth.\footnote{I think this truth needs to be one level of intensity higher. so eiether uncover truths and convey truth or uncover truth and convey Truth. I think the first}
An individual's experience can be generalized into universal reality.

What is mysticism?
The easy answer is a religious practice of experiencing communion with the Divine.

Ok yeah, let's see if we can't go from there. I'll stop wordsmithing at the end of each word from now on.

What is mysticism?
Mysticism is fundamentally a method of uncovering truth.
As a Catholic, who knows that our mission is to bring the whole of Creation to the Almighty, Mysticism also carries with it the goal of conveying Truth to the world.
No ok this is bad. I know I said no more drafting, but that was a lie. This gets moved down.


\section{Draft 1.5, Musing about how to begin the post}
Right now I feel like there's a couple of ways that I could begin this musing.\footnote{N.B. for the readers, if you have strong feelings about my including meta writing like this in future blogs (positive or negative) please do not hesitate to let me know}

\begin{enumerate}
\item I could start how I did, musing on this musing's genesis and its placement in the broader scope of me developing as a writer.
It's my month of craft, which does mean that there's something to be said for doing so, but I don't know if I really like that, especially given where I want to go.
\item I could begin as most essays recommend, by clearly stating my hypothesis.
That has the benefit of being up front, at the cost of me not having the rest of the musing to explore what I mean to say first.
Then again, I'm at well over two thousand words of thinking about it, so that shouldn't really be an issue either way.
\item I could begin as second-level essayists recommend, and start with a metaphor or anecdote, such as the way that benzene's structure was first hypothesized, or the way that the sewing machine was\footnote{allegedly} invented.
I could ideally weave that with an anecdote about some mystic, for all that I don't know if I know any of their stories well enough.
\item I could explain mysticism or science as though describing the other, or describe both using the same words.
That could be fun, for all that I know that it will also be difficult to do.
\end{enumerate}
As often happens when I enumerate\footnote{hah, get it, because in LaTeX the numbered lists are generated in an enumerate environment} my options, one seems most exciting, and that tends to be where my ability to generate new ideas stops.
I'm going to try to describe the two methods of inquiry without distinguishing the two.

From there, I guess that I could start going into like how both work when they differ? 
I'm not really sure where, if ever the two actually diverge, but I'm sure that it's somewhere.
What else did I do in the first draft?

Oop, it kind of seems like that's as far as I got in the first draft of this musing. Such a shame, but I wonder if the second draft might give me more inspiration once I have typed something slightly\footnote{look at me, being so optimistic and everything} more coherent.
\section{Draft 1}
Nearly every book on writing I've ever read\footnote{rephrase this in draft two to make it clear it's any book that even tangentially discusses writing} gives a lot of the same advice.\footnote{wow, wild. spice this up more}
Chief among them is to always have a small notebook with you to note down ideas as they arise.
I've never really been a fan of holding a pen and pad of paper, but I do live in the twenty first century.
I have another option.

And so, every so often I'll open my notes app and type down a phrase that suddenly strikes me, or I'll open my voice memos and sing something.\footnote{I really need to clear that out}
Most of the time, I forget to ever look at them again, and the few times that I remember, I either forgot the context of a phrase or no longer find it striking.
Every so often, though, I find something that strikes me even weeks later.

Today, I'd like to muse about one of those.
This past Saturday, for some reason, I was struck by the connection of science and mysticism.
A quick\footnote{and admittedly fairly sparse} search doesn't show that the field is full of people putting forth their explanations, so I'm going to do my best to try.\footnote{I will also be reading the two papers I found to see what they have to say}

The first paper I found\footnote{\href{https://www.jstor.org/stable/1203270}{Mysticism and the Philosophy of Science by Anthony N. Perovich, Jr.} (look at me actually using a footnote to cite)} seems to be discussing the concept of mysticism as a claim of inherent validity of a certain belief, at least in the first few pages.
Since I don't need to worry\footnote{here, at least} about considering whether mystics from other faith traditions are touching on the actual Truth, I don't know how useful this article is going to be.
However, it's only 20 pages, and is probably worth at least a skim right now.
As I get further into the article, it seems like the sort of article I generally love, where we dive deeply into the semantics of individual words.
That being said, it's the sort of article I love reading with a group, where someone else has already done a deep dive to distill a lot of the useful information out.\footnote{distill is probably the wrong word here, even though it's the cliche. It's not about a reduction and concentration, but repackaging to make understandable.
I'm sure I'll find a better word by the second draft}
I'll have to reread this article later, but it is absolutely not related to what I want to discuss today.
It's a fascinating dive into the way that we can, without acknowledging certain truths as better than others, make sense of disparate mystical experiences and scientific theories.

The second paper\footnote{\href{https://www.academia.edu/download/101343293/RHJ_New_Age_M_S_.pdf}{Mysticism in the New Age: Are Mysticism and Science Converging? by Richard Jones}} explicitly focuses on Eastern Religion, and so is not relevant to my discussion here.
I'll probably read it too, since I'm curious what it has to say, but that's probably a problem for next year me.

Now, because I am aware that I am incredibly easy to prime\footnote{whether I'm easier than anyone else, I won't make a claim here. Still, as far as I can tell, it's among the easiest psychological effects to reproduce, which is fun}, I'm currently thinking about definitions.
What is mysticism, and what do I mean here by science?

As with all people, I find that the methods I use to approach the world change as I continue to grow and live.
I had a very long period where my preferred analytical technique was constructing definitions.
If something cannot be rigorously defined, the logic goes, it is not a meaningful category.

Of course, this implicitly binaries the world.
Defining music, for instance, is not something that can be satisfactorily done.\footnote{in that every definition I've ever been given other than \say{you know it when you see it} includes things that aren't music (ex: notes in time says that the dropping of rain on an empty roof is music. Since a fundamental part of music to me is observation (which I don't remember if I've mused about before. If not, then my preferred thought experiment is: imagine writing and performing a piece on the piano for a friend. That's obviously music. If you then remove the friend, playing only for yourself, it's still music. If I use a DAW (digital audio workstation) to produce the music and play it back for me, I would still call that music. If I program a computer to fill in the chord progressions for me, I'd still call that music. If I program a computer to produce a convincing piece and play it for me, that's still music. If it records an audio file that I don't listen to, but someone else does, that's still music to me. If the audio file is never listened to, though, is it music? If not, then how can the potential exist without the action? It's at this point that I realize I need to learn more metaphysics. Where was I?), that is not music) or excludes things which are music (most often John Cage's 4'33", which is a very often misunderstood work of music that I love as a conceptual piece}
Much of life, as it turns out, exists in the hazy in betweens that are not clearly labeled and defined.
Nonetheless, it's useful to have an idea what I'm talking about, if only to aid in shared meaning.

When I speak of science, I am not referring to the scientific method\footnote{hypothesis, research, testing, etc.} that I and so many are taught is fundamental to the life of a scientist.
As a professional scientist, very little of what I do fits neatly into those boxes, and yet I do not believe that the way that I do science is lesser for it.
A fundamental part of science is simply observation and measurement.

However, I am also not really speaking about the rote parts of science.
Instead, what I'm probing at is the way that so many scientific discoveries\footnote{breakthroughs, if you will} are described.
The discovery of the shape of benzene, for instance, is described as coming to the scientist in a dream.\footnote{I found a citation on wikipedia, but the link is broken. Searching for the book again finds it on the Internet Archive, but not in any physical location I have access to}
There is a joke I've heard a few times talking about how the great discoveries in science take place when sleeping, and the great discoveries in mathematics take place when sleep is removed.
As I think about it, both versions are actually helpful for my thinking about mysticism.

Now, this does raise a question: what is mysticism?
Adapting from Wikipedia\footnote{I'm watching a long video essay right now about plagiarism, and it's making me more aware of how much I generally borrow ideas}, mysticism is a connection with greater forces.
Depending on the tradition and the specific interpretation the mystic has of the tradition, it is equally a form of being taken away from themselves as becoming fully themselves for the first time.
Regardless, mystics are revered in nearly every culture for the same reason: they have access to a fundamental and hidden truth about reality.
By listening to mystics relay their experiences, however, the common person can see the truth of reality, if only dimly.
If you've ever heard a top scientist or mathematician discuss their passion, it might be easy to see the connection.

I read Donald Knuth's CV this past weekend with my brothers.\footnote{add a segue. this is messy. Also probably do something with that long footnote. Maybe delete because, while interesting, not really relevant.}
One striking thing that I noticed is that he wrote a piece for Organ based on the book of Revelations.
Now, that in and of itself is interesting, but not necessarily pointing to anything.
The fact that he also wrote a book where he analyzes Chapter 3 Verse 16 in every\footnote{to the best of my knowledge, he's Lutheran, which means that he probably doesn't have every book in the bible, just the ones in his} book of the Bible.
Even though that discussion came after I jotted down the note, it feels significant to the discussion.

Of course, as soon as I started to say that, I realized that I have a bias in what mystics I'm familiar with.
As someone who's taken music history\footnote{it's wild to me how often music history has come up on this blog recently. Must be something about the time of year we're in}, the mystics I've been exposed to in detail tend to be mystics who also had significant musical output, such as Doctor of the Church Hildegard.

Ok so other than the simple point that scientists often describe breakthroughs as though they've suddenly received some divine revelation, and that's picture perfect mysticism, what am I trying to say?
I think that might be it, but I no longer know if I know enough about either subject to really write a whole essay about it.
I would at least want to be able to point to two stories, one of a mystic and one of a scientist, to show the parallels in the narrative.
Failing that, I'm not really sure what I can do.

I suppose that I could muse on the nature of mysticism?
Like how we are generally willing to accept that there are people with special knowledge, especially in quantum mechanics.
Might be worth riffing on that for a little, just to see what we see.

What is mysticism?
Mysticism is a term generally used to describe a religious altering of mind.
Of course, as with anything else used in religious contexts, the word is used to describe wide and sometimes contradictory examples.
Many better writers than I have written about the difficulties in defining religion, especially given how averse some groups are to having their beliefs labeled as religious.
As such, the fundamental part of the mystic experience for me is the belief that knowledge was given to the mystic, rather than the mystic discovering the information themself.

How, then, is a scientist a mystic?\footnote{at first I had modern day, but part of my argument should be, if it isn't already explicit, that the mystic experience still happens to religious today and happened to scientists in the past}
When looking at the world through the view of a scientist, a few things are fundamentally required to be true.
First, the universe must be sensical.\footnote{I think my autocorrect wants sensible, which makes sense, because Google NGrams shows that sensible has had orders of magnitude more popularity over the entire corpus searched. OED notes sensical appearing a few centuries later than sensible, though does not in any way dispute the word as real or legitimate. Sensical is apparently used less than once per million English words, which I think means that my entire blog has a higher than standard concentration of the word just from the three times I've used it here. I think that my entire corpus might be higher than the English standard actually}
That is, things happen for a reason, and the same starting conditions will result in the same outcome.\footnote{Quantum mechanics puts an asterisk on that, but there's nuance I can add if I want to be a sophist. Sophistry is a fun skill to develop, for all that I know it's a bad one}
I initially was going to say that reproducible is a different condition than sensical, but I don't know if that's true anymore.

Second, the universe must be measurable.
Russell's teapot is an example of how science can fail to learn things.\footnote{I'm not getting into it here, but the long and short of it is that science is useless for making claims that it cannot interface with facts.
I know that the point of the example is supposed to be the ridiculousness of religious thought, but}

Let's compare that to the world view required to be a Catholic.\footnote{Yes, I know that modern science was inarguably (spell check wants unarguably, which has recently been overtaken for popularity) and obejctively inspired by the Church, but that's not really the point of the argument here.}
The universe was created by an all powerful, all loving Creator, who has given us what we need to understand the world, and by extension, learn more about He who created us.\footnote{Hmm, I wonder how many heresies that sentence has.}
Of course, there's the nuance of Catholic, so Christ is a requisite part of the world view, and that is certainly an important aspect of how Catholics and Catholic mystics view the world.
However, Christ's incarnation and sacrifice can be somewhat derived if you start with the first proposition and then follow it with \say{and G-d gave authority to his Church,} for all that it's a bit of a hand waving explanation.

Ok, let's refocus.
Both mysticism and science\footnote{I'm explicitly not saying scientism here, because scientism is, in my experience, at least, the belief that the world is nothing more than the measureables.
As with every other form of inquiry, science is fundamentally limited in the sorts of questions it can ask and answer.
It's ridiculous, to me, at least, to claim that there's a single method of inquiry that encompasses everything.
Refocusing again} require the universe to be fundamentally sensible.
One fantastic secondary side effect of this assumption is the ability for knowledge to be distributed.

If knowledge is portrayed, as it so often is, as light, then each of us can be Prometheus\footnote{almost wrote Pythagorous, but that's a whole different mystical thing}, sharing the fire and light of knowledge with those around us.
Fire, like knowledge, yearns to spread.\footnote{insert obligatory information yearns to be free}
Taking a step back from the forced cliche, however, we do see that it resonates.

It is far easier to follow a derivation than to come to it for the first time.
Calculus was something that two of the greatest minds of their generation were barely able to scratch the surface of.
Now, the average college student is expected to be able to learn the sum of what took them years of study in a few scant semesters.

Similarly, many Catholics today pray the Chaplet of Divine Mercy, as St. Maria Faustyna Kowalska of the Blessed Sacrament\footnote{yes I needed to look up the name, no I don't feel guilt about it. I wonder if the y versus i in the name is similar to what I remember seeing a bunch in the early days of the latest Ukraine and Russia conflict, where the choice of spelling made implications about the validity of Ukranian as a language.} was taught in her revelations.
We are assured of the same benefits, just as we are able to do as much with Calculus as Isaac Newton\footnote{he was made a knight by a false monarchy, so I will not give him the title of Sir}.\footnote{Shoot, I said that I wouldn't get into specific examples, but here I am. Might be time to try draft two to see if I can't make this at least a little more coherent. Should probably take a break and do other work first, though. It'll be good to give my mind time to settle some of these thoughts.}

\end{document}