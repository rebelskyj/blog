\documentclass[12pt]{article}[titlepage]

\newcommand{\say}[1]{``#1''}

\newcommand{\nsay}[1]{`#1'}

\usepackage{endnotes}

\newcommand{\1}{\={a}}

\newcommand{\2}{\={e}}

\newcommand{\3}{\={\i}}

\newcommand{\4}{\=o}

\newcommand{\5}{\=u}

\newcommand{\6}{\={A}}

\newcommand{\B}{\backslash{}}

\renewcommand{\,}{\textsuperscript{,}}

\usepackage{setspace}

\usepackage{tipa}

\usepackage{hyperref}

\begin{document}

\doublespacing

\section{\href{digital-diaries.html}{Digital Diaries}}

Prereading note: this post was written as an assignment, so drafts 4-6\footnote{hopefully} lack much of my snark and\footnote{ibid} will read much more like an academic essay.

In the final draft, I hope to have restored some of the charm\footnote{read snark} that I tend to have in my writings.


\section{Draft 7}

Diaries today are becoming more and more of a digital phenomenon.

That is, people are deciding to record their thoughts on computers, rather than paper.

This transition from analog to digital is not wholly uncontroversial.


Many people have a distrust of digital diary keeping.

However, their objections almost always orient around the alleged fragility of digital diaries.


For those unfamiliar with the Internet, online storage can seem horribly fragile.

Unlike a bound book, which can last indefinitely,\footnote{barring exposure to fire, water, or neglect, pestilence, degradation of ink, or simply just being lost} digital diaries require constant upkeep.

If written in a word processor, a blog\footnote{a neologism, short for \say{web log}} post could become unreadable if the file format becomes obsolete or unused.\footnote{though MSDos(.? The old operating system emulator) existing does throw some doubt on that idea}

But, most blogs are, as the name suggests, hosted on the Internet.


A second concern is that posts on the web may be pulled down or somehow become unavailable.\footnote{not that diaries are ever burned or otherwise destroyed intentionally}

The fact that two different agencies,\footnote{the Internet Archive’s Wayback Machine and Google} both make constant backups of the Internet is seen as a lackluster response.\footnote{the fact that there is no similar analog equivalent remains unstated}

And, unlike physical diaries, anyone can access any blog from anywhere and anytime, which frees the researcher from needing the funds to go to the specific library where a diary is held, or find a way to have a copy made.

The copies are automatically made to every computer accessing the site.


Since these backups exist so widely, it is far less likely that we will undergo a similar loss to the burning of the Library of Alexandria or the burning of the linguistic library in Brazil.\footnote{\href{https://www.nationalgeographic.com/science/2018/09/news-museu-nacional-fire-rio-de-janeiro-natural-history/}{Seen Here}}

Had the files in the library been digital, they would have been hosted in a different site from the physical artifacts, preventing that horrible loss of knowledge.


That tragedy calls out the most important problem with preferring physical media over digital media.

Yes, you can duplicate a physical document.

If you don’t care about the exactness of the replicated document,\footnote{I.e. the exact material, penmanship, and so on} it is a fairly trivial process\footnote{though more effort than printing or backing up a digital file} to transcribe a backup.\footnote{or, heaven forbid, scan and upload it}

But, as the recent burning of the Brazilian library shows, even vitally important documents aren’t always backed up.\footnote{all joking aside, I find it absolutely horrible that some languages are now completely gone from the collective human knowledge}

The Internet, however, backs up everything.


And to me, the heart of digital diary and the Internet is reflected by this fact: neither promises permanence, only equality.

Every work hosted on Wordpress, for instance, is equally likely to be there in fifty year’s time.\footnote{barring the author deleting it}

The famous and forgotten will both exist in perpetuity.

Therefore, to me, the arguments against online diaries, especially now, some 20 years after the first published arguments, are nothing except baseless fear of the future.


\section{Draft 6}

Diaries are becoming a digital phenomenon.

That is, people are deciding to record their thoughts on computers, rather than paper.

This transition from analog to digital is not wholly uncontroversial.


Many people have a distrust of digital diary keeping.

However, their objections almost always orient around the alleged fragility of digital diaries.\footnote{Diaries, On-Line Diaries, and the Future Loss to Archives; Or, Blogs and the Blogging Bloggers who Blog Them. C. O’Sullivan}


For those unfamiliar with the Internet, online storage can seem horribly fragile.

Unlike a bound book, which can last indefinitely, digital diaries require constant upkeep.

If written in a word processor, a blog\footnote{a neologism, short for \say{web log}} post could become unreadable if the file format becomes obsolete or unused.

But, most blogs are, as the name suggests, hosted on the Internet.


A second concern is that posts on the web may be pulled down or somehow become unavailable.

The fact that two different agencies,\footnote{the Internet Archive’s Wayback Machine and Google} both make constant backups of the Internet is seen as a lackluster response.

O’Sullivan complains that \say{the Wayback Machine does not have word or subject search capabilities,}\footnote{Diaries, On-Line Diaries, and the Future Loss to Archives; Or, Blogs and the Blogging Bloggers who Blog Them. C. O’Sullivan

 p.71}, which would hold merit if not for the fact that physical diaries and libraries also lack subject search without the work of dedicated stewards.

Additionally, blogs all have word search capabilities, as modern browsers contain that feature.

And, unlike physical diaries, anyone can access any blog from anywhere and anytime, which frees the researcher from needing the funds to go to the specific library where a diary is held, or find a way to have a copy made.

The copies are automatically made to every computer accessing the site.


Since these backups exist so widely, it is far less likely that we will undergo a similar loss to the burning of the Library of Alexandria or the burning of the linguistic library in Brazil.\footnote{\href{https://www.nationalgeographic.com/science/2018/09/news-museu-nacional-fire-rio-de-janeiro-natural-history/}{Seen Here}}

Had the files in the library been digital, they would have been hosted in a different site from the physical artifacts, preventing that horrible loss of knowledge.


That tragedy calls out the most important problem with preferring physical media over digital media.

Yes, you can duplicate a physical document.

If you don’t care about the exactness of the replicated document, it is a fairly trivial process to transcribe a backup.

But, as the recent burning of the Brazilian library shows, even vitally important documents aren’t always backed up.

The Internet, however, backs up everything.


And to me, the heart of digital diary and the Internet is reflected by this fact: they don’t promise permanence, only equality.

Every work hosted on Wordpress, for instance, is equally likely to be there in fifty year’s time.

The famous and forgotten will both exist in perpetuity.

Therefore, to me, the arguments against online diaries, especially 15 years after the publishing of O’Sullivan’s article, are nothing except baseless fear of the future.


\section{Draft 5}

Diaries today are becoming more of a digital phenomenon.

That is, more and more people decide to record their thoughts on digital displays, rather than analog records.

However, this transition from analog to digital is not wholly uncontroversial.


Many people have a distrust of digital diary keeping.

But, examination of these objections tends to show that they are rooted in either classism or appeals to tradition.

They almost always find themselves orienting around the alleged fragility of digital diaries.\footnote{Diaries, On-Line Diaries, and the Future Loss to Archives; Or, Blogs and the Blogging Bloggers who Blog Them. C. O’Sullivan}


And, for those unfamiliar with the Internet, online storage can seem horribly fragile.

Unlike a bound book, which can last indefinitely, digital diaries require constant upkeep.

If written in a word processor a blog\footnote{a neologism, short for \say{web log}} post could plausibly become unreadable if the file format becomes obsolete or unused.

But, most blogs are, as the name suggests, hosted on the Internet.


A second concern is that posts on the web may be pulled down or somehow also become unavailable.

The fact that the Internet Archive’s Wayback Machine\footnote{a constantly updating archive of the internet} or Google’s own caching system both make constant backups is seen as lackluster.

O’Sullivan complains that \say{the Wayback Machine does not have word or subject search capabilities,}\footnote{Diaries, On-Line Diaries, and the Future Loss to Archives; Or, Blogs and the Blogging Bloggers who Blog Them. C. O’Sullivan

p.71}, which would hold merit if not for the fact that physical diaries and libraries also lack subject search without the work of dedicated stewards..

Additionally, digital diaries all have word search capabilities, as modern browsers all contain that feature.

And, unlike physical diaries, anyone can access any blog from anywhere and anytime, which frees the researcher from needing the funds to go to the specific library where a diary is held, or find a way to have a copy made.


Thirdly, since these backups are spread over many different servers, it is far less likely that we will undergo a similar loss to the burning of the Library of Alexandria or even the very recent burning of the linguistic library in Brazil.\footnote{\href{https://www.nationalgeographic.com/science/2018/09/news-museu-nacional-fire-rio-de-janeiro-natural-history/}{Seen Here}}

Had the files in the library been digital, they could have been more easily duplicated, and would have been hosted in a different site from the physical artifacts, preventing that horrible loss of knowledge.


That tragedy calls out the most important problem with preferring physical media over digital media.

Yes, you can duplicate a physical document.

If you don’t care about the exactness of the replicated document, it is a fairly trivial process to transcribe a backup.

But, as the recent burning of the Brazilian library shows, even vitally important documents aren’t always backed up.


The Internet, however, backs up everything.

Yes, we may not have a guarantee that this generation’s Beowulf will survive if not printed.

However, even many of the manuscripts from that time are still gone.


And to me, the heart of digital diary and the Internet is reflected by this: they don’t promise permanence, only equality.

Every work hosted on Wordpress, for instance, is just as likely to be there in fifty year’s time.

The famous and forgotten will both exist in perpetuity.

Therefore, to me, the arguments against online diaries, especially 15 years after the publishing of O’Sullivan’s article, are nothing except baseless fear of the future.


\section{Draft 4}

Diaries, like many written records, are becoming more and more of a digital phenomenon.

That is, more and more people decide to record their thoughts on digital displays, rather than analog records.

And, like the other forms becoming digital, the transition from analog to digital is not wholly uncontroversial.


For many reasons, people have a distrust of digital diary keeping.

But, even a mild examination of most of these objections shows that they are deeply rooted in either classist thoughts or appeals to tradition.

They almost always find themselves orienting around the alleged fragility of digital diaries, regardless of the factuality of these claims.\footnote{Diaries, On-Line Diaries, and the Future Loss to Archives; Or, Blogs and the Blogging Bloggers who Blog Them. C. O’Sullivan}


For those unfamiliar with the Internet, online storage can seem horribly fragile.

Unlike a bound book, which can last indefinitely, digital diaries require constant upkeep.

If written in a word processor, for instance, a blog\footnote{a neologism, short for \say{web log}} post could plausibly become unreadable if the file format becomes obsolete or unused.

However, since nearly old computers are still functional, and old operating systems are constantly being ported to new machines, it is unlikely that we will ever have files that we truly cannot open.

They may be difficult to interpret, but no more so than damaged manuscripts.


A second concern is that posts on the web may be pulled down or somehow also become unavailable.

The obvious rebuttal to this statement, namely the Internet Archive\footnote{a constantly updating archive of the internet} or Google’s own caching system is seen as lackluster.

O’Sullivan complains that \say{the Wayback Machine (the Internet Archive) does not have word or subject search capabilities.}footnote{Diaries, On-Line Diaries, and the Future Loss to Archives; Or, Blogs and the Blogging Bloggers who Blog Them. C. O’Sullivan p.71}

That argument would hold merit if not for the fact that physical diaries lack search capabilities, and libraries holding them do as well.

What searching methods are available come only when dedicated people add them.


However, the lack of searching capabilities is never seen as a flaw in traditional diaries.

Unlike physical diaries, anyone can access any blog from anywhere and anytime, which frees the researcher from needing the funds to go to the specific library where a diary is held, or find a way to have a copy made.

They can also search, since every modern web browser has search and find capabilities.


Additionally, since these files are spread over many different servers, it is far less likely that we will undergo a similar loss to the burning of the Library of Alexandria or even the very recent burning of the linguistic library in Brazil.\footnote{\href{https://www.nationalgeographic.com/science/2018/09/news-museu-nacional-fire-rio-de-janeiro-natural-history/}{Seen Here}}

Had the files in the library been digital, they could have been more easily duplicated, and would have been hosted in a different site from the physical artifacts, preventing that horrible loss of knowledge.


That tragedy leads to the third problem with preferring physical media over digital media.

Yes, you can duplicate a physical document.

If you don’t care about the exactness of the replicated document, it is a fairly trivial process to transcribe a backup.

But, as the recent burning of the Brazilian library shows, even vitally important documents aren’t always backed up.

What guarantee does a random, insignificant citizen of the world have that anything they write will ever be relevant to historians?


The most honest answer is that they don’t.

Most likely nothing any given blogger has to say won’t be relevant.

Nonetheless, the Internet protects and safeguards it.

Yes, it is true that we may not have a guarantee that this generation’s Beowulf will survive if not printed.

However, even many of the manuscripts from that time are still gone.

The Internet makes it more likely that the unimportant words will live on.


And to me, that truly is the heart of digital diary keeping, and by extension, the internet.

They doesn’t promise permanence, only equality.

Every work hosted on Wordpress, for instance, is just as likely to be there in fifty year’s time.\footnote{barring the author destroying it}

The famous and forgotten will both exist in perpetuity.

Therefore, to me, the arguments against online diaries, especially 15 years after the publishing of O’Sullivan’s article, are nothing except baseless fear of the future.


\section{Draft 3}

Diaries, like many written records, are becoming more and more of a digital phenomenon.

That is, more and more people\footnote{especially in younger generations} are turning, not to their notebooks, but to their keyboards when they decide to put to paper\footnote{that expression may not work as well here} what’s in their mind.

And, like these other records, the transition from analog to digital is not wholly uncontroversial.


For many reasons, people have a distrust of digital diary keeping.

But, even a mild examination of most of these objections shows that they are deeply rooted in either classist thoughts or appeals to tradition.

They almost always find themselves orienting around the alleged fragility of digital diaries, regardless of the factuality of these claims.\footnote{Diaries, On-Line Diaries, and the Future Loss to Archives; Or, Blogs and the Blogging Bloggers who Blog Them. C. O’Sullivan}


For those unfamiliar with the Internet,\footnote{shoot, is this a capitalized thing?} online storage can seem horribly fragile.

Unlike a bound book, which can last indefinitely,\footnote{barring exposure to fire, water, or neglect, pestilence, degradation of ink, or simply just being lost} digital diaries require constant upkeep.

If written in a word processor, for instance,\footnote{an unlikely scenario, but one that is mentioned} a blog\footnote{a neologism, \say{shortening web log}} post could plausibly become unreadable if the file format becomes obsolete or unused.

However, since every currently obsolete file storage\footnote{to the best of my knowledge} currently has an interpreter, it is unlikely that we will ever have files that we truly cannot open.

They may be difficult to interpret, but no more so than damaged manuscripts.


A second concern is that posts on the web may be pulled down or somehow also become unavailable.\footnote{a much more believable scenario}

The obvious rebuttal to this statement, namely the Internet Archive\footnote{a constantly updating archive of the internet} or Google’s own caching system is seen as lackluster.

O’Sullivan complains that \say{the Wayback Machine (the Internet Archive) does not have word or subject search capabilities.}\footnote{Diaries, On-Line Diaries, and the Future Loss to Archives; Or, Blogs and the Blogging Bloggers who Blog Them. C. O’Sullivan p.71}

That argument would hold merit if not for the fact that physical diaries lack search capabilities, and libraries holding them do as well.

What searching methods are available come only when dedicated people add them.

Regardless of the search capabilities, the files\footnote{and physical remnants} still exist.

Unlike the physical diaries, however, we can access\footnote{almost} any blog from anywhere and anytime,\footnote{assuming an internet connection} which frees the researcher from needing the funds to go to the specific library where a diary is held, or find a way to have a copy made.


Additionally, since these files are spread over many different servers, it is far less likely that we will undergo a similar loss to the burning of the Library of Alexandria\footnote{if we accept ancient history as real} or even the very recent burning of the linguistic library in South America.\footnote{/href{https://www.nationalgeographic.com/science/2018/09/news-museu-nacional-fire-rio-de-janeiro-natural-history/}{Seen Here}}

Had the files been digital, they could have been more easily duplicated, and would have been hosted in a different site from the physical artifacts.\footnote{given how cheap cloud storage is today}


That leads to the third problem with preferring physical media over digital media.

Yes, you can duplicate a physical document.

If you don’t care about the exactness of the replicated document,\footnote{I.e. the exact material, penmanship, and so on} it is a fairly trivial process to transcribe a backup.\footnote{or, heaven forbid, scan and upload it to the internet}

But, as the recent burning of the Brazilian library shows, even vitally important documents aren’t always backed up.

What guarantee does a random, insignificant citizen of the world have that anything they write will ever be relevant to historians?


The fairest answer is that they don’t.

Most likely they won’t be relevant.

Nonetheless, the internet protects and safeguards it.

Yes, it is true that we may not have a guarantee that this generation’s Beowulf will survive if not printed.\footnote{though the fact that the Library of Congress is printing out every tweet (for instance) makes this much less likely in my mind}\,\footnote{not to mention the fact that we also don’t have many of the works from that time period, which may have been even better than Beowulf}

But, we have a much higher chance that any thought of a random individual will be as accessible to future generations as that epic.


And to me, that truly is the heart of digital diary keeping, and by extension, the internet.

They doesn’t promise permanence, only equality.

Every work hosted on Wordpress is just as likely to be there in fifty year’s time.\footnote{barring the author destroying it}

The famous and forgotten will both exist in perpetuity.

To me, the arguments against online diaries, especially 15 years after the publishing of O’Sullivan’s article, are nothing except baseless fear of the future.


\section{Draft 2}

Diaries, like many written records, are becoming more and more of a digital phenomenon.

And, like these other records, the transition is not wholly uncontroversial.

For many reasons, people have a distrust of digital diary keeping.

However, these objections are almost always classist, unreasonable, or Ludditical.

They almost always find themselves orienting around the alleged fragility of digital diaries, regardless of the factuality of these claims.\footnote{Diaries, On-Line Diaries, and the Future Loss to Archives; Or, Blogs and the Blogging Bloggers who Blog Them. C. O’Sullivan}


For those unfamiliar with the digital world, online storage can seem horribly fragile.

Unlike a bound book, which can last indefinitely,\footnote{barring fire, water, neglect, pestilence, degradation of ink, or simply just being lost} digital diaries require constant upkeep.

If written in a word processor, for instance,\footnote{an unlikely scenario, but one that is mentioned} the blog post may can hypothetically become unreadable if the file format becomes obsolete or unused.

However, since every currently obsolete file storage\footnote{to the best of my knowledge} currently has an interpreter, it is unlikely that we will ever have files that we truly cannot open.


A second concern is that posts on the web\footnote{a much more believable scenario} may be pulled down or somehow also become unavailable.

The simple rebuttal of the Internet Archive\footnote{a constantly updating archive of the internet} or Google’s own caching system is seen as lackluster.

O’Sullivan complains that \say{the Wayback Machine does not have word or subject search capabilities.}\footnote{Diaries, On-Line Diaries, and the Future Loss to Archives; Or, Blogs and the Blogging Bloggers who Blog Them. C. O’Sullivan p.71}

That argument would hold merit if not for the fact that physical diaries lack search capabilities, and libraries holding them do as well.

Regardless of the search capabilities, the files still exist.

Unlike the physical diaries, we can access all of the blogs from anywhere with an internet connection, which frees the researcher from having to find the funds to go to a library where a diary comes from.


Additionally, since these files are spread over many different servers, it is far less likely that we will undergo a similar loss to the burning of the Library of Alexandria\footnote{if we accept ancient history as real} or even the very recent burning of the linguistic library in South America.\footnote{\href{https://www.nationalgeographic.com/science/2018/09/news-museu-nacional-fire-rio-de-janeiro-natural-history/}{Seen Here}}

Had the files been wholly digital, they could have been more easily duplicated, and would have been hosted in a different site from the physical artifacts.


That leads to the third problem with physical over digital media.

Yes, you can duplicate a physical document.

If you don’t care about the exact document,\footnote{I.e. the exact material, penmanship, and so on} it is a fairly trivial process to transcribe a backup.\footnote{or, heaven forbid it, scan it and upload it to the internet}

But, as the recent burning of the library shows, even drastically important documents aren’t always backed up.

What guarantee does a random, insignificant citizen of the world have that anything they write will ever be relevant to historians?

The short and long answer is they don’t.

Most likely they won’t be relevant.


Nonetheless, the internet protects and safeguards it.

Yes, we may not have as good of a guarantee of this generation’s Beowulf surviving on parchment if it isn’t printed out.\footnote{though the fact that the Library of Congress is printing out every tweet (for instance) makes this much less likely in my mind}

But, we have a much higher chance that the random thoughts of a random individual will be as accessible to future generations as that epic.


That truly is the heart of the internet.

It doesn’t promise permanence, it promises equality.

Every work hosted on wordpress is just as likely to be there in fifty year’s time.\footnote{barring the author destroying it}

The famous and forgotten will both exist in perpetuity.

And that, along with the different archiving methods, brings to the next point.

This is the first time in human history where we can not only see what was written, but pinpoint to the exact second when a piece is written, edited, or deleted.

Diary studiers point to the spread of the clock as a phenomenon leading to the rise of the diary and see this as a good change, and yet don’t feel the same way about the rise of digital media.

To me, this is, especially 15 years after the publishing of O’Sullivan’s article, nothing except baseless fear of the future.


\section{Draft 1}

Diaries, like many written records, are becoming more and more of a digital phenomenon.

And, like these other records, the transition is not wholly uncontroversial.

For many reasons, people have a distrust of digital diary keeping.

However, these objections are almost always classist, unreasonable, or Ludditical.

They almost always find themselves orienting around the alleged fragility of digital diaries


One common complaint about digital diaries is their alleged fragility.\footnote{Diaries, On-Line Diaries, and the Future Loss to Archives; Or, Blogs and Blogging Bloggers Who Blog Them}

For those unfamiliar with the digital world, they can seem horribly fragile.

Unlike a bound book, which can last indefinitely,\footnote{barring fire, water, neglect, pestilence, or degradation of ink} digital diaries require constant upkeep.

If written in a word processor, for instance,\footnote{an unlikely scenario, but one that is mentioned} the file may become unreadable if the software becomes obsolete or unused.

However, almost any old file system has seen some sort of official use, and so interpreters exist.

It’s unlikely that we will ever have files that we truly cannot open.


A second concern is that posts on the web\footnote{a much more believable scenario} may be pulled down or somehow also become unavailable.

The simple rebuttal of the Internet Archive\footnote{a constantly updating archive of the internet} or Google’s own caching system is seen as lackluster.

O’Sullivan complains that \say{the Wayback Machine does not have word or subject search capabilities.}\footnote{Diaries, On-Line Diaries, and the Future Loss to Archives; Or, Blogs and the Blogging Bloggers who Blog Them. C. O’Sullivan p.71}

That argument would hold merit if not for the fact that physical diaries lack search capabilities, and libraries holding them do as well.

Regardless of the search capabilities, the files still exist.

It is far less likely that we will undergo a similar loss to the burning of the Library of Alexandria\footnote{if we accept ancient history as real} or even the very recent burning of the linguistic library in South America.\footnote{\href{https://www.nationalgeographic.com/science/2018/09/news-museu-nacional-fire-rio-de-janeiro-natural-history/}{Seen Here}}

Had the files been wholly digital, they could have been more easily duplicated, and would have been likely hosted in a different site from the physical artifacts.


That leads to the third problem with physical over digital media.

Yes, you can duplicate a physical document.

But, as the recent burning of the library shows, even drastically important documents aren’t always backed up.

What guarantee does a random, insignificant citizen of the world have that anything they write will ever be relevant to historians?

Even if it isn’t, the internet protects and safeguards it.

Yes, we may not have as good of a guarantee of this generation’s Beowulf surviving on parchment.

But, we have a much higher chance that the random thoughts of a random individual will be as accessible to future generations as that epic.


That truly is the heart of the internet.

It doesn’t promise permanence, it promises equality.

Every work hosted on wordpress is just as likely to be there in fifty year’s time.\footnote{barring the author destroying it}

And that, along with the different archiving methods, brings to the next point.

This is the first time in human history where we can not only see what was written, but pinpoint to the exact second when a piece is written, edited, or deleted.

We no longer can question which draft of a manuscript is older.

\end{document}
