\documentclass[12pt]{article}[titlepage]
\newcommand{\say}[1]{``#1''}
\newcommand{\nsay}[1]{`#1'}
\usepackage{endnotes}
\newcommand{\1}{\={a}}
\newcommand{\2}{\={e}}
\newcommand{\3}{\={\i}}
\newcommand{\4}{\=o}
\newcommand{\5}{\=u}
\newcommand{\6}{\={A}}
\newcommand{\B}{\backslash{}}
\renewcommand{\,}{\textsuperscript{,}}
\usepackage{setspace}
\usepackage{tipa}
\usepackage{hyperref}
\begin{document}
\doublespacing
\section{\href{villanelles-2.tex}{On Villanelles (Redux)}}
First Published: 2022 October 14

\section{Draft 1}
Well, I have once again fallen behind on both poems and writing blog posts.\footnote{also stretching, but that's not relevant here.}
There's a Tumblr page that posts weekly prompts for short writings, and last Friday I responded to it with a poem.
I kind of like the idea of continuing to do that, and sonnet form is starting to get boring to me.

So, it feels like a good time to break back out villanelles.
I've written about them \href{on-villanelles.html}{before}, but they remain a fascinating poetic form to me, for all that they're significantly harder to write.

\end{document}