\documentclass[12pt]{article}[titlepage]
\newcommand{\say}[1]{``#1''}
\newcommand{\nsay}[1]{`#1'}
\usepackage{endnotes}
\newcommand{\1}{\={a}}
\newcommand{\2}{\={e}}
\newcommand{\3}{\={\i}}
\newcommand{\4}{\=o}
\newcommand{\5}{\=u}
\newcommand{\6}{\={A}}
\newcommand{\B}{\backslash{}}
\renewcommand{\,}{\textsuperscript{,}}
\usepackage{setspace}
\usepackage{tipa}
\usepackage{hyperref}
\begin{document}
\doublespacing
\section{\href{board-game-night.html}{Family Board Games}}
First Published: 2023 August 30
\section{Draft 1}
Last night my family had our first\footnote{virtual} board game night.
It was a really fun time, for all that it was much shorter than our historic marathon nights.
We checked that we were able to play two games,\footnote{we were} and we played through one of them.

Playing Ticket to Ride virtually is a much different experience than playing it live.
Thankfully, it is not a game that has significant intrigue, and so the fact that I was playing with the rest of my family from behind a screen didn't make too much of a difference.
It was also really nice to not have to carefully place each of the incredibly small plastic trains on the page.
The game layout also makes it incredibly clear how much of the game is left.\footnote{for those who've never played, Ticket to Ride ends when someone (almost) runs out of trains. In the digital version, there's a constant tracker of how many trains everyone has.}

One interesting feature that the game offered is that, like some variations of chess, each player only has so much time to make every choice they will over the course of the game.
I chose forty\footnote{it bothers me every time that I write the word that it doesn't have a u} five minutes per player, because that was the default for the game, and I didn't know what changing it would do.
I am now curious what would happen if you ran out of time, though.
Would it skip your turn?
Make you queue your move?
I should try it sometime.

\begin{itemize}
\item No work on presentations. I might try to spend some time tomorrow or this weekend on it.
\item I cleaned even more minimally.
I want to try kegging instead of using a secondary for my next batch of lemon wine, though, which means I need to clean soon.
\item Writing my blog continues to go well.
\item I finished up the next chapter today and plotted out the next chapter.
I should really plot out the book a little more than what I'm doing, which is sentences at a time for the future books, five total sentences for this book, and ideas for exactly the next chapter. 2/4+
\item I wrote no poetry yesterday. Sadly.
\item I posted the letter, which is progress. I did not write, though.
\item Maybe stretching was a bad goal.
\item Less trouble falling asleep, though I still felt the need to sleep in.
Unlike yesterday, I feel very comfortable with that choice.
\item I really need to get back into the swing of praying.
I haven't, and I regret it.
\end{itemize}

\end{document}