\documentclass[12pt]{article}[titlepage]
\newcommand{\say}[1]{``#1''}
\newcommand{\nsay}[1]{`#1'}
\usepackage{endnotes}
\newcommand{\1}{\={a}}
\newcommand{\2}{\={e}}
\newcommand{\3}{\={\i}}
\newcommand{\4}{\=o}
\newcommand{\5}{\=u}
\newcommand{\6}{\={A}}
\newcommand{\B}{\backslash{}}
\renewcommand{\,}{\textsuperscript{,}}
\usepackage{setspace}
\usepackage{tipa}
\usepackage{hyperref}
\begin{document}
\doublespacing
\section{\href{motivation.html}{Motivation}}
First Published: 2019 January 27

Prereading note: in the interest of my attention span and available time, I left out a lot of the discussion, as well as a lot of the reasoning I have for feeling this way.
Mea culpa.
\section{Draft 1}
To many people,\footnote{these weasel words show that I don't want to claim it's true} there are two kinds of motivation: extrinsic and intrinsic.
And, to most of these same people, it's believed that intrinsic motivation is better.
Mostly, this is because external sources may lead you astray or not exist to keep you working, or so the theory goes.

However, I claimed that I am extrinsically motivated.
The inevitable straw-man argument came out, namely that since I choose what to listen to, I must be intrinsically motivated.
But, that's not what I was saying.
What I was saying\footnote{and believe} is that, in my experience so far, I choose activities because of external motivators.
Continuing, I've realized I even stop doing activities when\footnote{positive} external motivations disappear.

The most clear examples I can think of are why I joined the two choirs I sing at in college.
For one, an audition choir, I received an email in June from the director saying that it would be a shame to let the hard work my high school choir director had put in go to waste.\footnote{it parsed a lot better in the email}
While getting a signature for that ensemble, my advisor told me that I should also join the early music group.
Once in both, I received more external motivation to remain in them than I can think of easily.

Whenever I\footnote{rightly} expressed that I felt I brought the average quality of singer down in the first ensemble, I was told that I was a valued member of the ensemble.
Whenever any other early musician learns that I play the cornetto, they\footnote{for some reason} express astonishment, as it's known as a hard instrument.

And, the final point to making extrinsic motivations work\footnote{for me} is that I've learned where to seek external motivators.
When I want to feel as though I've done a good job, I know where and who to talk to.
When I want an excuse to not do something, I do the same.
So, while to some, it may seem that I am intrinsically motivated, it helps me live my life better knowing that I am extrinsically motivated.
\end{document}