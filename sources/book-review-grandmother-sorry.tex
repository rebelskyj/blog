\documentclass[12pt]{article}[titlepage]
\newcommand{\say}[1]{``#1''}
\newcommand{\nsay}[1]{`#1'}
\usepackage{endnotes}
\newcommand{\1}{\={a}}
\newcommand{\2}{\={e}}
\newcommand{\3}{\={\i}}
\newcommand{\4}{\=o}
\newcommand{\5}{\=u}
\newcommand{\6}{\={A}}
\newcommand{\B}{\backslash{}}
\renewcommand{\,}{\textsuperscript{,}}
\usepackage{setspace}
\usepackage{tipa}
\usepackage{hyperref}
\begin{document}
\doublespacing
\section{\href{book-review-grandmother-sorry.html}{Book Review of My Grandmother Asked Me to Tell You She's Sorry}}
First Published: 2022 April 11
\section{Draft 1}
One advantage of the Libby app is that I am able to put a book on hold so that when it is available, I can download it for fourteen days.
Since many of the books I want to read others also want to read, it means that I get books spaced out, instead of needing to read every interesting book at once.

One downside of this is that I often have forgotten why I wanted to read the book in the first place.
That happened with one book that I read recently, Fredrik Backman's \textit{My Grandmother Asked Me to Tell You She's Sorry.}
It's the story of a young girl whose grandmother dies of cancer.

The girl is bullied a lot at school, and her relationship with her grandmother revolved around the stories that she would tell of different fantasy lands.
When her grandmother dies, Elsa receives different letters that must be delivered to each of the different people who live in the apartment complex with her.
Along the way, she learns about the many ways that her grandmother was a complex person who had relationships with everyone in the complex.

I loved the book a lot.
The first half or so was a struggle to get through, but I don't think that I have ever cried more while reading a book.
I couldn't keep tears out of my eyes for around the last quarter of the book.
All in all, I would highly recommend the book.
\end{document}