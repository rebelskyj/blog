\documentclass[12pt]{article}[titlepage]
\newcommand{\say}[1]{``#1''}
\newcommand{\nsay}[1]{`#1'}
\usepackage{endnotes}
\newcommand{\1}{\={a}}
\newcommand{\2}{\={e}}
\newcommand{\3}{\={\i}}
\newcommand{\4}{\=o}
\newcommand{\5}{\=u}
\newcommand{\6}{\={A}}
\newcommand{\B}{\backslash{}}
\renewcommand{\,}{\textsuperscript{,}}
\usepackage{setspace}
\usepackage{tipa}
\usepackage{hyperref}
\begin{document}
\doublespacing
\section{\href{flash-fiction.html}{Flash Fiction Friday}}
First Published: 2023 November 10

\section{Draft 1}
There's a tumblr called Flash Fiction Friday Official.
I'm not entirely sure why they feel the need to call themselves official, given that I haven't seen anyone else claiming the title, but it's that nonetheless.
The concept is fairly simple.
Every Friday, they\footnote{since it's a collective, as I've learned} post a new prompt, and anyone can respond to the prompt for the next 24 ish hours.

I've done it a few times.
At first, I tended to do some short form fiction.
As time went on, and I felt less inclined to write small short stories, I used it as a way to write some poetry.
Of course, I've fallen almost completely off the wagon of doing the prompts at all.

Today, since I'm a little over a week into my one month goal of writing as much as possible, I looked to see what the prompt was.
It was \say{By any other name}, which of course made me initially think of the Shakespearean reference.
I immediately wrote that line down and thought about how untrue it was in so many situations.

Of course, that immediately pushed my mind into something poetic.
I don't think that I want to write a poem today for the prompt, if only because poems take me far longer per word and I haven't really been writing any poetry recently.
So, that means I need to write prose.

One interesting thing I've realized about my FFF\footnote{the obvious initialism for the site} responses is that they tend to be far more emotionally driven and first person centered than most of my writing.
In part, I think that's because they tend to be far more poetic, and I find that my poetry tends to be much more emotionally connected than my prose writing.\footnote{is there something to unpack in that?
Probably.
Do I have any intention of unpacking it?
Not even a little right now}

Since I didn't want to write a poem, I then started thinking of what story I could tell.
Because the prompts are \say{Flash Fiction}, there is a maximum word count of I think 1000.\footnote{though I should really confirm this number}
That's both a lot of words and not very many all at once.

I'm not writing fanfiction, unlike a large number of the people who respond to the prompts.\footnote{not in a judgemental way, just in a statement of fact way}
In some regards, this adds to the struggle.
In the very limited words I have, I need to not only introduce characters and scene, but then have something happen.
When working with fanfic, these constraints are slightly different.

With a single name, you can give the audience the entire background of a known character.
There are of course many difficulties to writing fanfic that are not true of an original story.
Because I'm creating the entire world wholesale, nothing I do in the 1000 or fewer words can be contradictory to what people know about the characters.
If I was using established works, though, then I would have to be careful to explain why my own interpretation diverges from the canon, if it does.

I'm not planning on writing the actual story for the day in this blog post, if only to keep the slightest barrier between the different online presences I maintain.
I do want to spend some time and words considering the story I want to write, though.

I think that I want to talk about how changing a name does, in fact, change the thing it describes.
Maybe it's just that I spend too much time around scientists, but there's a lot of thoughts around me that language does not, in and of itself, contain any meaning or power.
I'm sure the fact that I read lots of fantasy inspired by Earthsea also leads me to the idea of names being important in and of themselves, but.\footnote{but nothing, honestly.
It absolutely does, and it's wild to me that people forget that she kind of created that entire genre trope.}

Probably because of the literary inspirations, I have the initial idea to write something fantasy adjacent.
I've also got the voice in my head telling me that I should write a very emotional and relationship focused story, for a few reasons.\footnote{in short, because of the aforementioned desire to do things poetically, and because I don't do much of that writing generally, so practicing seems like a good idea right now.}
Something that keeps popping into my mind is also the idea of how a relationship is, in many regards, defined by the words we use to describe it.

I don't want to write a didactic story, where I expound on that explicitly.\footnote{though why I don't is probably worth investigating.
It certainly has something to do with the fact that I don't love didactic fiction, but there's probably more to it than that}
However, being too subtle comes with its own drawbacks.
I don't have hundreds of thousands of words to carefully dance around a topic.
The medium really encourages direct, if not blatant writing.

Ok, I do feel pretty strongly that it's in my best interest as a writer to do some realistic fiction exploring relationships.
Now I suppose I should figure out the perspective.
First person has the advantage of feeling intimate right away.
Third person has the advantage of seeming objectivity, which contrasts to the emotional statement that I'm trying to make.
Second person has the advantage of being a little weird, and in many ways feeling even more intimate.

From that little reflection, it definitely seems like I'm leaning towards second person.
Ok, so then the question becomes who I'm addressing.\footnote{obviously the reader, but that's too meta for me right now}
That question also goes hand in hand with the presentation style.

Lately I've really fallen in love with epistolary fiction.\footnote{that is, fiction told through letters.}
Some famous examples include much of Dracula and This is How You Lose the Time War.
I like the fact that there's an explicit passing of time, and I also like the way it connects me to the letters I write to friends generally.

Ok, so then the question becomes how many letters to send and what story/ plot I want to send out.\footnote{hmm, is this what people mean when they say that they're plotting a book?
Because I kind of like this, and would like to get better at it.
Right now my NaNo book is kind of stalled because I feel like I've got nothing to do but fill space before the final climax of the book, since I've hit more or less every plot point that I meant to hit.}
So, things that I know I want are a relationship developing.
Since I have such a limited word count, it could make some sense to start not at the first meeting.

I think I want it to take place more or less in the present, so I probably need to address why they're writing letters to each other.
Starting it with just like \say{I see why Dr. so and so recommended we write each other letters} could be fun.
Ooh hey, look at that, we've got an implication of them going through marital issues.
Ok so I guess that is a question, do I want the couple to be married already?

Ooh, something fun I could do is start each of the letters with a different name, like (person), then love, then etc..\footnote{it feels weird to put two periods back to back, but I think that's the proper stylizing for etc. at the end of a sentence.
I suppose I could look it up, but that seems like a lot of work that I don't really want to do.}
That feels maybe a little forced, but maybe it could work.
With a thousand words, I have room for five to seven of the standard length letters I write, which is pretty nice for storytelling purposes.

Ok to frame this for me\footnote{since I'm about to go try and write the story, will report back with findings}: second person, directed in an epistolary fashion, where I use the openings of each letter as a way of framing what the emotions are at each point.
I should probably brainstorm a few different names, but I think I'm ready to try writing the story for today!
Exciting.

Well, I have a rough draft done.
It's about 500 words, which is nice and in the middle of the allowed word count.
I don't know if I love the story as a whole, so far, but I think that there's some potential in what I have.
I think that author's notes are allowed, so I might throw one in and go \say{so this is what I was trying to do, idk if it worked though.}
There's always a part of me that cringes at author's notes that try to explain what their goal was in creating a work, but I don't really know if there's any way around it.
Anyways, I feel like I should probably try to revise the story.

Update a few hours later: I no longer feel like there's enough time in the day for me to feel like I want to edit the short story.
I also don't know if I'll post it.
On the one hand, it isn't bad writing, and I kind of like the story.
On the other, I can tell a bunch of ways that it could be much stronger, which is almost always true for something I write.
I don't know what my goal is in putting the writing out there, I guess.

I suppose that for the people who make the prompts, seeing responses is probably validating on some level.
They do also tend to repost anything you submit with compliments.
Ok, the more I think about it, the more that I feel like I should just post it and accept that it's not as strong as it could be.
Maybe I'll be lucky and someone will be willing to comment on how to make the story better.\footnote{for all that I do sincerely doubt that will happen. The site is not super encouraging of unprompted criticism, which I generally think is a positive.}

Well, it's posted now.
We'll see if I get any interaction with it.
In either case, I will probably forget it exists for a while, because I posted it from my writing tumblr, which is very far from connected to my real one.
My real one is also kept fairly isolated from any other social media that I keep, which probably says something, but I don't really want to get into what it says.

I might try to look at the post/ the prompt in the future when I run out of things to write about, but for now, I'm just going to leave it in the drafts.

For all that this is probably a rambling mess, I think that it was actually really helpful for me to think about what I wanted to write today.
Maybe I should spend more time doing free writing to plan out the books/whatever I want to write in the future.
It's certainly something that's worth thinking about.
Maybe tomorrow before I write for NaNo I'll spend some time thinking about where, exactly, I'm trying to lead the story.

Daily Reflection:
\begin{itemize}
\item Did I write 1700 words for NaNoWriMo? I did! I also have in the intermediate days that I haven't posted here.\footnote{more on that later}
\item Did I write a chapter of Jeb?
I finally finished the chapter I've been working on for a few days and I'm halfway through another chapter.
I'm hoping to have enough energy to finish it off before I go to bed, but we'll see.
\item Did I blog? So I have managed to write a post every day this month.
The past two days, though, I haven't felt comfortable posting them to the semi public reality that is my website.
Is that still a blog post? Great question.
\item Did I stretch? It's been far too long since I stretched. After finishing this, though, I'll do it.
\item Am I doing better at prayer than a rushed and thoughtless rosary? No. My prayer life is starting to suffer, which in part is probably because of how sick I've been feeling these past few days.
Still, that isn't really an excuse, so I'll make sure to pray tonight.
\item Am I doing a good job writing letters to friends?
I'm ready to write letters tomorrow, but I'm ok with the fact that I didn't write any today either. I'm sick and I need to be ok with that fact.
\end{itemize}


\end{document}