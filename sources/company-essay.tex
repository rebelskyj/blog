\documentclass[12pt]{article}[titlepage]
\newcommand{\say}[1]{``#1''}
\newcommand{\nsay}[1]{`#1'}
\usepackage{endnotes}
\newcommand{\1}{\={a}}
\newcommand{\2}{\={e}}
\newcommand{\3}{\={\i}}
\newcommand{\4}{\=o}
\newcommand{\5}{\=u}
\newcommand{\6}{\={A}}
\newcommand{\B}{\backslash{}}
\renewcommand{\,}{\textsuperscript{,}}
\usepackage{setspace}
\usepackage{tipa}
\usepackage{hyperref}
\begin{document}
\doublespacing
\section{\href{company-essay.html}{Company Essay}}
First Published: 2018 November 19
Prereading Note: As this was an assignment with a corresponding word count, each draft ends with a word count.

\section{Draft 4 19 Nov}
Steven Sondheim has used much of his career to push against traditional ideas of what a musical should be.
Throughout his shows, he subverts traditional musical theatre ideals such as love, marriage, and idealism.
In \textit{Company}, he not only subverts the idea of how love should be portrayed, but he also dispenses with the idea of linear plot.
He does not even stop there.
Sondheim and Furth's \textit{Company} moves even one step beyond atemporal theatre, dispensing on the idea of a moving plot almost entirely.
Instead of presenting a narrative of growth and change, \textit{Company} instead presents a narrative of reflection, with little sense of time passed between scenes.
As staged by Marianne Elliott, \textit{Company} is a show in two pieces: the recurring thread of a dreaded 35\textsuperscript{th} birthday, and a series of vignettes showing Bobbie's relationship with her married friends.

The birthday party is staged four times in the initial script, at the beginning and end of each act.(Sondheim \& Furth 3,71,75,117).
Elliot chooses to add a fifth staging in the middle of the first act.
Additionally, rather than staging each birthday the same way, each time we see Bobbie celebrating her birthday, the scene has changed.
By staging each birthday differently, Elliot drives the idea of \textit{Company} as a reflective, rather than actioned show.
Each time Bobbie remembers the scene, she's in a different mental state, and so remembers it differently.

The show opens with \say{(Bobbie's) empty apartment} (3).
She is holding balloons spelling the number 35 and a cake sits on the table.
The other actors \say{enter from various parts of the stage (in the production, also from the audience) and group themselves around (her) apartment,} where they speak about the nature of gifts they have brought for the celebration, and almost all exhort her to \say{just take (the present) back} (4,5).
From the dialogue of her friends, delivered \say{intoning}, it is clear from the very beginning that what we are seeing is a recollection of the birthday party, not the actual party.

Later in the act, we see the initial room scaled down, complete with a cupcake and small balloons.
Bobbie squeezes through the doorway, as she can no longer simply walk through.
As this staging is not present in the initial script, there is no dialogue here.
Instead, Bobbie simply looks at the mini bottle of bourbon and, shrugging, drinks it.
It's clear that she's meant to be more accepting of her age now than she was just a few minutes ago, when the show began.
Of course, it is tempered from the fact that she'd just finished smoking marijuana with friends (36-50).
Given that marijuana is known to cause hallucinations, it makes sense that the most dream-like of the birthday parties would show up while she's \say{stoned} (36).

To close the first act, Bobbie stands outside the set, which before had only happened between scenes, or as a way for characters to break into scenes where they didn't belong (mostly during musical interludes).
This time however, Bobbie is still meant to be holding the audience's attention, as a spotlight is left on her.
In the initial script, it calls for \say{all the birthday guests (to look) at (Bobbie) as in Act One, Scene One} (71), while the production has a doppelganger be the object of her friends' attention.
In this way, it shows that Bobbie has finally begun to distance herself from her memories.
She can now see what was, without forcing herself to define it as what is.

The second act begins like the first, with Bobbie entering an empty white room.
Unlike the first scene, however, Bobbie is forced to deal with oversized, comically large balloons.
She deflates them, and her friends come in and begin to sing about how wonderful she is.
When Bobbie moves, however, her friends do not change their blocking to reflect it.
By leaving the blocking as if she hadn't moved, choreographer Liam Steel reminds the audience that Bobbie is looking back on the memories.
Her friends are gesturing to the memory of her, which both she and her friends can see.
However, the audience, as observers, cannot.

The second act progresses until, as expected, the birthday scene arises again.
Interestingly, the initial script calls for Bobbie to be observing the stage \say{unbeknownst to us} (118) while the show has Bobbie watching from upstage, in the same undefined space that she had occupied at the end of the first act.
Of course, both shows have the same intent.
Bobbie is watching the stage, hoping to avoid her friends.

The turnaround that this scene represents is a great way to see that nothing has changed during the duration of the show.
Rather than Bobbie being in the room before her friends appear, her friends are in the room waiting for her to appear.
But, when she fails to enter, they leave and she enters.
In such a way, we see that the scene was reversed, and we're back to where we began.

But, there's far more to this show than simply the birthday sequences.
While the birthdays provide a framework and regrounding to remind the audience that the show is repeating, the vignettes are where it truly becomes obvious that we are living in a dreamscape.
The show opens to a single, solitary, white room.
It is ringed with neon lights, and Bobbie enters.
As her friends appear, they crowd the stage.

But, after the party has ended, rather than simply dropping a curtain and changing the scenes, Elliott and Bunny Christie's set draws the audience into the transition.
A small hallway floats in from stage right, and Bobbi enters it, then begins to fumble with keys, giving time for the entire set to shift and another room to appear.
The fact that each room and set is wholly self contained after that reinforces the idea that, since memories feel self contained, so too do the rooms.
The few times that we see people occupying space outside of neon rimmed rooms is almost always during a musical interlude, which is clearly not meant to represent the remembered reality.

Speaking of music, rather than most musicals' approach to music, where it appears to grow naturally from the dialogue, such as a love-lost man musing about the beauty of his love, and slowly shifting to song, the music in \textit{Company} starts and ends abruptly.

The first musical break occurs in the first scene, as a mockery of the traditional musical opener (7).
Rather than having some portion of the cast or a dedicated story teller open the set by explaining the show, we hear and see dialogue and blocking as if a play until everyone suddenly begins to sing that \say{there was something we wanted to say} (10).
The next time we have a musical break happens much the same.

Bobbie is sitting, confused and somewhat frightened by the display her friends are presenting they spar.
Suddenly, Joanne enters and begins to sing.
The couples join in, as if nothing is odd, but Bobbie looks as confused as the audience feels.
This seems to happen throughout the entire show, where the audience sees Bobbie realizing that the songs her friends sing are not truly a part of the memory, but have been joined because she links the ideas.

Throughout \textit{Company}, the audience is slowly pushed into the idea that the show is not taking place in the present.
Rather, the entire show is designed to force the audience into the realization that the show is a representation of an aging woman's view of what she's missed by not marrying.
Rather than making it an easy ending, where she either realizes that she regrets it, or doesn't, Sondheim, Furth, and Elliot leave the audience to question whether Bobbie will seriously begin to seek a relationship
In all five of the couples we see, Bobbie sees elements she wishes to embody, but also elements she hopes to never see.
In such a way, the show gives the audience a more real view of a thirty five year old's view of the pros and cons of marriage than the typical musical theatre.
She is able to understand both how she would and would not benefit from marriage.

1374
\section{Draft 3 19 Nov}
Steven Sondheim has used much of his career to push against traditional ideas of what a musical should be.
Throughout most all of his shows, subversive elements to the status quo of musical theatre occur.
In \textit{Company}, this takes the role of its totally non-linear story.
However, not content to merely dispense with the normal idea of a linear progression through time, Sondheim and Furth take the show one level further, dispensing on the idea of a moving plot almost entirely.
Instead of presenting a narrative of growth and change, \textit{Company} instead presents a narrative of reflection, with little sense of time passed between scenes.

Marianne Elliott's staging of company is a show in two pieces, a birthday support structure, with vignettes interspersed for flavor.
The support structure comes from the five times we see Bobbi's birthday, once at the beginning and end of each act and once in the middle of the first act.
As a note, the initial script calls for only four birthday sequences (Sondheim \& Furth 3,71,75,117).
However, rather than staging each birthday the same way, each time we see Bobbi celebrating her birthday, the scene has changed.
In doing so, the show drives home the point that it is meant to be a travel through Bobbi's mind, with her looking at her celebration of 35 years changing based on how she feels.

The show opens with \say{(Bobbi's) empty apartment} (3).
She is holding balloons spelling the number 35 and a cake sits on the table.
The other actors \say{enter from various parts of the stage (in the production, also from the audience) and group themselves around (her) apartment,} where they speak about the nature of gifts they have brought for the celebration, and almost all exhort her to \say{just take (the present) back} (4,5).
From the dialogue of her friends, delivered \say{intoning}, it is clear from the very beginning that what we are seeing is a recollection of the birthday party, not the actual party.

Later in the act, rather than oversized balloons, we see the initial room scaled down.
Bobbi squeezes through the doorway, rather than being able to simply walk in.
As it was not in the initial script, there is no dialogue here.
Instead, Bobbi simply looks at the mini bottle of bourbon and, shrugging, drinks it.
This acceptance of her situation comes right after scene four, where in the midst of smoking with her friends, her exes sing a song about how horrible she is (40-50).
Given marijuana's propensity to cause people to hallucinate, it makes sense that the least real of the memory sequences would be shown here. 

Finally, to close the act, Bobbi stands outside the set, which before had only happened between scenes, or as a way for characters to break into scenes where they didn't belong (mostly during musical interludes).
This time however, Bobbi is still meant to be holding the audience's attention, as a spotlight is left on her.
In the initial script, it calls for \say{all the birthday guests (to look) at (Bobbi) as in Act One, Scene One} (71), while the production has a doppelganger be the object of her friends' attention.
In this way, it shows that Bobbi has finally begun to distance herself from her memories.
She can now see what was, without forcing herself to define it as what is.

Of course, the second act begins and makes us forget this is the case.
Bobbi enters, as she did in Act One.
Unlike the first scene, however, Bobbi is forced to deal with oversized, comically large balloons.
She deflates them, and her friends come in and begin to sing about how great she is.
However, when she leaves the center, where they've been gesturing, they continue to gesture to where she was.
This blocking helps to remind the audience that Bobbi is looking back on the memories.
Her friends are gesturing to the memory of her, which both she and her friends can see.
However, the audience, as observers, cannot.

As expected, the show ends how it had begun and ended up to this point, with Bobbi celebrating her birthday.
Interestingly, the initial script calls for Bobbi to be observing the stage \say{unbeknownst to us} (118) while the show has Bobbi watching from upstage, in the same undefined space that she had occupied at the end of the first act.
The turnaround is interesting, however.
Rather than her being there, and her friends appearing, her friends are there without her.
When she fails to enter, they leave and she enters.
The mirroring of staging shows us that nothing has really changed in the show.
We see that nothing has changed from beginning to end. 

But, there's far more to this show than simply the birthday sequences.
While the birthdays provide a framework we can analyze the show through, the vignettes are where it truly becomes obvious that we are living in a dreamscape.
The show opens to a single, solitary, white room.
It is ringed with neon lights, and Bobbi enters.
As her friends appear, they crowd the stage.

But, after the party has ended, rather than simply dropping a curtain and changing the scenes, Elliott and Bunny Christie's set draws the audience in.
A small hallway floats in from stage right, and Bobbi enters it, then begins to fumble with keys, giving time for the entire set to shift and another room to appear.
The fact that each room and set is wholly self contained after that reinforces the idea that, since memories feel self contained, so too do the rooms.
The few times that we see people occupying space outside of neon rimmed rooms is almost always during a musical interlude, which is clearly not meant to represent the remembered reality.

Rather than most musicals, where the music appears to grow naturally from the dialogue, such as a love-lost man musing about the beauty of his love, and slowly shifting to song, the music in \textit{Company} starts and ends abruptly.

The first musical break occurs in the first scene, as a mockery of the traditional musical opener (7).
Rather than having some portion of the cast, or a dedicated story teller open the set by explaining the show, before inviting the audience in, we hear and see dialogue and blocking until suddenly everyone starts singing about the fact that \say{there was something we wanted to say} (10).
The next time we have a musical break happens much the same.

Bobbi is sitting, confused and somewhat frightened by the display her friends are presenting they spar.
Suddenly, Joanne enters and begins to sing.
The couples join in, as if nothing is odd, but Bobbi looks as confused as the audience feels.

Throughout \textit{Company}, the audience is slowly pushed into the idea that the show is not taking place in the present.
Instead, the entire show is built to force the audience into the realization that the show is nothing more than a representation of an aging woman's view of what she's missed.
Rather than making it an easy ending, Sondheim, Furth, and Elliot leave the audience to question whether Bobbi will seriously begin to seek a relationship
In all five of the couples we see, Bobbi sees elements she wishes to embody, but also elements she hopes to never see.
In such a way, the show gives the audience a more real view of a thirty five year old's view of the pros and cons of marriage.

1247 Words
\section{Draft 2 19 Nov}
Steven Sondheim has used much of his career to push against traditional ideas of what a musical should be.
Throughout most all of his shows, subversive elements to the status quo of musical theatre occur.
In \textit{Company}, this takes the role of its totally non-linear story.
However, not content to merely dispense with the normal idea of a linear progression through time, Sondheim and Furth take the show one level further, dispensing on the idea of a moving plot almost entirely.
Instead of presenting a narrative of growth and change, \textit{Company} instead presents a narrative of reflection, with little sense of time passed between scenes.

Marianne Elliott's staging of company is a show in two pieces, a birthday support structure, with vignettes interspersed for flavor.
The support structure comes from the five times we see Bobbi's birthday, once at the beginning and end of each act and once in the middle of the first act.
However, rather than staging each birthday the same way, each time we see Bobbi celebrating her birthday, the scene has changed.
In doing so, the audience becomes confused as to the reality of the show.
Is the goal to see the five ways that Bobbi's birthday could go, as she reflects on the future?
Is it the way that she remembers it based on how she's feeling at that exact moment?
Or, is it a reflection on the slowly deteriorating mind of a woman attacked from all sides?
The show doesn't seek to answer this question.

As exciting as each of the birthday sequences is, the vignettes are where the life of the show comes through.


Ok so predraft time again so i don't keep flow of consciousness writing

Company tells non linear story
This is expressed through the scenes that have no relation or reference to each other, the birthday parties.
So what? it's cool idk im sure something will come when I get there
Wait i need an argument.
Because "non linear story" is sort of a given, maybe yeah company is about reflection.

Moving on.

The staging of the vignettes supports the thread of remembered experience far better than the birthday parties, though.
Every part of the performance suggests this sort of memory, from the staging, to the blocking, to even the dialogue.

When Bobbi  (restarting so it's clean)

\section{Draft 1 18 Nov}
Much of what low level analyses of a show attempt to do is explain what happens in the show.
At a first glance, that makes sense.
When describing an event to people, we have an innate desire to make it fit a narrative story.
Things begin, cause other events, and then end.

Of course, theatre is often seen as a way to reflect life.
Unlike life, we get happy endings, which resolve the problems and assure the audience that everything will work out.
Like life, however, things begin.

Well, that is, in most theatre.
Sondheim, Furth, and Elliott's \textit{Company: A Musical Comedy}, on the other hand, does not attempt to convey a narrative.
Instead, it presents a series of vignettes, few of which seem to occupy any chronology.
In fact, many of the vignettes overlap each other, just as our own memories overlap and rewrite information.
\textit{Company: A Musical Comedy} attempts to use its plot not to advance or tell a story, but rather as a reflective view of a 35 year old woman's tangle with her relationships.

The clearest way to see that the show is meant to convey a sense of recollected time, rather than transpired time, is in its use of a looped narrative.
Throughout the show, we see Bobbi's 35\textsuperscript{th} birthday party a total of 5 times, despite the script only specifying 4.

However, rather than having the celebrations remain constant, the script calls for them to change.
The show opens with \say{(Bobbi's) empty apartment.} (Sondheim \& Furth 3)
She is holding balloons spelling the number 35 and a cake sits on the table.
The remainder of the ensemble \say{enter from various parts of the stage (in the production, also from the audience) and group themselves around (her) apartment,} where they speak about the nature of gifts they have brought for the celebration, and almost all exhort her to \say{just take (the present) back.} (Sondheim \& Furth 4,5)

The next time we see the birthday celebration is at the end of the first act.
Bobbi stands outside the set, which before had only happened between scenes, or as a way for characters to break into scenes where they didn't belong (mostly during musical interludes).
However, this time, Bobbi is still meant to be holding the audience's attention, as a spotlight is left on her.
In the initial script, it calls for \say{all the birthday guests (to look) at (Bobbi) as in Act One, Scene One,} (Sondheim \& Furth 71) while the production has a doppleganger be the object of her friends' attention.

In between these two pictures of her 35\textsuperscript{th} birthday party, we are exposed to a series of vignettes.
In the first, Bobbi is at the home of one of the couples.
She enters by walking through a doorway into another self supported neon room.
It's clear that she's digging through her memory for the \say{key} to unlock the memory from her struggles.
When she opens the room, she takes her seat and the lights open on the room.

She chats with friends, until the musical interlude, where Joanne enters and speaks about the joys of being in a stable, committed relationship. (Sondheim \& Furth 26)
The rest of the ensemble joins in, until the song ends.
As if nothing had happened, the scene resumes.

The next scene simply has Bobbi chatting on the porch to a different couple.
There is nothing to suggest whether this happened before or after the initial scene.
This happens a few more times, and then we see the act closing party.

There are only two constants to the entire act.
First, Bobbi always has a drink of bourbon in her hand, and second, she's always in the same red dress.
The reset of the characters go through costume changes.

When the second act begins, it looks a lot like the first.
However, this time, Bobbi is forced to deal with oversized, comically large balloons.
She pops them, and her friends come in and begin to sing about how great she is.
However, when she leaves the center, where they've been gesturing, they continue to gesture to where she was.

Later in the second act, rather than oversized balloons, we see the initial room scaled down.
Bobbi squeezes through the doorway, rather than being able to simply walk in.

The show ends as it did the second act, with Bobbi not celebrating her birthday while her friends do.
This time, however, there is no other Bobbi.
Instead, all of her friends are there awaiting her arrival, so as to wish her a happy birthday.
As she fails to arrive, however, they grow demoralized and leave.

Only once she is sure that her friends have departed does she enter the room.
She sees the candles her friends had left burning, and, rather than try to extinguish them with her breath, as she had each of the prior times, she uses a fire extinguisher.
Of course, there is still one candle burning after this.
She blows it out, and the lights fall from the stage.

843 words
\section{Draft 0 18 Nov}
Sondheim, Furth, and Elliott's \textit{Company: A Musical Comedy} attempts to use its plot not to advance or tell a story, but rather as a reflective view of a 35 year old woman's tangle with her relationships.

Throughout the show, we see Bobbi's 35\textsuperscript{th} birthday party a total of (4? 5?) times.
From memory, there's the first one, the one with the friends facing the fake Bobbi, the one where the balloons are gigantic, which i think was the second act opener, the one where the entire room is tiny, and the ending one.
So that would be five times.
Great.

Throughout the show, we see Bobbi's 35\textsuperscript{th} birthday party a total of 5 times.
Each of the times we see the celebrations, they have changed.
The first time we see the party, Bobbi enters a white room, ringed in neon.

She is holding balloons spelling the number 35 and a cake sits on the table.
Slowly, her friends enter.
They speak about the nature of gifts they have brought for the celebration, and almost all exhort her to \say{just take (the present) back.} (Sondheim \& Furth 5)

The second time we see Bobbi's celebrations is at the close of the first act.
She stands outside of the self-enclosed set, clearly meant to show that she is reflecting, rather than remembering.
This impression is reinforced by the view the audience is presented with.
We see a second Bobbi, facing upstage so as to hide her face.

This Bobbi appears to be participating in the party.

When the second act begins, Bobbi again enters the single solitary room.
This time, however, there are gigantic balloons, which she viscously deflates.

Later in the act, she enters the room, which has become shrunk to less than half the size.
As she has in each scene, she takes a drink of bourbon.
This time it is from a minibottle, which she drinks straight.

The show ends as it begins, with Bobbi celebrating her birthday.
As with the first act's closer, she stands in the undecorated piece of the stage, and looks at her peers.
This time, however, there is no other Bobbi.
Instead, all of her friends await her arrival, so as to wish her a happy birthday.
As she fails to arrive, however, they grow demoralized and leave.

Only once she is sure that her friends have departed does she enter the room.
She sees the candles her friends had left burning, and, rather than try to extinguish them with her breath, as she had each of the prior times, she uses a fire extinguisher.
Of course, there is still one candle burning after this.
She blows it out, and the lights fall from the stage.

Word Count: 452
\end{document}