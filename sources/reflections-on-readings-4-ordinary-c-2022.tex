\documentclass[12pt]{article}[titlepage]
\newcommand{\say}[1]{``#1''}
\newcommand{\nsay}[1]{`#1'}
\usepackage{endnotes}
\newcommand{\B}{\backslash{}}
\renewcommand{\,}{\textsuperscript{,}}
\usepackage{setspace}
\usepackage{tipa}
\usepackage{hyperref}
\begin{document}
\doublespacing
\section{\href{reflections-on-readings-4-ordinary-c-2022.html}{Reflections on Today's Gospel}}
First Published: 2022 January 30

1 Corinthians 13:3: \say{If I give away everything I own, and if I hand my body over so that I may boast but do not have love, I gain nothing}

\section{Draft 1}
As always, it's fascinating to me to see how different my interpretation of the readings is three years later.
Last time I read these, I was struck by the commands in the readings, especially the first reading.
This time, I really hear the call to love more than everything.
Without love, there can be no gain in my life.

Tonight was also the first RCIA\footnote{Rite of Christian Initiation for Adults} class at my parish.
It was really interesting in conjunction with today's readings, because we talked about what our purpose in life is.
According to\footnote{my remembering of} the Baltimore Catechism, our purpose in life is to know, love and serve the Lord.

The part that stuck with me is that we love through helping our neighbor, and we serve the Lord through the liturgy.
I always sort of assumed it was the opposite, but it's really interesting to me how this ordering, service as liturgy and love as traditional service meshes more with Catholic teaching in my brain.
I know I had something else to add here, but I can no longer remember it.
\end{document}