\documentclass[12pt]{article}  
\newcommand{\say}[1]{``#1''}  
\newcommand{\nsay}[1]{`#1'}  
\usepackage{endnotes}  
\newcommand{\B}{\backslash{}}  
\renewcommand{\,}{\textsuperscript{,}}  
\usepackage{setspace}  
\usepackage{tipa}  
\usepackage{hyperref}  
\begin{document}  
\doublespacing  
\section{\href{succeeding-at-creating.html}{On Succeeding at Creating}}  
First Published: 2025 April 17

\section{Draft 2: 17 April 2025}  
In a wondrous display of kismet, for my break I started reading what I had thought was one of the very easy books of science I had checked out.  
Instead, it was\footnote{is?} a collection of essays from top scientists and mathematicians writing against reductionism.  
I'm struggling to articulate exactly what about that helped me with the sense of creating, but I think that I will do better here.

Science is, depending on the activity, a creative endeavor.  
There is minimal creativity involved in measuring a molecule, sure, but the knowledge that gives means that new questions can be asked.  
And, of course, measuring may not be so simple.

Einstein famously derided the parts of his relativity that suggested the existence of black holes.\footnote{thanks freeman dyson for that fact just now}  
It is now seen as one of the greater proofs of general relativity.  
Without getting into the quagmire I have with regards to knowledge, teaching, and revelation, I think that there is something to be said for the fact that we are terrible judges of ourselves.

Having written that line, I now realize that it sums up my feelings on succeeding at creating.  
I cannot know how something I craft will change the world, both because the future is unknowable and also because I am too close to it.  
Einstein saw errors when his math showed black holes; I know that there are other instances, but again, my mind runs dry.

So, I guess that what I should consider when judging the success or failure of anything I create is less whether the audience I intended is receptive to it, and more whether what I created can inspire something new.  
General relativity is a successful creation not in spite of its unexpected consequences, but in fact because of it.  
If what I make causes no new questions, leads to no new joy or inspiration, then it is a failure.

I think that is an answer I can accept.  
To succeed at creation is to create something which catalyzes another creation.

Catalysts are themselves something interesting that I should consider talking about at another time.  
They seem so strange, because they neither destabilize the starting material nor stabilize the product, nor even provide energy directly to a reaction.  
Instead, they simply make the changes easier to occur.  
Where does the effort to change go?\footnote{not in a literal, physical sense, in the metaphorical. I have faith that I could, if so desired, map energy flows}

If life is a chemical system, then the energy needed to overcome inertia and move to an easier path should go somewhere itself.  
To succeed at creating something, somehow that must make creation easier in a general sense.  
If ideas are in the ether, a plane separate from our own, is each act of creation a cracking of the window separating the two?

\section{Draft 1: 17 April 2025}

The goal of today's folly is to figure out what, exactly, it means to me to be successful at a creative endeavor.  
A better question might be why I think that it's important to succeed, but I don't really think so.  
After all, success just means accomplishing goals.  
There should be a goal behind the act of creation, at least in my mind.  
Now that I'm ok with the fact that I'm going to be thinking of creation as an endeavor one can succeed or fail at, let's think about what that means.

In this society, success is incredibly easy to define: fame.  
Some might argue for wealth, and I wouldn't fight them on that point\footnote{hm I must be hungry}, but it would not be my personal truth.  
So, should I judge the success of something I create based on how famous it makes me, or at the very least how much it improves my bank account?

No.

That's an easy enough answer.  
I don't want my life to be commodified.  
The harder question is whether or not I do judge success from that metric.

There is also the question of the professional creative.  
If my income comes\footnote{oh, income like in and come, like it comes in} from painting, then the world's willingness to pay for a painting is a metric of its success.  
Of course, that then means that we run into the locus of control issue.  
It's a famous and well known truth that very few artists are appreciated in their time.  
I don't think that it's reasonable to say that a work suddenly became successful hundreds of years after its makers death because someone decided to spend lifetimes of an average worker's income on it.

I guess what I'm trying to play with here is the idea of intent.  
That is, is success defined as how well you accomplished what you set out to accomplish?

What if you have multiple goals?  
What if you didn't think about what every goal was at the beginning of the project?  
What if your goals change?

So, if I don't think that we should judge success from external metrics and doubt our ability to judge it fairly from internal ones, where does that leave us?  
I'm not entirely sure.  
In the initial formulation for this folly, I was thinking about audience reception, how well your message was conveyed to the audience, and stuff of that nature.  
However, even just a little work in thinking shows me how all metrics fall short.

What are some ways that I could look at a work and judge it a success?  
Finishing something is a form of success all its own.  
Then again, what does it mean to finish a project?

If I make something for someone and they like it, is it a success?  
If they don't like anything on the day I gave it to them, but on a normal day they would, is that a success?  
If I make something perfectly according to a template but the receiver hates the template, would that be a success?

I don't know if my mind is working slower than normal, but I cannot seem to come up with answers, only more questions.  
Perhaps it is something of a self fulfilling prophecy: I said that this site was full of follies and now find myself unable to say anything of meaning.

In specific terms, did I succeed with this post?

I wrote it, which is one of my goals, and it will be finished soon, which is a form of success.  
I explored how I felt, even if only lightly.  
I tried to connect my new thoughts to my future actions, and I think that I did a decent job there.

Is there perhaps just minimal utility to judging success of an endeavor?  
Nearly everything I have done which felt judged or as a success thing was inherently comparative.  
Comparison does nothing to the work except harm it.\footnote{it can help with creation of a future work, but that's not the question}

A professor emeritus told me that the goal of every conversation should be setting up the possibility of a future conversation.  
So, is the success of a creative endeavor in how much it orients you to continue creating?  
We are not lone figures, though; is the success of creation how much it net orients the world to that craft?  
No exercise exists in isolation; is success how much it orients the world to creation?

What does it even mean to create?

I cannot enact any cause without the sum of the forces which have acted on me, regardless of how indirect.  
I cannot create any new energy,\footnote{matter is questionable, because of the whole interconversion of matter to energy} so is everything I do just rearranging?  
When I make music, it dies away seconds after I finish.  
What does it mean to create something so ephemeral?

Ozymandus reminds us that in the eyes of even just the human race, anything about us becomes ephemera.

Let's see if we can't reign this energy in.  
I don't know about another draft, but at the very least I want to see what stepping away for a bit does for me.  
I think that it helps me to take time away from writing, and I have another hour before my scheduled time comes to an end.

\section{Daily Notes}

\begin{itemize}

\item Obligations:

\begin{itemize}

\item Professional

\begin{itemize}

\item Write the thesis

I made progress here today! Wild how recentering myself does that.

\item Revise the thesis

\item Edit the thesis

\item Research for the thesis

A little bit! At the very least, talked with a group mate about some stuff I hadn't really thought of, which helped.  
Also ambushed a friend and got their help with some coding questions.

\item Read the books that might be useful for the thesis

\item Start citation tracking

\end{itemize}

\item Personal

\begin{itemize}

\item Learn the songs for to jam

\end{itemize}

\item Self:

\begin{itemize}

\item Silence

I worry that I might be getting too comfortable with silence, but that might be a ridiculous fear.

\item Typing practice.

I'm going to do it as soon as we finish this daily note

\item Keep the phone out of the room for bed

Nope, but I was also up for hours in the middle of the night last night, so being able to read was nice.

\item Pray St. Michael Chaplet in the morning

Too eepy, sadly.

\item Stretch in the morning

...

\item Read at night

Not the book I meant, but a book!

\item Poetry at night

\item Clean the home

Yes!

\item Stretching, standing, drinking water

Drinking water, at least.  
It's nice to start being more hydrated.

\item Posture

Eh, I think so.

\item No wasted time

I think still doing well here, no aimless scrolling, watching videos is occurring only as background noise, if even then.\footnote{which is part of why I'm worrying I might be too comfortable with silence}

\item Eat more than 2 meals a day

I think so! Both yesterday and soon to be today!

\end{itemize}

\end{itemize}

\item Goals and Growth:

\begin{itemize}

\item Ends:

\begin{itemize}

\item Letter writing, get into more

Nope!

\item Handwriting, pick and make the new one

Day two of hand journaling as a way to use up ink went well!  
I am really finding that forcing myself to use a fude tip does really quickly actually make writing with it enjoyable.  
I think that the penmanship is still a work in progress, and I'm wondering if a more looping print might still be preferable?

\end{itemize}

\item Means:

\begin{itemize}

\item Typing speed, improve it.

Right after notes here!

I want to work on getting more accuracy right now, I think.  
The issue with that is that I don't really have a great idea how to only type correctly in a way that also improves my typing speed.  
I guess that being conscious of each keystroke is probably the best way to do that, but I don't know that for certain.

\item Reading, do more of it

Yes! I am really enjoying this book series, and I think that I might have just forgotten to enjoy reading recently.

\item Blogging, do it

WOo!

\item Writing things that are not the blog and thesis, do

Eh, debatable. The morning journals have mostly been for me to recenter my ideas about work, but.

\end{itemize}

\end{itemize}

\end{itemize}

\end{document}