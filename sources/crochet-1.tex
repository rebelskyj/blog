\documentclass[12pt]{article}[titlepage]
\newcommand{\say}[1]{``#1''}
\newcommand{\nsay}[1]{`#1'}
\usepackage{endnotes}
\newcommand{\1}{\={a}}
\newcommand{\2}{\={e}}
\newcommand{\3}{\={\i}}
\newcommand{\4}{\=o}
\newcommand{\5}{\=u}
\newcommand{\6}{\={A}}
\newcommand{\B}{\backslash{}}
\renewcommand{\,}{\textsuperscript{,}}
\usepackage{setspace}
\usepackage{tipa}
\usepackage{hyperref}
\begin{document}
\doublespacing
\section{\href{crochet-1.html}{On Crocheting Continued}}
First Published: 2022 April 20


\section{Draft 1}
Crocheting is something I've blogged about \href{crocheting.html}{once before}.
I didn't talk much about it, which is fair.

The biggest project I've ever worked on for crocheting is a blanket that I began in December 2019.
I'm currently about 1/3 finished with it, though it's been at least a calendar year since I've worked on it, so I should get back to it.

Currently, I'm in the habit of making hats.\footnote{a hat-bit if you will}
I've been averaging one a week for the two and a half weeks I've been crocheting them.
This week's hat reminded me that I really prefer to crochet using yarn rated for a larger hook than the one I use, because otherwise the material ends up feeling too flimsy to me.
But, I neglected to realize this until I had finished an entire hat, so to make it thicker I'm now going through the entire design backwards and adding another layer of thickness.
In many respects, the way I make hats makes this much easier than it otherwise could be.

The way that I crochet hats is by making a circle of 10 double crochet in the round.\footnote{American notation}
I then, rather than adding a discrete row, just keep crocheting onto the first stitch again, making a spiral.
In the second \say{row}, I put 20 stitches in.
There are 30 in the third and so on until I get to the 70 stitches that's been working for me lately.
I should mention that I crochet into the back loop only, which adds a pretty spiral pattern into the hat.
This also worked well for me working backwards, because it gives me an easy place to crochet into.
Anyways, once I've reached the 70 stitches, I just keep going in the spiral without adding stitches until the hat gets long enough.

For this hat, I ended the string, because I thought I wanted to crochet from the top down again.
However, upon consideration, it appears that bottom up works better for the second layer, so I could have saved two ends by just going backwards from the bottom.

As I should say more often, c'est la vie.
\end{document}