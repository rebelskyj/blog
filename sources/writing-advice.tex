\documentclass[12pt]{article}  
\newcommand{\say}[1]{``#1''}  
\newcommand{\nsay}[1]{`#1'}  
\usepackage{endnotes}  
\newcommand{\B}{\backslash{}}  
\renewcommand{\,}{\textsuperscript{,}}  
\usepackage{setspace}  
\usepackage{tipa}  
\usepackage{hyperref}  
\begin{document}  
\doublespacing  
\section{\href{writing-advice.html}{Some Writing Advice}}  
First Published: 2025 April 14

\section{Draft 1}

I've written more than a few posts here about writing.  
Having just skimmed a few of them, it is really interesting to see the way that my own writing has developed in the years that I've been writing this blog.  
But, more importantly, I've realized that I have internalized a lot of writing advice that I find helpful at different times and places.  
Since I'm being asked about advice for writing more lately, an also because I'm writing more, it feels like it would be good to get it al down in one place.  
After all, extended memory is far better for me.

There are a variety of ways I could structure the advice, depending on whether my goal is prose or being used as a quick reference.  
If I wanted it to be reference, a list of some sort is probably the ideal.  
However, I think that part of what's important is also taking time, and not everything needs to be optimized for time efficiency.  
With that in mind, I'll be doing it as flowing prose.

It's hard for me to say what the most important piece of writing advice I have received or could give is.  
It does depend a lot on what the barrier to writing is.  
In general, I tend to hear that the biggest problem people have is starting, and so in general, advice on starting is the best.

More or less every book of prose advice I've read reminds its readers that no writing comes forth perfect from the pen or mind of a writer.\footnote{different prose books put the act of creation at different points. The more poetic tend to say it comes from the act of writing, while the more practical tend to say it comes from the person creating.}  
This is similar to the advice that nearly every doctoral student needs to be told at least once, though slightly different, \say{the greatest enemy of goodness is perfection.}  
That is, the idea that what we put on the page needs to be perfect can and often does mean that we are stuck in a hell of our own creation, unable to get anything on the page.  
And so, the first piece of advice is to remember that you cannot revise an empty page.  
Closely related, any writing is better than no writing.

I hope it's clear why these two are so essential.  
In more or less every situation where I find myself in need of writing, having literally anything on a page is preferable to having a blank page.  
I can think of a few places where it is not true, but even those situations, writing still needs to be produced eventually.

However, knowing that the writing needs to be done might not be enough to silence the voice in one's head assuring that the words we write will never be good enough.  
Different authors use different metaphors, but most involve consciously acknowledging\footnote{can you tell that I've been learning a bit about zen Buddhism? Also wow the three kinds of Buddhism seem really cool and different in fun ways, and I should explore them (as with all things, once I have my Ph.D. and once again have time to exist as an individual, of course)} the inner critic we have, and then moving past it.  
For some, this means putting the critic in a jail cell.  
For others, it involves killing the editor.  
In all cases, though, the advice can be boiled down to an acceptance that the writing we produce will not be good.

I've seen the extreme version of this advice from a writer I follow on social media.  
Not only should you accept that the writing will be bad, they say, you should explicitly write at the top of the document that you are intentionally writing something bad.  
Then, if the writing that you produce ends up being terrible, you were right.  
If, as it turns out, the writing was fine, then you have ended up with what was ultimately the goal.

The next piece of crucial writing advice moves past the specific project, and into the advice of becoming a better writer more generally.  
As I talked about \href{optimizing-lifting}{the other day}, there's something to be said for making sure that you want to be the kind of person who has a specific hobby.  
If you want to be a writer, it is important to write.  
And so, a great piece of advice for being able to write better is simply to write more.

I know that there's this idea ingrained in so many of us\footnote{or, at least, ingrained in me from my time in high school football} that we practice like we play.  
That is, mediocre practice results in poor habits and therefore poor final results.  
To some extent, I think that is true.  
If you never put forth effort on a drill, you're likely to not be able to put forth effort in the real situation.  
However, I think that, especially for creative tasks, it is important to remember what it means to have good practice.

There's \href{https://xkcd.com/1414/}{an XKCD}\footnote{wow the old internet joke that there's an XKCD for everything really does feel truer the more I write} about this very concept.  
Simply by writing more, you improve the parts of you that write.  
If you text your friend how you are feeling, you get better in touch with describing emotion.  
If you text your friend what you ate for dinner, you get better at conveying emotion through narrative, narrative, and descriptive writing.  
And so, the essential piece of writing advice is, quite simply, write more.  
Try some dumb writing prompt, rant about the dumbest movie you were forced to watch, pen the overly sentimental and soppy\footnote{sappy? no cool soppy is also a word, I do have a vocabulary, wild} love note that you could never in a million years actually send to the object of your desires.

Close to this, most advice on prose will also focus on the importance of consuming media as well.  
Top athletes point out the same.  
To improve at a craft, it is often very helpful to not just practice, but to also watch those who are good at what they do.  
So, if there's a genre or area you want to write in, reading in that area will help.\footnote{and now I am to help my undergrad with coding. Where was I?}

Moving from the broad back to the specific, structuring longer documents, especially longer stories, is often difficult.  
I've seen a variety of advices, which are really specific to the kind of writer that someone is.  
However, regardless of how much or little one enjoys planning books, there are likely to be scenes which are a struggle to think about.  
I've seen the advice of planning books as though they are road trips.  
That is, it's often good to simply know that you need to go from point to another without thinking about the exact path you may use.  
The path can be figured out later on.  
I've also seen the advice of starting from the very end of the book and working backwards.  
I hear that it makes the entire book much more fun, because you know exactly what you are working towards.

A piece of writing advice I think I saw from Ernest Hemmingway was to intentionally stop writing before all the words are gone.  
In doing so, you both avoid burn out and start to train your brain to want to write more often, because leaving desires frustrated is a great way to increase desire.  
It's closely related to another piece, which is that it's essential to also avoid burnout.  
While it may feel bad to write, sometimes that's just initial activation.  
Different authors set different thresholds, but most will say that, if after 500 words, it still feels bad, simply give up on writing for then.

And almost finally, it's often recommended, especially when editing and revising, to try your best to put yourself in the mind of the character.  
Do they react to stimuli the way that they should?  
If not, either change their reaction or change the character.

Finally, it can be fun to write multiple endings to a book simply to see how and where the plot would need to diverge in order for the new ending to be equally satisfying.  
I think that I've failed with that here.

\section{Daily Notes}

\begin{itemize}

\item Obligations:

\begin{itemize}

\item Professional

\begin{itemize}

\item Write the thesis

I did not, but I did spend some time talking with people about the Thesis.

\item Revise the thesis

\item Edit the thesis

\item Research for the thesis

I did some Gedankenexperiments, which might help me.  
It does feel weird to call ideas for running code Gedankenexperiments, but I think that's how it works.

\item Read the books that might be useful for the thesis

Have the book, and will read after writing time.

\item Start citation tracking

\end{itemize}

\item Personal

\begin{itemize}

\item Learn the songs for to jam

\end{itemize}

\item Self:

\begin{itemize}

\item Silence

I walked to work in silence today, and it was nice.  
I think that I just need to stop finding enjoyable content, and then the not listening becomes easier.  
I do need to also make time this week for music walks, though.  
That's like a form of silence.

\item Typing practice.

I did none yesterday, but yesterday was also just such a travel day, so I'm really ok with that fact.  
After writing with my partner today, I will again go back to practice before starting work for the day.

\item Keep the phone out of the room for bed

No, but I was finishing a podcast, so I'm somewhat ok with that.

\item Pray St. Michael Chaplet in the morning

I was so very behind this morning, and forgot that I had an actual scheduled event.  
Whoops!  
I did go to two seders this past weekend, though, which was also very prayer filled.\footnote{and probably something to think about, the way that my different heritages celebrate and exist and etc.}

\item Stretch in the morning

\item Read at night

Finished the podcast that I was advised to listen to.  
That was fun!

\item Poetry at night

I have not, but also now that I will soon have another fountain pen, this would be more doable perhaps.

\item Clean the home

Nope!

\item Stretching, standing, drinking water

I drank so little water this weekend and I can really feel it.

\item Posture

I've been doing better and better about this! I think that others are beginning to tell too, because there's definitely a different energy I'm getting from people lately.

\item No wasted time

Nope! But, I generally was ok with relaxing this weekend. Now that the week has begun, we're onto the grind season.

\item Eat more than 2 meals a day

... Sure! We'll say that happened.

\end{itemize}

\end{itemize}

\item Goals and Growth:

\begin{itemize}

\item Ends:

\begin{itemize}

\item Letter writing, get into more

Not so much, though I did realize that I have a few more people's addresses than I thought. I didn't remember to read the books on etiquette over the weekend, but if I do so today, that might be helpful.

\item Handwriting, pick and make the new one

Thought about it a little last night.  
I realized that my signature might be a great place to start, because I don't really need it to be a single swooping line, and figuring out how I want my letters to look there might also be great.

\end{itemize}

\item Means:

\begin{itemize}

\item Typing speed, improve it.

Spent a fair amount of time the other day, and am trying to spend some effort here writing with proper finger placement. I have hopes that this will in time become faster, but I have to wonder when I'm supposed to use each different shift key.

\item Reading, do more of it

Read a fair amount while at home!  
I do really love Street Cultivation, for all that it is just such a dark series.  
Very much it believes in the kind of world where power concentrates and abuses.

\item Blogging, do it

Look at this!

\item Writing things that are not the blog and thesis, do

Well, not everything can be a winner. I am planning to start a pen journal, and that should be fun.  
I've also been doing better about reaching out to friends which is fun!

\end{itemize}

\end{itemize}

\end{itemize}

\end{document}