\documentclass[12pt]{article}[titlepage]
\newcommand{\say}[1]{``\#1''}
\newcommand{\nsay}[1]{`\#1'}
\usepackage{endnotes}
\newcommand{\1}{\={a}}
\newcommand{\2}{\={e}}
\newcommand{\3}{\={\i}}
\newcommand{\4}{\=o}
\newcommand{\5}{\=u}
\newcommand{\6}{\={A}}
\newcommand{\B}{\backslash{}}
\renewcommand{\,}{\textsuperscript{,}}
\usepackage{setspace}
\usepackage{tipa}
\usepackage{hyperref}
\begin{document}
\doublespacing
\section{\href{atonal-precomposition.html}{Breaking the Octave: Atonal Preomposition}}
First Published: 2018 December
\section{Draft 1 (8 December)}
As a music major, sometimes I get asked which part of music history is my favorite for listening.
My answer of \say{everything that's before Baroque or after Romantic} doesn't tend to get received well, especially since that seems like such an odd answer.
But, there's a theme that connects them: a lack of tonality.

Now, it's important here to define what tonality means.
There're \href{https://en.wikipedia.org/wiki/Tonality}{a lot} of ways to define tonal, and therefore atonal.
Personally, I like the definition that has it synonymous with key, because it fits my calling both old and new music atonal.

Anyone even glancingly familiar with modern and post modern music will be familiar with the idea that it is atonal.
Composers tried to see what music might sound like without a tonal center.

From the other direction, however, music that's very old also lacks a key.
Early chant is almost entirely modal.
That is, they focus less on the exact notes in the octave,\footnote{because the only choice is whether to flat B} and instead focus on the arrangement of the notes.
I'm not going to explore that concept today,\footnote{if only because I don't know it that well} but just mention that as a reason for this essay existing at all.

When writing modern atonal music, one important piece in the way that many composers work is to divide the octave.
Since equal temperament\footnote{for better or for worse} says each semitone is identical, there's a lot of freedom in composition.
So, how do we, as the title says \say{Break the Octave}?

Well, an octave is made of 12 semi tones.
So, since I want my scales to all be one octave,\footnote{because the idea of octave equivalence is still important to me. I should really try a non-octaval scale} we can divide 12 by however many notes we want, and see how many scale options there are.

So, if we want only one note, there are 12/1 = 12 options.
That fits with what we know.
If the only note in your scale is the root, then there are 12 notes, so 12 roots.
This scale can be expressed as (o), signaling that the scale proceeds in octaves.
Since there are no 3rds or 5ths, this scale has no major or minor triads.

Since math is cool,\footnote{i.e. a/b = c iff a/c = b} we also see that there's only one option that contains all 12 tones.
Given that we talk about \say{the} chromatic scale, rather than \say{a} chromatic scale, this seems easy enough.
It can be expressed as (m), since the only interval in the scale is the minor second.
Since every note of the scale is represented, every major and minor triad is possible, for a total of 24.

Next we break the octave in two: 12/2 = 6.
There're six scales constructible with the tritone: c\&f\#, c\#\&g...
As there are no 3rds or\say{perfect} 5ths, this scale also has no major or minor triads.
This could be expressed as (T), since the only interval is the tritone

Going in the other direction, we see that there are two ways to break the scale into 6.
These two patterns are known as the \say{whole tone scale,} because they're made entirely of whole tones.\footnote{whole steps/major seconds}
The two patterns are: C,D,E,F\#,G\#,A\# and C\#,D\#,F,G,A,B.
You might notice that each of these scales has a diminished third in them.\footnote{A\#-C and D\# to F}
Since performers struggle with odd intervals, the enharmonic spellings are often used to make it easier for performers.\footnote{e.g. Bb-C or A\#-B\# for A\# and C}
There are only major thirds, and no perfect fifths, so these also lack triads.
It can be expressed as (w), since it's a series of whole steps.

At this point, you may wonder why I care about major and minor triads.
Personally, I feel like just because a piece may be atonal, that doesn't mean that it has to be dissonant.
While working with an early musician, he talked about the idea of horizontal dissonance and vertical dissonance.
Horizontal dissonance is where intervals and notes sound dissonant because of what preceded them, while vertical dissonance is what is typically thought of as dissonance.\footnote{large integers required to express the hypothetical relationship of notes}
Personally, I really like vertical consonance, and so I like writing with it.
Knowing what scale patterns I can use to make horizontally dissonant and vertically consonant pieces is fun to me.

Moving on, 12/3 = 4.
There are four ways to break the octave into major thirds.
They are: C-E-G\#, C\#-F-A, D-F\#-A\#, and D\#-G-B.
Again, there are only major thirds here, and so there are no triads available.
Interestingly, since a major third is the same distance as two whole steps, two tritonic scales are included in each hexatonic scale.\footnote{add in mono,bi... tonic for each}
This can be expressed as (M3), since that's the only interval.
This scale pattern is hard because it requires diminished fourths.

Next, there are three ways to break the octave into minor thirds: C-Eb-Gb-A, C\#-E-G-A\#, and D-F-G\#-B.
As there are only minor thirds, there are also no triads.
As in the hexatonic scale, the quatratonic scale contains augmented seconds.
This can be expressed (m3)

Now we get to the fun bit of math.
So far, 12 has been evenly divisible by each number we've tried.\footnote{1,2,3,4,6,12}
What happens when that isn't true?
When that's the case, there is no way of splitting the octave into just one interval.
The most often studied of these generated scales is\footnote{ignoring diatonic and pentatonic because they're not generated like these are} the octatonic scale, expressed (wh).
We see that 12/8 = 3/2, so that means every two movements, we need to move up three pitch classes.
A whole step is two, and a half step is one, so that constructs the octatonic scale.
There are three octatonic scales possible: C-C\#-D\#-E-F\#-G-A-A\#, C-D-D\#-F-F\#-G\#-A-B, and C\#-D-E-F-G-G\#-A\#-B.
You'll notice that since 12 doesn't break down into 8, there is overlap between each of these scales.
The fact that there are 3 ways to break the octave into 12 might also make you think that there are 8 ways to break the octave into three.
Let's try breaking it again, this time averaging.

One important note about the multiple interval construction is that unisons don't count.
That is, there's no way to move 1/2.
So, since a major third is 4 half steps, need an average of four movements per note.
If we do 5 and 3, we still satisfy this.
That is, if we move (4m3M3).
Unfortunately, there, like in the diatonic scale, we are forced to spell out each of the intervals, which means that there are 12 of this scale, and that it doesn't work well.

Moving on, we see that we can construct a monotonic, bitonic,\footnote{composed of tritones} tritonic,\footnote{which oddly enough doesn't contain a tritone} quatratonic,\footnote{again containing tritone relations}, hexatonic, octatonic, and dodecatonic.\footnote{which contains all of the intervals}

Another interesting thing to note about the octatonic scale is that any two octatonic scales contain the entire chromatic scale.
This appears to be true of the nonatonic scale (whh) as well.

The nonatonic scale is also interesting, and is the real reason I wrote this essay.
While making a hexatonic chorale, I realized that the voices were moving with some half-steps as well as whole steps,\footnote{if you want a copy, email me at \href{mailto:flyingrebelpipes@gmail.com}{flyingrebelpipes@gmail.com}} and assumed that it would be octatonic movement.
But, instead, it was nonatonic.
The cool bit of a nonatonic scale is that it contains a hexatonic scale\footnote{and therefore also a tritonic scale} within it, as well as half of the other hexatonic scale.
Since 12/9 = 4/3, there are four different nonatonic scales, each of which sharing 6 notes with each of the other three.\footnote{grouped as a set of 3}
The four scales are:\footnote{numbering wholly arbitrary and decided by me}\\
0: C-C\#-D-E-F-F\#-G\#-A-A\#\\
1: C\#-D-D\#-F-F\#-G-A-A\#-B\\
2: D-D\#-E-F\#-G-G\#-A\#-B-C\\
3: D\#-E-F-G-G\#-A-B-C-C\#\\

We see that 0,1,2 all share D,F\#, and A\#.
0,1,3 all share C\#,F, and A.
0,2,3 all share C,E,G\#.
And, 1,2,3 all share B,D\#,G.

Likewise, you can see which notes aren't in any of the given scales by seeing which ones the others share.
So, since 1,2 and 3 all share B,D\#, and G, we see that 0 contains none of those.
It's cool that the shared pitch classes between three of the four scales is a tritonic scale, and that the shared pitches between 0\&2 is one of the two hexatonic scales.
Each nonatonic scale contains 12 major and minor chords, 6 of each.

Next, if we divide the 12 into 4 a different way, we see that there are a few more options.
If we break the minor 3rd (m3) [3/1] into [6/2], then make it [M3M], we get:\\
C-E-F\#-A\#\\
C\#-F-G-B\\
D-F\#-G\#-C\\
D\#-G-A-C\#\\
E-G\#-A\#-D\\
F-A-B-D\#

Since two minor thirds [3] added equal one tritone [6], the six patterns each share some notes with each other.
We can do this a one other, breaking [3/1] into [6/2],\footnote{the fraction head must still be a divisor of 12 for the combinations to work, which is why we can't do [9/3]} and then [4m].
That is, movement of a perfect fourth and then a half step.
We get:\\
C-F-F\#-B\\
C\#-F\#-G-C\\
D-G-G\#-C\#\\
D\#-G\#-A-D\\
E-A-A\#-D\#\\
F-A\#-B-E

This brings us to a total of 16 possible quatratonic scales, which is pretty cool.

Next, we can try breaking the octave into 10.
[12/10] = [6/5], so there will be 6 scale patterns.
We also need to move up a tritone every five notes.
[2+1+1+1] works, so the pattern is [Mmmm].
The scales possible are:\\
C-D-D\#-E-F-G-G\#-A-A\#\\
C\#-D\#-E-F-F\#-G\#-A-A\#-B\\
D-E-F-F\#-G-A-A\#-B-C\\
D\#-F-F\#-G-G\#-A\#-B-C-C\#\\
E-F\#-G-G\#-A-B-C-D\\
F-G-G\#-A-A\#-C-D-D\#\\
F\#-G\#-A-A\#-B-C\#-D\#-E

This also suggests that there are more ways to break the tritonic scale.
But, let's look at the chordal options here.
Each of the decatonic scales contains (??) triads, (??) m and (??)M.
For the first scale, these are:\footnote{working in 12 tone notation}
\begin{enumerate}
\item 0-3-7 m
\item 0-4-7 M
\item 2-5-9 m
\item 3-7-10 M
\item 5-8-0 m
\item 5-9-0 M
\item 7-10-2 m
\item 8-0-3 M
\item 9-0-4 m
\end{enumerate}

Whoops, I did that wrong.
So, that's another set of nonatonic scales, that are bad because their pattern is  (WhhhWhhhW).
Weird.
Ok so to break the octave in 9 we did 3 movements every two notes.
Here, we need to do 6 movements every 5.
[2+1+1+1+1] should work.
That would do 0-2-3-4-5-6-8-9-A-B.
Whoops I'm dumb.

Ok anyways, so the scales for 10 notes are:
\begin{enumerate}
\item C-D-D\#-E-F-F\#-G\#-A-A\#-B
\item C\#-D\#-E-F-F\#-G-A-A\#-B-C
\item D-E-F-F\#-G-G\#-A\#-B-C-C\#
\item D\#-F-F\#-G-G\#-A-B-C-C\#-D
\item E-F\#-G-G\#-A-A\#-C-C\#-D-D\#
\item F-G-G\#-A-A\#-B-C\#-D-D\#-E
\end{enumerate}

1\&2 share C,D\#,E,F,F\#,A,A\#,B.
1\&3 share C,D,E,F,F\#,G\#,A\#,B.

It might be faster to look at what each scale misses.
1 doesn't have C\# or G.
2 doesn't have D or G\#.
3 doesn't have D\# or A.
4 doesn't have E or A\#.
5 doesn't have F or B.
6 doesn't have F\# or C.

So, since the notes missing are a tritone apart, we see that the common notes in 1,3 and 5 are one of the hexatonic scales, and the other one is in the other three.

Next, we see that since we can't break [12/5],[12/7], or [12/11] down, there's no way to represent the scale that doesn't require writing the entire scale out.
Ok this was a fun way to kill 90 minutes.

Oh shoot!
The decatonic scale contains a lot of triads.

In Scale 1:
\begin{enumerate}
\item D-F-A m 
\item D-F\#-A M 
\item D\#-F\#-A\# m
\item E-G\#-B M
\item F-Ab-C m
\item F-A-C M
\item G\#-B-D\# m
\item Ab-C-Eb M
\item A-C-E m
\item Bb-D-F M
\item B-D-F\# m
\item B-D\#-F\# M
\end{enumerate}

Cool, 6 major and 6 minor.
Wow this was almost 2000 words.
I need to spend less time doing this

The next thing I should work on is seeing how I can break the scales down into single movement triads.
That is, how one note changes FM to Am to CM.... 

\end{document}