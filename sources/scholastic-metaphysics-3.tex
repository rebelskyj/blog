\documentclass[12pt]{article}[titlepage]
\newcommand{\say}[1]{``#1''}
\newcommand{\nsay}[1]{`#1'}
\usepackage{endnotes}
\newcommand{\1}{\={a}}
\newcommand{\2}{\={e}}
\newcommand{\3}{\={\i}}
\newcommand{\4}{\=o}
\newcommand{\5}{\=u}
\newcommand{\6}{\={A}}
\newcommand{\B}{\backslash{}}
\renewcommand{\,}{\textsuperscript{,}}
\usepackage{setspace}
\usepackage{tipa}
\usepackage{hyperref}
\begin{document}
\doublespacing
\section{\href{scholastic-metaphysics-3.html}{Scholastic Metaphysics Chapter 2}}
First Published: 2022 March 3

\section{Draft 1}
This week I will be smart and read the chapter to try to respond to the questions at the back, rather than freeform note-taking while I read\footnote{pp 40 and 41}
\begin{enumerate}
\item What is the meaning of \say{a being}?
What are its two distinct but inseparable elements?
\item How can we reach explicit reflective awareness of the \say{is} in being?
Do all metaphysical systems agree on this focus on actual existence as the central core of all real beings?
\item What is meant by the \say{vocation of human beings} as endowed with intellect arising from the relation of intellect to being?
In what sense can being still remain a \say{mystery} for us?
\item Explain the difference between \say{real being} and \say{mental being}?
Examples of each?
What is the key criterion for our distinguishing between the two?
\item Explain the fundamental importance of action as the self-manifestation of being if we are to have a \say{universe}?
Could there be at least one completely inactive being?
\item Finite (all limited, created) real beings go out of themselves to relate themselves to others through action for two reasons: what are they?
Does it make sense to speak, as Maritain does, of \say{the intrinsic generosity of being}?
\item In the philosophical vision of St. Thomas, action is the key to a realist epistemology, or theory of knowledge.
Why?
Why can it then be called a \say{relational realism}?
Why does it also follow from this vantage point that all our human knowledge of real beings (at least in this life) must be incomplete, imperfect?
\item Why in this book do we take the person as the best model for what it means to be a real being?
Compare briefly the ancient, medieval, and modern approaches to the philosophical study of being.
\item What is the point of choosing interpersonal dialogue as the preferred starting point for a metaphysical study of being?
Why is it especially effective in refuting Kant's attempt to block access to any realist theory of knowledge or metaphysics?
\end{enumerate}
I think it might be helpful for me to try answering these questions both before and while/after reading the chapter, so that I can see how well my starting assumptions/things I have learned in the past line up with what the book claims.

So, my starting guesses as answers:
\begin{enumerate}
\item \begin{itemize}
\item What is the meaning of \say{a being}?

My guess from the little bits of Thomistic/Scholastic Metaphysics that I've done is that a being is made up of the potential and action, so what they can become versus what they are.
\item What are its two distinct but inseparable elements?

Whoops, I assume same answer here.
\end{itemize}
\item \begin{itemize}
\item How can we reach explicit reflective awareness of the \say{is} in being?

We can't. We can approximate it though.
\item Do all metaphysical systems agree on this focus on actual existence as the central core of all real beings?

No, because lots of metaphysical systems explicitly disbelieve in actual existence
\end{itemize}
\item \begin{itemize}
\item What  is  meant  by  the  \say{vocation  of  human  beings} as endowed with intellect arising from the relation of intellect to being? 

I have no clue what this is asking, I'm excited to find out.
\item In what sense can being still remain a \say{mystery} for us?

We cannot know the full reality of life.
\end{itemize}
\item \begin{itemize}
\item Explain the difference between \say{real being} and \say{mental being}?

I have literally no idea, once again.
\item Examples of each?
\end{itemize}
\item \begin{itemize}
\item Explain the fundamental importance of action as the self-manifestation of being if we are to have a \say{universe}?

Without action, there is no way to have a universe, which is inherently in action.
\item Could there be at least one completely inactive being?

Logically: yes, if there can be a being with only action, then there should be one with no action.
Practically: no because then it is not a thing.
\end{itemize}
\item \begin{itemize}
\item Finite (all limited, created) real beings go out of themselves to relate themselves to others through action for two reasons: what are they?

Desire to know themselves.
Desire to understand the Divine.
\item Does it make sense to speak, as Maritain does, of \say{the intrinsic generosity of being}?

Absolutely!\footnote{Complete wag}
\end{itemize}
\item \begin{itemize}
\item In the philosophical vision of St. Thomas, action is the key to a realist epistemology, or theory of knowledge.
Why?

Without action there is no way to interconvert.
\item Why can it then be called a \say{relational realism}?

Great question, thank you for asking.\footnote{practicing for my thesis exam}
\item Why does it also follow from this vantage point that all our human knowledge of real beings (at least in this life) must be incomplete, imperfect?

Without being able to fully understand the relational aspects of something, there is no way to get perfect or complete knowledge.
\end{itemize}
\item \begin{itemize}
\item Why in this book do we take the person as the best model for what it means to be a real being?

Thomas does.
\item Compare briefly the ancient, medieval, and modern approaches to the philosophical study of being.

Nope.
\end{itemize}
\item \begin{itemize}
\item What is the point of choosing interpersonal dialogue as the preferred starting point for a metaphysical study of being?

Because Kant does not.
\item Why is it especially effective in refuting Kant's attempt to block access to any realist theory of knowledge or metaphysics?

Kant assumes that we cannot know anything but ourselves, so starting with the knowledge that other people's existence has meaning defeats it.
\end{itemize}
\end{enumerate}

Time to read!
\begin{enumerate}
\item \begin{itemize}
\item What is the meaning of \say{a being}?

Something which is.
More precisely, that which is.
\item What are its two distinct but inseparable elements?

is, which is the fact of existence and essence.
Or, the fact and the substance?
Is meaning that it is, and that which meaning what it is.
\end{itemize}
\item \begin{itemize}
\item How can we reach explicit reflective awareness of the \say{is} in being?

Explicit reflective awareness refers to the fact that it is a wonderful mystery that things are at all, let alone what they are.
There are two approaches that the author suggests.
\begin{itemize}
\item Downward exploration, where we seek deeper and more fundamental facts about a thing.
\item Outward exploration, where we seek to connect every thing to every other thing until we have a universe within our thoughts.
I think at least, I am not totally sure what the goal is in outward exploration.
\end{itemize}
\item Do all metaphysical systems agree on this focus on actual existence as the central core of all real beings?

Nope.
In many systems, the term being means limited essence or form, which means that the Divine, which exists outside of a bound form, must not be in existence.
Since St. Thomas posits that existence in itself is the basis of being, the Divine fits within the meaning, and therefore actual existence treats all real beings.
\end{itemize}
\item \begin{itemize}
\item What  is  meant  by  the  \say{vocation  of  human  beings}  as  endowed  with intellect arising from the relation of intellect to being? 

Our fundamental calling is to illuminate the universe and bring the unknown into the known.
\item In what sense can being still remain a \say{mystery} for us?

It kind of just felt like he said that was true without an argument, but it isn't a particularly difficult concept for me.
Human knowledge is inherently incomplete.
\end{itemize}
\item \begin{itemize}
\item Explain the difference between \say{real being} and \say{mental being}?

Real beings exist.
That is, they do not only exist in thought.
E.g. being kind, being a man.
Real beings can generate mental beings, but the reverse is not true.

Mental beings, by contrast, are only real as ideas, such as the past, future, dreams, abstractions, math, and conceptions of absence.
\item Examples of each?

See above.
\end{itemize}
\item \begin{itemize}
\item Explain the fundamental importance of action as the self-manifestation of being if we are to have a \say{universe}?

St. Thomas believes that action is the primary distinguisher between the real and mental being.
Additionally, it is only through action that real beings interact with each other, allowing the vocation of knowing to take place.

\item Could there be at least one completely inactive being?

Well, a mental being almost certainly.
As a real being, there is no \say{practical} difference between that and no being.
Therefore, it is equivalent to no being.
\end{itemize}
\item \begin{itemize}
\item Finite (all limited, created) real beings go out of themselves to relate themselves to others through action for two reasons: what are they?

\begin{itemize}
\item We are poor in that we are limited and imperfect
\item We are rich in that we exist and can share this imperfect knowledge with others
\end{itemize}
\item Does it make sense to speak, as Maritain does, of \say{the intrinsic generosity of being}?

Yeah, I think so.
Especially if we agree with the concept of our vocation, there is an inherent need to connect.
\end{itemize}
\item \begin{itemize}
\item In the philosophical vision of St. Thomas, action is the key to a realist epistemology, or theory of knowledge.
Why?

We can only see things in the way that we interact as actors.
As a result, there is no way to know without interaction.
Additionally, action inherently reveals essence, and so provides knowledge.
\item Why can it then be called a \say{relational realism}?

Since we are seeing action as springing from essence, rather than essence itself, we can only learn through relation.
\item Why does it also follow from this vantage point that all our human knowledge of real beings (at least in this life) must be incomplete, imperfect?

Since we cannot see essence itself, we can only learn as through a veil.
Additionally, no single act can fully express the nature of a finite being, and we cannot observe all things, both because of physical and mental blocks.
\end{itemize}
\item \begin{itemize}
\item Why in this book do we take the person as the best model for what it means to be a real being?

I mean, in part because Aristotle did and our metaphysics grows from his.
However, the argument is that, while our modern collective metaphysics may overprioritize the subjective experience, there is something integral to individual experience.
That being said, there is still a concept that we as people share certain traits, and so the person balances these two, the subjective and objective.
\item Compare briefly the ancient, medieval, and modern approaches to the philosophical study of being.

The ancient focuses on the objective, ignoring the person.
The medieval focuses on the centrality of shared human experience, the We.
The modern focuses on the centrality of individual human experience, the I.
\end{itemize}
\item \begin{itemize}
\item What is the point of choosing interpersonal dialogue as the preferred starting point for a metaphysical study of being?

Sharing the agreement of a faith-friendly view like Thomistic metaphysics is a harder hurdle than the acceptance of knowledge through relation with other humans.
\item Why is it especially effective in refuting Kant's attempt to block access to any realist theory of knowledge or metaphysics?

Since we can communicate, there is an inherent contradiction to Kantian philosophy?
I'm a little lost about what Kant argues, so a rebuttal of his argument doesn't really help me that much.

\end{itemize}
\end{enumerate}
\end{document}