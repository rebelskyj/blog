\documentclass[12pt]{article}[titlepage]
\newcommand{\say}[1]{``#1''}
\newcommand{\nsay}[1]{`#1'}
\usepackage{endnotes}
\newcommand{\1}{\={a}}
\newcommand{\2}{\={e}}
\newcommand{\3}{\={\i}}
\newcommand{\4}{\=o}
\newcommand{\5}{\=u}
\newcommand{\6}{\={A}}
\newcommand{\B}{\backslash{}}
\renewcommand{\,}{\textsuperscript{,}}
\usepackage{setspace}
\usepackage{tipa}
\usepackage{hyperref}
\begin{document}
\doublespacing
\section{\href{book-review-after-dragons.html}{Book Review of After the Dragons}}
First Published: 2023 June 8
\section{Draft 1}
For as much as the books I read tend to fall into the categories of: recommended to me, read already, closely related to a book recommended to me,\footnote{so by the same author, stocked next to each other, someone says \say{oh that's like such and such}, etc.} I do also read other books.
It's sometimes hard to find books that don't fit into one of those categories, but that's just because I am surrounded by a number of book-lovers with distinct tastes and have a very low bar for what counts as closely related.
That being said, I'm usually glad when I do.

A few weeks ago, I went to the library to check out a copy of Art Spiegelman's \textit{Maus.}\footnote{Review should come at some point but I think I need to digest my feelings around it more because I don't want that review to be rambled.}
While I was there, I walked by the table of new arrivals.\footnote{or something. I'm not totally sure what the label was for the book display. What's important is that there was a display.}
I saw a book called \say{After the Dragons,} by Cynthia Zhang, and decided to give it a read.

It's squarely in the realm of Magic Realism\footnote{which I had remembered as fantastic realism for some reason}, which is a genre that I think more people need to be exposed to.
Magic Realism is subtly different than Urban Fantasy, though the distinction is important.
In Urban Fantasy, Magic is at least on some level hidden from the average person, or was.\footnote{in a number of books, the idea that Magic stops being hidden is a plot point.}
Some popular examples of Urban Fantasy are the Harry Potter series, Lilo and Stitch,\footnote{which I found on TVTropes, and am incredibly proud that I managed to avoid getting sucked into} Buffy the Vampire Slayer, Twilight, and Dracula.

Magic Realism, by contrast, is where we have our existing society, but Magic is just there.
I keep seeing claims that it originated in South America, but I've seen plenty of examples of it in Russian literature.
As an example, I would argue that Kafka's Metamorphisis is one, though there are newer ones for sure.
Disney's Encanto or Turning Red would also be clear examples of the genre.
I would also argue that Frankenstein fits better into this genre.
A lot of Stephen King's works fit here as well.

Anyways, four hundred words of exposition later, After the Dragons takes place in China, where dragons are slowly going extinct due to man-made pollution.\footnote{Magic Realism often uses the fantastic as a metaphor. As far as metaphors go, this one was not hard to understand.}
We follow a Chinese-American here on a study-abroad equivalent\footnote{he's a visiting researcher for the summer, but that's close enough in my mind} who slowly falls in love with a Chinese man who has a degenerative illness.
Dragons are how the two meet and bond, as the student wants to study them, and the Chinese man takes care of feral dragons that he finds on the streets.

It was a quick read, but an enjoyable one.
There were some lines in the book that screamed propaganda to me, but I've been told I'm overly sensitive to that.
Because the book was so short, though, I felt like there were a lot of portions that I would have preferred be fleshed out more.
I would still recommend someone read it, though I wouldn't read it again without a real\footnote{read: someone asks to read it with me or makes it clear that they would consider their life better for me rereading it} reason.

443/169
\end{document}