\documentclass[12pt]{article}[titlepage]
\newcommand{\say}[1]{``#1''}
\newcommand{\nsay}[1]{`#1'}
\usepackage{endnotes}
\newcommand{\1}{\={a}}
\newcommand{\2}{\={e}}
\newcommand{\3}{\={\i}}
\newcommand{\4}{\=o}
\newcommand{\5}{\=u}
\newcommand{\6}{\={A}}
\newcommand{\B}{\backslash{}}
\renewcommand{\,}{\textsuperscript{,}}
\usepackage{setspace}
\usepackage{tipa}
\usepackage{hyperref}
\begin{document}
\doublespacing
\section{\href{nanowrimo-5.html}{NaNoWriMo 2023!}}
First Published: 2023 November 1
\section{Draft 1}
Well, it's another November, which means it's time for another installment in \say{I\footnote{Initial draft said my first name, but I am unsure if I've ever posted it here, and may as well do a little bit of not self doxxing if I haven't} try my hardest to drive myself insane by doing something related to NaNoWriMo again.}
In 2018, when I began this blog\footnote{I keep wanting to say 'blog, and I think that I might just start doing that from now on}, I did not have the drive to actually write a novel.
Instead, I \href{nanowrimo.html}{attempted to add fifty thousand words to my 'blog postings over the course of the month.}
As I am sure you\footnote{the hypothetical reader} can guess, I was unsuccessful at that mission.

By 2019, I had forgotten about my musings well before November rolled around.
2020 and 2021 likewise passed by me without anything to point at.\footnote{on this site, at least. I am almost certain that there are any number of artifacts that I can look at.}
I suppose that there may not actually be all that much from 2021, given that I worked on the same journal from November of 2019\footnote{interesting, I had not connected that I finished that journal midway through what could have been another NaNoWriMo} through August of last year.

Last year, as I certainly remember, I \href{nanowrimo-2.html}{attempted to do NaNoWriMo} for what I had a vague idea was my first time ever.
As it turns out, I had made an account back in 2011, and had attempted to do NaNoWriMo that year.
I, unsurprisingly, failed at that goal.

Last year, however, I was successful, at least in terms of writing 50000 words of fiction and finishing one narrative.
The two did not align with each other, as I needed to start a sequel in order to get to the needed word count.
Of course, I gave up on the story almost as soon as I could.
One year later, I have no interest in revisiting that story.

I also wrote a story this past April, as part of a writing competition for the website where I currently post my ongoing novel.\footnote{the more I write, the more uncomfortable I get with ending sentences in prepositions.
I know that there's nothing inherently wrong with the prior sentence reading \say{for the website I currently post my ongoing novel on}, but the construction reads weirdly to me now.
I wonder if that's a sign that I'm getting better at writing.}
I made it to the word count for that book\footnote{fifty five thousand, five hundred and fifty five, to be exact (on the word count, not on what I wrote. I think I got somewhere closer to 57,000)}, but was tired of writing it by the end of the word count and rushed the ending.

Well, rushed may be a bit of an understatement.
The book was set over the course of twelve years.
I spent all but fifteen hundred of the words on the first three weeks, then rushed through the next twelve years in the remaining word count.
One commenter's reply was that they were grateful that I did not leave the work unfinished, for all that the ending was incredibly unsatisfactory.

So, that brings us to November.
As I mentioned \href{reflection-october-23.html}{yesterday}, I am also trying to revive this 'blog and get further ahead in my other book at the same time.
Will it work?

Who knows.

Will I burn out before the month is done?

Almost certainly.

Am I still going to ride this train for as long as I can, in hopes that I might be able to salvage a number of words from the situation?

Absolutely.

Anyways, the book this year is something I've never tried before.
I realize I've only really written third person limited in the past, so I'm writing this book in first person.
Now, whether that's going to be a good idea is yet to be determined.

The general plot of the book is that there's an order of vampire fighting werewolf priests in the modern day United States.\footnote{well, in the modern day.
I think that the plan is for the order to exist across the whole world, but the main character, and therefore, story, all will take place in the US.
Right now I have it set near Ann Arbor, but that is entirely just because I wanted a single location cue and it was the first vaguely metropolitan location that popped into my mind}
I'm not sure how much of the balance I'll have between conversation and action, but given that I've spent all of today's content\footnote{two words less than eighteen hundred} on the main character getting ready to meet with the monastery, odds are good that this book will also end up somewhat rushed.

In any case, it'll be the book I try to write.s
I'm hopeful that I can tell a story, and a story with faith, since I haven't really touched my faith in any of the writing I've done.\footnote{well, at least I haven't touched on my faith intentionally in any of the fiction writing.
The many reflections on the readings I've done here aren't counted, for obvious reasons.
It's also probably obvious that I have a Catholic world view given the other writing I've done, but I am not the person to analyze that.}
I don't know if I'll be able to do it to anyone's satisfaction, let alone my own, but I have been thinking a fair amount about how to interface my faith and writing.

I feel like that topic deserves far more consideration than I can give it at this point in the night.\footnote{I wrote this post in three sittings, the third of which begins with the final sentence of the prior paragraph.}
It's currently past my bedtime, and I'm well above my word goal for the day.\footnote{which is approximately 5000}

Daily Reflection:
\begin{itemize}
\item Did I write 1700 words for NaNoWriMo? Yes! I did two fifteen minute sprints\footnote{with pauses as needed to complete quests}, 1798/1700. Nice I am 100 words ahead of schedule now.
\item Did I write a chapter of Jeb? I did! I just finished it, and, as seems always true, it felt much harder to write the chapter at the beginning\footnote{and middle and end} than it ended up being.
\item Did I blog? I am blogging now! Currently working on a streak, which is nice.
\item Did I stretch? Shoot! I will do that as soon as I finish here
\item Am I doing better at prayer than a rushed and thoughtless rosary? I remembered to do an\footnote{admittedly rushed} Angelus just before noon, when my alarm went off. 
Otherwise, I tried to be more mindful about my prayer last night, for all that I don't know if it really was.
\item Am I doing a good job writing letters to friends?
I found my stamps, envelopes, and cards.
I also addressed an envelope.
That feels like enough for the day, especially since I want to make the letters feel as easy as possible.
Tomorrow I'll try to address another letter, but we'll see how I'm doing on time.
\end{itemize}

I\end{document}