\documentclass[12pt]{article}[titlepage]
\newcommand{\say}[1]{``#1''}
\newcommand{\nsay}[1]{`#1'}
\usepackage{endnotes}
\newcommand{\1}{\={a}}
\newcommand{\2}{\={e}}
\newcommand{\3}{\={\i}}
\newcommand{\4}{\=o}
\newcommand{\5}{\=u}
\newcommand{\6}{\={A}}
\newcommand{\B}{\backslash{}}
\renewcommand{\,}{\textsuperscript{,}}
\usepackage{setspace}
\usepackage{tipa}
\usepackage{hyperref}
\begin{document}
\doublespacing
\section{\href{time-discounts.html}{Time Discounts and a Better Me}}
First Published: 2025 April 3
\section{Draft 3: 3 April 2025}  
Anything in the future is worth less than having it now.  
For all that the previous sentence feels like the rantings of a child, they really underpin a lot of modern economic thought.\footnote{though, given the state of the economic world, maybe it's entirely appropriate that it sounds like a child having a tantrum}  
In economic terms, most everything either appreciates or depreciates, and most do so in some sort of exponential fashion.

In retirement advice, this is usually used as a way to encourage 20 and 30 somethings to enjoy their lives a little less.\footnote{wow I've gotten steadily more cynical over the drafts}  
In this post, however, I'm going to use it as the way of encouraging me to become slightly better.

I think that I can improve as a person.  
Even though I do agree that the exponential growth implied by most basic theories of compounding are unrealistic or unsustainable\footnote{see: planet}, there is still something to be said for compounding generally.  
Every day that I go forth with a skill is a day that I can improve the lives of those around me through that skill.  
Since I live in a society with capital requirements, I also have a chance to use any skill for a profit.

It's far more than that, though.

There's an old interview my grandmother once had, where she debated someone about whether or not everyone should get a liberal arts degree.  
As you might expect from everything about me, she argued very much in favor of universal liberal arts education.  
In her mind, and as a partial consequence, my mind, education is an inherently positive thing.\footnote{ugh having all the posts I want to write keep coming up in the posts I do write is getting frustrated. Curse of knowledge rises a few more slots}  
We are changed by everything we experience, and education helps any such change to be positive.  
When we learn facts, we are able to connect events and realities.  
When we learn theories and heuristics, we are better able to make sense of the world.  
And, more than anything, I think that learning is about finding joy in the world.

The Almighty created the heavens and the earth for humanity.\footnote{oof this may be a heresy}  
We are called to seek the Divine, and are given the wonder of the universe to better understand Divinity.  
We are given talents and seeds for growth and asked to plant and nurture them in ourselves and others.

Through all of this, there's something to be said for laying groundwork today for a better tomorrow.

The nearer term the future and the greater the expectation of gains, the more I should be willing to give now.  
That feels reasonable.  
If spending two hours on a project makes me feel marginally better for the following hour, I'm not sure if it's worth it.  
If spending two hours is supposed to me feel marginally better for an hour in three weeks, I know that it's not.

There's also the cost to be considered.  
Something that takes minimal effort is worth doing, even if the expected gain is low.  
I know that after stretching I feel better in the moments to follow, the rest of the day, and the following days.  
I even tend to feel better during it.

The initial premise of this post was self improvement.  
What can I do to make myself better, and how much is bettering myself worth?

I'm reading a book right now called \textit{Radical Acceptance}, and one of its key points is that only after accepting the realities of ourselves can we change any of them.  
The author is pretty clear that the goal is not attempting to change things, but I can take from books what I want.  
Still, there's a large difference between accepting reality and settling for reality.

How do I want to improve?  
Let's spend a minute and think.

I want to be more productive.  
I want to be more in shape.  
I want to be more kind to myself and others.  
I want to learn how to be vulnerable.  
I want to write, just so very very many words for my thesis.  
I want to experience the world around me, highs and lows.

Wow that went all over the place.

Ok so, writing and being productive obviously go hand in hand, since the productivity I'm searching for is generally writing right now.\footnote{gotta love homonyms}  
Being in shape is probably not a great goal as an end in itself, and wanting to be in shape for how I look is also not a good goal, since bodies change and I should never strive for a bodily ideal that absolutely is not sustainable for the rest of my life.  
I do, however, want to be able to experience the world around me.

When a friend asks if I want to go for a run, I'd like to be able to keep up.  
I don't want to be winded when I'm walking somewhere or going up stairs.  
I want to be able to move my body into different poses.  
I want to be able to pick stuff up.

When I am more centered and grounded, I am more able to experience the world.  
When I am more centered and grounded, I am more able to be kind.  
When I am more centered and grounded, I am more able to accept weakness, and therefore show vulnerability.

So, what can I change to be more productive, in shape, centered, and grounded?  
Luckily, the answers are all the same: be more mindful and work out more.

\section{Draft 2: 3 April 2025}  
A common idea in economics, especially modern economics, is that a dollar tomorrow is worth less than one today.  
There are plenty of arguments for this\footnote{inflation, uhhhh not being able to do stuff with it?}, but most boil down to the obvious that it's always better to have than to not have.  
When you combine this with general ideas of compounding interest\footnote{generally assume that the stock market returns 8 percent per year on average}, you get some ideas about saving like \say{any dollar you spend today is the same as spending two dollars in ten years or four dollars in twenty or etc etc.}\footnote{there's the fun \say{if you divide 72 by whatever percentage per time you have compounding interest, you get about the doubling time} fact that I always enjoy}  
I don't really love that argument, because like a guitar that costs five hundred dollars today, even if it wouldn't be worth 2000 to me in twenty years if I was to buy it then, could very well emotionally be worth that amount then.  
Also, as I talked about in the previous draft, I can only ever experience the present.

I'm now realizing that previous sentence is really the issue underlying a lot of my failure to self care.  
Any time that I spend doing something unpleasurable now with hopes of it paying dividends in the future is fundamentally happening to an alien me.  
I have no idea what I will be like even tomorrow, at least on an emotional level.  
However, much as I can think through choices in ways like \say{doing this will likely harm someone in another country}, or \say{this will have bad effects in fifty years}, I think that I can trust that I will always want to be supporting the future me.  
A future me who turns to a life of crime is likely having a harder time than I am now, and deserves all the help I can give.

So, then, what is the right way to do self care knowing that money spent now is money gone forever, anything gained now will either depreciate\footnote{e.g. you lose some huge percentage of the worth of a new car just by driving it off the lot} or compound\footnote{my degree, for instance, will do many things for me, not least is potentially shield me if I need to flee a country}, and any action I do is ultimately laying another brick in the road of my life.\footnote{see the post overmorrow}

I'm just now rereading the initial prompting, and it was not about self care at all!.

\section{Draft 1: 3 April 2025}  
In economics, there's an idea of time discount.  
In short, it's the idea that money today is worth more than money tomorrow.  
For whatever reason, I've taken at least a few surveys that ask questions on the nature of \say{would you rather one hundred dollars at X point in the future or fifty dollars today?}  
And, in general, I think that I generally have a fairly low time discount, if only because I have a decent understanding of general inflation rates and a pretty low acute need for 50 dollars.

Like most economic theories, this can be immediately applied to absolutely everything, and I'm sure I'm not the first person to treat taking care of myself as a time discount problem.  
The me of the future will reap a large number of gains for any number of small inconveniences now.  
Getting a degree, especially an advanced degree, is very often posited in the language of a time discount.  
\say{It's worthwhile to get paid peanuts now,} they say, \say{because in the future you will make far more!}

Still, my goal here is not to think about how things are worth less to me in the future generally, but to specifically try to reframe the things that I do to take care of myself in this way.  
At the easiest, let's take eating.

I know, beyond a shadow of a doubt, that I feel better on the day after I ate well than the day after I ate minimally or generally poorly.\footnote{of course, what poorly means is entirely context dependent, but really this is just saying present me affects future me}  
I have faith in the medical science that tells me that setting up habits now will make the me of the future healthier.  
I have no difficulties in putting into practice ideas of saving for the future, and yet I do not put the effort to make sure that the future I fund will be enjoyable by the me of then.

I do know that I have more faith in money, being external to me, than my own health, which is dependent on so much.  
Still, even in the short term, I know that I am happier, better able to handle life's struggles, look better, and feel more energized for less sleep when I spend more time working out.  
I even find that I am more productive, I think even when taking into account the hours spent not working or doing other hobby.  
However, on any given day, I am almost always more productive when I do not work out.

I don't really know what else I'm actually trying to say here, other that like when I think about actions that I want to do or want to want to do\footnote{I think that I have mused about this in the past. \href{do-versus-done}{seems like the closest}, though even this is less of that.}, I should really stop framing the present me as much.

There is, of course, the counter argument to a lot of saving.  
After all, the age of your life when you are most able to take life by the horns\footnote{I think that life is a bull right?} is the age when you're most encouraged to be planning for the future.  
In some regards, this is obviously a holdover from when society was more Christian, though I'm not sure if it's initially a Catholic or Calvinist way of doing the world.  
However, there is also research saying bad events are more meaningful than good.  
All that this paragraph is to say: I don't want to let my aspirations for the future prevent me from experiencing the present.

That's applicable in finance and self care, even if I think that they take up different spheres.

\section{Draft 0: 2 April 2025}  
Wow, look at me, writing parts of drafts before the day I post.  
Anyways

There's a concept in economics called time discount.  
I know that on at least a few occasions I've been asked a series of questions that starts with something like \say{would you rather have 100 dollars today or 50 dollars today?}.  
This has an obvious and correct answer.  
After choosing 100, I am then prompted between 50 today or 100 tomorrow.

\section{Daily Reflection: 3 April 2025)}

\begin{itemize}   
\item Intentionality:  
\begin{itemize}  
\item At least hourly, stand up, drink water, take two deep breaths, and do a stretch of some sort

I think that I did actually do a pretty good job of this yesterday. I at least paid attention to my body and tried to be standing daily.  
\item Be proactive about avoiding overwhelm and when feeling overwhelmed, stop and figure out why.

It's wild how quickly asking myself what, exactly I'm feeling and why stops me from feeling bad.  
\item Light a candle and read by candlelight each night. Along with this, leave all electronics outside of the bedroom and/or move them away at least an hour before bed time.

Didn't do great on this one, found a video I really wanted to watch right before bed.  
\item Candle time in the morning before electronics. Use the time for prayer

Didn't do this. Found a video I wanted to watch this morning.  
\item Focus on good posture, especially straight back and making sure that neck isn't awkwardly positioned.

Did shockingly well at this one, for all that wow it really feels like my upper back is seizing up

\item Don't waste time, and in particular, be mindful about making sure to take breaks and rest. Especially make sure to do rest which revitalizes the me of tomorrow, rather than rest which simply keeps me in stasis.

Did generally ok with this, though while teaching\footnote{which, I know, does not need to be a multitasked event} I did definitely find myself twiddling my thumbs, and as mentioned, watched videos this morning instead of the mindfulness. I did, however, stretch for a full half hour last night. It's wild how much more fun stretching is when you feel flexible.

\item Interpersonal Relationships:  
\begin{itemize}   
\item Figure out what belongs in a normal letter to a friend.

Read a book from 1860 yesterday about gentlemanly etiquette.  
It had a section on letters, which did give the advice of only writing letters when I have a reason to.  
Outside of that, it was very clear that one needs to write in a clean hand and with good spelling and grammar and etc.  
The clean hand is probably something I can work on. I have a hand which is beautiful and legible and should work on using it more.

In general, the advice was to simply write in a letter what you would say to a person in person, which was pretty helpful.  
\item Get back into writing letters.   
\item Work to message friends at desired intervals.  
I know that I said today was going to be the day that I made the list, but I don't know if it is.  
\end{itemize}

\end{itemize}  
\item Professional:   
\begin{itemize}   
\item Do the Thesis and other research requirements. Upcoming deadlines:  
\begin{itemize}  
\item Brain dump about science communication (Overdue)  
\item Brain dump a publicly accessible chapter (Overdue)  
\item Have final convergences for the results I'm trying to reproduce (due 4/4)  
\item Draft of the first paper (due end of month, but I want to make sure that I've reupdated it sooner than later)  
\item Finish revising and editing the overview of a program chapter (due 4/7) and send it to the boss  
\item Revise the Science communication and publicly readable chapters (due 4/7)  
\item Send the science communication chapter to the boss (4/14)  
\item Brain Dump the background to the program (4/14)  
\end{itemize}  
\item Only do the work I feel called to when I've finished the tasks set to me for the day or outside of normal working hours (post 1725)

Turns out that reminding myself that falling into obsession is bad means that I don't!  
\item Start making the giant citation document so that I don't have to search for citations later.

The old\footnote{read: mid 1900s} articles are in Zotero now.  
\item Work towards future career:   
\begin{itemize}   
\item Figure out the difference between my public-facing and field-facing presentation affects. As I focus on becoming a better presenter, I need to become aware of the difference and how to switch them.  
\item Need to look for jobs  
\end{itemize}   
\end{itemize}   
\item Health:  
\begin{itemize}   
\item Spiritual:   
\begin{itemize}   
\item Get back into the Lenten goals (pray chaplet of St. Michael, give money equal to amount I'm spending on myself, stop scrolling social media, stop playing video games)

I printed the chaplet out and brought it to my bedside, did not read it though.  
\item Be intentional about prayer. That means both making time for prayer and actually doing it.  
  
While teaching the children's Sunday school analogue\footnote{it's on a Wednesday, so not totally sure what to call it}, I made them do some silent prayer, and prayed during that.  
\end{itemize}   
\item Physical:   
\begin{itemize}   
\item Start focusing on posture again, especially while sitting.

Really did that yesterday, and I'm beginning to wonder if I might end up pushing my shoulders too far down and back. I have to imagine I'll be able to figure out what that's like, but given how tight my chest is when I try to sit like that, it's probably a ways off.  
\item Go to group fitness classes more regularly and more often. If not, do workout at home

Since the First Wednesday of the Month is always a long day for me, only ended up with a half hour\footnote{or was it twenty minutes?} stretch/yoga time  
\item Feed myself simply and healthily. Healthy here means trying to generally avoid processing.

On Wednesdays I get suspiciously cheap burgers, and I accept that those are almost certainly hyperprocessed.  
Otherwise did ok! Forgot berries at home which meant I had oatmeal with jam yesterday, which felt wrong somehow.  
\end{itemize}

\item Mental:   
\begin{itemize}  
\item Clean Life:   
\item Remove dirt and clutter from physical spaces (standard definition of clean):   
I am still actively behind on this, but I have hope that I will claw my way back to a comfortable place.  
\begin{itemize}  
\item At least once a week, each room has nothing on the floor  
\item At least once a week, all surfaces which are not inherently storage are cleared off  
\item At least once every two weeks, each room is vacuumed  
\item At least once every month, all non-storage surfaces are explicitly washed/cleaned  
\item At least once a week, I get rid of at least one item that I notice (meaning throw away or in rare circumstances gift or donate)  
\item Clean sight lines. Is my space set up in a way that orients me towards my goals for the space? If not, how can I make it so?  
\end{itemize}  
\item Spend time each day thinking about the goals for the day, and getting them out of my head and onto the page.

This has been really helpful for that, as has making sure that, while my tea brews in the morning, I start by thinking about the goals for the day.  
\item Continue to explicitly confront the voice in my head that says that people hate me.  
  
Haven't really needed to do!  
\end{itemize}  
\end{itemize}   
\item Hobbies:   
\begin{itemize}   
\item Reading  
\begin{itemize}  
\item Start reading and returning the library books I have.  
\item Finish the book on mindfulness I started. (also make a list of the exercises in the book and try them out)  
\item Read more poetry  
\end{itemize}

\item Music:   
\begin{itemize}   
\item Work on guitar  
\item Learn the songs that jam partner suggested and/or requested I learn  
\item Get back into the album.  
\end{itemize}   
\item Writing:  
\begin{itemize}   
\item Write poetry more often, ideally nightly.  
  
Shoot! Forgot this one  
\item Find a way to add meta data to my blog posts and then add the meta data  
\item Not only write blogs, but also post them.

Yesterday I even managed to do this without my writing buddy!!\footnote{miss you}  
\item Get back into writing the web novel  
\end{itemize}   
\item Other hobbies, do them.  
\end{itemize}
\end{itemize}
\end{document}