\documentclass[12pt]{article}[titlepage]

\newcommand{\say}[1]{``#1''}

\newcommand{\nsay}[1]{`#1'}

\usepackage{endnotes}
\newcommand{\1}{\={a}}

\newcommand{\2}{\={e}}
\newcommand{\3}{\={\i}}

\newcommand{\4}{\=o}

\newcommand{\5}{\=u}
\newcommand{\6}{\={A}}

\newcommand{\B}{\backslash{}}

\renewcommand{\,}{\textsuperscript{,}}

\usepackage{setspace}

\usepackage{tipa}

\usepackage{hyperref}

\begin{document}

\doublespacing

\section{\href{arranging-for-bagpipes-ii.html}{Arranging for Bagpipes Part 2}
}

As you may remember from my last mention of this topic, some songs are very difficult to arrange for bagpipe. Today, I was lucky enough to find a song that was easy.

I found myself today with the best problem an artist can have: a hyper focused muse. I could not focus on anything else until I had set the hymn /say{How Can I Keep From Singing} to bagpipes. Unlike the anthem that I previously discussed, this son was far easier. It’s pentatonic,\footnote{5 notes to an octave} and only occupies an octave, from the dominant to the dominant. As you may remember, the bagpipe has a range of an octave and a second starting from the subdominant. So, I put the song in the key of d, added a few grace notes, and was done. It’s nice when things go well.

\end{document}
