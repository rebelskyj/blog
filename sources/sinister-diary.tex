\documentclass[12pt]{article}[titlepage]
\newcommand{\say}[1]{``#1''}
\newcommand{\nsay}[1]{`#1'}
\usepackage{endnotes}
\newcommand{\1}{\={a}}
\newcommand{\2}{\={e}}
\newcommand{\3}{\={\i}}
\newcommand{\4}{\=o}
\newcommand{\5}{\=u}
\newcommand{\6}{\={A}}
\newcommand{\B}{\backslash{}}
\renewcommand{\,}{\textsuperscript{,}}
\usepackage{setspace}
\usepackage{tipa}
\usepackage{hyperref}
\begin{document}
\doublespacing
\section{\href{sinister-diary.html}{Sinister Diary}}
First Posted: 2018 October 17
\section{Draft 2}
Today I began another form of record keeping: a \say{Sinister Diary.}
Now, for those of you who don't know,\footnote{most likely, if you're right handed, haven't taken Latin, and lack pedantic friends} sinister is the nominative masculine singluar Latin word for left.\footnote{sinister,sinistra,sinistrum}
So, what is a \say{left diary}?
It's just a diary that I'm writing with my left hand.

In my diary making class, we talked about different ways of sparking creative journal keeping.
One of these is to make a diary entry with your non-dominant hand.
Now, I have always wished I could write ambidextrously, so this was a good spark to me.

Of course, calling anything written in my left hand's\footnote{lack of} penmanship implies that that written in my right hand is \say{dextrous.}
Dextrous is the British form of the American \say{dexterous,}\footnote{apparently I'm becoming British more than I thought. Next thing you know I'll be spelling it oxydized} coming from the Latin \say{dexter,} meaning right\footnote{like the hand} or skillful or proper.\footnote{I shamelessly use wiktionary}

Now, far be it from me to describe anything written I do as skillful, but it's certainly easier to read than my left-handed writing.
Also, I ran into another problem while writing sinisterly.

When I was a young, impressionable freshman,\footnote{as opposed to the young, impressionable junior that I am now} I took Latin.\footnote{no this story doesn't end poorly}
I had the brilliant idea to write all of my Latin with my left hand, so as to embed it more deeply in my mind.\footnote{the inner workings of my mind are a mystery even to me}
Of course, the professor shut that idea down on the grounds of legibility.\footnote{or, more precisely, the lack thereof}
But, today, when I tried writing with my left hand for the first time in a while, I noticed that I was thinking of the Latin\footnote{and also Spanish because my mind groups things oddly} translations for a lot of what I was writing.
This continued to the point where I began thinking, and even writing in in\footnote{broken} Latin.
As bad as it was, it was certainly fun.

So, I think I'll continue my sinister diary, if only to have that as a conversation starter.
\section{Draft 1}
Today I began another form of record keeping: the \say{Sinister Diary.}\footnote{ooh, does the go in the quotation? If so it should be capitalized}
Now, for those of you who don't know,\footnote{so, if you're right handed, haven't taken latin, and also not pedantic} sinister is the Latin\footnote{latin?} word for left.\footnote{sinister,sinistra,sinistrum}
So, what is a \say{left daily allowance}?\footnote{also, note that diary comes from the word for daily allowance}
It's just a diary that I'm writing with my left hand.

In my diary making class,\footnote{liberal arts are weird} we talked about different ways of sparking creative journal keeping.
One of these is to make a diary entry with your other hand.

Of course, calling anything written in my left hand's\footnote{lack of} penmanship implies that that written in my right hand is \say{dextrous.}
Dextrous is apparently the British form of the American \say{dexterous.}\footnote{apparently I'm becoming British more than I thought. Next thing you know I'll be spelling it oxydized}
Dexterous comes from the Latin \say{dexter,} meaning right\footnote{like the hand} and also skillful or proper.\footnote{I shamelessly use wiktionary, like all real scholars}

Now, far be it from me to describe anything written I do as skillful,\footnote{I might make the claim about my music, especially when asked} but it's certainly easier to read than my left-handed writing.
Also, I ran into another problem while writing sinisterly.

When I was a young, impressionable freshman,\footnote{as opposed to the young, impressionable junior that I am now} I took Latin.
I had the brilliant idea to write all of my Latin with my left hand, so as to embed it more deeply.
Of course, the professor shut that idea down on the grounds of legibility.\footnote{or, more precisely, the lack thereof}
But, today, when I tried writing with my left hand for the first time in a while, I noticed that I was thinking of the Latin\footnote{and also Spanish because my mind groups things oddly} translations for a lot of what I was writing.
This continued to the point where I began thinking in\footnote{broken} Latin.
But, I did certainly enjoy it.

Hence, I'll be continuing my sinister diary.
\end{document}
