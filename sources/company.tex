\documentclass[12pt]{article}[titlepage]
\newcommand{\say}[1]{``#1''}
\newcommand{\nsay}[1]{`#1'}
\usepackage{endnotes}
\newcommand{\1}{\={a}}
\newcommand{\2}{\={e}}
\newcommand{\3}{\={\i}}
\newcommand{\4}{\=o}
\newcommand{\5}{\=u}
\newcommand{\6}{\={A}}
\newcommand{\B}{\backslash{}}
\renewcommand{\,}{\textsuperscript{,}}
\usepackage{setspace}
\usepackage{tipa}
\usepackage{hyperref}
\begin{document}
\doublespacing
\section{\href{company.html}{Company Review}}
First Published: 2018 November 12

\section{Draft 1}
Tonight, I had the lovely fortune of seeing George Furth and Steven Sondheim's \textit{Company} performed at the Gielgud Theatre.
It was a show where the main character's sex was flipped.
No longer about Bobby, this show was about Bobbi.

It was incredibly enjoyable.
The set mostly took the form of small, modular pieces which were surrounded by neon lights.
They moved back and forth and up and down stage, and in some instances even up and down vertically.
The music was beautiful, as is to be expected.

Interestingly, the orchestra was above the actors, rather than in a pit.
The actors and actresses all performed brilliantly.

As to the production itself, for most of the show I'm unsure how different it was from the original script.
I went in mostly blind, as I do for most musicals.
However, I had seen a recording of the song \say{Barcelona.}

In this version, where the main character sleeps with a man, rather than the original which had the genders switched, there were some differences in how I felt as an audience member.
Despite the fact that the staging seemed almost identical, it's interesting how different some of the lines and blocking felt when delivered from a person of the opposite sex.

Overall, it was very enjoyable
\end{document}