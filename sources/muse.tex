\documentclass[12pt]{article}[titlepage]
\newcommand{\say}[1]{``#1''}
\newcommand{\nsay}[1]{`#1'}
\usepackage{endnotes}
\newcommand{\1}{\={a}}
\newcommand{\2}{\={e}}
\newcommand{\3}{\={\i}}
\newcommand{\4}{\=o}
\newcommand{\5}{\=u}
\newcommand{\6}{\={A}}
\newcommand{\B}{\backslash{}}
\renewcommand{\,}{\textsuperscript{,}}
\usepackage{setspace}
\usepackage{tipa}
\usepackage{hyperref}
\begin{document}
\doublespacing
\section{\href{muse.html}{To My Muse}}
First Published: 2022 November 24

\section{Draft 1}
At this point, I'm more and more sure that I'm the only one who reads this blog, for all that I rarely do so.
In many respects, that makes this mostly a write-only space.
But, I am currently listening to an audiobook about the creative process, and there are some interesting ideas in it about creativity.

A brief search of my musings\footnote{thanks grep!} shows me that I have mentioned my muse in apparently only two posts, both of which are from my initial blog.
In \href{arranging-for-bagpipes-ii.html}{one}, I talk about how I was blessed with a hyperfocused muse that day.
In \href{making-a-mixtape.html}{the other}, I mention how my muse should be considered to have the name Janet that day.

The book talks about how writing more makes you more creative, though it does so in more of a spiritual sense than I think I am personally willing to ascribe to it.
Still, it made me think about the fact that what I do now does shape who I am in the future.

So, to the future me, I hope that the writing I do now is helping you write better.
To my muse, I hope that my willingness to keep writing will encourage you to keep giving me ideas.
\end{document}