\documentclass[12pt]{article}[titlepage]

\newcommand{\say}[1]{``#1''}

\newcommand{\nsay}[1]{`#1'}

\usepackage{endnotes}

\newcommand{\1}{\={a}}

\newcommand{\2}{\={e}}

\newcommand{\3}{\={\i}}

\newcommand{\4}{\=o}

\newcommand{\5}{\=u}

\newcommand{\6}{\={A}}

\newcommand{\B}{\backslash{}}

\renewcommand{\,}{\textsuperscript{,}}

\usepackage{setspace}

\usepackage{tipa}

\usepackage{hyperref}

\begin{document}

\doublespacing

\section{\href{othello.html}{Othello Review}}

\section{Draft 1}

Tonight I had the pleasure of watching Othello at the Globe Theatre.

In a wonderful turn of events, I was a groundling.\footnote{one of the people standing on the floor}

I was right next to the stage, and even leaning on it for the piece of the play after intermission.


Sadly, the set and lighting didn’t blow me away.

The lights were fixed and immobile, and the set more or less was as well.

What I can comment on, however, was the music.


The show began with natural trumpets,\footnote{trumpets without valves or keys} which was nice.

Come the scene where Cassio becomes drunk, they are replaced with valved trumpets.

Before the first intermission, they bring out the cornetti,


In the second act, the cornetti play lamentations as the piece falls to its tragic fate.

The ending dance, however, returns with the beautiful jazz trumpeting.

Other instruments included a lute during the drunk scene, played masterfully by Iago, drums and other percussion played by instrumentalists, and whistles.


Finally, as is requested by Shakespeare, there is singing.

The drunken songs sounded drunk and merry.

The whole cast song at the end was sung brilliantly.\footnote{and, if I know anything, Baroquely}

But, the song that struck me hardest was Desdemona and Emilia’s duet of the Willow Tree.

They flowed between two part harmony and unisons flawlessly and beautifully.

I stood entranced for the first\footnote{for those of you unaware, I am not myself a Shakespeare fan for a variety of reasons, which may come in a future musing} time in the show.

Nothing existed for me except the two flowing voices and the story they told.


And truly, that’s all that I can ask of a show.

There was a moment where time stood still, and I found myself drawn, not into the story or characters, but simply into a place where I feel what the characters feel.

Even in professional theatre, those moments can be hard to come by, but the cast brought me nearly to that point time and time again, and to the point in the soulful duet.

But, as all good things do,\footnote{like Desdemona} it too came to an end.

\end{document}
