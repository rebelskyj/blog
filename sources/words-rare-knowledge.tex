\documentclass[12pt]{article}  
\newcommand{\say}[1]{``#1''}  
\newcommand{\nsay}[1]{`#1'}  
\usepackage{endnotes}  
\newcommand{\B}{\backslash{}}  
\renewcommand{\,}{\textsuperscript{,}}  
\usepackage{setspace}   
\usepackage{tipa}  
\usepackage{hyperref}  
\begin{document}  
\doublespacing  
\section{\href{words-rare-knowledge.html}{On Words for Rare Knowledge}}  
First Published: 2025 September 6

\section{Draft 2: 6 September 2025}

Now that I'm officially a doctor, I get to live the dream that I've had on some level motivating me since I was a child: the power to define and redefine words as I see fit.\footnote{or as I see absences in the lexicon, who can say which it used to be or even what context it originated in}  
I've also been thinking a lot about knowledge, and since I was unable to get a chapter on theory of knowledge and science in particular in my thesis\footnote{dissertation?}, this next period of the page might be filled with my attempts to come to one.\footnote{really bringing the term folly to be meaningful}  
I'd like to find some like fundamental set of reading for theory of knowledge, and it just as writing that occurred to me I can do that.  
Turns out \say{epistemology} is the study of theory of knowledge, which is definitely something I once knew.  
Although I am a firm believer that reading primary sources is often useful, as a chemist I am also a firm believer that often there's synthetic information\footnote{synthesized?} that is not easily or often transmitted in the original sources.  
With that in mind, I've found a few texts that might help me with my epistemology, which is great!\footnote{no, there is not an entire suitcase of my three dedicated to books and loose pages, that's such a ridiculous assertion, who would even consider that question? (ignore that when I studied abroad I had three cases, one was school, one was clothes, one was musical instruments of course)}

Where was I?

Oh right.

As a doctor and a person interested in knowledge, I want to be precise in my language usage, especially here.\footnote{I want to be precise independent of these two considerations, but ignore that fact please}  
The world, at least as I've known it, loves the idea of secret and hidden knowledge.  
Or, being fully clear, it loves the idea of rare knowledge.

What does it mean for knowledge to be rare?

It can mean any number of things to any number of people.  
From last draft\footnote{dear new readers, I apologize for the self-referential nature of the site}, we have that this knowledge can be hidden,\footnote{occult} require the learner\footnote{student? seeker?} to change their view on reality\footnote{nature? the world? these are not the same I guess},\footnote{eldritch}, be understood at some intuitive and uncommon level,\footnote{gnostic} or be mysterious.\footnote{arcane}  
In that set of distinctions, I did of course level the fact that knowledge can be hidden of its own accord (it's not always obvious to people the outcomes of their actions) or by an agent (as is popular in conspiracy thinking), the fact that the intuitive knowledge may be ancillary or required, and the fact that mystery means many things to many people.  
Still, these are some good starting places.  
Let's see what other words people use for rare knowledge, and see how we might use these distinctions.

We've:  
\begin{itemize}

\item recondite: from the Latin past participle of \say{to conceal}.  
Cool, so occult knowledge is knowledge which is hidden of its own accord and recondite knowledge which is hidden by an external agent.  
The MW\footnote{Merriam-Webster} \href{https://www.merriam-webster.com/dictionary/recondite}{definition} for the word suggests....

\item condite: pickled or preserved.  
There's something to be said for knowledge rare by virtue of having been preserved and therefore unquestioned.  
I cannot think of many times that I'll need to care about it, but I guess we'll keep it in the back pocket.  
Looking at its synonym page:

\item esoteric: for the specially initiated.  
I think that's good enough on its own.  
Esoterica are things which require understanding some fundamental set of axioms and facts\footnote{what is an axiom?} before they can be understood.  
Much of physical chemistry is esoteric.  
Great.

\item profound: difficult for normies.\footnote{can you tell I'm not directly quoting?}  
I have no interest in using profound most of the time.  
By virtue of being rare I feel like most knowledge has to be profound, but maybe I'm too optimistic in others.  
I feel like if learning something is easy, people will do it.

\item abtruse: as with profound. We're going to skip words I don't plan to use moving forward.

\item hermetic: related to the works of Hermes Trismegistus. Generally I'll use this as knowledge which must be revealed from a physical agent to be learned or understood.  
That will stand in contrast to some word meaning revealed by a non-physical agent\footnote{likely revelation/ revealed} and is often going to be recondite and esoteric.  
The truly hermetic works may very well be condite.

\item cryptic: MW says hidden or ambiguous meaning. I'm going to take ambiguous, and say that something like a Zen Koan is cryptic.  
That is, the goal is less the knowledge itself as the pondering which is of value.

\item mystic: MW says mysterious which makes sense when I look at the letters.  
I'm going to use mystic knowledge to refer to supernatural knowledge, standing in contrast to hermetic knowledge.\footnote{hmm, do I want something for natural but non-agentic? Like \say{I understand this because I stared at this rock forever}...eh see if there's any other words}

\item numinous: MW says supernatural or mysterious, also filled with divinity or spiritual.  
I'll use numinous rarely, I have to imagine.  
I guess internally I feel like mystic has an element of experiential knowledge, while numinous doesn't.  
Great.

We'll see if that difference bears out.

\item acintya: from looking up koans, I learn this is the Sanskrit word for unanswerable questions.  
I'll use that in place of Koan?

Maybe not.

Looks like a better usage is for knowledge which obscures future knowledge.  
I can absolutely see the use in something like that.  
Since eldritch knowledge requires viewing creation\footnote{yet another option} differently, it will likely obscure other facts.\footnote{which I use often, saying things like \say{as a scientist} versus \say{as a musician}}

\end{itemize}

What about magic?

I think that magic might be for things which aren't known but are actionable.  
That is, if I have a box that tells me the weather, that's magic.  
For most people,\footnote{myself included. I get logic gates and I get electrons and I get (kind of) coding, but the in between terrifying} a computer is magic.  
Magic is that which is not known.

Do I want to somehow distinguish the weather box that is unknowable from the computer, which is theoretically knowable?  
Do I want to distinguish being able to understand parts but not the whole\footnote{no, that's just emergent properties?}?

We'll see if it comes up.  
Why is this relevant?  
Oh, that's probably a great thing to cover too.

New draft time?

Nah, let's\footnote{me, and future me} figure out the reason why here.

I think that knowledge is not commonly held for a number of reasons.  
I was talking to friends last night, and one of them mentioned that in Psychology, there's a growing idea that it is useful to learn about the emotional words with no direct cognates in a given language, as they can potentially help others to better understand or name their experience.  
Even before that, though, the idea of unknown or not-widely known knowledge being for a number of reasons is something that's entranced me.

Is this related to my faith questions at all?  
Honestly I don't know.  
Probably, since on some level it's also my whole \say{what do I know and what do I believe?} thing.  
I'm hopefully\footnote{The desire to make the word a parenthetical or footnote was great, don't worry} never going to be a pure Descartian \say{all I can prove is that I am having thoughts}, and generally am completely fine with the idea of knowledge, objective reality, and the transmission of both.

That's probably a good thing to have printed on the top of each page of philosophy I read.  
There's something to be said for \say{touch grass} as a general catch-all for \say{if you have an issue with acting in the world from learning something new, maybe you shouldn't base your life on it}.  
Then I do get to the whole concept of a useful lie, which is ope another whole thing I need to consider.

What are things which are untrue but used to lead to greater truth?

Model is a word, useful lie is a term.  
I think that both ill serve my purpose for now.  
A lot of esoteric knowledge is recondite because learning the useful lies which prepare you for understanding the occult knowledge hide the deeper fact.  
As an example, people hate the idea of the electron cloud because we learn the ball and stick model, and then the orbital model.  
The fact that electrons co-exist in the same space feels even harder to understand because we have explanations for the behavior.  
Should I try to come up with an example of each form of hidden knowledge and explanation?

Probably.

Will I?  
Eh.

Let's just write up the words here and call it for the day.  
I don't want to burn out on writing, and this is a solid more than two thousand words.\footnote{I'd ask where this writing speed was during my thesis, but the answer is equal parts this is fully what I want, this is unfiltered, burnout, and it doesn't need to be accurate in the same way}

\section{Draft 1: 6 September 2025}

Something that's bothered me for a while is the idea that we have so many words which mean the exact same thing.  
In general, words have slightly different flavors of meaning, which is totally fine for me.  
When talking about knowledge, especially knowledge which is not widely held, there does not seem to be this same distinction.  
Since I've already accepted that what I write here is going to be explicitly and intentionally\footnote{both in the sense of using intention and the sense of it not being an accident} using language in a way that is not the norm.  
With that in mind, I want to start distinguishing the words for rare knowledge, especially because I am about to embark on a train journey where I have the intention of reading a number of philosophical works.\footnote{yes, for those who have seen what I packed, I do consider music theory a form of philosophy}

So, then, what words do I need to find a specific flavor for?  
I'm going to go through them as I can think of them, and then compile at the end.

Having just watched \href{https://www.youtube.com/watch?v=JW7obfLcW9U}{a video about Albert the Great}, I will accept that occult means hidden.  
That is, occult knowledge is not simply unused by virtue of being boring or just undesired; it is actively suppressed or kept from sight.  
I feel like I want a way to distinguish knowledge which is hidden on its own account and knowledge which is hidden by some deeper power.  
That's something to keep in mind going forward.\footnote{at this point I remembered that my tradition in the follies is to reflect first, write second, so there I go}

I really want the word eldritch to come with connotation of causing a fundamental change in the receiver.  
That is, knowledge is eldritch when being able to understand it means that you must see the rest of the world differently forever.  
I have a whole idea of a post about the idea that all knowledge is fundamentally eldritch, but then I think that I'll get into something circular.  
For now, I'll just stay with eldritch knowledge cannot be learned without changing oneself in a deep and permanent fashion.

Gnosis is a popular term whenever you talk to Gnostics or annoying Catholics.\footnote{with love, as an annoying Catholic}  
It's the Greek word for knowledge, but generally has a connotation of being secret.  
It also has a denotation, according to the internet, of being intuitive.  
From this, I'll take the intuition portion.  
Knowledge is gnostic\footnote{lower case to separate from the religious sect} when it can only be understood at an intuitive level.  
Or, at least, it is gnostic when it is understood at an intuitive level.  
Hmm, I feel like there's a difference between fundamentally gnostic and practically gnostic words.  
Another place we might have an option.

Arcane is another word.  
Arcane just comes from \say{arcanum}, a second-declension noun meaning secret or mystery.  
I don't really want to get into the whole \say{mystery means something different in the Catholic tradition}, but I'm happy for arcane knowledge to be information\footnote{oh wait is knowledge and information actually synonymous? (are knowledge and ... ? I don't know grammar anymore, and I am doctor, so do not need to. Shoot, I should probably bring that up.} which is counterintuitive or mysterious.

Ok I realize now (see above footnote)\footnote{I do wonder how rare my parentheticals are} that I need to reframe this entire post.  
So, Draft Two, here we go.

\section{Daily Reflection 6 September 2025}

\begin{itemize}

\item Did you journal by hand today?

No, also not yesterday, for all that I had planned to.

\item Did you do a folly?

Not yesterday; not having WiFi\footnote{more and more it bothers me that we capitalize it when it stands for nothing} does a lot to prevent me from writing on my preferred site.

\item Did you in some way, shape, or form advance the web novel?

Not yesterday, but I have about half of it bound and the other half folded. I'm planning to spend some time tonight doing the rest of the binding, but I did also just bind a book for a friend, and so I don't want to burn out of it.

\item Did you work on music, whether education or creation?

No, but I did listen to a lot and I made a playlist for a\footnote{if luck goes my way} future friend.

\item Did you work on book binding?

Yeah! As mentioned above.

\item Did you work on another hobby?

I packed! I also played some video games, which are a hobby for all that I don't want them to be.  
Also I have been reading a lot of a web novel I really enjoy, \say{Chaotic Craftsman Worships the Cube}, which I hope to once again finish soon.

\item Did you stretch? Really?

Not a lot, but some! I will do more stretching when I get back to my apartment, but in order to get back to my apartment I need to accept an absence of WiFi or internet generally.\footnote{No, I haven't heard of a hot spot, why do you ask?}

\item Prayer?

Not a lot, but I do still say a \textit{Lux Aeterna}\footnote{if other common prayers get to be called by their first few words in Latin, so does it} every time I go by a cemetery\footnote{wayyy more e than I thought that word should have}, and I went to a Shabbat dinner last night, which included some explicit prayer.

\item Meditation?

I was felling incredibly stressed going to the Shabbat last night, and so I did some intentional breathing then.  
Other than that, I guess that I'm trying to quiet my mind before sleep lately.  
Right now I'm not wearing headphones and not listening to anything in particular, despite the fact that I am in public and so therefore near enough to people that I cannot say I'm in silence.

\item Reading?

Other than the web novel mentioned above, I caught back up with the book about soccer my brother and I enjoy.

\item Minimizing screen time?

Not at all! Wooo, go me. I will hopefully start on this later, when I can think on paper rather than here.\footnote{though there is something to be said for the fact that pre-writing like that will mean that my thoughts are inherently miles away more filtered than they could be here... Eh, something to consider}

\end{itemize}

Current Pen List\footnote{for my own posterity, mostly}

\begin{itemize}  
\item Hongdian Black with Fude Nib: Diplomat Caribbean (8/30ish)  
\item Jinhao Shark: Diplomat Caribbean (8/30ish)

\end{itemize}

\end{document}

  