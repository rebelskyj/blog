\documentclass[12pt]{article}[titlepage]
\newcommand{\say}[1]{``#1''}
\newcommand{\nsay}[1]{`#1'}
\usepackage{endnotes}
\newcommand{\1}{\={a}}
\newcommand{\2}{\={e}}
\newcommand{\3}{\={\i}}
\newcommand{\4}{\=o}
\newcommand{\5}{\=u}
\newcommand{\6}{\={A}}
\newcommand{\B}{\backslash{}}
\renewcommand{\,}{\textsuperscript{,}}
\usepackage{setspace}
\usepackage{tipa}
\usepackage{hyperref}
\begin{document}
\doublespacing
\section{\href{aleatoric-lite.html}{Aleatoric Lite Music}}
First Published: 2022 January 26

\section{Draft 1}
As I mentioned yesterday, I'm currently writing a piece which could be described as aleatoric lite music.
More or less, my goal was to create a piece that can be continued for as long as needed.
Then I decided to add constraints, because what's the point of life without a little\footnote{read: lot of} chaos?

My first constraint was that I wanted it to be a round and/or canon melody.\footnote{for those curious, the distinction mostly comes down to the amount of times a melody is repeated}
Those are fairly easy to write, as it turns out, so that was great to learn.

My second constraint changed.
Initially, I wanted to have each voice have the melody, writing four part voice leading to make that fit.
However, the issue with that is that then you functionally have to write four different parts.
So, I decided that I would add a new constraint: there would be one harmony set, and the voices would trade to produce the different arrangements for voices.
I'm still unsure between my current plan of having 4 voicings and having 12, but since the tenor\footnote{baritone} line and the bass line as melody currently are the same basically, I'm not sure what difference the audience might hear between the two options.
Though, as I write this I realize that one of the goals was to keep it musically interesting moreso for the performers than for the crowd, so I guess the 12 options is more fun.

A new idea that's come to me since writing this blog is to have non-even loop lengths.
In all honesty, that should have been my first idea for aleatoric music.
The general idea behind the concept is that you have, for instance, a three beat line, a four beat line, and a five beat line.
If you start them all together and then let it run, with each voice looping as soon as it finishes, you end up with 4*3*5 =60 beats.
From 12 beats of written music, that isn't so bad.
Of course, that scales really well.
If you write a 12 beat and a 13 beat part, for instance, those 25 beats you've written generate 156 distinct beats.
A 13,14, and 15 beat part gives you 2370 beats of music for 42 beats of writing.
The issue for me is that generally if you want these beats to sound good, you're pretty limited to a single chord, though I guess it isn't at all uncommon to have them in the same diatonic world.
Once I finish the four part writing, I might try generating something that doesn't evenly go into 32\footnote{so anything but powers of two really}, and see if maybe that sounds nice.
The biggest issue there is that timing becomes harder, because unless you copy paste the same melody line over and over, people tend not to like phasing.
I guess I'll check with the conductress and see if she has any thoughts on the looping idea.
Maybe she'll like it, though I really doubt it.

Anyways, I've got the first two measures of the four part harmony written, and the whole canon is finished as well.
At this rate I should be finished by the end of the week, but that's a dangerous thing to claim.

Words:
532 and 28
\end{document}