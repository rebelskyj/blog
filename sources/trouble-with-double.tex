\documentclass[12pt]{article}[titlepage]
\newcommand{\say}[1]{``\#1''}
\newcommand{\nsay}[1]{`\#1'}
\usepackage{endnotes}
\newcommand{\1}{\={a}}
\newcommand{\2}{\={e}}
\newcommand{\3}{\={\i}}
\newcommand{\4}{\=o}
\newcommand{\5}{\=u}
\newcommand{\6}{\={A}}
\newcommand{\B}{\backslash{}}
\renewcommand{\,}{\textsuperscript{,}}
\usepackage{setspace}
\usepackage{tipa}
\usepackage{hyperref}
\begin{document}
\doublespacing
\section{\href{trouble-with-double.html}{Trouble With Double}}
First Published: 2019 January 08
\section{Draft 1}
Today, as with most days for the next few weeks, I dove.
Our goal was to run through every dive, with the added hope that I might be able to put inward two and a half back in.\footnote{for those of you who know my personal life, that dive is the dive that caused my (allegedly) only concussion}
Instead, I remembered why I hate inwards.

In front category dives, I tend to have an easy time.
Regardless of how I ride the board, I can generally save it.

In back dives, I just need to remember to swing my arms.
And, I do that often enough.

For reverse, I always know that I've gotten off the board too quickly.

But, for inward, the only element I really need to throw a quickly rotating dive is a good toe drive.
That is, I need to throw my feet back as they come up from the board.
Throwing my feet is scary, and makes me think I might hit the board.\footnote{no, I've never even gotten too close to the board}
So, today's inward double attempts were interesting.

On the first, I underthrew just enough that I landed in what the lifeguard called a \say{perfect cannonball.}\footnote{it hurt a little}
The second I underthrew to the point that it turned into a one and a half, because I knew I wouldn't make it, and I've never been horribly fond of hurting my back.\footnote{contrary to popular opinion}
After recognizing this problem and addressing it through an inward dive,\footnote{which was very, very over} I thought myself ready to again attempt inward double.
Instead of it being under, I had a wonderful toe drive.
In fact, I rotated so well that I was able to kick out at the double above the board.

Unfortunately, this meant that I was over on the dive, and it didn't feel pleasant.
Had it been a leadup for inward two and a half or double on one meter, it would have been great.
As it was, however, it was just painful.
But, tomorrow is another day, and another chance to try again.
\end{document}