\documentclass[12pt]{article}[titlepage]
\newcommand{\say}[1]{``#1''}
\newcommand{\nsay}[1]{`#1'}
\usepackage{endnotes}
\newcommand{\1}{\={a}}
\newcommand{\2}{\={e}}
\newcommand{\3}{\={\i}}
\newcommand{\4}{\=o}
\newcommand{\5}{\=u}
\newcommand{\6}{\={A}}
\newcommand{\B}{\backslash{}}
\renewcommand{\,}{\textsuperscript{,}}
\usepackage{setspace}
\usepackage{tipa}
\usepackage{hyperref}
\begin{document}
\doublespacing
\section{\href{reflections-on-readings-1-advent-b-23.html}{Reflections on Today's Gospel}}
First Published: 2023 December 3

\section{Draft 2}
One line in particular stuck out to me in today's readings.
In the first reading, the Prophet Isaiah says that \say{we are the clay and you are the potter.}\footnote{Isaiah 64:7B}
Now, my mind immediately leapt to two ways that this can be interpreted.

First, despite the way that most modern people interact with clay, it does not come to us perfectly pure and ready to be used.
Instead, it has to be refined more or less depending on the soil that it is in.
We, the people of G-d are clay.
Much like last week's metaphor of sheep and goats, we can point to that in contrast to the silt and sand of sinfulness.

Once clay is gathered, however, it is rarely simply formed and fired.
Nearly every culture has a tradition of incorporating dust from a previous, often failed\footnote{if only because the successfully fired pots would likely be used} project.
It is at this point that I feel the need to reflect on the different \say{we's} that the Prophet could be referring to.

First, we could mean each of us individually.
We are all lovingly and perfectly shaped by G-d our Father.
However, even as we are shaped, we are not alone.
Pottery is a fundamentally useful craft.
It creates objects to be used with others.
Similarly, whatever the Almighty shapes us into, it is meant to relate to the rest of the world in some way.

Additionally, we are not the first to be born.
Prayers from those before have their positive effects, like the powdered pieces of a previous project.
Or, just as it is easier to shape a new pot once you've made an older one, it is also easier to find the Lord when your parents have known Him.

Of course, we could also be referring to the entire People of G-d.
We are each a small piece of the clay in the pot of creation that the Lord forms.
In that regard, the fragments of old pieces can be thought of as the parts of ourselves and our cultures we bring from times before Christ.

Nearly every Parish I've visited uses an evergreen wreath to hold its Advent candles.
Why?

Of course, there's the obvious answer of \say{we've always done it that way, and it's pretty,} but of course, we can not have always always done it a certain way.
Someone had to be first.\footnote{I've just now learned it's apparently initially a Lutheran tradition, which is fun and interesting and adds a whole meta level to my \say{The Church takes the best from everything that's somewhat wrong}}
Some among you might know that we took the concept of the Christmas tree from Germanic Pagans, for whom evergreen trees were a reminder that the depths of winter would pass.
That sentiment lead us to using the same for Advent wreaths.

There was a truth in the pagan tribes that still resonates with the Church today.
G-d who fashioned the entire universe can be seen in all that He has created.
The beauty of evergreen boughs, especially in contrast to the white of winter, reminds us of so many different truths.
In such a way, the beneficial parts of pre-Christian lives and cultures can enrich the lives of the faithful, just as the advent wreath enriches many believers' Advents.\footnote{wow this second draft went way better}

\begin{itemize}
\item Hobbies:
\begin{itemize}
\item Did I embroider today? Yes! It's been fun, and I got to spend like an hour working with friends, which was really great. I know one of my goals this year was to grow closer to the people I care about, and I think that this helped.
\item Did I play guitar today? Once again, a scale and my etude were the majority of what I played.
\item Did I practice touch typing today? I did a few lessons. It's fun for me to see that I lose almost all of my progress each day whenever I log back on. I think that if I did more lessons I would probably lose less between days. Eh.
\end{itemize}
\item Reading
\begin{itemize}
\item Have I made progress on my Currently Reading Shelf? I read maybe 3 minutes today, which isn't great, but today passed by so much faster than I thought it would.
\item Did I read the book on craft? I decided to take the day of from reading the book on craft, because it's Sunday.
\item Have I read the library books? Last night I read a chapter. It was really interesting, and I'm excited to see where the rest of the book develops.
\end{itemize}
\item Writing
\begin{itemize}
\item Did I write a sonnet? Did it! I think I'm doing better at imagery and stopping myself from writing into a corner with bad rhymes.
\item Did I revise a sonnet? I still don't know if I actually know how to revise a sonnet.
\item Did I blog? Woo! I did it.
\item Did I write ahead on Jeb? I think that I'm still comfortable saying that I won't work on Jeb on Sundays, for all that I know that means tomorrow morning I'll need to panic write.
\item Letter to friends? I did not. It's in my fanny pack, though, which means that I can in theory write it whenever.
\item Paper? It's Sunday so I will not be doing work.
\end{itemize}
\item Wellness
\begin{itemize}
\item How well did I pray? Not great. Went to Mass and did a rosary last night, both of which are good.
\item Did I clean my space? No.
\item Did I spend my time well? I think mostly! I had a day full of seeing people, and the content I consumed was mostly content that I intentionally chose
\item Did I stretch? I went diving! Which counts as both stretching and exercise in my book.
\item Did I exercise? See above.
\item Water? Drank a bunch during and before Mass because I needed to sing. After that, though, I have not been doing a great job remaining hydrated.
\end{itemize}
\end{itemize}

\section{Draft 1}
Happy new year!
I was reminded today\footnote{reminded here meaning that I'm sure I could have pieced it together and may have in the past but it was surprising to hear} that this year is the shortest possible advent at three weeks and one day.\footnote{Christmas is on a Monday, which means that the fourth Sunday of Advent is one day before Christmas, making it the shortest possible.}
That's not really relevant to the rest of this reflection, but it's an interesting fact nonetheless.

Advent is the beginning of the liturgical year.
It is a time of preparation and reflection before the celebration of the coming of our Savior.
Of course, as is often brought up, Christmas only has meaning because of Easter.

Yesterday's first draft talked about iconoclasm and its opposite.
Every year, I feel a little more frustrated at the state of discourse surrounding Christmas and its accompanying rituals.
These days, both sides of the extremes\footnote{horseshoe theory rears its ugly head} claim that such beloved traditions as Christmas trees are inherently pagan.
Of course, there is truth to that claim, at least.\footnote{the claims that Christmas itself is a pagan celebration are less historically supported and mostly bolstered by people whose explicit goal in research is discrediting early Christian history, which is not what I generally consider a balanced source}

Christmas trees were, as reported by the early Church, used in Germanic pagan winter solstice celebrations.\footnote{the fact that the only thing we know about most pagan European cultures comes from the Church seems to get ignored sometimes}
However, just as music can bring people to the Almighty, and just as people and nations can be baptized, the Church in her glory recognized that rituals can serve to help lead us to Christ.
The Church never claims to be the only place that truth is found, just the best.
Anyways, all this to say, there is nothing wrong with remembering that life will come again and that winter will come to an end.

Today's readings remind us that everything will come to an end.
Christ compares His second coming to a man returning from a voyage abroad.
However, the Gospel is not where I want to focus tonight.

Instead, I would like to focus on a single line from the first reading.
\say{We are the clay and you are the potter.}\footnote{Isaiah 64:7B}
Now, when I think of clay today, I think of it almost exclusively in an artistic context.
What little I've worked with clay has come from prepared and purchased material of high purity.

Obviously, this is not how all clay has been gathered throughout history.
Clay can be gathered from nearly any soil, though some soil obviously has more or less clay in it than others.
It needs to be separated from the silt and the sand that also makes up soil.
I'm sure that someone wiser than me could connect that to the metaphor of sheep and goats from last week, but I can't seem to make the leap tonight.

The other piece of pottery lore that I remember right now is that nearly every culture has a tradition of incorporating a small portion of an already fired work into new clay.
There are some really interesting materials reasons for it, and indeed, having a small amount of reused clay improves the final product.
Now, depending on what we means\footnote{I'm becoming the insufferable philosopher I dread aren't I}, I can connect that to the reading.

Assuming we refers to each of us as individuals, it can be a reference to the fact that it's easier to find the Truth if you are raised in a family who believes.\footnote{I assume that's true, given that most people historically followed the religion of their parents}
It is not essential however, which is part of where the Church's mission to convert the entire world comes in.

If we refers to the People of G-d, then the pieces of prefired clay are traditions like Christmas trees and gift giving.
We are able to most fully express our love for the Almighty not by turning away from absolutely every part of our old life, but by baptizing and purifying everything we do.
The best pot is formed by taking parts from a pot that did not work.
Similarly, the best culture\footnote{I feel uncomfortable sometimes saying that some cultures are better than others. I think that it is kind of an objective thing sometimes, but of course there's a lot of nuance that is needed} is not created from new cloth entirely, but incorporates parts of the previous.

I don't know if that's at all coherent, but it resonates with me.
Probably worth revising and seeing if I can't make it a little better.

\end{document}