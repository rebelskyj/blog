\documentclass[12pt]{article}[titlepage]
\newcommand{\say}[1]{``#1''}
\newcommand{\nsay}[1]{`#1'}
\usepackage{endnotes}
\newcommand{\B}{\backslash{}}
\renewcommand{\,}{\textsuperscript{,}}
\usepackage{setspace}
\usepackage{tipa}
\usepackage{hyperref}
\begin{document}
\doublespacing
\section{\href{back-in-shape.html}{Musing on Getting Back in Shape}}

First Published: 22 January 2025

\section{Draft One}  
I know that somewhere I've mused before about my fitness goals.  
Though they've changed over the years, I think that they've remained relatively static since college, or at least in the past few years.  
These days, my goals are primarily to gain flexibility and endurance.

In college, I would often joke about the fact that I had steadily gone from sprinting sport to more sprinting of a sport.  
Football, despite being nominally only a few seconds per play, ends up generally being a few minutes of sprinting back and forth until possession changes.  
As a person whose only real event in swimming was the fifty freestyle, those few minutes became twenty to thirty seconds\footnote{less than thirty most of my high school career, sadly never less than 20}.  
Then, in college, I transitioned to diving, where each three second burst of effort in a meet was surrounded by at least five to ten minutes of rest.

Through all of this, I was relatively strong.  
Even today, I'm fairly sure that I can, at least once, lift more than most of the people I know.  
There's something to be said for focusing on strengths, but I don't think that applies here.

For one, I have always preferred being a jack of all trades to a master of one.  
For the other, given that I'm already stronger than average, the number of situations where I'll need to lift something beyond my current limits but within my theoretical limits is relatively small.  
By contrast, the number of times that I'll want to rush somewhere without losing my breath, or be able to contort my body to get somewhere\footnote{because, much as there are any number of benefits to my size, there are some downsides} is far higher.  
With that in mind, while working through the group fitness offerings, I prioritized cardio fitness and flexibility over strength.\footnote{As it turns out, even the strength portions are relatively cardio heavy, and a few friends of mine who've gone said that they don't think that I'd get much out of it. I don't think that's true, given my lack of cardio fitness, but there is something to be said for aiming to improve a trait by focusing on it}  
Today, I tried my first barre class.

In retrospect, I'm not entirely sure what I expected.  
If I force myself to come to a conclusion about what I thought we'd be doing, it was something like plies and other such movements on the barre\footnote{is it just bar in this context? Given how pretentiously French everything in ballet is, I'll assume not}.  
Instead, it was, as the instructor said, a mix of plyometrics, calisthenics, and ballet.  
We did do a little bit on the barre, but mostly we used it for balance.\footnote{which, again, in retrospect, is what it's always used for}

I struggled with the workout a lot.  
I hadn't realized quite how out of shape I am, but that was a great reminder.  
Somewhat surprisingly, to me at least, the fact that it was so painful and hard for me motivated me to go more regularly.  
I rely on my body for so much, so seeing that I'm letting it down in general is not something that I take lightly.

I still didn't make it to yoga today, but I have hopes for tomorrow.  
Otherwise, I think that I was in general decent at following the flowchart.  
I'm not sure how much of it is a thing that I need to do in order and how much is just a \say{yes, this is a list of activities I need to ensure that I do at some point.}  
Still, I think that I'm doing better for having the organization, even if it is far from optimal.  
After all, the only way to improve is to start.  
Wherever I am right now is where I am.  
The only way to get closer to my goal is day by day.  

\end{document}