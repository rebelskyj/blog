\documentclass[12pt]{article}[titlepage]
\newcommand{\say}[1]{``#1''}
\newcommand{\nsay}[1]{`#1'}
\usepackage{endnotes}
\newcommand{\1}{\={a}}
\newcommand{\2}{\={e}}
\newcommand{\3}{\={\i}}
\newcommand{\4}{\=o}
\newcommand{\5}{\=u}
\newcommand{\6}{\={A}}
\newcommand{\B}{\backslash{}}
\renewcommand{\,}{\textsuperscript{,}}
\usepackage{setspace}
\usepackage{tipa}
\usepackage{hyperref}
\begin{document}
\doublespacing
\section{\href{thanksgiving-2023.html}{Thanksgiving Ahome}}
First Published: 2023 November 23
\section{Draft 1}
I often feel like even when I reflect on a day, the hours pass by in a blur and I have no clue what I've done with any of them.
I don't really want that to be the case, especially today.
So, as much as possible, I'm planning to treat today as a \say{day in the life} log.

Let's start where my morning started.
At seven am, my alarms went off.
I groaned and kept my eyes closed for a few minutes, too tired to stand up.
By the time I managed to remind myself that I was about to go make bagels, an activity I'm excited for, it was closer to seven fifteen.

Downstairs, I found that the bagel dough had risen a lot overnight.
It was an incredibly soft and supple dough, which I was grateful for.
I felt called to make larger bagels than we have in the past.

I suppose I should explain.
My family has a tradition that's around 8 years old now of making bagels on Thanksgiving morning.
If I remember correctly, it began because I was a little sad that my family didn't really have any fun traditions, and we were also planning to bring a bunch of friends over for Thanksgiving.
I also was in a trend of baking bread at that point, and I wanted some way to feel connected to my grandmother.
The combination of all of those meant that I thought it could be really fun to make bagels on Thanksgiving morning.

The book we first made the bagels out of said that a single recipe would make two dozen.\footnote{I love old cookbooks, which never assume that you're cooking for anything less than an extended family}
However, bagels have clearly grown much larger since the book was written, because each of the two dozen bagels was far, far smaller than what I could get at any bagel store.
This year, since I did not measure anything, I decided to make bagels that felt closer in size to what I'm familiar with in the rest of my life.

Somehow, this meant that instead of the 30 or so bagels I expected to make, I had 18.
That did mean that the boiling and baking went much more quickly, because my family's oven can easily hold two trays of bagels.
The pot was barely too small for six of the bagels at the size I made them, but it worked out fine.

After shaping and boiling,\footnote{while finishing up an audiobook}, my father came down.
We chatted as I put the bagels into the oven, and we cleaned up from the mess.\footnote{read: he did most of the cleaning, which I appreciated}
As the bagel finished, we threw a butter braid into the oven, because we all wanted something sweet in addition to bagels.

My mother came down, and we chatted for a bit over coffee and bagels.
I delivered a few to a friend of the family, which is another long running part of the tradition.
Since the first year we've made thanksgiving bagels, we've also delivered them to people around the community.
In years where we made a few hundred, we gave out far more.
Nowadays, the list of deliveries is far shorter, which is a relief.

Coming home, I helped arrange the table and then chatted with my family.
At around ten fifteen, I started thinking about writing.
I did a few typing exercises, since I would like to be faster at typing.\footnote{at this point, I'm honestly limited more by my ability to key stroke than my ability to conjure up words to stroke onto the page}
I planned out the work I want to do today.\footnote{even though it's a holiday, I enjoy most of the things that I do, and I want to stay in the habit of doing things like blogging and writing the books I'm writing}
I know for certain that writing my to do list on paper is more effective for me, if only because I tend to ramble and journal around my to do list when I do it virtually.
However, the journalling is a nice way to reflect on my thoughts, and it's far less coherent than I ever want to put out into the world, so it is what it is.

At that point, it was time for me to start writing this blog.
I sign off for now at twenty to eleven, because the rest of the family is up and about and we're going to start chatting and planning the cooking for the day.
We're more or less unscheduled until 1300, at which point I'll need to spatchcock turkeys and start cooking
This year, we're shooting for a 5 pm eat time.

Well, I'm signing back on at 2100 or so.
I chatted with my family, which was fun, and ended up spatchcocking the turkeys at close to half twelve.
Due to some confusion with spices\footnote{there was an argument about what spices to use and how to grind them and how to add them to butter}, it took a while to get them ready to bake
Since my hands were covered in turkey juice and, quite frankly, turkey, I did not weigh in.
Given that we were still well ahead of time, I didn't worry about it.

The internet assured us that it should take about eighty minutes to cook the turkey.
We didn't believe that, and assumed two to two and a half hours.
However, as it turned out, we only needed ninety minutes, if only that.
I have strong memories of this being true in past years as well, at least since we've started spatchcocking.
However, despite the fact that the turkey is fairly overcooked, by traditional standards, at least, the breast meat remains juicy.\footnote{the fact that we put about a pound of butter between the skin and turkey breast probably doesn't do anything to help that.}

While the turkey started to cook, we quickly browned the giblets, neck, and backbones.\footnote{we cooked two turkeys}
Once they had a little bit of color, we added them to a pot of chicken stock we'd made\footnote{read: our fantastic extended family made and brought} and started simmering the two together.
The stock ended up getting used for stuffing\footnote{dressing, if we're being technical, since we didn't stuff it into anything (well, other than ourselves and the leftovers into their containers)} and gravy.

After that, we started making cranberry sauce and other such dishes.
We finished everything about an hour ahead of schedule, at which point we did some cleanup and had dinner.
After dinner, we finished cleaning, hung out for a while, and had dessert.
At some point during the day, someone came to visit.

I'm sure other events happened throughout the night, but I cannot remember them.
Right now, part of the family has gone to bed, and the rest of us are all watching a dumb Christmas movie.\footnote{while doing different activities as well.
I'm working on this blog, my little brother is editing all of my self published writing,}
That feels like as good of a place to sign this off as any!

Also, I found out just now that I've never posted about bagels.\footnote{at least, I don't have anything when I search for a blog post starting with the term bagel.
It's possible that I have, instead, something starting with recipe.
Eh, that's something to learn another time}

Daily Reflection:
\begin{itemize}
\item Did I write 1700 words for NaNoWriMo? Not yet! Will do after this\footnote{and editing my writing, because my little brother has finished}
\item Did I write a chapter of Jeb? No. It's thanksgiving, maybe I'll do some writing tomorrow, once the day kind of nominally calms down.
We'll see.
\item Did I blog? Again! A blog, that's nice. It was fun
\item Did I stretch? I cooked all day, which was pretty exhausting. Exhausting may be the wrong term. It was tiring, as a day well spent often is.
\item Am I doing better at prayer than a rushed and thoughtless rosary? Nope!
\item Am I doing a good job writing letters to friends? Still no.
\end{itemize}
\end{document}