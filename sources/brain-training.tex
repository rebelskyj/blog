\documentclass[12pt]{article}[titlepage]
\newcommand{\say}[1]{``#1''}
\newcommand{\nsay}[1]{`#1'}
\usepackage{endnotes}
\newcommand{\B}{\backslash{}}
\renewcommand{\,}{\textsuperscript{,}}
\usepackage{setspace}
\usepackage{tipa}
\usepackage{hyperref}
\begin{document}
\doublespacing

\section{\href{brain-training.tex}{Brain Training}}
First Published: 2024 December 13

\section{Draft 2}
Because I am actively trying to return my library books right now\footnote{there are a variety of reasons for this, but most of them boil down to me realizing that I'll never finish if I don't start and my research being a lot of start a test, wait for hours, check results}, I've started to actually try to read them.
A past version of me was incredibly optimistic, and thought that I would be able to read through tomes upon tomes of information about historical science and synthesiz it into my public facing talks.
Of course, life happened, and I did not end up doing so.\footnote{Ok so to be fair, I don't know if I can really entirely blame life, especially given what will follow}
In attempting to read some books which are little more than collections of essays, I have come to realize that I've somewhat lost the ability to read essays, especially social science essays.
Thinking about my brain as a muscle may have problems from a psychological standpoint\footnote{that is a statement that is just obviously true, but}, but it's been something useful for me in the past.
Just like how swimming is harder after taking a break, so too is it harder not just to think, but to think in any particular thought pattern after taking an extended break from it.
I recently noticed where my mental strength had waned on Sunday as I tried to write a hymn harmonization.
For whatever reasons, I most often find myself writing polyphonic music, or at least homorhythmic music.\footnote{I know that there's a term for music that shares rhythm but is multiple melodies}\footnote{Oh cool the term was homorhythmic}
Hymns, by contrast, are almost always homophonic.

That is, there is one single line that can be clearly pointed to as \textbf{the} melody, and the rest act to support it.
At a more fundamental level, though, I do not work in the realm of tonal harmony that often.\footnote{I use the definition of tonal meaning something approximating \say{a single diatonic scale at a time, usually through the entire piece, with emphasis on the I chord, the IV chord acting as sub dominant, V as dominant, and resolutions at I. Usually this means that the seventh resolves to the first and the fourth resolves to the third.}}
Sure, many of the folk songs that I write take the standard I IV V approach, but the music I listen to, cover, and especially write for choirs do not rely on that very limited harmony.
Hymns, as a genre, however, operate completely within that sphere.
Because I have minimal interest in writing tonal homophony, I was very comfortable with the fact that I have lost those skills.\footnote{For those then asking why I'm practicing it now, the answer boils down to the fact that the conductor I'm hoping to write music for generally likes homophony and relatively tonal music}
By contrast, I've realized that I cannot read essays, especially in the way that I used to, and that is something which concerns me, at least a little.
I don't know where the mental fortitude to slog through dense words that are written from an expert in a field to another expert in the same field has gone, but I do know at least a few reasons why I've lost it.
First, scientific papers are rarely written as essay.
These days, most of the time the abstract and conclusion are all that need to be read.
If attempting to copy an experiment, than a brief skim of the experimental section is usually sufficient.
I am also rarely attempting to do pure literature reviews, where I synthesize a number of papers into a single document without adding new information of my own.\footnote{that feels like it might have come off somewhat aggressively towards the social sciences. That wasn't the goal, and I absolutely think that synthesis is value, even if it is not explicitly novel in the same way as measuring something for the first time.}
Now I guess I have to ask myself whether that's a skill I want back, and if so, how much effort I'm willing to put into it.

Goals:
\begin{itemize}
\item One offs:
\begin{itemize}
\item Talk to boss about Ph.D. timeline
\item Pick a topic for a science communication article
\item Find an occasion I could write a song for. Done, I'm going to set the Ave Maria
\item Make a list of the stretches I'll do each day
\item Find a place to volunteer
\item Paper hit list
\item Compile a list of people I want to write letters to
\item Muse about macros and micros
\item Compile a list of 20 meals that I can make, with their ingredients (inc. shelf stable or lifetime), time, effort level, and nutrition info
\item Figure out my motivation for each book and have it as the bookmark
\item List of things that need to be cleaned and the frequency
\item List of things in my life
\item Make a list of musings to do
\end{itemize}
\item Weekly:
\begin{itemize}
\item Read a pop sci article a week, making notes about how they work -> Decided I no longer want to do this
\item Spend 30 minutes 2x a week working on writing the song -> will work sunday
\item Ten minutes 4x a week on drawing -> Instead of drawing yesterday, drew today
\end{itemize}
\item Daily:
\begin{itemize}
\item Define how I'm feeling each day at start and end -> Still good
\item Practice guitar daily (at least one scale and a chord progression) -> Still on
\item Muse daily -> Woo! A streak
\item Stretch Twice a day -> I think that I may have forgotten yesterday, but I have today
\end{itemize}
\end{itemize}

\section{Draft One}
I found myself struggling to think of a topic for today's musing.\footnote{No, I don't have any good answer for how I pick between calling it a musing and a blog}
Potentially relatedly, I've been trying to get through the long list of library books that I checked out from the library at my school.
Among them are books that have come recommended to me and a number of books I've found by simply wandering the stacks.
I often forget that History of Science is its own field, and that tends to be to my own detriment.
Because of the fact that it is History of, the field's output more closely resembles that of history than that of science.
That does make sense, given that it is tools and techniques from history which enable the research.
However, it does mean that I'm being confronted with a fact that I haven't had reason to realize: I've lost my ability to read essays.
Hmmm, this is a little too off the topic.

\end{document}