\documentclass[12pt]{article}[titlepage]
\newcommand{\say}[1]{``#1''}
\newcommand{\nsay}[1]{`#1'}
\usepackage{endnotes}
\newcommand{\1}{\={a}}
\newcommand{\2}{\={e}}
\newcommand{\3}{\={\i}}
\newcommand{\4}{\=o}
\newcommand{\5}{\=u}
\newcommand{\6}{\={A}}
\newcommand{\B}{\backslash{}}
\renewcommand{\,}{\textsuperscript{,}}
\usepackage{setspace}
\usepackage{tipa}
\usepackage{hyperref}
\begin{document}
\doublespacing
\section{\href{crochet-teaching.html}{On Teaching Crochet}}
First Published: 2023 December 13


\section{Draft 1}
First, I want to apologize for the url for this posting.\footnote{I'm realizing that I might be the only one who cares about what URL I use, but that's not going to stop me here}
As much as I do think that it is important for the urls I use to be meaningful, I have more and more begun to realize that I am not yet good at using command line, and so can only find blog posts by the beginning of their names.
Since this post is about teaching crochet, it could easily be teaching-crochet.
However, since I am pretty sure that I'm going to be reflecting more on the crochet than the teaching, I think that it's worthwhile for me to have the title focus on crochet, rather than teaching.\footnote{and yes, I do realize that this could all be avoided if I learned how to use either sed or grep or most likely both, but I don't know if I want to learn magic just yet.}

Anyways, today I helped teach some friends and colleagues how to crochet.
I hadn't realized quite how ingrained the muscle memory was for me, but I found that I kept needing to go far slower than I expected to explain a concept.
Understanding the way that the yarn moves and needs to move to make crochet work is something that becomes intuitive, but certainly does not begin as such.

We were trying to make granny squares, which may not have been the best idea, in retrospect.
The first fifteen or so stitches in a granny square look really ugly even if doing everything correctly, and it can be difficult to understand where mistakes arose when they inevitably do.
I got to practice a little bit of classroom management, though, which was really nice.

Daily Reflection:
\begin{itemize}
\item Hobbies:
\begin{itemize}
\item Did I embroider today? Shoot!
\item Did I play guitar today? A little, mostly just some chords to see how much I like E minor.
\item Did I practice touch typing today? Took the day off.
\end{itemize}
\item Reading
\begin{itemize}
\item Have I made progress on my Currently Reading Shelf? I'm about halfway through the audiobook.
\item Did I read the book on craft? Shoot! I'm really falling behind here.
\item Have I read the library books? More ok with falling behind here.
\end{itemize}
\item Writing
\begin{itemize}
\item Did I write a sonnet? Yesterday's sonnet was a lot, so hoping today's is less so.
\item Did I revise a sonnet? Nope!
\item Did I blog? Technically.
\item Did I write ahead on Jeb? Plotted the chapter, for all that I didn't start it.
\item Letter to friends? Nope!
\item Paper? I tried out a few different Latin squares.
\end{itemize}
\item Wellness
\begin{itemize}
\item How well did I pray? Better?
\item Did I clean my space? Markedly, but not much.
\item Did I spend my time well? Pretty well! It's hard to do a day when everything feels like it's overlapping.
\item Did I stretch? Nope.
\item Did I exercise? Drat!
\item Water? Started drinking more again, which is nice.
\end{itemize}
\end{itemize}\end{document}