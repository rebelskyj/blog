\documentclass[12pt]{article}[titlepage]
\newcommand{\say}[1]{``#1''}
\newcommand{\nsay}[1]{`#1'}
\usepackage{endnotes}
\newcommand{\B}{\backslash{}}
\renewcommand{\,}{\textsuperscript{,}}
\usepackage{setspace}
\usepackage{tipa}
\usepackage{hyperref}
\begin{document}
\doublespacing
\section{\href{dungeons-dragons-rex-0.html}{DnD Character Backstory}}
First Published: 2023 January 10

Prereading note: I'm making a backstory for a character I'm playing in a campaign.
He's a Star Druid Firbolg.
\section{Draft 1}
I have been asked to write down some of my story, though I do not quite know why.
My companions call me Rex Sylvae.
I am ever glad that we have avoided going near to the true lands of the Fey, as I have grown attached to my head and this name.

Ah, but I should explain.
In any anthropology of Firbolgs, there is bound to be a mention of our names, or lack thereof.
The claim is that Firbolgs do not use names themselves, only taking on use-names when interacting with other races.
This is not wholly true.

Firbolgs are not Fey.
However, we remember when we were.
Other races claim that a name represents who they are.
This is dangerous to one who interacts with the Fey.

After all, it is far easier to bind a name than a soul.
By saying that the two are the same, however, binding the soul merely requires binding a name.
And so, Firbolgs recognize names as what they are: a means of communication.

It is difficult to express this in the common tongue, as \say{being} encompasses both attributes and the whole.
I will do my best, however.
I would not say that I am Rex Sylvae, anymore than I would say that I am a wanderer.
I wander, yes, and I am called Rex Sylvae, yes, but I am more than either of those.

Of course, that does not explain why I would prefer to avoid the Fey.
Fey, knowing the mortal temptation to bind oneself to a name, are fond of asking who people are.
All Firbolgs answer in the same way when asked, as was taught by our own ancestors long ago.

You respond \say{I am a king of the forest,} as Firbolgs are above the Beasts of the forest.
Of course, Old Sylvan lacks articles, so that utterance is the same as \say{I am King of Forest,} the title of their One King.
To challenge that assertion would be madness, and the Fey who can tell truth from lies would hear it as true.

But, the Fey are long gone from much of this world.
And so, I, young and newly on my own, wandered by a race I had never seen before.
One asked me in the Common Tongue who I was, and so I responded as though this was some new form of Fey.
Of course, it later turned out that the asker was an Orc.

Now that we've covered the way I am called, I should move to more pressing matters.
While I would not say that I am Rex Sylvae, I would call myself a Druid of the Stars.
The Grove of Stars, though monolithic in name, is far from it.
Each of us is bound by two commandments: we gather together when the cosmos align, and we follow our stars.

Of course, with directions as vague as those, there is much dissent among the followers.
At one end, there are those that slavishly plot their star's course, ensuring that they remain beneath it always.
At the other are those who believe that, so long as they know where the stars should be, they are following their commandment.

Obviously, both of these are ridiculous positions, but they serve to illustrate the idea.
We are all bound to a number of stars, however weakly, and so there is no way to truly remain beneath every star which is yours.
We also know that stars unobserved can fade away, which is why many take to their nightly Litanies.
The question of whether a star you cannot see can still hear your Litany is not a settled one.

My family lay near the middle.
While we tried to stay near our birth star, we recognize that circumstances may require divergences.
As I grow older, I more and more believe the commandment to be more of a spiritual than physical demand.
That is, rather than tracking the placement of the stars as projected onto the land, we must act in nature with ourselves.

As I planned to present this idea at the next Alignment, I began to Dream.
At first, I wrote these dreams off as the result of a spoiled meal or some other, natural cause.
As they kept happening, though, I realized that they were trying to warn me.
Something cataclysmic is going to happen by the next Alignment.

Of course, Star Druids are known for our apocalyptic views.
The temple we built is made to withstand any disaster that we could imagine, warded as it is against all forces.
In my Dreams, it lies shattered.
I fear that the Bound Hand has something to do with this.

My Dreams and stars pointed me in the direction of three individuals, who I have now joined with in an adventuring party.
I hope that we will be able to avert whatever vision I keep seeing.
Until then, though, so long as we are bringing about a better world, I do not know what I can do.
\end{document}