\documentclass[12pt]{article}[titlepage]
\newcommand{\say}[1]{``#1''}
\newcommand{\nsay}[1]{`#1'}
\usepackage{endnotes}
\newcommand{\1}{\={a}}
\newcommand{\2}{\={e}}
\newcommand{\3}{\={\i}}
\newcommand{\4}{\=o}
\newcommand{\5}{\=u}
\newcommand{\6}{\={A}}
\newcommand{\B}{\backslash{}}
\renewcommand{\,}{\textsuperscript{,}}
\usepackage{setspace}
\usepackage{tipa}
\usepackage{hyperref}
\begin{document}
\doublespacing
\section{\href{reflections-on-readings-18-ordinary-c-22.html}{Reflections on Today's Gospel}}
First Published: 2022 July 31

Psalm 90:14 \say{Fill us at daybreak with your mercy, that all our days we may sing for joy.}

\section{Draft 1}
Today's readings are certainly less cheerful than the past few Sundays have been.
We begin in Ecclesiastes, where we are told that \say{All things are vanity}\footnote{Ecc. 1:2}
That statement stands in stark contrast to a lot of meditations that I use, at least.
Rather than emphasizing the fact that everything which is good comes from He who is Goodness, and therefore can orient us to the eternal, the author of Ecclesiastes instead focuses on the fact that most of our daily interactions are not directly with the divine.
As a result, they are inherently transitory and empty.

This cynicism is extended in the Gospel passage, where we are told to focus on growing rich in what matters to the Almighty.
Interestingly to me, immediately after this passage in the Gospel, Jesus tells his followers not to worry about food or clothing.
He uses the imagery of birds and flowers, who do not work yet are beautiful and fed.

This is also really clear in the second reading.
The pieces of us which are earthly lead only to ruin and sorrow.
If we exist focused on the secular, we will end up living lives of vanity.
We are instead to take on Christ and focus our eyes towards heaven.
\end{document}