\documentclass[12pt]{article}[titlepage]
\newcommand{\say}[1]{``#1''}
\newcommand{\nsay}[1]{`#1'}
\usepackage{endnotes}
\newcommand{\1}{\={a}}
\newcommand{\2}{\={e}}
\newcommand{\3}{\={\i}}
\newcommand{\4}{\=o}
\newcommand{\5}{\=u}
\newcommand{\6}{\={A}}
\newcommand{\B}{\backslash{}}
\renewcommand{\,}{\textsuperscript{,}}
\usepackage{setspace}
\usepackage{tipa}
\usepackage{hyperref}
\begin{document}
\doublespacing
\section{\href{blogging.html}{On Blogging}}
First Published: 2023 December12

\section{Draft 1}
Wow, five years ago I was posting my theatre final for my blog post.
It's really interesting to me how much my view of this blog has changed since starting it.
As Hanukkah starts to wind down,\footnote{side note, I forgot how nice it is to write by candlelight. Even though I also have a real lamp, there's something about the dynamic movement of a burning candle or candles that really makes all writing feel more special. Starting each morning with a burning candle and hand writing a letter is probably something I would do well to bring back} and the semester\footnote{allegedly, I'm somewhat removed from it right now} does as well, I think it could be fun to explicitly muse on why I'm doing these musings, and why I started initially.

As the header for this site implies, one major reason that I write my blog is because my father does.
He had his own reasons for wanting to blog, which, if I remember correctly, include wanting to leave us a legacy and wanting a space to answer questions that he frequently received.
Another reason that I initially began this blog was because I wanted my friends and family to be able to keep up with my exploits and adventures while I studied abroad.
I think that the only other reason that matters is that I was in a class on diaries which assigned us to keep a diary.
I argued\footnote{apparently somewhat successfully, to the point that I apparently got cited in an article the professor wrote about diary} that blogging is a modern form of diary, one which comes with its own benefits and drawbacks.\footnote{Let's see if I can't find it. Aha! It was \href{analog-vs-digital.html}{my fifth post}. Hmm I think that I also posted the paper somewhere. Let's see if we can't find that too. \href{digital-diaries.html}{Aha!} there it is. I love that I posted these things publicly, even if I would never dream of doing so now. Then again, I do know that a large number of academics have taken to posting their papers on their sites, so maybe it's going to come back for me. Where was I?}

Of course, I, like every writer under a commitment, then struggled to find something to write about literally every day.\footnote{also to find somewhere new to take a photo, but that's only somewhat tangentially related.}
One way that I got around this\footnote{a dear friend and reader pointed out that I use the phrase \say{for all that} very regularly in my writing. I don't mind that fact, but it's probably healthy for me to start writing outside of what I'm perfectly comfortable with. Or, rather, (wow, look at that), I think that in my month of working on form, it's probably in my best interest to interrogate the habits I have when writing (somehow distinct from writing habits)} was by posting my different writing assignments to the blog.
In some regards, I've kept up with that, however slightly.
These days, I tend to use the platform to iterate on ideas, but don't actually put the text of my assignments or other writing into the blog.

Why?

One reason is that I don't know if I want this blog to be widely attached to my name.
I'm going to keep it up indefinitely, but that doesn't mean that I want it to be easily accessed.
Right now, it's actively not SEO\footnote{i hate that people call it seo optimized, because that's search engine optimization optimized, and that feels redundant. Then again, I suppose it's not technically wrong, so I guess I'll put the optimized in the real text} optimized, and there's no direct link from the main site,\footnote{rebelsky.com, a domain that doesn't point anywhere} or even my subsection\footnote{\href{j.rebelsky.com}{which right now} just links to music (huh I didn't realize that I'd made most of the scores I've written available there. Might be worth using to point people to when they want to perform my music (read: when I've convinced them to perform it))} of the site.
It is linked on my GitHub, which a colleague pointed out.
Still, I don't think that's too much of an issue.

However, the more that direct text is copied between the blog and other sources, the more that they can be tied together.
My writing tumblr, for instance, is nominally anonymous.
I'd really rather all of my writing stay, at least in theory, separated from my personal self.

Other than that, though, as I keep doing funded research, I am more and more concerned about accidentally leaking private research information or causing our research to get scooped.
That's probably a baseless concern, but it's easier to stay in a habit of not leaking data than to have to develop it after graduate school.

So, what's the point of the musings these days?

In part, it's about holding myself accountable.
The daily reflections I post force me to, if only for a second, go through my mental checklist of the day and see how well I did.
More than that, though, the daily reflections remind me that I need to do things every day.
The number of times that I've been getting ready to go to sleep, read something like \say{did I drink water today?} and then drunk water\footnote{I did refill my water bottle and drink water just now} is almost embarrassingly high.

One portion of the original inspiration that still remains is that I do still want to be more like my father in more ways.
He has\footnote{had? I don't know the last time I saw it, but I'm sure that he didn't get rid of it} a sign with the quote \say{no matter how tall I grow, I still look up to my father}\footnote{or something to that effect}.
I know that he has his own personal relationship to the saying, and that I have mine.
Honestly, I'm not sure if I've ever told him that the sign has meaning to me.\footnote{dad, if you're reading this, love you, and also thanks for reading, sorry I'm so behind on yours so often}
Writing this lets me feel connected to him in a new way, which is really nice.

A major reason that has evolved from the original inspiration is keeping track of my life.
I originally made the blog so that friends and family could theoretically feel like they knew how my hours passed.
Looking back on the posts, though, I am reminded of my own memories there, especially the ones that I did not explicitly record.

Not finally but finally for my reflection tonight, I find that writing these musings makes me more reflective.
I've talked a lot about how I want to improve at craft, but, as I mentioned \href{conclusions.html}{recently,} I also struggle with expressing why I have goals or arguments.
I want to be better at the craft of writing so that I can better communicate in whatever endeavors I choose to spend my days following.
I want to be better at communicating because I do, deep down, believe that I have insights and ideas that could help the world and lead people to Christ.
Honing my craft allows me to be more effective at spreading whatever small scale message I want to spread, in addition to the larger meta goal.

For all that\footnote{If you were waiting for this to show up in the main text, you're welcome}, being reflective is an end in itself.
As my recent posts have demonstrated\footnote{to me, at least,}, my thoughts on a subject can and often do change rapidly and wildly if I take the time to verbalize\footnote{or whatever you'd call writing as a format} them.
In basically every post I've revised recently, the later drafts are not only more polished, but they argue something different, sometimes even contradictory to the earlier drafts.
I find that my ability to sort through my thoughts is improving.
Even if it were not, though, the fact that I am thinking more deeply is a goal in and of itself.

So, while I don't think that I'm going to get back into the habit of posting drafts of my class assignments to this blog\footnote{not that I really get assignments like that anymore. The only officially prescribed writing I have left for my degree is the thesis and any papers I publish. I know that I'll need to make handouts for differnet classes that I'm in, but that's its own can of worms and we'll see how I feel about doing more or less work once the term begins}, I do still intend to keep writing these posts.
What has this musing done to help with my goals of: being like my father, being more reflective, keeping track of my time, and keeping myself accountable?\footnote{wow, the meta reflection, look at that}
I don't really know.
Today was an uneventfully eventful day\footnote{in that it could have been a day that fundamentally altered the trajectory of my life in a massive way visible in the moment (not that every day couldn't, but in that this was a scheduled event) (wow vagueblogging is hard), but did not end up doing so}, and I\footnote{unlike my father, in some regards, at least} don't want to make my private life quite this public yet, and didn't touch on any of the events that happened.
I thought again about why I'm keeping up this blog, which is something that I do at least monthly, and so might not have needed to spend an entire post on.

Oh, I did just think of another reason that I'm writing this blog still.
As I've mentioned a few times, there's a site I use to motivate my writing these days.
It encourages putting out a large quantity of words, and a daily blog post is a way for me to generate more words.

Sorry, aside aside\footnote{wow that is a fun and technically grammatical construction}, I do think that reflecting explicitly is still good, especially since I think that this might be the first time that I actually explicitly stated why I want to get better at writing.
And, of course, there's going to be the daily reflection at the end, which will hopefully help with the final portions.

Daily Reflection:
\begin{itemize}
\item Hobbies:
\begin{itemize}
\item Did I embroider today? Something I embroidered was received.
\item Did I play guitar today? Still in its case from last night, will do so after I post this.
\item Did I practice touch typing today? I made it through c, if only for a lesson. H did not go well enough, and I was once more forced back last night. I do find that I'm doing better and better at using my middle finger for c, for all that\footnote{wow look at that, this one came out by accident} I'm still not quite at perfect finger use for each letter. I'm not onto G, which is fun
\end{itemize}
\item Reading
\begin{itemize}
\item Have I made progress on my Currently Reading Shelf? I officially started one of the audiobooks, and it really is making it a lot easier to read. I don't know why I have a mental block against reading physically right now, but it's better to find a workaround than to suffer.
\item Did I read the book on craft? I brought it with me during my far too long drive today, expecting to have to wait outside my car.
I didn't have to, and so didn't end up reading it.
\item Have I read the library books? Nope!
\end{itemize}
\item Writing
\begin{itemize}
\item Did I write a sonnet? Ope.
\item Did I revise a sonnet? Nope!
\item Did I blog? Kinda! Mused at least.
\item Did I write ahead on Jeb? Wrote a whole and fun chapter, and even did some plotting beforehand to see if it makes writing easier\footnote{it really did, and I might have to continue doing this}
\item Letter to friends? Nope!
\item Paper? I spent some time debating different search algorithms that I could use, along with pros and cons of each.\footnote{random versus grid is always such a tough call}
\end{itemize}
\item Wellness
\begin{itemize}
\item How well did I pray? Not good.
\item Did I clean my space? Infinitesimally.
\item Did I spend my time well? Eh, I think that I could have spent it better, but I certainly spent it better than I have been recently.
\item Did I stretch? Still no.
\item Did I exercise? Shoot!
\item Water? Completely spaced drinking water for much of today. Remembered to drink a fair amount before the open mic though.
\end{itemize}
\end{itemize}\end{document}