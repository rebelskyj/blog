\documentclass[12pt]{article}[titlepage]
\newcommand{\say}[1]{``#1''}
\newcommand{\nsay}[1]{`#1'}
\usepackage{endnotes}
\newcommand{\1}{\={a}}
\newcommand{\2}{\={e}}
\newcommand{\3}{\={\i}}
\newcommand{\4}{\=o}
\newcommand{\5}{\=u}
\newcommand{\6}{\={A}}
\newcommand{\B}{\backslash{}}
\renewcommand{\,}{\textsuperscript{,}}
\usepackage{setspace}
\usepackage{tipa}
\usepackage{hyperref}
\begin{document}
\doublespacing
\section{\href{ballad-rules.html}{Ballad Rules}}
First Published: 2019 February 01
\section{Draft 1}
As I mentioned \href{reflection-january-19.html}{yesterday}, this month I'm trying to write a limerick a day.
Now, this means that I actually need to know how to write them.
The rules of a limerick are \href{https://en.wikipedia.org/wiki/Limerick_(poetry)}{apparently} really flexible, which is sad.
Maybe I'll choose a different type of poem.

I think I'll instead do the ballad form.
That is ABCB, where the A and C lines have 8 syllables, and the B's have 6 syllables.
Since it's a folk tradition, the rules are pretty vague from there.
So, since for the daily sonnet I wrote 14 lines, I should probably try three stanzas of 4 lines, so that I have a smaller number of lines to write.

Maybe next month I'll try a \href{http://www.poetrydances.com/nonet.php}{nonet}.\footnote{Other options include \href{http://www.shadowpoetry.com/resources/wip/rondel.html}{this},\href{http://bensonofjohn.co.uk/poetry/formssearch.php?searchbox=Roundel}{this},\href{https://www.poetryfoundation.org/poetrymagazine/poems/32637/triolet}{the time that this may be}, or \href{https://www.poetryfoundation.org/learn/glossary-terms/villanelle}{this}}
\end{document}