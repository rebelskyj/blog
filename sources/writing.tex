\documentclass[12pt]{article}[titlepage]
\newcommand{\say}[1]{``#1''}
\newcommand{\nsay}[1]{`#1'}
\usepackage{endnotes}
\newcommand{\1}{\={a}}
\newcommand{\2}{\={e}}
\newcommand{\3}{\={\i}}
\newcommand{\4}{\=o}
\newcommand{\5}{\=u}
\newcommand{\6}{\={A}}
\newcommand{\B}{\backslash{}}
\renewcommand{\,}{\textsuperscript{,}}
\usepackage{setspace}
\usepackage{tipa}
\usepackage{hyperref}
\begin{document}
\doublespacing
\section{\href{writing.html}{On Writing}}
First Published: 2022 February 8

Prereading note:\footnote{not included in word counts (neither is this)} this is really rambly, but helped me organize my thoughts. I apologize to my readers.

\section{Draft 1}
Yesterday, inspired by my impending\footnote{if self imposed and slightly ignored} deadline of sleep, my musing on the topic of reading was markedly shorter than a normal post from me.
Today, since I'm writing in a bit of self-created downtime during my day\footnote{also far earlier than normal writing hours}, I have a bit more time to consider this topic.

Lately I've rediscovered my desire to become a writer.
The urge has ebbed and flowed over the years, but definitely flows\footnote{I guess since ebb means go out flow must mean increase} more and more as time goes on.
It's also something very difficult for me to get into.

For one, I think that I mostly want to write fantasy.
Unfortunately, fantasy comes with a wide array of issues.
In part, the market is saturated with stories by people who look and live like me\footnote{within error bars of course}, and my voice my not have anything special to say.
In part, it seem so intimidating to get into, because where would I even begin.
On that same vein, where would I stop?
Editing and revising has always been a bane of my existence.

As I've noted before, I often will go through double digits of drafts for big papers and essays.
If I'm writing a 500 page book\footnote{since that's a nice pretty number}, going over it 10 times means that I have just read the equivalent of 5000 pages, many of which are going to be bad, at least in my own view.
I guess I could try to edit less, which is likely what I would have to do.

I also feel like I set bad self goals.
It's really disheartening to not meet a self-imposed goal, but I also need my goals to be something that requires effort, or there's no real gain to meeting them.
I learned recently that Terry Pratchett\footnote{GNU Terry Pratchett} allegedly wrote only 400 words a day, which is less than my postings here try to be.
On the other hand, I also remember reading that he would often fully rewrite novels once the first draft was finished, which I can't imagine counted towards that number.
On the opposite end of the spectrum, there are web-novel authors I read who put out 2000 words a day 5 days a week.
From this, it seems like a goal of 2800 good words or 10000 decent words a week would be what I would need to be a full time author.\footnote{I guess I don't know for certain if the web authors are full time, but I'm going to assume that they are}
If I want to keep this blog going\footnote{and I do}, then those words would need to be in addition to the writing I already do here, since I don't want to turn this blog into a web serial.

I've been reading recently about how to live a more productive life, and how to read more\footnote{two different topics}. 
Both point out the same issue: everyone is currently using all 24 hours of their day, so in order to add something, something needs to be removed.
I definitely have habits I would be ok with removing, which mostly pertain to the mindless scrolling I do on the internet, but the issue is that I also have lots of awkward time gaps where I use that hobby.
For example, if I have three hours at work where I have like a minute off every five, that translates to 18 minutes I could do something else, which is more time than I have currently spent writing these 600ish words.\footnote{hey nice I could stop my blog for the day here and be good}
But, there's also the startup and cooldown time I have in every activity.
Mindlessly browsing is just a few seconds on either direction, but in general writing has much more of each for me.
Maybe if I just deleted my mindless scrolling apps I would find more time in my day, especially if I tried to make writing more accessible.

Conversely, since writing fantasy might hit the same primal feeling as reading it does, maybe some of my reading time could become writing time.
The issue there is that I read in non-planned or scheduled times, so it doesn't help me to build a new habit.
Especially since I'm already trying to rebuild the habits of writing poetry and having a healthy sleep schedule, trying to make a healthy writing habit appear out of nowhere seems a little optimistic.
But, who knows, I've just deleted one of my time consuming apps, and we'll see if I can write more in the future.

708 80
\end{document}