\documentclass[12pt]{article}[titlepage]
\newcommand{\say}[1]{``#1''}
\newcommand{\nsay}[1]{`#1'}
\usepackage{endnotes}
\newcommand{\1}{\={a}}
\newcommand{\2}{\={e}}
\newcommand{\3}{\={\i}}
\newcommand{\4}{\=o}
\newcommand{\5}{\=u}
\newcommand{\6}{\={A}}
\newcommand{\B}{\backslash{}}
\renewcommand{\,}{\textsuperscript{,}}
\usepackage{setspace}
\usepackage{tipa}
\usepackage{hyperref}
\begin{document}
\doublespacing
\section{\href{defining-diary.html}{Defining Diary}}
First Published 2018 October 23
\section{Draft 6: 23 October 2018}
Consider the word diary.
What does that concept encompass?
Is a series of thoughts about life, such as Marcus Aurelius' \textit{Meditations} a diary?
Or, would an autobiography be considered a diary?
Through a series of dichotomies, the form of a diary is more easily shaped. 
This tree will be explored only in the route that leads to a diary.
Other limbs are pruned in the interest of time.

The first division in this tree is that of reality.
Writing either has a goal of portraying reality, which is writing \textit{realis}, or does not, writing \textit{irrealis}.
However, addressing reality can be a goal of a work even if it isn't the the sole, or even primary goal.
As long as a goal of the work is the representation of reality, the work falls into the realm of writing \textit{realis}.
As diaries nominally take the form of a chronicle of life, they are concerned with reality.

Within writing \textit{realis}, another dichotomy can be drawn.
This new dichotomy is the break between writings that are concerned about a single person, writings \textit{solus}, and writings that are not, writings \textit{multis}.

In the interest of fairness, there is a reasonable rebuttal to this dichotomy.
Many works are not entirely about a single person, but contain sections that are.
Should these writings, therefore, lie somewhere between a writing \textit{solus} and writing \textit{multis}?
In this example, the key to understanding the dichotomy lies in the scope of observation.
While the work at a whole would be a writing \textit{multis}, the sections concerned with a single person are writings \textit{solus}.

This idea of the classification of works changing based on the scope they are viewed through is important piece in defining a diary.
Certain sections of a diary may place focus off of the author, and therefore belong as writings \textit{multis}, but the diary as a whole has its focus on the author, and so belongs as a writing \textit{solus}.

Again we cleave this twice divided category.
Writings \textit{solus} are either presented as written by their subject, writings \textit{sui}, or presented as written by another, writings \textit{alius}.
The important distinction to make here is the word \say{presented.}
Although a \say{blurb} (short pre-informative piece explaining a person's expert status before a presentation) may be written by its subject, it is still phrased in the third person as a literary convention.
As a diary is generally seen to be a record of self, it belongs in writings \textit{sui}.
Here, we see \textit{Meditations} fall off the branch of diary, as it is written in the second person as a form of address. 

All writings \textit{sui} either attempt to convey the passage of time, as writings \textit{tempus} do, or do not, as writings \textit{atempus} do.
Although a given entry in a diary may not convey the passage of time, and therefore be a writing \textit{atempus}, the chronicling of dates conveys the passage of the subject's life, making the diary a writing \textit{tempus}.

The final division to define diary is that of chronology.
Writings \textit{tempus} can be written after the bulk of their narrative has occurred, in writings \textit{praeter}, or written as the narrative progresses, writings \textit{iam}.
That is, autobiographies are written as a retrospective account of life, as if weaving the threads of an  author's life into a coherent tapestry.
Diaries, as a contrast, may attempt to weave this tapestry, but as they chronicle events as they occur, the tapestry woven lacks a pattern.
So here again, although each entry in a diary may be a writing \textit{praeter}, as it presents the threads of the day after they've occurred, in a neatly bundled piece of time, each entry is written before the tapestry as a whole can be seen, making it writing \textit{iam}.

By exploring the definition of the literary form of \say{diary,} the value of dichotomizing writing becomes apparent.
Through finding the definition of a form through defining what it is not, it becomes easy to decide whether a work belongs in that form.
That is, to see if a writing is a diary, one must simply ask whether it meets the requirements above.
If the work focuses on telling a contemporaneous account of the author's passage through time, it is a diary.
 
\section{Draft 5: 23 October 2018}
Consider the word diary.
What does that concept encompass?
Is a series of thoughts about life, like in Marcus Aurelius' \textit{Meditations} a diary?
Or, is every autobiography a diary?
Through a series of dichotomies, the form of a diary is more easily shaped. 
This binary-branching tree will be explored only in the route that leads to a diary.
The other limbs are pruned from this discussion.

The first division in this tree is that of reality.
Writing either has a goal of portraying reality, which is writing \textit{realis}, or does not, writing \textit{irrealis}.
Of course, just because addressing reality is a goal of a work doesn't mean it's the sole, or even primary goal of the work.
As long as the goal of addressing reality is a goal, it falls into the realm of writing \textit{realis}.

Diaries nominally take the form of a chronicle of life, and are thus concerned with reality.
Within the division of writing \textit{realis}, another dichotomy can be drawn.
This new dichotomy is the break between writings concerned about a single person, writing \textit{solus}, and the writing that is not, writing \textit{multis}.

Now, a rebuttal as to the validity of this dichotomy may be posed.
Most works are not entirely about a single person, but do contain sections that are.
In this example, the key to the dichotomy lies in scope.
While the work at a whole is a work \textit{multis}, the sections concerned with a single person are writing \textit{solus}.

Now, this idea of works changing based on scope is an important piece in defining a diary.
Although certain sections of a diary may place focus off of the author, and therefore belong as writings \textit{multis}, the diary as a whole has its focus on the author.

Again we cleave this twice divided category.
All writings \textit{solus} are either presented as written by their subject, writings \textit{sui}, or presented as written by another, writings \textit{aliud}.
The important distinction to make here is the word \say{presented.}
Although a \say{blurb} (short pre-informative piece explaining a person's expert status before the aforementioned presents) may be written by its subject, it is still phrased in the third person as a literary convention.
As a diary is generally seen to be a record of self, it belongs in writings \textit{sui}
Here, we see \textit{Meditations} fall off the branch of diary, as it is written in the second person, as a form of address. 

All writings \textit{sui} either attempt to convey the passage of time, as writings \textit{tempus} do, or do not, as writings \textit{atempus} do.
Although a given entry in a diary may not convey the passage of time, and therefore be a writing \textit{atempus}, the form as a whole conveys a passage through the life of the subject, making it a writing \textit{tempus}.

The final, and most difficult to convey division is that of chronology.
Writings \textit{tempus} can be written after the bulk of their story has occurred, writings \textit{praeter}, or written as the story takes place, writings \textit{iam}.
To explain, autobiographies are written as an account of life, trying to weave the threads that had occurred in the author's life into a coherent tapestry.
Diaries, as a contrast, attempt to weave a tapestry without a pattern, chronicling the events as they occur.
Although each of the entries may be writing \textit{praeter}, since the writing is not attempting to outline the whole of the author's history into a coherent thread, rather simply chronicling the events as they occur, the diary as a whole remains writing \textit{iam}.

By exploring the definition of the literary form of \say{diary,} the value of dichotomizing writing becomes apparent.
Although a dreaded prospect by many, the idea that genres can be grouped into mutually exclusive categories makes defining a given work as belonging to that genre easier.
So, to see if a writing is a diary, one must simply ask whether it meets the requirements above.
If the work focuses on telling a contemporaneous account of the author's passage through time, it is a diary. 
 
\section{Draft 4: 23 October 2018}
Of all of the writing that is, was, or will be, only two kinds exist.
All writing is either non-fictive (concerning itself with reality) or fictive, which does not.
Now, to address the inevitable rebuttal of writing that concerns itself somewhat with reality, imagine a lamp.
A lamp is either lit or not.
Even though it may not burn its brightest, it still burns, and is lit, or is completely extinguished.

Diaries nominally take the form of a \textit{curriculam vitae}, and are thus concerned with reality.
Within the division of non-fictive writing, another dichotomy can be drawn, that which is biographical (concerned with a single person's life), and that which is not.
Again, a rebuttal as to the nature of this dichotomy may be posed.
There are writings that contain a series of biographies.
Each of the biographies within the whole are biographical, while the sum work is not.
In such a way, a writing can be biographical or not, depending on the scope with which it is viewed.

Oddly, the question of scale makes the defense of a diary as biographical more tenable.
Certain entries in a diary may focus on others, but the diary as a whole still focuses on the author.

Again we cleave this twice divided category.
All biographical writings either find themselves as narrative (expressing the life through time), or not (expressing life atemporally).
Many non-narrative biographies take the form of a \say{blurb,} or short pre-informative piece explaining a person's expert status before the person presents.
At the peak of the genre, the blurb expresses no notion of time, only of accolades.
Again, although any given entry in a diary may be non-narrative, the diary as a whole conveys the life of an author through time.

Up to now, the divisions of writing into a tree have not followed typical genre listings.
This next division, however, is a common one.
Narrative biographies are either autobiographical or not.
That is, narrative biographies are either written by the subject or not.

Here, the objection may be raised that the division between self and not should be placed above that of narrative, since blurbs are often written by their subject.
However, since blurbs are written in the third person, tend to convey only publicly available knowledge, and portray the subject as a platonic ideal, good blurbs look the same, regardless of authorship.

Now, one of the less controversial pieces of typical diary writing is that a diary is written by a self about a self.
This lends it nicely into the autobiographical category, and sets up for the final binary division of writing needed to specify the diary.

All autobiographies are either retrospective, concerned with chronicling the events before them, or not, and are concerned with recording the events as they happen.
Here, we again confront the issue of scale.
While each entry of a diary is written retrospectively, the diary as a whole is not.
That is, although entries are written after the events, the aftereffects of the events recorded occurs after the entry is scribed.

With these divisions in mind, a a diary is a non-retrospective, autobiographical, narrative, biographical, non-fictive piece of writing.
Now, the question may be raised as to the purpose of defining a diary, and what goals were meant to be achieved.
As today's political climate shows, without consistent definitions, no debate can be fruitful.
In giving a specific definition for diary, a more fruitful debate as to the merits of diary as a form can be explored.

\section{Draft 3: 23 October 2018}
If we look at all writing that is, was, or will be, there are only two kinds of writing.
There's non-fictive writing, writing that concerns itself with reality, and fictive writing, which does not.
As a metaphor, think of a lamp.
A lamp is either lit or not.
If it is slightly lit, it is still producing light.

Diaries nominally take the form of a \textit{curriculam vitae}, and are thus concerned with reality.
All writing concerned with reality is either biographical (concerned with a single person's life), or not.
Now, here a potential division can occur.
What about, for instance, a series of biographies?
To me, each of the biographies should be seen as their own work.
In such a way, a writing can be biographical or not, depending on the scope with which it is viewed.

But, since diaries are a single person's record of their own life,\footnote{at least in common definitions} it seems fair to say that diaries are still biographical.
Now, all biographical writing is either narrative or not.
A narrative biography is concerned with expressing a life through time.

Many non-narrative biographies take the form of a \say{blurb,} or short pre-informative piece explaining a person's expert status before the person presents.
While each entry in a diary may be non-narrative, the diary as a whole conveys information through time.

Now, the divisions so far have been fairly novel.\footnote{at least as far as I could see}
All narrative biographical non-fictive writings are either autobiographical (focusing on the life of the author), or not.
Now, some might argue that a blurb can also be autobiographical or not.
However, since blurbs are written in the third person, and tend to convey only publicly available knowledge, a good \say{autobiographical} blurb should be indistinguishable from a good non-autobiographical blurb.

So, since a diary, as mentioned above, is a writing by a self about a self, it is autobiographical.
Finally, all biographies are either retrospective, where the idea is to record events after they've happened, or not.
Now, again we run into the problem of scale.
While each entry of a diary is written retrospectively, the diary as a whole is not.
That is, although entries are written after the events, more events take place and are recorded after the writing.
In general, the diary takes place contemporaneously.

And so, a diary is a non-retrospective, autobiographical, narrative, biographical, non-fictive piece of writing.
In the interest of length, the rest of the tree remains unfilled, though it is fillable.

\section{Draft 2: 19 October 2018}
Diaries, like many aspects of stereotypical female adolescence, are a nebulous concept.\footnote{what differentiates a horse from a pegasus from a unicorn? If a unicorn has wings grafted on, does it stay one?}
But, like any scientist, I hate undefined concepts.\footnote{well, hate's the wrong word, maybe \say{am uncomfortable with}}
So, I've been thinking about what a diary really is.

I tried defining it a few different ways.
First was through exclusion.
In \textit{A Brief History of Diaries,} \textit{Meditations} by Marcus Aurelius is considered a diary, as is a shopkeeper's logbook.
Both of those didn't seem to fit what I would consider a diary, but I couldn't figure out why.

Then came the idea of\footnote{like a good little linguistics child} splitting all types of writing into binary divisions.
It took a few tries, but I think I've gotten something that works.
Of course, writers hate dichotomies,\footnote{not that they're special, just that the two English people I said the idea to were immediately hostile, and I've never tried grouping music into categories, although the pushback to the Hornbostel--Sachs doesn't seem to be very major to me (although I'm also almost 60 years late to that fight [wow the 1960's are more than 50 years away] so what do I know)} so I needed to make sure that each category was a real dichotomy.

If we look at all writing that is, was, or will be,\footnote{as far as I can guess} there are only two kinds of writing.\footnote{ooh maybe this would make an opening line}
There is writing concerned with reality,\footnote{non-fictive writing} and writing that isn't concerned with reality.\footnote{fictive writing}
Of course, an immediate push back to this would be \say{what about writing that is partially concerned with reality?}
That's non-fictive.
It is concerned with reality.
It's like saying \say{a room either has some form of lighting, or it doesn't.}
Dimmer lighting, or lighting only in one corner still satisfies.

Obviously,\footnote{wow, the more I write the more I realize how dangerous that word is. Just because I'm finding something obvious as I write about it doesn't mean that everyone ever will} diaries have a focus on reality, so they are non-fictive.
Next, all non-fictive writing is either focused on a single entity, with the rest of creation\footnote{another word I'm realizing may be taken differently than intended} serving as a backdrop to it, or not doing that.

It may be helpful to give examples of writing in each category.
In the biographical category, there are diaries and biographies.
In the non-biographical category, we see things like epic poetry.\footnote{ok so this one may be a bad kind of dichotomy, because it seems completely arbitrary. I can't think of anything that I couldn't argue isn't concerned with a single entity unless I don't allow breaking, which doesn't work later. I also got into the whole internal dialogue of \say{if you believe that God is completely incomprehensible, and that we only see aspects, and every religion sees its own aspect, why do you believe you're in the \nsay{One, Holy ... Church}?}}

So if I ignore that, all writing concerned with reality is either narrative or not.

BREAK

Narrative writing is writing concerned with telling a story.
Think about a typical biography.
Conversely, there are non-narrative writings, such as instruction manuals.

Within narrative writing, there is autobiographical and not.
Autobiographical writing is writing concerning the author.
So, a member of an ethnic group writing an ethnography of the group, or authors writing about themselves.

Finally, all autobiographical writing is w

To me, a diary\footnote{in order from most to least specific} is a non-retrospective, autobiographical, narrative, biographical, non-fictive piece of writing.
In order, I'll explain each of these pieces, from least to most specific.

As a person who's taken Linguistics, I understand the importance of grouping things into dichotomies.
As an artist, I understand the frustration of false dichotomies.
So, I've attempted to group all writings into mutually exclusive categories.
First is fictive and null-fictive writing.

Confusingly, my definitions are based the opposite way here.
Null-fictive writing is writing concerned with portraying reality, while fictive writing is not.

All writing concerned with reality either has a singular entity at its focus: biographical, or not, null.

All biographical writing is either narrative: biographies and whatnot, or null: blurbs.

All biographies are either autobiographical: about self, or not: all other biographies.

Finally, all biographies are either retrospective: all the action takes place before writing, or not: diaries.

\section{Draft 1: 17 October 2018}
What's a diary?
To me, a diary\footnote{in order from most to least specific} is a non-retrospective, autobiographical, narrative, biographical, non-fictive piece of writing.
In order, I'll explain each of these pieces, from least to most specific.

As a person who's taken Linguistics, I understand the importance of grouping things into dichotomies.
As an artist, I understand the frustration of false dichotomies.
So, I've attempted to group all writings into mutually exclusive categories.
First is fictive and null-fictive writing.

Confusingly, my definitions are based the opposite way here.
Null-fictive writing is writing concerned with portraying reality, while fictive writing is not.

All writing concerned with reality either has a singular entity at its focus: biographical, or not, null.

All biographical writing is either narrative: biographies and whatnot, or null: blurbs.

All biographies are either autobiographical: about self, or not: all other biographies.

Finally, all biographies are either retrospective: all the action takes place before writing, or not: diaries.
\end{document}