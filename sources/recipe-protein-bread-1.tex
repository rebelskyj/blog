\documentclass[12pt]{article}[titlepage]
\newcommand{\say}[1]{``#1''}
\newcommand{\nsay}[1]{`#1'}
\usepackage{endnotes}
\newcommand{\B}{\backslash{}}
\renewcommand{\,}{\textsuperscript{,}}
\usepackage{setspace}
\usepackage{tipa}
\usepackage{hyperref}
\begin{document}
\doublespacing
\section{\href{recipe-protein-bread-1.html}{An Initial Attempt to Make Macro Bread}}

First Published: 21 January 2025

\section{Draft 1}
As I've mentioned in the last few posts, I'm trying to organize my life under a few different principles.  
Notably, I wanted to make sure that, even though the USDA seems less than clear about their need, I get enough phytonutrients of each kind.  
When I looked through my diet and what I could easily add to my diet\footnote{such as carrots, which I love to gnaw on}, I was missing aliums and purple fruit.  
With that in mind, I decided to try to make a bread that incorporated them.

At the store, dried figs and daikon\footnote{which I normally see labeled as daikon radish but} were the mixins that spoke to me most.  
Below is the recipe I vaguely followed, along with a quick back of the envelope calculation for macronutrients.

\begin{itemize}  
\item Add approximately 500 grams of flour\footnote{529} and 650 grams of water\footnote{648} to a bowl, mix until a thick paste is formed, and add in about a tablespoon of yeast.  
\item Upon being allowed to rise\footnote{about three hours, in my case}, shred approximately 250 grams of daikon\footnote{267} and squeeze to remove moisture.  
\item Finely chop 250 grams of fig, and also add to the dough\footnote{254}.  
\item Also add 100 grams of rye flour\foonote{103}, 100 grams of almond flour\footnote{103}, 25 grams of vital wheat gluten, 250 grams of greek yogurt to the mixture and begin to knead.  
I ultimately needed to add another 300 grams of flour\footnote{total of 840 grams or so} and 10 grams of gluten to get the dough to be the appropriate level of sticky, which is to say not at all. I also added salt somewhere in there, but that was eyeballed at best.  
\item I oiled the ball and let double, then divided into twelve even balls of approximately 200 grams.  
\item I forgot about them overnight and so found them to be overproofed, so reshaped, heated oven to 350 for 30 minutes as they reproofed, and baked at 350 for 30 minutes.  
\end{itemize}

How did they turn out?  
Honestly pretty well.  
I'm happy with the taste, if only because the fig comes through, but the daikon doesn't at all.  
I didn't love how sulfurous the daikon smelled while shredding or baking, though, and so might have to take steps to prevent that in the future.  
Chief among these, I think that I'll try shredding into cold water that's been doped\footnote{oop my materials chemist came out} with either salt or baking soda, since those seem to work for others.  
Other than that, I might add more of either the daikon or fig, since they're shockingly under noticed, for being a full 45 percent of the mass of the dough.\footnote{ok not entirely, because I did squeeze about half the mass of the daikon out as water. Still, that's like 400 grams, so still one part in three.}

What else went wrong?  
I don't love the texture.   
I'm not sure if that's due to the relatively low flour to other ratio, the overproofing, or something to do with using whole wheat flour.  
Regardless, I will probably try the next batch without rye, since I didn't notice much by way of flavor from it, and it definitely does a lot to add that kind of texture.  
I'll also try not to let the dough overproof, but that's far more optimistic.

Nutrition Information as Eyeballed:  
\begin{itemize}  
\item 840 g\footnote{that's what my notes said, so I'm willing to believe I missed ten grams of flour somewhere. Oh I think it might have just been spillover} of whole wheat flour. Its bag says 100 calories, 4 grams of protein, 20 grams carbs, 3 grams fiber, and 0.5 grams fat per 28 grams.\footnote{I guess I shouldn't be surprised that it's almost all one of the macros by mass}  
Somehow, that means I used exactly 30 servings of flour, for a total of 3000 calories, 120 grams of protein, 600 grams of carbs, 90 grams of fiber, 15 grams of fat.  
\item 650 grams water, which is nutritionless  
\item 267 grams \href{https://fdc.nal.usda.gov/food-details/2709803/nutrients}{daikon}, which is 16 calories, 0.68 grams of protein, 0.1 grams of fat, 3.4 grams of carbs, and 1.6 grams of fiber per 100 grams. That means a total of 43 calories, 1.8 grams of protein, .267 grams of fat, 9 grams of carbs, and 4.3 grams of fiber total.\footnote{it is at this point that I opened a spreadsheet to do the math}  
\item 254 grams of fig, and that's all I'm going to be putting in here because I don't think anyone needs me to do the math.\end{itemize}

In total:  
\begin{itemize}  
\item 5150\footnote{5136.696} calories  
\item 210\footnote{209.429} grams protein  
\item 85\footnote{84.8475} grams fat  
\item 900\footnote{900.66} grams carbs  
\item 145\footnote{144.44} grams fiber  
\end{itemize}  
which, assuming I split my rolls perfectly, comes out to about:

\begin{itemize}  
\item 430 calories  
\item 17.5 grams of protein  
\item 7 grams of fat  
\item 75 grams of carbs  
\item 12 grams of fiber  
\end{itemize}  
  
In total, about 24 calories per gram of protein, well below the 40 threshold I need to average over the day.  
It is interesting how calorically empty daikon is, and how relatively dense fig is.

  


Daily reflection:  


Didn't do great about maintaining the schedule I set, but did do more than expected, all things considered.  

\end{document}