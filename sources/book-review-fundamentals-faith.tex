\documentclass[12pt]{article}[titlepage]
\newcommand{\say}[1]{``#1''}
\newcommand{\nsay}[1]{`#1'}
\usepackage{endnotes}
\newcommand{\1}{\={a}}
\newcommand{\2}{\={e}}
\newcommand{\3}{\={\i}}
\newcommand{\4}{\=o}
\newcommand{\5}{\=u}
\newcommand{\6}{\={A}}
\newcommand{\B}{\backslash{}}
\renewcommand{\,}{\textsuperscript{,}}
\usepackage{setspace}
\usepackage{tipa}
\usepackage{hyperref}
\begin{document}
\doublespacing
\section{\href{book-review-fundamentals-faith.html}{Book Review of Fundamentals of the Faith}}
First Published: 2023 December 14
\section{Draft 1}
I've officially accomplished one of my goals for the month, and I've finished a book from my currently reading shelf.
The book is \textit{Fundamentals of the Faith} by Peter Kreeft.
It's a just under three hundred page book of apologetics about the Catholic Faith.

Now, there are many ways that I approach these book reviews.\footnote{I think. I'm not going to actually fact check that claim right now, though.}
One piece that I feel like is a fairly common aspect of them is my explaining how I came to the book.
This book, at least, has a fairly interesting story, which makes it one I want to share right now.

Since coming to graduate school, I've had the chance to meet a lot of fantastic people.
Last year, I helped teach religious education, and chatted a fair amount with one of the teachers for another age group.
This past year, she had discerned joining a religious order, and offered me a book of essays.
Of course, when someone who radiates holiness\footnote{as she did when last I saw her} offers you a book, it seems only prudent to read it.

So, of course, I waited almost six months before starting the book.
From the first page, though, I was hooked on it.
Kreeft strikes the difficult balance between incisive wit and deep broad truth.

As much as I'd meant for this musing to be a long reflection on the book, it's late enough right now that I would really rather be asleep.
A few brief thoughts before that happens:

As a faithful Catholic, he did, of course, profess extra ecclesiam nulla salus.
As a faithful Catholic, he did, of course, also point out that what we consider the Church and what the Almighty considers the Church are not always perfectly aligned.
In discussing that, he also brought up the parable of the man who keeps hiring more workers to his vineyard.

My entire life, that parable has placed cradle Catholics\footnote{like me} in the position of those hired first.
Kreeft inverts that.
We, as Catholics, do legitimately have an easier path to salvation\footnote{as he claims, at least. I think that there's something to be said about to whom much is given, much is expected, but that's a discussion for when I'm more awake} by virtue of knowing exactly what the Lord wants us to do and how He wants to be worshipped.
It was striking to consider that, in many regards, I do have an easier time finding Truth than those outside of the faith.

The other most striking aspect of the book was its claim to absolute truth.
As a scientist, one of the most fundamental aspects of the field is that no knowledge is certain.
Everything is either theory\footnote{hypothesis that has yet to be disproven by data} or law\footnote{model that works}.
Nothing is inherently true, and to do science correctly requires, on some level, a willingness to find that everyone else has always been wrong.

In the midst of a culture that is incredibly relativist right now, the fact that Kreeft was so willing to say that not only is the Church true, but it is the Truth, was really striking to me.
Also, he continually brought up the fact that the most central teaching of the Church is not being kind to our neighbors or loving G-d, or anything else like that.
The most central tenant of our faith is that Jesus Christ, True G-d and True man, was incarnate and died for our sins.\footnote{I mean that's already at least four or five things, but you get what I mean}

That was something that I've found myself coming back to as I consider the book.
For most of the modern history of the Church, it was reasonable to assume that everyone was familiar with the explicit teachings of the faith, and so we could discuss the secondary considerations.
In a time before relativism, it was valuable to explain how Christianity is like other faiths, who have their own piece of the truth.
Nowadays, as the normal belief is that belief systems are more alike than different, the opposite tack is needed.

Anyways, all this to say, I really enjoyed the book, and I feel like I should read more explicitly Catholic creative nonfiction in the next year.

Daily Reflection:
\begin{itemize}
\item Hobbies:
\begin{itemize}
\item Did I embroider today? I was ill today, so did not.
\item Did I play guitar today? Just a touch, but enough, I think.
\item Did I practice touch typing today? Took another day off.
\end{itemize}
\item Reading
\begin{itemize}
\item Have I made progress on my Currently Reading Shelf? I finished one of the books!\footnote{as evidenced by my writing a musing on it}
\item Did I read the book on craft? I did not expect to have an after choir activity, but I'm glad I did.
\item Have I read the library books? Shoot!
\end{itemize}
\item Writing
\begin{itemize}
\item Did I write a sonnet? Yesterday's sonnet wasn't great, hopefully today's is a little better.
\item Did I revise a sonnet? Still no
\item Did I blog? Less techincally
\item Did I write ahead on Jeb? Wrote the chapter for tomorrow.
\item Letter to friends? Nope
\item Paper? I thought about the best approaches moving forward.
\end{itemize}
\item Wellness
\begin{itemize}
\item How well did I pray? Bad.
\item Did I clean my space? Yeah! It's nicer now, which was the goal.
\item Did I spend my time well? Not really. I suppose that I'm recovering from illness, so some of that is to be expected. Still, I feel like I could have done better.
\item Did I stretch? I really fell off this train.
\item Did I exercise? See above.
\item Water? I drank so much water today. The nicest part of illness is how much it encourages me to drink water.
\end{itemize}
\end{itemize}\end{document}