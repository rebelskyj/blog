\documentclass[12pt]{article}  
\newcommand{\say}[1]{``#1''}  
\newcommand{\nsay}[1]{`#1'}  
\usepackage{endnotes}  
\newcommand{\B}{\backslash{}}  
\renewcommand{\,}{\textsuperscript{,}}  
\usepackage{setspace}   
\usepackage{tipa}  
\usepackage{hyperref}  
\begin{document}  
\doublespacing  
\section{\href{creating-for-myself.html}{On Creating for Myself}}  
First Published: 2025 September 15 (I'm so bad at hitting post)

\section{Draft 3: 12 September 2025}

As I was packing for my train trip, I found myself also wanting to pick up reading again.
In addition to the books of music theory and epistemology I printed, I also brought a few print books that I've wanted to read.
So far as I can remember, the books came from my mother, who gave me most of her theology collection a year or so before she died.
It hurts to realize that I was the only one of her children who was holding to the faith then, even more because I no longer hold as closely.
However, even though all that I do is fundamentally colored by both the way she raised me and her current absence, that is not what the point of that anecdote was.
Rather, it was to explain how best I think that I came into a copy of \say{The Nonviolent Alternative} by Thomas Merton.

At first, I had thought that it was a full tome with one coherent through line.
Instead, it is apparently a posthumously collected series of letters, reflections, essays, and the like by Merton.
It opens with a fairly lengthy essay by the collector, which puts Merton's work in the broader context.
Given that this book came out just a year or so after Merton died, though, there was not much room to see the long term ripples.
I don't normally spend two full paragraphs on background, but for some reason, I feel the importance of noting what brought the essay I read to me.
The first Merton writing in the book is \say{Original Child Bomb}, subtitled \say{Points for meditation to be scratched on the walls of a cave}.

In 41 verses, Merton tells the story of the atomic bombing of Hiroshima.
Unlike every other framing of the story I have seen, the science plays almost no role.
The scientists exist solely as phantom figures who occasionally oppose the bombing of a civilian populace.

In 41 verses, Merton shows the way that the modern war apparatus fundamentally precludes a just war.

My first two drafts today focused on the reasons I create.
Reading the biography of Merton at the front of the book, I was reminded of the real reason I should do everything: I am called to do so by my faith.

The struggles that I have with the Catholic Church, especially as it exists in America, are not unique to my current age.
Indeed, Zahn noted back in 1971 or 1980\footnote{I think, depends on edition} that \say{scandals abound, not least of which involve bishops who have never found it within themselves to express pubic concern over the napalming of innocent civilians but who are quick to publish fervent denunciations of the napalming of Selective Service records.}
Merton saw this same secular theology that the bishops of that age had.
Where I find this fact hard to face and find myself struggling with my face because of it, Merton remained a faithful and obedient monk.

It seems as though \say{Original Child Bomb} was just many of the writings that Merton put out over the years.
From the implications I have seen, it was not a particularly long project for him.
I wonder if it required drafting, or if by that point in his life Merton was just so filled with the Spirit and had such a well-formed voice that the lines came out at once.
I had meant to read the entire book in a sitting, but I found that I could not keep reading after the first.

The hobbies I want to do I want to do because I think that I can make the world better.

If Merton failed to do so, despite being so clearly a man of faith and being someone in conversation with leaders, what hope do I have?

On the other hand, despair is the easiest option.
I may not have his reach or his fluidity with prose, but I have my own unique view on things.
Merton's view on Just War Theology seems as though it was relatively unique, and so far as I can tell, it remains relatively unique.
He wrote \say{Original Child Bomb} near the eve of his earthly life, I am\footnote{Lord willing} near the beginning or early middle of mine.

So, what does creation do for me? 
Creation gives me a way to scream at a world gone mad, hoping to restore some sense of semblance.
In prose which hopefully lacks polemics, I aspire to my writing bringing others closer to truth, love, and beauty.
In making art, I hope to create beauty, which by its nature orients humanity towards the beautiful and true.
In creating anything at all, I reveal my spark of Divinity, which will hopefully light myself and the world aflame.

Wow that was fun.
I could do some drafting about what I want to do, but that doesn't feel quite as meaningful or necessary right now.
I'm happy with the way that these three drafts have helped me to come to an idea of not just what I want to do, but also why.
I don't know if my web novel currently does anything to be beautiful or lead others to truth and love.
Rather than stop it, however, I guess that it would be better if I was to focus on that.

So, then, I suppose that what I have taken from today's folly is that I don't create for myself, nor do I really want to.
Instead, I want to create so that I can bring the world to a better place.
The only way that I can improve my ability to create something beautiful is to practice. 
And thus, my practice will continue.

\section{Draft 2: 12 September 2025}

Why do I create?

I'm not totally sure, if I'm being totally honest.

There exist the obvious answers: the spark of the Divine within me kindling the world around, because I like all people fear my mortality and respond by trying to leave a legacy, my belief that I am better able to do things than others, the impacts I have sen that I can make.  
There are probably less obvious answers which would come to me if I thought more or deeper about the question.  
What I am nearly positive of, though, is that I do not create for myself.

Even though I do want to grow as a person and in my skills, I don't think of what I do as being for myself.  
Although that's a fun contradiction for me to explore, I don't think that's what I want to use this draft for.  
Rather, I want to explore the different forms of creation that I've either lost, fallen away from, or simply want to begin ab initio.\footnote{I don't believe in italicizing loan phrases because I think that it has semantic meaning in English, not simply as Latin}

This site is the obvious place for me to begin.  
If I was writing this for myself, I don't know what that would mean.  
Would I SEO the site to make money on the side?  
Should I be posting my most extreme views, so that everyone around me can learn about the most radical takes I have?  
Should I be curating a clean image of myself, such that no employer, regardless of how dedicated, could find something negative about me?  
Should I be making it something that helps me to think about the world?

Let's answer those one by one.

SEO is a special kind of hell.  
As I've listened to some videos\footnote{don't ask} about the platform culture, I more and more believe and see that there is something fundamentally broken about the idea of platform as service.  
That's something I can explore later, but for now simply knowing that I hate that concept is probably sufficient.

Much as I enjoy testing the limits of my freedom of speech, I don't know that this is a good idea.  
At the very least, I have seen the way that writing angry and inflamed\footnote{or inflammatory, I suppose} posts has made others in my life less happy and more angry.  
That's not really a goal of mine.  
Also, I quite enjoy life on this side of prison walls.  
Given the current political climate, I do start to worry about the things that I say coming back to cause me demonstrable harm.

At the other extreme, though, I do not want to push myself into a mold.  
I don't want to stay within the lines solely to avoid being beaten down by someone else.

Alas, the remaining paragraphs have been lost somehow.
In general, I think that I said that I want to write for a variety of reasons, and that's ok.
However, since then I've read a bit of Thomas Merton, which has been really life changing, and so now we go on to draft again.

\section{Draft 1: 12 September 2025}

I just read a beautiful piece of prose on writing from a dear friend of mine.  
In reading it and doing my own daily reflection, both digitally and on paper,\footnote{how's that for nested lists?} I found myself thinking about how hard it is to start any hobby, maybe especially one that I've dropped before.  
I spent a fair bit of time\footnote{read, less than thirty seconds} considering what I wanted to title the post, and I ended up deciding that a post about trying to shift the reason I write would be a good one.  
After all, much of the advice I see about creation says that we should not work for the adoration of crowds, but rather create for the sake of our own muse.

I then thought about how modern and Western that thought was.

I do not exist alone, and so it is ridiculous that I should create for myself alone.  
Even though I've met almost none of the readers of my web novel in person\footnote{something that I'm happy to continue being true, to be clear}, I still should think of them as complete and beautiful humans.  
Life lived in service of the other is the most fulfilling, so many theologies say.

So then, what does it mean for me to create something for myself?  
Is that something that I should even try for?

I think that generally when people say that we should create art for ourselves, it comes from a place of saying that we should not focus our muse on some hypothetical audience.  
Instead, we should create the art that we are called to.  
However, the fact that what I consider art at all, let alone what art I craft by nature is culturally constructed rarely enters the equation.  
Neither does the fact that, especially if I have an audience, there is an argument that I have some kind of duty to them.  
I of all people am always a fan of the idea that we owe so very much to each other and that we have a duty to fulfill.

So where does that lead me?

By confirmed viewership, my web novel is by and away the most engaged with art that I create.  
By positive impact to the intellectual\footnote{this word added after self introspection} world, the science communication videos I'm planning will likely be the greatest.\footnote{my dear brother requested that I start a podcast, which also seems fun}  
By self introspection, this site seems the best.  
By positive impact to the world, knitting and donating hats is probably the best I can do.

There exists in economics the idea of relative advantage.  
That is, even if organization A is better at production of everything than organization B, they may be more better at one than another.  
In that case, economics says that A should produce the thing that they're most able to outperform in and then purchase the rest.  
There are clear benefits to this approach, especially when taken at small scales.

If I have a neighbor who is great at sewing and I am great at farming, then the two of us can both have nicer clothes and food by sharing.  
Atomizing the roles we have ends up reducing the atomization of people, because I am then reliant on his sewing, and he is reliant on my farming.

When this same approach is taken to the global stage, however, there are any number of issues.

I don't like that I live in capitalism, or whatever the current system which holds me is.  
Rebellion takes a number of forms, and art is almost always among them.  
I do not think that I can generate art for the state, regardless of what the state is.  
So, then, does writing this site for myself end up doing good by virtue of speaking truth to power?

Am I speaking truth to power?  
I doubt that anyone with political strength reads this.

Am I screaming into the void?

If I scream into the void long enough, will he scream back?

I don't know the answer, and I don't know that this current draft is taking me anywhere good.  
I had also wanted to muse on the idea of what hobbies I want to get back into and why, and these words have probably been helpful in framing my mind.  
As I've been writing these past few follies, I do find myself realizing that there's a lot that I gain by simply word vomiting free association on the page.  
Second drafts find themselves hitting places I don't know that I would have otherwise found.  
Since the goal of this site is to help me explore myself, thought and idea, that's the best that it can do.

\section{Daily Reflection: 12 September 2025}

\begin{itemize}

\item Did you journal by hand today?

I did!

\item Did you do a folly?

Nope, missed about a week of them. one day was on the train, which I did.  
One day was with friends, and it was a full day.  
Two days ago was a long bus ride and then I was tired.  
Yesterday I just failed.

\item Did you in some way, shape, or form advance the web novel?

Not in the slightest.  
That's a shame,

\item Did you work on music, whether education or creation?

Nope!

\item Did you work on book binding?

Nope!

\item Did you work on another hobby?

Nope! I guess socializing, watching anime, and playing video games, but nothing that I'm happy about.

\item Did you stretch? Really?

Not really.

\item Prayer?

Nope. Well, I did do morning prayer while with friends and went to Mass with them, so I guess kind of yes.

\item Meditation?

Eh.

\item Reading?

Not really. I did finally finish rereading a web novel I enjoy.

\item Minimizing screen time?

Not really at all in the slightest.  
Today, though, I do really want to try reading the Nonviolent Option.  
Depending on whether or not there's WiFi on the train, may or may not reward myself with finishing some shows.

\end{itemize}

Current Pen List\footnote{for my own posterity, mostly}

\begin{itemize}  
\item Hongdian Black with Fude Nib: Diplomat Caribbean (8/30ish)  
\item Jinhao Shark: Diplomat Caribbean (8/30ish)  
\item Pilot Preppy: Private Reserve Electric DC Blue I think (I think since late june. I think)  
\item Sheaffer: Private Reserve Spearmint (since 7/15) (I Think)

\end{itemize}

\end{document}