\documentclass[12pt]{article}[titlepage]
\newcommand{\say}[1]{``#1''}
\newcommand{\nsay}[1]{`#1'}
\usepackage{endnotes}
\newcommand{\1}{\={a}}
\newcommand{\2}{\={e}}
\newcommand{\3}{\={\i}}
\newcommand{\4}{\=o}
\newcommand{\5}{\=u}
\newcommand{\6}{\={A}}
\newcommand{\B}{\backslash{}}
\renewcommand{\,}{\textsuperscript{,}}
\usepackage{setspace}
\usepackage{tipa}
\usepackage{hyperref}
\begin{document}
\doublespacing
\section{\href{running.html}{On Running}}
First Published: 2022 July 29
\section{Draft 1}
As I mentioned in my last \href{reflection-june-2022.html}{monthly reflection}, I wanted to be able to run a 5K this month.
I claimed to have run a 5K at reunion, which is debatably true.
I entered the race, and I was given a reward for finishing.

But, I didn't run it continuously.
Instead, I took lots of walking breaks.
Today, in contrast, I ran a full three miles continuously.\footnote{which I am aware is not a 5K, but it's close enough that I'm counting it}

As I was cooling down from the run, I thought a lot about why running a 5K straight felt so strange.
More or less, I think it comes down to the fact that\footnote{regardless of the truthfulness of it} I in many respects feel like I'm past my prime as an athlete.
I know that I'll never be able to dive as well as I did in college,\footnote{mostly because I don't have pool access} I'll likely never swim as fast as I did in high school,\footnote{which I acknowledge is an unfair comparison because I tapered and shaved for it} and I probably won't ever get my maxes up to where they were last summer.\footnote{if only because I don't love lifting}

So why does running 3 miles matter to me?
As far as I can tell, I have never run that far in my life.
All through high school, I ran at most a mile at a time.\footnote{or I snuck walks in in the middle of longer runs}
Throughout college, I had a very frequent goal of running a 5K, but I was never able to get there.
There are all sorts of excuses that I can make as to why I never did, but they fundamentally boil down to the fact that I've always thought of myself exclusively as a sprint athlete.
As any coach or modern self-help book can tell you, a fixed mindset where you believe you can't do something will generally see you being correct.

I think about a poster which hung\footnote{hanged? I can't remember which is which. Let's see: cool hanged is people, hung is literally anything else. Posters are not people (allegedly) so we're good} somewhere in high school.
It read \say{Whether you believe you can or you can't, you're probably right.}
I never saw myself as a person who could run 3 miles, and I was right.
Now I know that I can.

Now that I'm done inflating my ego, it's time to self-correct.
Despite the fact that I wasn't walking for any portion of it, my mile pace was less than 50 seconds faster per mile.
I also apparently worked far harder, because my average heart rate was more than 30 points higher.
I took an extra nine steps every minute, which is kind of funny.
The part that's wildest to me is that the peak of my heart rate during my first 5K was about the lowest that my heart rate was at during my run.

Anyways, the next goal is being ready for the \say{degree dash} that I signed up for, which apparently involves a 5.39\footnote{the average number of years it takes to complete a PhD}\footnote{hmm, one source says 5.39, one says 5.41. That's a big difference} mile run.
It's weird that despite being a bigger difference in how far I can run than I was at, it feels much more doable of a run.
Anyways, enough rambling from me.
\end{document}