\documentclass[12pt]{article}  
\newcommand{\say}[1]{``#1''}  
\newcommand{\nsay}[1]{`#1'}  
\usepackage{endnotes}  
\newcommand{\B}{\backslash{}}  
\renewcommand{\,}{\textsuperscript{,}}  
\usepackage{setspace}  
\usepackage{tipa}  
\usepackage{hyperref}  
\begin{document}  
\doublespacing  
\section{\href{mastery.html}{On Mastery}}  
First Published: 2025 April 15

\section{Draft 2: 2025 April 15}

This post comes with four goals: explaining why I like the \say{four stages of competence} model of mastery, explaining how they work, justifying the utility\footnote{how do you, dear reader, use utility and usefulness differently? The first feels more like something I do, while the second is the possibility? I'm not sure though, that feels unsatisfying} of the model, and reflecting on ways that I can better incorporate it into my life.  
One thing I realized while writing the last draft is that I think of each skill as consisting of two elements: the result and the method.  
When strumming a guitar, for instance, there is both the way that the pick\footnote{plectrum, if you need to feel fancy} needs to move up and down the strings and also the way that your entire body moves to do make the pick move like that.  
Both are important to mastery, but mean very different things, and in my experience, at least, do not develop at the same pace.  
Pedagogically, I think that many explicitly teach a \say{bad} initial method, because the perfect efficiency and smoothness of an expert require so many small systems working in tandem.  
By breaking that down into the parts, you lay the groundwork for becoming skilled, not just when looking at the final product, but also when watching.

Why do I like the four levels of mastery?  
First, it has my favorite of things: binary options where you go through each combination.  
A learner progresses through all four combinations of incompetence versus competence and unconscious versus consciousness.  
Second, it only flips one sign at a time: a learner is unconsciously incompetent, then consciously incompetent, consciously competent, and finally unconsciously competent.  
Third, it has the nice feature of using different negations for the two words, meaning that one could, in theory, abbreviate it as ui, ci, cc, and uc.  
Finally, it's generally easy for others to understand.  
Unlike the other mental models I use to guide my life, the four stages tend to be relatively simple for people to immediately grasp, as soon as I tell them what the stages are.

So, how does one go through them?  
At first, you do not know that you do not know a skill.  
Imagine cutting an onion for stew.  
Before learning that professional chefs cut their onions into completely precise squares of a given size, I at least just kind of cut the onion into some random size.  
I had no clue that there was a benefit to perfectly even pieces.\footnote{though, of course, I now know that there's a benefit to cutting into non-even pieces, but that's a conversation for another time}  
The result portion of the skill is far easier to progress out of this stage than the process, as one simply requires noticing what went wrong after the fact, while the other requires observation during the skill.

As you realize that your onions are not perfect little 4 millimeter chunks, you move into the conscious incompetence stage.  
Here, you know what you need to do, but cannot make it happen.  
When watching a chef, you might also notice that they hold and move the knife differently than you.  
As you try to model that behavior, the process portion can move here as well.

When you can finally get those perfect little dices, but it takes painstaking effort, you've moved into conscious competence.  
Someone calling your name while you cut makes you create larger chunks, but as long as you ignore it, you're fine.  
In this stage, while you focus, you can move the knife at a rapid and smooth clip, but only while you focus on it.  
Perhaps unsurprisingly, it is incredibly difficult to get both portions of a skill to this point at the same time.

I liken it to learning piano.  
I can play the left hand of a piano decently,\footnote{in this example} and the right hand as well.  
Playing both at once, however, results in my brain moving in far too many directions.

It is also regression into this stage that causes players to freeze, such as when taking free point shots or kicking field goals.

When watching the TV chef, they can chat while blazing through pounds of onions.  
This is because they have reached that final stage of mastery: unconscious competence.  
You do not need to think of the skill and stages to achieve it, you just do it.

I find that I tend to need to bring either the method or result to this stage before I can get the other to conscious competence.  
In general, most people I've seen suggest that you bring results up first, because, unsurprisingly, most people care about results.

Now, then, why should you use this model?  
How is it useful?

I find it useful, because it reminds me that I will obviously be bad at a skill.  
Also, the conscious competence phase is so painful for so many people, because it is when you finally realize just how terrible you are.  
Because I can now point to that not just as a necessary component of learning, but an active step forward, it becomes far more motivating.  
Rather than evidence that I have no skill, my failures become evidence that I am finally able to start learning.

It also lets me know when I can stop working on a skill.  
When entering that last stage of mastery, it can become hard to focus on the skill in question, which can make progress stall.  
After all, the whole point of it is that the skill becomes unconscious.  
Conscious effort is what creates growth.

So, when I find myself zoning out while practicing, I find it useful to take a look at the result.  
If it ends up looking like what I want, then I tend to trust that I have the skill at the level that I want it.  
Therefore, my time can be spent working to develop something new.

I should really do that more often, which is how this can help me in the future.

\section{Draft 1: 2025 April 15}

I have N goals here today.\footnote{I tend to start with a number, realize the number was wrong, and then change to N and back to the final number, which somehow is often the initial number}  
First, I want to explain why I really like the \say{four stages of competence} model of mastery.  
Second, I want to explain what each form means, both in the abstract and in the specific in my lived experience.  
Third, I want to explain why this framework is useful.\footnote{how is this different than why I like it? You'll see}  
And finally, I want to reflect on how I can be more conscious\footnote{or, ideally, unconsciously competent /s} about the framework

I like the four stages of competence model of mastery in large part because it does my favorite thing in lists: create a set of binaries and then go through each combination.  
In this case, the two binaries are unconscious versus conscious and competence versus incompetence.  
The method also has the nice thing of the different levels shifting a single vector direction in the matrix at a time.

As much as I love the method for the way that it functions linguistically, I do also like it for the way that it helps me understand my experiences, but that's the third subsection.  
I also like the method for the fact that it is relatively simple to understand.  
Much as\footnote{wow look at me not using As much as twice in the same paragraph} I love other mental models that I use, many of them require a fair level of explanation.  
The names of the four levels themselves are usually enough to get people to understand what they mean.

So, what are the four stages of competence or mastery, and what do they mean?\footnote{yes, I realize that directly above this I said that the levels were self explanatory.  
I like putting words down, though, because the site I use rewards me for doing so}

First comes unconscious incompetence.\footnote{Oh, I do also love that it uses two different negation forms, because then you can shorten to un/in, /in, /, un/}  
In this stage, you don't know what you don't know.  
This is the default state of humanity towards any task.  
Before I pick up a violin, I have no idea what, if anything, it means to play it well.

Even once starting, however, this stage does not immediately disappear.  
The first few days of playing a violin, I may be aware of some of the issues, but be missing bigger picture problems or other small areas.  
It's for this reason that so many people recommend finding a teacher for a new skill, because they can help you move out of this stage as quickly as possible.

When you finally internalize the many things you need to do in order to be good at a skill, you have moved to what can be the most painful and disheartening part of learning a skill: conscious incompetence.  
In this stage, you are aware of the many things that you do wrong, and nonetheless are unable to perform the task.  
When learning to dice an onion in a semi-professional manner, for instance, you might know that you need to make cuts every centimeter exactly.  
Knowing that this will result in perfect squares, however, does not suddenly grant you the muscle control and focus needed to move your hand exactly enough.

In the second half of this stage\footnote{yes, I do subdivide the levels of incompetence, because I find that a helpful division. I don't know if mechanical and facility (ease? speed? smoothness?) is a good division for all four stages, but so far it sure has been}, you can cut the onion into perfect little squares.  
However, when watching a professional chef, it still seems as though you are moving at the most glacial of paces.  
You can do the action, but not with the speed needed to call yourself skilled.

As your speed slowly increases, you slowly shift into the third stage, conscious competence.  
In this stage, you can cut the onion quickly and precisely, but it takes significant concentration.  
If you let your concentration slip, then, even if your pace remains the same, your cuts become less even.\footnote{hmm how do I do ease versus mechanical here? Physical process of doing the skill and fluidity! There it is, so then each skill can be broken, not just into what it is, but into mechanical and fluidic (I think that would be the right form of the word). To get through the stages, you need to be able to do both at the appropriate level, and they can be at different stages! Aha}  
Or, your cuts are great, but as your concentration wanes, so too does your pace.

With hundreds of pounds of onions cut, you finally move into the true mastery of a skill: unconscious competence.  
At this point, you are able to just dice an onion.  
It ceases to be a set of instructions and starts to be a single task in itself.  
Professional chefs who can chat along with someone while blazing through onions demonstrate this perfectly.

This is, of course, also a dangerous place to find yourself as a teacher.  
When you have fully internalized the motions and methods for any skill, it can be incredibly easy to forget any single part of them.  
It can be just as easy to forget even more.  
A way that many computer science professors love to point this out is to have someone explain how to tie your shoes, without using any physical motions.  
It's shockingly difficult, especially if you haven't had reason to attempt to do so in a long time.

Since I realized that I want to break skills into two kinds (see footnotes), I'm going to restart here.

\section{Draft 0: Meta Rambling about the Place}

I'm realizing more and more that I very much have a method for how these blog posts are written.  
I start with a story that's only related to the premise of the post by virtue of me making it so, and then connect it to what I actually want to talk about.  
With that in mind, today I think that I want to try just jumping into the content, rather than finding my way to it.\footnote{so, ignore this paragraph, basically}

I don't know where I first heard it, but I've really liked the idea that there are four levels of mastery: unconscious incompetence, conscious incompetence, conscious competence, and unconscious competence.  
A quick search seems to imply it was invented at a business school in 1960\footnote{somewhat surprisingly, \say{four levels of mastery} popped it up quickly, though the actual article on Wikipedia calls it the four stages of competence, which makes more sense.}, and that does kind of track with the framing.  
My goal here is to do two things\footnote{I think that starting with \say{goal of post is} might make it better?}

\section{Daily Notes}

\begin{itemize}

\item Obligations:

\begin{itemize}

\item Professional

\begin{itemize}

\item Write the thesis

Found out from someone else yesterday that my boss has officially announced that I'm defending in the fall, so that's exciting.

\item Revise the thesis

\item Edit the thesis

\item Research for the thesis

\item Read the books that might be useful for the thesis

Reading through one of the books yesterday was actually really helpful!!  
It is, as far as I can tell, also the initial place where one of the ways to simplify the math was introduced, which is wild.

\item Start citation tracking

\end{itemize}

\item Personal

\begin{itemize}

\item Learn the songs for to jam

\end{itemize}

\item Self:

\begin{itemize}

\item Silence

I did it! On the walk home and then the walk to work today.  
It's weird, and I do find that my mind goes into so many more places when I let myself have time to just be.

\item Typing practice.

Today!

\item Keep the phone out of the room for bed

I did! It was somewhat nice, though I did wake up to far more messages than I had expected.  
I did also check it much earlier than I would otherwise like.

\item Pray St. Michael Chaplet in the morning

Nope!

\item Stretch in the morning

Nope! Last night I could almost put my palms on the ground though, which felt absolutely fantastic.

\item Read at night

My candle is now too dim for this, and so I might have to either figure out how to make it brighter, read before turning off the lights, or find some third solution. Ope, it's dimmer because the wick is shorter\footnote{because I trimmed it}. It was less flickery, if only slightly. I do also have a candle I know burns brightly, so I could probably just switch to that one as well.

\item Poetry at night

I did! I had a line that hit me before night time stretching, which felt really nice.

\item Clean the home

I was home exactly long enough to go through my night time routine last night.

\item Stretching, standing, drinking water

Stretched last night! Stood nowhere near enough, drank probably not enough water.

\item Posture

It's becoming more and more unconscious, at least when standing. I also think that it's getting slightly better when I'm sitting, though that's much further from good still.

\item No wasted time

I think that I did ok with this yesterday! I do struggle with spending time on non-productive tasks\footnote{read: I felt really guilty spending an hour this morning trying out the new inks that I got with my writing buddy.}

\item Eat more than 2 meals a day

Oof I ate so little yesterday. Today's goal is to actually consume the oats that are sitting next to me, the lunch that I packed\footnote{leftover curry rice}, and then something for dinner.

\end{itemize}

\end{itemize}

\item Goals and Growth:

\begin{itemize}

\item Ends:

\begin{itemize}

\item Letter writing, get into more

Nope! I did find some inks that I absolutely love though!

\item Handwriting, pick and make the new one

I did a little bit, just to get the inks flowing.

\end{itemize}

\item Means:

\begin{itemize}

\item Typing speed, improve it.

Shoot! I never ended up doing this yesterday.  
Welp, that's ok, I am finding that my time is still being well used.

\item Reading, do more of it

Listened to a little bit of the book I'm going through while cleaning this morning.

\item Blogging, do it

Look at this!

\item Writing things that are not the blog and thesis, do

I got the new inks! And so I played around with four or five different colors and saw what they all looked like! There were some pretty ones, and I may not be as into green ink as I had initially thought. I was shockingly into the sheen inks (and honestly, the shimmer inks), and so now I know that for the next time I go to make an ink purchase.\footnote{which should really happen only after I finish all of these inks}
\end{itemize}

\end{itemize}

\end{itemize}

\end{document}