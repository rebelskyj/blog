
\documentclass[12pt]{article}  
\newcommand{\say}[1]{``#1''}  
\newcommand{\nsay}[1]{`#1'}  
\usepackage{endnotes}  
\newcommand{\B}{\backslash{}}  
\renewcommand{\,}{\textsuperscript{,}}  
\usepackage{setspace}   
\usepackage{tipa}  
\usepackage{hyperref}  
\begin{document}  
\doublespacing  
\section{\href{computer-mistakes.html}{On (Computer) Mistakes}}  
First Published: 2025 November 22

\section{Draft 1: 22 November 2025}

I've been commissioned\footnote{kind of} to write a song for a friend's upcoming wedding.  
As part of this, I've realized that I have really wanted to get some sort of way to have my computer automatically sing back the music I compose.  
I've had mixed feelings about this for ages.

After all, I have not hidden my general feeling that having performers is good.  
I've been clear that I don't like AI.  
Why, then, am I so adamant about wanting computer to sing for me?

Also, I'm not sure if I've written about it here, but in general I am also opposed to the idea that the composer is the sole determinant in what the piece of music is in terms of notes and shapes.  
Making computer music is sort of the antithesis of this view.

In general, I'm realizing that it comes down to two main considerations: first, I like the digital samples that come from Musescore, and am willing to listen to those.  
Second, I don't know that the music I write is worth being sung.  
Until I hear the words and melodies together, there's a chance that I would waste my money and a choir's time by having them sing it.

Third\footnote{yes, two is not the number here}, I also don't generally enjoy the process of producing.  
I recently got invited to a group and weekly discussion of musicians, and I realized that what I mean by making music and what they mean is so fundamentally different.  
Almost all of the discussion was on recordings of works, generally in what's vaguely describable as singer-songwriter genre.  
Most of the music I've written is somewhat \say{classical}, for as much as I don't love the term.  
Also, most of the works given were at least somewhat produced.

Production is always the part of music that I hate the most.  
When I took a digital music class, I found myself dreading the parts of the course that were most applicable to the work I would generally be doing in my current role as a musician: adjusting filters and plugins to slightly modify the sound of recorded audio.  
I don't entirely know why I dislike it, other than the fact that it's a digital skill that I've never needed or felt the need\footnote{we're not getting into the difference now} to learn.  
Unlike most manual skills, where I can at least understand that I don't have the time at the exact moment to work on the skill, I have never found that sort of peace with really any digital skill.

All of this culminated in me purchasing a VST for choir on Thursday night.  
Yesterday, I was far too exhausted to do anything, and that's a worrying trend.\footnote{I slept through multiple alarms today and missed a workout I had been scheduled for. I'm hopeful the fine isn't too bad}  
I think part of the exhaustion comes from the sense that I'm constantly running.  
The sense that I'm constantly running comes from not knowing my life ahead of time.  
The not knowing my life comes from not journaling.  
The not journaling...

Anyways, once I had purchased the application, I tried to download the music.  
It was\footnote{is} a nearly 60 GB file\footnote{which makes sense, given that it's a bunch of uncompressed audio that should let me hit any note and any word}.  
For whatever reason, the creators of the software believe that it's totally fine to require double the final storage space for downloading.\footnote{as someone who crashed a few servers by generating all files then deleting them, rather than sequentially, I understand why it's easier. Doesn't make it any less annoying}  
So, on Thursday I managed to clean about 80 percent of the files I would need.

This morning, I decided to clean the remaining percentages.  
After mucking around on my computer for a little while, deleting files from hidden folders with abandon\footnote{dear future me: sorry for the many things that I'm sure are now broken} and uninstalling countless software, I decided to also shift to cloud storage for some files.  
Now, fervent readers of the blog may remember a previous time I attempted this\footnote{or not, I'm unsure if I actually mused about it  then. Ughhh I still need to come up with my new terminology} cloud storage.  
At that time, I was running the code that became my dissertation locally, which involved\footnote{and, to be fair, still involves} generating and deleting tens of thousands of files a minute.  
Obviously, the sync was not happy with that, and so I disabled it.

When all was said and done here, however, I had just under the necessary storage, but had enough if I was able to use the \say{purgeable} storage.  
I quickly googled how to purge it, got the answer of generating an arbitrarily large file and then deleting it, and ran the code.  
What I forgot in doing that, however, is that the computer was also willing and able to pump everything off of my computer.  
So, bright side: I have more storage free on my computer than I had since probably the day I got it.  
Downside: literally nothing was local.

And, of course, the software still refused to install.  
I ended up finding a workaround\footnote{open the DAW that they ship with the VST, download the audio files from there, then go to the software and check for updates}, but that was very frustrating.  
I came here to start to write about that, and then realized I never posted \href{pilates.html}{my last post}.  
I tried to post it, but of course, the blog folder was now in the cloud.

Github, for whatever strange reason, is not great at dealing with files that are not stored locally.\footnote{funny how that works}  
By this point, I was frustrated beyond normal\footnote{forgot to mention: even once all the files were downloaded, I couldn't load the actual make words application, for whatever reason}, and so I just did a hard reinstall of the blog.\footnote{so... if it looks wrong, that's why}

That's where I'm at right now, but I can look at the bright sides.  
First, my computer is free of a lot of the random files I generally want to hoard but have no intention of using any time soon.  
Second, I have the software that I have wanted for a while, and I'm nearly positive that it's even working at this point.  
Third, my blog works again, and I'm even blogging right now!\footnote{look at me, getting words on a page}  
Fourth, uhhhh this is something that I did, which is always an accomplishment.

However, the title of this post has computer in parentheses.  
Much of the problems that came up today happened because of prior events.  
Right now, I'm thinking about a soccer-based web novel I'm reading.  
One of the characters is asked why he isn't mad when referees make bad calls, and his response is that if his team played perfectly, there wouldn't be a chance for refs to make a mistake.  
That is, the thing he's concerned about happened twelve moves earlier, when someone made a bad judgement or mistake.

That's something that resonates with me right now.  
Most of the things that are not going according to how I'd like right now are entirely because I made previous bad decisions and have been doubling down on them.  
Why am I exhausted?

Primarily because I decided to suddenly go from waking up whenever to waking up at 630 to waking up at 5 am.  
While I do generally think that it would be best for me to be awake at 5 am, I have to modify other parts of my behavior first.  
For example, I need to make sure that I'm going to bed at a reasonable hour and getting enough sleep.  
Given the fact that yesterday and Thursday I came home and immediately fell into ninety minute naps, and that today I again slept for three extra hours when I gave myself the chance, that's clearly not happening.

Of course, there's far more to it than just the lack of sleep.  
I often live my life in phases of crash and run.

I'm not feeding myself emotionally and spiritually/mentally.\footnote{is there a meaningful distinction? I'm not entirely sure.}  
This is evident in the fact that, upon waking, I still simply want to lie down and simply not have to think or move or experience.  
That's a symptom of burnout, I've learned.  
Even though it's hard to figure out where burnout ends and depression starts, I've got a few cues I'm using as general tells.

Chief among them is that, when depressed, I don't want to do exactly nothing.  
I want to be watching bright pictures, or listening to something, or playing a dumb game.  
Something, that is, which takes just enough of my mind to keep me from thinking.

When feeling burnt, by contrast, I find that I really just want to lie and let time pass me by.  
Since I'm still feeling some of that, it might be a good time to take a break and go lie down until no longer bored.  
I've mentioned uninstalling games\footnote{maybe only in my analog diary} a few times, and that's also been about storage needs.

It's nominally, however, been about the recognition that I will sit and play some games for hours on end, leaving and feeling nothing so much as numb.  
I don't like that, even if it is sometimes a necessary thing.  
My goal is to more and more find the things I can do that also let me let time pass by/recover from feeling too much mental energy gone without having nothing to show for it at the end.

No, that's not totally right.

I want to find the activities that actually restore me, rather than simply tide me by.  
It's the unfortunate reality that the better things go, the better they keep going.  
When thinking about the plate spinning analogy, as I am right now for the first time in a while, that makes complete sense.

Spinning plates are at their most balanced when they're spinning quickly.  
The hardest part is getting each plate to spin.  
So, not only is maintenance easier than starting, maintenance can be delayed a little bit without causing a plate to wobble too much.  
That's something worth thinking about.

Also worth thinking about is food.

I've stopped eating oats most mornings, bringing my diet to a breakfast pastry, lunch\footnote{which, admittedly, is usually colorful, full of nutrients, and tasty, along with the fact that I get multiple, which means I'm also getting more calories}, and then rarely anything approaching a dinner.  
That's not great, because I do think that at least two full meals a day is important for me.  
Of course, when I come home, I rarely feel like I have the energy to cook.  
Maybe I do need a microwave, because the idea of turning on my stove feels like too much sometimes.  
Worth thinking about, at least.

So, nearly two thousand words in, I'm realizing that the mistakes I made on my computer are, generally, symptomatic of the general mistakes I'm making right now.  
Rather than consider what I'm about to do hiding, I'm going to go recover as much as I can before working with a friend on one of the many tasks I have that aren't essential but are good to have done.

Current Pen List\footnote{for my own posterity, mostly}

\begin{itemize}  
\item Hongdian Black with Fude Nib: Empty  
\item Jinhao Shark: Diplomat Sepia Black. 10/6  
\item Pilot Preppy: Diamine Bilberry. 10/6  
\item Shaeffer (blue): Empty  
\item Diplomat: Empty  
\item Kaweko: Stipela Sepia. 10/6  
\item Monteverde: empty  
\item Shaeffer Calligraphy: missing

\end{itemize}

\end{document}