\documentclass[12pt]{article}[titlepage]
\newcommand{\say}[1]{``#1''}
\newcommand{\nsay}[1]{`#1'}
\usepackage{endnotes}
\newcommand{\B}{\backslash{}}
\renewcommand{\,}{\textsuperscript{,}}
\usepackage{setspace}
\usepackage{tipa}
\usepackage{hyperref}
\begin{document}
\doublespacing
\section{\href{dietary-needs-4.html}{Dietary Needs Mostly Concluded}}

First Published: 17 January 2025

\section{Draft 2: 17 January 2025. Realized that I want to make sure I'm getting the proper micronutrients from fruits and vegetables, and wanted to start listing that}

I just finished meeting with a nutritionist.  
In general, the advice was good, practical, and not particularly enlightening.\footnote{not in a bad way, just in a \say{oh good I'm not horribly off base with my estimations}  }
That being said, I was advised to generally eat more often, which is absolutely a good idea.  
I'd like to plan to start eating lunch and dinner daily, along with two snacks of peanuts and maybe a fruit.  
The macro breakdown in draft one shows me that's probably doable, and so now it's time to make sure I get my \href{https://www.health.harvard.edu/blog/phytonutrients-paint-your-plate-with-the-colors-of-the-rainbow-2019042516501}{recommended plants}.  


\begin{itemize}  
\item Red: lycopene containing. Examples they give\footnote{that I enjoy enough to consider having as part of a daily lunch or snack} are strawberries, cranberries, raspberries, tomatoes, cherries, apples\footnote{though other sources disagree with that}, and red peppers.  
\item Orange and yellow: beta cryptothanxin\footnote{which is not beta carotene, containing a hydroxyl group on one end.}, which becomes \href{https://en.wikipedia.org/wiki/\%CE\%92-Cryptoxanthin}{Vitamin A}. Yellow peppers, carrots, oranges, bananas, mango, pineapple, corn  
\item Green: sulforaphane, isothiocyanates, and indoles\footnote{cancer stoppers I guess.}. Spinach, asparagus, alfalfa sprouts, collard greens, green tea.  
\item Blue and purple: anthocyanins\footnote{make me life forever}. Blueberries, blackberries, concord grapes\footnote{I wonder why those in particular}, raisins, plum, fig, prune, lavender.
This one might be a little harder, since the berries are generally available only in small quantities, and I don't love any of the other options. I might need to start doing dry plums, though then I should look and see how many a day I might need.  
\item White and brown: allicin\footnote{onions, mmmm, keeps me from tumors}. Onion, garlic, leeks, parsnips, mushroom  
\end{itemize}

Of course, frozen fruit is an equally viable option.  
The site claims I should shoot for about 4.5 cups a day of fruits and veggies\footnote{nine servings, I imagine, though \href{https://www.health.harvard.edu/nutrition/how-many-fruits-and-vegetables-do-we-really-need}{Harvard seems to disagree} Green, vitamin C, and beta carotene seem to be the important things for that. Ah, one cup of each is considered a serving, other than fruit juice and dried fruit, which are both halved.}  
The book I was using for macros does not provide specific guidance, so that seems like a reasonable enough goal.  
Writing my shopping list for next week, then:\footnote{this is more a journal entry than blog post, but}  


\begin{itemize}  
\item Rotisserie Chicken. Good source of protein, and I can just take chunks to work for my lunch.  
\item Frozen meatballs\footnote{maybe}  
\item Hard cheese for lunches\footnote{because I'm going for a snack board kind of vibe}  
Upon reflection, a soft or spreadable cheese could also be nice, because I will have bread with lunch  
\item Frozen cherries or strawberries or fresh cherry\footnote{I forget the specific term} tomato  
\item Carrots  
\item Spinach\footnote{I do also own green tea, which I should remember to consume during lunch}  
\item Dry plums?  
\item Mushroom  
\item Peanuts  
\item Whole Wheat Flour  
\item Vital Wheat gluten  
\item Seeds for bread  
\item Apples  
\end{itemize}

My lunch each day will shoot for 30 grams of protein, and I'm hoping for it to include:  
\begin{itemize}  
\item Chicken, about 100 grams\footnote{about 20 grams of protein}  
\item Some \href{https://fdc.nal.usda.gov/food-details/328637/nutrients}{hard cheese}, probably 50 grams\footnote{10 grams protein}  
\item A small loaf of a fruit\footnote{probably prune} and nut studded\footnote{stuffed? I never know what it means when people say a loaf is studded} bread\footnote{dietary info to come, might mean I need less of the chicken or cheese?}  
\item Some form of mushroom, ideally at least? Maybe make a little pate?  
\item Spinach, one handful  
\item One apple  
\item Mug green tea  
\item Carrot  
\end{itemize}

The snack container, having just been measured, is about 3.5 oz, so my snack(s) will be:  
\begin{itemize}  
\item 3.5 oz peanut\footnote{about 24.5g, 600 calories (wow that's a lot of calories}  
\item Piece of fruit? I think that I still may need to get in a red, in which case maybe strawberries.  
\end{itemize}  
Here we go!

Daily Notes:

\begin{itemize}  
\item Practiced guitar? Not yet today, generally decently though.  
\item Twice daily stretching? It was going well until Tuesday  
\item Poetry? Kind of completely stopped.  
\item Blog? Lost track of time last Friday, stayed out too late, lost Saturday to that\footnote{and staying out too late}, lost Sunday to that, and here we are.  
\item Net cleaner home? I think so! I'm optimistic about going forward.  
\end{itemize}  
I only drew once, on Sunday, but I did manage to write and send out a letter\footnote{also on Sunday}.  


\section{Draft 1: 17 January 2025}\footnote{did absolutely write 2024 at first, did have to correct myself}

I just finished meeting with a nutritionist!\footnote{there are a number of unseen benefits to continuing education. This was absolutely one of them}  
In general, there was nothing particularly life altering about the advice she gave.  
I feel mixed about that, as I tend to when no easy solutions present themselves.

On the one hand, it's nice to know that I wasn't just failing to do the proper research or draw the correct conclusions.  
On the other, it's kind of frustrating to go into a situation with a  problem and leave the situation with the same problem.

That being said, though, there were three pieces of advice that she gave which I think may become helpful for me in the coming weeks.\footnote{no, \say{don't worry too much} was not one of them}

\begin{itemize}  
\item Make sure to be eating every 4 hours.\footnote{When awake, though I didn't ask about what happens when I wake up in the middle of the night} I keep seeing different times for how often one should be eating, and while I know I've seen some data suggesting other timelines, 4 hours is a reasonable enough goal to keep.\footnote{I should also be eating breakfast, but that's its own issue}  
\item Break the protein up throughout the day, and in particular, make sure to eat snacks.\footnote{nuts and seeds were examples of high protein snacks}  
\item Set timers to stop working and, even if I don't eat anything, check in with myself to see if I do want to eat anything.  
\end{itemize}

Other pieces of advice that I found useful, though less so:  
\begin{itemize}  
\item Make sure to count the protein in everything I eat. I tend to do this, but I do sometimes think I might forget to when I'm ball parking the total consumption.  
\item Don't be afraid of premade meals. In particular, rotisserie chickens and frozen meatballs are my friend.  
\item More fruits and veggies is never a bad thing.  
\end{itemize}

Let's see what peanuts\footnote{the cheapest nut, by far} are like in terms of their protein efficiency.  
Looks like it's around 170\footnote{plus minus ten} calories per ounce, and about 7 grams of protein in that same space.  
Converting to grams, that means I'd need about 364 grams of peanuts a day, or about 13 oz.\footnote{I did not convert back, but I did redo the math without the conversion because I still consume my food in ounces, in general. Or, at least, I feel like I can estimate an ounce better than 30 grams, even though they're the same amount}  
That's really not bad!

So, assuming that I eat two meals\footnote{because shooting for two a day is a good starting goal}, if I'm awake for 16 hours a day, I should be eating at least two snacks.  
I'm going to start shooting for 3-5 ounces\footnote{which is .75 to 1.25 cups, which might be FAR too much, but who knows} of peanuts at each of these snacks.  
That means I'll be getting between 21 and 35 grams per snack, for a total of 42 to 70 grams per day.  
Oh, yeah ok if I can eat 10 ounces of peanuts a day, I should have no trouble finding the other 20 grams from somewhere.  
Looking at the price per gram of protein, it's about 2 cents an ounce.\footnote{which I only now realize I calculated in the second posting}

Since I will generally be in my office from 8:30 to 5\footnote{I feel like these hours are generally decent, and somewhat reflective of reality}, I can do two snacks and a lunch\footnote{starting next week, since I'll be able to do my pseudo meal prep}, which is great.  
Assuming that I can manage 25 grams of protein in the lunch\footnote{I'm thinking part of rotissiere chicken, fruit, and probably bread (it took me far too long to remember that bread exists), which is a fun thing that is full of tactile sensation and also requires effectively no prep. Since I do also want to hit the other two kinds of vegetable (I don't care about starchy, and peanuts are legumes)}, I only need that much again for dinner!  
That's so much easier, and if I make my lunch:

\begin{itemize}  
\item Wildly, apples are not red fruits in the sense of containing lycopene.  
\end{itemize}

\end{document}