\documentclass[12pt]{article}[titlepage]
\newcommand{\say}[1]{``#1''}
\newcommand{\nsay}[1]{`#1'}
\usepackage{endnotes}
\newcommand{\B}{\backslash{}}
\renewcommand{\,}{\textsuperscript{,}}
\usepackage{setspace}
\usepackage{tipa}
\usepackage{hyperref}
\begin{document}
\doublespacing
\section{\href{dungeons-dragons-6.html}{Dungeons and Dragons Again}}
First Published: 2023 November 13

\section{Draft 1}
As I mentioned last week, I've started a new dungeons and dragons campaign.
I found out after this first session that we're supposed to be in a like turn of the first century time period, which makes my decision to play a plasmid somewhat of a choice.
The session started as most first sessions do, we all were introduced to each other in a tavern.

My character immediately drew a lot of attention, because I finally decided on a shape and build.
I was a vaguely humanish shaped blob, and my skin was shaded in reds blacks and browns, as though a lava flow.
As a result, while I was known to Giants as Goob\footnote{not that anyone in the party learned this}, I was known to most of the smaller races as Ember.

We met a small girl named Margaret, who was a Paladin on a quest currently to destroy the pirates who had kidnapped her.
The first combat went smoothly, and Goob easily withstood all of the damage that it had drawn, keeping the rest of the party safe.
It also managed to free a Tiefling from her captivity, and was pretty sure that she watched and followed the party as they continued on to find the rest of the pirates.

As we marched, the party found an arrow at our feet with a message in Elvish.
Thankfully, Margaret knew Elvish, and was able to translate.
The note said that we had allies.

Some members of the party were less than thrilled with this turn of events, feeling as though it was a bad idea to trust people we could not see.
The rest of us, Goob obviously included, felt like we were in no position to turn down help, and it wasn't as though they'd really offered us a choice.
We were ambushed by six pirates, who our unseen allies quickly dispatched.
With them gone, we looted their corpses.\footnote{My character was neutral evil, which I took to mean was very willing to spill blood if it meant that my character was compensated for the time.
Since Margaret, being twelve, did not have coin to offer, my character took on itself the choice to loot the people we killed instead.}

Adventuring a little further, we came to a staircase which descended to 25 or so pirates.
Goob had some fantastic attacks, dealing well over forty damage on multiple turns.\footnote{wow a zealot barbarian with the Fire Giant Fist Feat is a fun thing to play.
It would have been better if I hadn't treated Strength as somewhat of a dump stat, but I thought that having higher Constitution would be a good choice.
It was, as it turns out (spoiler alert) but not a good enough one.}
Because Goob played his role as tank admirably, most of the party took minimal damage.

Margaret, of course, being young and known to the pirates, also drew a fair amount of fire.
Goob, seeing this, rushed to draw the aggression of the other patrols.
In a few almost sickeningly powerful blows, he dispatched almost the entire patrol in front of them.
Unfortunately, the surviving members of each patrol focused their fire onto Goob, and it died.\footnote{more accurately, despite the fact that I took half damage because I was raging, the final attack did 40ish damage pre reduction and I had eight health.
I then failed my first death saving throw and they stabbed me to make sure I was downed.}

This was, as I learned afterwards, the GM's first ever player character death.
He felt terrible about it, which I feel bad about.
I was not super attached to this character, for all that I think it was fun.

I did really enjoy trying to get into the mind of an alien who had been raised by giants trying to interact with humans and astral elves.
I roleplayed the fact that, having an amorphous body, I was not going to be constrained to typical biology.\footnote{technically, as written I am unsure if that is allowed for the plasmid, but the GM was willing to let it slide, especially since there was no real mechanical benefit to be gained.
There's a lot to be said for \say{this is cool and I promise I'm not going to use it to try to do any game breaking shenanigans} (though there is also something to be said for \say{I am absolutely going to take this as far as you will let me, up to and including breaking the game. Tread warily})}
Among other things, this meant that I tried to copy the gait of members of the party, and that I had two eyes\footnote{normal!} which rotated about my head, one traveling clockwise and the other counterclockwise.\footnote{abnormal}

I also often forgot that people could not just move their head or body in directions, doing so.
In character, at least, the party was sad to see me die.
Out of character, I think that they're mostly just excited to see what character I bring next week.

One thing I realized when I rolled my first critical of the night was just how much I love rolling large numbers of dice.
For all that I don't have a ton of time in the next few days,\footnote{shoot, I have to prepare a lecture for next tuesday.
I should probably do that tomorrow? I don't think I have anything explicitly scheduled or any experiments that are particularly time sensitive.
I even have a few long calculations to run, and didn't have a ton to do while they did.
Great} I might spend a bit of time finding the way to build a character that gets to roll a lot of dice.

Immediate ideas:
\begin{itemize}
\item Monks apparently get large numbers of attacks
\item Rogues get sneak attack, which is a fun lil 2d6 right now\footnote{we're fourth level}
\item I think I heard half orcs crit on 19, which doubles my odds of getting double dice
\item I have to imagine that there are more feats than just the giant foundling one that give you an extra die while attacking, even if it is limited in how many times you can use it
\end{itemize}

It's not a great list, I'll fully admit, but it at least is a place to start looking.
Barbarian does have the benefit of the Zealot subclass, which gets an extra d6 on the first attack of each turn while raging.
That was a lot of fun tonight, especially on the one critical I had.\footnote{2d6 plus 2 necrotic damage from being a zealot, 2d10 fire damage from being raised by fire giants, 2d12 plus 4 damage for raging and swinging a battleaxe was really fun}

I guess one thing is that rolling 2d6 is better than 1d12, even though the damage output is slightly lower, so I should look for melee weapons that have more than one hit die as well.
Also, because it's already late\footnote{see the fact that I'm writing this after the game ended}, I'm not going to revise today's post.
That's a shame, because it does limit the word count for today, but such is the way of life.

Daily Reflection:
\begin{itemize}
\item Did I write 1700 words for NaNoWriMo?
I did! I actually wrote a bit of a fight scene that I'm almost proud of, which is really nice.
\item Did I write a chapter of Jeb?
I finished the chapter that I've been allegedly working on for a few days.
I also spent some time plotting out what I'd like to start happening in the next few chapters.
\item Did I blog? Look at this beautiful blog post wow!
\item Did I stretch? No, and I forgot to yesterday as well.
I'm starting to feel stiffer, though that might also just be the recovery from illness.
\item Am I doing better at prayer than a rushed and thoughtless rosary? 
I don't think last night's rosary was rushed, and I did remember to take some time to pray the Angelus at around noon.
That being said, tonight's rosary will likely be rushed.
\item Am I doing a good job writing letters to friends?
I wrote a letter!
There was a bit of downtime in my day, and I filled it by writing a letter.
Now I just need to post it, but I can do that tomorrow\footnote{presumably, at least}
\end{itemize}


\end{document}