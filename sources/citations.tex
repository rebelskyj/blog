\documentclass[12pt]{article}[titlepage]
\newcommand{\say}[1]{``#1''}
\newcommand{\nsay}[1]{`#1'}
\usepackage{endnotes}
\newcommand{\1}{\={a}}
\newcommand{\2}{\={e}}
\newcommand{\3}{\={\i}}
\newcommand{\4}{\=o}
\newcommand{\5}{\=u}
\newcommand{\6}{\={A}}
\newcommand{\B}{\backslash{}}
\renewcommand{\,}{\textsuperscript{,}}
\usepackage{setspace}
\usepackage{tipa}
\usepackage{hyperref}
\begin{document}
\doublespacing
\section{\href{citations.html}{Citation Styles}}
First Published: 2018 October 20

Prereading note: Draft 0 of this post is much more rambly and ranty than normal.\footnote{apparently I'm slowly shifting into my inspiration}

\section{Draft 2}
A common comment about internet subculture is that it fractals infinitely.
That is, within any interest group, you can always find divisions, going down as far as you're willing.
So, I feel better about complaining about this niche dispute, that of citation.
There are two normal ways of citing in academic writing.\footnote{at least as far as I've seen}
There's in-text, and there's footnoting.
For a variety of reasons, I find footnoting to be the objectively better form.

To me, the most important part of an essay is the flow of the prose.
I'd much prefer to read an essay with a poorer argument, but a better cadence and rhythm.\footnote{and is part of the reason I have a problem with a lot of academic writing, which ignores the importance of this}
Footnoting encourages this flow, because shifting citations doesn't affect the cadence of a sentence.
Conversely, if an in-text citation changes position, or if the citation itself changes, the cadence of the sentence changes.

Second, in-text citations serve as a distraction.
If you know me at all, you probably know that I'm incredibly distractible.
When I see an in-text citation, I'm reminded of the fact that the argument comes from somewhere else, and feel an urge to read the initial argument.
Or, at the very least, I stop reading the paper for a second, and start seeing the paper.
That is, I stop seeing the symbols as a dialogue with the author, and start seeing blobs of ink on paper.

A third reason is completely arbitrary and subjective.
In-text citations are associated with poor quality prose for me, if only because they were required in my formative years.
We would be required to use all of the different kinds of citations,\footnote{using titles, author names, and page numbers in varying locations both in the general prose, and also in the parentheses} regardless of which would flow better.

Finally, footnotes allow the author to make notes that may not be useful in the argument, but could still be useful to the reader.
In almost all of the semi-academic\footnote{i.e. works that are meant for use in by academics, but aren't meant to be published in a journal and can be done as somewhat pleasurable reading} reading I've done, the footnotes contain information that was useful to me as a student, even though it didn't have any direct relation to the core thesis.
If using in-text citations, one is forced to either leave out the information, try to work it into the argument, or use footnotes and in-text citations.
I have problems with each of these options.

The first option, excluding the information, makes the reading harder for a reader, and often contains prose that has sent me down wonderful educational explorations.
The second, working it into the argument, bloats the prose and renders the initial argument harder to discern.
The third option, that of mixing both styles, is just ugly to me.

Now, a fair complaint here is that in-text citations don't deal with parenthetical expressions, and so focusing on that is unfair.
Another complaint may be that good writing doesn't need asides, as all relevant information should be in the text.

I'd dispute both of these responses.
To the first, I believe that\footnote{in general} an option that is more robust while not sacrificing any usability\footnote{which feels the case to me in footnotes} is a better option.
To the second, I believe that it presents a limited view of good writing.
If in doing research on a topic, and finding that a certain path of inquiry may be interesting, but no longer relevant to the central argument, putting the information in a footnote can help a future scholar.
I know that footnotes have also been useful to me in understanding the main text.
When I was unsure about the meanings, I was able to consult footnotes to read what they meant.
However, to rebut the response that the information should just go in a real parenthetical,\footnote{i.e. in parentheses} if I don't need to read the information, then the parenthetical adds words that I have no need to read.

So, in conclusion, I much prefer reading my citations like this\footnote{Rebelsky 2018} than like this (Rebelsky 2018).
\section{Draft 1}
There are two normal ways of citing in academic writing.\footnote{at least as far as I've seen}
There's in-text, and there's footnoting.
For a variety of reasons, I find footnoting to be the objectively better form.

For me, the most important part of an essay is the flow of the prose.
I'd much prefer to read an essay with a poorer argument, but better flowing prose.
Footnoting allows this very easily, because shifting citations doesn't affect the cadence of a sentence at all.
Conversely, if an in-text citation changes position, the cadence of the sentence changes.

Second, in-text citations serve as a distraction.
If you know me at all, you probably know that I'm incredibly distractible.
When I see an in-text citation, I'm reminded of the fact that the argument comes from somewhere else, and feel an urge to read the initial argument.

A third reason is completely arbitrary.
In-text citations are associated with poor quality prose for me, if only because they were required in my formative years.
We would be required to use all of the different kinds of citations,\footnote{using titles, author names, and page numbers in varying locations both in the general prose, and also in the parentheses} regardless of which would flow better.

Another reason is that footnotes allow the author to make notes that may not be useful in the argument, but could still be useful to the reader.
In almost all of the classics and linguistics semi-textbook\footnote{i.e. works that are meant for use in a classroom, but aren't tertiary, and rather are the author's own research} reading I've done, the footnotes contain information that was useful to me as a student, even though it didn't have any direct relation to the text.
If using in-text citations, one is forced to either leave out the information, try to work it into the argument, or use footnotes and in-text citations.
I have problems with each of these options.

The first option, excluding the information, makes the reading harder for a reader, and often contains prose that has sent me down wonderful educational explorations.
The second, working it into the argument, bloats the prose and renders the initial argument harder to discern.
The third option, that of mixing both styles, is just ugly to me.

Now, a fair complaint here is that in-text citations don't deal with parenthetical\footnote{ooh is that the right word here? They aren't in parentheses, but it feels right so I'll keep it} expressions, and so focusing on that is unfair.
Another complaint may be that good writing doesn't need asides, as all relevant information should be in the text.

I'd dispute both of these responses.
To the first, I believe that\footnote{in general} an option that is more robust while not sacrificing any usability\footnote{which feels the case to me in footnotes} is a better option.
To the second, I believe that it presents a limited view of good writing.
If in doing research on a topic, and finding that a certain path of inquiry may be interesting, but no longer relevant to the central argument, putting the information in a footnote can help a future scholar.
I know that footnotes have also been useful to me in understanding the main text.
When I was unsure about the meanings, I was able to consult footnotes to read what they meant.
However, to rebut the response that the information should just go in a real parenthetical,\footnote{i.e. in parentheses} if I don't need to read the information, then the parenthetical adds words that I have no need to read.

So yeah, footnotes are great and I hate in-text citation.

\section{Draft 0}
One thing that I've begun to notice in a lot of English essays,\footnote{especially mediocre ones} is the tendency for the titles to be alliterative.
And, that's something that frustrates me for a few reasons.
First,\footnote{I still find it odd that I was conditioned to use \say{first} instead of \say{firstly} in such a short time} the titles tend to be much harder to speak aloud.
Second, they tend to be less descriptive.
But, that's not the point of today's post.

Rather, it's an internal complaint about my own inability to write an assigned essay.
As I mentioned \href{writers-block.html}{yesterday}, I don't tend to find myself unable to write anything.
Rather, it tends to be the issue of writing the correct thing.
And yet, with a paper that's currently assigned, I find myself with the problem of being unable to find anything to write about.

I've tried to force myself to write anything, and it's been some of the most strained, painful writing I've done in a while.
There're probably many reasons for this.
I'm going to try to go through them, and see if I can't resolve them internally.
Partially it's due to the fact that the prompt for the essay is so vague as to allow almost all writing, while specific enough to limit any of the pieces of critical analysis I would've enjoyed doing.\footnote{so really anything on music}
Partially it's that this week's meeting of the class felt painful to everyone involved.
Partially it's that I have so many other assignments, which are so much more fun to do.
Part of it is that I'm not allowed to cite the way that I prefer,\footnote{i.e. with footnotes} and am instead forced to use in-text citations.

Now, I guess an aside, I should mention why I hate in-text citations.
Or, I could just write today's post about that.
Yeah that sounds better than whining.
I'll still leave this as Draft 0 in the interest of full disclosure.\footnote{or something}
\end{document}