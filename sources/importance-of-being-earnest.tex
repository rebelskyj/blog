\documentclass[12pt]{article}[titlepage]
\newcommand{\say}[1]{``#1''}
\newcommand{\nsay}[1]{`#1'}
\usepackage{endnotes}
\newcommand{\1}{\={a}}
\newcommand{\2}{\={e}}
\newcommand{\3}{\={\i}}
\newcommand{\4}{\=o}
\newcommand{\5}{\=u}
\newcommand{\6}{\={A}}
\newcommand{\B}{\backslash{}}
\renewcommand{\,}{\textsuperscript{,}}
\usepackage{setspace}
\usepackage{tipa}
\usepackage{hyperref}
\begin{document}
\doublespacing
\section{\href{importance-of-being-earnest.html}{The Importance of Being Earnest Review}}
Prereading note: I find myself beginning to put my words to the page\footnote{or digital analog of that at least} as the time quickly approaches midnight. As a result, I apologize for any roughness in this posting.
\section{Draft 1}
Tonight, I had the wonderful fortune of seeing \textit{The Importance of Being Earnest} at the Vaudeville Theatre.
Overall, the show was enjoyable, though different from what I had expected.

What was\footnote{from my reading of the text} a show of subtlety and wit became, instead, a loud and forceful and exuberant farce.

From the beginning, I should have known it would be so, as the curtain onstage had a quote from Oscar Wilde:\footnote{author of the show} \say{If one tells the truth, one is sure sooner or later, to be found out.}
As the show begins, a piano and accompanying orchestra play loudly and beautifully, as the stage slowly lights, and the once opaque curtain becomes transparent.
The curtain opens, and we get a chance to see Algernon playing the piano.
From his first actions, it is clear that this will be a show, not of subtly implied innuendo, but rather over the top theatrics and staging.

The blocking was continuously adding a second thread to the show.
At the beginning, when Jack and Gwendolen are flirting during the conversation, their blocking is theatrical and beautiful.
In scene three, the three servants add another layer to the show, reminding us both of the existence of the stage behind us, as well as the motion of life.

The lighting was continuously sublime.
As peaks and valleys in action happened, so too did the lights subtly do so.
In the second scene, as the day progresses, the stage left light slowly rose from perspective, and the fog on the stage cleared as well.
It was a beautiful way to suggest the passage of time.
In the third scene, where we see the world from the other side, just a short time later, the sun is facing from stage right.
The little details like that made this production eminently enjoyable
\end{document}