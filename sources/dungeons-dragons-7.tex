\documentclass[12pt]{article}[titlepage]
\newcommand{\say}[1]{``#1''}
\newcommand{\nsay}[1]{`#1'}
\usepackage{endnotes}
\newcommand{\B}{\backslash{}}
\renewcommand{\,}{\textsuperscript{,}}
\usepackage{setspace}
\usepackage{tipa}
\usepackage{hyperref}
\begin{document}
\doublespacing
\section{\href{dungeons-dragons-7.html}{Dungeons and Dragons Again}}
First Published: 2023 November 20

\section{Draft 1}
As I said last week, I need to make a new character.
Time passing, as it tends to, far too quickly, means that I did not get the new character made at all during the intervening hours.
As I start this musing, I have less than 7 hours until the character is introduced to the group, which means that it's time to generate character which rolls more dice.

I have the following things to optimize:
\begin{itemize}
\item Race
\item Class
\item Background
\item Feats
\item Point buy spread
\end{itemize}

I'm level four and get a bonus first level feat, which should help.
Let's start with Race, since I doubt that anything will really change too much.
Other than a few racial feats, this is fairly self contained.
I think my goal for a race is easier time criticalling, since crits double dice.

Race options that give benefits:\footnote{as tempting as it is to use NPC races, I'll have some restraint}
\begin{itemize}
\item Bugbear gets +5 feet on melee attacks, and I can think of places that would be useful.
Also, +2d6 on a surprise round, which can happen often if we prepare well.
It's a clacky sometimes for sure.
\item Centaurs get the option to charge and then bonus attack as a bonus action. Situational for sure but
\item Custom lineage gives me a free feat, which can probably increase my dice rolls somehow.
\item Glitchling lets me reroll a 9 or lower, increases odds of hit
\item Grung lets me add a DC 12 Con saving throw to inflict 2d4 poison damage. I have to imagine most of the time people will succeed on that check, but not always I suppose
\item Half Orcs get an extra damage die on criticals with melee weapons
\item Lightfoot Halfling can hide behind any medium creature, which works well with Rogue
\item Variant Human also gets a free extra feat
\item Khenra lets me reroll any one as long as I can see my twin, so if I can convince someone else to play the race, could work.
\item Kobold lets me gain advantage on attack rolls proficiency modifier times a day
\item Kor lets me reroll every one, which is very strong.
\item Green Merfolk lets me hide when only lightly obscured by nature, similar to halfling
\item Minotaur lets me use a bonus action to attack with my horns if I take the dash action. Unfortunately, I think that I only get one bonus action a turn, so even if I dash as a bonus doesn't work sadly.
\item Amonkhet Minotaur has the same thing as half orc where i get an extra die on criticals
\item Ixalan Orc also gets savage attacks
\item Revenant doesn't work in any way for this goal, but does have the fun benefit of letting me bring my character back from the dead.
Could be worth considering another time, esp since they cannot die
\end{itemize}

Unsurprisingly, there are not too many options, especially given how many classes there are.
Now I get to go through every feat available at or before fourth level to see how they balance, since I think that'll be more impactful than backgrounds, and the two together should let me figure out my Class more effectively.

Relevant feats:\footnote{here defined as lets me roll clicky clackies on an attack roll}
\begin{itemize}
\item Vital Sacrifice. I can take 1d6 necro damage to deal an extra 2d6 damage on an attack or spell. ok if I build a char resistant to necro damage, that's legitimately tempting
\item Strike of the Giants: prof. bonus times I can add a die roll to a weapon attack
\item Squire of Solamnia\footnote{ignoring the prerequisite of campaign} lets me add 1d8 prof. times
\item Spell sniper doubles range and lets me pick up a cantrip. Add that to warlock's um eldritch blast and we get a fun thing
\item Sentinel lets me use my reaction to attack someone attacking an ally, or if they leave even if disengaging
\item Savage attacker 1/turn reroll my dice. Question becomes what I mean by rolling dice, because wow that's fun
\item Polearm master lets me opportunity attack when they enter my space and also i can bonus action d4 attack. that plus bugbear and sentinel seems really fun
\item Piercer lets me reroll a piercing damage die once per turn and adds a damage die to any piercing critical.
That stacks nicely
\item Orcish Fury lets me add one damage die once per long rest. That's meh, but better than nothing.
\item Metamagic Adept - Empowered spell rerolls a damage die.
\item Metamagic Adept - Twinned spell doubles the attack dice
\item Martial Adept - Precision attack adds a die
\item Martial Adept - Quick toss lets me do a bonus attach thrown weapon which adds the die for damage
\item Mage Slayer makes it so I get to opportunity attack any mages in range
\item Lucky I'm pretty sure is banned
\item Great Weapon master lets me bonus attack another creature when I successfully reduce a monster to 0 hp or crit
\item Flames of Phlegethos lets me reroll all ones on fire damage from a spell
\item Fighting Initiate - Great Weapon Fighting lets me reroll 1 and 2s on melee damage dies from two handed and versatile weapons
\item Elven accuracy lets me reroll one non Str or Con die roll once whenever I have advantage. I've seen some people say that means three chances to crit
\item Cruel lets me add 1d6 damage to an attack prof bonus times a day.
\item Baleful Scion lets me do 1d6 necro damage and heal that much prof bonus times a day
\item Agent of Order lets me do 1d8 force damage prof bonus times a day
\end{itemize}

Ok wow that's a lot of options.
Best options are those that do not limit the times I can use them explicitly, and come in the category of giving more places where I can attack and making attacks have more dice.
\begin{itemize}
\item More attacks:
\begin{itemize}
\item Sentinel
\item Polearm Master
\item Mage Slayer
\item Great Weapon Master 
\end{itemize}
\item More dice per Attack
\begin{itemize}
\item Vital Sacrifice
\item Savage Attacker
\item Piercer
\item Fighting Initiate - Great Weapon Fighting
\item Elven accuracy
\end{itemize}
\end{itemize}
Ok so I think that it's better for me to do more dice rather than roll more often.
I feel like I generally use reactions and bonus actions, so the second category is probably better
Aw shucks, Vital Sacrifice explicitly says I can't reduce the damage I would take in gaining the boon, which is a shame, but does make sense because otherwise I build something with immunity to necrotic damage.

Most likely backgrounds won't matter, and I'm very quickly running out of hours to figure this out.
I think, at least at the level we are, Fighter makes the most sense.
Champion at third level doubles my chances of a critical, and that's really my only goal right now.
For all that Great Weapon Fighting is considered subpar, it does increase the number of dice I get to roll.\footnote{because I reroll 1}

I absolutely want Vital Sacrifice, and then I have to choose between Half Orc, elf, and Kor.
If I choose Elf, I take Elven accuracy, which lets me reroll when I have advantage.
If I take Half Orc, I get an extra damage die when I crit.
If I take Kor, 1 on attacks become not one.

Ok so 1 will always miss, which means I roll no attacks.
Kor feels better for that reason.

So I take Kor, Champion Fighter, and Vital Sacrifice.
I still have my 4th level feat or raise statistics, point buy, and background to set up.
There are far too many backgrounds but here we go:

\begin{itemize}
\item Astral Drifter gives me Magic Initiate which gives me access to cantrips.
Relevant ones:
\begin{itemize}
\item Booming Blade: if creature moves after being hit, 1d8 thunder
\item Magic Stone: ranged weapons are nice
\end{itemize}
\item Giant Foundling gives me Strike of the Giants
\end{itemize}

I think Giant Foundling is the one to go for, especially because it works thematically.
Now the question is whether it's worth losing a level in Fighter for a level in rogue, which gives me access to sneak attack.
I'm leaning towards probably not, because next level I get an extra attack.
After that, though, might be worth switching over so I get extra dice.

Now time to generate stats and pick a weapon.
I want something two handed so I benefit from great weapon master.
Maul is the only two handed weapon other than double bladed scimitar, so I'll take that.
Mauls don't do piercing damage, so that feat is out.

I'm going to do the nice sweet \say{dump dexterity, max out strength and constitution,} build.
Assuming I'm allowed to take variant ability score increases, I'll have
\begin{itemize}
\item Str: 15+1
\item Dex: 9
\item Con 15+2
\item Int 10
\item Wis 10
\item Cha 12
\end{itemize}

From there, at fourth level I want a feat that increases my con by one, and then ideally adds hit points. Something that adds ac would also be nice.
Vigor of Hill Giants seems like the choice, because I restore Con and Prof extra hit points every short rest.
Since I know that I'll be taking some damage from my feat, that's a good idea I think.
Oh, I wear chainmail, so Dex doesn't matter, which is nice.
Welp, sixteen hundred words later I now have my initial build for a creature.
This took far more time than I thought it would, but in retrospect, I've never really read through all the options before.
I'm sure that there will come a point where rogue will allow me more dice to roll, I just don't really know when that would be.

Having now finished the session, things went well.
I realized upon arriving that, even though I knew the mechanics of the character, I did not know such relevant information as name and appearance.
I looked up Kor, and they're apparently very washed out elf looking people.
Alfred was the only name that I could think of, so that's what he was named, and I decided that since he was also a giant foundling, he and Goob knew each other.

Anyways, we did not have any combat this session, which means that the only d6 I rolled were to inflict damage on myself with the hope that I might be able to use it next session.
It was a fun time!

Daily Reflection:
\begin{itemize}
\item Did I write 1700 words for NaNoWriMo? Woo! Found out my plotting included some stuff I have accidentally already covered, so we're good.
\item Did I write a chapter of Jeb? I finally finished the chapter that's been driving me nuts for about a week now.
It's annoying, but I hope that I'll be ready to do more in the next few days.
\item Did I blog? A part of me feels badly about using work I was planning to do as a blog post.
That part of me is less tired than the rest of me.
\item Did I stretch? I've fallen so far off of this goal.
\item Am I doing better at prayer than a rushed and thoughtless rosary? Yesterday's rosary wasn't too rushed, and I went to Mass, which was prayerful.
\item Am I doing a good job writing letters to friends? Still no. It is annoying to me that I can't just write letters, for all that I respect the fact that there are a lot of things I want to do that I cannot do.
\end{itemize}

\end{document}