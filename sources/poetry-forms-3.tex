\documentclass[12pt]{article}[titlepage]
\newcommand{\say}[1]{``#1''}
\newcommand{\nsay}[1]{`#1'}
\usepackage{endnotes}
\newcommand{\1}{\={a}}
\newcommand{\2}{\={e}}
\newcommand{\3}{\={\i}}
\newcommand{\4}{\=o}
\newcommand{\5}{\=u}
\newcommand{\6}{\={A}}
\newcommand{\B}{\backslash{}}
\renewcommand{\,}{\textsuperscript{,}}
\usepackage{setspace}
\usepackage{tipa}
\usepackage{hyperref}
\begin{document}
\doublespacing
\section{\href{poetry-forms-3.html}{Writing through Poetry Forms}}
First Published: 2022 April 15


\section{Draft 1}
Despite my claims about needing a while to adequately learn the telestich, I'm happy enough with the speed I wrote yesterday's that I'm going to move on to the double acrostic.
I feel like this is going to be really hard because the way that I had to modify my word choices was very different in the two single acrostic forms.
But, you know what they say: nothing ventured, nothing gained.

Attempt\\
Attend to the pleasant aroma\\
That wafts from an effort since past\\
Take heed of the slow-moving draft\\
Each time that it comes to your side\\
Meet the day with outward calm\\
Pass times until all used up\\
Together with every attempt\\

So not the best poem I've ever written, but I think it's not horrible for a stream of conscious writing.
According to the book of forms, the poem's content should center around the word being built.
I don't know if I've done a particularly good job with that today, but there's always tomorrow.
\end{document}