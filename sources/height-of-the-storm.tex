\documentclass[12pt]{article}[titlepage]
\newcommand{\say}[1]{``#1''}
\newcommand{\nsay}[1]{`#1'}
\usepackage{endnotes}
\newcommand{\1}{\={a}}
\newcommand{\2}{\={e}}
\newcommand{\3}{\={\i}}
\newcommand{\4}{\=o}
\newcommand{\5}{\=u}
\newcommand{\6}{\={A}}
\newcommand{\B}{\backslash{}}
\renewcommand{\,}{\textsuperscript{,}}
\usepackage{setspace}
\usepackage{tipa}
\usepackage{hyperref}
\begin{document}
\doublespacing
\section{\href{height-of-the-storm.html}{The Height of the Storm Review}}
First Published: 2018 October 29
\section{Draft 1}
Tonight, I had the wonderful fortune of seeing \textit{The Height of the Storm}  at Wyndham's Theatre.
It was a very confusing show, layering time, subjective and objective realities colliding within a fracturing mind.
It hit me very powerfully, although I haven't had to watch someone around me undergo it.
I have felt some of the deja vu that the show plays with, especially when thinking about my grandparents.

The stage was interesting.
When I first saw it, I felt a little nauseous, because it felt almost as if a two point perspective drawing made in three dimensions.
However, as the show progressed, it began to feel more and more natural.
\end{document}