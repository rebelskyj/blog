\documentclass[12pt]{article}[titlepage]
\newcommand{\say}[1]{``#1''}
\newcommand{\nsay}[1]{`#1'}
\usepackage{endnotes}
\newcommand{\1}{\={a}}
\newcommand{\2}{\={e}}
\newcommand{\3}{\={\i}}
\newcommand{\4}{\=o}
\newcommand{\5}{\=u}
\newcommand{\6}{\={A}}
\newcommand{\B}{\backslash{}}
\renewcommand{\,}{\textsuperscript{,}}
\usepackage{setspace}
\usepackage{tipa}
\usepackage{hyperref}
\begin{document}
\doublespacing
\section{\href{bagpipes.tex}{On Bagpipes}}
First Published: 2022 October 24

\section{Draft 1}
Yesterday I got to play bagpipes at a friend's wedding.
It was a lovely service, and I'm grateful that I got to take part in it.

While I was setting up, I remembered the last service I saw bagpipes at.
I was at a funeral for a friend's family member.
It was there that I thought about how fitting bagpipes are for emotional performances.

Unlike most any other instrument, which requires consistent and continuous breath support, in the bagpipes you blow just the top of your breath each time.
When your breathing turns to gasps, you can\footnote{in theory I assume} still play pipes.

Anyways, yesterday's occasion was really happy and joyful.
It's always shocking to me how much people really appreciate hearing music, especially from instruments that they don't hear often.
Most of the parents and assorted older members of the wedding came up to express their gratitude for my playing, and the bride and groom both were happy for ti.
It's nice when my hobbies actually get to be helpful.
\end{document}