\documentclass[12pt]{article}[titlepage]
\newcommand{\say}[1]{``#1''}
\newcommand{\nsay}[1]{`#1'}
\usepackage{endnotes}
\newcommand{\1}{\={a}}
\newcommand{\2}{\={e}}
\newcommand{\3}{\={\i}}
\newcommand{\4}{\=o}
\newcommand{\5}{\=u}
\newcommand{\6}{\={A}}
\newcommand{\B}{\backslash{}}
\renewcommand{\,}{\textsuperscript{,}}
\usepackage{setspace}
\usepackage{tipa}
\usepackage{hyperref}
\begin{document}
\doublespacing
\section{\href{reflections-on-readings-22-ordinary-c-22.html}{Reflections on Today's Gospel}}
First Published: 2022 September 1

Luke 14:14 \say{blessed indeed will you be because of their inability to repay you. For you will be repaid at the resurrection of the righteous}

\section{Draft 1}
The common theme this week I saw was a fairly straightforward exhortation to seek the Lord our G-d.
The best way to do so, as the Gospel points out, is through corporal works of mercy.
That is, seeing Christ in the poor and needy, and giving to them.

As I thought about the Gospel, something else came to mind as well.
We can never repay the Lord for any of the good that He gives us.
In that way, striving to be like the Lord means doing good to those who cannot do good to us.

The Second Reading is also powerful this week, especially since I'm currently reading the part of the Bible referenced in it.
In this reading, St. Paul reminds us of the difference between seeking Christ and seeking the Father without Christ.
Without Christ, the Lord's presence is too awesome for us to behold.
\end{document}