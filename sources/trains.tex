\documentclass[12pt]{article}  
\newcommand{\say}[1]{``#1''}  
\newcommand{\nsay}[1]{`#1'}  
\usepackage{endnotes}  
\newcommand{\B}{\backslash{}}  
\renewcommand{\,}{\textsuperscript{,}}  
\usepackage{setspace}   
\usepackage{tipa}  
\usepackage{hyperref}  
\begin{document}  
\doublespacing  
\section{\href{trains.html}{On Trains}}  
First Published: 2025 September 12 (no service on trains and then forgotten)

N.B. Draft 2 is rambly, sad, and generally disorganized.
That is to say, Draft 2 is like me right now.
\section{Draft 2: 8 September 2025}
When I was in high school, I talked with my best friend about doing a road trip across all of Interstate 80.
We didn't end up doing it before undergrad, because we were too busy.
In the summer before my graduate school, then, I really planned on doing it.
That summer was 2020, however, and suddenly the idea of going cross country by car became both more and less feasible.
I am now done with graduate school, and by most objective standards have reached the pinnacle of schooling; this is my last chance for a post graduation travel celebration.
Even though I haven't been writing as much here as I would've liked this summer\footnote{thanks thesis}, I do feel like the fact that I traveled a lot is still clear.
From June 1st to yesterday, I put more than 6000 miles on my car.

I'm pretty sure 6000 miles gets me most of the way across the US.
Even if it was mostly in a spiderweb shape, rather than a line, then, it's as though I did what I had set out to do.
So, trains felt like the best way to visit friends, because I had the hope that they'd be less of a nightmare than flying.
In what is no surprise at all, they are.

I'm hoping that these weeks of travel can be for me a kind of resetting.
I will be entering corporate America starting in October, and while I am very excited for the fact, I am also somewhat nervous.
I don't know what that life will be like, and I feel like I'm kind of wasting my degree.
However, I am also almost entirely burned out of thinking, especially about chemistry.
This month will, if nothing else, be a period of time with no obligations on my mind.

I'm finding myself more and more emotional as I think about the trip.
I think that part of me is still feeling cheated by the fact that my mom was unable to watch me graduate.
Another part of me is just generaly worried about not having a place I can retreat to for the rest of the month; normally when I feel like life is too much I can either go to my apartment, or if things are particularly bad, back home.
I suppose that since I'm only doing my bookings a week or so in advance, I can always cancel, but that feels like the worst answer.

Anyways, trains are great because they're way more spacious and travel feels much more real than on a plane.
When I see cars pass by slowly or be passed by slowly, I am connected to the sense of getting between points in a way that airplanes cannot do.
Unlike driving, I can stand up and stretch, take a nap, or just close my eyes for a bit without negative consequences.

The train is just now hitting the part of Montana responsible for having mountains on the horizon.\footnote{I assume that what I see are mountains and that I'm still in Montana}
This is not my first time seeing the Rockies, or even my first time traveling through them.
I do think that this might be my first time going through the Rockies\footnote{rather than over, as in a plane} since what feels like my family's last great road trip.
I must have been ten or so, and we decided to drive to the West Coast to see my mom's family.
I remember watching math lectures on DVD with my brothers, far too much Homestar Runner, and listetning and singing along to so much great music.
I remember going to see family, and I remember the lawyers in my family telling me that I should become a lawyer.
I think that I remember going to a National Park or four.

I don't remember my mom's voice as we drove.
I don't remember what I wore.
I don't remember the feeling of love as she wrapped me in a hug, back when she was still taller than me.

Life is a train, I'm sure I've heard said before.
While being on these first hours of train time, I've been thinking about the 1970's song \say{City of New Orleans}, which is about a train.
It's also about America, the death of small towns, and the loss of innocence.\footnote{I think}
The train stopped for a half hour or so in some middle of nowhere town in rural Montana; we were ahead of schedule.
With the town less than an hour behind me, I have already forgotten its name.
I remember that there was an Ace, and that they were fundraising for some children's event.

If life is a train, what are the stops?

Last night I slept through all of the state of Minnesota.
Or, at least, I have no memories I can pin to the state.
It is more than possible that I was at some point awake, whether complaining to myself about a neighbor\footnote{seriously, do you need to call people?} or just bemoaning my inability to sleep through the night.
How much of my own life have I slept through?
What vast swaths of temporal land have I just let pass me by completely unnoted?

If life is a train, I would have said that my mom was the conductor.
I don't mean this in a domineering way, she was just the person I went to for all my questions and life concerns.
Without her, I find that I don't know who to turn to when I have questions.

If life is a train, what place does fath have?

That feels forced, but I'm thinking about how trains are so far from a faith object.
There's a comment on my web novel I think about a lot, which caused me to learn that concrete the noun is much younger than the adjective.
That is, someone looked at concrete and went \say{yeah that seems pretty definite}.
Similarly, trains are behemoths of steel and oil.
They travel with no freedom, bound by the tracks that countless men laid.
I have been thinking about my own relationship to faith a lot lately.
Am I like the train, living a life not just needless of faith, but in active oposition to it?

If life is a train, what am I?
\section{Draft 1: 8 September 2025}
I'm officially just over a day into my post graduation\footnote{both undergraduate and graduate} celebratory trip.
I decided to celebrate by going on a long train trip.

Mostly, that decision came because I've been feeling really burned out of driving lately.\footnote{shocking what driving 6500 miles in three months will do, especially when you took most of June off}
I want to visit many of the friends that have asked me to visit them through the ages, and I wanted to not deal with flights, because I hate planes in many regards.
I had a memory of one of the guys I dove with while studying abroad\footnote{I'm pretty sure as I think about it that it was one of the two men in my adult lesson class, before I was immediately recommended to just join a club team.} talking about how good the train system in the US is for getting around the country.
Since I didn't want to drive, didn't really want to fly, the old classic trio\foonote{planes, trains, automobiles} left me with a single other option.

And so, yesterday I drove down to the singular city which Amtrak uses to connect the east and west\footnote{as far as I can tell. I'm willing to believe that there's a way to go between them without using Chicago but it would be more constrained for sure}, saw a friend\footnote{and left my car with him}, and boarded a train to follow the advice once given to the founder of my home town \say{go west}.
My initial thoughts upon getting onto the train were that trains are just miles away better than planes.
Each seat is more than large enough for me to comfortably sit, there's no judgement for wandering, and there was literally no security.
I brought what I'm pretty sure is more bags than I was supposed to.
I'm the only one in my row of two seats, which is also really nice.
I love that the seat assignments are hand-done still.

Anyways, perhaps because of the nature of the novelty of train tripping, perhaps because of the absence of WiFi, and perhaps because semidaily\footnote{feels like bidaily should be twice and semi should be every other day} musing is my jam, I did not write a folly.
With that in mind, given that it's currently not yet 11 am central and that I am in the mountain time soon, I figured now would be a good time to blog aobut trains.
Except, having written these three and a bit paragraphs, I realize that I don't really have that much else to say about trains.
I guess it's time to go ahead and flip the coin between finishing binding the book and reading a paper book.
I think that I'll start with binding the web novel because I am enjoying a podcast\footnote{technically? I'm listening to a video which is also streamed as a podcast. It's very fun, and would highly recommended. I'm learning about China's only female emperor.} and want to finish it.
We'lll see what I feel like doing by the time that the podcast is over.
Likely either sit and silently do the binding, another podcast, or read \say{the Nonviolent Alternative}, which I think was in my mother's theology collection before she gave it to me.
I'll be back for draft two!
\section{Daily Reflection 8 September 2025}

\begin{itemize}

\item Did you journal by hand today?

I did!
I also did yesterday, even if I didn't write my folly then.

\item Did you do a folly?

Not yesterday, sadly enough. I'm about to do one now though.


\item Did you in some way, shape, or form advance the web novel?

No. That's a little bit sad, but it's also only Monday. I'm sure that in the next few days I'll have the time and mental space to do so.

\item Did you work on music, whether education or creation?

I was getting ready to bind my undergraduate composition professor's favorite text\footnote{or, at least, the one he recommended me to read. (Geometry of Music)} when I realized that I had packed a book on program music instead. That was sad.

\item Did you work on book binding?

See above.

\item Did you work on another hobby?

Video games, otherwise not really.

\item Did you stretch? Really?

I did! Last night at like 9pm I did a quick 5 minute stretch in the back of the train car. Might do more after this.
\item Prayer?

I packed a rosary, at least.
\item Meditation?

Kind of. I historically\footnote{read, as far back as I can think of, so at least a year} have daydreamed my way to sleep, but last night in particular I went for the whole \say{I may or may not be able to sleep like this, so at the very least we can do the Mythbusters approved meditate which gets me most of the benefits.}
Wow meditation is boring.
\item Reading?

I read Chaotic Craftsman so much yesterday.
However, it is an incredibly long book, and so I only made it through I think about 200 chapters.\footnote{I think that it's currently on chapter 900 and I just finished 600}.
\item Minimizing screen time?

Not even at all in the slightest.
A part of me is unsure how much I really want to, but I guess that's a question for another time.
I suppose that I do really want to read the books that I packed, so I might start on that.
Assuming I can finish one of them within the next 15 hours, I might be able to give it away, and therefore both reduce the size and weight (reduce both?) of my luggage, and also give a gift to my hosts.
\end{itemize}

Current Pen List\footnote{for my own posterity, mostly}

\begin{itemize}  
\item Hongdian Black with Fude Nib: Diplomat Caribbean (8/30ish)  
\item Jinhao Shark: Diplomat Caribbean (8/30ish)
\item Pilot Preppy: Private Reserve Electric DC Blue I think (I think since late june. I think)  
\item Sheaffer: Private Reserve Spearmint (since 7/15) (I Think)
\end{itemize}

\end{document}

  