\documentclass[12pt]{article}[titlepage]
\newcommand{\say}[1]{``#1''}
\newcommand{\nsay}[1]{`#1'}
\usepackage{endnotes}
\newcommand{\B}{\backslash{}}
\renewcommand{\,}{\textsuperscript{,}}
\usepackage{setspace}
\usepackage{tipa}
\usepackage{hyperref}
\begin{document}
\doublespacing
\section{\href{dietary-needs-2.html}{Dietary Needs}}

First Published: 9 January 2025

\section{Draft 2: 9 January 2024}

My last draft really went off the rails, because I got too far into the biohacking mindset. At the base minimum, I should be consuming:

\begin{itemize}  
\item 90 grams of protein  
\item 130 grams of carbohydrates  
\item 17 grams of linoleic acid  
\item 1.6 grams of alpha-linoleic acid  
\end{itemize}

Below, we see that,\footnote{ooh an unconsidered benefit of drafts} at the absolute minimum, I can meet this in a smoothie that does this relatively easily.  
Since the fat and carbohydrates are incredibly simple to achieve, it's worth considering what 90 grams of protein otherwise looks like.  
Common ways I get protein are:

\begin{itemize}  
\item Rice  
\item Pasta  
\item Fish  
\item Chicken  
\item Gelatin  
\item Beans  
\item Eggs  
\item Milk  
\item Cheese  
\item Yogurt  
\end{itemize}

To get 90 grams of protein from each of these sources, I would need to consume\footnote{going off of \href{https://fdc.nal.usda.gov/}{USDA's food search}}:

\begin{itemize}  
\item \href{https://fdc.nal.usda.gov/food-details/2512381/nutrients}{Rice}\footnote{I assume the usual 2:1 water:rice ratio many use, and that the entirety of the water is into the rice, so I divide the USDA's protein by 3} needs about 3.8 kilograms\footnote{assuming white long grain, which is my usual}, and has around 52 calories per gram of protein.   
\item \href{https://fdc.nal.usda.gov/food-details/168927/nutrients}{Pasta}\footnote{assuming the same water absorption} needs 2 KG, and has 28.5 calories per gram  
\item \href{https://fdc.nal.usda.gov/food-details/173671/nutrients}{Fish}\footnote{assuming cooking doesn't change the size much} needs 500 grams, and is around 5 calories per gram\footnote{which, I do note, means that it's almost all protein, since each gram of protein is 4}  
\item \href{https://fdc.nal.usda.gov/food-details/2646171/nutrients}{Chicken}\footnote{same assumption as fish, which will go forward as needed, and assume that the grains expand the same tripling} needs about 480 grams, and is about 7.7 calories per gram  
\item Gelatin, as mentioned below needs about 150 grams, and is about 5 calories per gram  
\item \href{https://fdc.nal.usda.gov/food-details/173734/nutrients}{Beans} need about 1.25 kilograms, 15.8 cal/g  
\item \href{https://fdc.nal.usda.gov/food-details/171287/nutrients}{Eggs}\footnote{which weirdly \href{https://fdc.nal.usda.gov/food-details/173423/nutrients}{gain a gram} of protein when fried} are apparently 50 grams per large egg, so the 710 grams I need is about 14 eggs,   
\item \href{https://fdc.nal.usda.gov/food-details/746782/nutrients}{Milk}: 2.7 kilograms of milk, about 27 cal/g  
\item \href{https://fdc.nal.usda.gov/food-details/170845/nutrients}{Cheese}: 404 grams, 13.5 cal/g  
\item \href{https://fdc.nal.usda.gov/food-details/2259793/nutrients}{Yogurt}: 2.6 kilograms, about 22.3 cal/g  
\end{itemize}

In the future, I'll hopefully figure out a way to take these numbers and construct a meal plan out of them.  


Daily:  
\begin{itemize}  
\item Practiced guitar? Yes  
\item Practiced accordion? Given up for now because only so many tasks to be added at once  
\item Twice daily stretching? yes  
\item Journal? as with accordion, given up for now  
\item Poetry? yes  
\item Blog? look!  
\item Net cleaner home? technically  
\end{itemize}

\section{Draft 1: 9 January 2024}

First, it appears that I was wrong in yesterday's musing. The oil requirements are just straight up grams per day, which makes my life significantly easier.  
In order to hit my nutrition goals, I should be consuming at a minimum:

\begin{itemize}  
\item 90 grams of protein  
\item 130 grams of carbohydrates  
\item 17 grams of linoleic acid\footnote{which according \href{https://en.wikipedia.org/wiki/Linoleic\_acid}{to Wikipedia} at time of looking, is about half of corn oil, so 34 grams of that}  
\item 1.6 grams of alpha-linoleic acid\footnote{which \href{https://en.wikipedia.org/wiki/\%CE\%91-Linolenic\_acid}{wikipedia} \href{https://en.wikipedia.org/wiki/Vegetable\_oil\#Composition\_of\_fats}{claims means about 160 grams of corn oil}, or 4 grams of flaxseed or linseed oil}  
\item I generally see 1500 calories as the starvation threshold.  
\end{itemize}

That is very little, and in total seems to add up\footnote{assuming corn oil is my only oil choice} to around 2300 calories. If I choose better oils for alpha linoleic acid\footnote{better here meaning optimal}, that number drops to 1222 calories per day.  
If I price out my costs,\footnote{assuming gelatin, sugar, corn oil and linseed oil all priced from my local grocery store that lists prices online} to hit the minimum I need for nutrition:

\begin{itemize}  
\item about 144 grams, or 5 ounces of gelatin, which comes out to 6.56 dollars  
\item 1 tablespoon of corn oil is about 14 grams, and costs about 4.5 cents, which means daily need would be around 10 cents  
\item Flaxseed oil is approximately as dense, one tablespoon costs 44 cents, and so the daily need is about 14 cents  
\item Sugar costs 80 cents per pound, and I need about a third of a pound per day, so that's about 30 cents per day.  
\end{itemize}

Total cost for the base need smoothie comes out to around 7 dollars and 10 cents.

If we swap the gelatin out for egg whites, we get a little bit of carbs, but, assuming the cost is only in protein\footnote{true to a first order}, we can start pricing out the price per gram of protein for different foods:

\begin{itemize}  
\item Gelatin: 7.5 cents  
\item Egg White: 4.7 cents  
\item Whey Protein: 6.5 cents\footnote{I'm remembering something about bioavailability, but that's probably not too relevant, skimming \href{https://pmc.ncbi.nlm.nih.gov/articles/PMC11171741/}{an abstract}, I see that all animal sources basically fine. Soy also approx 100, as is potato, interestingly. Oh wait, this is percent of amino acid intake, which have requirements in the mg/kg/day, which is so much less as to seem irrelevant to me; I'll just trust protein labels.}  
\item Chicken breast comes out to around 4 cents per ounce  
\item Flour technically has protein, though incomplete, and so is being discarded\footnote{\href{https://www.frontiersin.org/journals/nutrition/articles/10.3389/fnut.2020.00141/full}{this article} claims it misses lysine, threonine, and methionine.}  
\item Generally seems like peanuts are complete, and come in at 1.9 cents, or 2.3 if I don't want to bother shelling them myself.  
\item Beans also appear to contain all amino acids, and come in at 1 cent per gram as pinto beans  
\item Rice can come down to 1.6 cents per gram  
\item Lentils are around 1.4 cents  
\end{itemize}

Gelatin and egg whites are both about 1.6 grams per gram of protein, whey protein is about 1.1, and pinto beans are about 5, so it does require significantly more bean consumption. It also takes me well over the minimum carb load for the day, which also saves me 30 cents in sugar.  
It would, however, require consuming a full pound of beans\footnote{raw, then cooked} each day.  
Egg whites probably strike a good balance of volume and cost.

Now that we have gone through this, we also need to remember the whole \say{there is a limit of calories I can, or at least should, consume in a day.}  
At the 3500 calorie mark, that means everything needs to be under 40 calories per gram of protein in order to be a valid source of all protein I consume.

Also, if I do the insane thing of bulk buying gelatin, I can get it for as cheap as 40 percent the cost, bringing it down to 3 cents per gram, which is honestly kind of tempting. It does, however, require purchasing 50 pounds of gelatin, which would serve me for around half a year.  
It would take us down to under three dollars a day.

According to the \href{https://www.fns.usda.gov/research/cnpp/usda-food-plans/cost-food-monthly-reports}{USDA}, as long as I'm spending under 10 dollars and nine cents a day, I'm doing well.

This has absolutely degraded from the initial goal of figuring out how to feed myself well, for all that it would, without a doubt, be a very easy and effective way to get myself all of the nutrients I need in a day.  
I do wonder if I would be able to eat enough fruit and vegetable in a day to feel satiated.  


\end{document}