\documentclass[12pt]{article}  
\newcommand{\say}[1]{``#1''}  
\newcommand{\nsay}[1]{`#1'}  
\usepackage{endnotes}  
\newcommand{\B}{\backslash{}}  
\renewcommand{\,}{\textsuperscript{,}}  
\usepackage{setspace}  
\usepackage{tipa}  
\usepackage{hyperref}  
\begin{document}  
\doublespacing  
\section{\href{melody-harmony.html}{On Melody and Harmony}}  
First Published: 2025 May 7

\section{Draft 3: 2025 May 7}

The previous drafts have been a lot more meandering than I'm used to lately, and I think that part of the reason is that this post was really amorphous in my mind.  
Rather than the standard folly, where I have something I know I want to express and find the way to do so, I don't know if I ever really knew what I was trying to express in the previous drafts.  
Thankfully, whatever net I cast still managed to bring back something worthwhile to gnaw on.\footnote{I love the viscerality of the word gnaw}

Like many people, I struggle with imposter syndrome at times.  
These days, I find it happening almost exclusively in musical realms, and almost entirely when it comes to performance.  
In many regards, I can acknowledge the feeling as ridiculous.  
I am not trying to make a living doing music, so of course I am not as good as those who do.  
I know what would help my ear, and I actively choose not to do it.\footnote{ear training is just so low on the list of priorities I have}

If I was just generally not as great as music as I wanted, I think that I would be more comfortable.  
However, I do also have some amount of pride\footnote{which the emotion book says is a good thing, recognition of actual accomplishment. Catholic me still says not to use it, but also hubris and pride are two separate words for a reason} in my ability to compose, which I think is where the cognitive dissonance comes in.  
Every professional composer I've spoken to is pretty clear about the necessity of not just being able to hear the harmonies and melodies one writes as they enter the page, but of being able to hear the absent harmonies.\footnote{e.g. \say{if you play 11 notes on a piano, you should be able to hear which one you didn't play}}  
I can, at least most of the time, relatively accurately hear any individual melodic line in my writing.  
I can, however, almost never hear the harmonies.

As I write a piece for a dear friend's wedding, I am finding the absolute limits of my ability to compose without paper.  
The frets of the guitar are welcoming and encourage certain harmonies and melodies.  
When I am away from it, though, I can still think about where my fingers and hand will need to be in order to make notes.  
So, I guess that the takeaway is that I need to accept that I cannot compose without paper, even though that hurts to admit.  
When composing for choir, I need an active audio playback.

\section{Draft 2: 2025 May 6}

When I was going into my senior year of college, I was debating different career options.  
Because I was formed for academia\footnote{I can justify this statement if anyone doesn't immediately agree}, I knew that meant I wanted to go to graduate school next.  
Being a major in music with a love for composition and chemistry with a love for quantum and analytical, I asked professors what I was missing to be a good applicant to graduate school and generally successful in the industry.  
When we had visiting musicians, I did the same.\footnote{I interacted far more with the visiting musicians than chemistry professors for some reason, probably the size of departments and the fact that the visiting musicians did far more interactive activities?}

I don't remember what I was missing to be a good chemist, other than I think literal information.  
There was probably something like \say{be better at your lab notebook}, which I have not really done, but.

To be a successful composer, though, every professor and visiting musician I talked to had the exact same answer: a better ear was essential.  
Even outside of composing, the different performance based professors I worked with were often shocked at just how bad my pitch memory was.\footnote{I was a junior in college before the idea of remembering what pitch I had been singing was even a concept in my mind}  
My ear is not great, and I have no real desire to do the work essential to training it.  
Back in college, before I had matured and learned the value of suffering\footnote{read: I now practice regularly}, I was even less willing.

And so, two futures of mine diverged and I took the one of least resistance.\footnote{the fact that it's generally agreed Chemistry is more profitable than music wasn't hurting the choice either}  
That spring, I then won a composition award from the music department.  
As it turns out, even though I do not have an emotional idea of what each interval is, nor can I play 11 notes on a piano and say which one is missing,\footnote{both of these were things that are essential, according to most of the composers} I can still write music that makes the academy happy.  
Given the general reception to the songs I've written since then, I can also write music that makes the average person happy too.  
The only issue is that I cannot do it without a pencil and paper.

I can improvise lyric if I need to, and can figure out what should come next if given parts of a line.  
Given a melodic fragment, though, I cannot put the end on a guitar.

As I write a piece for my friend's wedding, this is coming up constantly.  
I am nearly positive that what I want is just some scalar walking between different chords, but don't have an idea in my hands or head how to do that when looking at the fretboard.  
What little I have been able to do past the moment of inspiration has been because I actively paused and reminded myself what chords were what scalar\footnote{I love using math music words in ambiguous contexts} distances from each other and how I could walk between them.  
Still, the more I did that, the less the new parts felt like something good.

I'm sure that if and when I go and put the notes in my sheet music editor, I will immediately know what I can do to make the song last forever.  
I love programmatic music, and especially for something which is meant to be somewhat ambient, having a number of touch points with transitions is great.\footnote{mmmm graphs}  
I can easily string a bunch of riffs or licks together, but need to have them first.

What was the point of this folly?

Really mostly me coming to terms with the fact that I don't intuit musical instruments like I do sheet paper.  
That does, in my heart of hearts, make me feel like less of a musician, but it probably isn't something horrible and worthy of despair.  
I don't think that I value the skills of composing on an instrument enough to work on them, so I guess it would behoove\footnote{ooh apparently this word is archaic now, comes directly from an Old English word that means the same! Not even like a \say{add this part to this part}. wild} me to internalize that I am ok with this fact.

I don't know how I feel about this musing right now, so may revisit it before posting tonight.  
With two minutes until the end of the hour, I'll call it here, though.  
On to the thesis.

\section{Draft 1: 2025 May 6}

While writing with my dear friend this morning, they commented that I have not been posting here lately.  
That's not incorrect.

Part of me feels like any writing I do here is writing that would otherwise be done on my thesis, which is not totally incorrect.  
There are only so many thoughts my brain can capture, break in, and pin to the page in a given day.  
However, I also know that I feel more grounded if I do these follies, and that being grounded lets me capture more thoughts.  
With that in mind, I'm going to consciously choose to spend the remainder of this working hour\footnote{currently 1122, final five minutes of an hour always reserved for stretching} writing this.  
If it doesn't get me to the point that I want, I'll try to return to it later.

So, what do I want to say about melody and harmony?

I was talking with a friend this weekend about how we write music.  
She has a great intuition and ear, and just sings what feels natural.  
I am so far from that, at least in general.

Sure, most of my songs begin with randomly singing a line or even just a few bars.  
However, what comes next is that I have to painfully figure out exactly what relative and absolute pitches I sang\footnote{absolute because many times I sing it not near my guitar and need to know whether to figure out a different chord pattern or if I can shift the song itself}.  
Even then, I only have a single line of music.

What comes next is the part of composition that I have always found easiest: composition on the staff.  
I am a product of the modern era, and really love having automatic playback.  
However, if I am away from my computer, I can write more pleasing melodic lines simply by using a staff than I tend to be able to do without the paper in front of me.  
I don't know what about me is well trained to write music.

Having written that sentence, I do realize that's untrue.  
I have spent a lot of time in my life explicitly and actively studying the melodies that I enjoy or dislike.  
Almost all of that study has been score study, which means that I have, at the absolute minimum, a good internal intuition of what a melody should look like on the page.  
Add to that the fact that I have internalized the formal rules that people have written over the ages for melodies that sound good\footnote{read: I know what normative melody is, which sounds good at the very least by virtue of being familiar to the listener's ears and is singable at the very least by virtue of being in the shape my throat recognizes (throat? is that where the song comes from?)}, and I guess that it should be no surprise that I can quickly jot down something that I like for a single voice.

Harmony is always something that comes later to me, which I attribute in equal parts to my high school choral experience and general love of early music.  
Both of these sources center melody, and have harmony fall out as a consequence, rather than the reverse, as is more common to the academic composer.

In playing guitar, though, most of what I've learned is various folk and folk adjacent\footnote{read: early rock and a lot of punk} songs by looking at chord sheets.  
I have mostly internalized standard variations on the 145 progression, which is certainly not bad, especially since most of what I write is folk and folk adjacent.  
Right now, though, I'm trying to write a piece for solo guitar for a friend's wedding.

In a stroke of inspiration late one night, I found a riff I really liked.  
The warm light of morning showed me it was just a walk up from A to D in tenths, which made figuring out some next options easy enough.  
However, because I do so little work with melody on guitar, I find that nothing I'm doing feels particularly natural.  
I've practiced enough scales that I can look at a tab and follow it, but apparently not enough to have internalized it.

So, what does this mean for me?

I want to work on transcribing it, because I feel like it will help with getting the progression through.  
Right now I feel like what I have is alternating melody and fingerpicked harmony, which is not necessarily a problem.  
Ideally, though, I think that I'd like there to be more of a melodic line throughout, if only because that is familiar.

\section{Draft 0: 2025 May 5}

I've talked a fair amount in the past about how I write music, and especially how I'm writing a piece for a friend's wedding.  
While lying in bed late one night, I awoke with inspiration for the hook for the piece I'm going to play this fall.  
Figuring out what comes next, though, has really been a struggle.

\section{Daily Reflection 5/7}

\begin{enumerate}

\item Top Priorities:

\begin{itemize}

\item Sleep:

\begin{itemize}

\item Keeping sleep time sacred?

I think so! I saw that it was around 9 pm yesterday so put the computer and whatnot away and went to sleep.  
When I woke up between 2 and 4, I didn't get back on my phone, even if I did check my watch for the few notifications it had.

\item Good sleep hygiene?

Eh, I was watching youtube in bed last night.  
Still, didn't use electronics when woke up in the middle of the night, which is a win in and of itself.

\item Sleeping enough?

Yeah! I woke up completely naturally at like 615, and I was unable to keep my eyes closed, so that's a good sign that I'm finally recovering from my sleep deficit.

\item How well rested do I feel?

Decently! I feel pretty good, but I also haven't left my home yet, so it's a little hard to know how much of that is that I'm actually well rested and how much is that I haven't done anything at all.

\end{itemize}

\item Feed myself:

\begin{itemize}

\item Did I eat breakfast?

Yesterday I had some matzah!  
Today I um plan to eat by 1030, because that's where I penciled in food for the day.  
Given how hungry I am now, though, might not be a bad idea to eat earlier.

\item Did I eat a second meal?

\item Did I eat dinner?

Yesterday I did! Had pizza.

\item Water?

Nowhere near enough. My water bottle is nearly empty, so I really should\footnote{want to and am working up the motivation to} refill it.

\end{itemize}

\item Family:

\begin{itemize}

\item Am I neglecting any familial obligations?

Nope!

\end{itemize}

\item Movement:

\begin{itemize}

\item Am I stretching at least 5 minutes per hour of computer time?

I did not stretch for the final like 2 or 3 hours at home, and it was clear that I felt so drained from that fact.

\item Am I generally making efforts to be limber?

I stretched this morning before this, and it felt good. Currently trying to do some foot stretches because they still hurt. I'm hoping it's not stress fractures, but who can say.

\end{itemize}

\item Spirituality:

\begin{itemize}

\item Time for prayer?

Nope.

\item Prayer?

Nope.

\item Time for sacred silence?

Nope.

\item Deep breaths?

Generally actually did ok with this one yesterday! Woo one of four.

\end{itemize}

\end{itemize}

\item Secondary Priorities:

\begin{itemize}

\item Thesis/ Ph.D. work:

\begin{itemize}

\item Keeping up on the writing deadlines?

Eh, I thought that I wanted to do one piece of writing a day and then did not do it yesterday.  
Still, I am generating a fair amount of content, and there's absolutely something to be said for the importance of organizing things before I just work on the project.  
Where the line is between being productive by planning and procrastinating by planning, I'm not completely sure, but I think that I'm still very much on the side of benefiting from planning more.

What are the current things I want to write?

\begin{itemize}

\item Draft of video one of youtube. Internal/nominal due date: Friday 5/9.

\item Science Communication Thesis Chapter Draft. Internal/nominal due date: Friday 5/9.

\item RebelFit Introduction Thesis Chapter Draft. Internal/nominal due date: Friday 5/9.

\item RebelFit Background Thesis Chapter Draft. Internal/nominal due date: Friday 5/9.

\item Publicly Reachable Thesis Chapter Draft. Internal/nominal due date: Friday 5/9. Hmm, can I count the things that I'm doing as an animation for that?  
Maybe.  
Still want to do it as a writing chapter, though, in case the boss isn't keen on the idea.

\item RebelFit Results Thesis Chapter Draft. Internal/nominal due date: Friday 5/9.

\item Background subtraction got weird, because one of the species didn't seem to appear in the spectrum. I realized that I've been working with a small subset of the spectrum, rather than the entire thing, though, which probably isn't helping.\footnote{note to self: that's something we can put in the paper \say{look, even if we have X times more data, linear increase not quadratic or whatnot}}

\end{itemize}

\item Reading the necessary things?

\item Making graphs?

Not at all

\begin{itemize}

\item Visual depiction of Latin Hypercube

\item Visual depiction of Grid Search

\item Visual depiction of random search

\item Visual depiction of Loomis-Wood Diagrams

\item Visual depiction of Spectral Stacking

\item Visual depiction of how the fitness of the spectral stacks is really reliant on the graphs being the right height

\item I guess that the stuff for intro to quantum video counts here.

\item Plots from the actual results of the runs, to make sure that it worked out.\footnote{SSC, AAT, if any vib states were good, what happened to the computations, etc}.

\end{itemize}

\item Organizing citations?

Nope!

\end{itemize}

\item Love:

\begin{itemize}

\item Taking risks?

Oh! Minorly, actually.

\item Making efforts?

Nope

\item Showing affection?

Decently!

\item Being honest?

Yeah!

\item Being open?

Yeah!

\item Being appropriately vulnerable?

I think so

\end{itemize}

\end{itemize}

\item Adjacent to Primary and Secondary:

\begin{itemize}

\item Typing Practice?

Did so this morning. Lots of work on C and Z, with a little on Q at the very very end.

\item Post-its being maintained?

Kind of! Realizing that the way I'm doing post-its might not be ideal.  
Yesterday I did some pseudo-assigned reading for the writing camp I'm attending, and all of the advice seemed to really stress having a good schedule.  
I have a nice bulletin board that I have been trying to figure out how to use.\footnote{as it turns out, I don't really love doing it with my web novel}  
Putting all the notes on it with dates might not be the worst idea!  
I think that I would need to break the time down into the rest of the week, month, and degree? Have three rows, and then spend some time each morning shifting things around? Maybe? Will see how the motivation takes me.

\item Applying to jobs?

Nothing new here from yesterday.

\item Reading the things I think could be good?

Nope! Today I actually scheduled time for it, though, which is probably going to be helpful.\footnote{yes, I do in fact reward myself for reading by going to a burger joint.}

\item Making manim videos?

Nope.

\end{itemize}

\item Cleaning?

\begin{itemize}

\item Office

Made sure it was good before I left yesterday, and will not be in today, so hoping to have it remain good.

\item Home

Scheduled time to do so today. I don't entirely know what it would mean for me to be all caught up with cleaning, but I think that, at the very least, it would involve vacuuming?

\item Car

I'm finally going to take the bulletin board out! That's something.

\item Computer

Not really, though not sure if I really need it.

\item Other as needed

\end{itemize}

\item External Obligations:

\begin{itemize}

\item Guitar for wedding?

Scheduled time to transcribe it today, and will finish the musing about it above.

\item Travel plans?

Going to call about one hotel just after this.

\item Talks for parks?

Two more have been scheduled.

\item Other requested talks?

\item Talks for conferences?

\end{itemize}

\item Tertiary Goals:

\begin{itemize}

\item Blogging?

I'm finally going to post this one.

\item Reading?

Nope, but I think that part of it is that I'm not reading anything worthwhile right now.\footnote{meaning that like none of the books I keep on my phone are dragging me in, and otherwise I don't keep books nearby enough}

\item Web Noveling?

No, but I would like to get back into it.  
Thankfully, the bulletin board\footnote{the more times I type it, the fewer attempts it takes to spell correctly. Also, yes I'm hoping that the repetition makes me actually get it} has my notes from before giving up.

\item Guitar?

Spent a solid few minutes last night working through a few exercises.  
I'm getting better at reading staff notation for the guitar, which feels really nice, if also more than a little strange.

\item Other hobbies?

Not so much.

\end{itemize}

\item Quaternary Goals:

\begin{itemize}

\item Letter writing

I'm thinking about bringing the books that I want to be reading to my cage, so that I can have more of a reason to visit it.  
There's something to be said about scheduling a default day which involves time there.  
I just really wish that it was open before 830 AM.  
Tragically, however, it appears that the building is the only space that has them available.  
Still, the 24/7 library is only like a block away, so if I started the morning there, could always shift to the library with my things later in the morning.\footnote{another advantage of bulletin board is that I can quickly shift things around. Then again, the same is true of floor time. Might need to have floor time, especially if I finish this post earlier than expected}

\item Handwriting/penmanship

Yeah! I have generally been handwriting things, even if I haven't been doing the loops and lines as much.  
The letters are getting gradually more even, and in general I think that I'm liking how the penmanship is developing.  
It is still really weird to be writing in lower case letters, and I do still catch myself returning to block capitals, but.

\item Picking a new signature

\end{itemize}

\end{enumerate}

\section{Daily Reflection 5/6}

\begin{enumerate}

\item Top Priorities:

\begin{itemize}

\item Sleep:

\begin{itemize}

\item Keeping sleep time sacred?

Yeah! I was spending time with a friend last night, so it could have easily spiraled out.  
Luckily they also respect their rest, and so I was politely encouraged to leave at an appropriate hour.  
I then went to sleep.

\item Good sleep hygiene?

Yeah! Went to sleep! Didn't do bed rot.

\item Sleeping enough?

About another 11 hours last night, which I hope brings me closer to a healthy level of total rest.

\item How well rested do I feel?

Generally ok! Still hurts to walk, but that's really the only thing slowing me down right now.

\end{itemize}

\item Feed myself:

\begin{itemize}

\item Did I eat breakfast?

Yesterday! Today I'm planning to eat it later.

Oh! I also had a few chocolates this morning. Look at me, feeding myself.

\item Did I eat a second meal?

Yesterday! I had dinner with friends.

\item Did I eat dinner?

Dinner was two full meals worth, so I'll count it double.

\item Water?

Generally ok?  
I think that I went through a full two liters of tea yesterday. Today I'm hitting a nice 15 ounces of espresso, which will hopefully both improve my productivity and desire to consume water.  
I do really prefer cool to cold water over room temperature, which is good for me to know, even if it not particularly actionable.  
I think that my electrolytes might be generally out of whack lately, which might mean that I shouldn't be chugging water so much as making sure that I also maybe like eat teaspoons of salt with them?

\end{itemize}

\item Family:

\begin{itemize}

\item Am I neglecting any familial obligations?

Nope! Need to listen to an album sometime this week, but will find space and time for that, I'm sure.

\end{itemize}

\item Movement:

\begin{itemize}

\item Am I stretching at least 5 minutes per hour of computer time?

Generally did ok with this yesterday, though I did ignore the last two or three hours of yesterday's time.  
That's not great, but also like I frame the activity much more as taking a break from work than actually stretching, which is probably wrong, but does explain at least part of it, to me at least.

\item Am I generally making efforts to be limber?

Yeah! Even when just sitting down, finding that I want to move more and more is probably good.  
Also like wow my shoulders are tight, and so did some stretching while walking. That felt nice, and so I'll probably spend more time today focusing on feet and shoulders, since those are my big areas of note\footnote{I'm currently operating under \say{if a part of me is easily forgotten, it's probably not an issue right now} school of recovery}.

\end{itemize}

\item Spirituality:

\begin{itemize}

\item Time for prayer?

Not really, kind of filling my day.

\item Prayer?

As a result, not really.

\item Time for sacred silence?

Nope! Wow I have done nothing with intentional quiet.

\item Deep breaths?

Not really, which I don't like in the slightest.  
Maybe I should remind myself with the stretching to also breathe.

\end{itemize}

\end{itemize}

\item Secondary Priorities:

\begin{itemize}

\item Thesis/ Ph.D. work:

\begin{itemize}

\item Keeping up on the writing deadlines?

What are the current things I want to write?  
Oh boy are there a lot of them, and I think that the goal is going to be the rebelfit intro/background, since I'm not sure how what I write is going to end up.  
Mostly going to focus on the whole \say{this is a list of algorithms I didn't use} for now.

Yesterday I also said that I wanted to get the new jobs resubmitted.  
Part of me wants to wait on that, but I think that's just the part of me that is terrified of success.  
Going on the list!\footnote{in addition to footnotes, I think that I'm also going to have a list of tasks for a day? I don't really know what I need to do to make my life ordered and functional, short of maybe like getting a secretary.}

\begin{itemize}

\item Draft of video one of youtube. Internal/nominal due date: Friday 5/9.

Would also be good to have this? Will spend time on it if I have extra energy after finishing the draft and submitting the jobs.

\item Science Communication Thesis Chapter Draft. Internal/nominal due date: Friday 5/9.

\item RebelFit Introduction Thesis Chapter Draft. Internal/nominal due date: Friday 5/9.

\item RebelFit Background Thesis Chapter Draft. Internal/nominal due date: Friday 5/9.

I really don't know if I can adequately differentiate this from the above, but hopefully it will clarify itself as I write.

\item Publicly Reachable Thesis Chapter Draft. Internal/nominal due date: Friday 5/9. Hmm, can I count the things that I'm doing as an animation for that?  
Maybe.  
Still want to do it as a writing chapter, though, in case the boss isn't keen on the idea.

\item RebelFit Results Thesis Chapter Draft. Internal/nominal due date: Friday 5/9.

\item I guess that I should probably also put here: \say{Do the background species subtraction and resubmit those jobs}, which I will nominally deadline at tomorrow, 5/6, because I think that I should do that today.

\item Introduction to Spectroscopy, especially rotational spectroscopy, both as classical and quantum framed. Let's say due 5/7.

I did this one yesterday! That means that it gets to be removed from the list! Woo!

\end{itemize}

\item Reading the necessary things?

I think that I don't really need to be reading (read: I should not be reading)\footnote{why was that a parenthetical and not a footnote? great question. Do I think that my boss will accept the footnotes that I'm leaving in my drafts for now? Probably not. Am I going to continue adding them until she explicitly tells me to remove them? Also yes}  
I should reorganize the Zotero, but that's not a high importance or urgency task.\footnote{I'm trying to consider tasks by urgency and overall importance when prioritizing. Unfortunately, everything is kind of either yes urgent or no urgent (is that just how my mind works?), so that's not super helpful, because the yes urgents are almost always yes important}

\item Making graphs?

\begin{itemize}

\item Visual depiction of Latin Hypercube

\item Visual depiction of Grid Search

\item Visual depiction of random search

\item Visual depiction of Loomis-Wood Diagrams

\item Visual depiction of Spectral Stacking

\item Visual depiction of how the fitness of the spectral stacks is really reliant on the graphs being the right height

\item I guess that the stuff for intro to quantum video counts here.

\item Ope! Plots from the actual results of the runs, to make sure that it worked out.\footnote{SSC, AAT, if any vib states were good, what happened to the computations, etc}. Post-it time.

\end{itemize}

\item Organizing citations?

Hah.

\end{itemize}

\item Love:

\begin{itemize}

\item Taking risks?

Kind of! The very slightest of risk yesterday, and it went well, which is, in retrospect, unsurprising.

\item Making efforts?

I think so! At the very least remembering what it means to make efforts.

\item Showing affection?

Yeah! I think appropriate amounts no less.

\item Being honest?

Generally! It was really weird to write with my friend this morning and find that they were so willing to be helpful.\footnote{I often comment things like \say{alas I do not have my backpack} as a way of externalizing that I need to get the backpack at some point. Friend just got the backpack, which I think felt bad to me because it made me feel like I was being passive aggressive?}

\item Being open?

Yeah! I think it's funny that I separate these two, because one would hope that they're synonymous, and in my mind they certainly approach it.  
However, there's some part of me that reacts to the words differently, so I'll leave it for now.

\item Being appropriately vulnerable?

Yeah!

\end{itemize}

\end{itemize}

\item Adjacent to Primary and Secondary:

\begin{itemize}

\item Typing Practice?

Will do now!  
Lots of work on the letter q. Sadly, the letter C was again an issue, and I even had a few lessons with Z. Generally looks faster than yesterday, which is nice.  
Perhaps unsurprisingly, focusing on the letter q did mean that my relative accuracy for it went down. I'm sure it'll rise up again as I keep working, but for now it is somewhat low.

\item Post-its being maintained?

I think so! Lost the list of post its from yesterday, which is a little sad and scary, but I'm sure they will turn up when I return to the office.

\item Applying to jobs?

Set up two meetings with the career office, one this Thursday so I can ask what should go in the materials and one next week so I can make sure that what I've put in them is good!

\item Reading the things I think could be good?

Nope!

\item Making manim videos?

Nope!

\end{itemize}

\item Cleaning?

\begin{itemize}

\item Office

Spent a good twenty or so minutes yesterday doing so, felt nice, meant that I could work at my desk. It feels like all the work has disappeared since then.  
The curse of entropy rears its head again.

\item Home

I did not get home until bed time last night, so did not. Today I will try to make time after work!

\item Car

none

\item Computer

Not really, but will plan to do some today. Oof I get a \textbf{lot} of emails that actually require responses.

\item Other as needed

Don't think any right now!

\end{itemize}

\item External Obligations:

\begin{itemize}

\item Guitar for wedding?

Subject of the above musing!

\item Travel plans?

Have reached out to some people!

\item Talks for parks?

Have signed up for more!

\item Other requested talks?

Nope!

\item Talks for conferences?

Nope!

\end{itemize}

\item Tertiary Goals:\footnote{mmmm off by N numbering}

\begin{itemize}

\item Blogging?

Here!

\item Reading?

Not really, but I would like to. I just always feel like I have too little time\footnote{read: the time flies by in some ineffable way. How has it been more than an hour that I've been working and yet only 30 minutes on this document?}

\item Web Noveling?

Nope! Thinking about it, and trying to remember where I last left off, though.

\item Guitar?

Noodled a bit this morning. As I figure out my schedule, will try to be better at actually practicing.

\item Other hobbies?

Going to set aside some time today to work on poetry/song lyric.\footnote{how do I distinguish them, you might ask? Really it's that I'm willing to break meter and flow and anything else much more if it's for a song, especially if I think that the melody or harmony requires it (probably will be in this musing)}

\end{itemize}

\item Quaternary Goals:

\begin{itemize}

\item Letter writing

Nope!

\item Handwriting/penmanship

Yeah! I did some work yesterday, and today I'm working on it in the sense of thinking about what I write as I do.  
I think that I do really want to keep the capital A that I've had since college, since I have actively been complimented on it before.  
I'm not totally sure if it fits the vibe of the other letters that I'm making, though, so that's something to certainly consider.

\item Picking a new signature

None.

\end{itemize}

\end{enumerate}

\section{Daily Reflection 5/5}

\begin{enumerate}

\item Top Priorities:

\begin{itemize}

\item Sleep:

\begin{itemize}

\item Keeping sleep time sacred?

OOf.

I went on a long walk/camping trip.  
Both because we had early mornings and because no one else in my life loves early bedtimes, I did not do great with that over the Friday and Saturday nights.  
Last night, however, I think I went to bed or sleep by 8, and I do feel significantly more human.

\item Good sleep hygiene?

Generally! I think at least.  
One plus side of camping, there is maximal incentive to minimize time in sleeping bag when not sleeping.

\item Sleeping enough?

Uhhhhh.

No, I don't think, but also like that may just be physical body muscle repair rather than actual need for sleep.\footnote{read: rest is not sleep}

\item How well rested do I feel?

Generally ok.  
Woke up feeling well rested this morning and ran out of energy halfway to work.

\end{itemize}

\item Feed myself:

\begin{itemize}

\item Did I eat breakfast?

On it right now! Ate something approximating breakfast on Saturday and Sunday.

\item Did I eat a second meal?

On Friday, Saturday, and Sunday, yes!

\item Did I eat dinner?

I slept through dinner last night, other wise yesish.

\item Water?

Nowhere near enough, but much more than normal.\footnote{as it turns out, when walking all day, one needs to consume more, not less, liquid}

\end{itemize}

\item Family:

\begin{itemize}

\item Am I neglecting any familial obligations?

I don't think so!  
Generally doing ok as far as I can tell.

\end{itemize}

\item Movement:

\begin{itemize}

\item Am I stretching at least 5 minutes per hour of computer time?

Not yesterday, but in general yes.  
Today I'm absolutely going to try to.

Also, realized that I do tend to do a fair amount of foot stretching while I sit at my desk, especially today, where my feet are in incredible pain.

\item Am I generally making efforts to be limber?

With how tight my entire body is right now, absolutely, even though it does feel like I'm tighter than ever before.

\end{itemize}

\item Spirituality:

\begin{itemize}

\item Time for prayer?

Yeah! As a group we decided to do the four sets of rosary mysteries during the walk, so 200 Aves.

\item Prayer?

As above.

\item Time for sacred silence?

Ehhhhhhhh. One plus side of being with a group is that there's no need for silence.  
I'll try to get back into it today, though, because wow I need to feel grounded again.

\item Deep breaths?

I think so! Certainly right now as I remember.

\end{itemize}

\end{itemize}

\item Secondary Priorities:

\begin{itemize}

\item Thesis/ Ph.D. work:

\begin{itemize}

\item Keeping up on the writing deadlines?

I sent in a paper draft and some chapter drafts in last week.  
What are the current things I want to write?

\begin{itemize}

\item Draft of video one of youtube. Internal/nominal due date: Friday 5/9.

\item Science Communication Thesis Chapter Draft. Internal/nominal due date: Friday 5/9.

\item RebelFit Introduction Thesis Chapter Draft. Internal/nominal due date: Friday 5/9.

\item RebelFit Background Thesis Chapter Draft. Internal/nominal due date: Friday 5/9.

\item Publicly Reachable Thesis Chapter Draft. Internal/nominal due date: Friday 5/9. Hmm, can I count the things that I'm doing as an animation for that?  
Maybe.  
Still want to do it as a writing chapter, though, in case the boss isn't keen on the idea.

\item RebelFit Results Thesis Chapter Draft. Internal/nominal due date: Friday 5/9.

\item I guess that I should probably also put here: \say{Do the background species subtraction and resubmit those jobs}, which I will nominally deadline at tomorrow, 5/6, because I think that I should do that today.\footnote{time for a post it}

\item Introduction to Spectroscopy, especially rotational spectroscopy, both as classical and quantum framed. Let's say due 5/7.

\end{itemize}

\item Reading the necessary things?

Don't think that there's a ton to read other than the spectroscopy and quantum books insofar as they can help me with the videos and background.

\item Making graphs?

I need to make a lot of graphs.  
What, though?\footnote{am I using this as my way of also getting things onto a page so that it's easier for me when it comes time to post it? Yes, absolutely. Am I also using this as a form of productive procrastination? Yes, absolutely. Is this also incredibly grounding after a long weekend? So absolutely and incredibly yes}

\begin{itemize}

\item Visual depiction of Latin Hypercube

\item Visual depiction of Grid Search

\item Visual depiction of random search\footnote{After writing the next two lines, realized that I hadn't capitalized}

\item Visual depiction of Loomis-Wood Diagrams\footnote{had to look up the term I've been using real quick. Saying diagram instead of plot or graph makes more sense to me}

\item Visual depiction of Spectral Stacking\footnote{does this need to be capitalized? Who can say?}

\item Visual depiction of how the fitness of the spectral stacks is really reliant on the graphs being the right height

\item I guess that the stuff for intro to quantum video counts here.

\end{itemize}

For now, not going to put deadlines, because don't really think that any are necessarily essential\footnote{that phrasing feels fundamentally wrong in some hard to pin down way}

\item Organizing citations?

I have not made progress on this, but would like to do so today.  
At the very least, as I read the rotational and quantum books, make sure that their info is in a new zotero folder.\footnote{which I'm going to shift all useful things into after making sure that I have good BibTeX keys for all of them}

\end{itemize}

\item Love:

\begin{itemize}

\item Taking risks?

Not really any, though I guess that I did try to set up meetings, even if not romantic, with people.

\item Making efforts?

As above, though also tried to get myself back into the habit of good/intense\footnote{are these the same? probably not, though from a flirtation standpoint, I feel like generally better to err on the side of more, not less}

\item Showing affection?

Oof, not great here.  
I forgot to ask any of the people I was with this weekend their thoughts on being called love or beloved, and I forgot to ask them how they feel about physical touch.

\item Being honest?

I think so! I tried to not deflect, at least.  
I did also go to confession for the first time since my mom's funeral\footnote{oof that's rough to say}, which was really hard for reasons I don't know if I can adequately explain here/want to.

\item Being open?

I think so! I was trying to be receptive to others.  
I had good conversations with strangers during the walk.

\item Being appropriately vulnerable?

I don't think that I was vulnerable appropriately, but it's a process, not a one off thing, so that's fine.

\end{itemize}

\end{itemize}

\item Adjacent to Primary and Secondary:

\begin{itemize}

\item Typing Practice?

Not at all, let's real quick do that now.

Generally a fair bit slower and less accurate today, but progress was absolutely made.  
I felt myself becoming smoother with each lesson, and even though it was exclusively p and w lessons, I think that the remaining keys got some good practice. Perhaps unsurprisingly, as I sit better\footnote{read: straighter}, I type both faster and more accurately. I should really\footnote{should here meaning think that I would benefit and enjoy more} go back to the external monitor, because wow is the laptop screen too low for me.

\item Post-its being maintained?

Will do just after I do my hourly stretch now!

Before next hour I should look up how to stretch foot, because wow it was cramping continuously.\footnote{opening a tab for that now, even though it does not get a post it}  
Oof there's a lot of them, and it's starting to feel overwhelming. Might want to make more categories than \say{now}, \say{not now}, \say{low priority}, or at the very least figure out what I can do.

\item Applying to jobs?

Nope! I will make a note for that though.\footnote{read: the note will say figure out what is needed to apply for a lecturing job and make a timeline of when jobs are closing. Ok that's two notes}

\item Reading the things I think could be good?

Haven't made much progress there, would like to spend some of tonight or tomorrow doing that, depending on how much work I am able to get done.

\item Making manim videos?

Nope! But I did spend a fair amount of time on Friday thinking about how to frame the series and where to start it.  
I think that \say{an intuitive explanation of quantum chemistry} feels like a lovely title, assuming that I can make it true.  
Luckily, someone else in my life expressed that they would love to watch it, not just because they support me, but also because they think that the information could be generally useful.

Also probably want to have something like \say{market research} meaning looking at the animations and videography I like and dissecting it.  
I'll add a post it to the dailies.

\end{itemize}

\item Cleaning?

\begin{itemize}

\item Office

Not yet, but have it on the list to do today. I'll do after the 10 AM Stretching break.\footnote{read, the one that ends the 10 hour}

\item Home

Negative amounts. I came in from the walk yesterday, dropped everything down, stripped, and collapsed in bed.  
Good things to do this week for sure though.

\item Car

I have camping stuff in the trunk and not camping stuff not in the trunk, so it's at least more ordered!

\item Computer

Nope! Will do so today after the 11am stretch.\footnote{ooh I guess post its don't have to go in the binder, I can keep them by me so I know what I said I was going to do when! Maybe? or hmmm idk.}

\item Other as needed

Generally I think that's everything.  
I guess \say{go to confession} counts as cleaning the soul!

\end{itemize}

\item External Obligations:

\begin{itemize}

\item Guitar for wedding?

I have more ideas for the piece, and plan to start transcribing it, because that should help with the composition.\footnote{entirely because I'm just noodling around a chord progression and understand melody lines better on a score than on the guitar (a musing? Today's musing? yeah ok}

\item Travel plans?

Nope! Get hotel room is officially high priority, and message each of four groups of friends becomes medium.

\item Talks for parks?

Nope! I think that working on the animation for manim would be a good start, though, and at worst I'm comfortable with last year's slides.

\item Other requested talks?

Nope! Oh shoot, I should really work on the \say{how to read science news} talk that was requested.\footnote{read: message the team and ask what they're looking for}

\item Talks for conferences?

As I write and make the graphs I think that I'll have a lot of this.

\end{itemize}

\item Tertiary Goals:\footnote{mmmm off by N numbering}

\begin{itemize}

\item Blogging?

Wow look at this, which I don't really know what I'm going to write for the actual text of the post, but I have spent fully 2000 words figuring out the rest of my mind on.

\item Reading?

Nope! I did read the very slightest bit of the book on my phone, and I finished the last chapters of the volume of a web novel my brother and I churn through.

\item Web Noveling?

Wrote last week's chapter, have yet to write this week's.

\item Guitar?

Some! Did a little with the whole \say{learn the actual instrument through the book}, but only the very slightest bit. Probably would be good and healthy and personally helpful for me to schedule time for it.

\item Other hobbies?

Nope! I don't know if I have other hobbies right now, or if I really want them.  
I guess that I was chatting with someone about song writing over the weekend, so could be good to return to that.\footnote{she also mostly does Christian music, and I have never really done music that connects to my faith, so that will probably be reallly helpful for me.}

\end{itemize}

\item Quaternary Goals:

\begin{itemize}

\item Letter writing

Nope! I will figure out where people live as I plan trips to them, though, which is part of the requirement to sending letters.

\item Handwriting/penmanship

A little! Yesterday at the cafe we went to before going home, everyone was very entranced with my pens, which was a fun and weird situation.

\item Picking a new signature

Not at al.

\end{itemize}

\end{enumerate}

\end{document}