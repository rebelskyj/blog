\documentclass[12pt]{article}[titlepage]
\newcommand{\say}[1]{``#1''}
\newcommand{\nsay}[1]{`#1'}
\usepackage{endnotes}
\newcommand{\B}{\backslash{}}
\renewcommand{\,}{\textsuperscript{,}}
\usepackage{setspace}
\usepackage{tipa}
\usepackage{hyperref}
\usepackage{nested}
\begin{document}
\doublespacing
\section{\href{beetlejuice-review.html}{Beetlejuice Review}}
First Published: 2024 January 14 (because I forgot to post, not bc forgot to write.)

\section{Draft 1}
Hi!

You may have noticed that I've been gone for the past few days.
Sorry about that, the beginning of a new year\foott{1}{not that the year doesn't begin on the first, but I just recently made it back to where I go to school, which feels like a beginning in itself} always comes with a lot of upheaval, and a new invigoration for the research I do does mean that my energies are focused there rather than on this blog.
As a result, I not only have not been writing this blog, but I also have not been working on the Saturday musings that I wanted to do.
So, today we'll be going back to old school blog: reviewing a show I just saw.

I just got back from a matinee performance of the musical Beetlejuice.
Unlike most musicals that I've been to\foott{2}{at least recently\endnotemark[3]}\endnotetext[3]{recently meaning the past almost decade, now that I think about it. I still think of high school as a somewhat recent phenomenon, but it was almost half a life ago wow}, I went into the show today with absolutely no idea what to expect.
What do I mean by that?

Most of the time, I have listened to the entire soundtrack of a musical almost ad nauseum before seeing it performed.
It is always fascinating to listen to the difference between a cast recording and a different cast\foott{4}{usually, at least} performing the same songs.
More than that, though, most of the time I try to be familiar with the show that I am about to see.
I read up on the show, and I try to get a sense of what is going to happen within it.

I think that I can name on one hand the things that I knew about this show going into it.
It was based on a 20th century\foott{I love that I can describe movies this way and it's somewhat meaningful of a distinction now} Tim Burton film.
There's a guy named Beetlejuice\foott{though in actual show, it's only ever spelled like the star, so I guess I'm not totally sure which is correct}, and he wants a girl to say his name three times.\foott{opposite of Rumplestiltskin I guessed. I was somewhat right}
Oh gosh that's far less than one hand.\foott{or at least it is for me.}
Those two facts were both true, and they both remain true.
The show, however, is so much more than that.

It opens at a funeral, where we meet our main protagonist: Lydia Deetz.\foott{For those who know I struggle with names, yes I needed to look that up just now, and no I did not get her name before the first act ended. Thankfully the cast was small enough that it didn't matter}
She sings a ballad about how invisible it feels to be sad.
As the song ends, the show begins in earnest, as Beetlejuice comes out to break the fourth wall and explain the premise of the musical.
It is clear that the entire show is meant to be a parody of many things, the classic musical form amongst them.

And yet, the show treats everything it does with a shocking amount of love and care.
The sets and scenery were absolutely stunning.
Sorry, I'm going to take a step back from the plot to just gush about the technical aspects of the show.

As the show opens, an animation begins to play on the backdrop.
I had never really thought about the options that using projected back sets could have, but it was mesmerizing watching the way that a clouded moon and trees could be drawn in, starting with what seemed like a single shaky hand.
The technical mastery remained in full force throughout the rest of the show.
Each room of the house is built with multiple forced perspectives, making the entire home feel disorienting.
The lights were completely invisible, drawing no attention to themselves, only to suddenly spring out at the audience when the parody sections of the show came back.
The blend between set and backdrop was also amazing.
Anyways, that's lighting and set, the two parts of technical theater that I care about.

The show is so obviously a parody, but one made in good faith.
The classic Broadway musical solo where the lead woman gets to play the role of a classic prima donna\foott{5}{in the operatic sense of the lead soloist, not in its new connotations} is played up, and the actress playing Lydia shone in all of those moments.
The show, for all that some of its jokes fell a little flat\foott{6}{I've never been a huge fan of the like \say{tee hee look how dumb the right in America is at this specific moment in time that we're writing this show that's nominally supposed to take place in the nebulous American now} thing that a lot of shows have taken to doing.}, never ceased acting earnest.

I'm sure there's more I could say about the show,\foott{7}{mostly about how absolutely incredible the lead actress was, if I'm being honest. She was absolutely stunning, and her voice was so emotional through every emotion that the\endnotemark[8] teenage character feels so strongly}\endnotetext[8]{I think? I'm assuming that she's meant to be a teenager, both because of the climax song and the fact that the attitude is general \say{theatre idea of a teenager}} but anything else would really start falling into the realm of spoilers, which I don't always love in my reviews.
Anyways, I am so glad that my new research group mate told me about the show and that I ended up going today.

Oh!
I was just about to say pulling back from the review, but that is one thing that I want to discuss.
The fourth wall is a tenuous concept in art, and one of the biggest trends I've noticed in my life is the general dissolution of it in most media.
One way that's often manifested in musicals is through this idea that I can't remember the name of that I encountered back when I studied music.
The general idea, though, is that music in a show is either part of the world of the show, where the characters are aware that they're singing and dancing, or something strictly for the audience.

In the early parts of the show, it is clear that the music coming from everyone exists in the real world, not in the world of the show.
Beetlejuice is the only one to directly address the audience, or whose music even seems directed outward.
As he is complaining about the people next to him, however, they mention that they can hear him.
At that moment, the fourth wall, which I thought had been shattered, reasserts itself.
We are once more separated from the antagonist, who is once again brought into the show.

As the show continues, the wall starts falling down more and more, and the classic scenes where groups come out and pose for the audience are numerous.
When the curtain call comes, it is hard to know whether the event is taking place within the universe of the show or outside of it.

I also really liked that the show had likeable characters.
I find that one issue I have with a lot of musicals lately is that we aren't really supposed to root for anyone, and nobody really grows.
Everyone in this cast, however, has a plausible\foott{9}{within the realm of like \say{sure, ghosts and demons would have no bigger concern than hanging out in a house, and only moody teenage girls can see them}. Plausible here is really meaning like internally consistent} and sympathetic goal, and everyone changes throughout the show to become someone more likeable.\foott{10}{obvious exceptions are the characters who exist for a scene or two and really mostly exist as scene dressing}
It was nice to be able to root for the main character, and even for the antagonist, to some extent.

Now pulling back from the telos\foott{10}{within a general meaning of the term} for this specific post, and moving to the telos of posting generally, I have been more lax about writing my daily reflections on days that I do not blog.
I wish that wasn't\foott{11}{weren't?} true, but alas, it has been.
With that in mind, though, I do think that my daily reflections are best saved for my private notes, which I have been keeping more regularly.
If anyone is particularly interested in reading them, I could be easily convinced to bring them back.


\endnotes{}
\end{document}