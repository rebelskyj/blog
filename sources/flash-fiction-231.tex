\documentclass[12pt]{article}[titlepage]
\newcommand{\say}[1]{``#1''}
\newcommand{\nsay}[1]{`#1'}
\usepackage{endnotes}
\newcommand{\1}{\={a}}
\newcommand{\2}{\={e}}
\newcommand{\3}{\={\i}}
\newcommand{\4}{\=o}
\newcommand{\5}{\=u}
\newcommand{\6}{\={A}}
\newcommand{\B}{\backslash{}}
\renewcommand{\,}{\textsuperscript{,}}
\usepackage{setspace}
\usepackage{tipa}
\usepackage{hyperref}
\begin{document}
\doublespacing
\section{\href{flash-fiction-231.html}{Flash Fiction Friday}}
First Published: 2023 December 15

\section{Draft 1}
It's another Friday,\footnote{more like Fri-yay, am I right?} which means it's time for another installment in Flash Fiction Friday Fthinking.\footnote{I'm sure that I could have come up with a better word, but it's still early and I'm developing a burgeoning caffeine addiction}
The prompt this week is \say{A promise to break}.
The prompt givers implied that it was a promise which would be broken, but that is not where my mind went.

I'm immediately reminded of a line in Seanan McGuire's Middlegame, which I don't remember perfectly, and so will not attempt to quote.
The gist of the line, though, is about compelling someone to continue, rather than stop.
I'm not totally sure how that reminded me of the prompt, except that I read it as a promise to break something.

Breaking is an inherently dynamic action, which makes it good for me, a person who struggles to have plots, especially in flash fiction.
So, if we assume that there's a promise which involves breaking, what does that require?

On some level, it requires two entities.
I suppose that one can, in theory, make a promise to oneself, but that feels not like the story I want to tell here.

I'm debating between realism, fantasy, and sci fi.
If scifi, there's the option of like \say{when the barrier falls do X}, which is similar for fantasy.
Actually, the FFF I've written since coming back from hiatus have all been shades of realism.
Let's lean into the fantastic, just for a day.

So, fantasy.
What's breaking, what's the promise?

One element of fantasy I love is how binding promises are.

Given that it's now very late out, I find that I want to write a sonnet to respond to the prompt, because I don't really have the energy\footnote{or, quite frankly, the time left in the day} to do a story, the rest of this blog, and muse.
As I walked home, though, I ran into an issue.
When I think of fantasy poetry, I think of ballads and ballad form, which is in triple meter.
I am currently in a sonnet phase, which requires iambic pentameter.
Still, there's nothing saying I can't just go for it.
Will report back when finished.

So, it wasn't the best sonnet that I've ever written, but it certainly wasn't the worst either.
All in all, I think that I'm actually pretty happy with it.
I managed to tell a narrative, and that's really what's important.

Now, readers of the blog might wonder what kept me from writing all day.
In order, there was a department party, a secret santa\footnote{officially snowflake, but we'll ignore that} reveal, a thesis defense and accompanying celebrations, and then two parties.
Each lasted just a little bit longer\footnote{or, in the case of the second party, quite a bit longer} than I was expecting, but I don't regret the time at all.
Time with people is never time wasted.

Daily Reflection:
\begin{itemize}
\item Hobbies:
\begin{itemize}
\item Did I embroider today? I completely forgot to take the embroidery home. I might need to go into the office tomorrow, though, so I could pick it up then.
\item Did I play guitar today? The smallest amount. I did also play penny whistle, though, which was fun and cool, if a little embarrassing.\footnote{because I did it in public both times, the second of which was at the second party, where everyone gathered around and sang silent night}
\item Did I practice touch typing today? I need to get back to this, but not tonight.
\end{itemize}
\item Reading
\begin{itemize}
\item Have I made progress on my Currently Reading Shelf? I am almost finished with the audiobook. I hope that we're past the hardest sections, but I don't honestly know if we are.
\item Did I read the book on craft? Shoot! Once again, the day has passed me by. For all that I have these daily reflections, I don't know how much they actually change my day to day habits.
\item Have I read the library books? I renewed my books, which is a step in the direction of believing that I'll read them. Maybe I have to return two books a month minus however many I read starting next month? That would make me sad but might make me read more books. At the very least, I'd decrease my bookshelf by 2 a month.
\end{itemize}
\item Writing
\begin{itemize}
\item Did I write a sonnet? Yesterday I actually liked the sonnet I wrote. Today's was pretty decent
\item Did I revise a sonnet? Nope
\item Did I blog? I'm not thrilled with this, but yes.
\item Did I write ahead on Jeb? Nope!
\item Letter to friends? Still no
\item Paper? I started redrafting the changes that I have to follow to make the code work in prose form.\footnote{there has to be a better way to say that, but I cannot think of it right now, I'm sorry}
\end{itemize}
\item Wellness
\begin{itemize}
\item How well did I pray? Bad.
\item Did I clean my space? Yeah! It's nicer now, which was the goal.
\item Did I spend my time well? Not really. I suppose that I'm recovering from illness, so some of that is to be expected. Still, I feel like I could have done better.
\item Did I stretch? I really fell off this train.
\item Did I exercise? See above.
\item Water? I drank so much water today. The nicest part of illness is how much it encourages me to drink water.
\end{itemize}
\end{itemize}\end{document}