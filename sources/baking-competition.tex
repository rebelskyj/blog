\documentclass[12pt]{article}[titlepage]
\newcommand{\say}[1]{``#1''}
\newcommand{\nsay}[1]{`#1'}
\usepackage{endnotes}
\newcommand{\1}{\={a}}
\newcommand{\2}{\={e}}
\newcommand{\3}{\={\i}}
\newcommand{\4}{\=o}
\newcommand{\5}{\=u}
\newcommand{\6}{\={A}}
\newcommand{\B}{\backslash{}}
\renewcommand{\,}{\textsuperscript{,}}
\usepackage{setspace}
\usepackage{tipa}
\usepackage{hyperref}
\begin{document}
\doublespacing
\section{\href{baking-competition.html}{Baking Competition}}
First Published: 2018 November 23
\section{Draft 2 (23 November)}
As I've mentioned time and time again, I enjoy making food, and I enjoy competing.
Somehow\footnote{probably because I'm horrible with secrets} my coaches throughout life have found out these two facts.
This is the first time that a coach has connected the two, however.

So, in a little over 6 weeks,\footnote{I think. I refuse to check} I'll be competing against nominally the entire diving squad, though, like many diving meets, it looks like most people aren't there to win.
Therefore, I need a winning recipe.

Baking competitions are a lot like diving.
A perfectly baked cookie will bring in less points than a good cake, because the cake is harder.
So, I need a recipe that balances ease, apparent difficulty,\footnote{I dream of a 5211A equivalent} reproducibility, taste, and appearance.
If I didn't care about the apparent difficulty, I'd likely just make my toffee chip chocolate cookies.
But, they seem too easy.

So, I think I'll try to make \href{http://catcora.com/recipes/desserts/chocolate-budino/}{\say{Chocolate Budino}} or, at least a variation on that recipe.
Now, like many people, I found out that youtube\footnote{this is a noun that isn't a proper noun} cooking videos exist, and fell in love with them.
They have some interesting tips for how to improve chocolate baked goods, which I'll likely try to incorporate.
But, this is definitely something I shouldn't put as much internal pressure on myself as I have.

\section{Draft 1 (21 Nov)}
I was informed last night that in a little over a month, I'll be participating in a baking competition.
Now, as I've mentioned more than once, I like to make food.
I've also mentioned that I like competition, so this should be a great thing for me.

But, here comes the hard part: what do I make?
My initial idea was to make my chocolate toffee cookies,\footnote{as mentioned in \href{toffee-recipe.html}{this post}} but with a few modifications.

I've begun watching youtube food videos, and I understand wholly why people watch them.
It's super entertaining and informative.
A few highlights from what I've learned is that you can brown butter to add more flavour\footnote{yes I'm British now, deal with it} to food, and that you can add espresso powder to chocolate foods to make them taste more deep.

So, I'd planned on browning the butter in the cookies,\footnote{not in the toffee, since that's already browned} and adding some coffee in as well.
However, I also realized that I had chicory coffee at home, and wondered if that might not be a better profile, since it would be a little nuttier.

But, as a wise friend helped me realize \say{Good cookies are good. Great cookies are good. Awesome cookies are great.}
What he meant was that people tend not to discriminate too finely between different levels of cookie.
I agreed, and so thought about what else I could make.

Bread was eliminated for much the same reason, because there's not a huge difference in how a normal person perceives a good to great loaf, and I don't know that I can make an excellent loaf 100\% of the time.\footnote{especially since I'll be in a scary kitchen}
So, since the challenge began with someone's banana bread recipe, it was suggested I do that.

But, banana bread for me has always just been a good comfort food.
So, like with cookies, it's not something that can really show how amazing I am at baking.

Then, I remembered that I used to make a recipe called \href{http://catcora.com/recipes/desserts/chocolate-budino/}{\say{Chocolate Budino.}}
It's effectively a chocolate pudding cake, and, when served warm, acts somewhat like a chocolate lava cake.
So, as above, I may add in browned butter and espresso to make it more rich, but that seemed like a good way to distinguish myself.

Then the options arose again.
I could either make it small, as in a cupcake mold, or large, the size of a 8" cake or so.\footnote{yes an 8" cake is 8", but it made sense that the time}
There are advantages and disadvantages to both, so I was not sure which I'll do.

Then the question became one of serving the cake.
I could present it on its own, but that lacks a lot of the cool factor that we love when we bake.
So, I decided I'd add some sort of side to the dessert.\footnote{no I'm not taking it too seriously, I just refuse to lose}

I like fruit, especially strawberries, so I decided I'd do strawberries on the side.
But, then the question becomes how, and I was unsure.

Next, people suggested ice cream.
But, I'm trying to showcase my talents, so buying ice cream is out.
Making ice cream is a pain, and no-churn ice cream is fake.

So, I realized what I really wanted was something creamy.
Hmmm, whipped cream.

I thought about what I tended to associate with chocolate and strawberries, and it's mint.
So, I thought I'd do a fresh mint whipped cream, and serve that with mint and strawberry.

Then the question returned to the size.
As another friend pointed out, a cake sized budino would be more easily decorated, as well as looking more professional.\footnote{man I have no clue how to write the words there. Do you say \say{as well as} and then a gerund? or a regular verb? or what}
So, I realized that I could do a full size cake, cut a nice slice out of it, and have a strawberry on the top, along with a sprig of mint and the cream.
I'm still not sure where the cream will go, but I envision it along the side of the cake slice, with the cake topped with some sort of decorative strawberry flower.
Maybe the cream could have a slice leaning on it with the mint sprig.

So, in conclusion, I try way too hard. 
\end{document}