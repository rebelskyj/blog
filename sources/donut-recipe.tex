\documentclass[12pt]{article}[titlepage]
\newcommand{\say}[1]{``#1''}
\newcommand{\nsay}[1]{`#1'}
\usepackage{endnotes}
\newcommand{\B}{\backslash{}}
\renewcommand{\,}{\textsuperscript{,}}
\usepackage{setspace}
\usepackage{tipa}
\usepackage{hyperref}
\begin{document}
\doublespacing
\section{\href{donut-recipe.html}{Donut Recipe}}
First Published: 2024 December 12
\section{Draft One}

Wildly enough, I don't think that I've ever written a blog post with the word \say{recipe} in the url.  
I'm almost positive that I've given at least a few recipes, so that might be a bit of an issue.  
However, that is not the purpose of today's musing.  
As the title\footnote{and likely URL} probably indicated, I'm going to talk about my donut recipe.

As with most of the things that I cook these days, there was minimal measuring involved.  
In general, this tends to work out, because I generally work with continuous ingredients.\footnote{can you tell that I've been thinking a lot about quantum chemistry lately?}  
That is, if the dough is ever so slightly too dry, I can add functionally any amount of water to the dough to hydrate it slightly more.  
In the specific case of the donut\footnote{my spellchecker and the cookbook I used last night insist it's doughnut. Hmm wonder what gardner says. Tragic, he's on team dough because of ingredient. However, given that it's a 1\.5  to one ratio, I think that I'm going to feel justified with moving the lexicon forward} dough, though, I do somewhat regret not measuring anything, because I used one of the only discrete ingredients in the average baker's toolkit\footnote{I did absolutely sit and think for a long little bit about what ingredients might be in a baker's kit that are functionally discrete. Chocolate chips are, but almost never will they be treated as such, since they normally are done by volume. (I also roped a friend into this) An entire whole spice, such as cinnamon bark or a vanilla bean, is similar. Fruits and vegetables maybe, especially if used whole (I never know what to do with onion)}: the egg.\footnote{more accurately, two eggs, but}

As a result, rather than simply describing a texture, I feel somewhat as though I need to at least approximate the recipe I used.  
My best guess is as follows:

\begin{itemize}  
\item Three cups bread flour\footnote{because it's what I had. Given the way I use it, probably not a bad idea to use bread flour or other high protein}  
\item 2 large\footnote{I think} eggs  
\item Heavy splash of orange liqueur\footnote{I feel a familial obligation to use Gran Marnier, but A: the grocery store did not have it, and B: the store brand was much cheaper}  
\item Two thirds a cup of sugar  
\item Heavy pinch of salt  
\item Heavy splash of vanilla extract  
\item Yeast, approx 1 tablespoon\footnote{entirely because I buy yeast by the pound, and I hate to measure}  
\item One cup whole milk  
\item 2 oz water  
\item 1 tsp yeast  
\end{itemize}

\begin{enumerate}  
\item Pour flour, sugar, liqueur, salt, vanilla, and first tablespoon of yeast into a large bowl.  
\item Crack in two large eggs  
\item Pour the milk on top and stir with a wooden spoon\footnote{you probably don't have to use wood, but it's what I did}. Texture should be about the same as slime, or slightly thicker. That is, it should be very sticky, but when you stir, you should easily watch it pull away from the edges of the bowl.  
\item Cover and let sit for two hours.\footnote{since everything I used came from the fridge, I put in a slightly warmed oven}  
\item After two hours, remember that for some reason you can never get yeast to rise when poured directly in milk, so add final tsp of yeast into water in a small container. Wait until frothy and stir into the dough.  
\item Cover and wait 6\-8 hours.\footnote{could probably wait less time, but like bed, you know?}  
\item Dough should be approximately doubled in volume. Punch it down by using spoon to lever the dough off the rim of the bowl.\footnote{I generally assume you use a bowl that will be completely filled and doming when the dough finishes rising} Because it is a very wet dough, might take some effort to deflate.  
\item Cover again and wait until clearly risen once again  
\end{enumerate}

Now, I am always a fan of doing things a little extra.  
The previous time I made an iteration of this recipe\footnote{24 May 2022}, I think that I wrapped the dough around oreos.  
This time, at request of the people I am feeding them to, I had three fillings: oreo, biscoff, and whole strawberries.  
With this in mind, recipe will continue:

\begin{enumerate}  
\item If filling donut with a solid, take enough dough\footnote{I usually need to sprinkle a little bit of flour on the dough constantly, don't be afraid of that fact} to cover the object and wrap it. Because we used high protein flour, you can stretch the dough a fair amount. Don't\footnote{a pun you can't make with the \say{approved} spelling} worry about that, the donuts will puff in the oven. The older cookbook I found recommends rolling to 3/8 inch thick and cutting from there, so if afraid, use that as a baseline  
\item As each donut is made, place it on a greased sheet pan\footnote{or something else}. It is ideal to wait at least five minutes after forming the donuts before frying them, though if you wait to heat your oil until you've finished shaping the donuts, you'll likely be fine  
\item When filling, dough, or shaper is exhausted, heat a pot full of a good frying oil to 350\.\footnote{There are so many schools of thought to this. If you have money to spare, I have heard great things about avocado oil. If you have slightly less, peanut oil is often recommended. I personally \say{splurge} (in the grad student sense) by buying canola oil rather than vegetable oil, because I like at least nominally knowing where the hydrocarbons are from. In general, high smoke point, minimal flavor is the goal}  
Follow normal frying safety when frying.  
\item When oil reaches 350F\footnote{I really hope no one reading this (lol) assumed 350 C and didn't keep reading ahead. Oh well, not changing it}, add as many donuts as you see fit. I found that in my wok, 12\-14 was about as many as I could reasonably fit, though I did manage 20 at once.  
\item Using a wooden spoon\footnote{again, probably optional, though I like to think that the wood is less likely to damage the donuts}, gently stir the donuts as they fry, flipping them if one side appears to be blonder than the other.  
\item Pull from oil when golden brown\footnote{if in doubt, another 30 seconds probably won't hurt}, drain, and let cool on paper towels.\footnote{J. Kenji Lopez Alt did find that they work better than cookie trays for draining oil}  
\item When cool enough to handle\footnote{so for me: immediately to 30 seconds later. To a saner person, a few minutes later}, dip in icing of choice.  
\item Allow to cool fully! This is an important one, because the inside will likely retain heat better than the outside.\footnote{why yes, I did have a mouthful of hot strawberry this morning, why do you ask}  
\end{enumerate}

By mentioning the icing, some might wonder about the recipe.  
The oreo donuts were topped with vanilla icing, and the strawberry were topped with a lemon icing.  
\begin{itemize}  
\item Vanilla Icing: Powdered sugar, splash of vanilla, and enough milk to make it barely a liquid while being constantly stirred. Adjust vanilla to taste  
\item Lemon Icing: Zest lemon into powdered sugar, wait a few minutes, stir in the juice of the lemon, again until just barely liquid when actively being agitated.  
\end{itemize}

I think that about sums it up!

Goals:  
\begin{itemize}  
\item One offs:  
\begin{itemize}  
\item Talk to boss about Ph.D. timeline  
\item Pick a topic for a science communication article  
\item Find an occasion I could write a song for  
\item Make a list of the stretches I'll do each day  
\item Find a place to volunteer  
\item Paper hit list  
\item Compile a list of people I want to write letters to  
\item Muse about macros and micros  
\item Compile a list of 20 meals that I can make, with their ingredients (inc. shelf stable or lifetime), time, effort level, and nutrition info  
\item Figure out my motivation for each book and have it as the bookmark  
\item List of things that need to be cleaned and the frequency  
\item List of things in my life  
\item Make a list of musings to do  
\item Block out time on Sundays for gospel reflection  
\end{itemize}  
\item Weekly:  
\begin{itemize}  
\item Read a pop sci article a week, making notes about how they work \-\> Picking the article today to read tomorrow  
\item Spend 30 minutes 2x a week working on writing the song \-\> wrote lyrics yesterday  
\item Ten minutes 4x a week on drawing \-\> Drew yesterday morning, ran out of time this morning, so will find time in the evening  
\end{itemize}  
\item Daily:  
\begin{itemize}  
\item Define how I'm feeling each day at start and end \-\> I have been, because the app makes me  
\item Practice guitar daily (at least one scale and a chord progression) \-\> So far so good  
\item Muse daily \-\> whoops, forgot yesterday  
\item Stretch Twice a day \-\> Also did not do this morning, will try to do after posting this. Otherwise doing well.  
\item Walk to the gym every day \-\> Not so much, but it's too cold for this goal, so deleting it  
\item Do daily affirmations \-\> I hate this, also going to delete it  
\end{itemize}  
\end{itemize}

\end{document}