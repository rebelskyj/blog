\documentclass[12pt]{article}[titlepage]
\newcommand{\say}[1]{``#1''}
\newcommand{\nsay}[1]{`#1'}
\usepackage{endnotes}
\newcommand{\1}{\={a}}
\newcommand{\2}{\={e}}
\newcommand{\3}{\={\i}}
\newcommand{\4}{\=o}
\newcommand{\5}{\=u}
\newcommand{\6}{\={A}}
\newcommand{\B}{\backslash{}}
\newcommand{\sub}[1]{\textsubscript{#1}}
\renewcommand{\,}{\textsuperscript{,}}
\usepackage{setspace}
\usepackage{tipa}
\usepackage{hyperref}
\begin{document}
\doublespacing
\section{\href{polymer-3.html}{Polymer Review Part 3}}
First Published 2018 November 30

Prereading note: I actually looked in my notes for this one!

\section{Draft 1}
Today we're talking about the factors that affect T\sub{g} and T\sub{m}.
Factors that affect T\sub{g} will affect all polymers, while the factors that affect T\sub{m} will only affect semi-crystalline thermoplastics.

There are four main kinds of factors that affect T\sub{g}: intramolecular forces, intermolecular forces, chain length, and timescale.

The intramolecular forces that affect T\sub{g} are: chain stiffness, side groups, and cross link density.
As chain stiffness goes up,\footnote{usually because you have bulky groups or groups that have double bonds and whatnot (really anything that increases the rotational energy required)} T\sub{g} goes up.
This is because as the amount of energy it takes to rotate along a bond\footnote{which \href{polymer-2.html}{yesterday} we said is what T\sub{g} means} increases, so too does T\sub{g}. 
As side groups become bulkier, they increase T\sub{g}, for the same reason.
As cross-link density goes up, so does T\sub{g}, 
This is because it's hard to rotate around a link, because the side group is suddenly large.
Also, if cross-link density isn't 0, there is no T\sub{m}.

The intermolecular forces that affect T\sub{g} are: side group dipole moments,\footnote{I think this is just polarity} side group chain length, and plasticizers.
As the side groups become more polar, T\sub{g} goes up, because it draws the different polymer chains closer together.
For that same reason, as side groups become longer, T\sub{g} goes down, because it keeps them further away.
For that same reason, plasticizers also tend to lower T\sub{g}.

Third, assuming the molecular mass is less than M\sub{e},\footnote{\href{polymer-1.html}{the mass required to form an entanglement}} the increase of molecular mass will increase T\sub{g}.
Above M\sub{e}, there is effectively no difference in T\sub{g} for chain length.

Finally, everything in polymers is time dependent.
This makes sense when you think about how free energy diagrams work.
At any temperature above 0K, there's some energy.
Given enough time, any bond can rotate, because it'll at some point be able to overcome its energy barrier.
Heating it just makes that happen faster.
So, T\sub{g} decreases as the time-scale you view it from increases.
This also works in reverse, so a polymer that's being stressed at high rates will remain solid at higher temperatures than one that is not.

In order of importance, the intramolecular forces are more important than the intermolecular forces are more important than the other two in governing T\sub{g}.\footnote{according to my professor}

So, assuming cross-link density is 0, eventually the polymer will become viscous.\footnote{T\sub{m}}
This is affected mainly by the same factors that influence T\sub{g}, which is why they have the relationship \href{polymer-2.html}{I talked about in Polymer Review 2}.
As interactions between the chains goes up, T\sub{m} goes up.
As chain stiffness goes up, T\sub{m} goes up.
As branching\footnote{side group frequency} goes up, T\sub{m} goes down.
As delta S\footnote{change in entropy} goes up, T\sub{m} goes down.
To explain this, T\sub{m} can also be expressed as delta H\footnote{enthalpy} divided by delta S\footnote{entropy}.
Turning a crystal, which is highly ordered, into a liquid, which is not, increases the entropy of the state a lot.
So, somehow that also works in reverse?
This part is where the physics goes beyond me.

To lower delta S, you can orient the polymer more.
This leads well into crystals which is almost certainly the topic of tomorrow's post.
\end{document}