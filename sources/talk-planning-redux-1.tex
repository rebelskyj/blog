\documentclass[12pt]{article}[titlepage]
\newcommand{\say}[1]{``#1''}
\newcommand{\nsay}[1]{`#1'}
\usepackage{endnotes}
\newcommand{\1}{\={a}}
\newcommand{\2}{\={e}}
\newcommand{\3}{\={\i}}
\newcommand{\4}{\=o}
\newcommand{\5}{\=u}
\newcommand{\6}{\={A}}
\newcommand{\B}{\backslash{}}
\renewcommand{\,}{\textsuperscript{,}}
\usepackage{setspace}
\usepackage{tipa}
\usepackage{hyperref}
\begin{document}
\doublespacing
\section{\href{talk-planning-redux-1.html}{Talk Planning, Redux}}
First Published: 2023 August 29
\section{Draft 1}
\href{talk-planning-1}{Last} \href{talk-planning-2}{year} \href{talk-planning-3}{I} gave a talk that was in part about tuning theory.
Unsurprisingly, especially given the fact I had been thinking about that talk for years beforehand, I am not entirely happy with the talk that I gave.
Mostly, I think that's because I fell into the same trap that everyone does.
Namely: it's easy to explain how 3 to the n is never equal to 2 to the m\footnote{other than the trivial case of m = n = 0, but come on, we're adults here}, which means any system is inherently flawed.

As I keep thinking, though, there are other elements I could touch as well.
After all, if you only want two notes, you can have a perfectly tuned scale with C and G.\footnote{why I always start my scales on C is beyond me.
Probably all the accidents relating to white note keys.
Eh, it's not worth being concerned about I suppose}
If you want three notes, then you can even have a well tuned one with C, F, and G.\footnote{I'm not totally sure what a minor 7th is supposed to be, but at least the first three notes are fine.}

The first issue, I suppose, comes between using three and five limit tuning.
Now, for those who don't already know,\footnote{read: for those who have spent their life better than me} limit tuning refers to the largest prime factor you can have in the denominator or numerator.
Going through in order:
\begin{itemize}
\item One limit tuning: you can play a single note. Arguably, since it's a single note, it is definitionally in tune
\item Two limit tuning: you can play octaves! Wow, harmony!\footnote{ok so actually, there's a concept for consonance and dissonance where it's how far the notes are harmonically (I mean duh, but my words aren't working, hence why I'm doing talk planning).
The most consonant interval is obviously the octave (since unison isn't really an interval), and it has a distance of one.
A fifth has a distance of two, as does two octaves. I think that matches my experience well, though I need to find the source paper so I can read it (I think I saw it in a 12tone video at some point). From a quick google, it seems like they might be doing log of the least common multiple of the two notes.
In that case, once again, there is no way to make a fifth and octaves be the same harmonic distance.
At a first order, though, a fifth (2:3) would be six as LCM, and so lies between two and three octaves in terms of dissonance.
Ok enough diversion, back to the actual text}
\item Three limit tuning: you now get the fifth.
Stacking two fifths\footnote{for this and everything, assume that I still believe in octave equivalence (for all that I'm more and more intellectually willing to not)} gives you a major second.
Stacking three gives you a minor third, and so on until you build the twelve tone scale.
Notably, it takes 7 fifths to build a diatonic scale, assuming you start from the correct note.\footnote{in this case, building from the fourth (fa, or F (hmmmm I wonder if there's something to the fact that the fourth (f) is fa (f) and the note f (f). eh probably nothing new)}
That works, but it gives you really wonky thirds, because
\item Five limit tuning\footnote{remember, it's primes, so four adds nothing new}: adds the major and by extension minor third.
When building up the ratios, 1 to 2 is an octave, 2 to 3 is a fifth, 3 to 4 is a fourth\footnote{because we define a fifth and a fourth to be an octave, and 2 to four is absolutely an octave}, 4 to 5 is a major third, 5 to 6 is a minor third.\footnote{because we define a major and a minor third to be a fifth, and four to six is a fifth}
The third in 3 limit tuning is\footnote{ok some quick math: 3/2 is G, 3/2 times 3/2 is 9/8 (octave equiv) is D, 3/2 more is 27/16 is A, 81/64 is E. 1.265} 81:64, which is not quite the same.
From this, you only need F, C, and G to produce a diatonic scale! Wow, progress
\item Seven limit tuning introduces the harmonic seventh.
For those who haven't heard of that, that's because we don't really use it in western music.\footnote{though there are a lot of arguments that the blues and jazz use it, since they use something known as a blue note, which is halfway between a major and minor third, like this is}
Since the harmonic seventh of C lies more or less equidistant between B flat and B,\footnote{968 cents, according to wikipedia (and no, I will not be getting into what a cent is right now, because I don't care that much)} there isn't much of a use for it, and by extension, any further limit in western tuning.
\end{itemize}
Ok how did I get here?
Right, how tuning theory is broken for multiple reasons.
I think that a five limit diatonic scale should work perfectly, assuming you only want to play in one key.
I have an inclination that it would break down even as fast as adding the dominant or subdominant keys.\footnote{but I should really work out the math. Let's see, we get A from F and then A from D if we do dominant. A from D would be 27/16, whereas A from F is 4/3 for F and then the fifth is 4 to 5 so 15/8. Nope, 5/3 actually. That makes more sense. 1.6875 is not quite 1.6 repeating, though they are shockingly close.
I suppose that playing the two together could work.
I also suppose taking the harmonic (i think that's the right word, the log based one) average of the two could work}
Yeah it breaks down immediately.

Ok I know that I've done other prep work for the talk, but my notes are all very far away, and getting to them feels like a lot of effort with minimal payoff.
Anyways, all this to say, the new talk on tuning theory will go deeper into the math than my last one did, even if I never give it to a single real person.

\begin{itemize}
\item As you see here, I did work on this presentation!
\item I cleaned minimally.
\item Here we go streak!!!
\item I wrote a touch more of the next chapter, and dreamed about what to write 1.2/4
\item Last night I wrote a whole sonnet\footnote{debatably, I think it ended up as ABBA ACCA DDDDDD which is technically different, it also was v incoherent}
\item I wrote letters to two friends!\footnote{well, there's an argument to be made that since I wrote letters to a married couple that I wrote two letters to one flesh
\item Shoot! No stretch.
\item I had trouble falling asleep last night, and so slept in.
I am honestly unsure if that was my best bet.
\item Prayer has remained less of a priority than it should be.
\end{itemize}


\end{document}
