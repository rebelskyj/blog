\documentclass[12pt]{article}[titlepage]
\newcommand{\say}[1]{``#1''}
\newcommand{\nsay}[1]{`#1'}
\usepackage{endnotes}
\newcommand{\B}{\backslash{}}
\renewcommand{\,}{\textsuperscript{,}}
\usepackage{setspace}
\usepackage{tipa}
\usepackage{hyperref}
\begin{document}
\doublespacing
\section{\href{pathfinder-2.html}{Pathfinder Again}}
First Published: 2023 November 7

As I alluded to \href{dungeons-dragons-5.html}{yesterday}, I'm currently playing a Pathfinder 2E campaign with some friends.
It's an interesting concept, where we groundhog's day\footnote{I'm not capitalizing this even though it's a holiday (should be capitalized) and referencing the movie, which was then verbed (should be capitalized). If pressed, I'll argue that it's like a double negative, for all that I just really don't want to capitalize it right now}
back to the start of the dungeon every time that we die.
Due in large part to that being the agreed upon mechanic, the game is primarily combat focused.

Our GM had the stated goal of getting better at creating and executing combats during the course of the campaign.
He is one of my alleged cousins,\footnote{if you don't know what I'm referencing, just substitute family friend} and the party consists of me, my brother\footnote{the one I have actual blood with (if you know from outside of this 'blog that the number of brothers by blood I have is not equal to one, you're welcome to ask me for clarifications, because I presumably know you in real life}, the GM's brother\footnote{who is technically the person I claimed to be cousins with, but that was more a system of convenience than anything else}, and the GM's former roommate.\footnote{who I don't know if I've ever actually met in meatspace (an expression I love)}
It's been a fun experience, in large part because of the lack of consequences.
Often when I play DnD or an equivalent game, I try very hard to both focus on the specific combat and having a good time there and also on the broader story arc.
That can be a little exhausting, because fighting optimally is fun for me, but is not necessarily reflective of the characters I enjoy roleplaying outside of combat.\footnote{for those not seeing the issue, the optimal way to fight in a game like DnD is to be a complete sociopath.
I generally like to play characters with empathy, which makes killing hordes of defenseless creatures have some cognitive dissonance}
In this game, however, I've decided to play a character who is concerned about the fact that he hasn't died yet.

More than that, the DM has given us free reign to rewrite our character sheets between attempted delves, which adds a layer of dehumanization as an RP'er.
My character no longer really knows what is real, since he truly remembers being born of Black Dragons, but now seems to be born of Blue Dragon.\footnote{Barbarian with the Draconic Primal instinct.
We started fighting monsters that were resistant to lightning, so changed the design so that now I deal Acid Damage instead.}
It's been really fun, and I'm excited to play again.
The rest of the post will be written tonight, when we're finished with the session.
It may be a first person log, from my character, depending on how I feel.

Well, this time I was the only fatality.
The issue with playing a game where everyone\footnote{I think, at the very least 4 of the five of us} has an undergraduate degree in chemistry, and the majority of us are getting degrees in something at least vaguely related to chemistry is that you run into some dangerously real world interactions.
In PF2e, there's a Class\footnote{I think, maybe it's an ancestry} that allows you to create metal structures.
There are a number of monsters which create acid, which our GM ruled can erode metal structures, because metal, as we know, dissolves in acid.

One of our group members, being a chemist, recognized the redox reaction that implies.
Iron, when dissolved in acid, converts to Iron Oxide.
The oxygen, obviously, comes from somewhere.
In the case of dissolving in acid, that tends to come from water.

Now, water, as we all know, is made of two hydrogens and an oxygen.
When you pull the oxygen out of water, you are left with hydrogen.
Through potentially relevant situations, the monster we were fighting ended up on fire.\footnote{persistent damage in pathfinder is great, and my only regret is not having more sources of it}
When the monster then ate away what the GM decided was 60 pounds of iron, that meant that we had reacted away approximately 550 moles of iron.

That ended up making something around 1200 liters of hydrogen gas.\footnote{don't ask me for that conversion, because I just relied on a friend's number}
The GM ruled that was worth 40 d6 fire damage.
I, stupidly, had just jumped next to the monster, and so was caught in the blast as well.
Had I been at full health, it would have almost killed me at once.
Since I was doing my job as a bartender and tanking the hit, I completely died at once.
I'm excited to be revived next week, and to have learned absolutely no lessons from the experience.

Update a few minutes later: I apparently got to come back to life.\footnote{in universe reason: we keep respawning because shrug.
Out of universe reason, it was early and everyone else wanted to continue exploring.}
Now we're going sailing, for some reason.\footnote{being totally honest, I tuned out a little bit while I was dead, because I role play very hard.}
Thankfully, I have taken the nicest feat in pathfinder\footnote{assurance, which just lets you always roll a 10 and ignore any circumstances on a skill of your choice.
As a barbarian, that means athletics, which is my most useful skill, and meant that we navigated with no struggles at all}
We found a river drake, and through situations completely outside of our control\footnote{read: I am the only member of the party who speaks draconic, and my character hates dragon related creatures, though the party doesn't know that} it tried to fight us.
It's very fun being a barbarian, because my strategy is very simple.
If I am close enough to punch, I use a war flail and smash it.
If not, I either throw a bomb or a javelin, depending on how much of a threat the object appears.

In the case of the water drake, it was very easy to dispatch, likely due to the fact that we are all fifth level and got a turn off before it did.
With a couple of lucky criticals, we dispatched it before it had a chance to fight.
A part of me feels bad, but that voice is a small one.
I realized the group did not know that I dislike dragons, which is the only part I feel bad about.

We then encountered some other creatures that also do not speak Draconic.
As a result, we were able to tell them that the river drake had attacked us for no reason.
Unfortunately, they also wanted the eggs, so we had a bit of a disagreement over how to make that happen.
After offering them other foods, they tried to fight us.\footnote{this time I don't think it's my fault, because I tried hard to make peace and failed unintentionally}

Annoyingly enough, I failed my single attack roll, which meant that I mostly just stood around as the rest of the party killed the monsters.
I did my job, however, and soaked up the damage that they wanted to output.
In such a regard, I really do a great job of playing damage sponge for the group.

Daily Reflection:
\begin{itemize}
\item Did I write 1700 words for NaNoWriMo? Once again, it was a successful day of writing for NaNo.
I still don't love the story, and I don't think that I'm going to at any point, but that's fine.
It's good for me to do something I only kind of enjoy sometimes.
\item Did I write a chapter of Jeb? 
So, I did not end up finishing last night's chapter. It ended up being exactly six hundred words shy, which I wrote this morning during my hour of writing time with a friend.\footnote{got a pr for it i think, with 2819 net words at the end of the hour}
I have not yet finished today's chapter of Jeb, but I have hope that I will do so before I go to sleep.
If not, I'll have to finish it tomorrow as well, or accept that a long chapter of a book a day may be too optimistic of a goal for me to take on in addition to everything else I've done.
I did, semi-relatedly, write a 1200 word draft of a methods section for a manuscript my PI has asked me to start writing. 
Maybe that explains some of my lack of desire to keep writing right now.
For all that my creative writing total is about 5200 words today, my total wordcount is apparently closer to 7000\footnote{I really don't trust the count on this site, so I don't use it for anything real}

Update as I kept working on this blog post: I'm now much closer to 6000 creative words today, and I'm almost positive that I'll finish the chapter tonight, if only because I still have like a hundred words left to write and I have nothing to write here.

\item Did I blog? Did I ever! Look at me, continuing to send out content into the void that is the website I have.
\item Did I stretch? I ended up falling asleep midstretch last night, which isn't great.
After this post posts, I'll start.
\item Am I doing better at prayer than a rushed and thoughtless rosary? 
Not really, honestly.
My rosary last night was done while half asleep.
I did the Angelus today, and I tried to focus on it, though.
\item Am I doing a good job writing letters to friends?
Not today!
I got another friend's address, which is a form of progress, I suppose.
In terms of putting ink\footnote{I write with a fountain pen, yes, why do you ask?
I really like the ink color, and wow it just writes so smoothly, it's such a nice change.
I tried to write with a mechanical pencil the other day and it was so hard.}
on paper, though, I did not make any progress.
There's always tomorrow.
\end{itemize}


\end{document}