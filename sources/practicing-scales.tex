\documentclass[12pt]{article}[titlepage]
\newcommand{\say}[1]{``#1''}
\newcommand{\nsay}[1]{`#1'}
\usepackage{endnotes}
\newcommand{\1}{\={a}}
\newcommand{\2}{\={e}}
\newcommand{\3}{\={\i}}
\newcommand{\4}{\=o}
\newcommand{\5}{\=u}
\newcommand{\6}{\={A}}
\newcommand{\B}{\backslash{}}
\renewcommand{\,}{\textsuperscript{,}}
\usepackage{setspace}
\usepackage{tipa}
\usepackage{hyperref}
\begin{document}
\doublespacing
\section{\href{practicing-scales.html}{On Practicing Scales}}
First Published: 2023 December 26
\section{Draft 2}
I've talked on this blog many times about the fact that I often struggle with finding a topic for my musings.
Today I began to look for inspiration in that best of places, my own blog.\footnote{accusations of plagiarism are much easier to beat when you can cite your own footnotes.
For that reason, if no other, I'm glad that I have my footnotes.
Given that I keep my different drafts, it may or may not be needed to have footnotes.
I also think that there's something to be said for the fact that my putting drafts above each other just means that I end up rewriting the entire thing, paraphrasing myself, rather than copy pasting.
I don't honestly know if that's a good or bad thing.
I think that my words tend to be better on repeating, but.
Onto the musing list it goes (since I'm now being a smart and compiling a list of things that I want to muse about).}
I found an old footnote about the fact that I don't really like the phrase eating your vegetables, and prefer practicing your scales.

There are a variety of reasons for that.
Obviously, there's the fact that I tend to enjoy vegetables,\footnote{there's something to be said about the fact that people often enjoy vegetables more when they're prepared better} which makes the expression not resonate for me.
Additionally, there is the fact that\footnote{I do love how I sometimes say there is and there's and I do not know when I use either. If I had to guess, I contract the first use and spell out future uses. Not worth my time to look up right now} I tend to hear eating your vegetables in the context of things which are beneficial to a small domain of tasks, while eating vegetables is something which is generally good and helpful for overall life quality.
And, most importantly,\footnote{at least for this musing, if the first draft doesn't make that clear} I tend to find that people use the expression not as a way of expressing need, but as a way to emphasize the difference between Amateur and Dilettante.\footnote{capitalized because I am redefining them, and I tend to find that essayists redefining a word capitalize it}

What do I mean by that?\footnote{this is a rhetorical device I need to kill within myself before I start writing my thesis in earnest}

I mean that I'll be using the words as they initially were meant\footnote{or, at least, as I'm choosing to believe that they were initially meant}, or for those less comfortable with that, I'll be using the words based on their Latin roots.\footnote{ok so Latin roots are not the end all be all for definitions. However, making up words as needed, even if they are already extant words, is not something that I am the first to do. Where was I? Right}
That is, an Amateur is one who loves a craft, while a Dilettante simply enjoys it\footnote{delights in it, one might say. I shan't (shant?), because I don't have that word in my own lexicon}.
What's the difference?\footnote{oof, I'm really using a lot of questions today. Maybe redraft again? We'll see what time it is when I finally finish this draft of the musing, especially given the fact that I have so very very many words in the footnotes here. I'm clearly too distractable. Maybe put the musing aside for a moment? Yeah that's good, let's just try not to have too many more sideboards (it's not sideboard? It's uh sidebar. That's the word)}

Well, what's the difference between love and enjoyment?
Love, to me, at least, has an element of growth.
Enjoyment, on the other hand, asks and assumes nothing of that.

To practice scales, then, is to accept the unpleasant parts of a craft because you understand the benefits.
As someone\footnote{I do honestly think it was in Atomic Habits} says, you don't get to choose whether you go in the dirt, only what dirt you go in.\footnote{he said something a little more vulgar, but the point remains}

I think it's an important distinction, and one that really will be helpful to me as I continue framing my life.\footnote{future tense is so hard in English. Prophetic future is a tense that I want to incorporate at some point, but}
What does it mean to practice scales?

On almost every instrument, one of the assignments that an instructor gives fairly early in the pedagogy is practicing scales and variations thereof.\footnote{like going up and down in thirds}
Why?\footnote{gosh this is feeling awful didactic today. Wonder why}
Even though almost no song is exclusively a scale,\footnote{opening line of Joy to the World is not an exception, because it's not an entire song (n.b. I came back to this paragraph because I realized that the way I wrote the rest of the sentence and the next few didn't let me put that joke in} the majority of most music, especially music in the canon\footnote{i.e. not popular music from the past 140 years} is primarily written in a scalar fashion.
The reasons for that are ancient and long enduring, but it is generally easier for most people to sing scalar melodies.\footnote{now, the chicken and egg question comes up here, because most children's taunts or child made up songs that I can think of are far more leap based than scale based. Hard to know for certain, and I live in a society and all}

However, unless you are dedicating yourself to the craft,\footnote{dedication is a better word for love versus enjoy. If I rewrite this in the future, keep that in mind} you almost never need to work on scales.
If we move to the kitchen, one of the first tasks that I've heard every culinary institute or restaurant requires of its chefs is chopping loads and loads of vegetables perfectly and uniformly.
As someone who merely enjoys cooking, I will probably never bother to do that.
However, I also accept that the fact that I do not practice julienning or any other technique on countless thousands of carrots does mean that there is a level of reproducibility and refinement that my cooking will never have.

So, as someone who does not believe that it's necessary to practice scales, I can choose to do so.
When I practice scales, it is with the knowledge that, assuming my goal is to improve at the craft, the best use of my time is probably spent working on scales.
However, there are many things that I do but do not love doing.\footnote{cooking, for instance}
Even if I enjoy them, it is not worth the effort of getting better for the sake of getting better.

When I work on embroidery, though, right now I am actively striving to improve.
There may be something questionable in the fact that I intrinsically tie love with desire to improve, but I don't know if I want to unpack that right now.\footnote{should that have been a footnote? unclear. it is needed to bridge the implicit love is improvement that I hadn't stated before (new draft absolutely needed). Dedication also works as a term}
Since I am dedicated to improving, I am willing to take time to simply embroider a specific stitch or test swatch over and over until it's perfect.

In crochet, however, I don't really care about getting better.
Right now I know that I am good enough at crochet to make anything that I want to make.
As a result, I'm not going to spend time, energy, and yarn, working on a specific skill simply to be better at that skill, even if it would be generally helpful to my overall craft.

Woof. This musing got away from me a little. Eh, I think that I got somewhere good, even if it's only coherent to me.

Daily Reflection:
\begin{itemize}
\item Hobbies:
\begin{itemize}
\item Did I embroider today? I plotted out the actual next project I have and then started it.
\item Did I play guitar today? Took today off.
\item Did I practice touch typing today? Spent a little bit of time on it, I don't know if I actually made any progress.
\end{itemize}
\item Reading
\begin{itemize}
\item Have I made progress on my Currently Reading Shelf? A little bit. I'm not enjoying the book right now, but that could have any number of causes.
\item Did I read the book on craft? I did not, for all that I thought about it a lot.
\end{itemize}
\item Writing
\begin{itemize}
\item Did I write a sonnet? I think that I'm done with sonnets for the month. I can do them quickly, and right now I want to just spend my time with my family\footnote{or, as this monstrosity implies, blogging}
\item Did I blog? Look at this! It even has some elements of growth.
\item Did I write ahead on Jeb? At some point I need to start again. If I want to be five chapters ahead by the third, I really need to pick up the pace.
\item Letter to friends? Nope! A friend set up a meeting time, though, so that's good enough.
\item Paper? I've started thinking again.
\end{itemize}
\item Wellness
\begin{itemize}
\item How well did I pray? Badly, but maybe better.
\item Did I spend my time well? I've decided that anything that I do with family counts as time well spent.
\item Did I stretch? Nope.
\item Did I exercise? Kind of! Because we had activities, if no other reason.
\item Water? Eh, not really.
\end{itemize}
\end{itemize}

\section{Draft 1}
As much as I do trust the book on writing, I want to write my musings as essays only.
So, despite the fact that they recommend writing my essays where the first draft is not a full essay, but as pieces, I will not be taking their advice.
Of course, I am starting my musing\footnote{today at least} from a much further point than normal.
I already know what I want to muse about, and I already even have some concepts ready.

I was struggling to come up with a musing idea today.
I read through some old musings that I made, and found inspiration.
Today I finally started embroidering again.
I thought that could be a good place to start, so I read through my old musings on embroidering.

In the footnote of \href{embroidery-2.html}{my second musing}, I mentioned that something good to muse about could be the concept of practicing your scales.
Practicing your scales, to me at least, is a better metaphor than eating your vegetables.
There are a few reasons for that.

First, I honestly tend to enjoy eating vegetables.
The concept of eating your vegetables tends to be something unenjoyable but necessary.

Second, and far more importantly, I tend to think of eating your vegetables as something you do that is undesirable but good for a specific activity.
Eating your vegetables is\footnote{arguably} something that you have to do for general life purposes.

So, what does practicing your scales mean?

First, I don't know that I've ever actually met someone who enjoys working on scales.\footnote{at best, I've heard that it's enjoyable because you're playing an instrument, with the implication that they'd rather be doing anything else}
Second\footnote{I don't really like the numbering, so I'm going to drop it here}, working on scales is not, strictly speaking, necessary.\footnote{except maybe on piano, but it isn't really like I ever plan to learn piano, especially not at the level where the advice for a piece is \say{take a year or two off and learn other music until you're good enough to start learning the music}. I'll say that I've never met an instrument where scales were unarguably (weird that it's inarguable but unarguably) needed}
Scales make music far easier, especially when you take scale not just mean playing the notes of a given key up and down stepwise, but also with some small intervals, like thirds and maybe fourths.\footnote{sixths and larger are just good for practicing leaps, which tend to happen very rarely. A common warmup I've done on basically every chromatic wind instrument (wind in the sense of blown, not just woodwinds) is to start on the fifth and then play down by half steps (e.g. G, f sharp, g, f, g e, g d sharp... g c g), so that I can practice intonation and leaps at once. Often I'll go for an octave after that. Initially the post said fifth, but I feel like fifths are a weird interval where you almost never see them stacked in a melodic line.}
The fact that they are not needed but make life far easier and better is really where this metaphor works.

So, at this point in the musing, we can either go for how we can use the metaphors or why scales are helpful.
Or, I guess that the initial musing talked about how I'm only now enough of an adult to practice my scales, so I suppose that I could focus there.
I think that I want to go with the first.

What does it mean to practice scales in other domains?

What are other domains?
I suppose that for many athletes, cardio or strength is the equivalent.
If you enjoy running or lifting, though, that metaphor breaks down a little.

Other domains include writing?
I suppose that the practice I'm doing typing could count.
Let's talk about that (in the next draft).\footnote{I think that this may be the first in text parenthetical that I've ever used}
What else could count? 
In some regards, the reading about writing that I do is practicing scales.

Oh gosh, there's a concern now that I could treat the parts of my craft I'm comparing here to practicing scales as something that is less than enjoyable.
I assume that this will not be an issue, since I don't plan on remembering this.\footnote{I mean um. no that's right, I suppose}

Ok so reading about writing, practicing touch typing, arguably even the writing that I'm doing right now.

In music, there's scales and fingering exercises, obviously.

In cooking, there's repeating recipes?
Practicing cutting?
Oh yeah that's the equivalent.
Practicing just cutting to make everything perfect.

Oh hey, that makes me think that practicing scales is what separates masters from amateurs.\footnote{ok so amateur is the wrong term here, because I like it meaning one who does it for love, and I think that it takes a level of love and dedication to actually do the writing. Ok actually, let's see if we cannot restart the piece with that in mind. Ope ok this should be main text}

As the above footnote points out, to be an amateur is historically to do something for love.
Where did that shift come from?
Oh duh, the word was initially used to refer to the rich gentlemen of England, who pursued a craft out of love, rather than to make a living.
When we shift to the fact that America doesn't really believe in class\footnote{in the same way}, it makes sense that there would be a semantic shift.
Dilettante is apparently a similar word.

Anyways, what separates the amateur from the not?

What's the opposite of an amateur?

Because of how fully the world has shifted to thinking of amateur\footnote{one day I'll learn how to spell} as non professional, it's hard to think of the type of person who does the work primarily for the need.
Oh wait, my opposite is not one who works for a living, but one who does not truly love a craft.

Dilettante\footnote{another word that's going to kill me as I continue trying to spell it} works great for that.
It comes from delight, with connotations historically and always as one who does something because it's fun.

Ok so practicing scales is what separates the Amateur\footnote{we'll use capital letters to denote my usage of the word, rather than the standard usage of the word} from the Dilettante.
Great! That sounds like a good musing idea.\footnote{the fact that I've been working on the post for an hour and only now got to here does do something to imply that the book on writing is right about at least some of the stuff}

So, let's start with \say{what does it mean to practice scales?}

Practicing scales is similar to something that other people talk about: eating your vegetables.

Ok so as much as I want to focus on dilettante rather than amateur, I do still want to title the file practicing scales.
I also do want to muse about that, for all that it doesn't take much to do that.
I feel like one hundred words is enough to explain it.
Let's try second draft and see what it looks like.
\end{document}