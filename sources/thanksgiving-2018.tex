\documentclass[12pt]{article}[titlepage]
\newcommand{\say}[1]{``#1''}
\newcommand{\nsay}[1]{`#1'}
\usepackage{endnotes}
\newcommand{\1}{\={a}}
\newcommand{\2}{\={e}}
\newcommand{\3}{\={\i}}
\newcommand{\4}{\=o}
\newcommand{\5}{\=u}
\newcommand{\6}{\={A}}
\newcommand{\B}{\backslash{}}
\renewcommand{\,}{\textsuperscript{,}}
\usepackage{setspace}
\usepackage{tipa}
\usepackage{hyperref}
\begin{document}
\doublespacing
\section{\href{thanksgiving-2018.html}{Thanksgiving!}}
First Published: 2018 November 22
\section{Draft 1 (21 Nov)}
Thanksgiving is an American holiday.
Or, at least, Thanksgiving being celebrated today is an American custom.
Here, on the other side of the pond, nothing really seems notable about it.
I'll still have all of my classes, and none of my traditions from home.

Now, before I start talking specifics about the traditions, I'd like to talk a bit about my traditions in general.
At some point or points in my early life,\footnote{middle and high school} I was assigned the homework of bringing a list of family traditions to school.
Of course, I didn't really think we had any.

As I look back, I realize we have many, and I'll probably write about them later.
But, I realized we also didn't have as many as I'd like.
We used to make pasta and bread, and don't any more.
We used to have large groups of people, and don't anymore.\footnote{help which of the two no mores is correct}

So, as Thanksgiving approached my freshman year, I had an idea.
Since people had been bugging me about making bagels,\footnote{side note, never tell people you know how to make anything that's that much effort} and I had a lot of friends who couldn't make it home for Thanksgiving, I suggested to my family that we host a large party for it.

They agreed, and so the schedule began.
I awoke at around 6am to begin the different bagel recipes.

We made three kinds of bagel dough: a normal bagelly dough, a Guinness dough,\footnote{guess what the secret ingredient was} and a tequila dough.\footnote{can you guess this one?}
Within the initial dough, we made regular, poppy seed, everything, cinnamon, and blackberry.\footnote{because we couldn't find blueberries}\footnote{I think that's all}

Of course, when you have around 200 bagels to make and shape, it's great to have friends come.
So, my 7 or so helpers and I made the outrageous number of bagels.
As the crowd arrived, the bagels were ready, so people ate them.
And, as friends had to depart for whatever reasons, they were able to take the bagels with them.

That's another nice tradition that I'm glad for.
My bagels fed not only the Rebelsky Family Thanksgiving Extravaganza,\footnote{which I believe is what it was called in 2017} but also the Women's Basketball House, and some others.

After bagels, we ate and played board games through the end of the day.

Last year functioned similarly, though I actually have a schedule!\footnote{also list of bagels!}\\
6:30 AM: Bagel Bonanza (100 cups of flour!) Begins!!!\footnote{I started around 6 so that if anyone showed up then they could enjoy it}\\
7:00 AM: Breakfast begins (pancakes)\footnote{for the helpers}\\
3:00 PM: Dinner!\footnote{breakfast ended before then, but no scheduled events took place}\\
5:00 PM: Dessert\footnote{I don't know if Dinner ended then or not}\\
6:30 PM: Games!\footnote{this didn't end. Some say you can still hear the players arguing about a loophole in the rules}

The bagels we made were:
	regular,
	cinnamon,
	cinnamon raisin,\footnote{i.e. take cinnamon and add raisin}
	salt,\footnote{which were disappointing imo}
	poppy seed,
	everything,
	blueberry,
	blackberry,
	strawberry,\footnote{no you have too many berries}
	apple and cinnamon,\footnote{i.e. take cinnamon and add apple chunks}
	asiago,\footnote{for some reason}
	Rosemary, and
	Rosemary and olive oil.\footnote{an oilier rosemary}

In total, we went through more than 100 cups of flour, and more than 400 bagels.
It was great!
I still have no clue where they all ended up, though I know that some left with everyone I could make take some.

But anyways, this year that will not be happening, which is sad.
Hopefully next year it will again. 
\end{document}