\documentclass[12pt]{article}[titlepage]
\newcommand{\say}[1]{``#1''}
\newcommand{\nsay}[1]{`#1'}
\usepackage{endnotes}
\newcommand{\1}{\={a}}
\newcommand{\2}{\={e}}
\newcommand{\3}{\={\i}}
\newcommand{\4}{\=o}
\newcommand{\5}{\=u}
\newcommand{\6}{\={A}}
\newcommand{\B}{\backslash{}}
\renewcommand{\,}{\textsuperscript{,}}
\usepackage{setspace}
\usepackage{tipa}
\usepackage{hyperref}
\begin{document}
\doublespacing
\section{\href{reflections-on-readings-4-advent-b-23.html}{Reflections on Today's Gospel}}
First Published: 2023 December 24

\section{Draft 2}
Today is the Fourth Sunday of Advent.
It's a strange set of readings, especially when we look at the readings that we have had for the past few weeks.
The first reading, as expected, is a reading from the Old Testament.
Unlike most of the other readings, today's does not come from one of the explicitly prophetic books.
Instead, we have an excerpt\footnote{I mean technically like 5 excerpts, given that the reading comes in many parts} from the Second Book of Samuel.
We follow David, who wants to build the Ark of the Covenant a better dwelling place.

Of course, as Catholics, we know that the Ark was a foreshadowing of Mary, who carried the Christ into the world.\footnote{Ok, so here's my inane question for the day: Christ means anointed. Is there a historical moment where He was anointed? I guess the only biblical scene I can think of is when he's bathed in funerary oils, but that might not be what people are talking about}
And so, it's only fitting that the Gospel today focus on Mary.
In particular, we see one thread continuing from the time of David the king to the time of Christ the King: The Almighty's angels.
The Angel Gabriel delivers messages both to David and to Mary.
And so, in the final hours before Christmas\footnote{personally, I've already been to Christmas Mass, so I guess it's already Christmas for me}, I think it's good to reflect on what Mary said to the Angel of the Most High, \say{may it be done unto me according to your word.}\footnote{Luke 1: 38 B}
The Pater Noster may be the perfect form of prayer, but Mary's Fiat\footnote{I do love that the best way to figure out a prayer's name is just to take the opening word or words from its Latin text} is bound to be a close second.
David wanted to build a fitting home for the Ark, but did not know what to ask.

Daily Reflection:
\begin{itemize}
\item Hobbies:
\begin{itemize}
\item Did I embroider today? I realized that I actually want to learn the stylistic aspects so tried to recharge a device that would let me plot out my embroidery.
I charged it, but that's as far as I got.
\item Did I play guitar today? I did! Even jammed with my brother for a little bit\footnote{him on grand piano, me on guitar}
\item Did I practice touch typing today? I made it through B! Have to repeat P again tomorrow, but that's a task for future me.
\end{itemize}
\item Reading
\begin{itemize}
\item Have I made progress on my Currently Reading Shelf? I finished the currently reading book from the library, now I just need to finish the rest of the books that I'm reading.
\item Did I read the book on craft? I read! Only about twenty pages. If I want to finish the book by the end of the month\footnote{year}, I need to read a little more quickly. It's really fascinating how much I feel like I learn in every word.
I'm also glad to know that the authors agree with me that there are no perfect synonyms.
\item Have I read the library books? Oh gosh, there's no chance at all that I finish the books in time. If I start reading just far too much, it would be possible for me to make it through, but it would mean that I kind of cannot do anything else.
\end{itemize}
\item Writing
\begin{itemize}
\item Did I write a sonnet? I feel bad about the amount of time that I've taken off from this goal. Let's go write one right now.\footnote{or, after posting the musing, at least}
\item Did I revise a sonnet? Revision no longer feels needed. I'm going to delete it starting tomorrow.
\item Did I blog? I'm not thrilled with this musing, but I don't have the mental space to keep trying.
\item Did I write ahead on Jeb? Sunday which is obviously a day off.
\item Letter to friends? I wrote out the Christmas notes that I want to send.
\item Paper? It's a holiday.
\end{itemize}
\item Wellness
\begin{itemize}
\item How well did I pray? I went to two Masses, which means that there was at least a little bit of prayer.
\item Did I spend my time well? Pretty well! I spent it with family, which is really nice.
\item Did I stretch? I am regretting the fact that I don't, and I need to find a way to fit it into my life from now on.
\item Did I exercise? See the prior item.
\item Water? I drank more water, if only because my voice kept breaking at my first mass.
\end{itemize}
\end{itemize}

\section{Draft 1}
It's the fourth Sunday of Advent, and it almost feels like Christmas.\footnote{the joke being that today is Christmas Eve (though arguably eve implies sunsetting, which may or may not be a part of the actual requirement here)}
Today's readings were pretty standard \say{be ready for Christmas} readings.
The first reading comes from the Second book of Samuel, where we see David trying to pay homage to his Lord.
He looks around, realizing that he lives in a beautiful home, while the Lord is being kept in a tent.
The prophet of the day\footnote{Nathan} assures him that the Lord will appreciate any efforts that Daniel puts forth.

The Lord, however, comes to Nathan in a dream,\footnote{We skip through different chapters in the book, so some of the message is left up to implication, rather than outright stated today} and tells him to tell Daniel not to build a temple.
More than that, though, the Lord says that he will establish an eternal kingdom from David's descendants.
As Catholics, we know what that kingdom is, and who the heir is.
The Gospel, of course, makes that explicit.

The angel\footnote{archangel, technically, or maybe Archangel. Angelic divisions have never been clear to me} Gabriel\footnote{literally: \say{G-d is my strength} or \say{G-d is strong}} comes to Mary.
Importantly, he is the same angel who came to Daniel to interpret his visions.
It's something that very rarely comes up in homilies that I've heard, probably for some good reason.

Of course, one reason could very easily just be that the Ave Maria\footnote{I do love that so many Catholic things are just named from the first few words of the Latin (the incipit, if you want to sound fancy). It's fun then how often the same holds true when we translate into English (ex. The Our Father, the Glory Be, the Hail Mary, the Angelus (which I guess we don't translate))} is allegedly a very unpopular prayer among Protestants.\footnote{I say allegedly because I've never heard accusations that I can remember}
The fact that the opening line comes directly from an angel\footnote{literally messenger} of the Lord should, in theory, help assuage concerns about the prayer.
Then again, I don't know if anyone attending Mass on the Fourth Sunday of Advent\footnote{especially when that's also the same day as, but a liturgically distinct day from, Christmas Eve} needs to be convinced that Mary is a good and holy person, worthy of admiration.\footnote{I think veneration is the ok one and adoration isn't but I'm not 100 percent sure and don't really want to look it up}

Advent in general, as the priest today reminded us, is a season focused on Mary.
It feels a little strange to me that we only really care about the last month of Mary's pregnancy, but I suppose that otherwise we'd have to start Advent as soon as Easter ended.\footnote{or even before, most years}
I'm sure there's something to think about in the fact that we only really talk about Christ as a newborn child and as a thirty three year old.
It is a little strange to think that, were I Christ, nothing that I've really done with my life so far would have been recorded.

Post writing note:
I'm not really happy with the musing right now, for all that I don't really know what else there is to reflect on in the readings.
Maybe I should come back to this in a little bit?
\end{document}