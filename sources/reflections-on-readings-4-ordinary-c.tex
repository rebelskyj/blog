\documentclass[12pt]{article}[titlepage]
\newcommand{\say}[1]{``#1''}
\newcommand{\nsay}[1]{`#1'}
\usepackage{endnotes}
\newcommand{\1}{\={a}}
\newcommand{\2}{\={e}}
\newcommand{\3}{\={\i}}
\newcommand{\4}{\=o}
\newcommand{\5}{\=u}
\newcommand{\6}{\={A}}
\newcommand{\B}{\backslash{}}
\renewcommand{\,}{\textsuperscript{,}}
\usepackage{setspace}
\usepackage{tipa}
\usepackage{hyperref}
\begin{document}
\doublespacing
\section{\href{reflections-on-readings-4-ordinary-c.html}{Reflections on Today's Gospel}}
First Published: 2019 February 03

Jeremiah 1:4: \say{The word of the LORD came to me:}

\section{Draft 1}
Today's three readings exhort us to three different calls.
In the first, the Lord speaks to Jeremiah, urging him to strength by saying \say{They will fight against you but not prevail over you, for I am with you to deliver you, says the LORD.}\footnote{Jeremiah 1:19}
We too are called to trust the Lord to protect us from harm and lead us to salvation.
The second exhorts us that \say{And if I have the gift of prophecy and comprehend all mysteries and all knowledge; if I have all faith so as to move mountains but do not have love, I am nothing}\footnote{1 Corinthians 13:2}
In the Gospel, Jesus drives his countrymen to anger by speaking truths to them.\footnote{When the people in the synagogue heard this, they were all filled with fury. (Luke 4:28)}

In all of these we see God's command.
God is Love, so of course we need love to be anything, for what are we without God?
And, though we should not test Him, we are called to stand for what is right and true, as God urges Jeremiah to do.
Finally, we are to take the love and trust we have and do as Jesus does, speaking the uncomfortable truth even at risk to our own safety.
\end{document}