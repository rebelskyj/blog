\documentclass[12pt]{article}[titlepage]
\newcommand{\say}[1]{``#1''}
\newcommand{\nsay}[1]{`#1'}
\usepackage{endnotes}
\newcommand{\1}{\={a}}
\newcommand{\2}{\={e}}
\newcommand{\3}{\={\i}}
\newcommand{\4}{\=o}
\newcommand{\5}{\=u}
\newcommand{\6}{\={A}}
\newcommand{\B}{\backslash{}}
\renewcommand{\,}{\textsuperscript{,}}
\usepackage{setspace}
\usepackage{tipa}
\usepackage{hyperref}
\begin{document}
\doublespacing
\section{\href{reflections-on-readings-3-advent-b-23.html}{Reflections on Today's Gospel}}
First Published: 2023 December 17

\section{Draft 2}
I love this set of readings, because each of them speaks to me in a way that I don't normally feel from the readings.
Intellectually, I know that every word in the Bible is beautiful, and each verse leads me closer to Truth.
Emotionally, though, I don't always connect.

I'd like to start with the Psalm.\footnote{shocking, I know, given that I never write about the Psalm}
The Psalm today really encapsulates the entirety of the readings.
It is the Magnificat, the Canticle of Mary.
That is, it is Mary's response to seeing her cousin Elizabeth while both are carrying their child.

How does this sum up the readings?
First, the Magnificat comes when John the Baptist leaps in his mother's womb.
The Gospel today is John's ministry and preparation for the coming of the Christ.

Second, it is a prayer.
The second reading exhorts us to pray unceasingly.

Finally, it is a song of joy and thanks to the Lord.
The Magnificat is a message of joy and hope to the downtrodden and despised.
The first reading, from the Prophet, also focuses on this.

The second reading is really just filled with a load of fantastic single lines.
I could spend pages discussing more or less every phrase in the reading, but I'll focus on one in particular.

St. Paul exhorts us to test everything, retaining what is good.\footnote{1 Thes 5: 21}
In a conversation with friends yesterday, I was once again reminded that this approach is not the way that everyone is formed in their faith.
In a different conversation, someone explained to me that questions can come from a place of judgement or of curiosity.
I think that these two conversations point to the same idea: we either seek to find the truth or confirm what we know.

In raising a child, I cannot understand why you would want them to trust unquestioningly.
I am reminded of a verse, where we are told that we are to have faith like children.
Every child I've ever known has wondered endlessly.
When they ask why something is true, they then wonder why the response is true.
Of course, there is an element of trust there.
A child believes what you tell her.

And so, we are reminded that we must have a faith like that.
I have hope that the Pharisees in the Gospel were questioning John on those lines.
If someone today preached that they had a new way of forgiving sins, I would be incredibly skeptical.
How much moreso would it have been for the faithful of the first century.

It is a common statement by Catholics that the Church is either the Truth and Holiness, or the most vile blasphemy possible.
We see reminders of that in the Gospel.
John is baptizing in the name of the one who is to come, the one who can forgive all sins.
It becomes a fulfillment of the Prophets first reading.

Christ's ministry was one of healing the ill and uplifting the poor.
Throughout the history of the Church, it has spread most by the downtrodden.
Even today, many of the conversion or reversion stories I hear come from someone who was in a terrible place.

The Prophet reminds us that we are all called to this ministry.
The Lord anointed him, sending His Spirit to rest on Isaiah.
We receive that same Spirit in baptism, and we are sealed in it during confirmation.

Daily Reflection:
\begin{itemize}
\item Hobbies:
\begin{itemize}
\item Did I embroider today? Embroidery remains trapped in the office. However, the person who got the first embroidered piece liked it.
\item Did I play guitar today? I think that I'm starting to like the higher harmonics twang of the new strings.
\item Did I practice touch typing today? SHoot!
\end{itemize}
\item Reading
\begin{itemize}
\item Have I made progress on my Currently Reading Shelf? I finished the first audiobook and started the second.
\item Did I read the book on craft? I read a little further and then immediately got inspiration, which is common in the book.
\item Have I read the library books? Still no.
\end{itemize}
\item Writing
\begin{itemize}
\item Did I write a sonnet? I forgot to write one yesterday, I think at least. That's a shame. Today's was pretty good, though.
\item Did I revise a sonnet? I had some fits and starts, which meant that I revised as I went.
\item Did I blog? I find that the second draft is way cleaner than the first, which makes some amount of sense.
\item Did I write ahead on Jeb? I wrote another quarter chapter, since I didn't want to do too much writing on a Sunday.
\item Letter to friends? I chatted with someone else who writes letters today. While working on my computer, I realized that I want my writing to count for the arbitrary word goals on this app, which might be part of why I struggle to get letters written. Given that the site explicitly encourages using the word counts as rewards for other tasks, though, maybe it's worthwhile to just tell myself a letter is worth 250 words and then lorem ipsum it?
\item Paper? I started writing an introduction to rotational spectroscopy, then realized that I don't know how to frame it. Some ideas include: imagine how things rotate normally, now imagine a molecule; imagine a molecule, now think about it rotating; there's light in space, why?
\end{itemize}
\item Wellness
\begin{itemize}
\item How well did I pray? Not great, but I hope that I can start to do better.
\item Did I clean my space? A little.
\item Did I spend my time well? Honestly pretty well! I got through a lot of what I wanted to do.
\item Did I stretch? No.
\item Did I exercise? I played basketball with the religious ed kids, which was fun. Turns out basketball is much easier when you have a sense of proprioception.\footnote{unsurprisingly, I guess}
\item Water? I actually think that was something I did well on today, for once. I feel like I drank an appropriate amount of water through most of the day.
\end{itemize}
\end{itemize}

\section{Draft 1}
This week's readings, in a shocking turn of events, are not a struggle for me.
Do not get me wrong.
I do not sit easily with these readings, completely unaffected.
My struggle with these readings is simply in how poorly I live them out.

The first reading comes from the Prophet.\footnote{Given that in the Gospel today, John calls him Isaiah the prophet (note lower case), I wonder whether I'm supposed to not call him the Prophet either.
Not worth my time to look up right now, but it is something that I'm curious about, so should look up at some point (I just know that I'm distractable enough right now that I would go down at least two rabbit holes before returning to the musing}
It opens with a really powerful verse,\footnote{Is 61:1} where we are told a few important facts.

In brief, the Lord anoints us and sends His Spirit on us.
When we are given the Spirit, we are given a mission.
In the case of the Prophet, the mission is to lift up the downtrodden, and a few specific examples are given.\footnote{yes, I recognize that they all apply to the people of Israel, but words can have multiple meanings, and I'm allowed to take the meaning that I want from a reading (as long as it lies within the acceptable interpretations, and as long as I don't lead anyone to error in doing so. Not acknowledging one of the interpretations of the reading doesn't fit there, I don't think)}

Of course, we as baptized are anointed and given the Spirit.
We too have a mission.
As Christ reminds us, everything that Isaiah said was his mission in this verse is also a mission we share.
We are to comfort the grieving, set free the prisoners, and strive for justice and liberty.

The second verse is where it shifts outside of the conventional modern Christian framework, though.
We are told that Isaiah was sent to proclaim a year of Jubilee.
Jubilee is a concept that has completely disappeared from the modern consciousness, for all that it is\footnote{in my opinion, at least} an absolutely essential component of a healthy society.
In a year of Jubilee, all debts are forgiven.

Now, there are some obvious reasons that we do not have this concept anymore.
First, we legally treat corporations as though they are people, and that has bled into the way we discuss things.
Copyright\footnote{I promise we're not getting into one of my internal rants about copyright, for all that I do feel like the Church's implied teaching is contrary to the understanding that most people I know have.} laws are described as letting corporations own the rights to their work, rather than artists.
There is no real reason that a corporation should have its debt forgiven, for much the same reason that there is no reason that a corporation or country having debt is inherently a problem.

The reason that debt is a problem to fall into is that we all know that there is a maximum amount of time that we will work.
At some point, we will stop producing, and there will be no new income to pay off our debts. 
Corporations\footnote{I'm going to switch to calling a country a corporation, because it is in many regards}, by contrast, can continue to cycle out workers indefinitely, and so there is no inherent timeline wherein they stop producing income.

Other than the fact that we treat corporations as people, though, the only reasons I've seen for not instituting Jubilee are based in the goal of raising corporations above people or in the goal of keeping oppressed down.
If we forgive debt, the most common narrative goes, then the companies which issued the debt will be in trouble.
Or, as comes up so frequently in discussions about student loans, if we forgive debt, then that just encourages people to take out more loans and live above their means.

In part, that is because of the system we have created, where student debt in particular is nearly impossible to discharge.
Even if we did not have a Year of Jubilee, and instead just stopped debts from reaching a point that they are impossible to pay off, then we would see a radical shift in the structure of modern economies.
Sorry, didn't mean to go on a rant here, returning to the readings.

The second reading reminded me of a conversation that I had with friends yesterday.
We were discussing, among other things, our own relationships to the faiths that our parents had raised us with.
One stark difference between us was that I was raised with the belief that, not only was it ok to question what we are taught, it is actively good.
It's always nice when the Bible supports the take that you have.

St. Paul writes that we should test everything, retaining what is good.\footnote{1 Thes 5: 21}
I do believe that statement would do wonders in helping the world as we know it.
However, it is a difficult statement to practice.

How do we know what is good?
Even outside of the concept of forming a conscience, there is the issue of large scale differences.
What is good for an individual may be bad for society, or the reverse.\footnote{I'm here taking the case of things which are explicitly beneficial for any individual who does them, but harmful for society, and the reverse. Things which help some people and harm others are in the forming conscience}
How should we balance the conflicting needs of a group to have its members in good condition and each member to live to his or her fullest?
I'm not really sure, and that's just an immediate issue that comes to mind with the question.

However, the existence of edge cases does not preclude heuristics.\footnote{I think that I need to get that message emblazoned somewhere in large letters.
I guess it's a little too formal for a lot of discussion, but it's certainly something I need to remember more.}
That is,\footnote{ah, the good old id est (literally Latin for that is, often abbreviated i.e.), also known as \say{I did a bad job explaining but I like the way the words taste, so I'm going to keep them but try again for something comprehensible}} the fact that there are times that a general principle does not apply does not mean that it doesn't work in general.
If I say, for instance, that children are short, \say{all} is absolutely implicit in that statement.
Even if we find a child who is tall\footnote{also what is tall?}, that does not mean that most children do not remain short.

And so, there are obviously cases where we can look at things and see that they are harmful.
Modern psychology agrees that believing that you are unable to improve yourself harms you.
Going into the spiritual, believing that to question your faith is to not believe it will lead you either to an unfulfilled life, where you cannot say what you truly believe, or a life where you lose your faith as you question it.
Any other examples I can think of are too emotionally charged, and so I'll leave them as an exercise to the reader.

Where were we?

Right, the second reading reminds us that we are to confront the beliefs that we have, making sure that they lead ourselves and the world to the Lord, our G-d.

I normally skip the Psalm, but I love the Canticle\footnote{I love the word canticle so much, because it means basically what you'd expect it to mean: something that's kind of like a song but not really} of Mary, which were the verses.\footnote{I don't think that verse is the right word, but I don't know if I've ever learned a different one}

And finally, we make our way to the Gospel.
I think that this was the first year that I realized that John was doing his preaching while Jesus lived his own private life.
As much as I know and truly believe that Christ was born around 30 years before he died, the lack of anything between his childhood and public ministry sometimes makes me forget that they must have happened.
While John was preaching his message of repentance, Christ was working in his earthly father's shop.\footnote{I assume, given that he's called a carpenter and that most people would have been working in the family business in those days I think}

There's absolutely something worth meditating on there.
First, there's the statement that Christ makes, which is that whatever we do to the least of people, that we have done to him.
I can imagine someone going to baptism in the Jordan, and then seeing Christ in his home.
Somehow, they cannot see that the Lord is incarnate in front of them.

More than that, though, it is a reminder that everything has its season.
If Christ had begun his public ministry in the Temple when his parents found him as a child, we would not have had the St. John the Baptist.
That would have meant that a number of prophecies went unfulfilled.

And, of course, it is here that we see the Pharisees doing as St. Paul commands.
They hear that a man is baptizing and preaching repentance.
Unsure what that means, they go to question him.
John replies honestly.

He is not Elijah, nor the Prophet, nor the Christ.
He is simply the voice in the desert\footnote{the change in punctuation still bothers me, but I haven't cared enough to look up why, so I guess it doesn't actually bother me that much} telling us to prepare the way.

Ok, this musing got a little out of hand, I think that I should revise it.
First, though, there's bound to be a way to tie the first reading to the Gospel.
I can't think of one, though, so I guess I'll just have to hope that inspiration strikes while I write the next draft.\end{document}