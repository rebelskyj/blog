\documentclass[12pt]{article}[titlepage]
\newcommand{\say}[1]{``#1''}
\newcommand{\nsay}[1]{`#1'}
\usepackage{endnotes}
\newcommand{\1}{\={a}}
\newcommand{\2}{\={e}}
\newcommand{\3}{\={\i}}
\newcommand{\4}{\=o}
\newcommand{\5}{\=u}
\newcommand{\6}{\={A}}
\newcommand{\B}{\backslash{}}
\renewcommand{\,}{\textsuperscript{,}}
\usepackage{setspace}
\usepackage{tipa}
\usepackage{hyperref}
\begin{document}
\doublespacing
\section{\href{board-games.html}{Board Games}}
First Published: 2022 August 22
\section{Draft 1}
This was supposed to come out Saturday, but life got in the way.

This past Saturday I spent more or less the entire day playing board games.
It was really incredible.

The day began with a 6 hour DnD session.\footnote{interrupted briefly and constantly by a small child}
It was really fun to try an officially published encounter, for all that it lacked the organic feel of homebrew.

After that, I went to a board game night.
There I found out that original rules Risk goes significantly faster.
I assume this is due to the three following reasons:
\begin{enumerate}
\item Starting territories are randomly assigned\footnote{so you don't have blocs}
\item There's no initial reinforcement phase\footnote{so especially with 6 players, it's just three tokens at a time}
\item The number of units gained from reinforcements grows quickly\footnote{4,8,10,12,15,20,25...}
\end{enumerate}
More or less, this meant that I had 15 soldiers\footnote{my allotment plus the 12 from a set} to take over Australia.
Once I did, I then gained 15 troops from eliminating someone, which spiraled out of control quickly.
I think the fact that all of us played an aggressive game also helped.

After that, we played some Anomia, which is always fun.

I then learned how to play euchre.
And, then it had been 12 hours of board games!
\end{document}