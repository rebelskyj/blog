\documentclass[12pt]{article}[titlepage]
\newcommand{\say}[1]{``#1''}
\newcommand{\nsay}[1]{`#1'}
\usepackage{endnotes}
\newcommand{\1}{\={a}}
\newcommand{\2}{\={e}}
\newcommand{\3}{\={\i}}
\newcommand{\4}{\=o}
\newcommand{\5}{\=u}
\newcommand{\6}{\={A}}
\newcommand{\B}{\backslash{}}
\renewcommand{\,}{\textsuperscript{,}}
\usepackage{setspace}
\usepackage{tipa}
\usepackage{hyperref}
\begin{document}
\doublespacing
\section{\href{nanowrimo-6.html}{NaNoWriMo 2023 Retrospective}}
First Published: 2023 November 29
\section{Draft 1}
Wow! I did it!
One day early and sixty two words over, I have officially completed NaNoWriMo number 25.\footnote{This is apparently the twenty fifth NaNoWriMo, which is really cool!}
Having now finished the story, I feel like now might be a great time to do a post mortem on the process.
 
I think that breaking it into pros, cons, and things I noted is probably a great way to do it.
I'm not sure if I want to do it in a list wise fashion or as they pop up, but it's probably a good idea to do lists, if only because I don't have to write each word successively\footnote{in that like I can write a word and then go up earlier in the document to interpend (I feel like that's the word for append in the middle, but I don't have the time to look it up right now) more words} and the three categories seem useful to me.
The good, bad, and ugly, is another way to frame this.
 
First, the good:
\begin{itemize}
\item I got to spend a lot of time writing with a friend. We tried to meet up daily, but that was a very soft goal.
Still, I think that we ended up meeting up at least half of the days this past month, including twice online.\footnote{one of those two times is right now.
It's wild how much simply having someone on the other end of a Discord call does to keep me productively writing}
I really enjoy parallel play\footnote{doing an activity next to someone also doing an activity}, especially with this friend, and I hope that we continue to do something similar in the future, for all that I know that it will be different now that NaNo is over.
I also know that I had a much easier time simply letting bad words flow onto a page than my friend did, and I worry that it might have been discouraging to write next to me.
\item I proved that I can, in fact, write a full\footnote{if unsatisfying to me (this belongs in con, not footnote, oh well} story in the course of a month.
Even while juggling writing this blog every day, keeping up\footnote{and even getting ahead at first} on Jeb, and doing my other writing in a day, I was able to keep the narrative in mind.
This is my second NaNoWriMo, but last year I think that I ended up finishing the story I wanted to tell in less than fifty thousand words, so I had to extend it past where was natural.
I nearly had the same issue this year, because I had two days left in the month but only 500 words left to write to hit fifty thousand.
Deciding what I meant by doing NaNo was a difficult choice, but I'm glad I chose to have it mean that I wrote 50k, rather than wrote explicitly the book every day in November. 
\item I got to explore writing a more explicitly faith based narrative.
For all that I know I could have done a much better job of it,\footnote{see cons} I enjoyed writing something that took place in the real world, where the Church exists, and where I could frame the Church as a good, if often human, institution.
I don't know if urban fantasy will forever be the genre for me, but especially since I have historically read a lot of it, there's no reason for me to think that I won't continue to do it at least a little. 
\item I got lots of practice with first person. Almost everything else I write is in a third person so limited as to be almost first person, but the explicit difference is nice.
I think that I could have been more intentional about the form, but I think that what intention I did use was nice. 
\item I got to practice plotting and then following the\footnote{admittedly very loose} plot.
I found that setting a loose story goal early on and then filling it in each day before writing worked really well.
I might do something similar with Jeb, since I currently have the one sentence part. 
\end{itemize}
 
The less good:
\begin{itemize}
\item I think that the book idea I had was better served as something that I wrote slowly and thoughtfully, rather than in the fairly rushed way that NaNo requires and expects.
I don't really have anything to add to this. 
\item I don't know if it's that, or if it's something else, but I didn't really love the book.
It might have been better as a short story.
For all that it was a cool premise, the one sentence synopsis of the book\footnote{imagine if there was a Catholic Order of werewolves with the charism of fighting vampires} was really about as good as I ever got.
I don't really know if anything I did in the book deserves a longer one.  
\item I don't think that I did a good job writing in first person.
I think that I struggle to put thoughts to page in a cogent fashion, and that absolutely came up here.
I couldn't keep the character's thoughts separate from what mine would be in the situation, not that I really tried. 
\item I think that I used a lot of cliches and did far more telling than showing, to the detriment of the story.
\item I don't think that I did a good job reflecting my faith in a positive way. The Catholicism was for more surface level to the story than I would have wanted.\footnote{of course, that probably ties to the first of the bad} 
\end{itemize}
 
The things I noticed:
\begin{itemize}
\item Every day it was far harder to start writing than to actually write.
I ended up adopting the motto of \say{the only way out is through and the only way through is forward,} which I typed at the end of each day's plotting session.
\item People get surprised when you tell them that you're doing NaNoWriMo, far more than when you tell them you have a blog or put out a web serial. Not sure what's up with that.
\item I have trouble writing specific sensory cues. This ties into the next item, so I'll talk about it more there.
\item I became much more aware of the way that books I read this past month were written.
The biggest difference between my writing and more or less everything I read was the sensory cues.
 
There's a concept of how well someone can visualize something in their mind.
It's a spectrum, as so many abilities are.
At one end of the spectrum are people who can close their eyes and see the world in, if anything, fuller color than with their eyes open.
At the other end, there are people who see nothing when they close their eyes.
 
I've always considered myself more on the former end of that spectrum.
However, as this month has progressed, I've begun to realize that there's a difference between what I can do and what I do do.\footnote{earth shattering revelation, I know. I don't do everything I'm capable of at every instant of every day.
If you thought I did, I'm sorry to burst your illusions (using you feels different now that I receive comments from at least two readers fairly frequently. I promise I'm not targeting you with the footnote but non-parenthetical here)}
Visualizing takes mental effort for me.
When I don't put forth the effort, things do exist in a void.
 
Obviously, this shines through in my writing.
Without meaning to, I simply elide through physical descriptors, simply because they are not relevant to the scene at hand.
It's something that a lot of commenters have noticed in my web serial, which is likely one reason that I have noticed it myself.
I know it's something that I should work on, especially if I want to really improve my craft.
 
I don't know quite how to fix it, so that's probably going to be one of the priorities in the coming months
\item I realized that I have finally shifted from being limited by the speed I can compose content in my brain to the speed that I can type.
It's kind of fun that the rate limiting step for me is now my ability to actually type.
 
Related enough that it doesn't get its own entry, I've realized that I don't actually know how to touch type properly.
I've decided to start actually learning for real, which will hopefully go well.
\item Having one writing project makes doing other writing projects easier.
There are flavors of productive procrastination to this, but I do really think that a large part of it is that once I've written one thing in a day, the next thing I need to write is easier because I don't have to convince myself that I like writing before beginning the next one.\footnote{yes, each day I do have to reconvince myself that I like to write.
Yes, that's as exhausting as it sounds, if not moreso honestly}
\item A consequence of becoming more comfortable writing is that I am becoming less and less satisfied with the way that I write, even as I start getting more external praise for the writing.
I don't know if this is a \say{we're near the move from conscious incompetence to conscious competence,} thing or just a \say{I'm now good enough that I see how much room I have to grow}, or what, but it's kind of fun.
I think that the main gripe I have with my writing right now is that it's rambly.
As I mentioned recently, I'm going to start working on craft in the upcoming months, and I hope that one consequence of that will be words that taste better not just one by one, but when taken as a whole. 
\end{itemize}
 
Now that I've done a quick and dirty reflection of how I feel about my NaNo this year, I'll go through some of the questions that I can imagine being asked about doing it\footnote{and since I'm still on the voice call with my writing friend, may ask the friend for some questions}:
\begin{itemize}
\item What is NaNoWriMo?
National Novel Writer's Month.
It's every November since 1999.
The goal is to write a full\footnote{50,000 words, which is shorter than most modern fantasy, but well within the realm of most classical fiction} novel in the span of the month. 
\item How did you come up with your story idea?\footnote{thanks friend for the question!}
I read a text post somewhere talking about dos and don'ts for including the Catholic Church in a work of fiction, especially urban fantasy, and was inspired.
\item How did you feel about it?\footnote{thanks again}
See above. 
\item Why did you choose to do this?
Great question!\footntoe{do I see any irony in only calling my question a great one? nope} I think that really I wanted to try to stretch myself this year. 
\item Do you plan to do anything with the novel?\footnote{an adapted thanks}
Probably not, at least for the foreseeable future.
While I think that there's something worth salvaging in the book, it's deep beneath a lot of crud I'll need to clean up, and I don't know that I want to do that right now.
I have the rest of my life, though\footnote{and to my future heirs, assuming that I am no more and you have a desire to edit it, go for it if I never got around to it and this blog post and the book for some reason still exist}.
\item How do you do it?
I have gotten good at sitting down in front of a computer and just typing.
I find that it helps to have a minute or so of free journaling at the start of each writing session, where you just open a new note file and start typing exactly what you're thinking, then slowly try to redirect your thoughts to the book.
At that point, take another minute to quickly plot out what you want to happen, and then start writing.
 
As an example:
 
Let's see, it's about half past 2000 right now and I don't really have anything else on my plate today, thankfully.
Once I finish this blog post I'm more or less done for the day, though I do need to figure out the title.
If I was actually going to be writing right now, I might be doing some planning for a short story I want to write.
A reason I don't want to write is (I don't have one right now)
 
I find that writing like that really helps me kill the voice in my head that tries to stop me from typing, for all that the fiction I write is written far less informally.
\end{itemize}
 
If you have other questions, please feel free to ask, and I'll respond privately.
If I feel like it's worth revising the blog over\footnote{i.e. if I have the mental headspace to revise the blog}, I'll update it with the new questions.
Thanks for following me on this journey!
Hopefully this blog will continue. 
 
Daily Reflection:
\begin{itemize}
\item Did I write 1700 words for NaNoWriMo? I did not today, but I finished NaNo\footnote{as you saw in the rest of this post} this item gets to be deleted for the next year or so! 
\item Did I write a chapter of Jeb? I did not finish it last night, but did finish it when I woke up early this morning and couldn't fall back asleep. 
\item Did I blog? I feel like today's was at least a little more coherent. 
\item Did I stretch? Nope! 
\item Am I doing better at prayer than a rushed and thoughtless rosary? I made it through like 2 mysteries last night without getting distracted, but made it through the full rosary, even if quickly. 
\end{itemize}
I\end{document}