\documentclass[12pt]{article}[titlepage]
\newcommand{\say}[1]{``#1''}
\newcommand{\nsay}[1]{`#1'}
\usepackage{endnotes}
\newcommand{\1}{\={a}}
\newcommand{\2}{\={e}}
\newcommand{\3}{\={\i}}
\newcommand{\4}{\=o}
\newcommand{\5}{\=u}
\newcommand{\6}{\={A}}
\newcommand{\B}{\backslash{}}
\renewcommand{\,}{\textsuperscript{,}}
\usepackage{setspace}
\usepackage{tipa}
\usepackage{hyperref}
\begin{document}
\doublespacing
\section{\href{species-counterpoint-3.html}{Species Counterpoint Progress}}
First Published: 2022 December 28

\section{Draft 1}
What a coincidence, apparently I need one week to do each counterpoint, at least so far.
Anyways, time to move on from \href{species-counterpoint-2.html}{second species counterpoint} to third species.

As a reminder, the rules are:
\begin{itemize}
\item Consonant downbeats\footnote{3,5,6,8}, with preference to imperfect consonances\footnote{3,6}
\item No note to note parallel perfect unisons\footnote{beat 4 to beat one here}
\item No hidden (parallel) perfect intervals\footnote{successive downbeats (or upbeats) need to not be the same perfect interval}
\item Offbeat disonances must be approached and left stepwise\footnote{C-B-C or C-B-A, as two examples}
\item Four notes written per note in the cantus firmus
\item Generally prioritize small skips and step-wise movement, with octave as the exception
\end{itemize}

Shockingly, third species did not feel significantly harder.
On a first pass, I only made my \say{fourths are consonant} mistake once, and the revision I did focused more on having a meaningful melody, which was nice to be able to have.
I am excited to see if this helps me with composing next week.
\end{document}