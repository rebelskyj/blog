\documentclass[12pt]{article}[titlepage]
\newcommand{\say}[1]{``#1''}
\newcommand{\nsay}[1]{`#1'}
\usepackage{endnotes}
\newcommand{\1}{\={a}}
\newcommand{\2}{\={e}}
\newcommand{\3}{\={\i}}
\newcommand{\4}{\=o}
\newcommand{\5}{\=u}
\newcommand{\6}{\={A}}
\newcommand{\B}{\backslash{}}
\renewcommand{\,}{\textsuperscript{,}}
\usepackage{setspace}
\usepackage{tipa}
\usepackage{hyperref}
\begin{document}
\doublespacing
\section{\href{reflections-on-readings-baptism-c.html}{Reflections on Today's Gospel}}
First Published: 2019 January 16

Isaiah 40:1: \say{Comfort, give comfort to my people, says your God.}

\section{Draft 1}
Today's readings, as are to be expected from the day of the Lord's baptism, are rather joyful.
To me, the line which best sums up this week's message is the first of Isaiah Chapter 40.
The readings focus on the comfort and knowledge that the Lord has come, loves us, and chooses us each and every day.

In the Gospel itself, I love the fact that Luke makes it clear that \say{the Holy Spirit descended upon him in \textbf{bodily form},}\footnote{Luke 3:22A}\,\footnote{emphasis mine} while the other two synoptic Gospels only state that the Spirit descends.
For me, at least, part of the faith that is so important is the physical nature.
As we're taught about sacraments, they're physical manifestations of internal realities.
Unlike other Christian faiths, we believe that the Body and Blood is truly present in the Eucharist.
\end{document}