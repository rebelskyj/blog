\documentclass[12pt]{article}[titlepage]
\newcommand{\say}[1]{``#1''}
\newcommand{\nsay}[1]{`#1'}
\usepackage{endnotes}
\newcommand{\1}{\={a}}
\newcommand{\2}{\={e}}
\newcommand{\3}{\={\i}}
\newcommand{\4}{\=o}
\newcommand{\5}{\=u}
\newcommand{\6}{\={A}}
\newcommand{\B}{\backslash{}}
\renewcommand{\,}{\textsuperscript{,}}
\usepackage{setspace}
\usepackage{tipa}
\usepackage{hyperref}
\begin{document}
\doublespacing
\section{\href{book-review-gideon-ninth.html}{Book Review of Gideon the Ninth}}
First Published: 2022 October 21
\section{Draft 1}
I'm still figuring out how to do these book reviews.
I think maybe a good, bad, other style could be good?
I haven't done that here, but there's always next time.

In addition to reading \href{book-review-atomic-habits.html}{\textit{Atomic Habits}} yesterday, I also finished a book that I was reading with a friend.\footnote{well, in theory at least}
That book was Tamsyn Muir's \textit{Gideon the Ninth.}

I'd heard a lot of praise for the book from different people unprompted, so I was curious how reading it would go.
Overall, I enjoyed the book a lot, for all that I still have no idea what happened.
The book begins, not so much \textit{in media res} as \textit{in medium mundum}.
Our main character is trying to escape captivity.
She has dreams of joining the military, if only to get off of the planet she's on.

Very quickly, we learn that this universe is one of necromancy and inter-planetary travel.
Gideon (our intrepid hero) is not a necromancer.
The only other girl on the planet\footnote{so not one of the elderly}, is, however.
The two fight about whether or not Gideon should be allowed to leave the planet, and it ends up in a duel.

Shockingly, the girl with a sword cannot beat the girl with a seemingly infinite number of skeletons in combat.
For losing, Gideon is forced to attend a meeting, where they learn that their God Emperor wants new immortal killing machines.
Both think this sounds like a great idea,\footnote{after enough duress}, and they fly off to a planet to meet the other houses.

From there, there's intrigue, romance, and violence.
Becoming an immortal necromancer stops seeming quite as wholly good and starts to seem more morally grey as the book continues.\footnote{shocking, I know}
All in all, it's an enjoyable read, with a couple of caveats.

There's a trend I've noticed as I've started reading more and more self-published/first works.\footnote{this book isn't self-published but is the author's first book}
Either the text contains literal paragraphs of explanatory text \textit{a la} \say{A gronkle is the currency in the realm. It is worth twenty two shenks, which are approximately one inch in diameter. One shenk is worth about as much as a single mana crystal, which is powerful enough to cast one fireball from. Unlike the gronkle-shenk relationship, which is enforced by the King (btw we're in a total monarchy)...}, or there's a glossary/index that we're supposed to refer to while reading.
Gideon is the latter half, at least for me.

I'll take some responsibility in this.
I often struggle to keep characters apart in books, but the author does just go \say{here's sixteen new characters. Anyways, from now on I'll be referring to them by name and where they come from interchangeably.}
However, the experience was still really enjoyable.
I'm excited to start the next book in the series.
\end{document}