\documentclass[12pt]{article}[titlepage]
\newcommand{\say}[1]{``#1''}
\newcommand{\nsay}[1]{`#1'}
\usepackage{endnotes}
\newcommand{\B}{\backslash{}}
\renewcommand{\,}{\textsuperscript{,}}
\usepackage{setspace}
\usepackage{tipa}
\usepackage{hyperref}
\begin{document}
\doublespacing
\section{\href{science-communication.html}{On Science Communication}}
First Published: 2025 April 2

\section{Draft 1: 2 April 2025}  
I'm wondering if the daily reflection I have might be too long. It's great and all, but spending twenty minutes every day on them feels like a lot.  
Then again, it isn't as though I really need the twenty minutes for anything.  
Time I choose to spend is time well spent.  
That's probably a good thing for me to keep reminding myself.

Onto the actual meat of this post.  
As I have mentioned here before, my current career plans have me going into science communication in some form or another.  
One of the leading voices in the field, at least as far as I know, is a Professor Emeritus at my school.  
I met with him a few weeks ago to get his advice on finding a career in the space.  
He gave me a lot of good advice, and in particular gave me a few questions that are important for me to answer.

This is a space for me to try to think of some initial answers to the questions.

His first thing to remember, though not a general question, was what the purpose of any given demonstration is.  
That is, what am I trying to communicate, and is what I have prepared effective for that task?

The two biggest questions were really, \say{why should I care about science?} and \say{why should my tax dollars go to funding (insert research that does not seem to improve someone's life)?}\footnote{e.g. molecules in space}  
I'm going to treat them as the same question, at least to start, but will keep in mind that they are not intrinsically the same question.

So, why should a random person care about science?

At a visceral level, I cannot think of an answer to this question.  
There's a meme that goes around sometimes that says something along the lines of \say{if you don't care about other people, I don't know how I can convince you to.}  
It feels similar to my feelings about science.  
Why should someone care about science feels like as ridiculous a question as why someone should care about others.  
Still, that's not a good way to win any hearts or minds, so let's fight past our own reactions.

My general life view says that we are all born interested.  
Before we know how to divide the world into science and art and language and sport, we simply absorb what's around us.  
I don't think that I have ever met a child who was not at least nominally interested in anything, when framed correctly.

Life, or to be more precise, the systems that we all live in, beat the joy out of learning and exploration for so many people.  
When I was in college, a frequent question I would ask my teammates and prospective students looking to join the team what hobbies they had outside of the sport.  
A number of them had no answer.  
This used to surprise me, until I learned more about the difference between going to school in rural versus urban America.

In rural America, it is not only acceptable, it is almost expected to be somewhat of a dilettante, at least in my experience.  
There simply are not enough people in the entire school to be able to segregate groups into a single activity.  
There were of course general trends and class divides, even if I did not see them at the time.  
After all, if you need to work every night to put food on the table, you aren't going to be able to do every extracurricular.

Almost every athlete did multiple sports, especially the top ones.  
The only exceptions to this that I can think of are some swimmers and soccer players, who joined club teams during the off season.  
In part, I have to assume that's because the programs were all no cut.  
If you showed up and put in work, you were guaranteed a spot in the team, and as far as I can tell, everyone generally had a chance to even compete.  
You might not end up on the varsity squad, but playing JV is still doing a sport.

I compare this to my friends' experiences going to schools in larger cities.  
Starting from early middle school, school sponsored activities begin actively cutting students.  
If you didn't know that you wanted to be a swimmer before entering high school, it is suddenly too late to do so.  
Even once you're on the team, though, you aren't safe.  
The fact that cuts exist at all means that there is a constant existential need to focus on the single activity.

I can relate to that now.  
The looming deadline of my thesis does make it harder for me to do other activities, because I am constantly asking myself if the time would be better spent doing something else.  
I feel beyond blessed to have ended up going to a small college, which gave me the chance to continue being an amateur.\footnote{Ok so in retrospect this is just my whole society needs to let us be amateurs rant, which I am nearly positive that I have posted before.}

Returning to the point of the question, and hoping that I don't go on the rant again, my initial answer to why someone should care about science is that they once did, and life is better when we care about more things.  
As the PE\footnote{writing professor emeritus takes a while and I'm never sure about capitalization. By initializing, I solve both issues} stressed, the first and primary goal of science communication should always be to have a conversation with someone else.  
Too, the primary goal of a conversation should always be setting the stage for the next conversation.  
In a conversation about science, the most important thing is to share feelings, not facts.

There's a recent Pope that made a similar point about conversion.  
People are convinced by experience and emotion, not by raw facts.  
Even within the sciences, we know this to be true.  
The scientists that everyone points to as the best lecturers and teachers are the ones whose presentation styles are animated and draw the listener in.

So, if someone asks me why they should care about science, my first question to them should\footnote{here being used in the prescriptive sense of trying to dictate my future hypothetical actions} be what they care about right now.  
It is important that the question doesn't feel like an attack or a deflection, so framing it as something like \say{I don't really know you well enough to know why you should or shouldn't care about science. What do you care about right now?}  
Then, and importantly, I need to actively listen to what they say, not to find the way that I can tie their interest to science, but simply because they're a full human, and deserve to be listened to.

Somewhere I need to do a reflection on how cults work, in part, by making people feel seen, and how I can weaponize\footnote{weaponizing gets a bad rap, apparently because people use it to mean attacking someone. I don't know what the non-aggressing way to describe taking a fact and turning it towards effecting the changes I want to see on the world would be. Arguably, any time I try to change someone's mind on anything I am, in fact, attacking them} that knowledge towards good.\footnote{oh, right, leverage is the word people use. It's got such a different meaning to me though. Meanings to me and others is another post to do}

However, despite the fact that the first goal of my science communication is a conversation and the second goal is therefore to have another conversation, the third goal is to make them leave the conversation at least a little more pro-science.  
As a result, it does still become important to figure out why they don't care about science.  
I have the internal idea\footnote{which I don't know if the data would support} that people will generally tell you why they dislike or don't care about something better if you ask them tangentially related questions.  
Deep in my heart of hearts, I do legitimately believe that most people want the world to be a better place.  
At the end of the day, if I cannot agree with someone on that basic premise, I might just have to accept that I cannot communicate with them about science.

However, wanting the world to be a better place still leaves a lot of room for disagreement.  
In general, I think that the quest for knowledge is a good in and of itself, not simply as a means to something else.  
Many people, however, want knowledge to have a use, especially knowledge that their tax dollars support, and that is also reasonable.

I can't ever imagine \say{you pay far less to fund science than you do to (insert other thing)} will ever be a useful line of questioning.  
So, what else can I say that researching prebiotic astrochemistry is good for?  
An answer I probably shouldn't and won't give but is true is that it keeps the sort of people who would do that kind of work from doing something else.  
We all know enough horror stories of mad science to not take the idea of keeping scientists placated at least a little bit seriously.

Many medical students do research as undergraduates before entering medical school.  
Giving them the chance to do research of their own equips them with the tools to better understand new advances in medicine.  
In general, there are plenty of versions of \say{funding any research is good because research helps people.}

A fair counterpoint is that we could get all those benefits by studying something useful.  
This is where we get to the real issue with communicating the need for science funding: there is very rarely a way to know what discovery will be important in a hundred years'\footnote{year's?} time.  
Newton's experiments with light were nothing more than a fun diversion until quantum mechanics became relevant.  
Gauss's understanding of electricity and magnetism only became important to the average person when we began using electricity.

On the other hand, all the research we did into vacuum tubes for computing was, in retrospect, not really needed.  
We now know how to make semiconductors without these tubes, and so computing can be done with far smaller pieces.  
Without knowing what a semiconductor is, there would have been no way for people to research them.  
Without a need for semiconductors in computing, the fact that silicon can easily be made to take on different semiconducting properties is mostly irrelevant.

All that to say, the biggest reason to fund any given research that sounds silly is that we never know where important facts will be discovered.

There is also the secondary point, which is that almost\footnote{only putting this here because I like to hedge my bets. \say{they're breeding mice to have cancer} is an example for like how ending childhood cancer could be portrayed.} any research can be made to sound ridiculous when framed by a bad actor.  
I do not know if all research can be framed to sound as though it has merit, but I think any research worth doing can be.  
Then again, I think that \say{we didn't know this and now we do} is a totally valid answer.

Of course, there's the final set of arguments, which is that a lot of seemingly frivolous hard science is in fact just basic science.  
Exactly one person needs to measure most things in chemistry and physics, and then everyone else can use that information forever.  
I do not need to take the gas phase rotational spectrum of water to know if it's in my sample.  
The more tools we have, the better we can build.

So, I guess that one of my answers does really boil down to the fact that having more tools available is always better, even if using more of them isn't better for any specific case.  
Without looking into the future, there is no way to know what problems tomorrow will bring, and so there is no way of knowing what information will be needed then.  
Even moreso, without knowing what question to ask, we cannot hope to find an answer.

\section{Daily Reflection: 2 April 2025)}  
N.B. Since I've realized that I will often give up halfway through a post, I've decided that I'm going to start each post now with the daily set of reflections.

\begin{itemize}   
\item Intentionality:  
\begin{itemize}  
\item At least hourly, stand up, drink water, take two deep breaths, and do a stretch of some sort

This went pretty well yesterday! I think I forgot about it a little while TAing, but not too terribly much  
\item Be proactive about avoiding overwhelm and when feeling overwhelmed, stop and figure out why.

I think I did ok about this. Yesterday I did feel horrible more than a few times, and simply realizing that wow I'm fundamentally just really sad about my mom is something that helped the overwhelm pass.  
\item Light a candle and read by candlelight each night. Along with this, leave all electronics outside of the bedroom and/or move them away at least an hour before bed time.

For some reason the flame in my candle really wildly swings every few seconds, which made reading a bit of a pain.  
I did, however, remove all the electronics from my room and read before bed, which was really peaceful.  
\item Candle time in the morning before electronics. Use the time for prayer  
Did not have candle time this morning, but I did also realize that there's no real point to my morning alarm, because I would rather be well rested than up early, so I've now set the alarm such that it is only for the absolute latest that I can get up and not miss anything.\footnote{8am, because I have an 830 appointment on Saturday}  
I did try to pray the chaplet to St. Michael, then remembered that I don't have it memorized. Will print it out to solve that issue  
\item Focus on good posture, especially straight back and making sure that neck isn't awkwardly positioned.

In general I've been trying, but it does really feel like it puts a lot of pressure on the part of my back just below my ribcage, which I'm not sure about the reason for or health of. I should look that up tomorrow

\item Don't waste time, and in particular, be mindful about making sure to take breaks and rest. Especially make sure to do rest which revitalizes the me of tomorrow, rather than rest which simply keeps me in stasis.

I did 45 minutes of stretching yesterday when I got home from grocery shopping, and it did do a lot to revitalize me before bed.  
That feels a little strange, but I mean that it took me from feeling exhausted and run down to just like sleepy, if that makes sense.

\item Interpersonal Relationships:  
\begin{itemize}   
\item Figure out what belongs in a normal letter to a friend.

I have requested a number of books on etiquette from the 20th century, and we will see if any of them end up being useful for what I need.  
\item Get back into writing letters.

Going to my cell in the library is a bit of a journey, but it's also something that I know beyond the shadow of a doubt is good for me, so that's a difficult tradeoff. It might help me to start actively working to spend time doing the reading I need to do in there.  
\item Work to message friends at desired intervals.

I still need to make the list of friends. That might be a tomorrow task, though.  
\end{itemize}
\end{itemize}  
\item Professional:   
\begin{itemize}   
\item Do the Thesis and other research requirements. Upcoming deadlines:  
\begin{itemize}  
\item Brain dump about science communication (Overdue)  
\item Brain dump a publicly accessible chapter (Overdue)  
\item Have final convergences for the results I'm trying to reproduce (due 4/4)  
\item Draft of the first paper (due end of month, but I want to make sure that I've reupdated it sooner than later)  
\end{itemize}  
\item Only do the work I feel called to when I've finished the tasks set to me for the day or outside of normal working hours (post 1725)  
\item Start making the giant citation document so that I don't have to search for citations later.  
\item Work towards future career:   
\begin{itemize}   
\item Do the reflections that were recommended to me (mostly focused around why I care about science communication. See above  
\item Figure out the difference between my public-facing and field-facing presentation affects. As I focus on becoming a better presenter, I need to become aware of the difference and how to switch them.  
\item Need to look for jobs  
\end{itemize}   
\end{itemize}   
\item Health:  
\begin{itemize}   
\item Spiritual:   
\begin{itemize}   
\item Get back into the Lenten goals (pray chaplet of St. Michael, give money equal to amount I'm spending on myself, stop scrolling social media, stop playing video games)

I did try to pray it the chaplet this morning, but I forgot most of the parts.\footnote{including, funnily enough, the number of choirs of angels. I assumed it was 7, because seven shows up so very often. Turns out it's nine. I do still also have the critique (unposted) about Archangel Uriel, but that's something to think about on another day.}  
\item Be intentional about prayer. That means both making time for prayer and actually doing it.

I prayed a little before bed and a little this morning, which is progress. I'm also being better about praying before meals again, which is good for me.  
\end{itemize}
\item Physical:   
\begin{itemize}   
\item Start focusing on posture again, especially while sitting.

As mentioned above, I do feel like this makes it harder for me to breathe. Will have to look up why that is and figure out if it's that those muscles are too weak or what.  
\item Go to group fitness classes more regularly and more often.

Yesterday I instead, as mentioned above, did a self guided workout for about 45 minutes while catching up on some content creators whose content I enjoy.  
It was pretty nice, but I did not push my core or other parts as much as in a normal class.  
Then again, I let myself spend more time in many of the poses, which was also good.  
I did find that my mat is stretchy, especially on movements\footnote{poses?} like downward dog.  
It's wild to me that I can touch my toes for the first time maybe ever, but certainly since college started. It feels nice  
\item Feed myself simply and healthily. Healthy here means trying to generally avoid processing.

I've been eating oatmeal for breakfast on work mornings, and it's been generally good for keeping me going through the day.  
I have even been generally good about remembering to bring in some frozen blueberries or mixed berries, which add health, taste, and color!  
\end{itemize}

\item Mental:   
\begin{itemize}  
\item Clean Life:   
\begin{itemize}
\item Remove dirt and clutter from physical spaces (standard definition of clean):

I'm still working to play catch up on this, but I have been trying.  
This morning I cleaned up some clutter, and if I have energy tonight, I will do so again.  
The hard part is absolutely that I have too much stuff, and I need to be more intentional about getting rid of things.\footnote{for example, there's no reason that I need to have dozens of empty wine bottles.
If I brew again, I am enough of an adult to buy my own bottles, and the empty space will make my life better.  }
\item At least once a week, each room has nothing on the floor  
\item At least once a week, all surfaces which are not inherently storage are cleared off  
\item At least once every two weeks, each room is vacuumed  
\item At least once every month, all non-storage surfaces are explicitly washed/cleaned  
\item At least once a week, I get rid of at least one item that I notice (meaning throw away or in rare circumstances gift or donate)  
\item Clean sight lines. Is my space set up in a way that orients me towards my goals for the space? If not, how can I make it so?  
\end{itemize}  
\item Spend time each day thinking about the goals for the day, and getting them out of my head and onto the page.

I've spend about 15 minutes on this so far today, and that doesn't count the three or four minutes that I spent planning the work I'm going to do today.  
Still, I think that it was time well spent.  
\item Continue to explicitly confront the voice in my head that says that people hate me.

Actively did so yesterday! Left an encounter with a friend feeling really angry, asked myself why, and realized that the feelings were misplaced.\footnote{I'm trying to stop calling my emotions dumb or wrong because emotions have no intellectual merit to them at all. Positive or negative, they are my body reacting to the stimuli around me. When I notice my emotions are negative, though, it is good for me to make sure that I intellectually agree with the gut reaction I had. When I don't, it's also good to try to reframe the experience, which helps shift the emotional reaction}  
\end{itemize}   
\item Hobbies:   
\begin{itemize}   
\item Reading  
\begin{itemize}  
\item Start reading and returning the library books I have.

Not so much yesterday, but that's ok.  
It remains a lower priority than the other activities I have been doing.  
\item Finish the book on mindfulness I started. (also make a list of the exercises in the book and try them out)

Read some yesterday! It was nice, and shockingly emotionally connective for me, which is always strange.  
\item Read more poetry

This is more aspirational than anything else. Still, it would behoove me to sign up for a poem of a day somewhere.\footnote{and then to actually read them}  
\end{itemize}

\item Music:   
\begin{itemize}   
\item Work on guitar

Practiced a little bit by candlelight last night! I need to bring home the exercises for guitar that are on my desk.  
\item Learn the songs that jam partner suggested and/or requested I learn  
\item Get back into the album.  
\end{itemize}   
\item Writing:  
\begin{itemize}   
\item Write poetry more often, ideally nightly.

Wrote some last night! Or, at least, wrote some words that I intended as poetry. Without strict meter and rhyming, I find that I can't really distinguish between prose and poetry at a fundamental level.  
Probably good to talk to someone about this.  
\item Find a way to add meta data to my blog posts and then add the meta data  
\item Not only write blogs, but also post them.

Did so yesterday, will do again today  
\item Get back into writing the web novel  
\end{itemize}
\end{itemize}

\item Other hobbies, do them.  
\end{itemize}   
\end{itemize}
\end{document}