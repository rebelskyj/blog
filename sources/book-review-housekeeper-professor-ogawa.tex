\documentclass[12pt]{article}[titlepage]
\newcommand{\say}[1]{``#1''}
\newcommand{\nsay}[1]{`#1'}
\usepackage{endnotes}
\newcommand{\1}{\={a}}
\newcommand{\2}{\={e}}
\newcommand{\3}{\={\i}}
\newcommand{\4}{\=o}
\newcommand{\5}{\=u}
\newcommand{\6}{\={A}}
\newcommand{\B}{\backslash{}}
\renewcommand{\,}{\textsuperscript{,}}
\usepackage{setspace}
\usepackage{tipa}
\usepackage{hyperref}
\begin{document}
\doublespacing
\section{\href{book-review-housekeeper-professor-ogawa.html}{Book Review of The Housekeeper and the Professor}}
First Published: 2022 November 21
\section{Draft 1}
I just finished rereading Yoko Ogawa's \textit{The Housekeeper and the Professor.}
The book tells the story of a housekeeper assigned to a former math professor who has suffered an accident which means he cannot remember anything more than 80 minutes in the past.
The professor is brilliant, in love with mathematics, and a genuinely kind person.
Most of the book focuses on the relationship between the professor and the housekeeper's son, who form a bond over their shared love for baseball.

Reading the book was an interesting experience for me.
The last time I read this book was more than half of my life ago,\footnote{probably} when my grandmother gave it to me.
I don't remember why she had the book, thought I would enjoy it, or what I took from it at the time, but throughout the entire reading of the book\footnote{short as it is}, I kept remembering vague snippets of my childhood.

For a book about the lack of memory someone has, it was an interesting metaexperience reflecting on my own lack of memory about my last reading of it.
Even without my own personal connection to the book, though, I did really enjoy it.
It's great as a person in science seeing a writer talking about the fundamental beauty of math, and the characters all feel real.
More than that, though, most all of the characters in the book are kind people.
Too often I find that fiction, especially realistic fiction, tries to paint a world where no one is motivated by the interest of the common good.
This book does not do that.

Overall, I give it a very biased score of 5/5, and would likely still give it a 4.5/5 even without my history.
I'd highly recommend reading it, especially since it's so short.
\end{document}