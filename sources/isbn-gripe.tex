\documentclass[12pt]{article}[titlepage]
\newcommand{\say}[1]{``#1''}
\newcommand{\nsay}[1]{`#1'}
\usepackage{endnotes}
\newcommand{\B}{\backslash{}}
\renewcommand{\,}{\textsuperscript{,}}
\usepackage{setspace}
\usepackage{tipa}
\usepackage{hyperref}
\begin{document}
\doublespacing
\section{\href{isbn-gripe.html}{Griping about ISBN}}
First Published: 2022 May 4

\section{Draft 1}
As I've mentioned now too many times, I'm currently writing a book.
I have a variety of ideas for what to do with it once I finish writing the words, but nearly all of them involve some level of self-publishing.\footnote{both because the genre I'm writing for doesn't have great mass market appeal and because I don't think it's necessarily good enough to be published.}
I learned that if you self-publish through the largest E-book distributor, the ISBN you get is associated with them.
That makes sense when I think about it for even an instant, but I've also learned that many independent bookstores will refuse to stock books with that publisher, for again, obvious reasons.

I also found out that the owner of an ISBN gains some controls over the book, which made me want to assign my own ISBN.
Of course, since we live in America, the company responsible for managing the numbers also sets the pricing scheme.

It is the most asinine pricing scheme I have ever encountered.

A single ISBN costs \$125.
Knowing nothing about ISBN, that could be a totally valid price.
10 costs \$295, or less than a quarter the cost per ISBN of buying individually.
100 costs \$575, which is less than a fifth the cost per book of 10, and less than 5\% the cost per individual ISBN.
That's so ridiculous that if you need 13 ISBN, it is cheaper to buy 100 than any other method.
Finally, 1000 ISBN costs \$1500.

For reference, that is 1.2\% the cost per ISBN of buying individually, about 5\% the cost per 10, and only slightly over 25\% of the per hundred.
When people say that it's hard to be an independent business, they never mention this, though they really should.
At 211 ISBN, it is now cheaper to buy 1000.
That's completely ridiculous.
It's a way of penalizing small creators and forcing them to go to larger publishing houses.

Anyways, that was a bit of a rant, I have no solutions, and I'm just generally annoyed by that fact.
Oh!
Any edition of a book needs a new ISBN\footnote{not unfairly tbf}, so unless you're planning on only ever releasing a single book in two editions\footnote{like paperback and ebook}, there is literally no reason at all to buy ISBN one at a time.
The option to do that is there, as far as I can tell, entirely to screw over new authors who aren't sure what they're doing.

The same company sells barcodes for \$25.
For reference, a barcode can be made online with any number of free tools in a matter of minutes.
For a single QR code, they charge \$75, which is even worse to me, since I doubt they take care of the hosting for the QR.
\end{document}