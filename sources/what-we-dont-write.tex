\documentclass[12pt]{article}[titlepage]
\newcommand{\say}[1]{``#1''}
\newcommand{\nsay}[1]{`#1'}
\usepackage{endnotes}
\newcommand{\1}{\={a}}
\newcommand{\2}{\={e}}
\newcommand{\3}{\={\i}}
\newcommand{\4}{\=o}
\newcommand{\5}{\=u}
\newcommand{\6}{\={A}}
\newcommand{\B}{\backslash{}}
\renewcommand{\,}{\textsuperscript{,}}
\usepackage{setspace}
\usepackage{tipa}
\usepackage{hyperref}
\begin{document}
\doublespacing
\section{\href{what-we-dont-write.html}{What We Don't Write}}
First Published: 2018 November 16
\section{Draft 1}
For some reason, writing that title made me think of the song by Jon Svetky, \say{What We Talk About When We Talk About Love}.\footnote{also apparently a book by Raymond Carver}
But, as I wrote a post for today, I got more than 500 words in when I realized it wasn't something I could post.

Most of the time, I know that I won't be able to post a writing before I even start it.
Or, in some cases, I get a paragraph or so in before I realize that it shouldn't be published.

But, maybe because of discussions I've had with my class in the Diary course I've been taking, I reflected today on what we don't write.
Obviously, despite \href{disclaimer.html}{my disclaimer}, I'm not going to post things that I think will reflect horribly on my in my immediate future.
So, I'm not going to write about how much I may hate a certain person, or how frustrated I am about a certain event that I have the wrong amount of control over, because it doesn't help me at all to write about it.\footnote{no, despite how it sounds, the second example is not targeted at anyone}

But, there are other things that don't go here.
As one of my classmates mentioned, we don't mention the everyday.
Of course, since I've been scraping the barrel sometimes to get a post, I do mention some of the everyday.

We don't mention what we're ashamed of.
Obviously, this is something that is different now from the typical image of a diary.
Here, I'm publishing my work with the expectation of others reading it, much like early diaries.
So, I can't put personal information that might be hurtful to someone near me, because that's unfair to them.

All this is to say, I've realized that, much as \href{http://www.cs.grinnell.edu/~rebelsky/musings/}{my inspiration} did in his third essay, I can't always publish how I feel, because it may not be appropriate to the facts and realities of relationships I have.
Anyways, before I start sounding maudlin,\footnote{hopefully before} I've realized how helpful it is for me to force myself to write something every day, and to publish it, even if I'm not ready to release it.
This has definitely been\footnote{and hopefully will continue to be} a good way for me to grow myself, both in terms of introspection and presentation.
\end{document}