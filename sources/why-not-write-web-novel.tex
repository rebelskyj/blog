\documentclass[12pt]{article}[titlepage]
\newcommand{\say}[1]{``#1''}
\newcommand{\nsay}[1]{`#1'}
\usepackage{endnotes}
\newcommand{\B}{\backslash{}}
\renewcommand{\,}{\textsuperscript{,}}
\usepackage{setspace}
\usepackage{tipa}
\usepackage{hyperref}
\begin{document}
\doublespacing
\section{\href{why-not-write-web-novel.html}{Exploring Hesitation to Restart Web Novel}}
First Published: 2025 March 26

\section{Draft 2: 26 March 2025}  
I do really love a good metanarrative, and this blog post might be a good example of one.  
I was\footnote{and am, I suppose} wondering why I have not been writing my web novel.  
I thought that I would\footnote{wow three I statements in a row} have the time, energy, and motivation to write another draft yesterday.  
Instead, 13 hours after leaving home for work, I returned.  
Everything that I planned to do at home last night took longer than I had assumed it would, and as a result I did not end up writing my chapter.

I forget who initially told me this, but someone once told me that there's no such thing as free time.  
That is, anything that I do has to come at the expense of something else that I am otherwise doing.

Somewhere else, I've seen that I need to have some amount of my time as rest.  
For a while, I think that the novel was a form of rest for me, but it absolutely stopped being one by the time that I stopped writing it.

With both of these in mind, it's obvious to me that I have been less able to do things lately than I was before.\footnote{see: grief is a huge obligation}  
If I want to be better about extending grace to myself, I need to make this a question stemming from curiosity, not judgement.\footnote{which, eh we'll see if I'm ever able to do}

So, let's go through the reasons that I might write a web novel, and see how motivating they are to me right now.\footnote{I was about to make a list but then I realized that I said I wouldn't last version}

In general, I would love if more of the things that I did were autotelic.\footnote{hmm, is that true? I suppose that I want everything that I do to be for the greater glory of G-d, and I want that to be my main motivation. If I assume that I have that motivation and that the actions I take are doing so, then an action is as though autotelic}  
Actually, I don't know if that's true.  
Some use autotelic to mean that a task is undertaken without external goals, and is therefore motivated by the thing itself.  
It gets to intrinsic and instrumental motivations.

Do I have intrinsic motivation to write?

Broader, do I have intrinsic motivations?

Even broader, is it better for me to have intrinsic motivations?

I guess things which are good in themselves are considered intrinsic motivations.  
I more and more realize that I think that I subscribe to a version of Divine Command Theory, which means that good is itself an extrinsically defined thing.  
However, since I also think that goodness is a moral constant, I suppose that practically speaking I can treat things that I do because I believe that they are Divinely ordained as good, and therefore intrinsically valuable.  
I'll even go the step further and say that I'm going to use intrinsically motivating as good in itself, where good means I think that it, on the whole, helps to bring the world towards G-d.

Ok, diversion aside, let's say that an autotelic action is an action I undertake without the explicit or implied belief that I will have an extrinsic benefit from doing it.  
For me, working out is not an autotelic action, because I exercise to stay in shape.  
Practicing my guitar is not autotelic, because I do so to be better at guitar.  
Jamming with a friend is an autotelic action, because the jamming is the goal.\footnote{or, making music is an end in itself to me.}

This helps me think about the book.  
What would it mean for me to be writing the book as its own end?  
I think that it would mean that I'm writing it to figure out where the story leads me.\footnote{I also more and more realize that I believe in some weird potentially inherently heretical metaphysics where knowledge, song, and story all exist external to humanity and we receive revelation which lets us see them, if only for a moment. I should really expound on that sometime}  
That is certainly a motivation I have.  
The initial premise for the book had the main character becoming a terrorist in the final book.\footnote{or at least a revolutionary}  
I'm less and less sure how that will happen given the way that the story is playing out right now.  
In short, I think that there is at least a small part of me that wrote the book and would continue writing the book for its own sake.

Is writing the book a way for me to hone my writing skills?  
If so, then the motivation to write the book becomes\footnote{I'm working with probabilities right now, so that's where my mind is at with separating things out} my motivation for improving my writing skills and my belief that writing the book improves my writing skills.  
Writing skills are not a monolith, however, and so I should really clarify what it means to improve at writing, at least insomuch as it relates to the book.

I believe that the primary ways the book helps me improve as a writer are that it teaches me to work on a deadline,\footnote{relevant for the thesis drafts} it teaches me to write faster,\footnote{always something that I want}, it helps me with considering a long narrative and pacing therein\footnote{wow, really relevant to the thesis}, and it helps me to understand the general human experience.\footnote{which is not, in fact, a writing skill, but is a motivation that I should explore on its own}

How motivated I am to improve in each of those regards is, as far as I can think of it, a function of how motivated I am to be good at the skill, how good at the skill I currently am, and the rate of progress that I think I will have in improving in the skill.\footnote{Wow I really think entirely in the mode that I last used my brain. Yesterday I tried to learn probabilities, and now all I can do is think about probabilities}  
This means that my overall motivation for writing the book to improve myself is the sum of the motivations to improve each skill multiplied\footnote{for some really arbitrary definition of multiply. Convolved? Functioned? Idk} with the likelihood that I will improve that skill by writing  
It can be better broken down into the sum of how quickly I think that writing the book will help me with a given skill multiplied by how motivated I am to be good at the skill, divided\footnote{again, for some arbitrary meaning thereof} by the skill I think that I have, or:

M(Write) = $\prod$ M(Be good at skill) P(Writing will improve the Skill) E(Speed of improvement by writing)  

Where M is the motivation overall, P is the probability, and E is the expectation value, which is itself a function of how easily I train the skill, how good the skill already is, how hard it is to train, and what it means to improve.  
In general, right now I do think that I am, at least consciously\footnote{there's such a ripe series of musings for me to do about what it means to fight your subconscious. I.e. if there's something you aren't doing, there's clearly some part of you that doesn't want to do it. There's a saying I see that people can lie with everything but their habits, which might be relevant here} very motivated to improve at writing.  
I think that writing the book will absolutely help me with keeping to deadlines, writing quickly, and understanding the human experience.  
I think that it will be neutral to slightly negative towards my ability to write in the scientific tone, but it might be positive at helping me to develop a specific register for writing, in such a way that I can then change it.

Ok, so the overall motivation appears positive in both regards, and I assume that if I have two positive motivations that they will add.  
Let's hope that this will continue to be true, and all reasons I once wrote the book are positive motivators.

The next reason I had to consider was that I was motivated by the idea of making money.  
That's really a few motivations hiding in a trench coat and pretending to be the same: the amount that money itself is a motivator, the amount that money is a decent stand-in for how much the external world values something, and the fact that one of the initial reasons I began to publish the book was my belief that I could write something better than the authors who make a lot on the platform.  
Money itself is not really a motivator for me, I more and more realize.  
Money as a stand-in for value is something that motivates me decently well, though not much better than anything that money can be exchanged for (pizza, ice cream, a medal).\footnote{I have commented a number of times that, much as I love receiving awards and medals, I hate having them (hate might be a strong word, dislike? do not like? hmm) after the fact. My motivation to win is entirely on getting a thing, not having a thing}  
Given that the majority of my readers are faceless entities leaving comments on a faceless book, money is really the only way for them to show value.  
I don't think that I'm really motivated by the idea that other\footnote{potentially worse} authors are making more, because I've seen a number of what I consider really well written books also not have a significant monetary value associated on the website, and I've seen how much work the authors who turn major profits put into the business side of the writing.

A part of me also worries that being paid to write will make me want to write it less and/or value the book less.  
Given that I've made absolutely no steps towards monetizing the book, I don't think that's a motivation I should take into account.  
Comments are about as meaningful to me as I think that money would be, though there is the secondary point that something people spend money on is something that they're more likely to recommend to a friend, which would increase my number of comments.  
All in all, though, I guess that I have to say possibility of monetization is probably just about zero as far as motivations go.

The next motivation I wanted to explore was the fact that I do things in order to prove\footnote{side note: to who??} that I can.  
I think that I have effectively proven that I can write a web novel while doing a Ph.D., and I've proven that I can write something that others want to read.  
However, I haven't proven that I can finish a story or tell a narrative that ends in a way that people like.\footnote{the one thing I published on the site has a fair number of angry comments on the last chapter because I spent the entirety of the book on a few days and then sped through the next two decades.}  
There is also the above portion of my motivation to prove that I can write something that others would pay for, but, as discussed, that's relatively minor, especially since I've had comments saying that people are actively looking for a way to pay me.  
All in all, I'll say the part of me that refuses to accept limits is a minor motivation at best.

My motivation to write something that my sibling enjoys is not something I've considered for a bit.  
Given the group chat I had with other friends who enjoyed commenting on the book as each new chapter released, I should probably extend it a little further.  
They all have other things to read which they enjoy, so I don't really feel like I'm depriving them of much.  
Then again, I do also love when people like things that I made.\footnote{There's a meme that goes around where someone says something like \say{when looking at your art, think of it as a cake at a party. People don't go \nsay{wow this cake isn't as good as this other cake}, they go \nsay{wow! two cakes!}} which feels somewhat relevant here}

And finally, a friend asked me how motivated I was to write the thing because I enjoyed it.  
That feels like a tough question, and was part of my question for an autotelic action.  
Does doing something because I think that I'll enjoy it make it not autotelic?  
Great question, and one I've just reached out to a philosopher friend of mine about.

Ok, so all in all, I do have a fair amount of motivation to start writing the book.  
I do really believe that on some level the thing that was keeping me from writing was the fact that I didn't have an explicit reason to point to for why I would.  
I have that now!

Now comes the hard question: how do I start writing it?  
I have been and plan to continue to be very busy at work.  
However, breaks are important, or so I'm told.  
I do often find that I stick myself into a rut while working, and forcing myself to take breaks is at least one way of confronting that part of me.

How often do I want to publish, how much do I want to be able to revise, how much will I revise, how long will each chapter be, how much of a backlog do I want to start with are all other questions.  
Let's answer them now.  
I want to publish at least once a week, and ideally three times a week again, because I love a MWF release schedule.  
I would ideally like to be at least 10 chapters ahead so that I can hopefully avoid writing myself into a corner re: making a choice that has bad consequences in three more chapters.  
I don't really think that I'll revise much, in part because I don't care that much about eking out every possible shred of skill into the book, and in part because a goal is to get better at quickly writing decent text.  
I want each chapter to be in the 2000 to 2500 word range, but will accept if they are again in the 1800 to 2200 range.  
I want to start with a ten chapter backlog, because that gives me the revision ability that I had hoped for.

How do I get the next twenty five thousand\footnote{wow what a big number} words written?  
I set up time and space to do so.  
In general, I think that I can and probably should start setting time aside on Sundays again for personal growth related activities.  
They, like swimming or most things in my life, make every day slightly harder but in return make every day markedly easier.\footnote{which is such a horrible thing to realize, because it really shows me how quickly my discount function takes effect when I'm in the day to day. Framing it as a time discount might help me going forward, though (another idea!)}

I'm a few weeks behind on keeping up with the living goals, but here it is!

N.B. I've decided to have the whole list of goals that I have for the month at the bottom of each posting, and I'll delete entries as is relevant. That way I can track everything each day!

\begin{itemize}   
\item Professional:   
\begin{itemize}   
\item Thesis Work:  
\begin{itemize}   
\item Work on the research within the thesis! Working!  
\end{itemize}   
\item Be a better mentor: figure out how to take time to help underclassmen as they need help while still getting my own work done. Figured that out!  
\item Leave work at work. I've set boundaries, am following them, and it's working!  
\item Work towards future career:   
\begin{itemize}   
\item Read the recommended readings about science communication.  
\item Do the reflections that were recommended to me (mostly focused around why I care about science communication)   
\item Work on the materials for the science outreach event in April: the handout while they work and a page for the families to take home. Done! I think.  
\item Figure out the difference between my public-facing and field-facing presentation affects. As I focus on becoming a better presenter, I need to become aware of the difference and how to switch them   
\item Need to look for jobs  
\end{itemize}   
\end{itemize}   
\item Health:  
\begin{itemize}   
\item Spiritual:   
\begin{itemize}   
\item Do the Lenten goals. Haven't been great about this one. Was doing ok at not listening to things constantly but then had a rough few days and found a new series.  
\item Be intentional about prayer. That means both making time for prayer and actually doing it. Whoops.  
\end{itemize}   
\item Mental:   
\begin{itemize}   
\item Clean my Life:   
\begin{itemize}   
\item Remove dirt and clutter from physical spaces (standard definition of clean). Doing better!  
\item Spend time each day thinking about the goals for the day, and getting them out of my head and onto the page. Haven't been doing great at, am slowly improving though! Need to find the optimal way that isn't effortful  
\item Start reading and returning the library books I have. Nope!  
\item Don't waste time, and in particular, be mindful about making sure to take breaks and rest. Kind of! Wasting time is somewhat happening  
\item Clean sight lines. Is my space set up in a way that orients me towards my goals for the space? If not, how can I make it so? Darn curse of entropy.  
\end{itemize}   
\item Interpersonal Relationships:  
\begin{itemize}   
\item Figure out what belongs in a normal letter to a friend.  
\item Get back into writing letters.   
\item Upon feeling a sense of dread at receiving a message from someone, remember that my lived experience says that most interactions are positive. More to the point, if my friends didn't like me, they would tell me or at the very least would not continue to keep me in their life. If alone, speak something to that effect.  I've been doing this! It works wonders  
\item Work to begin messaging friends.  
\item Potentially start giving small gifts, though many people also dislike clutter, so think carefully about that one. Have started giving friends things as I have them, and they're well received. Thoughtfulness is the key.  
\end{itemize}  
\end{itemize}   
\item Physical:   
\begin{itemize}   
\item Go to group fitness classes more regularly and more often.  Doing! Will keep though.  
\item Feed myself simply and healthily. Mostly doing! Will keep   
\end{itemize}   
\end{itemize}   
\item Other:   
\begin{itemize}   
\item Music:   
\begin{itemize}   
\item Figure out something to work towards on guitar. Done! A friend asked me to play at his wedding, so will need to be good for that.  
\item Work towards it   
\item Spend time making efforts to improve as a singer, not simply passively singing. I'm getting rid of this goal because I no longer really have it.  
\item Spend time making efforts to improve as a musician, not simply passively growing. see above.  
\end{itemize}   
\item Writing:  
\begin{itemize}   
\item Find the mental block towards writing my web novel. Wow! Look at this!  
\item Write poetry more often, ideally nightly. Uhhhhhh nope, should restart though.  
\item Not only write blogs, but also post them. Ideas include:\footnote{as a living list!}  
\begin{itemize}   
\item 26 for 26   
\item Listening to an album and writing about my experience with it. Unsure if this is best done with one I have prior knowledge of or a new one, but regardless, sit and listen without other stimuli. One of my lab mates and I have been exchanging albums, so this could be fun  
\item The arts I've been doing lately   
\item Why I care about science and communicating it   
\item the block between me and my web novel   
\item My general disposition to authority. I'm not entirely sure what that meant, but it was a note I wrote to myself while half asleep, so I'll leave it here.  
\item How to feed myself.  
\item Motivation, autotelic motivation, etc. etc.  
\item The idea I have that knowledge (music, writing, science, etc.) is revealed, never gained (hard to express, meaning a great choice)  
\item Understanding the human experience, esp. re. writing  
\item Doing the work to improve myself and time discount  
\item My writing style(s) and what they show  
\item The four levels of mastery  
\item A fiction: story told through bullet points  
\item A short film: watching the screen as a breakup letter is written  
\item \say{the only way out is through, and the only way through is forward}  
\item Internal and External Arguments within and without\footnote{what's the term for not just within?}  
\item Curse of knowledge, esp re. Music Theory  
\item What does it mean to succeed at a creation? (reception, viewership, etc.)  
\item Inertia (re: my mind sticking to the thought patterns I have at a moment)  
\end{itemize}   
\end{itemize}   
\end{itemize}   
\end{itemize}

\section{Draft 1: 25 March 2025}

I keep musing within my musings about why it is that I'm not doing my web serial right now.  
I have a few reasons that have seemed plausible, and so I'm going to explore them as well as anything else that comes up as I muse today.  
I've been very into structuring documents lately\footnote{unsurprisingly, given that I'm setting up and writing my dissertation right now}, but I want this to be less structured, if only to force myself out of the happy little boxes that I'm putting myself into.\footnote{read: no lists or subsections today}

The first and most obvious reason an external observer might have for me no longer keeping up with the book\footnote{I don't know why I am so opposed to listing the title or even the shortened version I tend to use} is the same reason that most people have expressed shock that I was writing a web serial at all: a graduate degree is intense and\footnote{in theory at least} all consuming.  
Especially now that I'm in the thesis writing stage, I really should\footnote{should is a word that is apparently problematic for a lot of people} be spending a lot of my mental and physical writing space\footnote{the best part about being in physics is that now I can use space and time interchangeably and justify it with spacetime being a thing}, if not all of it\them on the thesis.  
However, I stopped writing the serial well before I started writing the thesis in earnest, and I was, in fact, able to write it while in a doctoral program, so that can't be all of it.  
Whether it's an actual reason or a convenient excuse I'm using is definitely up for debate, and I'm sure we'll come back to this in time.

The second reason\footnote{I know I said destructuring, but I also need the reader to understand that writing paragraphs that are text numbered feels completely free form to me at this point. The more I'm doing with planning, the more I feel like a bunch of nested bulleted lists tends to be the ideal way to write (side note: maybe explore that as a fiction idea?)} that an external observer might use is the death of my mother this past October.  
Technically speaking, I stopped writing before October\footnote{the 28th of August is when I posted that I was going on hiatus. Oof}, so that's not explicitly timeline accurate.  
Of course, my mother was actively dying in late August, and had been in the hospital for a while before that, when it wasn't clear if she was going to recover.

The death of my mother does tie neatly into the comment from the first reason: time.  
Not only was I grieving the loss of one of the pillars of my life, I am\footnote{I don't know why I'm using past tense, and much as I hate intra-sentence tense disagreement, I think that it stylistically works here. Readers are of course free to disagree} playing catch up, or at least feel like I'm playing catch up for the time that I was less than productive while actively dealing with her dying.  
Even outside of time, grief is absolutely something that has taken up a lot of my mental space and time.  
Something that countless authors have spoken\footnote{written?} about is the way that grief is almost all consuming at first.

The death of my mother also directly affects the novel for a really key reason that I'm not sure the average external reader would know: she was one of the main reasons that I wrote it.  
There was a solid month or so that I had to force myself to write each chapter, beginning with typing \say{the only way out is through, and the only way through is forward. This is something you can do to notably improve your mother's experience as she deals with her cancer}, deleting it, and then writing the words to the book.  
When it became clear that she was not, in fact, reading it or likely to be able to read it again, that baseline reason disappeared from the logic.  
Given that it was, at least allegedly, the sole thing that got me writing for at least a month, the loss of the reason is definitely a big part of the loss of motivation.  
I think that I had forgotten or blocked out the fact that I used her cancer as an explicit reason to write the book.

Obviously, this ties to another barrier to writing: I associate the book with her and it's painful to do things that I associate with her, knowing both that I am no longer able to connect with her about them and, maybe more importantly\footnote{for all that I'm just now coming to the realization}, that the more I do them without her, the less I'll associate the actions with her.  
That's not just me fearmongering, it's, as best as I understand it, the state of the field in psychology and neurology.  
Part of me was, and probably still is,\footnote{deciding what belongs in text and what belongs in footnotes is really hard when the entire thing is me reflecting and musing on emotions and what's in my mind explicitly} grasping tightly onto anything that I have that still reminds me of her.  
Looking at the blank text file where the book once was, or seeing the drafts that are still unwritten, though painful, also makes me immediately think about her.  
Grasping tightly is never healthy, though, and I think that I'm finally starting to loosen up my grip.\footnote{I listened to the audiobooks for the first series we ever read together (rather than like as a mother reading to a child). It was really hard, and I think that I actively like the books less now for having done so. Then again, I did do that in November, so the grief was without a doubt far rawer then}

I also have the general wall of starting anything.\footnote{how's that for a smooth transition}  
I've taken enough time off of writing that restarting the book is just that, starting again.

Part of me is worried that fans will suddenly hate the way that the book is written.

Part of me knows that so much of the book was written with the general love for life and optimism that I have had for most of my life.  
I'm beyond terrified at the idea that this optimism, which I have internally as such a key part of my identity, might no longer exist.  
I also worry that the writing will become darker, and therefore ruin the style that I've worked towards.

Part of me knows that the book is, to put it mildly, unique in the writing style.  
I've continued to read about writing style and how to structure prose, and I think that there's the internal argument within me\footnote{that's redundant, but I do also feel like I have internal arguments outside of me and external arguments within me, maybe} between adapting my writing style, and therefore the book, into something more mainstream, the part of me that wants to lean into the uniqueness, and the part of me that wants to not think about the style as I write.  
Writing that sentence, I know that it's the exact same issue I have always had with poetry and music.

More than that, though, it's the same argument that so many pop-adjacent\footnote{I don't know what else to call it. I'm using popular here in the musicological sense, which is non-academy and (these days generally) non sacred. That isn't to say pop the genre as labels define it, just music that people do independent of the academy} musicians use for not learning theory.  
They have a unique sound and don't want to force themselves into the box that learning theory will do.  
I always decry this argument as nonsense, because learning new tools is never a bad thing.

However, there is absolutely something to be said about the rubik's cube dilemma.\footnote{I know it has an actual name, but my little sibling is or was trying to solve a rubik's cube without any help}  
That is, once you know how to solve a rubik's cube, there's no way to go back and try to derive how to solve it yourself.  
Or, rather, knowledge changes the way that we see the world.

I know this is true, and have commented on it a lot.  
There is even a name for this, the \href{https://psychology.stackexchange.com/questions/3801/what-is-the-name-of-the-cognitive-bias-where-an-expert-overestimates-the-knowled}{curse of knowledge}.\footnote{see \href{https://xkcd.com/2501/}{this comic} for a humorous example}
In music, I feel like I've never thought of it as a curse.

Because music is such a universal human behavior, there are countless thinkers\footnote{from the academy, of course} who have put forth their reasons for why the curse of knowledge doesn't apply to songwriting.  
In short, the idea tends to be that we all listen to so much music, and our brains are so hard wired to find patterns, that we have internalized most musical rules.  
Education simply lets us make the choices conscious, rather than unconscious.\footnote{then we get to that whole \say{four levels of mastery} thing that I find so intuitive and that no one else seems to use}  
There's also the secondary point that I don't see a lot of people talk about, which is that most people, especially today, learned their instruments from some sort of system that derives on some level from the academy.\footnote{of course, music being so intertwined with culture means that like the blues, which was famously initially something completely independent from the academy, is now a staple of the knowledge the academy can give you (why the blues progression is what it is and how to quickly vary it)}  
I guess that I'm growing a little less sure of the position that knowledge is not a curse when it comes to finding a voice.

There's the related but completely independent argument that much creative expression is not solely about using our own voice, but also about sharing it with someone else.  
If the consumer does not take in what you are trying to convey, were you successful in expression?  
Even deeper, if the consumer doesn't consume the product because it's so off putting to them, were you successful at expressing yourself?  
In some regards, this ties to the question of why I am writing the web novel.

Is it something autotelic\footnote{new word I learned meaning something done for its own sake}?  
Is it, as I've mentioned before, a way for me to practice and hone my writing skills?  
Is it, as I've thought about before, something I do with hopes of one day turning a profit?  
Is it about demonstrating that I can do whatever I set myself to, believing that I am unlimited?\footnote{something that an expert tells me is that I have an internal worldview that believes that I am unlimited, and so therefore am hard on myself when, as it turns out, I am not}  
Is it, as the initial drafts were, about writing something for my older sibling to read?  
From a friend, is it something that I do because it's pleasurable?

This feels like a good place to stop the reflection for now, return to work\footnote{ah the first point returns}, and come back to this later.
\end{document}