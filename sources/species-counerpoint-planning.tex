\documentclass[12pt]{article}[titlepage]
\newcommand{\say}[1]{``#1''}
\newcommand{\nsay}[1]{`#1'}
\usepackage{endnotes}
\newcommand{\1}{\={a}}
\newcommand{\2}{\={e}}
\newcommand{\3}{\={\i}}
\newcommand{\4}{\=o}
\newcommand{\5}{\=u}
\newcommand{\6}{\={A}}
\newcommand{\B}{\backslash{}}
\renewcommand{\,}{\textsuperscript{,}}
\usepackage{setspace}
\usepackage{tipa}
\usepackage{hyperref}
\begin{document}
\doublespacing
\section{\href{species-counterpoint-planning.tex}{Species Counterpoint Planning}}
First Published: 2022 December 19

Prereading note: This will be rambly

\section{Draft 1}
As I \href{musescore-4.html}{mentioned recently}, I'm really enjoying playing around in Musescore.
But, I don't like being bad at composing, especially when I'm doing it.
One thing I've tried to do before is species counterpoint.

Species counterpoint is a set of compositional exercises that's pretty (in)famously used in a lot of composition curricula, especially historically.
Generally you begin with a whole note melody that you write a harmony to.
The harmony has to follow some general rules\footnote{that are probably memorable from introductory music theory} such as limited leaps\footnote{depending on the exact style of species counterpoint, leap rules vary a lot}, forbiddance of parallel perfect intervals, and so on.
Additionally, within each species there are specific rules.
There are five species\footnote{types of harmony line written}:
\begin{enumerate}
\item First Species: Whole note against whole note.
In this form, you generally are only allowed \say{consonant}\footnote{scare quotes necessary} intervals.\footnote{P1,m3,M3,P5,m6,M6,P8 and then everything larger gets reduced down}
\item Second Species: Half note against whole note.\footnote{two notes written per note}
Here the second note may be dissonant, but only as a step-wise passing tone.\foonote{so if you have a C below an E, you can go to B and then A, since C and A are consonant with E}
\item Third Species: Quarter note against whole note.\footnote{four notes written per note}.
As before, only the first note needs to be consonant, and dissonances need to be approached and left stepwise.
\item Fourth Species: Suspended whole notes offset by a half note.
Generally if the suspension becomes dissonant, it needs to resolve down by a step.
If not, you can leap or move up.
\item Fifth Species: Melody against whole note.\footnote{use whatever from each of the other species}
I think then the rules are fairly loose, though generally still consonant on first note unless suspended over.
\end{enumerate}
The pedagogical resources I've seen generally recommend working with your \textit{cantus firmus}\footnote{given melody} both above and below the written melody for practice.

After mastering\footnote{for whatever definition you want} the five species, you can then add another voice and start over.
As it turns out, that causes 55 lessons if I want to get up to three voices in fifth species against a single cantus firmus.

From \href{book-review-atomic-habits.html}{a book on habits I read}, it's best if new habits can be accomplished in under three minutes.
Historically, writing a single first-species line takes less than three minutes, so I'd like to try doing six days of counterpoint a week, moving lessons each week.
At first, both because I want to slowly build the habit, and because I know first species with a single voice well, I will just do a single exercise.
I'd like to spend the seventh day\footnote{Sunday, first day of the week technically} setting a song to four-part string arrangement, both because I like how it sounds in Musescore ,and because I want to get more practice actually writing music.

The plan is:
\begin{itemize}
\item One Voice First Species
\item One Voice Second Species
\item ...
\item One Voice Fifth Species
\item Two Voices, First and First Species
\item Two Voices First and Second Species
\item ...
\item Two Voices First and Fifth Species
\item Two Voices Second and Second Species
\item ...
\item Two Voices Fifth and Fifth Species
\item Three Voices First, First, and First Species
\item ...
\item Three Voices Fifth, Fifth, and Fifth Species
\end{itemize}
If I start today, I'll be finished near the beginning of 2024.
It's good to set big stretch goals I hear.
\end{document}