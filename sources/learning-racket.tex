\documentclass[12pt]{article}[titlepage]
\newcommand{\say}[1]{``#1''}
\newcommand{\nsay}[1]{`#1'}
\usepackage{endnotes}
\newcommand{\1}{\={a}}
\newcommand{\2}{\={e}}
\newcommand{\3}{\={\i}}
\newcommand{\4}{\=o}
\newcommand{\5}{\=u}
\newcommand{\6}{\={A}}
\newcommand{\B}{\backslash{}}
\renewcommand{\,}{\textsuperscript{,}}
\usepackage{setspace}
\usepackage{tipa}
\usepackage{hyperref}
\begin{document}
\doublespacing
\section{\href{learning-racket.html}{Learning Racket}}
First Published: 2022 January 6
\section{Draft 1}
As someone who does a lot of math-adjacent work\footnote{i.e. as a scientist}, I often have to do lots of repetitive calculations.
Since I am easily bored, I tried to find a way to not have to do the same equation 100 times with only minor modifications.
Growing up in a CS heavy household, I realized that I could try to use computers to do some of the math for me.

So, I asked my family \say{What language should I learn?}
They responded\footnote{in what should have been an obvious choice} Python, because lots of reasons which boil down to:
\begin{itemize}
\item It's easy
\item It's commonly used
\end{itemize}
But as I worked in Python, I became dissatisfied.

Anyways, flash forward to today where I'm now learning Racket.
Racket is built on Scheme which is built on Lisp.
Lisp is a language\footnote{language family?} that developed in parallel to C\footnote{FORTRAN? unsure where this begins}-like languages, which is based on the idea of lambda calculus.

It uses a lot of parentheses, and it has been fun to figure out how to use it.
Today I wanted to see if 12 steps to the octave really is the best approximation\footnote{minimizes the wolf tone}.
Turns out it is in fact the best up to 47 divisions of the octave!
And I did that in Racket, which is fun and exciting.
If you want my\footnote{bad} code, feel free to message me.
\end{document}