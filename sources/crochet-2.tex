\documentclass[12pt]{article}[titlepage]
\newcommand{\say}[1]{``#1''}
\newcommand{\nsay}[1]{`#1'}
\usepackage{endnotes}
\newcommand{\1}{\={a}}
\newcommand{\2}{\={e}}
\newcommand{\3}{\={\i}}
\newcommand{\4}{\=o}
\newcommand{\5}{\=u}
\newcommand{\6}{\={A}}
\newcommand{\B}{\backslash{}}
\renewcommand{\,}{\textsuperscript{,}}
\usepackage{setspace}
\usepackage{tipa}
\usepackage{hyperref}
\begin{document}
\doublespacing
\section{\href{crochet-2.html}{On Crocheting Again}}
First Published: 2023 June 29


\section{Draft 1}
It's been a little more than a year since \href{crocheting-1.html}{the last time I blogged about crocheting.}
When I last wrote it, I was in the habit of making hats\footnote{a hat-bit, if you will}.
I don't actually know if I have made any hats since the one I blogged about here.
I did make a few bags with a new stitch that I learned, which was really fun, though shockingly slow to build up.\footnote{the stitch was solomon's knot, which I feel like should be very fast, since it's such a loose weave.}

But, as is so often true of my hobbies, I fell out of the habit of crocheting for a while.
Somehow, the algorithms that serve me content seemed to think that I wanted to get back into crocheting, though.
In particular, there's a pattern going around right now that's apparently viral.
It mostly just looks like strawberries.\footnote{or, as one group mate phrased it, tomatoes. I hope that he's unique in believing that}

So, after four years of crocheting\footnote{wow that's way too big of a number. I did not graduate that many years (not doxxing myself right now) ago, that's so fake}, I finally made a pot holder.
I think that's generally recommended to be among the first projects people start on, but I've never been particularly good at following conventional ordering for learning skills.
The strawberry stitch builds into a pot holder really nicely, since it's a completely linear project.
It's a four row pattern that then loops.
Odd rows are in red, and even are in green.\footnote{and wow I do not know how to effectively switch colors.}

The first row is single crochet three, double crochet five stitches into the fourth, repeat that until you have three stitches left, then single crochet three.
For a foundation, the site I found the pattern on recommended doing a row of single crochet with 4N-1 total stitches where N is the number of strawberries you want.
The second row is single crochet in each single crochet, then single crochet the five stitches together and add a chain and slip stitch for sizing.\footnote{as it turns out, when you crochet together, you're apparently supposed to pick up each stitch individually, yarn over and replace, and only then take them all off. So, instead of the hook having five red loops, it has six green. It took me far too long to realize that.}
Row three is single crochet one, double crochet five into one, single crochet three, repeat the second and third steps until the end, which is single crochet one.
Fourth row is identical to second row.

The overall pattern makes a nice offset set of strawberry shapes, though I would like to explore with different number of stitches in the strawberries or separating them.
That's for another time, though.
It also built up a little thinner than I would have liked, so I think I'll need to sew a cotton sleeve to the other end for thickness.

\begin{itemize}
\item Once more, I did not make the progress I wanted to on my home.
\item I hope not to break my blogging streak tomorrow, but that will mean that I need to write before playing DnD with friends.
\item Air quality is returning to breathable, though it hasn't yet. After this I will stretch, since I haven't hit my daily goal yet.
\item I slept terribly last night, so slept in.
\item Today I listened to less CCCiaY, though I will do a bedtime rosary.
\item I wrote a little under two chapters today, which was fun. Interestingly, I was able to write one chapter basically straight through in under forty\footnote{which should really have a u}minutes, and then while writing the second noticed that it was late.\footnote{and got jump scared by a music video playing in full screen and my app crashing (thankfully right after I saved to another program)}
\item Three days have passed without poetry. 
\item I wrote and mailed a letter this morning! It was really fun! I also took the advice of a book I'm reading and lit a candle while I wrote. It did make it far more fun.
\end{itemize}

537/155
\end{document}