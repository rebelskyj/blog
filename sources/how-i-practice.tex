\documentclass[12pt]{article}[titlepage]
\newcommand{\say}[1]{``#1''}
\newcommand{\nsay}[1]{`#1'}
\usepackage{endnotes}
\newcommand{\1}{\={a}}
\newcommand{\2}{\={e}}
\newcommand{\3}{\={\i}}
\newcommand{\4}{\=o}
\newcommand{\5}{\=u}
\newcommand{\6}{\={A}}
\newcommand{\B}{\backslash{}}
\renewcommand{\,}{\textsuperscript{,}}
\usepackage{setspace}
\usepackage{tipa}
\usepackage{hyperref}
\begin{document}
\doublespacing
\section{\href{how-i-practice.html}{How I Practice}}
First Published: 2018 December 8
\section{Draft 2: 8 December}
Earlier this week I was asked to reflect on how I practice.
I can boil my practice routine down into three pieces: making non-practice time useful, impromptu practice, and scheduled practice.

Non-practice time being useful is probably the bit of how I practice music that is weirdest to others.
It's somewhat like \textit{The Music Man}, where I visualize what I would want the piece to sound like, but there's more to it.
For example, when I know I'll be unable to practice cornetto for a while, I'll make a special point of listening to different professional cornettists so I can evaluate what of their playing I like and dislike.
The variety is also important, so that you can figure out what separates you from any given professional as well as what separates you from all professionals in performance.
I also sometimes listen to recordings I've made of myself, though I am loath to do so.

When I talk about how I don't always practice music, something that people will comment is that I play instruments a lot.
And, sure, that's true.
But, playing the hurdy gurdy or hammer dulcimer shouldn't really help me play cornetto, especially since the instruments are so different.
However, much like how lifting improves performance in most athletic endeavors, I've found that playing any instrument, or even just listening to music, helps you with every instrument you play.
I have no clue why this is, but it's been incredibly helpful for me to understand.
So, when I'm at Grinnell and bored, I'll start playing one of the many instruments I have access to and know how to play.
Here in London, I'll pull out my ukulele or penny whistle, or even just headphones while I'm on the endless subway rides.
The idea of using wasted time sort of flows into my other kind of using wasted time method.

For instance, when I practice bagpipes \footnote{the instrument that everyone loves to hate} I often just use the practice chanter silently.
The important piece of bagpipes is having quick and sure fingers, so just fingering the practice chanter is often all that I need to improve.
When I don't even feel like doing that, I'll often visualize the music that I'm supposed to be learning.
Sometimes that's me playing the melody in my head,\footnote{often just a short phrase} sometimes it's me thinking of fingerings, and still other times it's just me thinking of the sound of the instrument.
One important thing about playing is knowing how you want the instrument to sound.
Sometimes I do feel like playing though, such as last night while waiting for trains to arrive.
There, I practiced playing my penny whistle.
It was time I wasn't able to be otherwise productive in, so I used it for music.

That idea of music at whenever time is more or less how my impromptu practicing works.
The only difference between it and how I practice instruments above is that the impromptu practices are me trying to learn something, rather than just playing by ear or memory.
That is, when I'm playing like I talked about in the wasted time section, I'm usually playing melodies I already know by heart or just fiddling around on the instrument.
When I'm impromptu practicing, I'm trying to learn something new.
It's an odd distinction, but it works for me.

Finally, there's \say{dedicated} practice time.
When I have a large block of time,\footnote{larger than 30 minutes} I'll tend to work on multiple instruments.
Multiple here means however many I'm planning on performing on within the next 3 months while at school, exist within easy reach at home when there, or with me while abroad.
So, I'll pick an instrument\footnote{the one think will be the hardest, or easiest, or I'll be performing on soonest, or randomly (depending on the day)} and start playing it.
This semester, that's tended to alternate between cornetto and bagpipe, and those are really the two I'll focus on in dedicated time.
It happens when I've set aside a piece of time to go and practice at a space.
I almost never use the fully allotted time, except for when I have an activity on the other side of it, because then I inevitably want to keep practicing.\footnote{good old confirmation bias}

So, that's how I practice.
\section{Draft 1: 6 December}
Today, I was asked to reflect a little on how I practice.
The prompt came because I was asked how I know what I should focus on when I practice, especially since\footnote{in my experience} people teaching lessons never tell you to stop working on something.
I said something along the lines of \say{Don't forget the old exercises, but stop focusing on them/trying to improve them,} and that's pretty true for me.
I never really stop playing pieces that I've learned before.

Sometimes, this is as simple as me playing the first line of a piece that I last worked on two years before because my mind gets caught there, since some of the warmup I do stays the same.
Other times, it's listening to old recordings of myself, so that I can hear how my playing has changed.\footnote{thankfully, usually for the better}
Even other times, I'll play old things when I feel too discouraged, as a sort of diagnostic \say{is it the piece or me that isn't working today?} questionnaire.
It's about 50/50 which one it will be.
Finally, sometimes it's just that I'll see the piece as I page through my folder.
So, I never really stop thinking about a piece.

But, that's not the whole of how I practice.
So, I don't immediately shred the idea of the pieces I've played before.
What do I do instead?

There's a couple pieces to the way I practice, which vary based on my opportunities.
The worst part of my time in London has been my difficulty with finding space to practice in.
So, I can't always\footnote{or even often} practice the instruments I learn with sound.
Thankfully, depending on the instrument, that's not a huge issue.

For instance, when I practice bagpipes \footnote{the instrument that everyone loves to hate} I often will just use the practice chanter silently.
The important piece of bagpipes is having quick and sure fingers, so just fingering the practice chanter is often all that I need to improve.
Other times that I can't practice, as I mentioned above, I'll listen to old recordings of myself.
But, I don't do that a ton, for obvious reasons.\footnote{poor recording, music I don't always love, poor quality of player}
Instead, I've found\footnote{after being explicitly instructed to do so by teachers} that listening to experts and professionals play is another way to improve.
So, when I know I'll be away from the cornetto for a bit, I'll make a special point of listening to different cornettists, so that I can evaluate what of their playing I like and don't dislike.
The variety is also important, so that you can figure out what separates you from any given professional as well as what separates you from all professionals in playing.
But, that's not always a great time, because sometimes I like to listen to other music that the instruments I play.

That's where the weirdest part of my practicing comes in.
I've found that when I work on any instrument, my performance on every instrument improves.
I think it's a lot like weight-lifting for sports.
Yes, you're probably better off spending any given hour practicing a sport, rather than lifting.
But, if you're not sure which sport you'll be competing in, or can't practice that sport for whatever reason, lifting will still improve your athletic performance.
Likewise, because every instrument I play has some sort of expressive device that differs from the others I play, I can focus on playing musically and expressively, even when reading music.
Somehow working on penny whistle flourishes helps me work on cornetto articulation.

Finally, when I really don't feel like practicing at all, I'll visualize music.\footnote{ok, I'll visualize it a lot anyways}
That is, I think about the piece I'll be performing, while I'm walking around, or sitting, or just bored, and think about what it would sound like in a perfect world.
As much as \textit{The Music Man} claims that it works, I'm surprised more people don't try it.

But, all this is more accurately titled \say{how I \textbf{don't} practice,} so I should probably explain how I do.
When I have a large block of time,\footnote{larger than 30 minutes} I'll tend to work on multiple instruments.
Even when I don't I sometimes try two or three.

Anyways, I'll pick whichever instrument I think needs the most care,\footnote{or I think will be the hardest, or easiest, or I'll be performing on soonest, or randomly (depending on the day)} and start playing it.
This semester, that's tended to alternate between cornetto and bagpipe, and those are really the two I'll focus on in dedicated time.

Oh, that's also another difference: the idea of dedicated practice and free practice.
I'll come back to that later.
So, I'll start on one of the instruments.
When I become unable to play it, because of either mental or physical exhaustion, I'll switch to the other.
This will continue until I run out of time or mental energy to do any more playing.\footnote{anymore?}
Then, I'll be done.

Of course, that's in the dedicated time.
That is, the block of time I'll set up and say \say{I'll be going here for this time frame to do this.}\footnote{I don't normally take the amount of time I plan on, and when I do, I tend to need more}
At other times, I'll just be finding myself in need of a way to occupy myself, and I'll just start playing because I can.
There's a place for both, and I think it's important, especially for me, to have both forms of practicing.
Anyways, that's how I practice.

\end{document}