\documentclass[12pt]{article}[titlepage]
\newcommand{\say}[1]{``#1''}
\newcommand{\nsay}[1]{`#1'}
\usepackage{endnotes}
\newcommand{\1}{\={a}}
\newcommand{\2}{\={e}}
\newcommand{\3}{\={\i}}
\newcommand{\4}{\=o}
\newcommand{\5}{\=u}
\newcommand{\6}{\={A}}
\newcommand{\B}{\backslash{}}
\renewcommand{\,}{\textsuperscript{,}}
\usepackage{setspace}
\usepackage{tipa}
\usepackage{hyperref}
\begin{document}
\doublespacing
\section{\href{flash-fiction-232.html}{Flash Fiction Friday}}
First Published: 2023 December 22

\section{Draft 1}
Another Friday, which means it's time to try to answer another prompt.
The prompt this week is among any option.
I had no idea how to respond to this prompt, and reading the description didn't make the prompt any more sensible.
However, it did give me some ideas.

The prompt implies a choice.
If we keep that in mind, then the prompt shifts to choosing between any option.
What could be a good response for that prompt?

My mind is falling immediately to love, but I have to wonder how much of that is just me choosing a new creative rut.
I think that the last few times that I've written a FFF, it's been about romance of some sort.
What else could we set as a number of choices?

I mean basically anything can be a set of choices, which makes this prompt feel almost too broad.
I guess that more or less any writing I produce will be good for the prompt.
A part of me really doesn't like that, though, because part of my ideal for prompts is that it constrains my writing.

So, where else does my mind immediately go?

Given that we did a family recipe for dinner\footnote{musing partially written, ready to be finished and posted tomorrow}, I'm finding that I'm thinking about choosing foods.
As the new Percy Jackson television series\footnote{musing to be made at a later date, once I've consumed the content (I've read that people are against the phrase content consumption, and I don't disagree with them, but I find that it does accurately describe the way that I feel (might be worth figuring out how I think and why I think about the things I do)) about the media I engage with} is coming out, there's choices about media we consume, hero's choices, and the like.
Since we play board games as a family, the idea of what game to play comes up a lot.

The hero's choices feel like something that I want to interact with.
What do heroes do?
Or, rather, what terrible choice could a hero be forced into?\footnote{the fact that I want it to end in tragedy probably says something about where I am right now, but that's the nature of life}

I'm still circling around the idea of romance, which may be something that I could do?

So, we have a hero\footnote{I read a dictionary once that defined heroine exclusively as the female accompaniment to the hero, which was shockingly formative. As a result, I treat hero as a gender neutral term} who needs to make a choice.
Maybe there's something to say about like deciding between the person you love and the work you need to do?
Alright, how can we frame it in a way that hasn't been done to death a million times?

Or, just as much in the choice, how do we do what we want to do with our full self, ignoring all the self doubt and negative self talk?

A hero has a choice.
How explicit do I need to make that?
I think that at this point I really need to focus on the writing, which means I need to stop writing about writing and start actually writing.\footnote{terrifying, I'm not going to lie}

Actually, wait, let's think about the normal questions we have.
Do I want to have first or third person?
I don't really know.
I think that we should write it and see what happens.

Ok so about five hundred words of writing later, I find that I got too far from the prompt.
I more or less wrote a retrospective, ten years later.
I think that I focused too much on the retrospective, rather than the actual set of choices.
Maybe I could do something looking forwards instead?

Ooh, I actually really like the FFF I wrote. Let's see if there's anything I can do to spice it up for take three, but if not, I am actually going to be ok with it.
I assume that I'll see something that I want to change, but you never know until you try.

Ooh, wow, the revision is way stronger, I think. I more or less replaced every word, which is kind of funny when you think about it. The scene played out more or less the same, though, which is interesting.
I do still feel like I rushed the ending, which is more and more something I realize I struggle with.
The ending of a story always seems to come too quickly, and I don't quite know how to wrap it\footnote{honestly, any narrative} up satisfyingly.

All this to say, good night.

Daily Reflection:
\begin{itemize}
\item Hobbies:
\begin{itemize}
\item Did I embroider today? I need to find the embroidery, and more importantly, I need to find time for the embroidery. Picking a project would probably help too. There has to be someone I can give a small thing to, and I can just make something for one of them.
\item Did I play guitar today? I meant to, but the hours passed by really quickly. They were spent with family, though, so I cannot really complain.
\item Did I practice touch typing today? It's too late, so I'm going to take a break for the day and say that's ok.\footnote{look at me rhyming}
\end{itemize}
\item Reading
\begin{itemize}
\item Have I made progress on my Currently Reading Shelf? I'm still working on the third book in the series. I think that I'm spending too much time in front of screens, though, so should do some analog reading.
\item Did I read the book on craft? Brought it in finally today, but it was too dark to read.
\item Have I read the library books? I looked at them and felt bad that they were sitting alone, which is like reading them.
\end{itemize}
\item Writing
\begin{itemize}
\item Did I write a sonnet? Yesterday's also wasn't great, and I'm too tired to write one tonight. Tomorrow I should write earlier.
\item Did I revise a sonnet? Nope
\item Did I blog? Yay!
\item Did I write ahead on Jeb? I finished the final chapter of the year, and I'm going to intentionally take a few days off before starting again.
\item Letter to friends? I saw a few friends.
\item Paper? Nope
\end{itemize}
\item Wellness
\begin{itemize}
\item How well did I pray? I should work on this
\item Did I spend my time well? Yeah!
\item Did I stretch? Drat.
\item Did I exercise? Crud.
\item Water? Nope.
\end{itemize}
\end{itemize}
\end{document}