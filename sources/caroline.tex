\documentclass[12pt]{article}[titlepage]
\newcommand{\say}[1]{``#1''}
\newcommand{\nsay}[1]{`#1'}
\usepackage{endnotes}
\newcommand{\1}{\={a}}
\newcommand{\2}{\={e}}
\newcommand{\3}{\={\i}}
\newcommand{\4}{\=o}
\newcommand{\5}{\=u}
\newcommand{\6}{\={A}}
\newcommand{\B}{\backslash{}}
\renewcommand{\,}{\textsuperscript{,}}
\usepackage{setspace}
\usepackage{tipa}
\usepackage{hyperref}
\begin{document}
\doublespacing
\section{\href{caroline.html}{Caroline, or Change}}
First Published: 2018 December 10
\section{Draft 1}
Today, I had the wonderful opportunity to see \textit{Caroline, or Change} at the Playhouse Theatre.
There were some pieces of the show that I really loved, like the long scene about how much the world will miss JFK, and the dinner scene, where I realized that if my paternal grandfather had been alive at a Hanukkah table, he might have expressed similar sentiments.

But, overall the show fell flat for me.
Partially this is my fault, because when two singers are singing different words, I can't hear either.
Since the show was written very operatically, that meant that I couldn't understand much of what went on at all.
I also didn't love the melodies that the composer used, but that's neither here nor there.
\end{document}