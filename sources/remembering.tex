\documentclass[12pt]{article}[titlepage]
\newcommand{\say}[1]{``#1''}
\newcommand{\nsay}[1]{`#1'}
\usepackage{endnotes}
\newcommand{\1}{\={a}}
\newcommand{\2}{\={e}}
\newcommand{\3}{\={\i}}
\newcommand{\4}{\=o}
\newcommand{\5}{\=u}
\newcommand{\6}{\={A}}
\newcommand{\B}{\backslash{}}
\renewcommand{\,}{\textsuperscript{,}}
\usepackage{setspace}
\usepackage{tipa}
\usepackage{hyperref}
\begin{document}
\doublespacing
\section{\href{remembering.html}{On Remembering}}
First Published: 2022 November 8
\section{Draft 1}
Yesterday was my grandmother's birthday.
It's a strange day for me, because it is a little sad.
I still miss her, if only intellectually.

On the other hand, though, I can no longer remember most of what used to make me sad about her absence, let alone her.
The only pieces of her voice that I can recall are from the voicemail that my dear brother had\footnote{has?} for the longest time.
When I try to picture her face, I can only see the photographs of her that we keep in our home.

It's strange for me to realize that I've been without grandparents for nearly half my life now.
A lot of the poems I've been writing this month have centered around memory, especially the generational memory that I feel like I'm missing.\footnote{If you'd like it, shoot me a message}
In retrospect, November is a month that always has so many calls to my past, so it's unsurprising that my reflection\footnote{as writing poems often is for me} would start centering on them.

So what is remembering?
Is it the emotional ties that we keep to what's now gone?
Is it the intellectual knowledge of something now lost?
Is it something else?
I don't really know, but I'm glad I at least had this to reflect on it for a bit.
\end{document}