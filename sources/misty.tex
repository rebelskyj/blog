\documentclass[12pt]{article}[titlepage]
\newcommand{\say}[1]{``#1''}
\newcommand{\nsay}[1]{`#1'}
\usepackage{endnotes}
\newcommand{\1}{\={a}}
\newcommand{\2}{\={e}}
\newcommand{\3}{\={\i}}
\newcommand{\4}{\=o}
\newcommand{\5}{\=u}
\newcommand{\6}{\={A}}
\newcommand{\B}{\backslash{}}
\renewcommand{\,}{\textsuperscript{,}}
\usepackage{setspace}
\usepackage{tipa}
\usepackage{hyperref}
\begin{document}
\doublespacing
\section{\href{misty.html}{Misty Review}}
\section{Draft 1}
Today, I had the immense fortune of seeing Arinz\'e Kene's \say{Misty} at the Trafalgar Theatre.
Misty\footnote{as far as I could tell} tells the story of a man writing a theatrical work.
Throughout the process, he deals with internal doubt, lack of support from friends and family, and pressures from the theatre administration.
It was an incredible experience, and not one that I can otherwise summarize.

The set began basically, with a cube-like shape on the stage.
Where its shadow might go, the stage was roughened.
Throughout, the stage and set are changed mostly through lighting.

The sounds of the show came both from prerecorded audio and from two live musicians.
At times, it was difficult to tell what was prerecorded, and what was being done live, which was also true of the performance itself.

Though serious, the show was at many points incredibly funny.
Though funny, many of the jokes still pointed to deeper issues.

Overall, it was an experience I would repeat in a heartbeat
\end{document}