\documentclass[12pt]{article}[titlepage]
\newcommand{\say}[1]{``#1''}
\newcommand{\nsay}[1]{`#1'}
\usepackage{endnotes}
\newcommand{\1}{\={a}}
\newcommand{\2}{\={e}}
\newcommand{\3}{\={\i}}
\newcommand{\4}{\=o}
\newcommand{\5}{\=u}
\newcommand{\6}{\={A}}
\newcommand{\B}{\backslash{}}
\renewcommand{\,}{\textsuperscript{,}}
\usepackage{setspace}
\usepackage{tipa}
\usepackage{hyperref}
\begin{document}
\doublespacing
\section{\href{reflections-on-readings-27-ordinary-c-22.html}{Reflections on Today's Gospel}}
First Published: 2022 October 2

Habakkuk 1:2a \say{How long, O LORD, must I cry for help
and you do not listen?}

\section{Draft 1}
Today's readings call us to remember the cost of discipleship.
These are not the costs that have been coming up in prior weeks: temporal pain and hatred, loss of material wealth, and so on.
Instead, these readings remind us that the Lord's ways are mysterious, and we are called to remain steadfast in spite of, and perhaps due to the trials of faith.

Habakkuk asks the same question that has plagued non-believers for ages: if the Lord is all-good and all-powerful, why is there suffering in the world?
More to the point, why is there suffering among the faithful.
I'm sure we can all understand how the Lord, especially as seen in the Old Covenant, would allow suffering for the faithless, but why the faithful.

These readings don't seek to answer that question.
We are given pieces of it, certainly, as in the Psalm, where the response calls us to keep from hardening our heart when the Lord speaks.
We are directly compared to the Israelites at Meribah, where they questioned the Lord's goodness even as he brought water from stone.
As ridiculous as it seems to be dubious in the face of such an obvious miracle, how often does that happen to us?

White light sent through a lens diffracts into a rainbow of colors because each wavelength moves slightly differently through a medium.
How wonderful is it that we were given such a world to work in, where we can see what colors everything shining are made of?
How often do we forget the beauty in the natural world because it's been buried under layers of dry material?

Christ himself tells us in the Gospel that even when we do all that we are commanded, that is no cause for reward.
Instead, we have merely done what is expected of us.
In what we are expected to do is see the Lord's Creation without a hardness of heart.
Seeing the beauty of the world He created is certainly something that I need to work on.
\end{document}