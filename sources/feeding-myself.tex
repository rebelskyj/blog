\documentclass[12pt]{article}[titlepage]
\newcommand{\say}[1]{``#1''}
\newcommand{\nsay}[1]{`#1'}
\usepackage{endnotes}
\newcommand{\B}{\backslash{}}
\renewcommand{\,}{\textsuperscript{,}}
\usepackage{setspace}
\usepackage{tipa}
\usepackage{hyperref}
\begin{document}
\doublespacing
\section{\href{feeding-myself.html}{On Feeding Myself}}
First Published 5 April 2025
\section{Draft 1: 5 April 2025}

A potentially concerning percentage of the posts\footnote{interesting that I'm more and more using post instead of blog or musing or anything that others I know use. Worth thinking about, so onto the pile (list of things to write) I have} I've been writing are about food.  
I want to be feeding myself, and I want that feeding to go quickly and healthily.

I did realize today, though, that I want this post to encompass more than simply feeding my stomach.  
I want to make sure that all parts of me, body, mind, and soul, are being fed.  
Still, I'm going to start with feeding the body.

Breakfast and lunch, at least during the work week, are getting better and better.  
Breakfast of oats with berries is a great breakfast,\footnote{accusations of it being horse food aside} and I've generally been eating a decent lunch.  
Dinner remains a mixed bag, though, which isn't great.

I think that there are a few interwoven parts to that.  
For one, I still have some part of me that really thinks that dinner should be a production, or at least something effortful.  
I don't really know what that means, and interrogating the idea isn't really giving me anything either.  
However, I guess it is good for me to know that I have that gut instinct, so that I can start confronting it.  
I think that the other main issue with dinner is that I do not have a schedule, which makes scheduling exactly when I will be eating what in my home more difficult.

Still, I've done any number of hard things, and this isn't as hard as that.

Third issue is that I don't have a microwave, so anything I want for dinner kind of has to be stove or oven meal, which also adds some time to the prep.

Potential solutions:  
\begin{itemize}  
\item Plan to eat dinner at work. In this way I feed myself, don't crash as much when I stay late, have motivation to stay and keep working, and have access to a microwave.  
Downsides: I then am eating all three of my daily meals at work, it does nothing to help my weekend self, and I only have access to a microwave.\footnote{ok that isn't true, I do also literally have the small countertop oven that I brought into the office a few years ago.}  
I guess I really mean that I cannot do food prep at the office the same way that I can at home.  
It would require me constructing meals and bringing them to the office.

\item Keep as I am doing. Works for lunch and breakfast, but adds stress to my day and means I'm very tired by the time that I go home for the night. Probably not the winner, for all that it does have the advantage of being exactly what I currently do.

\item Do to dinner what I do to lunch: premake a bunch of meals and then heat one up each night. This can work with the first idea, but it means that if I do so at home, I have to deal with the slow pace of oven or stove.

\item Make dinner the night before, when I would have been making the dinner for night of.  
In this way, I can still get the cooking itch scratched, and if I don't prep something, then I will know that I am responsible for feeding myself.  
\end{itemize}

I guess there's also the tertiary consideration, which is that I can also just keep more easily consumed food at home.  
If I do that, then I can just grab like a handful of peanuts when I feel hungry, rather than needing to either just suffer or make something effortful.

Ok with my thoughts laid out like this, I think that I have my answer for now.  
Night before, I will cook dinner for the following day.  
If, for whatever reason, I am not able to get the meal done, then I'll just grab two lunches to work, and eat one for dinner.\footnote{I should \textbf{really} get through the bag of rice that I brought in at some point}

I'll also get some peanuts the next time I go shopping and then have a snack that I can consume at home.  
Goal part two of feeding my body is explicitly getting some, if not recipes, then at least food things that I can do, knowing that a past me has found their macro profiles to at least resemble something decent.

(N.B. I wrote everything from here until the start of the list after the second entry, because I realized that I want to do the back of the envelope protein content for what I'm getting from breakfast and lunch).

Assuming that I eat my oats and box lunch like a good child, I should be getting about\footnote{ok so one ounce of oats dry is apparently 120 calories, which is about 5 grams. Another site has it as cups of prepared, which is weird to me, because that's entirely based on the amount of water I add, and (unfortunately for my diet and fortunately for basically everything else) the water I consume is not appreciably protein filled. Still, I think I probably eat 2-4 ounces of oats a day, so 10 to 20 grams of protein. Probably about 50 grams from the 200 grams of pork I'm now eating a day (I was still hungy after eating the 100 gram bits, so). I'll assume the cheese that I'm eating right now is fairly representative, and so I'm getting like 25 grams from that. Oh Wild, I don't actually need protein at dinner! Cool} my entire daily need for protein in the first two meals I eat, so I can go entirely based on vibes!

\begin{itemize}

\item Pot of beans has been doing me well. If I thought ahead, I could do from dry\footnote{because wow I should be cycling through them}, and I know that I prefer them with a little kick.\footnote{read: measure cayenne with your heart}  
Beans are great in terms of the macros.  
I've taken to starting with frying up a handful of frozen onion and celery, which adds at least something to the dish, and I also tend to pour in some frozen peas at the end, which helps the dish overall.  
It's also relatively low effort, which means that if I make a batch\footnote{as I kind of have to do, especially if I'm going to be using dry. I guess there's nothing actively making me cook the entire bag at once, but let's be real. I will not cook a half bag when it is no extra work to cook a full one.} I will need to actually eat the whole batch before the beans ferment.\footnote{don't ask me how I know that the stages of a bean dish left in the fridge tend to stall out at the fermentation step, rather than ever molding.}  
I don't really need rice with that, though rice is nice. Probably worth thinking about making up some rice at start of week and doing some portioning?  
Ooh! If I get japanese curry blocks, I can have curry beans and rice.\footnote{double checking that this isn't offensive}.

\say{But!} you might cry out here, \say{is a curry block not incredibly processed?}

Great question.

\item if I thinly slice meat, rather than boldly chunk it as I have been, I can take small bits of that and it'll reheat quickly.\footnote{yet another option over rice}

\item Since I don't really need to hit any macro goals, plain curry rice\footnote{Japanese curry block over rice} is an easy and tasty option.  
If I add vegetables to the curry, becomes even healthier!  
I can also apparently make my own curry block at home!\footnote{I won't because I don't really care that much about palm oil, but it is still good to know. Then again, if I do something resembling a block, I can instead keep roux blocks and spice mix ready to go in my freezer, which is also fun! Let's plan on that when I get an afternoon, since I like being able to completely choose my flavor profile. Does mean I would have to splurge for dry mushrooms, but.}  
My friend recommended carrot, potato, and onion as a trio of vegetables for the curry.  
I guess one issue is that I then have to cook them and make the roux, which kind of means that it shouldn't be a daily event.  
Honestly, though, that's not a huge issue. I don't think I actually have the cooking urge so much as have tied it in my mind to being able to eat dinner\footnote{which is true}

\end{itemize}

Ok cool, I do love japanese curry, and I have been wanting an excuse to buy potatoes.\footnote{they spoil at nearly normal vegetable pace, but I treat them like onions (indefinitely good) or apples (also good forever if u do it right). Maybe I just need to treat them better?}  
So, the to do list is:  
\begin{itemize}  
\item Clean out fridge\footnote{read: get rid of the food which is clearly bad and or that I know I will not eat. Food waste is bad, but it's a smaller sin than self harm, which not feeding myself is at least a passive version of}  
\item Figure out food cooking timings (next part of blog)  
\item Figure out how often I need to go grocery shopping to get the things that I want to have  
\item Get the things for next week  
\item Make food  
\item Keep a schedule  
\end{itemize}

One thing that I know about myself is that I cannot rely on weekends for cooking.  
I can rely on it for shopping, which is maybe strange, but.  
I guess there's also the fact that I will have an incredibly different schedule in a little over a month\footnote{read, no longer have most of the demands on my time that I have right now}, and will be visiting friends\footnote{hopefully. I do need to schedule that!} over the summer, so I guess I know that it won't be a forever thing.  
Still, probably good to give myself the bare bones of how I would feed myself going forward.

So:  
\begin{itemize}  
\item I eat a small loaf of bread each day M T R.\footnote{on Wednesdays I often treat myself to a bagel and on Fridays I don't eat meat, which means that I do kind of need to feed myself differently because of the missing seventy five grams from the two pieces. Then again, it's ok if one day a week is protein deficient}  
If I make the buns as I have been, I can make 24 at once.  
Let's say that's three or four loaves a week, which means I need to make bread every six to eight weeks.  
That feels about true to what I've had, and of course does not include the fact that I do tend to eat more than a loaf on some days or eat them on off days.  
Cool, once I figure out how many I have left, I can schedule when to make the next batch.

Oh! I did also just buy a lighter whole wheat flour, because I don't necessarily have to eat the aggressive winter wheat, spring wheat is also still whole.  
That'll be fun.  
I do also want to start using the entire container of yogurt, because I have kind of stopped eating it at home lately.  
That might change with my schedule, but I doubt it.

\item I currently don't eat meat on Fridays because it's Lent\footnote{and, despite my reservations about food restriction, I think that it's still probably better for me to follow the guidance of abstinence on Fridays}. I should consider whether it's worth continuing that past lent.  
Pros: day without meat is better for the environment and myself, if the data are to be trusted.  
The American Church is laxer than most of the world when it comes to the Friday meat restriction.  
Technically we are still supposed to give it up.  
I'm on a high meat diet\footnote{ok I do realize that 200 grams of meat a day is not, by almost any American definition, a high meat diet.  
It feels high, though, because I'm eating meat daily, and usually as a block of its own.  
If I put it in curry, I suppose that would make it more spread out.  
Off topic though, return} right now entirely because I want to make sure that I'm not harming my body.  
If I keep not eating meat on Fridays, I will in theory start to get more meals that are meatless and can find ways to increase protein.\footnote{there's the ever available and expensive option of simply adding gelatin}  
I don't know what I started this note with, but that's just a thing to consider.

\item I eat about 100 grams of cheese every day I have a bun, so approximately 3-4 times a week. I must be buying huge blocks of cheese, because wow I get a lot of days' cheese from them. Still, next time I'm at the store, note how many it will make.

\item Around 200 grams of pork a day of cheese, which means a little more than two weeks a kilo.  
Since I think that the porks I've bought are all in the 8-10 pound range, that means I'm good for probably a month per pork.\footnote{oof that back of the envelope math feels like \href{https://xkcd.com/2585/}{XKCD's joke}, but it also is probably true in the end. Back of envelopes here, not precision}

\item Fruit: one to three a lunch, on all days ideally.  
That means that I need at least fifteen a week.  
I also find that they're only really good for a week or two.  
With that in mind, I guess that I should probably be buying a weekly supply.  
I don't know if driving an hour a week is necessarily worth it to go to Costco, but maybe it is, especially when I'm going somewhere nearby already.  
  
\item Vegetable: I have been loving crunching down on a romaine heart, but I think that literally anything which has that same general vibe\footnote{holdable in my hand, fittable in my mouth, crunchy, generally dense} would be good enough.  
Also probably an every 1-2 week thing.\footnote{am I currently literally googling (using it for brand dilution) \say{things like lettuce}? yes. Yay! I can eat chard still, but it's a little bitter. Collard greens are also very bitter when raw. And tragically, they like so many of the remaining leafy greens are related to mustard, which I have an allergy (allegedly) to. Ugh.

Endive!! Not super bitter! Not a mustard!  Escarole is like endive, and also an option.

Iceberg lettuce is always a popular choice for a reason, though I tend to find that it's too flavorless (no I will not consider dressing)}

\end{itemize}

Ok cool, that's everything I eat I think?  
Oats and frozen fruit I of course need to buy as I run low, but they last forever, and so don't need to be considered in terms of perishables.

So, I'm going to be making Japanese inspired curry on Tuesday.\footnote{I honestly think that Tuesdays could be a good food prep day of the week for me, though it does mean that Mondays are the old day. Also depends on shopping day. If I shop on the weekend, waiting a few days to cook is fine}  
Oh wait, I do also want to get through my beans.

Um.

Ok so for now let's try curry beans and rice, as much as that's not really a thing.  
I'm still going to be doing potato, carrot, and onion, since those are all delicious and my heritage yearns for me to eat more potato.  
This means my shopping list is carrot\footnote{assuming (likely accurately) that the ones I have aren't great. They're also a decent choice for the munching sensation of romaine, but different enough that I don't think interchangeable}, onion\footnote{assuming (with who knows what level of accuracy) that they aren't good}, potato, curry block, fruit\footnote{read: apple or pear unless something calls to me}, and crunchy green\footnote{lettuce or endive probably, though chard is an option. Leek? That's almost certainly wayyy too oniony}.

I'll also inventory my fridge and freezer and toss the expired and rotten food.\footnote{please don't judge my life. If it helps, the rotten food is often like homemade pickles that I was curious the lifespan of (and forgot to eat)}  
What food is not expired, I will also attempt to eat quickly, in such a way as to get rid of it.

Great!

Now, about feeding the rest of me.

I want to get back into typing practice I think.  
Two of the rate limiting factors for me right now are legitimately my typing speed and accuracy.  
That feeds three things: the part of me that loves quantifying growth, the part of me that is writing a bunch, and the part of me that likes being good at things.\footnote{two of those are the same, so how can we break it into something else? I don't really know, honestly. Uhhhh what was the third thing initially?}  
On weekdays, I think that this will mean that I spend the five minutes immediately after daily reflection doing typing practice.  
Goal in that space is of course accuracy and correct finger placement.\footnote{how useful is working for a few minutes a day when I spend the rest of the day teaching myself bad habits? Great question.}  
Actually, if I start with the practice, then I will be primed to get the finger choices and placements correct, so five minutes before doing the reflection are to be spent on the practice.

What else am I trying to feed?

I absolutely need to do guitar every day and work out every day.  
A forty minute workout in the morning is a lot, but forty minutes is forty minutes, and I'm no longer finding that I am only productive in the morning.  
If anything, it kind of feels like I'm having the waking version of the previous me's experience with alarms.\footnote{roughly speaking, for every minute earlier I wanted to leave the house, the alarm had to go two minutes earlier. So, getting up and being ready an hour before normal meant the alarm was two hours.  
These days that is not super true, but I do find that my productivity crashes when I feel like I've done a task and spent a decent amount of time.  
If I do it right, that break happens at lunch which restores me to work again.}  
  
Five minutes of stretching at night and five minutes of guitar in morning and evening means that in total I'm scheduling less than an hour of my life right now.  
I'm debating whether or not I should let myself catch up on content during the morning stretch, but am leaning towards no.  
I'll absolutely get more workouts in if I only let myself catch up on content when being stretching or cleaning, especially because I do want to keep up with content.

I want to read more.  
If I just say that I'll read each night, will I?  
One issue is that I very much cannot read or write by candlelight, at least with it as flickery as it is.  
Apparently trimming the wick can help with that.\footnote{which can't hurt to try?}  
I think that I will!  
But, I also don't want my nights to be filled with a number of activities, even if they are nominally restful.  
I think that reading analog nonfiction will help me to bed, as will writing poetry.

As the end to the previous sentence implies, I also want to write more poetry.  
Doing so after reading might actually be the best, because it gives me a space to process what I read and get my final thoughts for the day onto the page.

Woo! Look at this, I have plans for how I will feed myself.  
Now I just have to actually go to the store, get the food, clean the fridge, make the food, and keep to the schedule.  
Basically nothing!

\section{Draft 0.5: 4 April 2025}  
I've mused more than a few times about how I would like to improve my mental and physical health, especially in context of food.  
In the past, I've focused more on the higher level concerns, like the overall macro and micronutrient profiles that I should aim for.  
The other posts have generally been focused on individual recipes that I made a single time, mostly so that I would have a point of reference in the future when I forgot what I did.  
However, the most important part of keeping myself eating healithily is actually having a sustainable way of feeding myself.  
With that in mind, this post is focused on how I'm getting nutrients now, how I'd like to be, how I think that I should be, and what the differences are.

So, let's start with what I'm doing right now.

My breakfast for weekdays is a bowl of oatmeal with frozen\footnote{read: they are frozen initially, they thaw out as I mix them with the oats} berries, usually blueberries but sometimes mixed berries.  
These days, my lunch on Monday, Tuesday, and Thursday are a block of pork\footnote{approximately 200 grams} and cheese\footnote{approximately 150 grams of a semi hard. I apparently don't like Jarlsberg, which makes sense because I have more and more found that I dislike swiss cheeses as I grow older.}, along with a romaine heart and one or two apples or pears, depending on what I got at the store that week.  
Dinner, on the other hand, has no regularity  
\section{Daily Reflection: 4 April 2025)}

\begin{itemize}   
\item Intentionality:  
\begin{itemize}  
\item At least hourly, stand up, drink water, take two deep breaths, and do a stretch of some sort

I know that I did a bad job of this, but I also realized that I had 20 minutes of break between 11am and 8:15pm yesterday, and that's not including the fact that I was actively working before 11. Still, the idea of self work and externalized schedules are different.  
\item Be proactive about avoiding overwhelm and when feeling overwhelmed, stop and figure out why.

Despite that, I somehow did not feel too overwhelmed yesterday. That was great.  
\item Light a candle and read by candlelight each night. Along with this, leave all electronics outside of the bedroom and/or move them away at least an hour before bed time.

Candle time is really nice.  
It remains a pain to write by flickering candle light, but it might be good if I were to get some sort of lid like in a lantern.  
No electronics was great too.  
\item Candle time in the morning before electronics. Use the time for prayer

Did not have candle time, but I did force myself to make time to stretch and pray before getting onto electronics.  
That was probably good.  
\item Focus on good posture, especially straight back and making sure that neck isn't awkwardly positioned.

Good posture is getting easier, and I can now bring my shoulders too far back for even posture.

\item Don't waste time, and in particular, be mindful about making sure to take breaks and rest. Especially make sure to do rest which revitalizes the me of tomorrow, rather than rest which simply keeps me in stasis.

I think that I did ok with this, though prioritizing remains incredibly difficult.  
I don't know what's more important between the writing I know I need to do for work, the writing I know I need to do for my personal well being, the derivations I need to be doing for my code, the code I need to work on, the things I must do in order to be healthy in all the senses of the word, the things that people expect of me, and the things that no one is asking me to do that I don't want to do but absolutely need to do.\footnote{oof the agonizing pain of grief hurts so very very much}

\item Interpersonal Relationships:  
\begin{itemize}   
\item Figure out what belongs in a normal letter to a friend.  
  
The books arrived at the library, so now it's time for me to pick them up.  
That's going to be fun  
\item Get back into writing letters.

I think that this is always meant to be a weekend thing.  
\item Work to message friends at desired intervals.

Shoot. One of these days I will compile the information.  
\end{itemize}

\end{itemize}  
\item Professional:   
\begin{itemize}   
\item Do the Thesis and other research requirements. Upcoming deadlines:  
\begin{itemize}  
\item Brain dump about science communication (Overdue)  
\item Brain dump a publicly accessible chapter (Overdue)  
\item Have final convergences for the results I'm trying to reproduce (due 4/4)  
\item Draft of the first paper (due end of month, but I want to make sure that I've reupdated it sooner than later)  
\item Finish revising and editing the overview of a program chapter (due 4/7) and send it to the boss  
\item Revise the Science communication and publicly readable chapters (due 4/7)  
\item Send the science communication chapter to the boss (4/14)  
\item Brain Dump the background to the program (4/14)  
\end{itemize}  
\item Only do the work I feel called to when I've finished the tasks set to me for the day or outside of normal working hours (post 1725)

It's unclear to me what's work I want to do and what is work that I need to be doing.  
\item Start making the giant citation document so that I don't have to search for citations later.  
  
Nothing new, but I've at least reorganized the books I have which are at least nominally related to the research.  
\item Work towards future career:   
\begin{itemize}   
\item Figure out the difference between my public-facing and field-facing presentation affects. As I focus on becoming a better presenter, I need to become aware of the difference and how to switch them.  
\item Need to look for jobs  
\end{itemize}   
\end{itemize}   
\item Health:  
\begin{itemize}   
\item Spiritual:   
\begin{itemize}   
\item Get back into the Lenten goals (pray chaplet of St. Michael, give money equal to amount I'm spending on myself, stop scrolling social media, stop playing video games)  
\item Be intentional about prayer. That means both making time for prayer and actually doing it.

I prayed the chaplet this morning!  
I haven't been scrolling or gaming, which is good.  
I have yet to get into the almsgiving, which is a shame.  
\end{itemize}   
\item Physical:   
\begin{itemize}   
\item Start focusing on posture again, especially while sitting.

Did that yesterday, today I am in pain and not sure why. I will still work on posture, but it is a lower priority.  
\item Go to group fitness classes more regularly and more often. If not, do workout at home.

Did a workout yesterday night, despite getting home late.  
Stretched this morning because I have the full knowledge that I am not getting home until very very late.  
\item Feed myself simply and healthily. Healthy here means trying to generally avoid processing.

I don't generally think of canned food as unprocessed, but realistically it's a pretty low level processing.  
\end{itemize}

\item Mental:   
\begin{itemize}  
\item Clean Life:   
\item Remove dirt and clutter from physical spaces (standard definition of clean):   
  
Did some work on!  
Almost certainly need to make a list before we get to it.  
\begin{itemize}  
\item At least once a week, each room has nothing on the floor  
\item At least once a week, all surfaces which are not inherently storage are cleared off  
\item At least once every two weeks, each room is vacuumed  
\item At least once every month, all non-storage surfaces are explicitly washed/cleaned  
\item At least once a week, I get rid of at least one item that I notice (meaning throw away or in rare circumstances gift or donate)  
\item Clean sight lines. Is my space set up in a way that orients me towards my goals for the space? If not, how can I make it so?  
\end{itemize}  
\item Spend time each day thinking about the goals for the day, and getting them out of my head and onto the page.

Wow look at this.  
\item Continue to explicitly confront the voice in my head that says that people hate me.  
  
Woo.  
\end{itemize}  
\end{itemize}   
\item Hobbies:   
\begin{itemize}   
\item Reading  
\begin{itemize}  
\item Start reading and returning the library books I have.

I just reorganized the shelf, and there are far fewer books than I thought that there were.  
That's pretty great, which means that I have more hope about my abilities to rewrite.  
\item Finish the book on mindfulness I started. (also make a list of the exercises in the book and try them out)

Did no reading yesterday, which I don't love, but I was also very tired.\footnote{as evidenced by my sleeping all the way until the overly generous alarm I set.  }
\item Read more poetry.  
  
I wrote some! That's better than nothing.  
\end{itemize}

\item Music:   
\begin{itemize}   
\item Work on guitar  
\item Learn the songs that jam partner suggested and/or requested I learn  
\item Get back into the album.  
\end{itemize}   
\item Writing:  
\begin{itemize}   
\item Write poetry more often, ideally nightly.  
  
Wrote some last night, and even tried to work with meter and rhyme, at least a little bit.  
\item Find a way to add meta data to my blog posts and then add the meta data  
  
Is metadata all one word? I think that it might actually be. Asked my sysadmin about it.  
\item Not only write blogs, but also post them.  
  
Another one down!  
\item Get back into writing the web novel  
\end{itemize}   
\item Other hobbies, do them.  
\end{itemize}
\end{itemize}

\section{Daily Reflection: 5 April 2025)}

\begin{itemize}   
\item Intentionality:  
\begin{itemize}  
\item At least hourly, stand up, drink water, take two deep breaths, and do a stretch of some sort.

If the fact that that I didn't end up posting last night isn't enough of an indication, I did not, in fact, manage to do this.  
Additionally, I don't think that I drank water or had good posture for most hours.  
From 1300 to 1900 I was completely locked in.\footnote{note to self, you are not allowed to work on derivations until you're finished with the day's work and completely caught up.}  
\item Be proactive about avoiding overwhelm and when feeling overwhelmed, stop and figure out why.

I think I did ok with this. Didn't feel very overwhelmed for most of the day yesterday, and so that was good.  
Today, because I was double booked, I ended up feeling overwhelmed when I got to the second activity, which was fine.  
I took literally 30 seconds, wrote down what I was thinking, and then felt better.  
\item Light a candle and read by candlelight each night. Along with this, leave all electronics outside of the bedroom and/or move them away at least an hour before bed time.  
I didn't get home until very late last night, and so I did not do this.  
\item Candle time in the morning before electronics. Use the time for prayer  
I also did not do this. Oof  
\item Focus on good posture, especially straight back and making sure that neck isn't awkwardly positioned.

Forgot about this for most of the day, and then was reminded by looking at others, and so I am once again reminding myself that posture is important.

\item Don't waste time, and in particular, be mindful about making sure to take breaks and rest. Especially make sure to do rest which revitalizes the me of tomorrow, rather than rest which simply keeps me in stasis.

I think that I generally did ok with this.  
I spent last night hanging out with some friends, and it was great to see them, even if it did mean that I was out far later than I would otherwise normally choose.

\item Interpersonal Relationships:  
\begin{itemize}   
\item Figure out what belongs in a normal letter to a friend.

Shoot, in the fugue/flow\footnote{honestly, that's got promise for a musing. What is the difference between a fugue and a flow state?} of yesterday afternoon I did not end up picking up the books.  
\item Get back into writing letters.   
  
Tomorrow is Sunday, and so tomorrow a letter is to be written.\footnote{Is to have been written is I think the construction I need for agenda's construction in Latin, but I may be wrong, since it has been the better part of a decade (oof I'm old) since I took the class and learned that content}  
\item Work to message friends at desired intervals.

I still need to make the list, but I do feel like I'm doing an ok job of contacting some friends.  
List really needs to start moving up the priority list.  
It's just hard because, while I adore and feel close and want to remain close with a number of people, I also feel like I am already so far behind on my thesis writing.\footnote{to get to anything resembling the goal I have for the length of my thesis, I need to be writing thousands of words a day every single day.  
While I have absolutely had periods of doing that, I don't know if they're the pace I have right now.  
Also, like I guess the thesis is a higher priority than almost any other research I would be doing, so that should leave me more space for writing, especially once the term ends}  
I have some objective standards for that, since I do, in fact, have a draft due nominally on Monday that I haven't even finished writing, let alone leaving for a week before revising and another week before editing to send it in.  
However, I will always be behind, and I need to start to come to terms with that.  
\end{itemize}

\end{itemize}  
\item Professional:   
\begin{itemize}   
\item Do the Thesis and other research requirements. Upcoming deadlines:  
\begin{itemize}  
\item Brain dump about science communication (Overdue)

On the bright side, I had the event today, so I can at least write about how it went in the form of a laboratory report.  
It's almost certainly not worth publishing, but that's why it gets to go in the thesis.  
If I start reading the books on teaching astronomy and chemistry that I have in my shelf, I can write something about that, if only as a \say{this could be a cool experiment for someone to use}  
\item Brain dump a publicly accessible chapter (Overdue)  
  
I've been trying to explain my research to people more and more, and it's hard because one answer is \say{I'm writing AI to replace the work of assigning molecular spectra}, which is in the broadest terms true, but that's really the only terms where it's true.  
Anything else, though, requires me to explain what rotational spectroscopy is.  
\item Have final convergences for the results I'm trying to reproduce (Overdue)

I figured out\footnote{at least one more reason} why the code wasn't converging correctly, and spent some time yesterday with the derivations trying to make the speed up I had plausible.  
However, I should instead just finish updating the code to run with both ideas that I have.\footnote{this is not me being data shy or coy, I just don't entirely have in my working memory right now what the two ideas are.}  
I started before the activity this morning, so that was good, but I also need to clean the code and delete the extraneous code\footnote{I think that about half the functions I wrote in the code aren't actually called anymore.}  
\item Draft of the first paper (due end of month, but I want to make sure that I've reupdated it sooner than later)  
  
I looked at it today, because I deleted the actual file which had my compiled comparisons between theory and experiment at some point.  
I knew that it was in the file, then realized that one benefit of cookies is that the site I haven't been to in at least 8 months still had the entire spreadsheet in the different forms I might want it to be expressed.  
That is literally all that I did, though.  
\item Finish revising and editing the overview of a program chapter (due 4/7) and send it to the boss

This really means finish writing.  
However, since I can't really distinguish between the background and the overview, I think that it might have to get broken into different sections.  
I don't know how, but what's most important is getting all the content on the page, both for the words and so that I don't have to remember to add more content.  
I can't revise an empty page and all  
\item Revise the Science communication and publicly readable chapters (due 4/7)  
\item Send the science communication chapter to the boss (4/14)  
\item Brain Dump the background to the program (4/14)  
\end{itemize}  
\item Only do the work I feel called to when I've finished the tasks set to me for the day or outside of normal working hours (post 1725)

So, yesterday was not a good example of that.  
I did have it in the to do list, but that isn't really so much a reason as an excuse.  
\item Start making the giant citation document so that I don't have to search for citations later.

I reorganized my bookshelf, so the books that I am most likely to put in my thesis are all in the same place now.  
\item Work towards future career:   
\begin{itemize}   
\item Figure out the difference between my public-facing and field-facing presentation affects. As I focus on becoming a better presenter, I need to become aware of the difference and how to switch them.  
\item Need to look for jobs  
\end{itemize}   
\end{itemize}   
\item Health:  
\begin{itemize}   
\item Spiritual:   
\begin{itemize}   
\item Get back into the Lenten goals (pray chaplet of St. Michael, give money equal to amount I'm spending on myself, stop scrolling social media, stop playing video games)  
\item Be intentional about prayer. That means both making time for prayer and actually doing it.  
\end{itemize}   
\item Physical:   
\begin{itemize}   
\item Start focusing on posture again, especially while sitting.

I don't know if I'm doing something wrong but wow is sitting up straight kinking my shoulder something fierce.  
\item Go to group fitness classes more regularly and more often. If not, do workout at home  
  
Haven't worked out yet today, but am planning to when I get home.\footnote{because I have the goal of getting work done, and I still don't let myself work at home.}  
\item Feed myself simply and healthily. Healthy here means trying to generally avoid processing.  
  
Ehhhh, most of my food since the last reflection was potluck style.  
I don't believe in mandating someone else's diet, especially when they are helping to feed me.  
\end{itemize}

\item Mental:   
\begin{itemize}  
\item Clean Life:   
\item Remove dirt and clutter from physical spaces (standard definition of clean):

It took me all morning yesterday, but the area above my desk is again ordered, and significantly moreso\footnote{I really feel like that's a word, but my editor assures me that it is not. Weird} than it was before.  
My home,\footnote{oof, I wrote the rest of my home at first. I promise that I do not live in my office nor mentally think of the office as my home} however, is in a static or slightly worsened state.  
At some point I'll have to take the time and commit to actually just giving a day to a deep clean, but that day is not today.

\begin{itemize}  
\item At least once a week, each room has nothing on the floor  
\item At least once a week, all surfaces which are not inherently storage are cleared off  
\item At least once every two weeks, each room is vacuumed  
\item At least once every month, all non-storage surfaces are explicitly washed/cleaned  
\item At least once a week, I get rid of at least one item that I notice (meaning throw away or in rare circumstances gift or donate)  
\item Clean sight lines. Is my space set up in a way that orients me towards my goals for the space? If not, how can I make it so?  
\end{itemize}  
\item Spend time each day thinking about the goals for the day, and getting them out of my head and onto the page.

I really need to start the day with this more explicitly.  
I don't know what medium to use, though, because I don't want to be on the computer early, but I do also primarily look at the computer, these days.\footnote{see: me constantly kvetching about the amount of writing that I am to do}  
\item Continue to explicitly confront the voice in my head that says that people hate me.  
  
Spent time with friends last night, and I didn't even think that they hated me once before leaving!  
\end{itemize}  
\end{itemize}   
\item Hobbies:   
\begin{itemize}   
\item Reading  
\begin{itemize}  
\item Start reading and returning the library books I have.  
  
They're organized now, into the four relevant kinds.\footnote{things I think or hope will be in my thesis, popscience, actual philosophy, and books I thought might be popsci but look to be much lower reading level than I thought}  
\item Finish the book on mindfulness I started. (also make a list of the exercises in the book and try them out)  
  
Have made no progress. I will do so tonight, though.  
\item Read more poetry  
\end{itemize}

\item Music:   
\begin{itemize}   
\item Work on guitar  
  
Oof. I haven't touched it in days.\footnote{I keep wanting to animize or anthropomorphize my guitar. Is it better to say she? he? they? xi? Great question. I think it is still good for now, because i still know it is not}
\item Learn the songs that jam partner suggested and/or requested I learn  
\item Get back into the album.  
\end{itemize}   
\item Writing:  
\begin{itemize}   
\item Write poetry more often, ideally nightly.  
\item Find a way to add meta data to my blog posts and then add the meta data  
\item Not only write blogs, but also post them.  
  
I didn't do either tomorrow, which isn't great.  
\item Get back into writing the web novel  
\end{itemize}
\item Other hobbies, do them.  
\end{itemize}   
\end{itemize}
\end{document}