\documentclass[12pt]{article}  
\newcommand{\say}[1]{``#1''}  
\newcommand{\nsay}[1]{`#1'}  
\usepackage{endnotes}  
\newcommand{\B}{\backslash{}}  
\renewcommand{\,}{\textsuperscript{,}}  
\usepackage{setspace}   
\usepackage{tipa}  
\usepackage{hyperref}  
\begin{document}  
\doublespacing  
\section{\href{faith.html}{On Faith right now}}  
First Published: 2025 June 7

\section{Draft 4: 7 June 2025}

I'm realizing right now that each draft is becoming not so much a draft as a background or full folly itself.  
Because I want to explore the entire issue today, though, the 1500 words I wrote to get to what I think it means to sin (Draft 3, and the sin bit is only the last 50 words. Draft 2 led to Draft 3), and the words in Draft 1 where I explore the motivation here of this post (finding the things that I believe at the deepest level) are nothing more than prior drafts.  
So, let's state the goal of this folly: I do not know if I can be a Catholic any longer.  
To be a Catholic, I must agree with the Church.  
The Church claims that the goal of all humanity is to get to heaven, and that means that all actions are either sin or love (see Draft 3).  
So far we're good, but this is where we break from pure philosophy and theology into the real world, where the Church teaches that some actions are inherently love or sin.

Within Church teaching, there are still a number of different ways to approach something.  
There's the common question of level of binding authority: if a priest says something, that means less than a bishop, means less than a pope, means less than the current pope, means less than a full ecumenical council.  
There are obviously shades to the binding, but generally the framework is the more that the Holy Spirit prevents error in the teaching, the more we must believe the teaching.  
In addition to how binding a teaching is, there is also the question of interpretation.  
It is dogmatically taught that Mary was assumed body and soul into heaven.  
Whether that happened before she died or after she died, or even while she died, the Church has not   
Finally, there is the question of interpretation.  
The world is a fundamentally different place than it was even a decade ago, let alone the two millenia that the Church has existed.  
I don't know if I can believe that words themselves are binding, only that the ideas they contain are binding.  
Different cultures will require different interpretations of the same truth.

From Fides et Ratio, and probably other sources, we know that truth cannot disagree with truth.  
Science and Faith are not just not contradictory, they are actively complimentary.  
The Church can speak on matters which science cannot: salvation, eternity, the non-physical.  
Science can lead us to the Church by predisposing us to truth.  
Also, like not being hungry or sick is good, and science is really good at making food and medicine, and especially when we need it to continue.  
Multiplying loaves and fishes fed the crowd for a day, synthetic ammonia feeds half the world every single day.

So, part of what I'm going to do here, and what I feel comfortable doing, is calling out places where what we teach as theology is actually an empirical fact.  
If we can measure it with science, then science can give us the answer to the question.  
If this disagrees with the Church, then we must remember that all on earth are fallible, and there is no perfect communication between others.  
As I once explained to a friend, I trust G-d absolutely; if He tells me to do something I (as a correct moral agent) do it without hesitation.  
If I do not know without a doubt that it's G-d, though, I have to consider how reliable my experience of reality is, and what I understand that I am being told to do.\footnote{mmm descartes}

If someone tells me that G-d told them something, there's yet another place for mistruth to come in: there is no perfect way to transmit knowledge.  
Someone telling me a Divine commandment means that I must trust that the person is correct in hearing the divine and also my interpretation of the words that the person gives, which are themselves an interpretation of what they believe the Divine said to them.  
There is the fun binary scaling we can start to do: did G-d speak (yes/no), did human agent understand correctly (y/n), did human agent attempt to communicate this correctly (y/n), did human agent communicate what they attempted to communicate effectively (y/n) (repeat those three for as many layers as needed), did I understand what they were trying to communicate?  
This is nice because even as it leaves many spaces for truth to die, it also gives space for the Holy Spirit.  
At any point, someone can try to intentionally mislead, but the interpretation might still end up being the truth.

Anyways, what do I dislike about the Church?

Ever since learning about the Jesuits' history, I've felt uncomfortable with a large aspect of my faith.  
The more that I look into the historic Jew-hatred of the Church, the worse that it gets.  
There have been binding prohibitions against Christians associating with Jews, and the Council of Florence expressly teaches the dangers of the Old Covenant.

Both as a Jew and as a believer, I find these problematic.

One of the really really really important things about covenants with G-d is that they are eternal.  
We do not need to fear the world drowning because He covenanted not to flood the world.  
We can hope for salvation because Christ covenanted with us, taking all our sins on himself.

However \href{https://catholicism.org/ad-rem-no-63.html}{it is very clear that} the Church does not want Catholics to follow the Covenant with the Jews.  
I understand this in the case of Gentiles, those who were never bound by the Covenant between Abraham and G-d.  
In the case of those like me, though, who were born into Abraham's physical, not just spiritual, line, what am I supposed to do?\footnote{oof the commas there did not serve to make the writing more comprehendable (comprehensible? ope yeah that's what my word spell likes)}  
The Early Church saw themselves as Jews.  
  
Peter said that others were not bound to join the Covenant not because it was wrong, but because it was a yoke which is too hard to bear.  
From a footnote in the previous draft, I know that I must always act in the interest of my own salvation.  
It is a commandment that I, who has been introduced into the Covenant of Abraham, bring my own children into the fold.  
On the other hand, as Peter points out, that makes it harder for them to be saved.

This is really where my issue comes into play.  
Every other covenant with the Lord that we have seen in the Bible treats them as eternal.  
Why is this, the inarguably most impactful one in terms of individual actions, somehow the one which is discarded?  
I know the Catholic answer, which is that the new covenant supercedes the old, but even that is only partially true.

Non-Christians often complain about Levitical law, and especially Christian interpretations of it, because we do not take everything.  
Catholics often reply with the idea of three forms of law under the Covenant: those which bound the Jews, those which bound the State of Israel, and those which were \say{moral} law, and therefore bound all of humanity.  
I do not take issue with this idea, nor do I even have an issue with saying that the Church has the authority to say which is which.  
I have no issue with the idea that what laws bind Jews, not the Israelites\footnote{I don't really know how to distinguish the people from the specific pre-Temple culture. I guess there's also the whole like \say{pre first, first, post first, second, and post second Temple periods and requirements}} could have been amended in the New Covenant.  
I have no issue with the idea that the New Covenant allows people to be bound only by the new version, rather than both Christ's and Abraham's.  
I just cannot think that the laws which bound Jews as G-d's people can be discarded wholesale.

I suppose that there is also the question of what it means for G-d to make a covenant with the nation-state of Israel in addition to the Jewish people.  
It is obvious, I feel, that any covenant with the Lord would not prescribe\footnote{proscribe?} actions which are sinful; sin is definitionally breaking relation with Him.  
It makes sense that there are moral truths which may not be obvious to an outsider but which are good to all people in the covenant:\footnote{when to use colon and semi colon???} all humans share in Adam and Eve's responsibility to tend to the world.  
It makes sense that there are rules which do not bind the rest of the world but do bind His Chosen People: \say{to whom much is given, much is expected} and also just like that's the whole point of making something sacred, you set it aside and treat it differently than the norm.

How do I draw the line between what commandments G-d gave to the Jewish people and the Jewish nation, though?  
What is the difference between the two?

I get to the issue of borders and immigration here, much as I didn't realize that one of my current and acute complaints with the Church was related.  
Many top people in the current administration claim to be Catholics.\footnote{is it uncharitable for me to frame it like this instead of saying that they are Catholic? maybe? I'm not totally sure. I think that there's something to be said that, just as Peter didn't think that it was good to bind Gentiles to Mosaic (there's the word that I was looking for. Oh shoot wait, what about the things in Abrahamic Law, are they binding? Are they released? Were there bindings?) law, describing them as not Catholic but claimants means that they have a reduced culpability when they preach heresy}  
Recently, one of them claimed that the Church supported closed borders, or something similar.  
In response, the bishops of America, including my own local bishop, drafted a letter where they pushed ever so slightly against that claim.  
In the letter I received, we were told that \say{The Church opposes both completely open borders and completely closed borders.}

This is not, so far as I have been able to find in any source, true.

The Church absolutely and emphatically has rejected closed borders.  
The Church has said that nations have a right to monitor their borders to maintain the safety and well-being of their people.  
That is not the same as saying that they have an obligation to, nor is it even saying that nations are themselves good.  
A completely open border is the same as saying that there is not a border, I would argue.  
The Church is generally pretty clear that any divisions between humanity\footnote{other than sex, and to some extent age} are fundamentally wrong, or at least not ideal.  
I assumed that the Church's stance on borders was like its stance on property: in an ideal world we would not have private property\footnote{the apostles, after all, shared all they had.}, but we must be able to choose to share what we have.\footnote{at least in part: the Church also believes in the universal destination of goods (or some similar phrasing), which says that you can morally take some things from some people (I think)}  
It would be correct to say that the Church rejects communism and socialism as concepts, because both explicitly require their citizens to own no property.  
It would not be correct to say that the Church rejects communal living, where no one has property of their own.  
After all, in heaven we have no property.

The letter really feels like a great example of the issue I have with the current princes of the Church, especially in America.  
In what is a very clear example of explicitly anti-Catholic teaching (taking immigrants from Catholic Churches during Mass), they were unable to unequivocally call the action bad.  
Instead, in addition to the above statement, they also said \say{Immigration policy must achieve a proper balance between migrant rights and sovereign rights.} and \say{secure borders help everyone}.  
This is fine, but is not a strong enough statement.  
Much as I find it inherently wrong when someone says \say{why are you complaining about X atrocity when Y is also occurring?}, it is wrong to say such things as \say{our U.S. immigration system has been broken for decades, no matter which party holds power.}  
That's true, sure, but it is not both parties creating and posting on official government channels videos which glorify the pain of deportation.

They go on to say \say{Distinctions must be made between immigrants who present genuine risks
and dangers to society and therefore may be lawfully expelled, and those
who have been here for years, have no criminal record, and have lived
peacefully and contributed to the common good.}  
There is a huge gulf between these two options.  
What about those who have a criminal record because there is no way to feed a family when you cannot legally work?  
Living totally peacefully, contributing to the common good, and presenting no genuine risk or danger to society is not in any way contrary to having a criminal record.  
Even more than that, though, the Church explicitly preaches forgiveness.

We have always said that there is no sin which precludes holiness.  
Why, then, do we now believe that there is no forgiveness for crimes?  
Is crime fundamentally damning in some way that sin is not?  
Is serving the prison sentence that our nation decided was fair\footnote{I'm not getting into the whole \say{as Catholics there is no moral way to defend the current system} except wait no that's exactly what I want to get} not enough to say that you have repaid the debt?  
Why is it that we believe there are actions that cannot be atoned for, despite the fact that we preach the reverse.

Our last Holy Father was clear that we are obligated to support the environment, the migrant, and the poor.  
Our last President professed being Catholic.  
There were a number of bishops who said that he, by virtue of being willing to sign into law a bill which would ensure that a woman struggling through the pain of a sudden loss of child could not suddenly face punishment for that\footnote{am I going to write the whole post about abortion soon? Gosh I hope so}, was fundamentally unfit to receive communion.  
There have been none who have said the same about our administration, despite these people literally and explicitly calling the Pope wrong and saying they don't have to listen to him.

Regardless of how I feel about any Church teaching, deference of intellect and will means that I cannot, in any way, shape or form, explicitly call the Church wrong as a faithful Catholic.  
I can say that my own formation did not lead me there, or that I think that the Church is not interpreting the teaching correctly.  
I cannot, however, publicly and directly disagree with the Pope and call myself a good Catholic.

The Church claims to stand above the culture, and yet we are caught up within it.

Ok so returning to the people of Abraham, if covenants are eternal, the Jewish people are who G-d covenanted with, and they still exist\footnote{all of which I am almost positive we are bound to believe}, then how is there a distinction between the laws of the Jews and the laws of the Jewish nation?  
The nation is the people.

I understand from a practical level how this works: clearly many of the banned actions explicitly referred to other nations which no longer exist.  
This does not mean the law no longer binds, only that the binding no longer comes into play.  
Just as I am in no way bound by the restrictions on those with children right now, that does not mean that the restrictions do not exist.  
Still, I suppose that it is helpful to note that some of the covenant is not practically binding, because it is no longer workable.  
Also, like any portion of the covenant involving the Temple is impossible right now, since there has not been a temple in centuries.  
I guess referring to the nation's laws like that is workable.  
Ok so that's no longer an issue, great.

I have traveled so far that this post is almost unfollowable to me.  
Still, I think that I'm at the point right now where I accept that what I have issue with is the teachers of the Church, not really or necessarily the teachings of the Church.  
As I reconnect with my Jewish heritage, it is possible that I will disagree with the teachings, but until such a time, I feel that I can call myself Catholic.  
I may disagree with the interpretations of the bishops, but I cannot say that my take is the Catholic one.  
I am willing to stand against the Church Militant because it is not inerrant.  
It is, however, still the Church.

Tl;dr: I am mad that our bishops are embroiled in the culture war and I still don't know how I, a Jew in the eyes of the Church, should and do exist in the world.

One of the reasons that it's been really hard for me to know that the Jesuits, until incredibly recently, would not have let me join is that I am so drawn not just to Ignatian spirituality in theory, but also Jesuits in practice.  
There is no way to know the bounds of our faith except by knowing what is outside of it.  
Science is a way to know G-d's creation, and knowing the creation helps us to see the Creator.  
Of the big three spiritualities, Franciscans are focused on service, Dominicans on teaching the normative faith\footnote{initially written as parroting Aquinas, but that's not charitable}, and Jesuits on the way that non-theological knowledge can lead us to salvation.

I am soon to be a doctor.\footnote{if all goes according to the current timeline}  
I cannot view the world except through the knowledge that the beauty of creation is there for us because our G-d loves us so.  
What does it mean that the part of the Church who teaches this only recently would let me join?  
If I were but a few decades older, it is impossible that I would have been called to join them, I suppose.  
The Church is meant to stand outside culture, leading all of humanity to heaven.

The Church is its members.  
The Church Militant, and many of its leaders, have done atrocious things.  
In the same way that I am comfortable with the idea that saying \say{this country did something awful} and \say{the leader of this nation did something awful in his official role as the leader of the nation} are generally interchangeable.\footnote{I do of course understand where that goes wrong sometimes}  
When the princes of the Church actively harm the most vulnerable among us, when they will not make clear stances against evil, and when the institutions they maintain, create, and endorse do evil, how can I in good conscience say that every single person in the world is best served by joining the Catholic Church on earth.  
Do I believe that all in heaven are part of the Church Triumphant?

Yes.

Do I believe that all are given the choice to join it at death?

Yes.

Do I believe that living a moral life helps prepare us to make that choice?

Yes.

Do I believe that living according to the teachings of the Church is the generally best way to live a moral life?

I don't know any more.

\section{Draft 3: 7 June 2025}

Lately I have been struggling with my faith.  
There are a lot of things that the Church teaches which are generally good for belief.  
However, there is much theology that is good or bad based on the individual.  
There are a number of works which have received a \textit{nihil obstat}\footnote{lit: nothing obstructs}, which says that there is nothing contained that is expressly contrary to faith and morals but which do not receive an \textit{imprimatur}\footnote{let it be printed, I think} which says that the content is good for the development of the Catholic faithful.  
That makes sense to me, and I don't really feel a need to justify it right now.  
As a Catholic, my goal is the salvation of the world, but explicitly the goal of my life is to end up in heaven for an eternity with the Father of Creation.\footnote{a question I often consider is if is better to damn myself if it meant that I could guarantee the salvation of N other people, or the reverse: is it ok to choose salvation if I know that it will lead to N people being damned? I know the correct answer, much like in the trolley problem, is \say{that's not a real situation}, but I think that it's important to look at the edge cases, even if they cannot exactly occur.  
This is probably the scientist in me, who is ok with the idea that the models we use may be non-realistic, but still useful.

For the purposes of this post, I think that I am going to go with it is always best to choose my own salvation.  
I am the only person in sole control of my soul.  
Hmm am I ok with that?  
Yeah, I think so.

That's another set of questions, I guess. I've asked five of the Catholics whose opinions I deeply trust what they think  
  
Ooh yay the seminarian was the first to write back! Edit: just before writing the quote from Corinthians, got the response back.  
He said to use double effect. My goal is Heaven, and so any consequences are fine (not in so many words, but effectively). In contrast, Church says you can never take an action you know to be wrong (choosing hell). So, it's really the standard catholic answer to the trolley problem \say{you can shift the lever to save people, but you can't do it with the expressed moral calculus of N less than M}... which is fine because I do believe in agency. At the extreme level, I must always choose the good.}

So, what is the absolute minimum that, as a Catholic, I believe is necessary to believe for salvation?  
Note that this is a distinct question from what is necessary for salvation.  
Just as it is not necessary to know how internal combustion works to drive a car, there are bound to be a number of truths which I do not need to believe in order to be saved, and there are a greater number of things which if I truly believe what is necessary for salvation, I would believe by virtue of being a thinking and semi-logical being.

It's very clear in Catholic theology, especially since Luther, that we do not earn our place in heaven.  
It is only Christ's unmerited and infinite love that lets us in.  
From this, I've always understood the answer to what gets us into heaven is that at the end of our mortal existence, we are given the choice between eternity with G-d and eternity without Him.  
I've also seen framings of it as eternal bliss or eternal torment.  
In either case, there is the transitory state known as purgatory, where we go if there are worldy attachments which keep us from fully saying Yes to the Almighty.

So, what are these worldy attachments?

The Church often distinguishes the Secular and the Sacred.  
Secular is literally just meaning \say{of the age}.  
I've understood this, especially in light of what else I know\footnote{to be clear, almost none of what I'm putting here is what is necessary and / or sufficient to get to the conclusions. I'm just putting what feels like the most important notes for myself as the if of an if then, so that I can trace my logic where needed. If I disagree with a conclusion, I can then go back and see what might need to be modified to stop leading to an incorrect location.

In analogy: I'm trying to get to point A, and know that it's north of me. It's totally possible knowing that will be enough to get me there, but it's also possible I'll need to be more precise (north east? northwest? which highway do I take?)  
Since the goal is the location, and I know whether or not I've gotten there (based on whether or not I feel Catholic at the end of the logic train)

This feels fundamentally Ignatian in practice/principle, and that is itself another reason that I'm uncomfortable, because Hitler did in fact point to the Jesuits as an example of how to do Jew hate well (something something, I think if up to your great grandfather was Jewish, you were not eligible to join them until like the 1980s)

Anyways, back to main text} to mean that Heaven is that which is eternal, and the process of Purgatory is where we burn away our attachments to anything which is not permanent.  
That's fine with me in principle, and I absolutely agree with some of the ways the Church, when asked, expressly teaches this.  
G-d is Love, so love is always good insofar as it leads to Him.  
However, it is absolutely possible to love G-d less because of attachment to a spouse or parent or child or etc..  
I even think that this can go further.

In the first draft, I talked about hydrogen emission.  
Even this is a transitory thing.  
There has not always been hydrogen, and there will come again a time when there is no hydrogen.  
Even though it is something we take as foundational in science, our attachment to this cannot be eternal.

Ok so to get to heaven I need to say Yes to G-d, which means releasing my attachments to anything transitory.  
What is eternal?

Paul, for all that I have my qualms with a lot of his writings, is one of the major authors in the Bible.  
His teachings are well accepted by the Church, especially when taken with proper context.  
To the best of my knowledge, First Corinthians 13:13 is not one of these verses that can be taken out of context.  
Depending on the translation, three things are eternal: faith, hope, and either love or charity, which is the greatest of the three.  
Here there's the fun sermon priests love about \say{in Greek there are many words for love. In English there is just one.}  
Talking to a dear friend once, she reminded me that English is a language that expressly constructs multiword meaning.  
Just like we don't have a one word infinitive but still have infinitive verbs, we do not have one word for eros, but can say erotic or sexual love.

Ok so to get to Heaven, we can only be attached to faith, hope, and love, especially love for others.  
That's totally cool with me.  
If I'm thinking of an omnibenevolent and omnipotent Creator, I would hope that Love is essential.  
Exactly what Faith and Hope mean here, I'm sure has meaning, but I don't know what they do in the practical.

This was a circuitous\footnote{more than 1000 in this draft alone if we count the monster of a footnote} way of getting to something my mother really impressed on me: all theology needs to start and remain focused in Love.  
I'm glad that I got there kind of on my own, even if it was one of those \say{I knew on some level the answer that I was getting to, and so the logic trail led me there}.

So, all theology must be rooted in love.  
Is Church teaching rooted in love?

What Church teaching do I disagree with?

What is my crisis of faith?

Honestly, I think that I might need to restart from there, rather than continue here.  
I feel comfortable with: heaven exists, heaven is union with the Divine, union with Divinity requires separation from that which is not Divine, anything which is Divine must be eternal and anything eternal must be Divine\footnote{though this second half maybe less clear on, is it an equivalence? sure I'll say yes for now}, and so to go to heaven we must be removed from anything which is not eternal.  
Of things on earth, Love is one of the few eternals, and so all actions which are carried out in love bring us closer to the Divine, and all actions which are carried out in absence of love pull us away.  
Great.

Now let's get into the meat: what are the things that the Church teaches are inherently actions with and without love?  
Oh yeah that's absolutely what the issue is: Sin, the Church teaches, is that which separates us from G-d.  
Love, as I showed above, is what brings us to G-d.  
Therefore Sin is what is not Love and vice versa.  
Is  it reductive to say that all actions are either Love or Sin?

Maybe?

I guess that we get to principle of double effect: is the primary goal of an action I take love or sin?  
If Love, then good, if sin, then bad.  
Grey areas exist, but I can get to them when I do.  
The Church teaches many things as fundamentally good and fundamentally bad, and so I can explore whether I agree with those, how binding the teaching is, and how open to interpretation the teaching is.\footnote{will probably need to cover this in another draft}

\section{Draft 2: 7 June 2025}

As an important preface, this post is me actively figuring out how and what I think as I write.  
It will jump as my thoughts do.  
My goal is to see whether I am, in fact, a Catholic.

A theologian I recently read claimed that Aquinas\footnote{if I remember correctly} said that if a Catholic truly and honestly finds that they cannot believe in the Church, it is a mortal sin to claim to believe what he does not, and only a venial sin to apostatize.  
I've never been super cool with the idea of mortal and venial sin as an important distinction, which I think is also somewhat of an Orthodox thing.  
Anyways, it is from this and something my mother raised me with that I come to this post.  
My mother always taught me that, as a Catholic, I have two moral options: complete obedience and complete freedom.

Complete obedience is, on the face of it, the easier option by far.  
The Church gives us leaders in the form of priests, bishops, and the pope.  
If you are willing to believe exactly what your ecclesial authority\footnote{the priest of the closest parish to you, his bishop, and the current pope} claim for morality, any sin you fall into as a result of incorrect theology is not on your soul.  
This is where we remember the whole \say{millstone around the neck} that is better for bad teachers than misleading.  
Of course, you do not get to pick and choose.  
You can't go \say{Fr. So and so says this thing I like, and Fr. So and so this one, and this Bishop said this, and etc.}

If you want to, on the other hand, make choices,\footnote{can you tell that my mother had Opinions about what the right choice was?} any sin is solely on you.  
When you see the Lord at the end of your mortal life, you have full responsibilities for your actions, because you are a moral agent.  
Developing your conscience, then, is essential.  
Even more than that, though, constantly evaluating your conscience is also essential.  
There is a famous thing about walking: people cannot do it in a straight line.  
No matter what you try to do, when walking without an external method to maintain straightness, you will veer and curve.  
Likewise, without interrogating your beliefs, it is easy to slip into what is easy, rather than what is true.

So, let's go through and see what this means for me and the Church.

The Church claims that the end destination for all humanity is heaven: eternal union with the Almighty, a personal agent who created everything and lovingly holds all of reality, constantly sustaining it.  
I'm not going to get into the theology\footnote{literally using the word here, I'm not going to break down G-d right now, just Salvation} here, but instead hone in on the Salvation.  
As a Catholic, what is essential to believe to enter Heaven?

If I can agree with everything essential, then there is no issue with me remaining Catholic.  
If I cannot, then I do owe the Church deference: I must do everything in my power to try to believe Her answer, even if it is hard.  
If I am unable to do so, though, then I cannot be a Catholic.

So, what is essential?\footnote{500 words into this draft I finally know where I wanted to start... We're at almost 5K total words right now, and so this is absolutely going to be an unreasonably long post, even if I finish it today}

I don't honestly know.  
So far as I understand it, we do not earn our place in Heaven.  
It is only Christ's unmerited and infinite Mercy and Love which brings us there.  
We accept or reject this Love, and that determines Heaven or Hell.  
There's also purgatory, for those of us who cannot choose Love in full, but would eventually be able to.

I'm not entirely sure if we are obligated to believe that Hell is eternal.  
As far as I can tell, Hell will be ended at a point in time.  
Heaven and Hell, by contrast, exist atemporally.  
Is this relevant to my belief? Maybe!

There is an idea that I have sen and really like, even if I don't know that I believe it, that Heaven and Hell are not, in fact, different places.  
Instead, the same omnipresence and full knowledge of this presence of the Lord is either something lovely or unbearable.  
Those who find it lovely are in Heaven, and those who cannot stand it are in Hell.  
This makes purgatory work really well too!  
A lot of theology talks about how faith is process of refining by fire, and there's a lot of metaphor of burning.\footnote{mmm fire}  
When you fully accept the Lord, there is nothing in you which needs to be burned away.  
When you reject him, you are constantly being consumed.  
And, when you have the last few things separating you from Him, purgatory burns that away.

Honestly, I think that I might be best served restarting with this.  
Let's try it.  
I know that it makes no objective difference whether the next words come here or above in a new draft, but it mentally feels very different.\footnote{which I guess is a form of objective difference, since it is measurable}

\section{Draft 1: 7 June 2025}

I'm just going to preface this with the full knowledge that I am aware this post is almost certainly going to meander not so much like a creek, but more like a laser beam sent into broken glass.  
Expect a number of hard pivots, soft transitions, and everything in between.  
With this disclaimer out of the way, let's think about faith.

I've talked a lot about faith on this blog, and I think that makes sense.  
Faith is an important part of my identity, and I do try to have my beliefs shape my actions.  
Since October, though, I've been finding my faith harder and harder.  
This is not that surprising: my mother was the one who grew me in the faith, and so much of the way I believe and believed is based on her.  
Without her, there's a gap and that's not great.

So, as of right now I am a professed Catholic.  
I believe all that is required to believe, and I generally keep normative opinions.  
I offer deference to the Church, and where she and I disagree, I try my best to believe what the authorities tell me.  
Is this going to be true at the end of the post?  
We shall see.

I've been listening to a philosophy channel more and more lately, if only because he puts out a lot of content and it's generally interesting.  
I've also been reading not no philosophy, and that's been really cool too.  
Something I think about a lot is the semi-modern\footnote{I think} idea that there must always be some truth you hold deeper than the rest.  
Mathematicians might know this from Godel, who showed that there are infinite truths that cannot be proven within any self-consistent system.  
That is, in order to find any truth, you have to start with foundational assumptions that cannot be questioned.

What are mine?

I don't know, honestly.  
I try my best to remain open to the idea that I could be wrong, and try to hold ideas as lightly as they deserve.  
If I cannot effect a change based on an ideal, then there is no point in being dogmatic about it.  
I guess that both of these, the openness and lightness, are foundational, but I have certainly been more dogmatic in the past.  
Yesterday someone expressed that they would love to have a child like me and I tried to say that I was a bad child because I followed rules badly.\footnote{in that I saw no gray areas often.}

Why do I say that I am Catholic?

That's an easier question, certainly.  
I've taken the Sacrament of Confirmation, where I pledged myself to the Church.  
Even though there is nothing technically requiring me to of full knowledge accept the faith, the cultural understanding of the Sacrament remains.  
That's certainly sufficient when I don't have questions, but.

I often talk about some of the reasons that I find the Church to be the best faith option.  
What reasons do I give?

I want to live in a world with fundamental meaning.  
This is not something which can be empirically tested or derived, but is something I do deeply hold.  
What is that meaning?  
Great question, no idea.  
I include in this that there are truths which are universal, and that morality is not something culturally constructed, but rather an absolute set of laws that we may or may not follow.\footnote{do I know what that absolute morality is? No. Do I think that there is no way to make it totally parsable by humanity, because it needs to account for literally every consequence of every action? Yes}

So, whatever worldview I take needs to have reality.

I don't couple this with my next point, even though most do: I want to live in a world that is self-consistent.  
That is, I need to believe that reality is measurable\footnote{no descartes demon} and the same.  
If I measure the speed of light today, it will be the same tomorrow.  
If I measure it here, it will be the same elsewhere.

On my last drive home, my mind, lulled into peace and contemplation by the endless rolling hills, started thinking about this question.  
I realized that, much as I have historically talked about Truth and truth as distinct things, I don't entirely know what I mean by these.  
There was a rhetorical idea that seemed interesting, and I will try it going forward: capitalization is reserved for those things which are, under my philosophy, non-dependent truths.

what do i mean by this?

take the classic astronomy benchmark: the hydrogen 21 centimeter transition.\footnote{initially i had 12, turns out i had them flipped}  
what does this truth rely on?  
i'm not going to go completely pedantic and say that it requires the definition of the centimeter: regardless of the units used to measure the transition, it is consistent across time, so far as we know.  
i am also not going to really get into the fact that it is an abstraction at best: there is red and blue shifts which make the line change energy, and there is an inherent linewidth to the transition, regardless of how low pressure and temperature the hydrogen is.  
however, the transition does require the existence of hydrogen.

is hydrogen an independent truth?  
also no.  
models of the universe claim that in the beginning there was no hydrogen, and in the end there will also be no hydrogen.  
where there is hydrogen, i feel comfortable claiming that its hyperfine transition will occur\footnote{on average} at the energy expressed with photons of 21 centimeter light.  
i just do not feel comfortable claiming that the universe requires hydrogen.

i've gotten somewhat off track, but i think that the normal things i use for why i believe in the church are that i want a faith system with certain characteristics: unique claim to truth, universality, objective system of morality, and coherence with other forms of knowledge.

the unique claim to truth, to me, at least, is the idea that, given our universe with true things that we may only have approximate knowledge of, i want my faith to be the one which is closest to the truth.  
in a perfect version, there would be a special relationship with the creator.\footnote{hmm, do i need to believe in a creator? not sure yet}

oh shoot, no i usually start with the idea that G-d\footnote{who is definitionally in catholic circles a non-dependent existence} is love, and that love is good.  
goodness is inherent to love.  
what if we search around here?

looking at the notes i made, something that i was really thinking about it starting with full belief in the church and working backwards, rather than starting from what i want\footnote{i think want is the right word for now} to believe and seeing if i end up at the church.  
since, as the start of this draft says, i am a Catholic, that may be the right way to approach my faith.  
from here on, if it is something that i take as a given in the current moral framework, it's going to get a capitalization.\footnote{so be ready for some german-style every noun gets capitalized. i will not be doing this to every word, just the nouns, to be clear}

so, under the Church, what is the goal of humanity?  
our goal is to bring creation to Salvation; this is best achieved through spreading the Gospel of our Lord and Savior Jesus Christ.\footnote{i have no doubts there are people who see these as intrinsically linked. they are welcome to that, i am unwilling to say that the best way for every person ever and always will be the Bible and the Tradition of the Church, if only because we are in an imperfect world and so interpretation of either can be wrong}  
from the footnote, something that often bothers ex-catholics, potential future Catholics, and those Catholics who try to understand their Faith more deeply is the idea of Tradition.

Catholic scholars often differentiate Tradition from tradition. Tradition, they say, is that which we know to be true because the Church has passed it on as Divinely revealed.\footnote{hmmm revealed is an adjective here, so no capital i guess (also note that i never gets capitalized, because the Church, so far as i know, does not teach that i in particular am a given. what am i, and all those fun buddhist questions)}  
tradition, on the other hand, is merely that which we have carried down through the ages, and does not necessarily bind us morally.  
as one of the ex-catholic podcasts i have started listening to\footnote{hmmm that's probably a place that caused some of this. then again, Truth is Truth} points out, though, that is often used as a bait and switch or at least is completely subjectively defined.

as an example, take the question of what acts a Married couple is allowed to partake in.  
there are a number of different claims which range from \say{only that which results in Procreation} to \say{anything goes so long as at some point in the relationship there is an openness to Life}.  
which of these is correct, and how sinful is being wrong?

ope, ok sorry getting distracted again.

i think that this might be a good time to start a new draft, this time focusing on the whole \say{Salvation} aspect.

\section{Daily Reflection 6/7}

\begin{enumerate}

\item Did you journal by hand, and do you feel like the stormy questions in your mind got on the page?

Not for the last two days, yesterday because I forgot my pens at home/\footnote{is one supposed to do spaces on either side of a front slash? neither side? trailing?} knew what I wanted to do right away.  
Today I didn't journal because I got sucked into my phone and am now at a coffee shop doing writing.

\item Did you do your best to sit in still silence?

Not really. I guess that I spent a little time, but mostly last night I enjoyed my new bean bag chair and played a game.  
Oh! I did intentionally not bring headphones to an event yesterday, and then got somewhat lost, so ended up with a nice five or so minute walk without distractions.

\item Are you making sure that each task is given your full attention, not just because the task deserves it, but because you deserve the luxury of doing a single thing at a time?

In general! Last night after the event, though, I did listen to a podcast\footnote{technically turned on the video recording of a podcast, which has occasional pictures that I tabbed over to look at. It was a really fun podcast, about my favorite kind of monster (objectively good scientist who just also is completely amoral, in a way which may or may not be part of why good at science)} while playing the new game I've become obsessed with.\footnote{it's a vampire-survivors-like (I don't really know if this has a real genre name. I've seen bullet heaven, but that's not necessarily the vibe. It's certainly not bullet hell. The genre is like minimal meta progression (you get experience slightly faster or start with more health or whatever) roguelite (perma death, new build each run) autoshooter (weapons fire on their own and aim on their own) where increasingly large swarms of enemies attack.) This one is also a rhythm game, so the way you load your attacks is by pressing keys to the beat. It means that I have to watch the circle really carefully while also avoiding enemies. Great time, and very stimulating to the brain, plus it has nice music and graphics and the controls feel nice}

\item Are you focusing on your posture and breath?

Not a ton, but more than none! I'm hopeful about continuing to improve that. Right now I'm working in a comfy chair, where I am just slightly too large, an the laptop I write on is being aptly named, both of which aren't great for the posture.  
I guess I can make the active effort for breath, though.

\item What in your body is holding tension right now? How can you fix it? When will you fix it?

Absolutely my hips. I'm crossing my legs when I lie down an insane amount. I also think my lower back, but that's just because the fascia there is always tight when I look.\footnote{look here mean \say{do anything even resembling a stretch which could stretch the lower back} which somehow includes dropping my chin to my chest}  
I can fix it by dedicating the time that I have to stretching this afternoon.  
I will do so this afternoon, immediately after clearing space to stretch. Part of me wonders about shifting the bed again so that there is space on one side of it for stretching...

Anyways, all things to consider. I think that there is bound to be something I want to listen to, if only because I saw that two of the podcast YT channels I follow put out new podcasts.

\item Comments on sleep?

I feel like it's been going decently. Last night and the day before I didn't get enough of a nap in, which is a shame.  
Today I woke up on my own a full hour after when the alarm was supposed to have been going off, which is kind of strange.  
Anyways, sleeping in that extra hour felt great, and now I feel really well rested and ready to take on the day.  
I do still think that I will try to push waking up back yet another fifteen minutes this week, because if worst comes to worst I can just get into triphasic sleep.\footnote{if I schedule things semi-optimally, first nap could potentially have me waking at like 8AM, which would get me on a semi-normal schedule for at least part of it, plus there's no feeling quite like getting up early and doing something only to be able to go back to bed before the day starts}

Generally I feel like the naps and the sleeps have more of a soft drift off, rather than immediately falling asleep upon hitting the mattress.  
I am at the wrong level of wakefullness during the night, as evidenced by me sending a completely garbled text to a friend at around 1am today.

\item How's eating going? In particular, how are you doing with eating plants and unprocessed food?

Eating has been fine-ish. Yesterday I had oats for breakfast, a few chunks of mystery meat\footnote{thank you friend!!} for lunch, two tacos for pre-dinner\footnote{because I was going to a group event and don't like looking like I eat a lot in public anymore. The fact that there was an absurd amount of extra food at the end of the event makes me realize I didn't really need to do that, though. Also like this is absolutely not the group of people who would judge me for consuming a lot, and I need to be more comfortable accepting myself and being me in public settings} and a large slice of NY style pizza and a smallish slice of cheesecake for dessert. Then I went home and had some rice and beans.

This morning I had some more rice and beans and am currently consuming an espresso with whipped cream.  
I'm sure I'll eat lunch and dinner at some point.

\item Are you neglecting any of your familial obligations? If so, how can you rectify this?

Made my brothers reschedule our weekly call, which I don't feel great about.

\item Cleaning: what is the biggest priority you have right now, and what is the next action item for it?

Right now the biggest priority needs to be clearing out the yoga mat so that I can and will stretch again.  
After that, I want to get the kitchen to a space where I can make brewing possible, which probably means starting with the table.

\item Thesis: current task. What's preventing you from finishing it? How will you remove that obstacle?

Writing apparatus chapter, I think.  
The presentation went well which is nice, even if I felt like it was pretty horrible.

Apparatus block is just apathy, and so I will make time to work on it tomorrow before Mass?\footnote{feels like a fine time, assuming that I don't end up working on it today}

\item Thesis: next task. What will you need to be able to do it?

Data analysis: the many many\footnote{2700 total, though the remainder of the first 900 (about 780) are on hold, and the next 1800 are doubled with yet another typo being fixed} jobs need to run and I need to decide what the data analysis pipeline will be.

\item What's the next job you're applying to?\footnote{note that this might be a \say{things we don't post} but}

Did a skills assessment and personality test yesterday.

There is also a lecturer position at my current institution that's open right now, so I should really get some teaching philosophy statements going

\item Are you intentionally trying to spend time with others?

Yeah! I am writing with a friend right now, went to Shabbat last night, and am going to Havdalah tonight. In between, I'm also going to a bar with a friend.

\item Are you doing your absolute best to ensure that you and those you interact with view the interactions in the same light? Are you sure?

Yeah! Well, no, but I also think that the fact that I'm being open and honest and chatting with people using the words I want to use is probably good. I'm also not really interacting in places where easy large-scale miscommunications can happen.

\item Are you keeping up on this daily set of reflection questions?

Didn't do them yesterday. Today I do them again. Why didn't I do them yesterday? The same reason I didn't journal, more or less.

\item Are you keeping up on writing the follies? If not, what's in the way?

Missed Thursday and Friday. Thursday I wanted to be productive in the morning and then didn't make time for it at end of day.  
Yesterday just flew by, between group meeting, skills assessment, giving a presentation, and Shabbat.  
Today I'm going to do it, though.

\item How are the long form follies coming? Do you feel like they're weighing you down right now?

About to write the big one, because now is the appropriate time to have my potential crisis of faith.

\item Are you writing poetry? When, and what were your takeaways from the previous day's writing?

...SHOOT.

\item Are you making music? If not, what is in the way?

Not a ton. I've been listening to more which is nice, but something inside of me doesn't really want to pick up the guitar or sing right now.  
I did sing in choir on Thursday, sing along with the prayer as I was able to last night, and will be singing at Mass and a concert tomorrow, so I guess generally I am making music, even if not self-directedly.

\item Web novel?

I wrote a short short story and posted it yesterday.  
Generally being well-received!

\end{enumerate}

\section{Daily Reflection 6/5}

\begin{enumerate}

\item Did you journal by hand, and do you feel like the stormy questions in your mind got on the page?

I did, and I don't really feel like there's anything too stormy today, which is really nice. It's great to have a routine, even if the routine isn't particularly active.

\item Did you do your best to sit in still silence?

Kind of. I didn't make time for it, but I did like take deep breaths a few times. I probably could / should have done so this morning instead of scrolling instagram, but that's the nature of life.

\item Are you making sure that each task is given your full attention, not just because the task deserves it, but because you deserve the luxury of doing a single thing at a time?

Not entirely, but generally yeah! It's nice to not be always trying to optimize. It's a little frustrating trying to get work done at home without wireless, but I think that it might end up being for the best for me.

\item Are you focusing on your posture and breath?

I keep catching myself not doing it, and that's no fun at all.  
When I do, though, it's easy enough to fix again.

\item What in your body is holding tension right now? How can you fix it? When will you fix it?

Yesterday I realized that my neck was really tense, and that wasn't great. I think that I got some stretching/ massaging on it which is great. Now I think my jaw is still tense, and otherwise just my hips.  
To fix these, I can massage my jaw and do some more hip stretches.

In an ideal world, this would also mean that I am stretching this evening at least once, but we'll see if that happens.  
One downside of the rearranging I am doing at home right now is that my stretching area(s) are all filled. I think that I have solutions to most of the issues, but I'm not totally sure if they will work. There's only one way to find out, and so that's another goal for the day.

\item Comments on sleep?

I think that I basically slept through the night last night.  
I was out until 11, which meant that I didn't take the time for an end of day routine.  
I think that the end of night routine is really important to me getting sleep, so I might have to force myself to do it even when I am tired.  
The nap felt great, and my new blanket was great.

\item How's eating going? In particular, how are you doing with eating plants and unprocessed food?

I did not end up being hungry for lunch yesterday, and so I just had a single bite of the sandwich. Other than that, I had burgers for dinner.  
That's not fantastic, but I will be eating some cherry coffee cake this morning and other than that probably the sandwich for lunch and in a perfect world I would also eat diner of some sort. I think that is absolutely doable though.

\item Are you neglecting any of your familial obligations? If so, how can you rectify this?

Nope! I listened to the album yesterday and even took notes, which is really cool.

\item Cleaning: what is the biggest priority you have right now, and what is the next action item for it?

Much as I wish that it was something else, the biggest priority right now is absolutely getting space for the stretching

\item Thesis: current task. What's preventing you from finishing it? How will you remove that obstacle?

I realized yesterday while doing data analysis that I had a block of code that was not indented when it should have been, and since I write in Python, where whitespace is important, that means that I had a whole section of conditionals that didn't get used. Whoops.

Fixed and resubmitted that.\footnote{even though I have no priority at all, only requesting one core at a time means that they all get run right away, which is good to know just always}

I need to finish the presentation that I'm giving tomorrow, do the literature review for group meeting, and write the next chapter I have due.  
That's all that I have to do.

\item Thesis: next task. What will you need to be able to do it?

Next task is writing the chapter for the week, which is really a lot more than I want it to be.  
That's all that I have to do for now, I think, but probably not.

\item What's the next job you're applying to?\footnote{note that this might be a \say{things we don't post} but}

I'm taking some skill assessments for a job application tonight(?). Maybe this afternoon. Presumably sometime after the rest of the group has gone away because I need to be alone for three hours. I guess that I could do that in the library in my cage, and honestly, that might be the move.

\item Are you intentionally trying to spend time with others?

Yeah! I forgot that I had signed up to go to a Shabbat this week, so that's going to be really fun.  
I went to board games last night, and I wrote with a friend today.

\item Are you doing your absolute best to ensure that you and those you interact with view the interactions in the same light? Are you sure?

Generally! I don't think that I had any places for miscommunication since the last time that I thought about it.

\item Are you keeping up on this daily set of reflection questions?

Yeah!

\item Are you keeping up on writing the follies? If not, what's in the way?

Forgot to post yesterday's but that's really it. Will give it that quick read through and then post.

\item How are the long form follies coming? Do you feel like they're weighing you down right now?

Have not touched. I still feel like they're more existentially pulling me down.

\item Are you writing poetry? When, and what were your takeaways from the previous day's writing?

I have not been, and it might be part of why I'm not sleeping as well.

\item Are you making music? If not, what is in the way?

I moved my guitar to the other side of my bed, and I've played a little bit on it last night and this morning.

\item Web novel?

That will hopefully be happening today!

\end{enumerate}

\end{document}