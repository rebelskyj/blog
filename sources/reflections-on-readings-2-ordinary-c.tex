\documentclass[12pt]{article}[titlepage]
\newcommand{\say}[1]{``#1''}
\newcommand{\nsay}[1]{`#1'}
\usepackage{endnotes}
\newcommand{\1}{\={a}}
\newcommand{\2}{\={e}}
\newcommand{\3}{\={\i}}
\newcommand{\4}{\=o}
\newcommand{\5}{\=u}
\newcommand{\6}{\={A}}
\newcommand{\B}{\backslash{}}
\renewcommand{\,}{\textsuperscript{,}}
\usepackage{setspace}
\usepackage{tipa}
\usepackage{hyperref}
\begin{document}
\doublespacing
\section{\href{reflections-on-readings-2-ordinary-c.html}{Reflections on Today's Gospel}}
First Published: 2019 January 21

John 2:3-5: \say{When the wine ran short, the mother of Jesus said to him, \nsay{They have no wine.} Jesus said to her, \nsay{Woman, how does your concern affect me? My hour has not yet come.}
His mother said to the servers, \nsay{Do whatever he tells you.}}

\section{Draft 1}
Today's readings are a set that my family has always enjoyed.
Mostly, it's due to the non-sequitur in the Gospel.
Mary says that there's no wine, Jesus says it isn't his problem, and then solves it anyways.

It also leads nicely into the joke I've always enjoyed telling friends when they ask why Mary is such a big deal in the Catholic Church.\footnote{other than the whole mother goddess being adopted from prior pagan traditions}
Jesus is a good Jewish boy.
Good Jewish boys listen to their mother.
Mary loves us and listens to all of our prayers.
So, if you pray to Mary, Jesus has to listen.

Of course, I have no idea how accurate this joke is theologically, but I hope.
\end{document}