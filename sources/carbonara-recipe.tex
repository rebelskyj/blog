\documentclass[12pt]{article}[titlepage]
\newcommand{\say}[1]{``#1''}
\newcommand{\nsay}[1]{`#1'}
\usepackage{endnotes}
\newcommand{\1}{\={a}}
\newcommand{\2}{\={e}}
\newcommand{\3}{\={\i}}
\newcommand{\4}{\=o}
\newcommand{\5}{\=u}
\newcommand{\6}{\={A}}
\newcommand{\B}{\backslash{}}
\renewcommand{\,}{\textsuperscript{,}}
\usepackage{setspace}
\usepackage{tipa}
\usepackage{hyperref}
\begin{document}
\doublespacing
\section{\href{carbonara-recipe.html}{Something I'm Calling Carbonara Recipe}}

\section{Draft 1}
I've mused a few times in the past few days that, although I write a post a day and averaged out, come up with about a post a day, I almost never have a single idea in a day.
Unfortunately, on days that I do not have any ideas, I also tend to forget the ideas that I've come up with on previous days

I had been considering writing about music, specifically writing music, but couldn't find a post with that exact title.
Searching through early posts on my blog, I remembered that one series I had was recipes I wanted to be able to remember, reference, and rely on in the future.\footnote{I love rules of three, and couldn't think of a better third r word right now}
Although most of my food exploration has been slowed in the past few months,\footnote{not because I do not cook (though lately that's been more of that), but because I've started cooking more wildly and less regimented.
Even though I still go through trends where I want to cook the same dish over and over, I find that what I cook tends to be a variation on \say{put everything that I feel about in the pot and maybe water if I want it to be soupier}, rather than the regimented recipes of my past}
I have at least one recipe that was a mainstay for me for a few weeks.

There are some important contexts to the story of the dish I'm going to describe.
First, although I'm going to call it carbonara\footnote{as, I realize, you the reader saw when opening the page, despite the fact that I write the title last when doing any of my musings}, I've seen a lot of people on the internet get very mad at calling anything that isn't exactly what their grandmother\footnote{allegedly} made carbonara.
This is not, in any meaningful sense, a traditional carbonara recipe.
However, of the pasta sauces I know names for, carbonara is the closest to describing what I cook.

The recipe requires the following ingredients:
\begin{itemize}
\item Olive Oil
\item Pasta, I recommend spaghetti that you know releases a lot of starch\footnote{I watched an interesting video where they seemed to suggest that more expensive pasta, which is often described as better, is mostly just better because it releases more starch into the cooking water.
Regardless of how true that is, I do find that starchier water makes this dish go better.
At worst, just use less water than you're used to, and there will be a higher starch to water ratio from that (agitating the pasta while cooking also helps release starch)}
\item Lemon
\item Salt
\item Egg
\item Black Pepper\footnote{I don't remember when I stopped using preground black pepper, but I think it was just about the exact moment that I started buying groceries for myself.
To the best of my knowledge, it isn't markedly more expensive, and it's certainly better tasting.
If you don't fresh grind your own pepper, please consider this your call to}
\item Parmesan cheese or whatever local variation you prefer.\footnote{more or less as long as it's hard and grates into powder you're probably fine.}
\item Optional: mushroom
\end{itemize}

The astute among you may notice that there are no specific ingredient amounts here.
That is for a very basic reason: I only believe in measurements by the heart these days.

That being said, there are some constants in most iterations of the recipe.
I almost always use a single, whole lemon.
I tend to use two or three whole large eggs.\footnote{well, whole when I begin the process.
I don't use the shell, for I hope obvious reasons}

Directions for cooking:
\begin{enumerate}
\item Put water and what seems like an appropriate amount of salt into the pot you plan to cook in.\footnote{note: as you make this recipe more times, you will likely find that the optimal amount of water is less and the optimal amount of salt is more.
The less water you add, the faster you get food, because the less time it takes to heat the water to boiling}
\item While the water starts heating, measure out the amount of pasta you want.\footnote{measure with your stomach.
I tend to go for about half a pound when making a full meal out of it, but you can go for what feels appropriate to you}
\item If using mushrooms, chop them to desired thickness at this point\footnote{I generally recommend thin but sliced, not diced, because I like large surface area mushroom chunks}
\item Once water is boiling, put pasta in water.
\item At this point, we are now on a timer, which is great and nice.
Grate the zest of one lemon into a bowl, along with two or three eggs,\footnote{depending on how hungry you are and how much sauce you want.
These are generally correlated but not always.
Sometimes there's also a voice in my head which suggests more protein would be healthy for me, and that tends to be a three egg and extra cheese kind of day}
and as much cheese as it takes to make the mixture into something which resembles a thick paste.
\item It's a well known fact that the two biggest issues that carbonara can face are a broken sauce and a sauce where the eggs have scrambled.
To prevent both of these, I recommend vigorously stirring the beaten egg mixture as you stream in what seems like slightly too much of the pasta water, before draining the rest of the pasta.
\item If using mushrooms, put in about twice as much oil as you feel like you should use and cook the mushrooms until at your desired level of doneness.\footnote{I tend to like mine browned on the outside and raw on the inside, but I recognize that I have less than common taste}
\item At this point, turn off heat on the stove and add the pasta to the oil and mushroom.
\item While stirring vigorously, slowly pour in the sauce.
If you have done all the steps correctly, it will thicken as the water evaporates and the egg proteins set, but will not grow clumpy.
At this point, squeeze either half or all of the lemon into the pot, again depending on how much pasta there is and how acidic you want it to be.
\item Serve immediately and eat before the sauce congeals.
\end{enumerate}
Now, there's a very valid argument to be made that this is not carbonara in any real sense.
There's no pork of any kind, at a bare minimum.
Arguably, calling the sauce you make here mayonnaise is not too far off base, since it's primarily an emulsion of oil and egg.

Taste wise, however, I have no complaints.
The lemon zest and juice make what would otherwise be an incredibly heavy meal into something almost refreshing.

Oh, I realize I forgot to discuss the reasons that I started making this dish.
Chief among them is the fact that I started keeping lemons in my home.
The curious among my readers might ask why I started keeping lemons in my home.\footnote{depending on the reader, either questioning why started, or why I haven't always kept lemons}
The long and short of it is that I had a friend who seemed appalled that I used lemon juice for all of my cooking needs.
He argued that fresh lemons are just objectively better in every regard.

Truthfully, I cannot say that I disagree with that take, having now started to use them.
Other than that, the main consideration was that I really just wanted pasta al burro\footnote{the fancy (read: Italian) way to say buttered noodles} every night for dinner, but knew that wasn't healthy for me.
This still scratched the itch, and had the benefit of a number of other macro and micro nutrients.
However, I stopped eating as many lemons, and also wanted to eat more different foods, so I haven't made this dish as much recently.
Maybe I'll start doing so again.
We'll see, I suppose.

Daily Reflection:
\begin{itemize}
\item Did I write 1700 words for NaNoWriMo? I did. It was a slog and a half, but I did it.
\item Did I write a chapter of Jeb? I did not, but I'm too tired from teaching my first university level class and driving home.
Tomorrow I have less on my to do list, and so I'll hopefully have more mental space to do the writing.
\item Did I blog? Woo! Look at this blog post.
\item Did I stretch? Still no.
\item Am I doing better at prayer than a rushed and thoughtless rosary? My rosary last night wasn't too rushed, but that's about all the prayer I did.
\item Am I doing a good job writing letters to friends? I brought the letters home with me, so I can at least hopefully make some time for letter writing. We'll see if I do.
\end{itemize}
\end{document}