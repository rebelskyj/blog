\documentclass[12pt]{article}[titlepage]
\newcommand{\say}[1]{``#1''}
\newcommand{\nsay}[1]{`#1'}
\usepackage{endnotes}
\newcommand{\B}{\backslash{}}
\renewcommand{\,}{\textsuperscript{,}}
\usepackage{setspace}
\usepackage{tipa}
\usepackage{hyperref}
\begin{document}
\doublespacing
\section{\href{why-versus-how.html}{On Why Versus How}}
First Published: 2023 December 20

\section{Draft 5}
Why and how feel like such close questions.
And yet, there is an unseen gulf of meaning between the two.
Far too often, we stop the mind from looking upward by focusing it downward.

Why asks the reason, the cause, the philosophical reason that something is.
How, by contrast, asks the mechanism, the physical description of what is.
That is, why is to metaphysics what how is to physics.\footnote{philosophers and physicists please don't hate mail me}

Children are always gazing upward, considering the nature of the world.\footnote{in my imagination, at least}
Children ask why the world is the way it is, not how it is.
When we tell them that the world is the way it is because of history, that reduces all of creation to a series of forced events.
We lose agency in this world.

Why is the sky blue?
Rayleigh scattering is true, but it is how the sky is blue.
Rayleigh scattering explains that some light is diffracted differently.\footnote{I think}
If physics is all that makes the sky blue, then there must not be anything more than physics, is what our minds are lead to believe.
By answering questions of why with how language, we fall into the trap of scientism.

More than that, though, we fall into an issue endemic to computer science.
Computer scientists are famous for creating solutions to problems without considering the secondary effects, or the reasons the problems may exist.
The most famous example that I can think of is a Stanford student who found that when you challenge a parking ticket, most jurisdictions will just drop the charge, because it isn't worth the hassle to fight it.
So, the student wrote an app to automatically contest all parking tickets.

Deep down, the question being asked was how do parking tickets work, not why do they exist.
I think that we can all agree it is generally better to not park in disabled spaces\footnote{assuming that we are not disabled} or in front of fire hydrants.
This why and how can also lead to an intellectual stagnation.

Let's take the example of the research that my group does.
We look for the origins of life in star forming regions.
From now on, each statement will have an implicit how question attached to it.

We use telescopes such as ALMA and NOEMA to measure the radio frequencies emitted from star forming regions and tie them to specific molecular transitions.

We have catalogs of molecular line information, which we can assign to the spectra coming from space.

We can purchase molecules and collect their rotational spectra.

We can shine known wavelengths of light on known molecules and detect whether or not the molecule absorbs at that wavelength.

We can see a change in voltage, which is directly correlated to the change in light.

Notice how even here, we've fallen almost completely away from the actual research goal of searching for the origins of life in star forming regions.
All of the how questions are absolutely essential, because without answers, we would not be able to measure transitions or assign them.
However, imagine if instead we start asking why.

I'm going to work backwards.

We can see a change in voltage, which is directly correlated to the change in light. (Why?)
The software we use is designed to function with an instrument that has a variable resistor. It measures changes in photons as changes in temperature as changes in resistance. V=IR, so that can make a change in I or V if you hold the other constant. (Why?)
It's a detector very similar to the ones that astronomers use? (Why?)
Calibrating based on temperature is easier?\footnote{I assume, and this is where my knowledge falls apart}

We can shine known wavelengths of light on known molecules and detect whether or not the molecule absorbs at that wavelength. (Why?)
We want to know what frequencies the molecule absorbs light at. (Why?)
When we know what frequencies the molecule absorbs light at, we can assign its spectra using a limited number of rotational constants. (Why?)
Rotational transitions are low energy, which means that a lot of the cold regions of space still have active populations in multiple rotational states, which allows us to quantify temperature and density. (Why?)
Knowing what space is like helps us to model it, and can help us learn where we come from. (Hey look we got to the question of origins)

We can purchase molecules and collect their rotational spectra.
I feel like this one follows the same train of thought.

We have catalogs of molecular line information, which we can assign to the spectra coming from space.
This one follows the same starting from \say{rotational transitions are low energy}.

So, one thing that I notice is that how questions lead me deeper into the subject, reducing the scope of the question, while why questions bring me higher.
There is absolutely a space for both, and I think that both are needed.
The ideal is to have feel firmly anchored on the ground and a head floating in the clouds, after all.\footnote{I hope that's a common saying and not one that I stole from I think Discworld? I know that it's somewhere in the Tiffany Aching series, but I cannot quite remember where}

It is important to look deeper at the question.
Physics as a field is built on the concept of reductionism.
For all that it has limits, as every other method of discovery does, it is an incredibly powerful tool when applied correctly.

However, the same is true for why.
Without motivating the work we do, there is no way to connect it to others.
Since all that we do is for the greater glory of G-d, and we are called to bring all creation to Him, we need everything we do to be in connection to the rest of the world, especially the rest of humanity.\footnote{there we go, it took all day (literally, I've spent more than 12 hours working on this musing off and on), but I finally got a why for the post}

Daily Reflection:
\begin{itemize}
\item Hobbies:
\begin{itemize}
\item Did I embroider today? No! I worked with my family, though.
\item Did I play guitar today? I played a little this morning! Things sound different at home.
\item Did I practice touch typing today? I'll get back to it tomorrow. I found that my hands are once again slowing me down.
\end{itemize}
\item Reading
\begin{itemize}
\item Have I made progress on my Currently Reading Shelf? I got a lot further through the second book with a friend.
\item Did I read the book on craft? Shoot!
\item Have I read the library books? Shoot!
\end{itemize}
\item Writing
\begin{itemize}
\item Did I write a sonnet? I don't remember if I wrote one yesterday. Looks like I didn't, so that'll happen after this.
\item Did I revise a sonnet? Will probably need to revise, given how tired I am.
\item Did I blog? Oh gosh, did I.
\item Did I write ahead on Jeb? I finished the chapter due today. Tomorrow I hope to write the last chapter of the year.\footnote{wow that's wild to think about}
\item Letter to friends? I started texting a friend with daily updates, and I reached out to a couple of friends.
\item Paper? I talked with my family about it.
\end{itemize}
\item Wellness
\begin{itemize}
\item How well did I pray? Oof.
\item Did I clean my space? This goal gets to be deleted because I'm at home, where extrinsic motivation exists.
\item Did I spend my time well? I spent it with family!
\item Did I stretch? Nope.
\item Did I exercise? Nope, but I do regret that.
\item Water? I always struggle to drink water when I'm home.
\end{itemize}
\end{itemize}

\section{Draft 4}
I often find myself considering language.
There are a lot of reasons for it, as I come to realize more and more with each passing year.
One of the more common issues that I find myself considering is the shade of difference between similar words.
Two of these words are \say{why} and \say{how}.\footnote{as you might expect from the title of the musing today}

So, what is the difference between these two words?

Why is focused on the reason behind an act, and how focuses on the process behind it.
This distinction, for all that it took me far too long to find, rather than simply looking it up, is a somewhat subtle one.
Often, we treat the words as interchangeable.

For example, think of that most basic scientific question: \say{why is the sky blue?}

Most often, we answer this with mechanistic answers.
The sky is blue, we explain, because of rayleigh scattering\footnote{true, kind of}, or because air is blue\footnote{also kind of true}, or any of a number of other answers.
That is, we tell children how the sky is blue.

Why is that a problem?

I think that it's a subtle thing.
By answering questions where someone looks for a reason with a mechanism, we start to point to mechanisms as reasons themselves.
That is, the world we create when why questions are answered with how lacks transcendence.\footnote{hmmm this might be the day for the rant, for all that I feel like it's probably not the musing for it}
When we see the world as nothing but a machine, moving thoughtlessly along, we feel as though actions cease to matter.
After all, if the why is the how, then free will makes no sense.

We can look and see the issue from the opposite perspective.
If we answer a how question with a why answer, we fall down to fundamentalism.

How do we celebrate Hanukkah?\footnote{I'm not sure whether I went to the most recent Jewish holiday because my family just watched An American Pickle, because the holiday just passed, because my Jewish identity is something I'm forced to reckon with almost daily, or almost any other reason}

We celebrate Hanukkah because Hashem preserved the Jewish people before, and He will continue to do so in the future.

I don't know if that quite works.
It feels almost like a non sequitor, and I realize now that there are ways to frame the question that start from a science question, rather than a religious point.

How does a computer program work?
We write computer programs to make life easier.
We can, hmm no.

By adding the work at the end of the sentence, you cannot really substitute a why.

How do we use computers?

Computers make life easier because they can do a lot of math really quickly.
As it turns out, problems which are easy for people to solve tend to involve concepts, while questions that are easy for computers to solve tend to involve caluclations.

That doesn't answer the question.
I think that this is showing me my own bias.
I have a far easier time accepting that a process explains a reason than that a reason explains a process.
Then again, I find myself incredibly uncomfortable with fundamentalism as a concept.
What do I mean by fundamentalism?

That's a fantastic question.
Fundamentalism is answering how questions with a why.
When finding the way that something works, we can always delve deeper.

How does rotational spectroscopy work?
Molecules rotate.

How do molecules form?
Atoms bond.

How do they bond/how are they formed?
You see, we can go on and on.

If we answer \say{how do molecules form} with \say{because it is favorable for them to do so}\footnote{even though this is itself somewhat of a mechanistic answer}, we shift the question from mechanisms to reasons.
Actually, as I think about it, the framing of an answer is often the difference between why and how.
Why do molecules bond? It's energetically favorable.
How do they bond? Two atoms draw close enough that they release energy by drawing closer.

I don't think that's the point I'm trying to make, though.
Let's try one last time, focusing entirely on the whole \say{using a how to answer a why kills belief in greater workings and makes the world as though automata.}

\section{Draft 3}
One question that children stereotypically ask is \say{why?}.\footnote{I don't know if the . should be there, but I'm going to leave it, because it's a statement}
One question that children ask far less often\footnote{if stereotypes are to believed} is how.
To be honest, I think that the lack of distinction in children's questions is the start of every issue that we have in communication.\footnote{ok that's a little strong, but I've been working on this post for a lot of words and still have no clue what I'm trying to say, so I'm going to go with it}

Think about explaining grass as green, either because it reflects green light, or because it has chlorophyll.
Each of those answers would satisfy the question of either why or how grass is green.
And yet, to me, at least, these two sets of responses feel different.
If they don't to you, this musing is only downhill from here.
Be forewarned.

So, what is the difference between why and how?

In basic terms, I generally feel like how is looking within, while why looks without.
Or, how asks the mechanism by which something operates.
Why assumes that there's a broader meaning to it.

Is that all, though?

If so, it does lend credence to the idea that why questions are questions of faith, while\footnote{while is a word that I'm going to be using far too much for the rest of this musing, sorry} how questions are questions of science.
This definition, of course, relies on science being defined as a mechanistic search for the way that the universe functions, while faith is concerned with understanding the reasons which make the universe.\footnote{not using why or how except very intentionally is so very hard for me. I'm being very brave.}

Something that I keep needing to remind myself is that there needs to be a takeaway from whatever I write.
Given that I don't know what the point of this musing.
I guess we can ask the two questions of this musing?

Why am I writing this musing?

I am writing this musing because I have a vague sense of dis-ease\footnote{ill ease? disease? huh is that where disease comes from? That makes sense} about the difference between the words, their meaning, and their use.
The two words need to be different, if only because they are different words.

Why else?

I don't think that there's a deeper meaning to this musing, at least yet.
So, let's assume that the why has been covered.

How am I writing this musing?

I'm writing this musing by making different drafts of the musing.
I'm writing this musing by trying to think about what I think the differences are.
I'm writing this musing by figuring out what I think the words mean in connection to other words.

Any other ways that I'm writing this musing?
I honestly think that may be it.
That's how I'm writing it.

So I guess the way that I responded to the questions confirms my thoughts about how they differ.
Why does seem to point to meaning, and how seems to point to methods.
Great. 

Let's move past that though.
That last sentence in the how feels striking to me.
Words have meaning almost exclusively in how they relate to other words.

When we answer a why question with a how answer, or a how question with a why answer, we do a disservice to others.
Or, instead, let's take the other view.
Most of the people that I know do not care so much about the precise usage of language.
Instead of saying that the rest of the world is wrong, let us see if I cannot find a way that I can shift my own views of language.

After all, as Randall Munroe once pointed out, language and communication requires two people.
What is different about the language that others use where this does not feel like an issue?
Only now, twenty five hundred words later, do I feel like it is a good time to look up and see if there's anything that others would think

Oh gosh \href{https://english.stackexchange.com/questions/5563/how-come-vs-why}{someone} pointed out that why and how come have the same usage, though how come is less formal.
If we put it into my framework, how come is also a mechanistic explanation?
I think?
Honestly, I don't know how I feel about it.
Oh good, the rest of the internet generally agrees with me that the difference is between process and reason.

I feel really dissatisfied with this end of musing, however.
Why?
Because I kind of feel like I've gotten nowhere after all these words.
How?
I have failed to answer the question, which is difference in usage.

I probably have time for one more draft.
Let's do everything in our power to focus exclusively on why I think it's a problem that we answer process questions with reason answers or vice versa.
I'm sure that there's a reason if I stop to think about it.
\section{Draft 2.5: metawriting to figure out how I feel}
Why do I muse?

I have talked about this a number of times.
In short, I muse because I find it helpful to my life and growth as a person.

How do I muse?

I don't think I've ever explicitly thought about this.
This question can be answered in so many ways, with varying levels of granularity.

I muse by posting my thoughts on the blog.
I muse by coming up with an idea, writing about it, revising it\footnote{optionally}, and then posting it.
I muse by typing words on a digital page.
I muse by having thoughts and expressing them.

Why do I hurt?\footnote{hypothetical. Right now I'm very comfortable}

I made a mistake while working out.
I fell down.
I don't stretch enough.

How do I hurt?

My leg aches, or my arm feels like daggers are being driven into it.

Does that help me?

Why is causal?
Like why presumes a reason, while how just explains mechanisms?

Why is grass green?

Because it doesn't absorb green light.
That's mechanistic.

Because it's filled with chlorophyll, which is green.
That's also still a mechanism.

How is grass green?

Because it doesn't absorb green light.
Because it's filled with chlorophyll, which is green.

Somehow, despite the fact that both of those answers are the same, it still feels vastly different to me.
I don't actually know if there's anything more that I can say about that.
Let's try, though.
\section{Draft 2}
I've been thinking recently about the difference between why and how.
The two words have to be different\footnote{because otherwise we wouldn't have both}, but I've been struggling to figure out in what way.\footnote{why and how both came up in my mind when I tried to figure out how to end the sentence (though not there, which is a clue)}

How feels like something descriptive.
That is, I figure out how to end a sentence, because I already know that I'm going to end a sentence.
Why, on the other hand, feels prescriptive.
If I am figuring out why to end a sentence, it is because I don't know what its purpose is.

Maybe that's the way to figure out the difference between how and why: write a few sentences that could use either word and see the way that the meaning changes.\footnote{I just wrote that sentence with a how unthinkingly, and that might be a part of the issue. I'm too thoughtless about the way I use my language.}

I'm going to restart this musing, and I will try my best to use either word only intentionally.
\section{Draft 1}
While writing my reflection on the readings, I mentioned faith like a child.
As I drove to religious ed, I thought about faith like a child, and I thought about faith and science.
One half formed memory I have\footnote{which I think I first encountered in a children's book series about someone who's a clone of a dead girl raised by the initial girl's parents (have asked my resident librarian what book that could be -Update: Double Identity by Margaret Peterson Haddix (also wow she wrote a lot of books, many of which were formative to me. As the resident librarian put it \say{She has written so many objectively twisted but deeply compelling plots})). Wild how my memory links. Also, now that I'm down the rabbit hole, wow early 2000s child lit was filled with a lot of wild dystopian literature. Then again, I also read Octavia Butler's Wild Seed, which though not children's lit (I think) or from the early 2000s, was still messed up. Ok back to the real musing (but new idea just dropped). Also I need to reread and then write about the Patternist series, because there's so much in there that's immediately relevant to modern political discourse.} was that the difference between faith and science is which questions they seek to answer.
Science seeks to answer how, while religion seeks to answer why.

Of course, I then immediately thought about the question that a young child asks.
Children do not\footnote{stereotypically, at least} ask how.
They ask why.
When a child asks why the sky is blue, however, we tend to give how answers.

That bothered me, for a reason that I'm going to try to work through here in this musing.

Ok, so the most obvious bother is that if the difference between why and how is science and faith, and the two live in different domains, then it's wrong to answer a why with a how.
Now that I've expressed the concern, I can set it aside.\footnote{because clearly something in that statement is wrong, so I should figure out what}

I'll trust that the sentence above describes my concern.
At least one of those claims is untrue, so let's list them.\footnote{I'm treating this first draft as something slightly more formalized than free association, but only slightly}

\begin{itemize}
\item Explicit: there is a difference between science and faith
\item Explicit: science answers how questions
\item Explicit: faith answers why questions
\item Explicit: the two have different domains
\item Explicit: you should answer how with how and why with why
\item Implicit: you should answer the question that people asked, not what you want to answer
\end{itemize}

Are there other claims? I don't really think so.\footnote{one thing I'm realizing is that the raw tex file that I write contains its own information that gets removed when I convert to HTML. Most of the time, I'm on team line break after each sentence (since a professor taught me how much easier it made editing and seeing edits), which LaTeX ignores (I don't know if the many conversions I use will break if I try to have special formatting for the word, so I'm going to pretend that the normal orthography (which may or may not be the proper term, depending on how important the positioning of letters is to the meaning of the word) is acceptable). However, every so often I find that I have two sentences that feel like a single thought, and so get to use a single line.
Also most footnotes tend to be a single line, which makes me think that I really do use line breaks to signal thought shifts. Wild}
There may be, however, so I'll be willing to add to the list if I need to.

Alright, let's try to answer the claims one at a time, and see if we can't find the one that isn't ringing in tune.

There is a difference between science and faith.
No, that's absolutely true.
For all that I've talked about how science and mysticism are intrinsically linked, the method of discovering knowledge is inherently different.
Science is different than math is different than faith.

By that, I mean that science gains knowledge\footnote{as a discipline, nominally} by measuring and reproducing measurements.
Mathematics generates knowledge by testing hypotheses using logic.
Faith generates knowledge by divine revelation.

Oh! A claim in there is that why and how are fundamentally different questions.
There we go, that's the sticking point that I have right now.

So let's see what the difference is between the two.
How describes process, while why ascribes meaning?
That feels kind of resonant.

That is, how is a descriptive thing, while why is a prescriptive.
Ok, that does also work with the difference between faith and science, so let's run with that take.
Now then, what's wrong with answering why questions with how or vice versa?

Ok, let's remove the explicit case of the children's question.
Let's look at the election.

Why do we have a president?

We have a president because we set up a system wherein we elect someone to run the country.
That is a why answer to a why question.
Oh, yeah, no, that's definitely the issue.

When a kid asks why the sky is blue, we treat it as though they're asking how the sky is blue.
The sky is blue because air is blue.\footnote{among other issues.
Generally, though, it's something about rayleigh scattering, which may or may not be something I've vaguely forgotten about.}
But, why is the sky blue? That's a much more difficult question to answer.
Honestly, there isn't an answer that I know of for why the sky is blue, other than the fact that the universe was created in such a way to cause the sky to be blue.

What are other questions that children stereotypically ask?

When a child asks why we have to go to school, on the other hand, we do give them a prescriptive answer.
We have to go to school because it's good to learn, or something.\footnote{the law tells us, or whatnot. There are plenty of other reasons we could give for why kids have to go to school}

Ok, so I guess my issue is really just with the answer to why the sky is blue.
That is, however the only children's question about science that I can think of.

Oh, wait.

Why is grass green?
That sounds like something that a child would ask.

We can answer that descriptively \footnote{because it doesn't absorb green light}.
Can we answer that prescriptively?

If not, does that show me the issue?
Is there a way to frame scientific questions that are why based, rather than how based?

Ooh, that may very well be the issue that I have right now.

Is how is grass green a different question than why is grass green?

On the face of it, yes?

\begin{itemize}
\item How:
\begin{itemize}
\item Grass is green because of chlorophyll
\item Grass is green because it preferentially reflects or emits green light\footnote{or scatters, I suppose}
\item Yeah that's about it.
\end{itemize}
\item Why?
\begin{itemize}
\item Because that's the best wavelength to absorb light for energy\footnote{though that's why chlorophyll is green, I suppose}
\item Because it was made to be green.
\end{itemize}
\end{itemize}

I guess those are shades of different meaning, for all that I cannot really point to anything specific about it.
However, what is the difference between a child's response to each of those?

If we assume that a child is an automaton, the child will ask why in response to everything else.

So:
Grass is green because of chlorophyll. Why?

Because grass gets its color from chlorophyll.
Why?

Because there's a lot of chlorophyll in grass.
Why?

Ok I can't find a good way to continue from there.

Grass is green because it emits green light.
Why?

Because chlorophyll emits green light, and that's what looks green.
Why?

Because that's the best wavelength to gain energy.
Why?

Because the sun emits light across a whole wavelength, maximizing at the green.
Why?

Because something something quantum mechanics.

Ok that's also not great.
That also does cover the actual why answer, which is slowly leading me to the actual issue I have.
Despite the fact that I have different senses of the meaning, my impulse is to answer the why questions more or less interchangeably.

What can I do with this? 

I think I might have fallen too far into the weeds.
Let's pull back and re evaluate.
\end{document}