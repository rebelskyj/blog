\documentclass[12pt]{article}[titlepage]
\newcommand{\say}[1]{``#1''}
\newcommand{\nsay}[1]{`#1'}
\usepackage{endnotes}
\newcommand{\1}{\={a}}
\newcommand{\2}{\={e}}
\newcommand{\3}{\={\i}}
\newcommand{\4}{\=o}
\newcommand{\5}{\=u}
\newcommand{\6}{\={A}}
\newcommand{\B}{\backslash{}}
\renewcommand{\,}{\textsuperscript{,}}
\usepackage{setspace}
\usepackage{tipa}
\usepackage{hyperref}
\begin{document}
\doublespacing
\section{\href{embroidery.html}{On Embroidery}}
First Published: 2023 November 4
\section{Draft 1}
I think I mentioned this in my \href{reflection-october-23.html}{last monthly reflection}, but last month I went to an embroidery showcase!\footnote{as it turns out, I did not. Interesting. Ah, I put it in my daily reflection, so unsure why it didn't count for the month. Anyways.}
It was a really cool experience, in part because i hadn't realized how diverse the field of embroidery was.
I had in my mind an image of just like the way you can add a small pop of color to a piece of clothing as the whole of embroidery.\footnote{it sounds ridiculous to say so now, but I didn't have anything else to connect the word to then}
I was\footnote{obviously, in retrospect} wrong about that.

The first exhibit\footnote{I feel like that's the wrong word. Table? unsure} was a display of temari balls.
Temari, as the display informed us, is a Japenese form of embroidery where you make intricate\footnote{and, relevant for the crowd of physical chemists I went with, very mathematical} designs in a ball filled with rice hulls.\footnote{traditionally, at least. As I looked up the art elsewhere I saw people using more or less anything you would use as a replacement for filling.}
The designs were beautiful, and they had pieces at every level of finish, beginning with raw rice hulls and ending with a beautiful almost fractal pattern.\footnote{the entire room was filled with signs asking for no photography, which we respected.}

That, of course, was only the first table we saw.
There was a lot of free form embroidery, which I had expected.
There was also a lot of cross stitch, which I had not realized was considered a part of embroidery.
Again, in retrospect, it makes sense that cross stitch would be featured in the Embroidery Guild, if only because the materials and skills cross over so well between the two hobbies.

There was another kind of embroidery that instantly caught my attention, however: counted thread embroidery.
Similar to cross stitch, it is worked in a regular fabric\footnote{wikipedia informs me this is called even weave fabric, and happens when the warp and weft are the same size. I had never considered that a fabric might not be like that always, but I suppose it makes sense.}
Unsurprisingly, as a child of the digital age,\footnote{we'll ignore the fact that not all of the people I've known feel similarly about grid based designs and imagery} the fact that the designs were worked onto grids was fascinating to me.
By and large, the designs worked in counted thread embroidery did not rely on different stitches to add texture and design elements, instead relying on color, material, and to a small extent, thickness of worked thread to make their designs.

Anyways, as the four of us walked through the exhibit, we were stopped multiple times by older women who encouraged us to join the guild.
It was really sweet, and all of us were tempted, though I think we all decided independently to wait to join until next year.
When one of the recruiters found out that we were all getting our Ph. D.s in Chemistry\footnote{which, in retrospect, is probably not the most common answer for a group of four twenty somethings at an event to give, for all that it's a common one for me}, she was elated and told us that there was a former chemical engineer for the state DNR in the guild.
We met her, and we bonded a little bit over the fact that embroidery, especially counted thread embroidery and cross stitch, are very rewarding if you have the sort of mind that a Ph. D. chemist does.\footnote{I don't know how to describe it, but it's there}

In a fun turn of events, she was the one who had set up the temari ball exhibit, and was more than happy to tell us a lot more about the craft.
It was fascinating to consider the fact that a lot of the skills I've been working to develop as an analytical chemist\footnote{e.g. tolerances, how to fix mismeasurements, how to understand what a physical change on one part of an object will do to the rest of it, how to fudge (which is technically slightly different than fixing a mismeasure)} apply really well to a craft like making a temari ball.
The fact that all four of us had taken a course on machining and CAD made the conversation all the more enjoyable.

When we'd gone through all of the art at the show, we stopped by the sale they had.
I got a book on designing Bargello patterns, mostly because it was filled with pretty designs and had a section explaining terminology.
The others each got their own different books, and we went to a craft store to get supplies.

Since then, I've started trying to learn how to embroider.
Right now, I'm still new enough that I keep being surprised to learn things.
It's fun, especially since I haven't been a novice in this way in something for a long time.
Dealing with not knowing what questions I should be asking is a skill I've let fall a little bit by the wayside, especially during my latest degree, which is meant to focus almost entirely on delving deeply into one or two small questions.
Still, it's something that I really enjoy, especially because I can see the ways that it intersects with so many other skills that I have or want to have at some point.
The fact that there will be a social aspect to the craft in the future, as my friends and I join the guild, only adds to that.

For all that the only print resource I have for embroidery is a book on designing Bargello patterns, I do not think that I will likely end up doing too much Bargello work.
Bargello embroidery, for those not in the know, is a form of counted thread embroidery based off of some extant art in Bargello, Italy.
Its emblematic style\footnote{as far as I have been able to ascertain} is relatively long vertical stitches being used to the exclusion of any other stitch.
As a person who personally loves the textural differences that vertical and horizontal lines can make, I don't think that I'll be too reliant on the style.
That being said, I am also now enough of an adult to do the scales of my different hobbies.\footnote{I think that the common phrase is eat my vegetables, but I happen to enjoy eating vegetables. That is, doing the parts of the hobby which feel less rewarding as a task you have done but which better enable you to achieve what you would like to do in the hobby.
I feel like the concept of practice your scales is something I could (should) absolutely go into much more depth over sometime this month, both because I'm realizing as this footnote grows ever longer that I have a lot to say about it, and also because I know I'll run out of ideas for what to blog about well before December rolls around (to say nothing of the fact that I also would, as of now, at least, like to continue this blog well into December and onward.
I'm a little sad when I look at the whole month of October and see only two posts.}
I'm perfectly willing to believe that practicing a simple Bargello pattern will become essential for my development as a fiber artist, and I'm willing to grit my teeth and bear it, even if I do hate working on the pattern.\footnote{not that I think I will. Truthfully the only part of embroidery I haven't fallen in love with is threading needles, but that just seems like a skill I'll get better at with practice, especially given how little I see people complaining about it/struggling with it}

Thus far, I've almost exclusively made a small pattern with gradually increasing numbers of threads, to see the way that the shape differs as it gets thicker.
It's interesting that the object seems shorter when it has more threads, especially since I can pull out a ruler and see that the grid based fabric is, in fact, still a grid.
I also find that I generally like the more filled in look more, which makes a fair amount of sense.
I've always been interested in texture as a part of creative media, but that interest has tended to be more in the way that light reacts\footnote{I tried a number of words, and even though I don't really like reacts, it's the best one I could find} with the media being worked.
I suppose the canvas I'm stitching into is, in many regards, the media I'm working, but it doesn't really feel like it, at least right now.
There is a part of me that really does enjoy looking at the canvas underneath the thread, which I'm now realizing might be my issue.

I do enjoy negative space in art, but I tend to feel like its use needs to be intentional.
Right now, the designs I'm working don't feel like the sort of art that need explicit negative space.
It's more than plausible that I will change my opinion as I continue into the craft and make more intricate artwork.
In fact, I'm almost positive that I will find a use case for nearly every thickness of thread I've worked so far.
I just don't know that they'll be my standard block.

Anyways, this has been a shockingly long and rambling musing.
In summary\footnote{for the youth, tl;dr}, I went to an embroidery show with friends a few weeks ago.
It was filled with really cool art and decorations.
I've started embroidering and I really enjoy it.

Daily Reflection:
\begin{itemize}
\item Did I write 1700 words for NaNoWriMo? I didn't start the morning with a writing coworking session, which meant that it took me far longer to write the NaNoWriMo words.
That being said, I still managed to get them all out.
\item Did I write a chapter of Jeb? I'll say I finished it, despite the fact that I need to find a way to wrap it up. It's only 2000 words, which feels too small nowadays.
Then again, that's longer than an average chapter for the book has been, so I guess I could stop here, but I'm about to go write with a friend, so that feels like a bad idea.

Preposting edit: I did end up finishing the chapter, and I added about five hundred words.
It's now the longest chapter of the month, clocking in at a solid 2465 words\footnote{by count of the software I'm writing uncompiled .tex into. I don't know if that's one to one for actual words written, but I assume that it's at least relatively true.}.
Actually, I think that this may even be my longest chapter ever for the book.
I'm not going to fact check myself right now, but it feels at least mostly true.
\item Did I blog? Woo! The streak is extended to day five.
\item Did I stretch? Oops! Forgot to stretch today\footnote{read: the hours passed by me in a flash because I motivated myself to write 100 words by reading 2000 (numbers rounded less than accurately, but the general gist was that I had many breaks)
I have stretched now, though I don't know if the way that I stretch these days is very similar to the way I used to stretch.
Lately, I've focused almost entirely on my neck, shoulders, and hips.
Historically, I think shoulders used to be a focus, but I can't remember ever really focusing on my neck or hips.
Eh, I'm older and live a different lifestyle than I used to.
It makes sense that my flexibility needs have changed.
\item Am I doing better at prayer than a rushed and thoughtless rosary? No. I was very rushed in my rosary last night and forgot to do the Angelus today.
\item Am I doing a good job writing letters to friends? There was some reason I thought that I would be able to say I did a good job today.
As I think about it, it was probably because I thought that I would, in the completely empty day I had, write a letter to a friend.
I suppose I could do that during the coworking session with a friend tonight.\footnote{in retrospect, writing all day before doing a fun writing hang may not have been my best choice. I count a net words of almost 5700, which is already a personal best.
Another hour of writing means that I'll have to find something else to occupy my writing anyways, since I've already done as much writing for NaNoWriMo as I'm letting myself (I forget if I mentioned why, but my goal is to write one fifty-ish thousand word novel this month. If I let myself write too much in a given day, that means that won't happen, probably)}
After all, there's nothing that says I have to be doing the same kind of writing.
You know what, I think I will!

Preposting edit: As of beginning this\footnote{second} edit, I have a total daily word count\footnote{not counting the letter I'm going to say I've written in a few more words} of almost 6400 words, trumping Thursday's record of 5367.
It's possible that I would have beaten that yesterday, but I had to draw a line for what words I would and wouldn't count in my word counts.
On the word writing site, where I'm explicitly incentivized to inflate the numbers, I'm more or less just writing everything that I can justify doing on a web platform.
On the spreadsheet that I keep for my own purposes, though, I only count creative writing.\footnote{fairly consistently, though I'm pretty sure that I have counted other writing before.
This month, at least, I'm not planning on doing that.}
That means that I'm not counting the work I did on a manuscript\footnote{research paper draft} for the group, despite the fact that the website thinks it was almost a thousand words.
Anyways, that's a very long winded way of saying that I am not counting the writing I'm obligated to do for work or anything analog in my counts for the month.
While that may not totally reflect the number of words I write, I feel like it's more accurate to my goal, which is writing more creative words.
Sorry, I lost the thread there.
I did end up writing a letter to a friend, and I think it ended up being around two hundred words.
As I said above, though, I'm not going to count it in my word total.
It is nice to have made progress on that monthly goal, though!
Now I just need to address and send the letter!\footnote{and write the other letter I've addressed an envelope, and address and write the other letters I was planning to send}
Unfortunately, as I mentioned in yesterday's musing, my writings do not always align perfectly with the word goals I have in the application.
When that happens, I find that I tend to put the extra words here, which is certainly an interesting reflection on how much I value this blog compared to the other two writing exercises I'm doing this month.\footnote{anyways, this now puts me to almost sixty eight hundred words, which really goes to show you how needlessly verbose this entire edit was, given that I started it at around 6400 (can you tell I needed just a few more words still?)}
\end{itemize}
\end{document}