\documentclass[12pt]{article}[titlepage]
\newcommand{\say}[1]{``#1''}
\newcommand{\nsay}[1]{`#1'}
\usepackage{endnotes}
\newcommand{\B}{\backslash{}}
\renewcommand{\,}{\textsuperscript{,}}
\usepackage{setspace}
\usepackage{tipa}
\usepackage{hyperref}
\begin{document}
\doublespacing
\section{\href{scheduled-life.html}{On a Scheduled Life}}

First Published: 20 January 2025

\section{Draft 1}  
As I mused about \href{planning-2025}{at the start of the year}, I want to stop wasting time.  
The easiest way to do that, as I can see, is to see what I actually do with my time, and also to make sure that the things I want to do are on the list.  
I opened all the posts\footnote{I should really add some meta data to them so I can sort more easily. Not entirely sure how to do that, though, so would have to ask for help} that even vaguely referenced schedules as I prepared to write this post.

However, before that I made my actual schedule.

\say{How?} you might be asking, either as someone looking for insight into my mind or as someone who wants to organize their own life.  
It was certainly not the most time efficient method, though I think that it may have ended up as the optimal method for me.

First, I sat and intentionally did not have audio going.  
I know that may be obvious to many, but your brain cannot think as well or deeply or freely if you're listening to an audiobook.  
From there, I started writing down my goals for the day, and that led quickly into the things that I generally want to do, and their regularity.

I traveled to a new location\footnote{for unrelated reasons} and took that messy document\footnote{5 hand written pages on top of already printed paper} and compiled it into two pages: goals for the day and my general musings.  
The goal for the day was nominally in linear order, but the musings were not in the slightest.  
My general musings had me breaking my life into:  
\begin{itemize}  
\item Things to do before work  
\item Things to make sure to do every day at work  
\item Things to do after work  
\item Weekly commitments  
\item some random musings on things I'd like to do soonish\footnote{brew something again, start writing my thesis in earnest}  
\item a list of activities that I wanted to do along with an annotation for the frequency I wanted to do it and its importance to me, e.g.\footnote{I feel like I'm using more itemized lists in my blog these days. Wonder what that's about}\textsuperscript{,}\footnote{also these are made up examples, because I want to illustrate the system better, and while I understand it, I think that it's generally helpful to show manufactured examples to make sure details are clear}  
\begin{itemize}  
\item [3W] Serenade a pigeon 6.5. This means 3 times each week I wanted to serenade a pigeon, and its importance is 6.5/10.  
\item [D] Touch grass 9. This means daily I want to touch grass, 9/10 importance.  
\end{itemize}  
\end{itemize}

Of course, I was not immediately able to annotate the importance of items without having the full list.  
More than that, though, I don't love absolute measurement scales, and so I started with my first item, decided that it wasn't very important, and assigned it a low score.  
The next item on the list seemed relatively more important, so I gave it a six.  
The next item was even more important, so I gave it a seven.  
Item 4 was in between items two and three for importance, so I gave it a 6.3, since I felt like it was closer in importance to item two than item one.  
I continued this through the entire list.

Now that I'm done with it, I do find it interesting that so many of my goals were in the 6 to 7 range and the 8 to 9 range, with nothing between 7 and 8, and only a single item below 6.\footnote{which, in retrospect, might be why it's so clumped.}  
Even if the exact values might change, it does show me that I have one goal I don't care about\footnote{the 4}, one goal I care about a lot\footnote{the 9}, groups of goals that are very important to me\footnote{8-9 range}, and groups of goals that are moderately important to me.\footnote{6-7 range}  
I do keep some absolute scaling, I guess, because nothing was a 10\footnote{life or death}, and nothing was a sub-4\footnote{things I don't care about}.  


I then took my annotated pieces of\footnote{white, printer} paper home and didn't look at it until today.  
Before bed, however, I made a list of tasks that I wanted to accomplish today in my bedside journal.\footnote{musing to come}  
Upon waking up and going through parts of my morning ritual\footnote{i.e. putzing (I always forget that not everyone throws the occasional yiddishism into conversation) around}, I sat down with the two pieces of paper and the goals from yesterday's journaling\footnote{transferred from the journal to another} and grabbed some more blank paper.

First, I wrote in large letters and a sectioned off block the nominally immovables in my schedule.  
That is, the things that happen at a set time.  
I say nominally because in that list was also when I work\footnote{which is flexible} and a few things that have rough end times.  
Below that I made a list of all the things that I had sorted yesterday, now ranked from most to least important.  
The new list looked something like:  
\begin{itemize}  
\item Touch Grass [5M] [D] [NA]  
\item Make sure Gravity exists [1H?] [W] [3W]  
\end{itemize}  
That is:  
\begin{itemize}  
\item Spend at least five minutes\footnote{5m} touching grass each day\footnote{D}, with no upper limit on times to do so.  
\item I think that it will take me around an hour\footnote{hence question mark} to make sure gravity exists, and I want to do it once a week, but if I had empty space in my schedule, I wouldn't want it to happen more than three times a week.  
\end{itemize}

One of my big goals was making use of the fitness class access pass I bought for the semester.  
They break all activities into one of four categories:  
\begin{itemize}  
\item Cardio  
\item HIIT  
\item Strength  
\item Mind Body  
\end{itemize}

I always want more cardio fitness, generally enjoy feeling stronger, and would love to be flexible and in touch with my body.  
As a result, I then went to the page where they listed each activity per category, e.g. for the made up category of \say{Fun}  
\begin{itemize}  
\item Touching Grass  
\item Singing Songs  
\item Testing Gravity  
\item Sleeping  
\item Yoga  
\end{itemize}

I did my normal way of deciding between options, ranking each item in relation to the rest, and started ranking them on the page:  
\begin{itemize}  
\item Singing Songs > Touching Grass  
\item Singing Songs > Testing Gravity > Touching Grass  
\item Sleeping >> Singing Songs  
\item Yoga $\ge$ Sleeping  
\end{itemize}  
Which means that\footnote{from bottom to top} I think that I would probably enjoy yoga and sleeping the same amount, though if I had to choose I'd lean towards yoga.  
Sleeping is significantly preferable to singing songs.  
Singing songs beats out testing gravity, which beats out touching grass.

After going through each of the four categories, I made my overall ranking, which unsurprisingly was almost entirely less than or equal to signs.\footnote{because I only picked my favorites from each category and there's something in each category I value}

From there, I went down and listed every single option on the calendar that the school provides, because I prefer being able to look through hand written pages, because I don't like their layout\footnote{I understand that overlapping class times makes things hard, but listing two events on top of each other because one lasts ten minutes longer hurts me}, and because there were only a few activities that I cared about. I did make this new list still broken down into category, e.g.  
\begin{itemize}  
\item Fun (Only Yoga)\footnote{because in going through the rest I realized that what I wanted from Sleeping was equally available from other sources}  
\begin{itemize}  
\item 715-8 M-F (runs from 7:15 to 8 AM any day of Monday through Friday)  
\item 715-815 TR\footnote{R means Thursday} (runs from 7:15 to 8:15 AM on Tuesday or Thursday)  
\item 1300-1400 U\footnote{Sunday} (runs from 1 to 2 pm on Sunday)  
\item 1500-1550 A\footnote{Saturday} (runs 3 to 3:50 PM Saturdays)  
\end{itemize}  
\item Thinking (Where I kept both Biting and Crying)  
\begin{itemize}  
\item 715-8 M-R B  
\item 730-9 F B  
\item 9-10 C  
\end{itemize}  
\end{itemize}

Etc. etc.

On a new page, I looked back at the group fitness goals I had\footnote{between 2 and 6 times a week}.  
Of the four activities I compiled, I wanted to make sure to do three of them at least once a week.  
I then went through and tried to find a way to do that.  
Because of prior commitments, there was only a single time each week I could do a barre class\footnote{cardio and flexibility! Perfect for someone like me with neither}, a single time a week I could do Zumba, and a nice yoga spot every morning before I wanted to get to work, so below the listed set of workouts with prior constraints removed, I made a list of what I wanted to do each day and when.

From there, we moved on to the actual scheduling part of my time.\footnote{Why did I spend the last thousand or so words on fitness? Because that's what I spent my time on, and it only seemed fair to say so}  
I started by listing the events that I wanted to do, starting with waking up.  
Using my time estimates, I was able to back track how early I would need to wake up in order to get each item I wanted to do before work done before going to yoga.  
Thankfully everything except for the 4 fit.  
It took some restructuring, because as I walked through the morning in my mind, I realized that I have a connection where some activities feels better to move into others.  
Like those logic puzzles where everything is given relative referents, I found a solution that worked, along with the quick time reminder of how long each would take.

Saturday and Sunday were the odd ones out, because neither has work, and so timing is weird.  
Since there's another yoga slot later in the morning on Saturdays, I just moved the entire schedule so wakeup is at the same relative time to yoga.  
Sunday I set the wakeup time with the other five days of the week, but decided to go to it last.

Because my schedule diverges daily after yoga, I then went through my Monday through Friday schedule, noting the activities that I wanted to do before going home, with time stamps where relevant.  
On the next page, I started to plot out what I wanted to do each night, and what I also wanted to do differently each night.  
I also started plotting out ideas for Sunday.  
At this point, Monday through Friday is nominally scheduled from waking to sleeping\footnote{though I will be very clear that there are two activities called \say{sit in silence}, and I always try to be pessimistic about how long things take because bonus time is fun}, and Saturday and Sunday are relatively empty.  
Looking at the ordered list of floating items, I was happy to se that most all of them were done, with only the weekly tasks yet to be assigned.  
I made a list of those items under the day I felt like they belonged, since floating items only need to be scheduled relative to the fixed items and each other if relevant.

I then made my weekly flowchart, which diverges and converges fairly often.  
Because routines are important, I tried my best to keep the divergent streams as similar to each other as possible.  
I also put literally everything I could think of as a necessary part of the schedule\footnote{turn on kettle for tea} as an example, because cognitive load is bad.

I opened another small journal and made nine lists:  
\begin{itemize}  
\item Things to do every morning (so everything I do before the divergence, with wake time removed, because I know when that is)  
\item Things to do every night  
\item Monday tasks...  
\end{itemize}

Finally, two hours after beginning, I was finished with my goal of plotting my ideal life.  
Most of it will be as simple as following a checklist, with the only\footnote{scheduled} cognitive load coming at the beginning and end of each day, where I go over my short term goals, and at the beginning and end of each week, where I do the same with slightly longer term goals.\footnote{Long term goals go here}  
I'm hopeful that this will work and that it will make my life better.

Now to look at all the previous times I've tried and failed to see if there might be any common patterns:\footnote{in alphabetical order by URL}  
\begin{itemize}  
\item My \href{goal-updates}{goal updates from last December}, which made me realize that I was aiming for far too many tasks, and that a lot of them weren't as important to me as they felt while sitting at a computer. That's one reason I did the entire scheduling process by hand, because I know that I think better unplugged.  
\item \href{how-im-keeping-track-of-my-time-abroad}{an early musing}, where I mostly just reflected on the various ways that I tracked time passing by, which hasn't really changed\footnote{except for using less social media to communicate}  
\item \href{living-scheduled}{about two years ago} I said that I found that having a list of items was really helpful. Looking at the calendar of blog posts, I was indeed going through a good period of, if nothing else, getting a blog out daily. Emotional things happened which threw me off my game.  
Of course that's the risk of making any human connection, and I find it worthwhile to make connections despite that fact.  
\item \href{planning-to-fail}{this post from a month or so later} points out that, short of actively making time for something, I will often lose it.  
\item \href{schedules}{About two and a half years ago}\footnote{oof 2022 is not one year plus away anymore} I realized that making a list of people to contact with frequencies, or setting up regular reminders, at least, works really well.  
I fell off of it about two years from the post, which makes sense given what was going on in my life (see the aforementioned risks of connection).\footnote{i do feel like I'm more and more treating footnotes as asides and annotations, rather than parentheticals. I should think about whether that's the goal or not. Certainly using hyperlinks is a good way to avoid needing to link, but. Eh a topic for another musing}  
\item \href{schedules-2}{Naively}, I assumed that an autopopulated list on my phone would save me. I did not like looking at it, and it was far too repetitive\footnote{which is something that I'm hoping nested lists like I have might help}.  
\item A few weeks \href{schedules-3}{later}, I spent a fair amount of time thinking about what's important to me and how I can make time for it.\footnote{wow I just got hit with a wave of sadness, remembering how healthy and energetic mom was then} \say{I think that I have the space to be able to do so, at least for now} is the relevant quote, because I was again, too optimistic in what I wanted to do. Might be good to have some sacrificial items that I can comfortably say I'll get rid of if I'm out of spoons.\footnote{Reading each night is definitely one of them. The afternoon workout can be skipped at least a few times a week}  
\item Speaking of \href{spoons-and-spell-slots}{spoons}, running out of energy is a very real thing, and many of the activities I do when low energy do not replenish me.\footnote{short of starting the day over, I'm realizing I don't know of too many that work reliably for me}  
\item In what is a weird happenstance \href{spoons-2}{I wrote \say{About eighteen months ago}} about eighteen months ago. It contained the poignant quote \say{I find it interesting that I continually look at scheduling my life as a
potential solution. Intellectually, it very well could be, but I have
such an aversion to scheduling myself that I don’t know if it will be
probable for me without changing something major about my life}\footnote{I've generally been skimming the posts, if you're wondering how I missed a good one}. It pointed out that rest, deadlines, and timers all help me, and, as the quote points out, I resist self scheduling.

As I mention there, the more tired I am, the more I am willing to call my own bluff and point out that any commitment to myself is only as binding as I want it to be.  
More than that, though, I've realized that I need activities to be scheduled immediately succeeding each other with no space for me to get lost.  
That is, if it only takes me twelve minutes to get home instead of the fifteen I scheduled, I cannot let myself go \say{ooh yay three free minutes to do whatever!}.  
Instead, I need to immediately move to my next task.

Being early to externally timed events, though, I'm ok with my tendency to disappear time.  
\end{itemize}

It's very possible that I missed some posts there, but I don't really care that much.  
I think that the important pieces for me in particular to remember about my aversion to schedules are:

\begin{itemize}  
\item Relative times are much better than absolute times. Or, don't set timers for things, just have a list of what follows what.  
\item Breaks are almost impossible for you to recover from. As the fact that you sat down for two hours and with complete focus wrote your schedule, before immediately spending the past 75 minutes writing this blog post without distractions points out, focused work for longer is better. Transitions are the killer.\footnote{so no pomodoro}  
  
That might be an issue, because I do also want to get into the habit of moving each hour and drinking water often. I suppose that bathroom breaks tend not to count as a break in my mind, so as long as I'm hydrating to an hourly rate, that should work.  
\item There are far more things that you want to do than you have time for.

I didn't schedule any time for my web serial, though dropping yoga would free 6 hours a week, and Sunday should have enough free time to make for writing if I needed.  
I could also drop my nightly scheduled reading time, or shorten the morning silence time.  
I also allocated 45 minutes to blogging each night, which is almost never the time it actually takes me.  
Unfortunately, 5 and 90 do average out to about 45, and that does generally seem like a good amount of time for me to spend on a general post. Maybe the excess time can be put to use writing the book.  
\item Time without stimuli is good and healthy and necessary.  
\item The less mental\footnote{and probably physical} effort that a task takes, the easier it is to do.

No, that isn't entirely right.  
\say{The easier it is to start a task, the easier it is} reads as far more accurate to me.  
As the struggle with transitions implies, starting a task is really difficult for me.  
The more that I can tell myself the task is automatic, the more likely I am to do it.  
\item Finally, you need to be doing well to be doing well.  
Almost every time that I've stopped this blog in the past\footnote{which is my shockingly accurate gauge of my general well being} I've not been doing well.
I don't really know how to fix that, except to make sure to avoid running myself ragged.  
\end{itemize}

All this being said, I do hope that I am able to maintain my life when the summer, with its constant interruptions to schedules, arrives.  


Daily:  
\begin{itemize}  
\item Practiced guitar? It is now on my morning list, but I have yet to do it today.  
\item Twice daily stretching? In addition to everything else I did yesterday, I also made my stretch list.  
\item Poetry? Need to get started with that again.  
\item Blog? A behemoth!  
\item Net cleaner home? I'm going to say yes, but I'm also going to spend some time right now fixing it. I do know that people say the more specific a task is the better, but I feel like right now there's enough low hanging fruit for fixing things that I don't really need to plan that much. I can't wait for the time when that is no longer true and I once again find myself having to decide what to clean.\footnote{Which is the point of nightly goals, use the remaining brain power before reset to decide. Transfer in morning in case light makes me change my mind}  
\end{itemize}


\end{document}