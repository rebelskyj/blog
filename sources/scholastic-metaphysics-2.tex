\documentclass[12pt]{article}[titlepage]
\newcommand{\say}[1]{``#1''}
\newcommand{\nsay}[1]{`#1'}
\usepackage{endnotes}
\newcommand{\1}{\={a}}
\newcommand{\2}{\={e}}
\newcommand{\3}{\={\i}}
\newcommand{\4}{\=o}
\newcommand{\5}{\=u}
\newcommand{\6}{\={A}}
\newcommand{\B}{\backslash{}}
\renewcommand{\,}{\textsuperscript{,}}
\usepackage{setspace}
\usepackage{tipa}
\usepackage{hyperref}
\begin{document}
\doublespacing
\section{\href{scholastic-metaphysics-2.html}{Scholastic Metaphysics Chapter 1}}
First Published: 2022 February 25

\section{Draft 1}
As I mentioned yesterday, I failed to read the chapter for the week.
But, today is a new day, so let's try again.

In the view of the author, philosophy is an attempt to connect human experience to the whole of creation.
As such, it is less focused on generation of new experience, but rather the attempt to connect these experiences to broader truths.
Metaphysics, then is focused on creating a unified view of reality from these discrete points.

That is, we try to find what properties and laws all things within the whole (universe) of creation follows.
This definition apparently comes from Aristotle, who coined the term.
Or, metaphysics is \say{the study of \emph{all beings} precisely insofar as they are \emph{real}}\footnote{p 6, emphasis in original}.

As in chemistry,\footnote{since I'm in theory reading this to connect to chemsitry}, we cannot directly inspect everything.
We must find what connects that which we can measure and find a way to apply this to the whole of reality.
Apparently we can also make an argument for the existence of G-d.
The argument belongs moreso in natural theology, but it is apparently an outcropping of this metaphysics?

In the Christian Tradition, we hold that there are two major ways the Most High speaks to us, the Book of Nature (creation) and the Book of Revelation (Revelation).
As both are created by Him, apparent contradictions between natural reason and theology must mean that one of the two is in error.
Metaphysics differs from religion, however, as metaphysics is solely concerned with the intellectual, rather than the assent and action to conform to the Lord's plan.

As cultures are limited in their understanding, so too are metaphysical explanations limited.
As with cultures, though, these explanations can be informed by all peoples' interactions with Creation.

There are\footnote{apparently} some common objections to the concept of metaphysics.
One is that metaphysics has no distinct subject matter.
Chemistry studies chemicals, biology life, and so forth, yet metaphysics claims to study everything.
However, metaphysics studies from only a single frame of reference, which is how all things are like all other things.
One important way that things are like other things is that they \emph{are} at all, which is something that most fields of study take as implicitly true.
Even as I read this, it is difficult for me to conceptualize the question the author poses: \say{How come there is a real universe at all?}\footnote{9}

The second argument is that we, as part of Creation, cannot understand Creation because we are within it.
Personally, that argument falls flat for me without reading the response, because scientists often claim to study systems from models or parts, rather than the entirety of the system.
I think that's similar to what the author argues, though he says moreso that we are endowed generally with an ability to do this.

Third, people argue that either empiricism, Kantianism, or relativism are better models of the universe.
Empiricism claims that we cannot know without deriving from experience.
Hume was a proponent of this.
Mostly the author just goes \say{this is a bad take, if we accept this then philosophy fails.}\footnote{approximate quote from 11}
Kantiansim, from Kant, claims that external reality does not shape our personal experience, rather our personal experience creates external reality.
The author points out that Kant's claim fails because if each person is creating the external reality, then how does that work when two people interact?
Relativism claims that there is nothing truly transcendental, rather only socially true.
Again, math's apparent universality would suggest this to be untrue to me, but the author rather just says that the argument is self-defeating.

In short, the author claims that objections to metaphysics themselves start with a metaphysical stance, such as that there are no universal truths.
That is still a statement about the nature of all reality.

Moving past the defense, the author claims that we would not seek knowledge if it did not appear good.
Though we cannot know fully, we still search to know more.
And, as we may have the desire to know, reality matches us by allowing itself to be known, the so-called \say{intelligibility of being}.
He claims that nothing has yet been shown to be unintelligible, but I think I remember something in math that claims the opposite.
Then again, math does not deal with the real, so it is not truly a contradiction.

There is the principle of non contradiction, which claims that things cannot both be themselves and not themselves at the same time.
That feels fairly definitional to me, which makes me wonder why you have to defend it.

There is also the principle of sufficient reason, which states that things which cannot themselves explain their existence must owe their existence to something else.\footnote{I think}

So there are apparently questions at the end of each chapter.
Maybe next week I'll just respond to each question rather than live-blog my reading.
\end{document}