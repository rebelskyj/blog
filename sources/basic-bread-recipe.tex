\documentclass[12pt]{article}[titlepage]
\newcommand{\say}[1]{``#1''}
\newcommand{\nsay}[1]{`#1'}
\usepackage{endnotes}
\newcommand{\1}{\={a}}
\newcommand{\2}{\={e}}
\newcommand{\3}{\={\i}}
\newcommand{\4}{\=o}
\newcommand{\5}{\=u}
\newcommand{\6}{\={A}}
\newcommand{\B}{\backslash{}}
\renewcommand{\,}{\textsuperscript{,}}
\usepackage{setspace}
\usepackage{tipa}
\usepackage{hyperref}
\begin{document}
\doublespacing
\section{\href{basic-bread-recipe.html}{Basic Bread Recipe}}

\section{Draft 3}
In my first year of college I gained\footnote{and lost} the moniker of \say{bread guy.}\footnote{the nickname apparently spread outside of my friendgroup(s), and when I've mentioned that I used to bake a lot of bread, I've had college friends exclaim \say{Oh! So you were the bread guy!} and then express displeasure over having missed meeting me until I became boring. (Dear Mom, I apparently was cool)}
Now, this was not an unforeseeable outcome.
I baked\footnote{I really feel like the past tense of bake should be \say{boke}} bread almost every day, and tended to have the same problem: by the time I finish making bread,\footnote{or a lot of things if I'm being honest} I don't really want it anymore, since I would've been smelling it for the time it took to cook.\footnote{apparently many people feel the opposite about these sorts of things}\,\footnote{no, despite the fact that I've made upwards of 50 gallons of bread (because how else would you measure bread dough?) I still don't know how long it takes to cook. It led to a problem this summer}

This problem should have an easy resolution.
I would just give it to my friends!\footnote{Yay friends!}
Of course, sometimes my so called \say{friends} wouldn't want to eat the bread I made.\footnote{and wow it hurts the ego}

I know, It's ridiculous to me too.
Who wouldn't want bread?

Nonetheless, I would need to get rid of the bread another way.
My next step would be to find the people who occupy the hard to define mental space of people you know, but have no emotions attached to.\footnote{I really hope that's a group of people others have in their mental spaces}
Once I found them,\footnote{or messaged them} I would also offer bread.
That also didn't have a 100\% success rate, which hurt less.

Sometimes, I would just walk around campus and hand it out\footnote{more often when I made little breads (which are objectively adorable)} to whoever I saw, and sometimes I would just leave it in public spaces near people who volunteered to dispose of it for me.
Nonetheless, I made a lot of bread and got known for it.

Now, the moniker served another purpose.\footnote{yay for forethought and multiple advantages}
The CS\footnote{Computer Science} department\footnote{I really hope I'm using the right nomenclature here, I wouldn't want \href{http://www.cs.grinnell.edu/~rebelsky/musings/busses-bussing}{another} snarky musing written about me} at Grinnell College\footnote{Best school in Iowa(the best state)!}was\footnote{is?} growing rapidly, so there were many students who might assume I was related to a member of the department.\footnote{not incorrectly}
I assumed\footnote{foolishly} that being known as \say{slightly odd person who feeds us} was better than being known as \say{child of very odd, grumpy, and sarcastic person,} if only because I was being described as my own entity in the first.

But anyways, back to what's important: bread.
My recipe is adapted from \href{https://artisanbreadinfive.com}{this website's book}.
I tend to make the dough in a 3 gallon ice cream container from Kilwins, as my family somehow acquired many of them.
It exists below.

Basic Bread Dough:

Ingredients:\\
5 pounds flour\footnote{yes, one bag of flour. I cook on the principle of \say{needing more than two tools to measure everything is a sign of poorly scaled recipes}}\\
8-9 cups water\footnote{depending on humidity and how long you plan to let the bread sit}\\
2 heaping tablespoons yeast\footnote{or not. It's yeast, it grows. The more you add the faster it grows. The less, the slower}\\
2 lightly heaping tablespoons kosher salt\footnote{wow that's such a vague measurement. I guess slightly better than \say{enough} but really, just add enough that the dough tastes right.(Disclaimer: I'm not responsible for injuries resulting in dough consumption). \href{https://www.fda.gov/ForConsumers/ConsumerUpdates/ucm508450.htm}{Apparently} raw flour is dangerous. But, what's life without a little danger?}\\
Note: if making a second batch, if you leave the dough that is leftover in the container, you can skip the yeast.\footnote{also your dough will start more sour, which if you want that is good}

To Make Dough:\\
Add all ingredients together.\footnote{should I have said: add all ingredients to container? Eh}\\
Mix until it seems to be one coherent blob of dough.\footnote{blob is a very technical term}\\
Cover, but not airtightly, and let sit at room temperature until doubled in size.\footnote{because buildup of pressure can mean your house being covered in dough. According to many sources, that's a bad thing}\\
Then, cover airtight\footnote{or not if you're lazy} for up to two weeks\footnote{I assume you could go longer, but you know, it's bread dough, it's pretty easy to tell when it's gone bad}
The longer you let it sit, the more sour it will be.\footnote{duh}

To Cook:\\
Sprinkle flour on top of dough.\footnote{it makes me feel so cool}\\
Pull out a piece of dough the volume of bread you want.\footnote{i.e. loaf size, roll size, big loaf size, small loaf size, big roll size... (I'm not the best at knowing the size of things)}\\
Generously coating with flour, gently fold it over itself until smooth everywhere except one spot.\footnote{it's hard to explain but easy to do}\\
After placing the bread with the spot down,\footnote{generally on parchment paper} quickly score the surface of the dough with a razor blade, aiming for around half an inch deep.\footnote{the most fun part}\\
Put onto a stone in an oven preheated for around 30 minutes at 350 F\footnote{no C because I don't use metric when I bake} with a stone and a baking tray\footnote{I recommend one you don't particularly care about} inside, the tray beneath the stone.\\
Pour water onto the baking tray.\footnote{which is why no to the caring}\\
Let cook until bottom is hollow when tapped, or properly golden brown.\footnote{if you grab the loaves enough the heat stops hurting}\\
If in doubt, a few extra minutes never hurts, and often helps.\footnote{don't take that too far and just leave for 3 hours. Once it starts looking browner than gold it's almost certainly done}\\
Pull out and let cool.\footnote{mmm, bread}\\

\section{Draft 2}
In my first year of college I gained the moniker of \say{bread guy,}\footnote{the nickname apparently spread outside of my friendgroup(s), and when I've mentioned that I used to bake a lot of bread, I've had college friends exclaim \say{Oh! So you were the bread guy!} and then express displeasure over never having met me. (Dear Mom, I have made friends)}
Now, this was not an unforeseeable outcome.
I baked\footnote{I really feel like this should be \say{boke}} bread almost every day, and tended to have the same problem: by the time I finish making bread, I don't really want it anymore, since I would've been smelling it for the time it took to cook.\footnote{apparently many people feel the opposite about these sorts of things}\,\footnote{no, despite the fact that I've made upwards of 50 gallons of bread (because how else would you measure bread dough?) I still don't know how long it takes to cook. It led to a problem this summer}

This problem should have an easy resolution.
I would just give it to my friends!
Of course, sometimes my so called \say{friends} wouldn't want to eat the bread I made.

I know, It's ridiculous to me too.
I would need to get rid of the bread another way.
My next step would be to find the people who occupy the space of \say{people I have no feelings towards but am aware of their existence and believe they feel the same about me.}\footnote{I really hope that's a group of people others have in their mental spaces}
Once I found them,\footnote{or messaged them} I would also offer bread.
Sometimes, I would just walk around campus and hand it out\footnote{more often when I made little breads (which are objectively adorable)} to whoever I saw, and sometimes I would just leave it in public spaces near people who volunteered to dispose of it for me.
Nonetheless, I made a lot of bread and got known for it.

Now, the moniker served another purpose.
The CS\footnote{Computer Science} department\footnote{I really hope I'm using the right nomenclature here, I wouldn't want \href{http://www.cs.grinnell.edu/~rebelsky/musings/busses-bussing}{another} snarky musing written about me} at Grinnell College\footnote{Best school in Iowa(the best state)!}was\footnote{is?} growing rapidly, there were many students who might assume I was related to a member of the department.\footnote{not incorrectly}
And, I figured being known as \say{slightly odd person who feeds us} was better than being known as \say{child of very odd, grumpy, and sarcastic person,} if only because I was being described as my own entity in the first.

So, bread.
My recipe is adapted from \href{https://artisanbreadinfive.com}{this website's book}.
I tend to make the dough in a 3 gallon ice cream container from Kilwins, as my family somehow acquired many of them.
It exists below.

Basic Bread Dough:

Ingredients:\\
5 pounds flour\\
8-9 cups water\\
2 heaping tablespoons yeast\\
2 lightly heaping tablespoons kosher salt\\
Note: if making a second batch, if you leave the dough that is leftover in the container, you can skip the yeast

To Make Dough:\\
Add all ingredients together.\\
Mix until it seems to be one coherent blob of dough.\\
Cover, but not airtightly, and let sit at room temperature until doubled in size.\\
Then, cover airtight\footnote{or not if you're lazy} for up to two weeks\footnote{I assume you could go longer, but you know, it's bread dough, it's pretty easy to tell when it's gone bad}
The longer you let it sit, the more sour it will be.

To Cook:\\
Sprinkle flour on top of dough.\\
Pull out a piece of dough the volume of bread you want.\\
Generously coating with flour, gently fold it over itself until smooth everywhere except one spot.\\
After placing the bread with the spot down,\footnote{generally on parchment paper} quickly score the surface of the dough with a razor blade, aiming for around half an inch deep.\\
Put onto a stone in an oven preheated for around 30 minutes at 350 F\footnote{no C because I don't use metric when I bake} with a stone and a baking tray\footnote{I recommend one you don't particularly care about} inside, the tray beneath the stone.\\
Pour water onto the baking tray.\footnote{which is why no to the caring}\\
Let cook until bottom is hollow when tapped, or properly golden brown.\footnote{if you do it enough it stops hurting}\\
Note: I guess you can add extra water whenever, I just don't because I'm lazy.\\
If in doubt, a few extra minutes never hurts, and often helps.\\
Pull out and let cool.\\

\section{Draft 1}
My freshman year of college, I was nicknamed \say{the bread guy,}\footnote{the nickname apparently spread outside of my friendgroup(s) even, and when I've mentioned that I used to bake a lot of bread, I've had college friends exclaim \say{Oh! So you were the bread guy!} I promise it's real}
Now, this was not an unforeseeable outcome.
I baked bread almost every day, and ran into my usual problem with baking bread.
By the time the bread would be finished cooking, I wouldn't really want to eat it anymore, since I would've been smelling it for the time it took to cook.\footnote{no, despite the fact that I've made upwards of 50 gallons of bread (don't ask why that's my measurement) I still don't know how long it takes to cook. It led to a problem this summer}

So, I would need to get rid of it somehow.
Thankfully, instead of just setting it on fire,\footnote{which happened once, and was a complete accident} or otherwise wasting it, I would give it to friends.
Of course, sometimes my so called \say{friends} wouldn't want to eat the bread I made.

I know!
It's ridiculous to me too.
So, I would need to get rid of the bread another way.
Generally this meant finding the random people you know that you wouldn't consider friends\footnote{I really hope that's a group of people others have in their mental spaces} and offering them some fresh baked bread.
Sometimes, I would just walk around campus and hand it out\footnote{more often when I made little breads (which are objectively adorable)} to whoever I saw, and sometimes I would just leave it in public spaces near people who volunteered to dispose of it for me.
Nonetheless, I made a lot of bread and got known for it.

There was a motive behind getting this nickname.
I knew that, since the CS\footnote{Computer Science} department\footnote{I really hope I'm using the right nomenclature here, I wouldn't want \href{http://www.cs.grinnell.edu/~rebelsky/musings/busses-bussing}{another} snarky musing written about me} was\footnote{is?} growing rapidly, there were many students who might assume I was related to a member of the department.
And, I figured being known as \say{slightly odd person who feeds us edible home made food} was better than being known as \say{child of very odd, grumpy, and sarcastic person,} if only because I was being described as my own entity in the first.

So, without further ado: my basic bread recipe.
It is adapted from \href{https://artisanbreadinfive.com}{this website's book}.
I tend to make the dough in a 3 gallon ice cream container from Kilwins, as my family somehow acquired many of them.

Basic Bread Dough:
Ingredients:\\
5 pounds flour\\
8-9 cups water\\
2 heaping tablespoons yeast\\
2 lightly heaping tablespoons kosher salt

To Make Dough:\\
Add all ingredients together.\\
Mix until it seems to be one coherent blob of dough.\\
Cover, but not totally, and let sit at room temperature for a time lasting between: just barely doubled\footnote{approx. 4-8 hours} or up to 2 weeks.\footnote{don't ask how I know}\\
The longer you let it sit, the more sour it will be.

To Cook:\\
Sprinkle flour on top of dough.\\
Pull out a piece of dough the volume of bread you want.\\
Generously coating with flour, gently fold it over itself until smooth everywhere except one spot.\\
Putting that spot down,\footnote{generally on parchment paper} quickly score the surface of the dough with a razor blade, aiming for around half an inch deep.\\
Put onto a stone in an oven that has been heating for around 30 minutes at 350 F\footnote{no C because I don't use metric when I bake} with a stone inside, and a baking tray\footnote{I recommend one you don't particularly care about} beneath it.\\
Pour water onto the baking tray.\\
Let cook until bottom is hollow when tapped, or properly golden brown.\footnote{if you do it enough it stops hurting}\\
If in doubt, a few extra minutes never hurts, and often helps.\\
Pull out and let cool.\\
\end{document}