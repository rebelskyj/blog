\documentclass[12pt]{article}[titlepage]
\newcommand{\say}[1]{``#1''}
\newcommand{\nsay}[1]{`#1'}
\usepackage{endnotes}
\newcommand{\1}{\={a}}
\newcommand{\2}{\={e}}
\newcommand{\3}{\={\i}}
\newcommand{\4}{\=o}
\newcommand{\5}{\=u}
\newcommand{\6}{\={A}}
\newcommand{\B}{\backslash{}}
\renewcommand{\,}{\textsuperscript{,}}
\usepackage{setspace}
\usepackage{tipa}
\usepackage{hyperref}
\begin{document}
\doublespacing
\section{\href{reflections-on-readings-15-ordinary-c-22.html}{Reflections on Today's Gospel}}
First Published: 2022 July 10

Deuteronomy 30:10 \say{If only you would heed the voice of the LORD, your God,
and keep his commandments and statutes
that are written in this book of the law,
when you return to the LORD, your God,
with all your heart and all your soul,}

\section{Draft 1}
Being totally honest, the homily I listened to was absolutely stunning and I'm just going to steal from it.
In the Gospel today, Jesus is confronted by a scholar who asks what the greatest commandment is and how he can follow it.

The priest pointed out that the scribe wants to have neighbor be as restrictive a category as possible.
After all, if we have to love everyone like ourselves, where can we keep room in our hearts for hatred?
But, Love is not satisfied with restrictions.
Love is overwhelming and seeks to exceed.

In such a way, our neighbor is the whole of humanity.
We are called to love the Lord with all of ourselves, heart, mind, body, strength and being.
It may be incorrect, but I always saw the two parts of this commandment as a what and how.
How do we love the Lord?
We love the Lord in part by receiving sacraments and praying, obviously.
But, we also love the Lord by loving our neighbor.

Anyways, that's something I'll have to think about and focus on for the next week.
How do I avoid prioritizing my own self-interest above others?
\end{document}