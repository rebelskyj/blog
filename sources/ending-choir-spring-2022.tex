\documentclass[12pt]{article}[titlepage]
\newcommand{\say}[1]{``#1''}
\newcommand{\nsay}[1]{`#1'}
\usepackage{endnotes}
\newcommand{\B}{\backslash{}}
\renewcommand{\,}{\textsuperscript{,}}
\usepackage{setspace}
\usepackage{tipa}
\usepackage{hyperref}
\begin{document}
\doublespacing
\section{\href{ending-choir-spring-2022.html}{Ending a Year of Choir}}
First Published: 2022 April 27

\section{Draft 1}
This past weekend was my last concert in the school-sponsored choir I'm in.
Tonight was the last day of the candlelight choir at the Catholic Student Center for the semester.
Both still have a social gathering planned, but the end of the formal singing is something that I'm reflecting on tonight.

I've loved the way that I've grown in both choirs.
The school-sponsored one has clear things that I can point to.
The section leader was himself a DMA student in conducting, and there was a DMA in Vocal Performance in my section as well.
Both really helped me grow as a choral singer over the past two semesters.
The repertoire we sang from certainly pushed me as well.
It was primarily new music, and most of it was fairly difficult.

The candlelight choir helped me grow also as a choral singer, though differently.\footnote{shocking, I know. This is what you get from late night postings}
I mostly sang tenor, which is somehow fundamentally different to singing bass.
I can't articulate why, but I constantly struggled with tuning, pitch, and intervals.
In part, the tessitura is much higher, and in part I think choral composers just write the voices differently.
I'm far more used to being the root of a chord than the third, for instance.

I also got to compose for the candlelight choir, and that was fantastic.
We sang two of my songs, both written for the choir.
It made me remember how much I love to write music, and was, in retrospect, part of what encouraged me to restart this blog.

Sadly, I'm unsure if I'll be able to continue either choir next year.
If I cannot, I hope that I will be able to find other opportunities to make music.
If I can, then I will be far happier. 
\end{document}