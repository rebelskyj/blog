\documentclass[12pt]{article}[titlepage]
\newcommand{\say}[1]{``#1''}
\newcommand{\nsay}[1]{`#1'}
\usepackage{endnotes}
\newcommand{\1}{\={a}}
\newcommand{\2}{\={e}}
\newcommand{\3}{\={\i}}
\newcommand{\4}{\=o}
\newcommand{\5}{\=u}
\newcommand{\6}{\={A}}
\newcommand{\B}{\backslash{}}
\renewcommand{\,}{\textsuperscript{,}}
\usepackage{setspace}
\usepackage{tipa}
\usepackage{hyperref}
\begin{document}
\doublespacing
\section{\href{reflections-on-readings-3-ordinary-c-2022.html}{Reflections on Today's Gospel}}
First Published: 2022 January 23


Nehemiah 8:8: \say{Ezra read clearly from the book of the law of God, interpreting it so that all could understand what was read.}

\section{Draft 1}
According to the priest at Mass tonight, today is the celebration of the Word of G-d.
If that's the case, then the readings are especially appropriate.
The first reading has Ezra reading the Lord's word to the people of Israel.
The Gospel has the Word made Flesh in the person of Jesus Christ reading the prophecy that he himself fulfilled.

There's something beautiful about that to me.
It's really fantastic to see how so much of the bible is mirrored between the Old and New Testaments.
Last time I heard these readings, I thought of them as \href{reflections-on-readings-3-ordinary-c.html}{\say{one of the more straightforward Gospels}}
It's so weird to me how these readings used to seem plain and simple.
Looking at them now, they seem incredibly deep, more-so to me at least than a lot of weeks. 

The second reading in particular is something I feel that I should read more often.
It reminds me of a quote from\footnote{I believe} St. Therese of Lisieux which says something along the lines that a garden of nothing but roses is not more beautiful than a garden with all the flowers of nature, and a daisy is not made better by being a rose.
There's no part of me that was not lovingly fashioned by the Lord and Creator of all things, and I need to remember that more in my day to day life.
We are not unloved beings in an uncaring universe, but deeply loved and beloved beings made in the image of the Lord, who is Love.
\end{document}