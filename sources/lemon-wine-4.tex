
\documentclass[12pt]{article}  
\newcommand{\say}[1]{``#1''}  
\newcommand{\nsay}[1]{`#1'}  
\usepackage{endnotes}  
\newcommand{\B}{\backslash{}}  
\renewcommand{\,}{\textsuperscript{,}}  
\usepackage{setspace}   
\usepackage{tipa}  
\usepackage{hyperref}  
\begin{document}  
\doublespacing  
\section{\href{lemon-wine-4.html}{Lemon Wine 2025}}  
First Published: 2025 June 3

\section{Draft 1: 3 June 2025}

I've \href{lemon-wine}{written} \href{lemon-wine-2}{twice} \href{lemon-wine-3}{before} about my experiences making a lemon country wine.  
It's been a while since I did any brewing, and it is something that I would really like to get back into.  
With that in mind, I think that a nice lemon wine would go really well with my thesis defense, especially if I can make nice labels.\footnote{and even if not, the bottles will still be pretty}

Apparently sugar is about 45 points per pound per gallon, which means that a five gallon recipe with about 5 pounds of sugar should start out at about 1.045.  
If that ferments to full, ends up around 5 percent, which is good to know.

So plan is:

\begin{enumerate}

\item Purchase sugar and lemons, maybe water (grocery store)

\item Purchase yeast, bentonite, yeast nutrient (Fermaid?), and anything else that looks fun (homebrewing store, make sure that the ingredients are kosher)

\item Clean and sanitize everything (this is obvious but)

\item Pour about four gallons of water into the fermenter

\item Boil about 5 pounds of sugar with about a gallon of water and some acid to help break the molecule down for yeast.  
Once boiled, add in the yeast nutrient (calculate how much to use or trust the box)  
Also add in hops as appropriate.

\item Add the mixture to the large pot of water, ensure that the yeast won't die.

\item Add yeast, stir vigorously, MEASURE DENSITY and temperature, cover, walk away

\item Check regularly.

When at about 30 percent fermented away, add known volume of sugar water with known density, measure new density, use that to figure out current ABV

\item Repeat until satisfied

\item Stabilize the wine. I think that's bentonite, but that might actually be for defogging...

\item When stable, back sweeten and acidize and otherwise balance.

\item Bottle and cork and label

\end{enumerate}

What do I need to do adjacent to this:

\begin{itemize}

\item Hop trials: how long should I boil how much hop for use in lemonade?\footnote{add vodka to water to simulate the lemon wine I guessss}  
Make like a quart at a time, and try a small matrix of boil time and hop percentage.  
I might need to also do something with wait time: how long between adding the hop and tasting?

Oh shoot: hops are often not recommended in the preferment because their flavors blow off

I guess that means that I should plan on dry hopping or adding hops to the liquid during the backsweetening/acidifing phase.  
Hmm are hops acid stable?  
I wonder if making an oleo saccharum or citrum from hops might be something workable.  
I'll try it if my hops are still here.

I guess this means that I need to get a more precise scale as well.

\item Make the oleos.

Get good quality lemons, clean vigorously, allow to dry.\footnote{sanitize? with the keg sanitizing solution?}  
Peel, crunch with citric acid and/or sugar and/or other acids of choice.  
Seal in container, allow to rest for a while.  
Consider boiling after straining? to stabilize and whatnot?

If I put in glass jars, I can can it, I guess.  
Canning feels like a LOT though.  
I guess a pseudo can\footnote{heat} still feels close enough.  
I will check if I have any glass containers, and if so, might just put in the jar, seal, and then boil for a bit to try to make it a little more shelf stable.  
I guess i could also freeze them?

For hops, grind hops with citric acid? sugar? both? something else? I guess I'll try full acid, full sugar, and a fifty fifty blend of the two.

Hmmm, if I do just sugar, then presumably the sugary hop liquid will be much much less sugar than I need to get all the appropriate hop flavor.  
The same may not be true in terms of acid: I really don't know if the lemon liquid will have the right ratio of lemon to acid. I might do like a 60/40 acid sugar ratio?  
Maybe 70/30?

The goal is something that's really lemony, but also acidic.  
I need to extract all of the oil I can, and so whichever base I use will probably end up being the only contributor that I need.  
Unfortunately, I don't know the relative ratios of hop to lemon to use for an ideal solution.

I guess one option is to make multiple versions of each, but that feels excessive in a number of ways.  
Still, I don't know if I have an issue with doing one acid and one sugar for both hop and lemon peel.

Great, so that's sorted.

\item Buy the bottles and corks/make sure i know where the corker I own went

\item Figure out how to make labels and do so.

\end{itemize}

Great, that's all that I really feel like needs to be done. I'm excited to get started, so might try to hit the supermarket after work today? Hmm, to consider I guess.  
This is, of course, yet another motivation to get the home cleaned: I want sufficient space for brewing.

\section{Draft 0.6: 3 June 2025 (Diverts into Math)}

I have \href{lemon-wine}{written} \href{lemon-wine-2}{twice} \href{lemon-wine-3}{before} about my experiences making a lemon country wine.  
It's been almost two years since the last post, and an equivalent amount of time since the last time I did any brewing.  
With that in mind, I think that I would really, honestly and truly, like to make lemon wine in time for my defense.  
If I get done soon enough, I would even love to have it as a thing at the post-celebration or to gift to my committee.\footnote{which would require learning how to make a nice label, but that's definitely a craft that I can get behind}

So, what do I want to plan on for the recipe?

8 pounds of sugar has historically fermented somewhat sweet, which is a problem if I end up doing an oleo saccharum of lemon\footnote{literally sugar oil, basically using sugar to extract lemon oil}.  
If, instead, I do an oleo citrate\footnote{using citric acid instead, and wow I hate that name}, I end up with something that I can mix to exactly my specifications, but it would then mean that I would be unable to just add the remaining mixture to my drinks through the summer and onwards, because it would be horribly sour.\footnote{although, that's not inherently a bad thing, as I think about it... what's wrong with having acid and sugar allowed at different additions?}

Ok so that sounds like I'll try to make an oleo citrate.  
I have a bunch of citric acid, so that's not a problem.  
I think that when I was optimistic, I bought ascorbic and tannic acids, so might mix a little of them in water to see what they taste like.

I'll do the quick math for how much sugar I should add to make the water at the appropriate level for my yeast.\footnote{hmmmm what ABV am I shooting for? apparently 10-13 is a normal and good amount, so let's just say ten for the nice roundness of it all}  
I think that I'm going to go for a 71-B again, because that's always been a reliable yeast for me.  
It can easily handle 10 percent, and so I want to shoot for a starting density somewhere in the 1.08 range.  
That's apparently around 208 grams of sugar per liter, which is about 800 grams per gallon, which is about a pound a gallon.\footnote{assuming my back of envelope math is correct}

In previous years, I used about 8 pounds, and that ended up giving me\footnote{perks of takign notes} about 10 percent, and the notes there suggest that I would need about 8 pounds to make it work.  
Ok, so I'll just plan on using the full ten pounds of sugar, boiling it with some acid to hopefully turn the sucrose into its constituent fructose and glucose, which will hopefully make the yeast work better, happier and faster.  
There is, of course, something to be said about step feeding, which makes the yeast happier in the short and long terms.  
The only issue is that I need to be able to measure the density at all points.

Let's see, can I do a density calculation with some unknown volumes?

Say I know that I have a fermenter full of liquid with standard gravity 1.03\footnote{3 percent heavier than water}.  
If I add in a known quart of liquid with standard gravity 1.1\footnote{which feels like decent sugar water?}, and get new standard gravity 1.033, then...\footnote{getting out pencil (pen) and paper for this bit, because brain isn't working without writing by hand and the conversions here don't support tex equaitons}

Ok so the goal is to find alcohol by volume.  
In general, if I know starting and ending densities, it's given to me by some constant times the difference.\footnote{i think it's like 110, but I can find exact value later}  
Of course, I do not know the starting volume well.  
When adding in the second draft of sugar, I know the current density, the volume of sugar water being added, and the density of the sugar water.  
From this, we can find that the initial volume was the new volume times the amount the sugar water's density decreases divided by the amount the base's density increases.  
That's nice.  
From there, since we know the initial ABV and we have the initial volume, we can find the raw amount of EtOH in the solution.  
As the liquid ferments, it becomes another change in density problem, but we add the starting ethanol content.  
Only at the very end, when I am sweetening, do I really care that much about the overall ABV, and rough guides are probably fine, so I could also say that the amount of ethanol in solution is only slightly decreasing, and treat it as fixed\footnote{5 gallon is much larger than a quart, so ABV unchanged} and just do constant delta density.

Eh, step feeding seems like the winner for now. Let's look at the usual advice for that.

Huh wild, most people only step feed when they really want to hit incredibly high ABV.  
I wonder why that is.

Well, guess I don't really need to do that.  
Let's still try for it, though, because I want to make sure that it ferments dry.  
We'll eyeball about four or five pounds of sugar\footnote{read: whatever seems like it will fit well in the pot with water}, dilute to about five gallons, introduce the yeast and nutrient, then monitor for the next three days.  
Every time density drops by a third to a half, add new sugar water as needed.\footnote{read: make sure sugar water is cool first, even though we start with boiling water the first time through.}

Hmm, do I need to get nice water for this?  
If spring water is cheap, it does save a bit of effort.  
Eh, we'll see if there's readily and cheaply available water\footnote{what is cheap? great question} when I go to get sugar.  
This draft is also not great, because it's just me figuring out that I want to step feed, and some of the math therein.

\section{Draft 0.5 3 June 2025 (deciding that sugar is my friend)}

I have \href{lemon-wine}{written} \href{lemon-wine-2}{twice} \href{lemon-wine-3}{before} about my experiences making a lemon country wine.  
It's been more than a year since the last one\footnote{going on two, actually,} and I think that it's even been that long since my last brewing.  
I officially have a day and time set up for my defense, which is fantastic, but means that I'm now reminded of the many things that I wanted to have done before the defense.  
I really want to be able to share my homemade lemon wine with people at the thesis defense, and so that means that I really need to get on it.

Honestly, I don't love the fact that it's a sugar wash, because that feels wrong somehow.  
Mostly it's that sugar doesn't add anything to the final product.  
By its very nature, it is either taken up by the yeast or left as flavor at the end.

It's not that I am some sort of purist who thinks that everything needs to have extra ingredients for the sake of being complicated.  
I love munching on a whole head of lettuce, after all.  
However, I have to wonder if making kilju\footnote{Finnish word for a sugar wine, what most people I can find in the homebrewing community call sugar wines} is really the best bet.  
An advantage is that I know exactly how much sugar is in it, because sugar is 1:1 sugar by weight.  
Another advantage is that it does really let both the yeast profile and the other flavors I add into it\footnote{lemon, hop} shine.  
And, of course, it is dirt cheap.  
I'm nearly positive that sugar is the cheapest source of sugar\footnote{I think it beats apple juice, and if it doesn't that is wild}.

By using sugar, the bottles that I want to get\footnote{read: will be getting} are significantly more costly than the ingredients inside of them.  
That's not necessarily a bad thing, because again, spending for the sake of spending is not ever my goal.  
So why am I so opposed to using sugar?

Partially I am just embarrassed when people ask me how to make the lemon wine, because it is fundamentally just add sugar to water, add yeast, then treat like regular lemonade.  
If I did some berries, that would make it far more of a berry lemonade than a pure lemonade, and I worry that the flavors will get muddied.  
Ok, I feel comfortable with my decision to do a sugar base again.  
Yay!

\section{Daily Reflection}

\begin{enumerate}

\item Did you journal by hand, and do you feel like the stormy questions in your mind got on the page?

I did! The only stormy questions are the blog posts that I absolutely need to get on the page.\footnote{instead of the usual plan of taking a real beak, today's plan is to emotionally wreck myself in the middle of the workday}

\item Did you do your best to sit in still silence?

For a little bit! This links moreso\footnote{a word I will bring into common parlance if it kills me (or others)} to the below, but I did do the silent sitting after posting yesterday, and I do really feel markedly better for having done so.  
Other than that, though, I guess that I wasn't doing a ton of stillness, but that's less because there was a constant rush and more because I completely lost track of time.

\item Are you making sure that each task is given your full attention, not just because the task deserves it, but because you deserve the luxury of doing a single thing at a time?

I generally wasn't multitasking yesterday.  
There were a few things I had at once, like making sure that my code didn't crash while I was reading, and when I was playing a dumb game, listening to some audio that I wanted to get through.  
Otherwise, though, I wasn't even able to listen to a book over the walk home, and wanted to spend time just not doing anything.  
I did scroll through instagram on the entire walk home, though, which is less than great.

\item Are you focusing on your posture and breath?

Generally. I can always do a better job, but each day that I stand with good posture, everything feels better more generally.  
It's also really nice to be able to feel the way my breath feels when I hold it or move the body in different ways.  
My default is always a stomach-led breath, but I'm more and more learning to love the way that my ribcage can expand if I let it.

\item What in your body is holding tension right now? How can you fix it? When will you fix it?

I think that I need to be more aware of just how much tension in my body is interlinked.  
Yesterday while stretching my neck, I felt my entire lower back start to release as well.  
That being said, I still feel like my shoulders are wanting to hunch forward, but that's more posture than tension.  
Because I'm focusing on that, though, that is the place I hold tension.  
Maybe I should look for more shoulder stretches.

I also haven't been able to straighten my arms since at least high school.  
Apparently that's something I can fix primarily by literally just trying to straighten them for a few minutes at a time.  
It really hits the tendons just below my triceps, which I suppose makes sense.

So, I can fix the shoulder tension by stretching the shoulders and by looking up more shoulder stretches.  
I can fix the arms not straightening by straightening my arms.

\item Comments on sleep?

Not really. I slept in an extra hour today, but I think that might be more a lack of motivation than an active need for more sleep.  
A part of me is thinking about whether more shorter sleep sessions might be better?  
Eh I'll try to stick with this routine for at least a little longer before really messing it up.  
It sounds like my cult of personality is in its uprise\footnote{there's a word I'm trying to remember that means something like this but in a more literary way. I can't remember it, though}, because someone is tempted to try out the routine after simply knowing that it's not not working for me.

\item How's eating going? In particular, how are you doing with eating plants and unprocessed food?

Eating wasn't great yesterday. I ended up with a bowl of oats\footnote{plant, relatively unprocessed}, a bunch of peanuts\footnote{plant, somewhat unprocessed}, a couple packs of instant ramen\footnote{... neither}, and some gushers.\footnote{arguably negative for each}

Today I've had some coffee\footnote{it's a ginger cinnamon clove latte, so it's basically nothing, but I'm counting it as food}, and a slice of lemon pound cake.  
After this writing session I might go grocery shopping? Or at some point today maybe.  
What do I need from the store, though? I know that there was something I was thinking of, but I can no longer remember it.

If I cook more beans, I can eat more beans, and beans are a great food for a variety of reasons.

\item Are you neglecting any of your familial obligations? If so, how can you rectify this?

Not really! I'm planning on doing the walk and song time today, because I need something away from computer that is not away from devices.

\item Cleaning: what is the biggest priority you have right now, and what is the next action item for it?

Biggest priority right now: I don't really know. Lots of things are bothering me, but since I think that I'm getting a new chair today and/or tomorrow, I guess that the highest priority is getting the space for it/them cleared.  
That's also the next action item.

Once the chair has home, it's probably good to look at my entire home again and think about what is where, what I don't need and can donate/gift/throw away, and what I need to find space for.  
I don't love keeping my instruments in my apartment over the summer when it's humid and hot, so I might once again occupy space. That might also motivate me to play them more, and also go to work in the office more.  
I don't know if either of those is a real perk, though.

Hmm, when can I dedicate time to just do a deep dive on my apartment?  
I'd like to be able to have friends over, if only one at a time, because there are a few games I have that would be fun for two.

Oh! I also wanted to have homemade lemon wine for my defense. Now that it's scheduled, I do really need to get that prepped and ready.\footnote{OH! That's what was on the grocery list: a bunch of sugar and lemons/lemon juice. Ooh if I did lemons, that would be so so so good. Ok so yeah I guess that I'm making lemon oleo saccharum. If I then combine that with lemon juice, I end up with lemon cordial, and if I then add in some more acid in the form of all the different citric and related acids that I own, I can really make the drink exactly what I want. I do love the hops that I have, so might try titrating it in. I don't quite know best practices for adding hops to country wines, but. Ope ok this footnote is way too long now. Maybe instead of musing about paper as I had planned, today will actually be about  lemon wine. That's definitely where the passion is right now}

\item Thesis: current task. What's preventing you from finishing it? How will you remove that obstacle?

Right now the current task is monitoring the runs to make sure they are not crashing, check the output of the jobs from yesterday, and making figures.  
This ties really closely in to working through the presentation I am giving on Friday.

Ope! I stand corrected.\footnote{I took a travel break after time with others, and during that time period was informed of this fact} My current task is making the list of where my thesis currently stands and an updated timeline.  
That can happen in parallel with monitoring jobs! 27 down, 873 to go!\footnote{they're all running right now, with between 1 and 5 cores. It appears that regardless of number of cores, I really only need like 1 GB of memory, which is pretty nifty! Whole run output is also only like 200 MB, which is only marginally larger than the size of the input data (which it currently is outputting). I am really curious about the run time, but that's for later}

\item Thesis: next task. What will you need to be able to do it?

Finishing the presentation for Friday. Realistically I just need to dedicate time to that.  
I could listen to the book on presenting more and actually start doing the exercises it contains.\footnote{wow the author is passive aggressive about them, but also he isn't wrong}  
Early on in the book it pointed out the importance of speaking presentations, and chunking in thoughts, not lines or sentences.  
Anyways, we'll see what ends up happening.

\item What's the next job you're applying to?\footnote{note that this might be a \say{things we don't post} but}

I made my USjobs account, and, perhaps unsurprisingly, there were no jobs that were even slightly related to what I do.  
With this in mind, I found a cool outreach position in Germany, and so I will start on that app.\footnote{I'm doing a very distracted post today, but honestly I'm not pressed about that}

Nooooo, the job closed! I swore it said it closed end of June, not end of May.  
Oh well, more jobs will come in time.

Huh wild, FBI jobs are not on the regular USjobs website. I wonder if the same is true for other things.  
I'll apply to one of the jobs at my current university that's posted, and know that I'll be rejected from that too.

\item Are you intentionally trying to spend time with others?

Yeah! Well, at the very least I'm writing with a dear friend now and attempted to schedule a lunch with someone yesterday.\footnote{it is happening, but attempt still felt like the right term}  
I should reach out to someone else I care about now.

\item Are you doing your absolute best to ensure that you and those you interact with view the interactions in the same light? Are you sure?

No, but I think that right now all of the unclarity\footnote{obscurity? that's not right, I don't know what the word is right now} that I have is good, if not actively healthy.\footnote{hmm, if not is such an interesting thing, because half the time it seems to suggest the first part is the likely one and the other half, the second part.}

\item Are you keeping up on this daily set of reflection questions?

Yes! Look at this monster.\footnote{some might argue that a 1500 word daily reflection (which is what I have right now on top of the templated parts) is excessive bordering on unreasonable. I have no real comments to that.} I'm wondering if doing this by hand might be better, but I am absolutely finding that the way I think while typing and hand-writing are fundamentally different, which is interesting in and of itself.

\item Are you keeping up on writing the follies? If not, what's in the way?

Yeah! Today I even got a new inspiration halfway through my daily reflection.

\item How are the long form follies coming? Do you feel like they're weighing you down right now?

I have yet to start, and right now I honestly don't think that I'm being weighed down, which is really nice.  
I do still really want to start, though.

\item Are you writing poetry? When, and what were your takeaways from the previous day's writing?

I wrote not a full page, but a full song last night.

Biggest takeaway: my default way to write a song is ballad form\footnote{four feet then three feet per pair of lines, the end of the three feet rhyme in pairs (at least how I do it. I think that the rhyming bit is more optional)}, and if I just let myself iterate over the exact same poem, each attempt becomes markedly better.

\item Are you making music? If not, what is in the way?

No. I still haven't moved my guitar and I just don't really want to make music right now. Not so sure why, but I do feel the mental block.

\item Web novel?

No, but I'm hopeful to make time for it today.

\end{enumerate}

\end{document}

  