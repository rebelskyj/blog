\documentclass[12pt]{article}[titlepage]
\newcommand{\say}[1]{``#1''}
\newcommand{\nsay}[1]{`#1'}
\usepackage{endnotes}
\newcommand{\B}{\backslash{}}
\renewcommand{\,}{\textsuperscript{,}}
\newcommand{\foott}[2]{\endnotemark[#1]\endnotetext[#1]{#2}}
\usepackage{setspace}
\usepackage{tipa}
\usepackage{hyperref}
\usepackage{nested}
\begin{document}
\doublespacing
\section{\href{flash-fiction-234.html}{Flash Fiction Friday}}
First Published: 2024 January 5

\section{Draft 1}
\endnoteversion[a]

Another Friday, which means that it's time for another musing on Flash Fiction Friday.
This week's prompt is a little late, in my opinion, but is \say{how it ends}.
There is a lot that I can do with that prompt.

Given the \href{album-update-24-01-03.html}{album work I'm doing}, something on the topic of star death or heat death of the universe is tempting.
Of course, the idea of writing a villanelle inspired by \href{https://www.poetryfoundation.org/poems/46569/do-not-go-gentle-into-that-good-night}{do not go gentle into that good night} remains something that I want to do.
Villanelle remains a poetic form that I am in love with, if only because there's something really inherently intense in the form.
Something from \href{https://www.poetry.com/poem/114806/the-hollow-men}{the hollow men} also seems tempting, though that is just because the text is so, so very into the public consciousness.

That being said\foott{1}{for my reader who appreciates the turn of phrase}, I do think that there's something good about looking far smaller.
Rather than taking the death of everything and animating it, it could be fun to turn the end of something small into a larger picture.
Things that end include relationships, classes, friendships, lives.

I guess one immediate question is whether I want this to be poetry or prose.
I haven't been writing a lot of poetry lately,\foott{2}{Ok as soon as I typed that I remembered that I did, in fact, write more than 20 sonnets last month.
That was a whole year ago though, so it doesn't really count, in my mind at least} and I do want to get into song writing.
Maybe trying to do something that's more elevated prose?\foott{3}{Then again, the last time I was proud of the prose I wrote my least interacted with fiction ever, so maybe that isn't the best idea}

I could try doing a villanelle inspired piece of prose?
No, I don't think that would work.
What makes a villanelle so compelling is the fact that it is only nineteen lines and full of repetition.

Having now come back about an hour later, I find that I'm inspired also by a song that my band\foott{4}{it always feels weird to say my band, but that is far fewer syllables than the band that I played with, back when we still did that} played \say{Til Forever Falls Apart}\foott{5}{No, I will not be taking questions on why I capitalize song titles but not poem titles}
That's a song about some nebulous end, but I think that it's about death?
Or maybe the potential of death.
The group also did \say{I Will Follow You Into the Dark} at the same time, and I get those two songs stuck in my head together.\foott{6}{Mashup potential?}
That song is more explicitly about a murder suicide pact, which is its own form of romance.

So, based on my musing so far, it does really seem as though my muse wants me to be writing poetry, probably about death and related to romance.
Since I don't think that the album will be about romance,\foott{7}{Except maybe incidentally in like a single song. There's always a chance that I am wrong, but it does really feel like that's where my muse is pulling me} I don't have to worry about putting out this set of lyric.
Since it's going to be anything I want, I want a villanelle.

That's AbA', abA, abA', abA, abA', abAA'.

That is, even though the entire poem is nineteen lines long, I only have to generate 13 lines of rhyming meter, five in the b rhyme and seven in the a.
I will have to break myself of the habit of wanting perfectly strict meter here, because villanelles sound way too sing songy when written in meter.\foott{8}{I'm sure that a better poet than I has managed to make it work, but I am not that poet right now}
Nothing left to do but try to write it, I suppose. Given that the flash fiction minimum is 100 words, I do need to have lines that average out to at least 5 words a piece, which is hardly a struggle usually.

Having now written a draft and sent it to a fantastic and eloquent friend, the friend mentioned no small number of issues with the poem.
Chief among them, I chose an incredibly trite refrain, and I did not think of any meter at all.
Now almost two hours later, however, I have a completely different poem that is miles away better than where I started.
All of the improvement is due to the fact that my friend patiently led me to water, then, when I was about to collapse from thirst, reminded me it was there.\foott{8}{I think I might have lost the thread of the metaphor}

There's really not much more for me to talk about, so that's nice.
Goodnight all!

\begin{itemize}
\item Blog? This is a little more disjoint than I would like
\item BiaY and CCCiaY? See yesterday's answer
\item Jeb? See yesterday's answer.
\item Album? Oh gosh I learned that I am not prepared to write the lyrics to my song
\item Exercise? I did not the bare minimum, but I did some light cardio for like an hour!\foott{9}{bare minimum is the squats, pushups, and plank}
\item Stretch? Just a tiny bit.
\item Book on Craft? A little! I also bought a book light which means I can read it tonight.
\item Library Books? I saw it again.
\item Guitar? I saw it, moved it, and did not touch it.
\item Prayed? Eh!
\item Water? I just remembered that I'm sitting next to a water bottle
\end{itemize}

\endnotes
\end{document}