\documentclass[12pt]{article}[titlepage]
\newcommand{\say}[1]{``#1''}
\newcommand{\nsay}[1]{`#1'}
\usepackage{endnotes}
\newcommand{\1}{\={a}}
\newcommand{\2}{\={e}}
\newcommand{\3}{\={\i}}
\newcommand{\4}{\=o}
\newcommand{\5}{\=u}
\newcommand{\6}{\={A}}
\newcommand{\B}{\backslash{}}
\renewcommand{\,}{\textsuperscript{,}}
\usepackage{setspace}
\usepackage{tipa}
\usepackage{hyperref}
\begin{document}
\doublespacing
\section{\href{toffee.html}{Toffee}}
First Published: 2022 January 15

\section{Draft 1}
I've \href{toffee-recipe.html}{written before} about how I make toffee, but I made it again for the first time in a long while today, and that seemed appropriate to blog about here.
As before, the recipe is 1:1 sugar and butter, heated to hot then cooled.
This time, however, as I have \href{candied-orange.html}{too much chocolate}, I decided to make the toffee the way most of the internet suggests I should: chocolate covered.
It came out well!

I guess it's unsurprising that a recipe which\footnote{allegedly, I did not measure the chocolate} is just 1 part chocolate, sugar, and butter, could come out well.
Nonetheless, it was really fun to see that I can, in fact, make things taste good.
I unfortunately didn't let my chocolate cool enough before mixing it into the cookie dough but that's life I suppose, I'll have moreso chocolate than chocolate chip cookie.
\end{document}