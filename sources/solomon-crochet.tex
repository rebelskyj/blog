\documentclass[12pt]{article}[titlepage]
\newcommand{\say}[1]{``#1''}
\newcommand{\nsay}[1]{`#1'}
\usepackage{endnotes}
\newcommand{\1}{\={a}}
\newcommand{\2}{\={e}}
\newcommand{\3}{\={\i}}
\newcommand{\4}{\=o}
\newcommand{\5}{\=u}
\newcommand{\6}{\={A}}
\newcommand{\B}{\backslash{}}
\renewcommand{\,}{\textsuperscript{,}}
\usepackage{setspace}
\usepackage{tipa}
\usepackage{hyperref}
\begin{document}
\doublespacing
\section{\href{solomon-crochet.html}{Solomon Stich}}
First Published: 2022 July 4
\section{Draft 1}
Today I went to a crafting event with some friends.
I knew that I needed a project, but wasn't sure what to work on.
Recently I've realized how limiting the way I know how to crochet is.

Don't get me wrong, most of what I crochet is meant to be warm and thick, so the fabric being so is a plus.
But, as these summer months go on, I'm beginning to understand the appeal of looser fabrics.
So, I decided to look for an easy project that would force me to use a new stitch.

Enter the Solomon Knot.\footnote{Or Lover's Knot, Knot Stitch, True Lover's Stitch, etc.}
Though I cannot find a reason for the name, it's functionally just a very loose set of single crochets with skips.
It took me a bit to get into the flow of the stitch,\footnote{partially because I'm using really bad yarn}, but now that I have I really enjoy it.

I like how I can decide exactly how large the holes are, though that really adds another level to keeping consistency in my stitch.
In the Solomon Stitch, you not only have to keep string tension the same, but also keep the loops you make the same size.
I'm making a bag out of this stitch, which I likely won't want or keep, but I've been surprised by myself before.

I've really forgotten how much I love crocheting, so it was great to spend time with friends and learn the knot.
\end{document}