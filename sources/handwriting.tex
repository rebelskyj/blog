\documentclass[12pt]{article}[titlepage]
\newcommand{\say}[1]{``#1''}
\newcommand{\nsay}[1]{`#1'}
\usepackage{endnotes}
\newcommand{\1}{\={a}}
\newcommand{\2}{\={e}}
\newcommand{\3}{\={\i}}
\newcommand{\4}{\=o}
\newcommand{\5}{\=u}
\newcommand{\6}{\={A}}
\newcommand{\B}{\backslash{}}
\renewcommand{\,}{\textsuperscript{,}}
\usepackage{setspace}
\usepackage{tipa}
\usepackage{hyperref}
\begin{document}
\doublespacing
\section{\href{handwriting.html}{On Handwriting}}
First Published: 2022 September 3

\section{Draft 1}
As I mentioned \href{reflection-august-2022.html}{in my monthly reflection}, I finished a journal that I'd been working on for a while.
Since I finished it, I went back to look at the other journals I'd written.
One of my major goals, back when I started filling these journals, was to improve my handwriting.

I still remember when I was in sixth grade and a teacher was completely dissatisfied with my handwriting.
She told me to come back to the classroom after school, where she would fix my penmanship.
A few minutes into the practice, she gave up and told me I should go to become a doctor.\footnote{that part I remember slightly less well, but it was definitely a joke, just whether about how common computers were or that doctors have bad handwriting I no longer recall.}

Other comments like that followed me through most of my schooling.
So, when I went abroad\footnote{coincidentally when I started this blog}, I decided that I could work on my penmanship.
It's really fun to look and see how each of the four notebooks looks notably\footnote{hah} different than the others in terms of writing.
The speed at which writing changes has slowed down for sure, even ignoring the fact that each journal lasted longer than the one before.

Still, since starting this journey to good handwriting, compliments of my penmanship happen fairly often, which still feels really nice.
I think a large reason I get compliments is that I have a very distinct pen.\footnote{is pen the term there? I'll assume so}
Since I was teaching myself to write well as a college student, there was almost nothing which restricted what I can do.

So, I write in all capitals, with the first letter of each word larger.
Double letters share everything but a vertical stroke if possible\footnote{so not in c,s, sometimes e}, and a's are written as a caret with a diagonal line from the bottom right to halfway up the left stroke.
The only reason a's are shaped like that for me is that a professor my senior year told me that he couldn't read my caret a's, and I needed some sort of horizontal stroke.
This seemed to work as a compromise.
For the newest journal, I decided to add a new trait to the handwriting, so now I have the top stroke of some letters\footnote{t,i,s,c,g,j,f still unsure of b,p,r} extend across the entire word.

All this to say, I like liking how my handwriting works.
\end{document}