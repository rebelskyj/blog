\documentclass[12pt]{article}[titlepage]
\newcommand{\say}[1]{``#1''}
\newcommand{\nsay}[1]{`#1'}
\usepackage{endnotes}
\newcommand{\1}{\={a}}
\newcommand{\2}{\={e}}
\newcommand{\3}{\={\i}}
\newcommand{\4}{\=o}
\newcommand{\5}{\=u}
\newcommand{\6}{\={A}}
\newcommand{\B}{\backslash{}}
\renewcommand{\,}{\textsuperscript{,}}
\usepackage{setspace}
\usepackage{tipa}
\usepackage{hyperref}
\begin{document}
\doublespacing
\section{\href{reflections-on-readings-24-ordinary-c-22.html}{Reflections on Today's Gospel}}
First Published: 2022 September 11

Exodus 32:9 \say{I have seen this people, how stiff-necked they are, continued the LORD to Moses. }

\section{Draft 1}
This week is one of the prodigal son readings.
The priest gave a really good homily that I'm going to shamelessly steal large portions of today's reflection from.

First, the context for the parable is clear in today's Gospel.
Christ is speaking to sinners and scribes alike today.
He begins with a parable about a herdsman with sheep.

Of course, losing a single sheep temporarily would be an everyday occurrence, so it would not be worth rejoicing over to find a lost sheep to your neighbors and friends.
Yet, the Lord rejoices in our return.
Again with the coin, finding a lost coin is not cause for an average person's rejoicing to the community.

Then we get to the parable that my brother hates and I always loved: the prodigal son.
Something the priest pointed out today that I'd never noticed, the second son doesn't inherit.
And yet, when he asks for an inheritance, his father loves him so much that he gives to his son something he does not deserve.

Too, our Lord gives us so many gifts that we in no way deserve or could ever hope to earn.
When he returns contrite, the father dresses him in fine clothes, symbolizing that he is the favored son.
Seeing this, it's easy to understand why my brother dislikes the reading.
Why does the elder son, who has been nothing but dutiful, not receive more than the sinful brother?

Again, this was being addressed to both Pharisees and tax collectors.
The message of the parable is two-fold.
First, for the sinner and the lost, it's a reminder that no matter how low we are, how depraved our sin, there is nothing the Lord rejoices in more than our return to Him.
Second, for the legalistic, the Lord's Love and Justice are not man's justice.

Something that a friend and I were discussing once were the two concepts of mercy and justice.
He saw them as two distinct pieces the Lord gives out, where justice is sometimes withheld for sake of mercy.
I disagreed with that then, and disagree with it now.
There is no weighing scales in the Lord's mercy and justice.
Rather, His justice is mercy.

When we repent, the evils we have committed against the Lord are gone.
The Sacrament of Reconciliation truly frees us from what is otherwise an unbreakable cycle of failures.
And yet, that's not how the temporal world works, or maybe even should work by some interpretations.

Certainly, it was wrong for the younger son to demand an inheritance he did not deserve and spend it on a life of sin.
And yet, his father saw him from a distance and ran to embrace him.
Even after being told that his son wanted nothing to do with him, his father still held out hope for his return.
When a sinner repents, the choirs of angels and saints rejoice with the Lord, for one more child has come home
\end{document}