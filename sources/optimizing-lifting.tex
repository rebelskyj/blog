\documentclass[12pt]{article}  
\newcommand{\say}[1]{``#1''}  
\newcommand{\nsay}[1]{`#1'}  
\usepackage{endnotes}  
\newcommand{\B}{\backslash{}}  
\renewcommand{\,}{\textsuperscript{,}}  
\usepackage{setspace}  
\usepackage{tipa}  
\usepackage{hyperref}  
\begin{document}  
\doublespacing  
\section{\href{optimizing-lifting.html}{On the Optimal Lifting Plan}}  
First Published: 2025 April 12

N.B. I know this says six drafts.  
Really what it means is that I spent a lot of today just figuring out how I wanted to frame the advice.  
The advice is simple: \say{lift weight}.  
The meaning it has to me, however, applies far more broadly, and I wanted to find the framing which best expressed my thoughts, to say nothing of finding out exactly what my thoughts were.  
As a result, the drafts circle the same content, but often approach it from vastly different angles.

\section{Draft 6: 12 April}

Maybe because of how I look or act, I have been asked more than once for advice on getting into lifting.  
Probably because I talk about music a lot, I've been asked about learning music a fair number of times.  
In both cases, I've imagined being asked far more than I have been asked, and yet I don't know if I've ever given the correct advice.

The best way to start lifting is to find something heavy, pick it up, and then set it down.  
After repeating this exercise as many times as one can, record the number, and walk away.  
The following day, repeat.

What anyone who has never lifted before will almost certainly learn is that the number of repetitions they can do will increase steadily for a while.\footnote{ignoring the noise of day to day fluctuations in ability, of course}  
Could they potentially grow that number more quickly by hyper targeting specific muscle groups or even specific muscles?

Maybe!

I don't really think so, though.  
Or, at least, the extra time that it takes to find out exactly what muscles can be targeted by what lift, learning the proper form to isolate the muscle group without harming oneself, and then setting up the specific lifts is almost certainly better spent simply lifting.  
When progress slows, or they notice something in particular holding them back, it's more than sensible to start crafting a more in depth routine.

If someone asked me how to learn guitar, I'd ask if there are any classic rock songs that they can listen to all day long.  
Why classic rock, and why does it need to be repeatable?  
Most classic rock songs have a pretty simple guitar pattern, and the guitar in them tends to be a major enough part of the orchestration that practicing it alone will still give the effect.

If they have a song, I'd tell them to learn the chords for it, and then just play until their fingers hurt day after day.  
If they don't, I'd tell them to do anything as long as they're fretting\footnote{not mentally, but on the board} notes until their fingers hurt.  
Having a metronome is only helpful when one can play notes long enough to correct themselves.  
Getting proper tone requires being able to hold fingers in the proper way, which requires nothing so much as practice.

Could they learn bad habits by practicing like this?

Absolutely!

However, if they play guitar daily to exhaustion, they can get sufficient calluses to play for a long time within a few weeks.  
Any bad habits that can be built up in that time can also be unlearned.  
And, a few weeks is enough time to know if one wants to become a person who plays guitar.  
If one is, then finding instruction becomes useful.

If you can play notes for fifteen minutes at a time, then spending five of them making sure that you seamlessly switch between chords is time well spent.  
Spending time with a metronome, if you are like any standard musician, will always pay great dividends.  
However, both require you be able to play.

That is the other part of why the advice is so good.  
Doing things is always harder than not doing them.  
Learning to lift or play guitar or anything else is easy to fantasize about.  
Realizing that, if you want to be able to improvise a killer metal riff, you need to know your scales on a fundamental level is far less fantastic.  
Before trying the grunt work of an activity, one can never know how they feel about it.\footnote{this isn't the \say{how do you know you don't like X}. If you have never eaten something green, then yeah you should try lettuce. If you know that you dislike kale, broccoli, and cauliflower, then cabbage probably won't be for you.}

\section{Draft 5: 12 April}

The best piece of advice for figuring out a lifting routine I've ever gotten was: \say{put your computer away, find something heavy, and pick it up.  
Set it back down, then lift it again.}

If I want to improve at a skill, I must do the skill.  
Until I am nearing the peak of my ability in something, almost any practice will improve me.  
The further from the peak I am, the less difference any practice will make.

For whatever reason, music remains the skill that I can best relate to.\footnote{ok, specifically, learning to play a specific instrument when growing up in an encultured way where they understand music but consider it something that only some do}  
If someone has never played an instrument, and they start just strumming a guitar aimlessly, they will become better at it.  
If they tried to follow the routine of an expert, they would almost certainly quit immediately.  
They lack the calluses on their fingers, the unconscious competence of doing many things at once, and the love of the instrument.

Similarly, if someone is newly embarking on a lifting journey, the most important thing is that the lifts they do will not injure them.  
As far as I know, almost every adult has picked something up without injuring themselves.  
They have likely even picked up something they consider heavy.

Our muscles grow when they are challenged.  
Picking up something heavy day after day will gradually make it feel lighter.

A fitness plan which hypertargets individual muscles, or even one which targets broad muscle groups, is not as useful as simply lifting.  
The more time we spend planning, the less time we spend doing, after all.  
If someone finds that after lifting something heavy for a few weeks, they're starting to see their progress slow, then starting to atomize the workouts can become useful.

But, I think that I need to remind myself more and more, when trying to learn a skill, I should stop looking up how to do it and start doing it.  
As my progress wanes, I can learn from the masters, but I need to be able to love the instrument for itself first.  
If I need my fingers to be callused, the calluses will grow just as fast from scales as from playing twelve bar blues as from noodling about.  
Ok this is good but should really be a slightly different framing.

\section{Draft 4: 12 April}  
What is the optimal lifting routine?

It is not some specific set of lifts.  
Nor is it some specific set of focused and targeted groups of lifts.  
The optimal lifting routine is the routine that makes you most able to do the motions in life that you want to do.  
This requires three things: knowing what one wants to do, knowing how lifting can help with that, and actually lifting.\footnote{In many regards, this is similar to the way algorithms work, which makes sense because my mind is one track in the regard of never truly letting go of research}

What do I want to do?

In short, I want to be unbound.  
Of course, there are any number of physical, mental, social, societal\footnote{my subgroup is far outside the societal norm}, and emotional limits that preclude many actions.\footnote{And, I am of course aware of the fact that failing to limit myself in one way is itself limiting.  
That is, committing to any particular path of action precludes doing any other act.  
However, failing to commit to a path closes off any other act just as much.  
By going for my Ph.D., I cannot travel the world and explore the unique art styles, learning how they were influenced by and influenced the peoples who made them.  
If I had chosen to do neither, though, I would not have done either.  
Basically, commitment is needed. (P.S. this was not initially a footnote, but I woudl really actually like to have a nice clean main text here)}  
That doesn't change my desire to be able to do what I will.\footnote{and, of course, what I think that the Almighty wills. Ideally these are one and the same, practically, who can say?}

How does lifting help me do what I want?

Since I want to be able to effect my will onto the world, any lifting plan I pursue should help me with this goal.  
My will is rarely to be in the gym for its own sake\footnote{ah gotta love considering means and ends a bunch. I get more and more why people say that philosophers should go touch grass}, and so a plan which has me spending less time is probably preferable to one which has me spending more time.\footnote{I'm not main texting this but, I am better able to get people to do what I want when I am more conventionally attractive.  
Lifting can make my body more toned and my muscles larger.}  
If my body fails me, then, as many say, \say{the spirit is willing but the body is weak}.  
More often than not, my body being the impediment to my will comes from endurance, rather than brute force.  
That being said, I am sure that I would be less constrained if I were better able to move weight.

How can I do these lifts?

Eh, basically what I've said here is that I don't need to lift, I need to do cardio.  
This isn't the lifting advice from above, though, which is fine.  
Final draft should be that though, so one last chance.

\section{Draft 3.1: 12 April}

What is the optimal lifting routine?

There are any number of answers to this question.  
Some swear by the weekly routine of \say{push, pull, legs}, others by some constantly evolving set of lifts, doing dozens at a time.  
However, there should only really be one correct answer: the optimal lifting routine is the one which most frees you.\footnote{ok so that's a framing I should really interrogate myself about. I wanted to have some answer, but then remembered that all things should lead to sanctification, then had makes you strongest, but even that is a vague term}

\section{Draft 3: 12 April}

Since the Industrial Revolution, society has told us that we are interchangeable cogs in a greater machine.  
As a partial result, advice towards self betterment tends to focus on man as machine.  
We can and should seek to improve anything by working on atomistic parts.  
If I want to be stronger, that means that I need better biceps, triceps, quadriceps, abdominals, etc.

And, of course, this is not entirely incorrect advice.  
There are many things that we can treat atomistically.  
If my nose is clogged, I can blow it.

A greater consequence of the Industrial Revolution was the enshrinement of schedules.  
A week has no fundamental meaning to the world.  
Unlike the day, which, prior to artificial light's conquering of the world, controlled when we could see, the week does not have any fixed meaning.  
We chose seven days, in part, because it has some relation as division to the lunar month.

It takes about a fortnight to go from new moon to full moon and from full moon to new moon.  
However, this desyncs very quickly.  
One can tell this by simply looking at a calendar which lists the full moons.

And yet, almost all workout routines I've seen are based on the idea that we should repeat some set of workouts weekly, or at least some general kind of lifts.  
Since we must live in this society, making our workouts work around the weekly schedules we must have is not the worst idea.  
Still, why do we break things into push, pull, legs?

Spending any time talking to physical trainers or reading up on the literature shows that these lifting routines, while often optimized to improve our ability to do the lifts at higher and higher weights, are often far from optimized in terms of getting us to be able to live a better life.  
And so, we get to the truly optimal lifting plan: pick up an object and set it back down.

Exercises which target a specific muscle group are fantastic for ego lifting.  
For functional strength, though, they are only really needed in rare circumstances.  
After all, I do not know a single time that I have needed to isolate a single leg's calf muscles while lifting or moving.  
Training one's body to lift heavy weights with good motor form means that the behavior becomes more innate.

And, of course, only by doing something can we learn how we feel doing it.  
I know now that, while running may be a healthy exercise, I don't really like it at all.  
I really enjoy yoga and most meditative forms of workouts.  
I even prefer core workouts\footnote{especially when I can actually do them} to running.  
Almost all benefits that any given workout has are shared amongst any workout.

Hmm might be a better framing

\section{Draft 2: 12 April}

One of the best pieces of workout advice I ever received was about how to start a lifting program:

\say{Pick the weight up.\\ Set it down.\\ Repeat.\\ Grab a heavier weight when you stop feeling the pain}

Now, why is this great advice?

For me, at least, the primary goal of lifting is to be able to lift heavy things when I'm in my day to day life.  
By reframing the entire lifting plan into just getting weights moved, I can remember that, when choosing between the lift that looks cool or the lift that looks ridiculous but will help me more, I should pick the latter.

I have had lifting plans for ages, and so I know the proper form for a lot of lifts, how slight modifications to them can focus on different muscle groups, and what lifts generally target what things.  
However, I also have done far more lifting than most people I know, and even I fear doing the appropriate weight to train back squats to failure.\footnote{I just fear failing that lift orders of magnitude more than any other lift}  
Why do people back squat?

That's a legitimate question.

As far as I can tell, it's because squatting is generally a motion people do a lot, and should do in a lot of settings.  
It's almost always healthier to pick up something heavy by squatting first, rather than just bending over or doing a lift which focuses on one's back.  
Still, that doesn't mean that everyone needs to do them, despite what almost every lifting plan will suggest.

In fact, almost none of the most popular lifts that people do are really the best for most people, at least as far as I've seen.  
We squat and bench and do bicep curls because they are what we have been told to do, and because, when doing an incredibly high level program, they are among the more effective for getting the last optimizations out of a lifting plan.  
For most of us, any lifting plan is an increase.

In general, advice since the Industrial Revolution has treated people as though they are interchangeable cogs, and as though the week is a fundamentally meaningful division of time.  
Oh that's a much better framing, let's start over with that.

\section{Draft 1: 12 April}

I saw something really recently\footnote{well after planning this post} that struck me pretty deeply.  
I don't remember exactly what it said\footnote{and no, there is no real correlation within myself between what I remember and how important it is to me, and yes, I do really wish that there was.}, but the general gist\footnote{why do I feel like jist should also be acceptable? probably because jel and gel are both words that I've used in the NYtimes crossword} of it was that we cannot live optimally.  
No matter how hard we try, we will have days that we do nothing, days where we will not be the shining and perfect light to the world.  
For some reason, that was really helpful to me, and, despite the fact that it's sort of opposite this post, it felt relevant to bring up here.

I forget where I initially saw the best lifting plan I've ever seen, but I think that it was probably a screencap of some old greentext.\footnote{if you don't know what greentext is, your life is better for it}  
It is equally probable that I saw it with some random video..  
Perhaps unsurprisingly, I consume a lot content that is at least somewhat adjacent to life optimization.  
In either case, the advice was \say{just pick up something heavy and then put it down. Repeat that, lifting heavier things when it gets too easy}.

Why is this the optimal weight lifting plan?

First,\footnote{oof I love lists apparently} it's important to remember why we are lifting.  
For me, the most important benefit lifting gives me is the ability to pick up heavy objects in my day to day life, ideally without showing any struggle.\footnote{yes, I do have an issue where I hate showing any forms of weakness how did you know}  
Next in importance is the fact that it's generally allegedly healthy to lift weights.  
I like looking well muscled, and lifting does help with that.  
And finally, I like showing people up in the gym.\footnote{absolutely my least healthy trait, I know}

I would like to think that at least the first three of these are pretty common goals for most people who want to get into lifting.  
So, why is it better to just pick up weights, rather than finding some optimal program?  
For me, any efforts that I spend towards optimization feel similar to actually working on a problem.  
I know that I am not alone in this sentiment, and so saying \say{the weight goes up and the weight goes down} means that I have nothing else to look up.

Second, focusing lifting on the idea of picking up and setting down progressively heavier weights means that one can remove some of the stigma behind what workout plan we might have.  
If we start with a low weight, we can make sure that we lift with proper form before raising the weight.  
If, instead, we were to start with bench press as the specific lift of the day, we might focus on just getting a heavy weight.

And, in general, I think that this is a good advice for much of life.  
If I'm wanting to learn something or do something, I should just do it.\footnote{obviously with exceptions}

\section{Daily Notes}

\begin{itemize}

\item Obligations:

\begin{itemize}

\item Professional

\begin{itemize}

\item Write the thesis

I'm debating whether or not I should be working on this this weekend.  
On the side of yes: I should always be working, and I do actually feel some level of motivation.  
On the side of no, I'm visiting home, told myself that I would be taking a break, and am generally feeling low motivation while at work.

\item Revise the thesis

\item Edit the thesis

\item Research for the thesis

I think that I found a textbook that actually explicitly lists the matrix formation that I need.  
However, I'm more and more realizing that I probably do, in reality, need to solve the matrix myself, which does require understanding the basis function.  
That won't be fun, even if it is important.

Boss also thinks it's good to list out what assumptions each thing makes.

Also, there have in fact, been advances in the field since the 1960s, so I should really read the newer theory papers, if only because they might be more understandable in terms of the math they use.\footnote{it remains a bother to me that the brits say maths}

\item Read the books that might be useful for the thesis

See above for the book I'm to read.  
Also, since one of the underclassmen is preparing for a preliminary exam, I had an impromptu lecture to the group yesterday about representations and reductions.  
I'm not entirely sure if what I said was accurate or helpful, so wow do I need to get that in more detail.  
On the plus side, I noticed one major error with what I was doing in my fancy new code, so we'll see if fixing that fixes the issue.

\item Start citation tracking

Continuing to load papers into the citation manager.  
I should start making annotated bibliography, because that's something that many people use for some reason.\footnote{why? Eh could be a fun appendix \say{here's the list of things that someone learning rotational spectroscopy should actually read and what they should get from it}, since like each textbook cites at least 500 different articles (and a shocking number of other textbooks, gotta love that the research in the field actually happens and is published in textbooks (or at least did, back in the good old days when the field was being actively advanced)). I've definitely found that each of them has complementary information, and I would save myself a headache by noting down what each one has at different points.}

\end{itemize}

\item Personal

\begin{itemize}

\item Learn the songs for to jam

\end{itemize}

\item Self:

\begin{itemize}

\item Silence

None at all, but I also don't know how good silence is for driving.  
As I type that, I realize that the answer is \say{incredibly}, but I still don't really love it.

I'm also realizing that GameLit audio books tend to be narrated with the assumption that you'll listen at well above single speed, if the fact that I can listen to other content at single speed far less painfully is any indication.  
Also, pauses which sound stilted and unnatural at single speed feel almost correct at the maximum speed.  
Then again, since many consumers in the field also buy books by length\footnote{and I am beginning to be one of them, much to my own (only now realized) dismay}, there might also just be the motivation to read as slowly as possible.

\item Typing practice.

Oops! Forgot to do that yesterday. I'm sitting in a really awkward position right now\footnote{read, functionally laying down with my head propped up}, but if I can type like this for writing, I can type like this for learning to type. I'm already able to feel how much faster and smoother my typing is, especially since I'm making far fewer mistakes.  
Punctuation is still awkward for me, which makes sense, since none of the drills I've been doing have punctuation in them.  
All this to say, I'll go practice typing once i finish this reflection.\footnote{and therefore before I start on the blog of the day}

\item Keep the phone out of the room for bed

\item Pray St. Michael Chaplet in the morning

\item Stretch in the morning

\item Read at night

I did! It wasn't the book that I told myself to read, but I restarted the book series that I wanted to read.  
I forgot how incredibly bleak it is, wow.

\item Poetry at night

\item Clean the home

\item Stretching, standing, drinking water

\item Posture

\item No wasted time

I think that I did ok with this, especially because I'm doing less scrolling and more reading.  
Then again, I have spent probably three hours in the past day and a half looking at pens and pen inks, and I do not need more of either right now.\footnote{especially since I have an order coming in soon}

\item Eat more than 2 meals a day

I had cake, lunch, and dinner last night! Woo, go me.

\end{itemize}

\end{itemize}

\item Goals and Growth:

\begin{itemize}

\item Ends:

\begin{itemize}

\item Letter writing, get into more

Nope!

\item Handwriting, pick and make the new one

Realized today that yes, I do really want to do cursive.  
That's great, but does mean that I have to really make an effort if I want my hand\footnote{script? Is that what people say?} to be distinct and pretty and legible.\footnote{I know that isn't the same order that I listed before}

\end{itemize}

\item Means:

\begin{itemize}

\item Typing speed, improve it.

Soon!

Wow that was fun, I didn't realize that there was a speed chart in the settings which shows how fast each letter is.  
Turns out A and N are my fastest, both at over 7 characters per second, which is kind of wild to me. A I guess makes sense, since I don't use that many z or q, but the fact that N is so fast when none of the other right index letters\footnote{also, how did I just now notice that I type p with my pinkie? Also, am I supposed to use both shifts, because I only use left shift. Probably something to ask someone at some point} are anywhere near as fast is strange to me.  
I don't know if I made much progress today, but looking at the graph of all my speeds, I do need to remind myself that only recently did I start focusing on form and making sure that each letter is pressed by the correct key with minimal errors.  
I'm positive that slows me down, and that's ok for now, especially since it's in practice time.

Does it carry over to my actual typing?

Certainly at least a little! I definitely need to look at the keyboard less right now, though wow is it hard to type with my eyes closed.\footnote{shockingly, that entire sentence had only a single error in it, even though I know that I did mispress a few keys. Oh well. Might be good practice for me to write with my eyes closed more, so that I have to really focus on making sure that each letter is being typed by the correct finger.}

\item Reading, do more of it

Why is reading a means and not an end?  
Oh right, because I really meant that I wanted to read the research books.  
Well, I brought the books on etiquette home, so I can at least start on those!

\item Blogging, do it

Look at this! I need to make sure that I have both low and high effort posts\footnote{that might not be the right framing. Deep and shallow? Basically things that I can really dig into and things that I can quickly get my thoughts on the page.  
First is good for revising and structuring thesis and writing as a whole.  
Second is good because I want to be able to craft something compelling without having to revise and revise and revise.  
It does mean that the shallow posts might have to be written slightly less stream of consciousness, but that is a sacrifice that I am willing to make}

\item Writing things that are not the blog and thesis, do

I cannot believe that I've written almost one thousand words in the past fifteen minutes, but wow.  
I did spend a few hours today going through literally every single ink seller in the country, and found my new favorite brand.\footnote{I am a simple, simple man. Also, I'm nearly positive that at some point in the not far off future, I am going to end up making my own inks from scratch}

\end{itemize}

\end{itemize}

\end{itemize}

\end{document}