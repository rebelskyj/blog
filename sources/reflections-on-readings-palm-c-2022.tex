\documentclass[12pt]{article}[titlepage]
\newcommand{\say}[1]{``#1''}
\newcommand{\nsay}[1]{`#1'}
\usepackage{endnotes}
\newcommand{\1}{\={a}}
\newcommand{\2}{\={e}}
\newcommand{\3}{\={\i}}
\newcommand{\4}{\=o}
\newcommand{\5}{\=u}
\newcommand{\6}{\={A}}
\newcommand{\B}{\backslash{}}
\renewcommand{\,}{\textsuperscript{,}}
\usepackage{setspace}
\usepackage{tipa}
\usepackage{hyperref}
\begin{document}
\doublespacing
\section{\href{reflections-on-readings-palm-c-2022.html}{Reflections on Today's Gospel}}
First Published: 2022 April 10

Philippians 2:9 \say{Because of this, God greatly exalted him
and bestowed on him the name
that is above every name.}

\section{Draft 1}
Today marks the final Sunday of Lent, and what is apparently in the top four most attended Masses in the year.
Today we celebrate both the arrival of our Lord to Jerusalem, and also mourn his death.
The congregation I was a part of had a part to speak in the Gospel.

We played the role of the disciples and the random members of the community.
The lines that we speak were coursing through my mind today, though differently before and after the Mass.
Before the Mass, I thought about how we shout \say{Crucify him}\footnote{Luke 23:21B}, because our sins are what led Christ to be crucified.
After the Mass, though I have been reflecting on how we also say that we have lacked for nothing from the Lord.
That line stays with me more, because the idea that we are sinful and Christ died for our sins is not something that I'm currently struggling with in my faith.

The idea that, despite all of our faults, and despite what common sense would tell us, we can, in fact, rely solely on the goodness of the Lord to protect us and preserve us is something that I'm struggling with.
Even those closest to him, who he had just eaten with, and who he founded his Church on betrayed and abandoned him, just as we do when we sin.
Still, the Lord forgave Peter and the Lord forgives us.

Another image from the Gospel that struck me today was the image of Jesus praying in the garden.
It's the first of the Sorrowful Mysteries of the Rosary, which is the set of Mysteries I've found I can pray the most easily.
The line that struck me today was \say{He was in such agony and he prayed so fervently that his sweat became like drops of blood falling on the ground.}\footnote{Luke 22:44}
What does it mean to have sweat like drops of blood?

Is it that the sorrow he felt was so piercing and distressing that the drops of sweat fell as painfully as a dripping wound?

Is it that the sweat itself was bloody and red?

Is it that the sorrow he felt now on our behalves was going to be mirrored soon in the blood of the Cross?

Ultimately I don't know, but all of the three feel like plausible enough explanations to me.

As I write this musing, though, I am struck by the two verses immediately before.
Our Lord prays that he might be spared the trial he is to undergo, and an angel appears to strengthen Him.
I don't know if I've ever remembered that verse before.
Normally, I've seen the verse used solely to remind us that even in the midst of suffering, the human person of Jesus was fully subservient to the Divine Will.
The inclusion of the second line, though, really speaks to me.
Even the Lord, who had perfect knowledge of his eventual resurrection and ascension, found solace and comfort from an angel.
How are we to live through the struggles in our own lives if we refuse to be comforted?

That's certainly something I'm thinking on right now.
An angel came to strengthen Him, because the Lord will always support us in doing His will.
That doesn't make the trial any easier, though.

Anyways, this is a long and rambling way for me to say that there are a lot of really deep and meaningful things that I think I could pull from this reading if I really gave it the time it deserves.
\end{document}