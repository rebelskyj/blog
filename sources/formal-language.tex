\documentclass[12pt]{article}[titlepage]
\newcommand{\say}[1]{``#1''}
\newcommand{\nsay}[1]{`#1'}
\usepackage{endnotes}
\newcommand{\1}{\={a}}
\newcommand{\2}{\={e}}
\newcommand{\3}{\={\i}}
\newcommand{\4}{\=o}
\newcommand{\5}{\=u}
\newcommand{\6}{\={A}}
\newcommand{\B}{\backslash{}}
\renewcommand{\,}{\textsuperscript{,}}
\usepackage{setspace}
\usepackage{tipa}
\usepackage{hyperref}
\begin{document}
\doublespacing
\section{\href{formal-language.html}{On Formal Language, or Why Experts are Bad at Explaining}}
First Published: 2022 February 16

\section{Draft 1}
Every so often enough ideas start percolating in my head to create a post for me.
Today's came to me from reading \say{Don't Teach Coding}\footnote{mentioned priorly}, thinking about the research I'm doing, a conversation with one of the readers of this blog, and being in a Catholic Church.
My thoughts, being percolated, are not yet coherent, so I apologize to whoever is reading this.

One issue with communicating high level information in a field to outsiders is that the language is necessarily formalized in a way that requires enough learning that outsiders cannot access the information without first becoming insiders.
As an example of this, our group makes \say{amorphous solid water}, not ice.
Why? Because ice as a term encompasses far more than what we do, and the different ways ices behave is relevant to our research.
Amorphous ice is different than crystalline ice, and water ices are different than other ices made from other liquids.\footnote{this second part should be fairly obvious I hope}

More than novel vocabulary, though, there is also the issue of redefined words.
As an example, the term \say{melting} seems fairly straightforward in day to day conversation.
In material science, though, it only refers to crystalline solids turning liquid.
In chemistry, though, it refers to any solid to liquid transition.
I'm sure in other fields it has other terms as well.

Computer scientists are of course intimately familiar with this issue.
Logic gates take common terms, like if, and, and not, and redefine them to mean exactly one thing.

Both of these issues, new words and redefining old words, have incredible benefits though, which is why it's so hard to be a beginner.
There's no way to talk about chemistry without electrons, but before knowing what electrons are, you have to define it.
That brings the second issue, which is that teaching a formal language using informal language requires a leap of understanding, which is never guaranteed.
I don't have any solutions, and I'm still not sure what I'm trying to say here, but I think it's fun how words work I guess.
\end{document}