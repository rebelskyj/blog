\documentclass[12pt]{article}[titlepage]
\newcommand{\say}[1]{``#1''}
\newcommand{\nsay}[1]{`#1'}
\usepackage{endnotes}
\newcommand{\1}{\={a}}
\newcommand{\2}{\={e}}
\newcommand{\3}{\={\i}}
\newcommand{\4}{\=o}
\newcommand{\5}{\=u}
\newcommand{\6}{\={A}}
\newcommand{\B}{\backslash{}}
\renewcommand{\,}{\textsuperscript{,}}
\usepackage{setspace}
\usepackage{tipa}
\usepackage{hyperref}
\begin{document}
\doublespacing
\section{\href{music-memories.tex}{On Music Memories}}
First Published: 2024 November 12
\section{Draft Three}My mother died between my last posting of this blog and now.  
Looking at the calendar, there were a solid two months between when I stopped and when we learned that her death was imminent.  
However, as I find myself avoiding both this blog and thinking about her, I find that I'm spending far too much mental energy that can no longer be used for anything I want to spend time on.  
So, today I want to use my musing to put down in writing some key memories I have with my mother in relation to music.

The first memory that keeps coming to my mind happened sometime while I was in high school or college.  
We were in her truck, driving to Williamsburg.\footnote{why we were driving there, whether we stopped there, any other details about the event I cannot recall at this point.}  
We started listening to some music on my phone through the car speakers, and ended up on a Johnny Cash album.  
Eventually we got to his Christmas album, where my mom and I joked about the fact that he didn't really sing so much as speak while a gospel choir behind him sang.  
What sticks out to me, though, is her discussing recording technology.

She claimed\footnote{and I unquestioningly believed, and have to this day never really bothered to confirm} that, before mono and stereo had gained supremacy, there was a period of time that music was recorded to be played through four locations at once.  
Apparently a lot of Johnny Cash was recorded in that form, and the sound was somehow fundamentally off for only being stereo.

Although not a single event, there was a recurring conversation through my life.  
I would be listening to Harry Chapin, my mother would express how much she loved his music, and my dad would make some comment about how it was too saccharine or not actually folk.\footnote{because apparently if you have a full orchestra and gospel choir you can't call yourself a folk artist}  
The music I find myself listening to when I want to just listen to music is so inherently shaped by the music she\footnote{and my father, who actually purchased the music generally} introduced me to.

Some of it\footnote{the Bad Examples, Dr. Hook and the Medicine Show, among others} may not have been the most developmentally appropriate music for children.\footnote{Arguably Bat out of Hell, even if I didn't really pick up on the sexual supertext (I don't know what to call something that is more blatant than outright stating) until recently}  
Right now I can't remember any specific instances of her singing along with the music, but given how much my brothers and I do, it feels like she must have, at least while we were younger.

Less a memory of her and music in particular, I do recall conversations we had starting in high school, where I really began to realize how close my family was.  
After one of my concerts\footnote{more accurately, a concert in which I was a member of the ensemble}, she mentioned that another parent was shocked at how her children were always at their siblings events.  
My mother made some retort along the lines of \say{I don't make that an option.}  
That framing doesn't convey what she meant though.

To my mother, family was something that deserved our complete support.  
It wasn't a question that we would be at each other's events because it wasn't a question that we would support each other.  
As she was in her final days, that was something that we returned to.  
Despite the fact that she was incredibly busy with her work\footnote{which her wake would have us think meant constantly causing formative positive experiences in everyone she interacted with}, she made herculean efforts to make it to every one of our\footnote{objectively excessively many} events.  
It was this unequivocal and unconditional love and support that gave me the confidence to be who I am.

Returning to the subject of music, my mother would often have her coworkers over for holiday parties when we were younger.  
Once my brother and I were at an age where we were competent at music, she would ask us if we wanted to play for the events.  
We, being the attention seekers that we were\footnote{are}, took her up on the offer.

Honestly, the last two reflections do a lot to clarify to me why I have so much difficulty coming up with single instances of memory.  
The way she raised me\footnote{not speaking for my brothers, though I am almost positive they feel the same way} was so full of love that any individual event falls into the general haze of positive love.  
I do remember her mentioning her mother's love of bagpipes when I took them up, and that helped me feel connected to my family.

She played guitar when I was younger, though stopped as we grew older and she became busier.  
In her last months, she wanted to return to guitar, but I don't think managed to do so.  
She's the reason that I took up guitar.

Much as I miss my mother, I wouldn't trade any of my memories of her for another day with her.  
Tears are starting to well in my eyes right now, though, which feels like a good reason to stop musing here.  


\section{Draft Two}

I find myself realizing that I don't recall ever truly spending time to reflect on my relationship to ordered sound.\footnote{or disordered sound, but music is a much more interesting thing to talk about than just \say{I am easily over stimulated}}  
Right now, I find myself incredibly reflective on a lot of things, even as I find it incredibly difficult to think at all.  
There are any number of reasons for this, but the most crucial of these is the fact that my mother died a month and four days ago.

I know the last time that I said goodbye to her, the last time I said goodbye to her body, and the last time I said goodbye to the idea of her as a physical part of my life.  
What I cannot recall with anywhere as much clarity, however, is the last thing that she said to me, when she last said it, or the last time I saw her awake.  
My little brother and I both realized at her wake that, for all that we had done our best as a family to prepare for her death, we had forgotten to get her to record a short voicemail that simply said she loved us.

Rather than dwell on that,\footnote{Tears formed in my eyes, and it immediately became hard to breathe. While that is absolutely a sign that I should spend time reflecting on this more, now doesn't feel like the time} I will instead focus on something that she instilled in me from a very young age: a love for music.\footnote{of course, she was not unique in this. My father, brother, extended family, schooling, and so much more did the same, but, potentially for the emotionally charged reasons, the most emotive memories I have of music are with her}  
She often expressed shock that all three of her sons ended up as musical as we did, which never made sense to me.\footnote{That isn't entirely true. She had been told often that she couldn't sing well, and as I think about it, she didn't tend to sing as far as I can remember. That's sad, though for other reasons.}  
From a young age, she encouraged all of us to do a lot of music, especially vocal music.

However, my musing today is not meant to focus on my relationship to performing music.\footnote{Small's \say{Musicking} aside, I do still think that there's a difference to me in being the consumer or producer of music.}  
Instead, I want to focus on my relationship to listening to music, and what music means to me.  
I might also spend some time thinking about some memories I hold deeply with my mother, if only so that there's a record of that.  
Actually, that sounds like a much better choice.

I'll save the general musing for another day. Today we will muse on   


\section{Draft One}

For some reason, I am convinced that I've written a musing on music and my own relationship to it before.  
However, a quick search does not bring anything up, other than a very short one which mostly discusses my own relationship with music.  
So, my goal here, in addition to restarting the blog again\footnote{For a variety of reasons, not least of which is that I want to be writing a lot more words}, is to spend some time really thinking about my relationship to sound in general, and music in particular.

When I was younger, I used to have a constant soundtrack following me while I was at home.  
There were any number of reasons for this, but one that I had not realized until I found myself in the boundary waters, unable to fill my life with constant sound, was that it kept me from dwelling on negative thoughts.

Now, writing that brings me to something that's been keeping me from writing here for the past month or so.  
Especially since returning to America and restarting this blog, I've tried my best to avoid identifying others, or even really to put too much of my own personal life in this blog.  
However, I don't know if that really serves me.  
Without writing it here, I have seen over the past month that I just don't write at all.

A month and four days ago\footnote{as of the day of writing this}, my mother died.  
The soundtrack which followed me in middle school kept me from thinking about the death of my grandmothers.  
The constant sound I use right now doesn't really feel like it serves the same purpose.

In part, I think that might have something to do with the way that my own relationship to sound has changed.  
\textit{Mr. Tanner}, a song which once occupied the entirety of my mind, no longer stops me from wearing deeper ruts in my unhealthy mental spirals.  
Studying music for a degree has given me new ways to abstract myself from what my ears experience.  
And, I think as importantly, I am twice as old as I was then.  
My relationship to my mind and body is fundamentally different.

Still, though, I do and have generally filled my time and ears with constant noise for a while, even predating my loss.  
For a while, that took the form of any number of audiobooks, especially while I was traveling places or otherwise engaged in activities that did not require my full focus.\footnote{playing video games, knitting, embroidery, and so on. Unfortunately practicing scales never became something I could do that for. Stretching, on the other hand, has always been a great way for me to feel great about listening to a book.}  
I cannot listen to audiobooks when needing to read, write, or otherwise focus my mind, however, and so I have thought a lot about the music I can use to work, if only because people are distracting and I work in a shared office space.

Music with words shares many of the same problems that an audiobook brings, though to a lesser extent.\footnote{of course, the fact that I do not listen to music at three times speed might have something to do with that}  
I find myself listening to the words more than to what I'm writing.\footnote{for all that I don't think I have a mental voice when I read, I know that I do when I write}

Classical music makes me remember my days as a student, and I find myself trying to follow the flow of counterpoint.  
Soundtracks, another popular option, never catch my interest enough to have playing for a while.  
I've always had an interest in minimalist music,\footnote{maybe not always, but for as long as I can easily remember, at least} and that helped for a little bit.

Music for Airports, in particular, gave me an incredibly productive little while to work.  
However, more and more, I find myself realizing that the music I truly prefer is process music.  
As I have written this musing so far, I've been listening to \say{Piano Phase} by Steve Reich.

It's a really interesting exploration of what happens when two pianos play the same simple motif at slightly different tempi.  
It explores the phase space of each note lying in relation to each other, and has a real sense of tension and release, despite being so few notes.  
However, it is repetitive enough that there it lets me continue to focus on my work.

In general, I find that right now I'm incredibly interested in the way that art can become obsessive.  
Villanelles, which are often seen as an inherently obsessive poetic form, are the only kind of poetry that my fingers seem willing to type.\footnote{outside of a few needlessly dark songs, but that's neither here nor there.}  
I'm sure that I could psychoanalyze that desire, and how it relates to the loss that I've experienced, but we've already left the initial topic of this essay so much that it hardly seems worth continuing this draft.\footnote{wow meta}  


\end{document}