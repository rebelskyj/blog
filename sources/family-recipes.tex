\documentclass[12pt]{article}[titlepage]
\newcommand{\say}[1]{``#1''}
\newcommand{\nsay}[1]{`#1'}
\usepackage{endnotes}
\newcommand{\1}{\={a}}
\newcommand{\2}{\={e}}
\newcommand{\3}{\={\i}}
\newcommand{\4}{\=o}
\newcommand{\5}{\=u}
\newcommand{\6}{\={A}}
\newcommand{\B}{\backslash{}}
\renewcommand{\,}{\textsuperscript{,}}
\usepackage{setspace}
\usepackage{tipa}
\usepackage{hyperref}
\begin{document}
\doublespacing
\section{\href{family-recipes.html}{On Family Recipes}}
First Published: 2023 December 23

\section{Draft 3: December 23}
Holidays are a time to remember.
For many, it is a time when the entire family shows up in the same space to share a meal and fellowship.
Traditions form as people bond to a specific food or ritual.
As older members die and newer people join the family, traditions mutate and change.

Sometimes, however, traditions last until they are questioned.
I'm reminded of one from my own family, where we used to trim the ends off of each roast before cooking.
When we finally reconstructed the narrative, it turned out that the only roasting pan one of the family owned was too small for the roasts.
Even as we got larger pans, however, we continued trimming the roast.

Yesterday, I was reminded of this as I prepared the only family recipe I get from my mother's half of the family.
Now, earlier drafts\footnote{I think that there's something to be said for referencing earlier drafts in these musings, since they're still available. Idk how I feel about it, though, so we'll see if this ends up being a one off} focused far more on the specific dish.
I think that it's worth taking a step back and thinking more about what family recipes tell us in general.

Until just a few years ago, for instance, I thought that we had a family recipe for green bean casserole.
In some regards we do.
We prepare the meal the same way, and it's a recipe we use for every important meal.\footnote{that's not entirely true, but it feels like it's true, which is something}
However, it turns out that the recipe we follow comes directly from the French's Fried Onion box.

I have read a number of so called family recipes that adapt\footnote{if even that} from a recipe book that someone's mother or grandmother or aunt read.
In the times before the internet, it was difficult to source information.
If someone brought a dish you liked, asking them for a recipe would not necessarily come with the recipe's provenance.

Honestly, my mind is blank today, and I do not know what else I could say.
I'm sure that there's more I could muse on, but I don't know what.

Daily Reflection:
\begin{itemize}
\item Hobbies:
\begin{itemize}
\item Did I embroider today? I still haven't figured out who I'd give the gift to, though I do like the last design I made. It could be fun to just make more letters.
\item Did I play guitar today? Just for a moment! It was fun, and I'm starting to enjoy E Major as a chord, for all that I still dislike B Major.\footnote{the chord, which is needed for the key of E Major}
\item Did I practice touch typing today? I did! I made it through P and am now ready to start on B tomorrow. I was once again stopped on c for a bit, which isn't fantastic, but it is what it is.
\end{itemize}
\item Reading
\begin{itemize}
\item Have I made progress on my Currently Reading Shelf? One of my library books is due soon, so I'm working to finish it tonish.
\item Did I read the book on craft? Today flew past sadly.
\item Have I read the library books? nope!
\end{itemize}
\item Writing
\begin{itemize}
\item Did I write a sonnet? Did not write one yesterday, will also not make one today.
\item Did I revise a sonnet? Still no
\item Did I blog? Eh, mostly yesterday, but
\item Did I write ahead on Jeb? Today is one of the days off.
\item Letter to friends? Started plotting some letters to friends.
\item Paper? Break day.
\end{itemize}
\item Wellness
\begin{itemize}
\item How well did I pray? A little better, if only because I woke up with nightmares.
\item Did I spend my time well? With family! That's the only thing I can hope for.
\item Did I stretch? OOp.
\item Did I exercise? Crud.
\item Water? Touch.
\end{itemize}
\end{itemize}


\section{Draft 2: December 22 (abandoned part of the way through)}
Holidays are a time of memories.
For many households in modern America, holidays are the one time a year that the entire intergenerational family is under one roof.
Flavors and specific preparations can be handed down from the people with the greatest attachment to them.
As the oldest members of the family die and younger family members are born, recipes and meal plans change.
Sometimes, things continue to be done the same way, simply because there is no need to question the way things are always done.

Sometimes, however, someone new\footnote{often a spouse} will ask what the point is of some tradition.
I'm reminded of one from my own family, where we used to trim the ends off of each roast before cooking.
When we finally reconstructed the narrative, it turned out that the only roasting pan one of the family owned was too small for the roasts.
Even as we got larger pans, however, we continued trimming the roast.

In my own household, we have very little of this generational memory.
In part, it comes from the fact that many members of my family in recent generations have disowned or been disowned by their relatives.\footnote{family here only refers to the ones who we are not estranged from}
\section{Draft 1: December 22}
Today I made one of\footnote{apparently the only from my mother's side, because her mother didn't really like to cook} the family recipes I've inherited.
We've always called it halishki\footnote{spelling nonexistent because oral tradition.}: a dish of dropped egg noodles in a chicken and cream sauce.
When you google the word, I mostly find a Polish word for a dish with noodles and cabbage in a pork based sauce.
There's more or less nothing in common with my family recipe and that.

However, this presupposes that my family recipe is Polish.
We really don't know where most of my family came from, especially this part of the family.
I've always been told it was somewhere Slavic, but that really does not narrow it down too much.
If we start searching for Polish drop noodles, we find kluski.
Wikipedia for kluski takes us to a similar dish in other portions of Eastern Europe, halusky\footnote{which has a diacritical over the s, making it pronounced more or less how my family's recipe is}.
Halusky is small dropped egg noodles, which does accurately describe the dish that I make, for all that it is also described as typically being served with cabbage.

However, I know that many members of my family have been unable to eat cabbage.
That could explain the lack of cabbage in the family recipe.
At this point, though, I was excited to learn more.

I got the last name\footnote{or the potential spellings for the last name} of the person we initially got the recipe from.\footnote{who died when my mother was newly born, which means there's no way to question anything}
The last name does come from the areas listed in the second Wikipedia article, which is pretty cool.
One of the two variant spellings is allegedly Jewish in origin, which is interesting.
I'm sure that there's a fascinating history there that is now lost to the mists of time.

So, what is the family recipe?
As far as I can tell, the recipe is more or less just chicken and dumplings in a cream sauce.
I don't see many recipes for chicken and dumplings that call for cream sauce.

How do I make it, though?
I think that every time that I have been personally a part of making it\footnote{three times now}, I've followed a slightly different recipe.
However, this time through I think will be my recipe going forwards.

I started by melting some butter in a large stock pot and then browning some chicken.\footnote{today was boneless skinless breasts, which I'm less thrilled about, but that's ok}
The goal was really just getting a bit of color on the meat and a touch of fond on the pan.
Once the chicken was browned, we put some butter in the pot and then some white onion.
Once the onion looked soft, I added in the chicken stock and let it come to a boil.

We made a very simple dumpling dough\footnote{or maybe batter} out of egg, flour, salt, and pepper.\footnote{as we joke, we make peasant food, and so use black pepper instead of white pepper}
We then spoon some portions of the dough\footnote{batter} into the boiling stock, chop some celery and add it in, and add the chicken\footnote{after chopping} back.
From there, I let it simmer covered fora bout an hour, and then uncovered it to let the stock reduce a fair amount.
Just before serving, we taste the broth for seasoning, and then add cream to taste and texture.

This time around, we made a huge recipe of it, which should be really nice for leftovers.
Historically, it's tasted far better the next day, as all of the flavors have had time to meld together.

Post Script:

As I look at the past six hundred words, I realize that I want to focus more on the family recipe aspect than the specific recipe we followed.
Let's see how that goes, moving from the specific case to more of a general one?\footnote{also, as you might have guessed from my musing yesterday, my goal is always to write a few more words at a time}
\end{document}