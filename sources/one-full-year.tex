\documentclass[12pt]{article}  
\newcommand{\say}[1]{``#1''}  
\newcommand{\nsay}[1]{`#1'}  
\usepackage{endnotes}  
\newcommand{\B}{\backslash{}}  
\renewcommand{\,}{\textsuperscript{,}}  
\usepackage{setspace}   
\usepackage{tipa}  
\usepackage{hyperref}  
\begin{document}  
\doublespacing  
\section{\href{one-full-year.html}{On One Year}}  
First Published: 2025 November 12

\section{Draft 2: 12 November 2025}

It is not a year since she died.

It is not a year since I sat shiva.

To my surprise, it is a year since I brought back this blog.

My last post before the reflection on Music was in July.  
I know that I went home to see my mother the weekend before Labor Day.  
She felt ill again, and went to the hospital; internally I joked that I was a bad omen, since coming home had sent her to the hospital twice in a row.

She was much more ill than we had thought again.

The doctors didn't know if she would make it again.

This time she didn't make it.

One year ago today I reflected on a few memories with music I have of her.

Earlier this year, I introduced the idea of a weekly album club to the remaining members of my family.  
Despite the fact that we're the entire Rebelsky family now, we do not use the chat we used to\footnote{Unimportant Family, whose last post came at the end of September before my mother died.  
We planned what we called \say{jailbreak}, or my mother going out into the world.}

I can't believe that it was the twenty fourth that we did that.

I remember when I was in middle school choir.  
My mother mentioned that she could hear me singing, and refused to see why that wasn't a good thing.

In the past year, I have not written here anywhere near as much as I would have liked.  
In the year coming, I hope that I will write closer to as much as I like.  
It's been a year since the first post I made since my mother left my home for the last time.

What will this next year bring?

One of my new work friends recently asked me if I have any new year's resolutions incoming.  
Honestly, I have forgotten that time exists.  
I mentioned my previous, and likely abandoned, tradition of N for N.

My mother would not want that.

So, I figure that this is as good a place as any.  
In the next year, I will grow.  
I will become more of the person my mother would want me to be.\footnote{that is, more of who I am, unashamedly. More than that, though, someone who cares fro those who need caring for}

I'm an hour past when I should be asleep, and what better time is there to finish writing, tears drying on my cheek as my shoulders still shake from half-repressed shudders of sorrow and grief?

My mother was a poet.

I wish I was too

\section{Draft 1: 12 November 2025}

There are countless experiences that we are unable to express in words.  
Most of the time, we think of these experiences as the incredible, the awesome, the sublime.  
Sometimes, however, it's the complete absence of difference that becomes notable.

Today I read through most of \say{On the Calculation of Volume}\footnote{book one}, which is a piece of literary fiction about someone stuck in a time loop.  
Something about that is really resonating with me right now.  
There are words for time passing by too quickly, there are ways to express time when it drags like a sullen child.  
There do not seem to be, however, words for the way that time is when it doesn't seem to exist.

I'm sure that my emotions must have changed from this time last year.  
We're a year from the day of the first post I made in a world without my mother.  
How has it been so long?

How has it been so short?

How long can I make my posts about my mother before people will want me to move on?  
What does it mean to move on?

I'm terrified to read my father's blog, because I know that he's in a similar place.  
They say it's good to face one's fears; let's see what he had to say recently.  
Wow there's a special kind of sorrow in the dramatic irony of a post in time.

In 2017, my father wrote \href{https://rebelsky.cs.grinnell.edu/musings/thirty-years}{\say{I hope we have at least another thirty years together}}.

Last year, I somehow missed my father \href{https://rebelsky.cs.grinnell.edu/musings/halloween-2024-10-31}{thinking about Halloween with my mother}.  
Maybe it's good for me to do this writing and reading now and here.

I don't think that I've cried in a few months.  
As I just read \href{https://rebelsky.cs.grinnell.edu/musings/nine-months-michelle}{my father's three quarter year reflection}, however, tears began spilling down my face.  
These emotions are definitely not new\footnote{definitely is a word I've only recently obtained the spelling for. Every time I write it, I remember the story my mom used to tell about how her own mother could spell check her essays over the phone, because my mom always misspelled the same words}, but I don't think that I've let them be.  
Maybe that's for the best.

The \href{https://rebelsky.cs.grinnell.edu/musings/anniversary-2025}{letter he wrote for their anniversary this year} is somehow even more painful.  
When I was listening to Not Dead Yet, I was driving towards my final outreach talk as a graduate student.  
I had to pull over for a moment, the grief was so intense.\footnote{I didn't actually, but emotionally I did, and there's something to be said for non literal speech here. As my mother used to say, lies by omission are still lies. It only tracks, then, that emotional truths are truths just the same.}

At a little over \href{https://rebelsky.cs.grinnell.edu/musings/eleven-months-michelle}{eleven months}, my father apparently came to terms with my mom's absence.  
I absolutely have not yet; I still immediately think of her when I need to ask a question about medicine or have a thing to share about our Catholic faith.\footnote{or, at least, what was our Catholic faith. Somehow, even as I find myself doubting so much else, the idea that she is in Heaven has not once entered my mind. Regardless of what the afterlife is, I know that she is in the best one.}  
I do so strongly relate to my dad's comment that it's sad no longer being able to have the excitement that comes from knowing I'm going to get to share something with her.  
He mentions my Ph.D. defense there.

That's such a hard thing for me to think about.  
That day, while we celebrated, one of the older members of the (astronomy) department came to ask my advisor what was happening.  
I don't know quite why, but she shared that my mom had passed.  
The prof nodded, walked away.  
He came back later and congratulated my father.

I'm sure there's something there, but the emotions have taken any possible story out of it.

\say{I was depressed, but not sad.} my dad writes in the eleven month reflection.

There's something almost unfair about the way that depression and grief hit me so differently.

He reflects on the journeys they never got to take together.  
Like my father, I too mostly enjoy trips for the people I can share them with.  
I don't know if my mother would have wanted to take a train trip, but I feel certain that she would have wanted me to travel after my defense.

It's so sad to realize that a year has come and gone and I did my best to ignore it.

And fuck,\footnote{is this among my first profanity here? I think so?} here's \href{https://rebelsky.cs.grinnell.edu/musings/one-year-michelle}{One year.}

My father reflected on how his mother turned to the faith of her Fathers\footnote{mothers?} when her husband passed.  
Even though she, too, is now gone, my father was able to connect with her in finding faith in the midst of grief.

I wish I was as brave and strong as my father.  
I cannot handle the idea of sobbing daily, even if that's the way I feel.   
The fact that he can let himself feel that is something I don't have words for.\footnote{wow, what a coincidence, given the theme of the day, I guess}

Also, sometimes I forget just how powerful of a writer my father can be.

\say{I typically assumed that things generally work out okay in the end.\\I guess I was wrong.}

The line break.

There's something powerful in the fact that my first post after her death was about music, and \href{https://rebelsky.cs.grinnell.edu/musings/surviving-concert-2025-10-19}{my father recently posted about a concert without her}.  
The artist is one of the many that I'm named after.

\href{https://rebelsky.cs.grinnell.edu/musings/thirteen-months-michelle}{And this brings us to the thirteen month}.  
I'm nearly positive that my maternal grandmother died on a Friday the Thirteenth.  
Something about that is so fitting, as she was always the most superstitious of those I knew.\footnote{I'm told she closed her eyes and held her breath whenever she drove over a bridge. Given how often she took the bridge from Iowa to Illinois, I really hope that isn't true. The Mississippi is a big river}

It feels almost like a betrayal to not do Thanksgiving at home this year.  
I know that we had many conversations about how little the traditions of Thanksgiving meant to any of us.  
Still, last year I felt connected to her as I made the one \say{family recipe} she insisted on every year: butter mushrooms.\footnote{the recipe is more or less in the name. Basically just slowly simmer a bunch of mushrooms in butter for as long as you can.}  
This year, there's none of her.

I don't know what making bagels this year will be like.

Last year I didn't have her, but the wound was fresh.  
Every year before, I had her to come down and make conversation.  
In years where there would others, she would chide me for not offering coffee.\footnote{I wonder what she'd think about my daily coffee drinking now?}  
In years without them, she'd just sit with me and talk for a while.

\say{, look, here’s yet another form we had to have her sign. She could barely write her own name at that point}.

There is something so incredibly painful in remembering the way that she slowly faded away.

Her own mother went to the hospital on a Monday, sorted her affairs, and left the hospital on Friday.  
I think that my mom had wanted to go out like that.

Then again, she also said that she wasn't ready to die just before I stop remember her being lucid.  
I still don't know what the last words she said to me were, nor do I know what the last words she heard me say were.  
I remember waking up to my brother knocking on my door.  
I remember walking to the hospital, passing by the same person at the entry desk.  
Unlike every other time, I don't think that she waved to us.

I remember standing beside her bed.

Or, I remember standing beside the bed I knew was no longer hers.

None of us had the words.  
We had the need to pray, however.  
I was the only one with the words of a prayer, and so we did the \say{Lux perpetua}, the one prayer I still find myself praying regularly.\footnote{every time that I pass by a graveyard or gravestone}

The next days are a blur.

I didn't realize that my dad had taken down my mom's medical texts.  
Those books sat in an unreachable part of our home for my entire life.  
I guess that it makes sense we should get rid of them; many probably would find homes where they could be of use.

My dad thinks that we won't associate any given book with him.  
I think that's probably true, but I do also have the firm suspicion that there will be books I cannot separate from him in my mind.  
War for the Oaks I think will always be a book from my mother in my mind.  
Love and Knishes is a book from my dad's mother.  
I didn't think of these facts until recently forced to think of death, though, so who can say.

I truly hope that I don't have to find an answer to that question for a long while yet.

OOf ok so here we go. Time to make a coherent final draft.

\section{Draft .8: 12 November 2025}

I'll be totally honest here:\footnote{at first this was a semi colon, but I think a colon is the correct punctuation. Also wow it's a pain and a half switching between Mac and windows for typing.} I have found that almost seven hundred words in to this, two full drafts later, I still don't know what this musing will be about.  
I guess that there is something in the fact that mentally I always think of these creations as musings (per my father) or blogs (per the fact that they're on the internet).  
Let's take a minute\footnote{me? procrastinate? never!} and look at why I decided to call it a folly.

Weirdly, I didn't make that choice until \href{verbiage.html}{this year}.

Perhaps because of the time of the year, perhaps because of the music I'm listening to, and perhaps because I just looked through the history of my posts here, but I find myself not quite sad, but certainly in a tearful emotion.  
It's so strange to think about the way that I started 2024, completely unaware of the fact that I would end it in grief.  
Looking at \href{reflection-2023.html}{my reflection at the start of the year}, I barely mention my family at all.  
I knew that my mom was dying of cancer; somehow that didn't make me decide to think about her.

Anyways, I want to look forward, I think?  
I think that this whole post might end up being posted as random musings, because I don't know what to say here.  
That's probably as good as it'll get.

\section{Draft 0.5: 12 November 2025}

I often have trouble creating and maintaining routine.  
And, of course, I have the normal issue of wanting to fix every issue in my life at once, rather than as they come up.  
With that in mind, I'm going to try to start accepting that each week is about creating a new good habit or removing an old bad habit.  
Given that it's working right now, I'm going to call this week's habit \say{get up earlier}.

Every time in my life that I can consistently wake well before I'm needed somewhere, I adore it.  
So, then, why do I not keep it up?  
Because entropy is always the secret winner.

One morning will come, I'll be incredibly tired from some event the night before, and so I will sleep in a little longer.  
After all, the habit should serve me, not the other way around.  
Then, more awake than I'm used to at night, I stay up late again.  
I sleep in again, and I'm suddenly waking just in time again.

So, how has resetting my sleep been going?  
Honestly, really well.  
The alarm has been set for half five since Monday.

Monday I slept again until 610.  
Tuesday I lay in bed until 550, despite having initially awoken a little before the alarm.  
Today I woke before the alarm again, and then started reading at 530.  
From there, I realized that I wanted to write, and so here I am.

In comparison to \href{flection-november-25}{my previous flection about the month}, there is a lot missing from this morning.  
I am not working on Celtic knotwork in the morning any of these days.\footnote{Though, I did figure out the main thing yesterday, and now I'm going to try to figure out a medium. I'm really leaning towards ribbon or rope for some reason}  
I have made exactly zero progress with the different pieces I'm working on in regards to composition.\footnote{the same friend's wedding also includes a trio singing a song, but the current arrangement is for a soloist over a trio. I just need to rewrite the harmony lines to make sure that there's only a need for two.  
Well, may as well do that now.}

See footnote.

\section{Draft 0: 11 November 2025}

This morning I woke up to find that it had snowed while I was sleeping.  
The snow, as it always is, was beautiful.  
The snow, as often happens to early snowfall, is gone now.

I'm realizing now that part of what's kept me from understanding the passage of time this year is the fact that the weather has not followed the passage of time.  
It was in the fifties last week.\footnote{I think}

Anyways, I'm not entirely sure what's going to come out of today's writing.

I've recently been reading more than before, and this evening I learned how to play bridge.  
I've also been working on an idea for a dear friend's present, but I'm now wondering if it might not be best made in a medium other than embroidery.\footnote{Right now I'm kind of thinking ribbon that's looped into the knotwork}  
After nearly two hours of banging my head against a wall, I finally found the reason that I could not get my knots to line up.  
That's nice and all, but it does leave a sour taste in my mouth.

Because I stayed late to play bridge, it's already my bed time, and that's something that I want to make real efforts towards respecting.  
So, I guess not every day ends up being a worthwhile post?

Hmmm, that's not satisfying as a thing to post, though.  
Guess this will wait for tomorrow, when I can give more time to thinking about the framing for this?

Current Pen List\footnote{for my own posterity, mostly}

\begin{itemize}  
\item Hongdian Black with Fude Nib: Monteverde Ocean Noir. 10/6  
\item Jinhao Shark: Diplomat Sepia Black. 10/6  
\item Pilot Preppy: Diamine Bilberry. 10/6  
\item Shaeffer: Private Reserve Ebony Green. 10/6  
\item Diplomat: Diplomat Caramel. 10/6  
\item Kaweko: Stipela Sepia. 10/6  
\item Monteverde: Diplomat Burgundy. 10/6

\end{itemize}

\end{document}