\documentclass[12pt]{article}[titlepage]
\newcommand{\say}[1]{``#1''}
\newcommand{\nsay}[1]{`#1'}
\usepackage{endnotes}
\newcommand{\1}{\={a}}
\newcommand{\2}{\={e}}
\newcommand{\3}{\={\i}}
\newcommand{\4}{\=o}
\newcommand{\5}{\=u}
\newcommand{\6}{\={A}}
\newcommand{\B}{\backslash{}}
\renewcommand{\,}{\textsuperscript{,}}
\usepackage{setspace}
\usepackage{tipa}
\usepackage{hyperref}
\begin{document}
\doublespacing
\section{\href{ratings.html}{On Ratings}}
First Published: 2023 December 16

\section{Draft 2}
What's the point of rating anything?
Ok, that's a little broad, let's try that again.\footnote{I've realized that opening with an overly casual tone helps me to write better. Or, at least, it feels better to me. I wonder if readers feel the same way}

A friend asked me to think about five star ratings for books, especially on the pseudo public platforms which aggregate ratings.
As I thought about that question, I came to a number of potential responses.

First, I think that five is too many options.
Really, I think that you only need two, maybe three if you want to be mean.
The two would be enjoyed or did not enjoy, and the third could be for did not finish if you wanted to separate books you read all the way through.
I would be very happy with just the two ratings, though.

So, what do I do when I'm given five stars?
There's an impulse to only use one and five star ratings\footnote{I'm not even getting into fractional stars right now, because I hate them as a concept.
If you want ten points, have the scale go to ten.
The only reason I'll give diving a pass on this is that the cards they use are set up in a really pretty pattern to make a number or the number and a half work well.
I still think that it would be worthwhile to do integers, but I'll respect their choice to have smaller numbers}
However, that feels lacking.
If I'm painting in black and white, I shouldn't just ignore the shades of grey\footnote{one of these days I'll figure out when I was anglicized (almost anglicised wow) into thinking that gray was spelled with an e} just because they're new and scary.

So, let's consider how I could use five stars.
One star could be the third bin that I had before, books that I could not finish because they were so poorly written.
Thankfully, very few books that I've encountered fall into this category, if only because most people who put books out in a public enough form that I find them have a competent grasp on the English language.\footnote{or can hire someone who does}

Five stars would have to be the absolute favorite books ever written.
The kind of books that you can't stop talking about when you read them, to the point that your friends feel obligated to either end the friendship or read the book themselves.

Even though I like shades of grey, I've always admired paintings that work in shades of black.\footnote{yes I know that definitionally there is a single shade of black. However, as a person with limited vocabulary, I fully acknowledge that there is a spectrum of color that I would call black even if I can distinguish the hues from each other.
I've seen some beautiful art that plays with that a a concept}
And so, I can find a use for two stars: books that I didn't finish because I didn't enjoy them.
These would be books that I'm certain others would enjoy, I just wasn't one of them when I encountered it.\footnote{the idea that what I enjoy is variable will come back later}

Four stars would be books that I really liked, but that I haven't made into a facet of my personality.
And, of course, that leaves three stars to hold the rest of the books, the majority of what I read.\footnote{realistically, since I know what I like, I have far more fours than threes}

But, this is a single dimensional view of the process.
For all that\footnote{you're welcome. (if you know what this is a reference to, that message is for you. otherwise feel free to ignore} I do think that most books I read fit that scale, it does have some gray areas.
Between a book with an astounding concept and fair\footnote{I love that fair counts as failing in a lot of places. It's like words mean nothing. Oh, wait, that's the theme of the musing today. Never mind} execution and a book that brings nothing new to the genre but does what it sets out to perfectly, which deserves a higher score?
As I once dove,\footnote{I just wrote this sentence and I'm already struggling to parse it. Readers that I know exist, is the same true for you?} I then find a use for the fractional stars.\footnote{much as I am loathe to}

In diving, two criteria determine your score: how well you did the dive and how hard the dive was.
The difficulty of the dive modifies how much the points for your dive are worth.
Doing that, I could have a few modifiers and apply them to my base score, putting books in the correct bin as needed.

Of course, this single bin approach to diving works because we can generally agree what the optimal dive looks like.\footnote{infinite flips and spins executed perfectly}
Books, however, serve a variety of purposes, and what makes a book successful in one domain is often what precludes its success in another.
Most modern sites are aware of this fact, and allow for tagging of books so that people can remember that one book is a cozy pastoral fantasy\footnote{readers, you know what I'm talking about} and another is a fast paced exploration of morality.\footnote{don't immediately have a book to place to this, but I'm sure that there's at least a few that I've read. Oh, I suppose that the one Discworld book where Vimes gets possessed by the Summoning Dark would be that. Or Mort, or any of the Death Books. Wow look at that}

Then, of course, the rubric breaks down.
I now, on some level, need to have separate rubrics for each tag and combination of tags.
This all ignores the fact\footnote{I keep wanting to use of course, which makes sense, because I've thought about this a lot and already written a draft today. If my thoughts aren't obvious to me right now, that's an issue} that modern sites aggregate the data.
Here we come to a new issue.

I, like most people, am aware on some level that my time on this earth is finite.
I do not have the time to read every book that comes out in a given month, let alone every book that comes out every month.
And so, I rely on the aggregation of other readers in deciding what to read, at least on some level.\footnote{honestly, these days most of what I read comes from a single recommendation, so I guess this maybe works for me the abstract reader, not me the physical body. Oh, also a lot of books I just pick up in the library because they look interesting (in their titles, since they're by and large textbooks or academic texts that aren't in CS, the covers aren't pretty), which also doesn't have aggregation information other than implicitly, since the library chose to stock it}
This comes with a conundrum.

When I rate something with four stars, it falls below a book that I rate with five stars.
And so, all the different criteria end up collapsing into that same rating system I introduced at the start of the musing: good\footnote{five stars} or bad\footnote{one star}.
I've heard it described as five stars means you were ok with it, and everything below five is how much you disliked it, which isn't that far off from the way that ratings end up playing out.

Daily Reflection:
\begin{itemize}
\item Hobbies:
\begin{itemize}
\item Did I embroider today? I sadly didn't end up going into the office today. Embroidery remains trapped there.
\item Did I play guitar today? I did! I'd been noodling on guitar and it reminded me of the opening to a song, so I sang through it.\footnote{it's wild that I'm at the point that a chord sheet (I think that's what they're called) is enough for me to play a song I know} After that, I played some scales.
\item Did I practice touch typing today? This day has flown by, and even though I wouldn't change most of what I did, that doesn't make the time come back.
\end{itemize}
\item Reading
\begin{itemize}
\item Have I made progress on my Currently Reading Shelf? I listened to a little more of the audio book! Soon it will be finished, which is cool.
\item Did I read the book on craft? On my daily goals this morning, I did put it on the list, but didn't get around to it, sadly enough.
\item Have I read the library books? I renewed my books, which is a step in the direction of believing that I'll read them. Maybe I have to return two books a month minus however many I read starting next month? That would make me sad but might make me read more books. At the very least, I'd decrease my bookshelf by 2 a month.
\end{itemize}
\item Writing
\begin{itemize}
\item Did I write a sonnet? Yesterday I actually liked the sonnet I wrote. Today's was pretty decent
\item Did I revise a sonnet? Nope. I think that I've learned how to do so, though, which was the monthly goal.
\item Did I blog? I actually kind of like the post today.
\item Did I write ahead on Jeb? I wrote about half a chapter. If I weren't so tired, I'd do more.
\item Letter to friends? I should remove this goal, since I doubt I'll make progress on it before the year is out.
\item Paper? I thought about what I'd want it to say at a large scale picture.
\end{itemize}
\item Wellness
\begin{itemize}
\item How well did I pray? Oof.
\item Did I clean my space? Yes, but I think the net was dirty.
\item Did I spend my time well? Kind of! I was somewhat thoughtless through the day, but that's kind of to be expected.
\item Did I stretch? Shoot.
\item Did I exercise? Um.
\item Water? Ope, that's absolutely something I forgot. Let's do that now.
\end{itemize}
\end{itemize}


\section{Draft 1}
Yesterday a friend asked me about my feelings on five star scales.
As I thought about it a little more, though, I realized that I have a lot of feelings about rating systems in general.
Today seems like as good of a place as any to discuss how I feel about them.

Rating systems, especially as they exist now, suffer from the tragedy of the commons.\footnote{ok so I fully recognize that the tragedy of the commons is a completely ahistorical idea, given that commons were famously well maintained by everyone. However, it's a phrase that works well enough right now}
Most rating systems are not used for individuals, but for public ranking.
That's an issue, for a lot of reasons.

Let's explore why.\footnote{not that I think any of my readers couldn't do this on their own, just that it feels like it could be useful for me to do this for myself}

If I have five stars to rate every book I read, for instance, there are a few ways that I could do it.
Assuming that 0 stars is not an option, of course, I only have five categories to place absolutely everything I read.
There are few easy books to place.

First, there are the books that I read over and over, finding something different in them each time.
The sort of books that we love because they feel able to grow with us just as much as we grow.
They, obviously, would get five stars.

On the opposite end of the spectrum, there are a lot of things that I've read that I could not get more than a few pages into\footnote{if even that} because there was absolutely no sense of grammar, tense, or spelling.
An argument could be made that ratings should be reserved for books I've finished, and not for everything I've tried to read.
Of course, if I have a rating system, I want it to be a record that I can use in the future.
Since that's true, I want to show everything that I've attempted to read.

So, that's one star and five stars taken care of.
What about books that I couldn't get through, just because I didn't enjoy them?
It feels very unfair to put them at the same place as books that I couldn't read because they were illegible\footnote{is that the right word? kind of feels like it shouldn't since I have a lot of internal ideas about legibility being a handwriting thing. A quick search tells me it's understandable, so I guess it works. Weird}, which implies that I should put it at two stars.

Looking at the other direction, there are a lot of books that I enjoyed, but would not plan to reread because I don't think that there's anything more that I want to gather from the book.\footnote{initially I said either, but I can't think of another reason that I wouldn't reread a book. Maybe my imagination is just lacking these days}
Or, there are books that I plan to reread, not because I find that they are thought provoking or help me grow, but because they're comfortable.
Similar to a nice bowl of Kraft brand mac and cheese, sometimes you aren't looking for the best, simply the most familiar.

And that leaves one star for everything else I read.
Three stars would mean that I finished the book, and either didn't enjoy it, or don't plan to reread.
Honestly, most books that I read should fall here, if we think that stars should be averaged.
Then again, if I consider that I know what I like to read, it would make sense that the books I read would generally be above the baseline.

Of course, this creates an immediate tension.
Five stars works best if there's even distribution between categories.
For instance, if I put all books that I did not finish as one star, I then have four stars to granulate my feelings, rather than simply three.
Others might immediately point out that fractional stars exist.

Honestly, I find that fractional stars start to become worse than useless for me very quickly.
Once I get above general feelings of no, sure, yes, and absolutely\footnote{honestly, the fact that I have a four point scale for life and that these are the four points does say a lot about me}, I don't think it's fair to grade books on a single axis.
How do I compare a book that has fantastic ideas but less than stellar execution with a book that is not as thought provoking but does what it attempts perfectly?
As a diver, there's the voice in my head that says you add modifiers.

For those who don't know, points in diving come from two, arguably separable places.
First, there's the score that the judges give.
The sum of the three middle judges' scores\footnote{so in 5 judge situations, they drop the high and low score each time. Since judges aren't themselves being judged, it doesn't matter whose score gets dropped in cases where multiple judges give the same score (actually, as I think about it, there was a statistic on some of the report sheets I've seen from meets with rankings of judges based on how often their scores were taken. I wonder if someone's ever explored that). In three judge meets, all scores get taken. I think that you're not supposed to have even numbers of judges, for the obvious reason}
is the first point in scoring.
In that regard, it is optimal to make sure that any dive you do is done to the absolute highest score possible.

However, there is a secondary consideration.
Each dive is assigned a degree of difficulty.
That value is multiplied by the first score to produce the final score for a dive.
In that regard, it is optimal to attempt the most difficult dive possible, because it has the highest modifier.

Of course, since diving is a sport judged by humans, there are elements of bias.
Dives that land on hands, for instance, tend to be scored more favorably than dives which end on feet.
More than that, though, a lot of judges are more forgiving of small errors in more difficult dive, whether because it's difficult to see every part of a more complex dive or because it just seems more impressive.

Before each meet, a diver therefore needs to consider how best to balance doing a dive well and doing a difficult dive.
Returning to the book, it could make sense to make modifiers.
Of course, since we're still\footnote{a little over a thousand words in. This musing will absolutely need a second draft} only on the single person use case for ratings.
I don't need to have hard numbers, especially if I read books fairly often.
Having vague ideas of the modifiers I want to apply is probably good enough.

Now, the astute might notice that diving does collapse those two scores into one at the end.
I could, in theory, do the same.
However, I do not always want to read the same thing.
I tend to have around ten books in my currently reading shelf.
If we look at them, I can categorize them as\footnote{using the current shelf plus things I know that I'm planning to start in the next few days}
\begin{itemize}
\item Two dense and difficult creative nonfiction books about topics I want to learn more about
\item Two dense theology books
\item Two literature books\footnote{i.e. books that have merit but are not written to be beach reads}
\item Two somewhat hard books that I want to learn from.
\item Four beach reads,\footnote{i.e. books that don't ask thoughts from me or that I can read while distracted.} some of which are fiction, and some of which are nonfiction
\end{itemize}

It isn't fair to compare a beach read to a literature book, especially because the very things that make one good in its own genre are\footnote{is} what would make it bad in the other genre or what disqualifies it.
Now, of course, nearly every book sorting site has ways of adding categories to the books you read.
If I then just rank each book with its modifiers and tags, I can sort the tagged books when I want to recommend or reread something.
And, to some extent, that could work.

However, there are more than two factors that I want to consider.
As much as it kind of feels wrong to say\footnote{ooh, I should explore why at some point}, the politics that the author espouses in a book are relevant to how much I enjoy them.
A book that has an interesting premise, is well written, but concludes with statements like how slavery is fundamentally a good thing, is not something that I want to recommend, if only because I don't want anyone in my life to think that I think slavery is fundamentally good.\footnote{I feel like I shouldn't have to say that, but here we are}

So, we get to the point that reducing everything to a single number becomes somewhat meaningless.
However, as I alluded to earlier, the danger with ratings is that they do not exist in a void.
When I rate something, the ratings generally become somewhat public.
Now, there's a new level of consideration that I have to use.

On the one hand, I want to give accurate and useful ratings for my own purposes.
On the other, I also want the authors I like to succeed.
I know that a lot of people actively sort to find the highest ranked books within a genre.
Truth be told, I cannot say that I am too different.

Of course, this does lead to a bit of a chicken and egg issue.
The fact that only high rated books get read means that people are more likely to give anything they like five stars, which means that the quality of anything with less than five stars starts to drop dramatically.
On one site I read often, grammar that is nearly impossible to understand is still given a three star by some commenters.\footnote{who fully acknowledge that they have trouble parsing the sentences}
As this cycle continues, each feeds more and more into the other, making it so that the rating system would really be best served by a simple positive or negative.
I remember seeing that Netflix had switched to that, and I think they found it was more effective for what they wanted.

So, right now I've only discussed books and rating them on a five point scale.\footnote{and a little bit of diving, but that's its own different thing}
I think that most of the lessons that I want to hit on are more or less the same.
Time to revise this and see if we can't make the writing sparkle a little more.
Might be worth framing the whole thing as \say{start from yes no, then grow into five stars, then grow into multi dimension that collapses, then go to tags, then back}, because there's fun in dramatic irony.

Right?\end{document}