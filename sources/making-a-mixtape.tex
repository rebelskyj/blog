\documentclass[12pt]{article}[titlepage]
\newcommand{\say}[1]{``#1''}
\newcommand{\nsay}[1]{`#1'}
\usepackage{endnotes}
\newcommand{\1}{\={a}}
\newcommand{\2}{\={e}}
\newcommand{\3}{\={\i}}
\newcommand{\4}{\=o}
\newcommand{\5}{\=u}
\newcommand{\6}{\={A}}
\newcommand{\B}{\backslash{}}
\renewcommand{\,}{\textsuperscript{,}}
\usepackage{setspace}
\usepackage{tipa}
\usepackage{hyperref}
\begin{document}
\doublespacing
\section{\href{making-a-mixtape.html}{Making a Mixtape}}
\section{Draft 2}
I couldn't think of anything to write today.
But, I was listening to a lot of music today, so I thought I'd try to make a mixtape.
The theme is inspiration, specifically academic inspiration.

Front Side:\\
Anthem by Harry Chapin\\
Anything I'm Not by Lenka\\
Corner of the Sky from the Pippin (New Broadway Cast Recording)\\
Country Dreams by Harry Chapin\\
Cover of the Rolling Stone by Dr. Hook and the Medicine Show\\
Dammit Janet by the Rocky Horror Cast\\
Dancing Queen by ABBA\\
Don't Stop Believing by Journey

Back Side:\\
Grey Seal by Elton John\\
A Hard Day's Night by the Beatles\\
Heartbeat (It's a Love Beat) by the DeFranco Family\\
How Far I'll Go from Moana\\
It's Still Rock And Roll To Me by Billy Joel\\
Kelly the Boy from Killane by the High Kings\\
Livin' On A Prayer by Bon Jovi\\
Long Goodbye by the Nadas\\
Mo Ghile Mear by Celtic Thunder

Explanation:
Like Chapin's protagonist, who's \say{searching for an anthem}, I start by searching for motivation to start a paper.
But, within seconds of writing a paper, I realize that it's hard to be \say{Anything I'm Not}, and I'm hoinh to procrastinate.
I'll go find my \say{Corner of the Sky} to hang out at until I have the motivation to work.
While in the \say{Country(,) Dreams} of avoiding responsibility come into my head.
But, if I want to make it \say{On the Cover of the Rolling Stone}, I'll need to get to work, and \say{Dammit Janet}\footnote{for the purpose of this essay, assume I have a muse whose name is Janet} I can't think of anything.
I just want to be \say{Dancing (to) Queen}, not writing this paper.
But I know that if I \say{Don't stop Believing} and write this paper, I might have time to go out.

I can't really work \say{Grey Seal} in here, but it's a good next song.
As I look at the clock, unsure of whether the 6 is an AM or PM, I realize it's been \say{A Hard Day's Night}.
And, as the deadline approaches, our heart may start pumping out of control.
But, if we remember that \say{Heartbeat (It's a Love Beat)} you can tell yourself that it's not panic, it's love of the topic.

As the paper continues, I often wonder \say{How Far I'll Go} in trying to finish the paper.
What's the tenuous connection I hope the professor won't latch on to.

Each paper and assignment is still somehow different from the one before.
But, as Billy Joel points out, \say{It's Still Rock And Roll To Me}.
Every work is different, but the same in principle.

\say{Kelly the Boy from Killane} is just a bop.

As I've mentioned a lot so far, as a deadline draws near, students often feel nervous, like they're \say{Living On A Prayer}.
And, as we look at the work we have left in the semester, and the time left to do it, we may have to say a \say{Long Goodbye} to our friends until the work we all have is resolved.

But, we have hope that we'll make it through.
\say{Mo Ghile Mear} is a song of hope, so it seems fitting to end the list.

\section{Draft 1}
As I've mentioned in many other posts, I like music.
So, I decided that I would make a mix tape today.
Cassette tapes apparently hold 60-90 minutes of music, so that seems like a good length to shoot for.
I'll assume 30 minutes per side, since I'm feeling unambitious.

Next I need a theme.
I think today's theme will be inspiration.
It's the point in the semester where it seems that inspiration is needed.\footnote{at least to me}
Of course, inspiration comes in a variety of places.
So, I'll start with \say{Anthem} by Harry Chapin.\footnote{3:56}
I'll follow it with Lenka's \say{Anything I'm Not}.\footnote{3:18. Total time: 7:14}\,\footnote{and apparently this mixtape will be in alphabetical order}
Next is \say{Corner of the Sky} from the Pippin (New Broadway Cast Recording).\footnote{2:57. Total time: 10:11}
Following that will be \say{Country Dreams} by Harry Chapin.\footnote{sorry, this may be a Chapin heavy playlist. I've been listening to a lot of him lately, so it's on my mind}\,\footnote{4:48. Total time: 14:59}
Next: \say{Cover of the Rolling Stone} by Dr. Hook and the Medicine Show.\footnote{2:54. Total time: 17:53}
After this: \say{Dammit Janet} by the Rocky Horror Cast.\footnote{2:47. Total time: 20:40}
Of course, I would be remiss if I didn't throw some ABBA in, so next is Dancing Queen.\footnote{one of the songs that I listened to for the 15:30 of continuous CPR I was to do}\,\footnote{3:51. Total time: 24:31}
Continuing the theme of powerpop from older generations' childhoods, \say{Don't Stop Believing} by Journey.\footnote{4:09. Total time: 28:40}

Backside:
\say{Grey Seal} by Elton John.\footnote{4:01}
It was so tempting to put another Harry Chapin song here,\footnote{bonus points to whoever knows what song it is} but instead, \say{A Hard Day's Night} by the Beatles.\footnote{2:33. Total time: 6:34}
And: \say{Heartbeat (It's a Love Beat)} by the DeFranco Family.\footnote{3:10. Total time: 9:44}
Following that, \say{How Far I'll Go} from Moana.\footnote{2:43. Total time: 12:27}
Next: since I realize I'm missing Billy Joel, \say{It's Still Rock And Roll To Me}.\footnote{2:57. Total time: 15:24}
\say{Kelly The Boy from Killane} as performed by the High Kings follows this.\footnote{3:29. Total time: 18:53}
Returning to nostalgia I have no right to have, \say{Livin' On A Prayer} by Bon Jovi.\footnote{4:11. Total time: 23:07}
And, following that, a song by an Iowa group\,footnote{yay!} \say{Long Goodbye} by the Nadas.\footnote{3:06. Total time: 26:13}
Finishing off the playlist, \say{Mo Ghile Mear} as performed by Celtic Thunder.\footnote{2:55. Total time: 29:08}

Time to give the playlist a\footnote{minimal} listen, then revise/add comments.
To begin: I started with Harry Chapin's \say{Anthem,} because it has a strong driving beat, which helps focus, and it has lyrics that I appreciate.
Also, the background instrumentation feels much sparser than most of his songs, which I appreciate.

Anything I'm Not follows, as it too has a strong pulse.
Unlike Chapin, Lenka has\footnote{to me} a much more cheerful outlook.
The song ends with an idea of escaping, and becoming free, which fits in so well with the next song.

Corner of the Sky is my favorite Pippin song, and one of my favorite musical theatre songs.
Like many graduates of my high school, I have fond memories of watching our choir director sing this song at our senior choir concert.
The song helps keep me focused when I really need to get work done.\footnote{the alleged point of this tape}
It speaks about finding where you belong, which the next Chapin song, \say{Country Dreams} does as well.

Unlike in the Pippin song, Chapin fully accepts that he's given up on his dreams.
But, he's accepted that, and still keeps on trucking.
Especially as deadlines approach, knowing that you've got something, even if it's not what you want, is all you can sometimes ask for.

But, we still dream big, and so to do Dr. Hook and the Medicine Show's in \say{Cover of the Rolling Stone}.
Of course, once you're at the top, there's always something that can go wrong.
A car could break down outside of a house, and you might need to go into a creepy house to get your way back to civilization.

Of course, that's nonsense, but \say{Dammit Janet} from Rocky Horror Picture Show has the strong beat that we appreciated in the first songs, and speaks of Brad's\footnote{and later Janet's} goal to achieve.

Next, \say{Dancing Queen} by ABBA starts with the line that students dream of during studies.\footnote{Friday night and the lights are low/looking out for a place to go/where they play the rock music}
After 20 minutes of concentration, I need something to lighten my spirits.

Of course, the next song, \say{Don't Stop Believing} speaks to me every time I know I don't have enough time to finish an assignment.
Somehow, I'll make it through.
And there ends the front side.

In the time it takes me to switch our metaphorical cassette tape over,\footnote{since I don't know where I would find, make, and play a real cassette tape} I need a quick and lively introduction.
Elton John's \say{Grey Seal} gives that to me.
Then, since homework is still continuing, \say{A Hard Day's Night} goes through my head when I've been inside a building working on an assignment, and am not sure which 6 the clock is pointing to.\footnote{if my mother is reading this, I always go to sleep promptly at 10:00 pm and wake at 8:00 am}
And, as the deadline approaches, our heart may start pumping out of control.
But, if we remember that \say{Heartbeat (It's a Love Beat)} you can tell yourself that it's not panic, it's love of the topic.

As the paper continues, I often wonder \say{How Far I'll Go} in trying to finish the paper.
What's the tenuous connection I hope the professor won't latch on to.

Each paper and assignment is still somehow different from the one before.
But, as Billy Joel points out, \say{It's Still Rock And Roll To Me}.
Every work is different, but the same in principle.

There's nothing special about \say{Kelly the Boy from Killane}, it's just a motivational song.

As I've mentioned a lot so far, as a deadline draws near, students often feel nervous, like they're \say{Living On A Prayer}.
And, as we look at the work we have left in the semester, and the time left to do it, we may have to say a \say{Long Goodbye} to our friends until the work we all have is resolved.
And, like any good playlist, it ends with hope.
Specifically, hope that someone will come set us free.
Here, \say{Mo Ghile Mear} stands in nicely.
\end{document}