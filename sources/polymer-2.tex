\documentclass[12pt]{article}[titlepage]
\newcommand{\say}[1]{``#1''}
\newcommand{\nsay}[1]{`#1'}
\usepackage{endnotes}
\newcommand{\1}{\={a}}
\newcommand{\2}{\={e}}
\newcommand{\3}{\={\i}}
\newcommand{\4}{\=o}
\newcommand{\5}{\=u}
\newcommand{\6}{\={A}}
\newcommand{\B}{\backslash{}}
\newcommand{\sub}[1]{\textsubscript{#1}}
\renewcommand{\,}{\textsuperscript{,}}
\usepackage{setspace}
\usepackage{tipa}
\usepackage{hyperref}
\begin{document}
\doublespacing
\section{\href{polymer-2.html}{Polymer Review Part 2}}
First Published 2018 November 29

Prereading note: as with the rest of these posts, it's written mainly from memory, and will\footnote{hopefully} be consulted and fixed for accuracy before examinations.
\section{Draft 1}
Polymers are super cool, for a variety of reasons.
The one that I'm thinking of today is that they take all of the things that you get told in chemistry class exist but don't matter, and then suddenly make them matter.
The two examples of that I'm thinking of are the fact that the main force holding polymers together is their Van Der Waals interactions, which is the force I've\footnote{so far} always been able to ignore as non-significant for a material.
The other is that polymers treat all energy as energy.
That is, the temperature that a polymer needs to be at to behave as a liquid is lower if mechanical energy is supplied.

Speaking of, there are a few phases that polymers exist in.
I'll briefly discuss them from lowest to highest temperature.

At the coldest, a polymer is in what's known as the \say{glassy} state.
Here, it is solid, brittle, and the chains are held rigid.

When you heat up, you'll eventually reach the glass transition temperature, T\sub{g}.
This is the temperature range where the polymer would be described as \say{leathery.}
It also corresponds to a quick drop in modulus.\footnote{which is basically strength}
This is due to the fact that below T\sub{g}, the energy needed to break the Van Der Waals bonds is higher, because the polymer is moving less.
Once you reach T\sub{g}, however, the polymers move enough on their own that the Van Der Waals forces effectively disappear.
The only thing that gives the polymer strength then is the entanglements, which we talked about \href{polymer-1.html}{yesterday} as M\sub{e}.
Above T\sub{g}, you're in the rubber phase.
This is where a polymer behaves a lot like what we think rubber does.
And, right near T\sub{g}, the material exhibits a lot of damping.

After T\sub{g}, you reach what's known as T\sub{m}, which is the temperature where the polymer stops acting like a solid, and begins acting like a fluid.
Apparently it's only melting when crystals do that, so that isn't what happens to many polymers.
One interesting tidbit is that polymers that are crystalline\footnote{and maybe amorphous?? I'm not totally sure} will have a T\sub{m} equal to 1.5x the T\sub{g} in Kelvin.

The information about T\sub{m} only applies to thermoplastics, polymers that don't have chemical crosslinks.
If a polymer has crosslinks, it never reaches the viscous phase.
Instead, it stays at the \say{rubber plateau} for modulus until the material ultimately burns.\footnote{i think}

The other thing to note that is different about amorphous and semi-crystalline polymers is that the drop at T\sub{g} is much larger in amorphous polymers than in semi-crystalline, because T\sub{g} is where amorphous regions fail, while T\sub{m} is where the crystalline sections fail.
So, since a semi-crystalline polymer has a much smaller composition of amorphous polymer, it loses less strength.

\end{document}