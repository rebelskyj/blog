\documentclass[12pt]{article}[titlepage]
\newcommand{\say}[1]{``#1''}
\newcommand{\nsay}[1]{`#1'}
\usepackage{endnotes}
\newcommand{\1}{\={a}}
\newcommand{\2}{\={e}}
\newcommand{\3}{\={\i}}
\newcommand{\4}{\=o}
\newcommand{\5}{\=u}
\newcommand{\6}{\={A}}
\newcommand{\B}{\backslash{}}
\renewcommand{\,}{\textsuperscript{,}}
\usepackage{setspace}
\usepackage{tipa}
\usepackage{hyperref}
\begin{document}
\doublespacing
\section{\href{reflections-on-readings-2-ordinary-c-2022.html}{Reflections on Today's Gospel}}
First Published: 2022 January 17\footnote{a day late, I know}

John 2:4: \say{Jesus said to her, \say{Woman, how does your concern affect me? My hour has not yet come.}}

\section{Draft 1}
As I mentioned \href{reflections-on-readings-2-ordinary-c.html}{last time I read these}, my family really likes the readings that happen this week.
This time around, the concept that the Lord's \say{hour ha[d] not yet come} really stuck with me, because it says that He was there with his disciples.

I know that I've always been taught that this was the beginning of His ministry, so I'm unsure why he already has disciples.
But, it's still beautiful that He listened to His mother without question.
\end{document}