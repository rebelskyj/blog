\documentclass[12pt]{article}[titlepage]
\newcommand{\say}[1]{``#1''}
\newcommand{\nsay}[1]{`#1'}
\usepackage{endnotes}
\newcommand{\1}{\={a}}
\newcommand{\2}{\={e}}
\newcommand{\3}{\={\i}}
\newcommand{\4}{\=o}
\newcommand{\5}{\=u}
\newcommand{\6}{\={A}}
\newcommand{\B}{\backslash{}}
\renewcommand{\,}{\textsuperscript{,}}
\usepackage{setspace}
\usepackage{tipa}
\usepackage{hyperref}
\begin{document}
\doublespacing
\section{\href{reflections-on-readings-holy-family-b-23.html}{Reflections on Today's Gospel}}
First Published: 2023 December 31

\section{Draft 2}
Today is the Feast of the Holy Family.
There are a number of optional readings today, which means that there is no way that I would be able to discuss every variation on the readings.
Instead, I am going to reflect on the readings that were read at the Mass I went to today.

Year B's version of the first and second readings focus on Abram turned Abraham.
We see his conversation with the Lord, where he bemoans the fact that all of his blessings will fall to his servant, rather than one of his descendants.
This conversation is striking to me for a variety of reasons.

First, the G-d of the Old Testament, for all that we talk about the anger, is also a kind and loving Father.
I could never imagine a Greek myth where someone complains to Zeus about the way that they have not been blessed the way that they wanted to be.
Of course, the Greek gods are gods in a very different sense than the Almighty.
I don't need to go into that, but it is something that I thought about as I started redrafting this musing.

Second, the Lord assures Abraham that his children will be as countless as the stars.
As a child, that felt interesting to me, because it seemed like we should be able to count the stars.
Or, at least, there is a way to count all the stars that the naked eye can see in the night sky.\footnote{zero, if you live in many places}
Similarly, there is, in theory, a way to measure all of the biological descendants of Abraham.
We could genetically sequence the world and find all of his living children, at least.

However, the stars in the sky are far more than the stars that we can count.
One of the biggest discoveries of Hubble was that there is no such thing as empty space.
For those who don't know, one of Hubble's projects was staring into a dark void, previously thought to be empty space.
What it found was more and more dimmer and darker stars.
Similarly, as the priest reminded us during his homily today, all Christians are adopted children of Abraham.

The second reading reminds us of the many ways that the Old Testament prefigures the new.
Abraham is willing to trust his son, his only child with Sarah his wife,\footnote{not his only child, one might note} to the Lord.
When the Lord says that the is to kill his son, Abraham trusts that he could bring his son back from the dead.
Thankfully, however, the Lord does not require Abraham to go through with the sacrifice.
In the New Testament, however, the Lord does give up his only son.

I've seen discussions of the Binding of Issac which point out that Abraham was ancient when he was asked to kill his child.
There was no way, the commenters claim, for Abraham to overpower his child.
That is, his child went to the sacrifice willingly.
That doesn't directly relate to the Gospel today, but it feels at least somewhat relevant.

Anyways, time to move to the Gospel.
This passage, like one of the readings from Luke a few weeks ago, contains one of the Canticles that we use frequently.
We have the Canticle of Simeon.
The Canticle of Simeon, like the Canticle of Mary, is a song of praise to the Lord.
Today's reading notes that Simeon is full of the Holy Spirit as he says the words.

It is also important to note that Christ is introduced to the Jewish people according to custom.
For all that modern Catholicism tends to overlook a lot of the way that Christ existed as a Jew in the Second Temple Era, it was a vital part of the early Christian theology.
Christ was brought into the Temple, the same way that every other Jewish boy child would have been.

However, this entry to the Temple is not entirely filled with joy.
A prophetess of the Lord sees Mary and tells her that a sword shall pierce her heart.

Sometimes I think about the song \say{Mary did you Know?}
It's often unpopular among the theologically inclined, because the answers to most of the questions are yes.
For all that these complaints are valid, I do think that there's something really powerful in knowing that Mary spent 33 years with her son, knowing the entire time that he would be murdered before she would die.
It is often said that the worst fear a parent has is outliving their child.
I cannot even imagine what it would be like to not only outlive your child, but to watch him grow and develop knowing, the same way that you know the Lord is G-d, that he will die before you.
Anyways, I think that's as far as I can get in understanding the readings right now.
I've still got some hours until the New Year\footnote{well, one of the new years, at least}, so there's always a chance that I'll be inspired to write more in the upcoming time.
If not, I'm pretty ok with where I've gotten.

\section{Draft 1, Realized there's some critical framing errors}
I found out today that there are optional first and second readings for Years B and C for today's readings.\footnote{revise this sentence please}
In the Year B readings, we focus on Abram\footnote{Abraham? I never know how we're supposed to refer to the biblical characters who get renamed when we discuss their actions before being renamed} and the promise the Lord gave him, that he would have descendants as countless as the stars.
As a person tangential to astronomy, that's an interesting thought to me.
Every time we look in what should be a dark region, we see more stars, fainter and further away.

The Second Reading comes from St. Paul, who reminds us of the first reading.
Abraham is gifted children countless beyond any number because of his trust in the Lord.

\end{document}