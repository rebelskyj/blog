\documentclass[12pt]{article}[titlepage]
\newcommand{\say}[1]{``#1''}
\newcommand{\nsay}[1]{`#1'}
\usepackage{endnotes}
\newcommand{\1}{\={a}}
\newcommand{\2}{\={e}}
\newcommand{\3}{\={\i}}
\newcommand{\4}{\=o}
\newcommand{\5}{\=u}
\newcommand{\6}{\={A}}
\newcommand{\B}{\backslash{}}
\renewcommand{\,}{\textsuperscript{,}}
\usepackage{setspace}
\usepackage{tipa}
\usepackage{hyperref}
\begin{document}
\doublespacing
\section{\href{todays-gospel.html}{Today's Gospel}}

\section{Draft 3}
Today's Gospel\footnote{24th Sunday of Ordinary Time in Year B} reading features one of the two lines that I find most striking in the Gospels.\footnote{the other comes during the Easter season}
Jesus exhorts Peter, \say{Get behind me, Satan!}\footnote{Mark 8:33}

Peter, the man who\footnote{whom? I'm not wholly sure how whom is used in the modern English language} Jesus loved and trusted so much so that he entrusted the Church to him, is called Satan.\footnote{the great betrayer}
To me, this truly shows two of the important pieces of my Catholic faith: we are to act and speak as we see true, not always meekly or gently, and that we are to love the sinner and hate the sin.
Jesus doesn't reproach Peter in soft words, or calmly.
In fact, he doesn't even do it kindly.
In no uncertain words, he tells Peter that he is sinning.

Nonetheless, 6 days later, he takes Peter to the mountain where he meets with Elijah and Moses.\footnote{Mark 9:2}
Even though the Gospels don't mention it, clearly Jesus forgave Peter for his actions, and Peter tried to accept this change.

This reading particularly speaks to me in today's climate.
We tend to take neither of the two messages we are told to take.
We don't tell the people we love to their faces that what they are doing is wrong.\footnote{the \say{people we love} is important, because there's no shortage of telling those we don't know or care about that we disagree with them}
We also don't do the other side of the message, and forgive those who do wrong.
I'm as guilty of this as anyone.
I judge quickly and quietly, then discount anything that someone who has spoken out of ignorance has to say.
Today's reading was a good reminder to me that I need to try harder to love, even when it's hard.

\section{Draft 2}
Today's Gospel\footnote{24th Sunday of Ordinary Time in Year B} reading features one of the two lines that I find most striking in the Gospels.\footnote{the other comes during Easter}
Jesus exhorts Peter, \say{Get behind me, Satan!}\footnote{Mark 8:33}

Peter, the man who\footnote{whom? I'm not wholly sure how whom is used in the modern English language} Jesus loved and trusted so much so that he entrusted the Church to him, is called Satan, the great betrayer.
To me, this truly shows two of the important pieces of my Catholic faith: we are to act and speak as we see true, not as we see convenient, and that we are to love the sinner and hate the sin.
Jesus doesn't reproach Peter in soft words, or calmly.
Nonetheless, 6 days later, he takes Peter to the mountain where he meets with Elijah and Moses.\footnote{Mark 9:2}
Even though the Gospels don't mention it, clearly Jesus forgave Peter for his actions.

Overall, this reading speaks to me, especially in today's climate.
We tend to take neither of the two messages we are told to take.
We don't tell the people we love to their faces that what they are doing is wrong.\footnote{the \say{people we love} is important, because there's no shortage of telling those we don't know or care about that we disagree with them}
We also don't do the other side of the message, and forgive those who do wrong.
I'm as guilty of this as anyone.
I judge quickly and quietly, then discount anything that someone has to say.

\section{Draft 1}
Today's Gospel reading features one of the two lines that I find most striking in the Gospels.
Jesus exhorts Peter, \say{Get behind me, Satan!}\footnote{Mark 8:33}

Peter, the man who\footnote{whom? I'm not wholly sure how whom is used in the modern English language} loved and trusted so much so that he entrusted the Church to him, is called Satan.
This is truly the best example of Jesus saying that we are to love the sinner, even if we abhor the sin.
Moreso, it points out to me that we have the responsibility to help those around us who make mistakes.
6 days later, he takes Peter to the mountain where he meets with Elijah and Moses.\footnote{Mark 9:2}
Even though the Gospels don't mention it, clearly there was some conversation during those 6 days where the misunderstanding was resolved.
\end{document}
