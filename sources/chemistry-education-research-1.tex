
\documentclass[12pt]{article}  
\newcommand{\say}[1]{``#1''}  
\newcommand{\nsay}[1]{`#1'}  
\usepackage{endnotes}  
\newcommand{\B}{\backslash{}}  
\renewcommand{\,}{\textsuperscript{,}}  
\usepackage{setspace}   
\usepackage{tipa}  
\usepackage{hyperref}  
\begin{document}  
\doublespacing  
\section{\href{chemistry-education-research-1.html}{Reflection on Chemistry Education Readings Week One}}  
First Published: 2025 July 14

\section{Draft 2: 14 July 2025}

A dear friend of mine asked me if I would be interested in doing an independent audit of a course on Chemistry education.  
That is, we will do the class readings and then discuss between the two of us what we think.  
I'm generally interested in pedagogy and teaching, and so agreed.  
Our general plan is to, starting today, write a weekly reflection on the week's readings, and then in some form or another also respond to the other's reflection.

Going into this project, I realize that it's important for me to state some things I'm coming in with.  
I more and more believe that the value of a good is its goodness, and that arguments towards utility are pointless at best.  
I also had a friend in my first year of the program who had personal/scientific\footnote{depending on how one defines either} issues with the professor for the course, so I am also bringing in some hesitation to trust, at the very least, the framing of the readings.  
This past Thursday I gave a lecture to some high school students about reading science, and the two things I wanted to stress to them were \say{there is objective reality} and \say{any presentation of reality is fundamentally biased}.

This week's readings were about broad level \say{science literacy}, both by the same author, written a decade apart.  
The first came from 2010, where he argues that \say{science literacy} is a meaningless term and offers\footnote{or at least, claims to offer} a way to restore meaning to the term.  
The second comes from 2020, and he focuses here\footnote{with a coauthor, it should probably be noted} on the idea that we (only now) live in a society more focused on emotional appeal than scientific reality and consensus.  
A lot of my issues with the first paper\footnote{unsubstantiated claims, disagreement with what feels like obvious reality} were fixed in the second, which makes the choice of the first into the curriculum interesting to me.

In general, the claim of the first is that science should be taught in a way which is useful, by which the author means either \say{helps people lead happier, more successful, or more politically savvy lives} or \say{help people solve personally meaningful problems in their lives, directly affect their material and social circumstances, shape their behavior, and inform their most significant practical and political decisions}, which are, in my mind at least, kind of very different claims.  
He also never defines science, which feels more than a little concerning to me.  
Saying that people are and are not using science well requires defining science.

The first paper did point me to some interesting ideas of what science literacy can mean, and helped me to recognize that I treat it and most forms of literacy as a fundamentally communal process.  
He claims here (though, it must be admitted, explicitly recants the claim in the other paper) that \say{(o)ne fair critique... is that Roth and Lee appear to have started with the assumption that knowledge is collectively held and meaning socially constructed.}  
While I personally believe that meaning is an absolute, coming from the Divine, I don't think that I would feel comfortable arguing that in a paper about science education.  
The first paper also reminded me of the fact that I apparently read more into subtle racism and classism than others in my life.  
The author is much more derogatory of the scientific relationship that \say{minority youth in high poverty urban environments} had than the relationships either other group had.  
At the end, he explicitly advises for trying to produce \say{competent outsiders}, as opposed to \say{marginal insiders}, and his bias as a biologist by training comes up, where he says that most who learn science have an \say{understanding of science (that) is fairly primitive, extending to experimentation but excluding probability and peer-review, and utterly neglecting the long and messy labors of authentic scientific work}, as though science must require pain and probability.

The second paper was much more practical, for all that it was still of minimal utility to me, personally.\footnote{perhaps because I already know many things}  
One of the big focuses in the article is that science does not produce \textbf{\textit{Truth}} (TM).\footnote{he doesn't emphasize it like this, but he might as well}  
In four columns of text, he somehow does a worse job of explaining that concept than the three second version my college professor gave \say{science is a model, not reality}.

In both, he argues for the concept that we should teach things which students will find personally meaningful, and that we should explicitly frame the topics in that way.  
Thinking about pedagogy, I have trouble with this, though I should probably ask my brother, who reads much pedagogy research.  
The utility of multiplication, in my mind, is highest when it's an unconscious thought.  
He talks about how discussions around heating bills are unscientific, but I have to wonder how different the group studied (elderly) and a younger cohort might treat it.  
Nearly everyone I know above a certain age has a good amount of multiplication tables memorized still.  
Nearly everyone I know below a certain age has to do multiplication for everything.

Type 1 and Type 2 thinking, or fast and slow thinking, or heuristic and thoughtful thinking are all ways to classify what neuroscience has apparently found\footnote{and my own experience has held up} are the major modes we use in life.  
The more innate knowledge becomes, the more it can move into the fast portion.  
Forcing me to, by rote, learn each letter and word means that I now look at words and do not need to think to understand them.  
Forcing children to do times tables means that they do not have to do the math of \say{I have three dollars, how many avocados can I get if they're 50 cents each}, but that they can just look and go \say{50 cents and three dollars, 6 avocados}.  
I would never think of framing the use of multiplication in that way, but I think that this points to the more fundamental issue that I take with this framing.

My grandmother once was on a television debate where she argued that everyone should get a liberal arts education.  
Her main thrust was that it fundamentally makes you a better person.  
I don't know why the author seems so resistant in the first paper to the idea that there are things which science cannot answer.  
He accepts this in the second, and yet still seems focused on the idea that the value of science education is in the measurables, rather than the intangibles.

My friend's reflection focused on Physical Chemistry, and the fact that we can't really make it applicable.  
My initial response is that we don't take Physical Chemistry until upper level students in a major, by which point there's a tacit (though very often incorrect)\footnote{parentheticals in text are side notes, footnotes are bonus thoughts} assumption that major students will use the major in life.  
The utility I have for quantum mechanics in my day to day life is in the work that I do.  
I also know that I very often approach questions in life from a quantized view now, in a way that I was taught out of in schooling.\footnote{at some point I'll write my rant about how Newtonian Physics makes what should be intuitive fundamentally alien}

I'm cautiously excited to keep reading, even if I don't know if I can dedicate the same time to these readings going forward.\footnote{then again, two hours really isn't that much, is it?}

\section{Draft 1.1: 14 July 2025 (Draft 1 got a little too aggressive, I think)}

A dear friend of mine and I have decided to read through the syllabus for a course that neither of us will be able to take on Chemistry education.  
More than that, we plan to each separately and then together discuss the readings for a given week.  
This is the first of that series.

We read two articles on science education by the same author, written a decade apart.

In general, I found minimal of use in either.  
The first was constantly contradicted by the second, and both suffered the same fundamental issue: without defining science, there's no good way to define science literacy.\footnote{at this point I read my friend's review of the readings so that I can see what they did}

In general, the first article, written in 2010, focused a lot on the idea that science literacy should be judged on its \say{usefulness}, which is defined very narrowly and individually.  
I take huge umbrage with this approach; literacy is fundamentally about interaction with the other.  
Of course groups of people are more likely to be using science than an individual in an individual experience; when looking at the problems cited\footnote{nearly all social science}, aggregate action is more important than individual.

Throughout both articles, I also saw a disheartening and concerning trend of unquestioned bias.  
When examining a case study from UK residents or Washington State residents, the idea of scientific interpretation never arose.  
When speaking on a study of \say{minority youth in high poverty urban environments}, by contrast, the author took pains to emphasize that the students were not necessarily taking science information in accurately, writing (with original emphasis) \say{Is it sufficient that they felt more comfortable with and interested in science \textit{as they interpreted it?}}.  
He also presumed that these students will cease to be interested in the specific subjects they have taken a love of.

My friend's reflection comments on the fact that Physical Chemistry, by most definitions and explanations, isn't useful to the everyday experience of people.  
Truthfully, I understand this, but feel like Physical Chemistry\footnote{capitalized because proper noun in my mind} is like mathematics: the value in learning it is in the operations that can become thoughtless.  
I think that it's incredibly important to teach times tables, not because I think that timed tests are great\footnote{even though I did love crushing my peers}, but because of something that a math teacher told me when I was in high school.  
The gist of it was that when things are ingrained, they are barrierless to being done.

\section{Daily Reflection}

\begin{enumerate}

\item Did you journal a full page today?

No, I stopped at a little under half of a page today.  
I don't know why I thought/think that one full page is the appropriate amount to be writing, but I think that most of the advice in regards to daily journaling assumes that I will be using an A5 or smaller notebook, not the letter size college gridded page.

\item What ink/pen did you use, and what were your thoughts on it?

Shark pen, which currently has Private Reserve's \say{Purple Mojo}.  
It's a nice purple, deep and rich which I like. Photographs a little bluer than it looks, and minimal shading.

Earlier this week, in the Pilot Platinum Preppy, I journalled with Private Reserve's \say{Electric DC Blue}, which is far less aggressive than the name might imply.  
It had almost no shading whatsoever, and a really really nice color.  
It's a little hard to describe it, but a dear friend suggested normal roller pen blue, which felt slightly off.  
A slightly more refined version of that, maybe, gets the gist across.  
Doesn't photo well.

\item How's prayer?

Terrible. I tried my best to do a rosary on Saturday and had to stop halfway through the Creed.  
Might be something that's worth pushing through this week, though, since I do feel pretty emotionally numb.

\item How's focus?

Eh, it's been ok.  
On Friday I wanted to get some work done but instead spent four or so hours trying to figure out what order to practice letters in.  
Since I don't write individual letters, though, it seemed more reasonable to figure out the relative frequency of not just letters, but of small combinations\footnotes{bigraphs, trigraphs, and what I must assume are quatragraphs. Internet suggests tetragraph or quadgraph, which both also feel reasonable enough.} of the letters too.  
Once I'd figured that out, I realized that I probably wanted to remove the longer strings from the weight of the shorters.  
After all, if every instance of \say{q} is followed by a \say{u}, then practicing \say{qu} will get me the proper weight of \say{q}.  
In general it made little difference, though subtraction did mean that more longer phrases entered the top of the page.

Outside of that, I've been pretty blase.  
I know that I say that I can use travel days as a place to get a lot of work done, but I struggled with that this week.  
I think that much of that comes down to how tired I was/am, and the rest comes from the general sense of aimlessness I have right now.\footnote{which would be a good thing to folly on (hmmm folly on is a terrible phrase, think of something better)}

\item How's sleep?

I slept just so very very much this past weekend.  
I think that a good portion of that was because I did actually need to catch up on sleep/recover from/prevent an illness.\footnote{hmmm, how do I use slashes in lists? feels like there should be a space but then the kerning looks weird to me}  
At some point, though, I have to wonder if the sleep was itself becoming a problem.  
Hotel beds are just way too cozy, I guess.

Other than that, back at home last night I did not sleep too well.  
My watch says that of the 10.5 hours I gave myself, I was asleep for 5, which feels not entirely wrong.  
I don't feel too bad now that I'm out of bed, though, which is nice enough.

\item How many meals, and how balanced?

In general I do eat pretty well when I'm traveling, and this past weekend was no exception.  
Thus far this morning I've had an iced coffee\footnote{hmmm, does drink count as food?}, most of a two-day-old donut, and some water.  
I'm getting fed a lunch at noon, though, which I have every intention of having been balanced.

I'm still living maybe too much like a stereotype of my demographic and relying on take-away pizza.  
However, I have food in my fridge and a desire to not let it go to waste.  
Might be great to just grab the carrots for munching season.

\item How's the posture?

Eh.  
Currently sitting a little straighter for the reminder, but in general, my neck feels incredibly tight and misplaced and my shoulders are forward.

\item How's the breath?

Thoughtless, by and large.  
Taking these two breaths, though, felt great.

\item How's the movement?

Minimal intentionality, which is fair enough.  
There's a VR fitness thing I've been enjoying lately, so I will plan to work on that tonight, I suppose.

\item How's the physical flexibility?

Terrible. My shoulders feel so incredibly tight, and my neck does as well.  
It's a struggle to touch my toes.  
It's almost painful to stretch a lot of the time, now.\footnote{might be that the fascia is overworked or something}

\item How's keeping up with the family obligations?

Decent! I listened to the album last week and met with the brothers.

\item How's the thesis?

Decent! I'm hoping to finish a draft of the final chapter needed for the minimum viable thesis this week, and otherwise make the rest of the thesis cleaner.\footnote{read: add in the citations and start making figures}

\item How's the poetry?

Nonexistent, which is more than a little sad. I don't know why, when I'm sitting and aimless, I forget that poetry is a thing that I wast to write.

\item How're the interpersonal relationships?

Eh, not a lot of changes, which also means that I haven't really been seeing or interacting with people as much as I generally want.  
I'd already been planning to work on that today, which means this is just a good secondary reminder to do so.

\item How's the music?

Honestly decent! I have plans to jam tomorrow and play at an open mic on Thursday, which should be really fun.  
I've been practicing the music for the show as well as the wedding song, and both have been going well!

\item How's the other writing?

Nonexistent. The journaling didn't even really happen, and as might be obvious, neither did this blog.

\item How's the cleaning?

Home is cleaner than it has been, and I have hopes and ideas for how to continue along this trajectory.  
Car needs to be emptied more than anything, but that's often true.

\item How's ordering the life?

Eh.

I realized today that my ideal storage/sorting system is one wherein I can explode out things in physical space, which is not realistic.  
When I have folders, the fact that there's a barrier\footnote{opening the folder} somehow does a lot to keep me from using it.  
A concept could be to have folders for after I've written on the paper, which might work?

I'll try it.

Outside of that, though, I have an idea for keeping my mind map real and visible.  
Right now I've got a fair number of tasks, and many of them are interrelated.  
I'm thinking about getting notecards and stringing them together with relative importance and order.  
Then, the more things that are dependent on a process,  the more that I can and should be working on them, even if the task itself feels low importance.

I'll see if it helps, but I'm also continuing to consider bullet journaling.

\item Water?

Honestly pretty decent.  
I was feeling bad Saturday night so started drinking water with more intention, and that helped.  
The air right now isn't great, and so I've been wanting more water just for that reason.

\end{enumerate}

\end{document}