\documentclass[12pt]{article}[titlepage]
\newcommand{\say}[1]{``#1''}
\newcommand{\nsay}[1]{`#1'}
\usepackage{endnotes}
\newcommand{\1}{\={a}}
\newcommand{\2}{\={e}}
\newcommand{\3}{\={\i}}
\newcommand{\4}{\=o}
\newcommand{\5}{\=u}
\newcommand{\6}{\={A}}
\newcommand{\B}{\backslash{}}
\renewcommand{\,}{\textsuperscript{,}}
\usepackage{setspace}
\usepackage{tipa}
\usepackage{hyperref}
\begin{document}
\doublespacing
\section{\href{reflections-on-readings-26-ordinary-c-22.html}{Reflections on Today's Gospel}}
First Published: 2022 September 25

Luke 16:25 \say{Abraham replied, \say{My child, remember that you received what was good during your lifetime while Lazarus likewise received what was bad; but now he is comforted here, whereas you are tormented.}}

\section{Draft 1}
Every so often as I read the Bible, I somewhat understand where the prosperity gospel comes from.
Then you get readings like today.
It's hard to think of a harsher condemnation of that heresy, even coming more than a thousand years earlier.
This Gospel tells the tale of Lazarus, a poor man, and a rich man.

We know nothing about why Lazarus was poor, or why the rich man was rich.
It's possible that the rich man made all his wealth only in the most noble of ways.
The rich man never chides Lazarus in the parable.
He never says a single thing ill to him.
It's not even clear in the parable if Lazarus ever even asked for aid from the rich man.

At the end of the day, however, all that only leads to further the message of the parable: it is not enough to give when you are asked.
If there is a hungry person at your door you are to feed them.
When you see the sick, you are to care for them.
It's both an incredibly simple and incredibly difficult command from the Gospel today.

And yet, for as radical as that command seems, we see that it is not a new command from Christ.
The First Reading today is from the Prophet Amos, who speaks similarly about the rich in Zion.

To tie this back to the beginning of my post, the prosperity gospel tells us that the rich are beloved by the Lord and the poor deserve their lot.
Richness is an implication of holiness, and poverty of wealth is a sign of poverty of Grace.
Yet, this reading tells us in no uncertain terms that this claim is utterly contrary to the Lord's Will.
\end{document}