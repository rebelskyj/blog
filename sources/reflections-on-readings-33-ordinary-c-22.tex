\documentclass[12pt]{article}[titlepage]
\newcommand{\say}[1]{``#1''}
\newcommand{\nsay}[1]{`#1'}
\usepackage{endnotes}
\newcommand{\1}{\={a}}
\newcommand{\2}{\={e}}
\newcommand{\3}{\={\i}}
\newcommand{\4}{\=o}
\newcommand{\5}{\=u}
\newcommand{\6}{\={A}}
\newcommand{\B}{\backslash{}}
\renewcommand{\,}{\textsuperscript{,}}
\usepackage{setspace}
\usepackage{tipa}
\usepackage{hyperref}
\begin{document}
\doublespacing
\section{\href{reflections-on-readings-33-ordinary-c-22.html}{Reflections on Today's Gospel}}
First Published: 2022 November 13

2 Thessalonians 3:10 \say{In fact, when we were with you, we instructed you that if anyone was unwilling to work, neither should that one eat.}

\section{Draft 1}
As a brief preface, I really only focus on the Second Reading today, and in particular a single verse from it.
That verse pertains to the nature of work in the Catholic worldview.

A question that I've had for a while is what the Church teaches about intellectual property.
So far as I can tell, it doesn't really have any teachings.
The Church does, however, have a number of teachings on physical property.

One of these is that St. Paul's exhortation requiring people to be willing to work comes with it the duty to give them what they need to work.
I, personally, take that to mean that intellectual property also belongs generally to the public.
Of course, the Church strongly states that we have a right to individual property, not just public use.
The defense of copyright from that perspective is that we have a right to the profit potential of an idea.

For many reasons, that strikes me as a hollow argument.
After all, most of the profit coming from ideas is not from individuals, but from corporations.
Corporations preventing other corporations from using an idea they've created is something I don't have strong feelings about, but I do strongly believe that an individual working for their own benefit should be able to use whatever tools they can that do not harm another.
That is, since sharing information does not take it from the sharer, there is no legitimate cause, in my mind, for people to have the right to hoard knowledge.
Additionally, is the right to profit really a right at all?

I am reminded of a quote from \href{https://incommunion.org/2010/11/24/the-social-doctrine-of-st-basil-the-great/}{Saint Basil the Great}, \say{The one who steals clothes off someone's back is called a thief. Why should we refer to the one who does not clothe the naked, while having the means to do so, as anything else? The bread that you have belongs to the hungry, the clothes that are in your cupboard belong to the naked, the shoes that are rotting in your possession belong to the barefooted, the money that you have buried belongs to the destitute. And so you commit injustice to so many when you could have helped them.}
Just as we would say that you should not forcibly take bread from someone with much to give to one who is little, but the one with much is in the wrong, we should not take the intellectual property from one who has it to give to one who does not.
But, that's the great thing about information.
It isn't taken when it's shared.

I have two more points about St. Paul's quote.
The first is that he does not say that people who do not work should not eat, only that if someone is unwilling to work.
Too often, society takes the assumption that anyone not working must not want to work.
However, doing work that is not fulfilling to you or beneficial to the community is not work that we are called to do.
One unwilling to do this meaningless work should not be punished, at least in my reading of the text.

Finally, just after Paul tells us this, he reminds us that paying attention to the doings of others is also not Christian.
That is, he does not say that we should not feed the person who does not work, but that they should not eat.
He then makes it very clear that it is not our place to decide who is and is not willing to work.
\end{document}