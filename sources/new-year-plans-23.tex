\documentclass[12pt]{article}[titlepage]
\newcommand{\say}[1]{``#1''}
\newcommand{\nsay}[1]{`#1'}
\usepackage{endnotes}
\newcommand{\1}{\={a}}
\newcommand{\2}{\={e}}
\newcommand{\3}{\={\i}}
\newcommand{\4}{\=o}
\newcommand{\5}{\=u}
\newcommand{\6}{\={A}}
\newcommand{\B}{\backslash{}}
\renewcommand{\,}{\textsuperscript{,}}
\usepackage{setspace}
\usepackage{tipa}
\usepackage{hyperref}
\begin{document}
\doublespacing
\section{\href{new-year-plans-23.html}{New Year Plans for 2023}}
First Published: 2022 December 23

Pre-reading note: This is rambly and I am sorry for that.
\section{Draft 1}
Every month I try to post a reflection on the past month and my goals for the upcoming month.
But, all the thinking tends to happen day of, and I generally don't find that I am too reflective for the rest of the month.
Today has the fun thing where planning for next year\footnote{23} and the day\footnote{23} are the same, which I'll retroactively say is what inspired this post.

Looking through my monthly reflections, a few things stick out to me.
More or less every month came with a\footnote{failed} goal of blogging daily.
I don't know why I have it as a goal when I clearly don't value it, as evidenced by my lack of daily blogs.
Maybe I should be smart and start making a list of blog topics to go through on days where my mind is empty.
That's actually a really good idea, so time to make that list.\footnote{currently empty, but many days I post a blog and then realize I wanted to post something else too, and the way the code for the site is written, that's not an option}

I tended to put keeping up with the BiaY podcast on there, which on a long-term level I've kind of done.
I think I'll be able to finish it by the end of the year.
That goal can remain more or less unchanged.

Stretching/exercise of some sort was a recurring goal as well.
I did not do a good job with it, and that's something I don't like.
Unlike blogging, where the rewards for doing so are debatable if extant, there are clear benefits to me.
I feel better, and I look better too.
Plus it's healthy, and I would like to be healthy.
I think I tend to fail to stretch on mornings that I am running late, so having a better sleep routine would help me there.

A lot of my goals fall into the category of writing.
In particular, I often had at least one of the three kinds of writing I do\footnote{prose, poetry, and song} as a goal.
Each of them serves me in a different way.
I've \href{species-counterpoint-planning.html}{already decided} that I would like to do more writing of song, and so I think I will keep that up.
I keep telling myself that I'm going to actually put my prose out into the aether for someone to\footnote{have the option to} read, and I think doing that would help me be better about actually writing more.
Poetry is a strange one.

I really like writing poetry.
I feel like it makes me more aware of how I'm doing mentally and emotionally.
I like that it teaches me how to use words more intentionally.
In less good motivations, I like that people seem surprised that I can write poetry.
I think doing a poem a day would be achievable for me, especially since I have a writeblr\footnote{writing-focused tumblr} that I mostly post poetry on.\footnote{as with any other external creation I've made, readers of the blog are welcome to ask me for it}
More than achievable, though, I think it would be healthy.
Maybe plotting out what poems I would like to write would be a good idea.\footnote{hey look my first thing to blog about}

Practicing music is sort of the last category of recurring goals I had.
I know on an intellectual level that I will improve my instruments faster and more reliably if I play them more.
I feel like, again, having a plan would help me stick to it.\footnote{wow, another scheduled post}

Moving one level deeper into my meta analysis, I've been reflecting on something that a creator I follow mused on.
Their point was that resolutions should be framed less as specific and measurable goals, and more as aspirational concepts.
That is, if your goal is really to get healthier, then \say{be healthier} as a theme is better than \say{run every day}, because of a lot of reasons.
On many levels I agree.
The general argument he makes is that life is a series of choices.
If your choices tend to skew in one direction, that's generally the life you end up living.
By setting your goal as something big and encompassing, you have the slight nudge to the direction you want to face your life.
The other point was that goals often change.
Having the resolution to go to the gym every day gets dropped if you develop repetitive strain.
If your goal was health, though, then you can put that energy into eating better or something.

The other thing that they point out is that years are bad amounts of time for goals.
Instead, they suggest having themed seasons.
I kind of like that, and so in addition to having a yearly and monthly reflection, I would like to also have quarterly reflections.\footnote{shockingly, four seasons means that a season is a quarter}
I'm still not sure how I feel about having seasons related to general aspirations.
But, the nice thing about something like this is that the worst that happens is I don't like it and then I'm free from having to do it again.
More than that, though, if it doesn't work out, I know that it's something I don't need to try in the future.
If it does work, though, then I have a new way of improving my life, which would be nice.

So, to summarize:
\begin{itemize}
\item I would like to blog every day. I need to find a way to make that easier on myself. My current thought is having a backlog of ideas.
\item I would again like to do the BiaY, and this year I'd like to add CCCiaY. There's not much to say there
\item I would like to be better about exercising, and need to figure out how to make that easier on myself. My current thought is having a better sleep schedule or fewer hard morning times.
\item I would like to write more and in the varied forms of prose, poetry, and song.
I've already worked out how I'll do the song, and that's been really helpful for me\footnote{for the four days that I've done it}.
I think that the same would be true if I made goals for prose and poetry, so they go on the backlog.
I also think that sharing my prose and poetry with the world would help me to write more, so I'll try to do that.
\item I would like to practice music more.
I think that having a plan would help with that.
\item I will try having monthly, quarterly, and yearly goals.
The quarterly goals will be nebulous concepts such as \say{Reading,} or \say{Health}, which I will use to help shape the way I live my life.
\end{itemize}


\end{document}