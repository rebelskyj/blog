\documentclass[12pt]{article}[titlepage]
\newcommand{\say}[1]{``#1''}
\newcommand{\nsay}[1]{`#1'}
\usepackage{endnotes}
\newcommand{\1}{\={a}}
\newcommand{\2}{\={e}}
\newcommand{\3}{\={\i}}
\newcommand{\4}{\=o}
\newcommand{\5}{\=u}
\newcommand{\6}{\={A}}
\newcommand{\B}{\backslash{}}
\renewcommand{\,}{\textsuperscript{,}}
\usepackage{setspace}
\usepackage{tipa}
\usepackage{hyperref}
\begin{document}
\doublespacing
\section{\href{pierogi-recipe.html}{A Pierogi Recipe: Or Why I Pronounce Ukulele the Way I Do}}
\section{Draft Two}
At first glance, today's title may seem like a non sequitur.\footnote{Lat. \say{it does not follow}}
What do pierogi\footnote{Eastern European potato dumplings} have to do with the ukulele?\footnote{A Hawaiian lute \href{https://en.wikipedia.org/wiki/Ukulele}{probably} based off of a Portuguese lute}
No, this title isn't\footnote{and wasn't} written as an intentional non sequitur to spark my creative energies.
Rather, I realized that the same thought processes inform my choice of both making pierogi and pronouncing ukulele.

To understand what I mean by that, at the bottom of today's essay are two pierogi recipes.\footnote{also, for those of you who have asked for my recipe}
One of them is a recipe I found online,\footnote{available \href{https://thestayathomechef.com/pierogi/}{here}} and the other is the recipe I used last time I made pierogi.\footnote{as best as I can recall/with mistakes I realized at the time fixed}
As a bonus, I also include the gluten free pierogi dough recipe I used.\footnote{for those of you who may want that for whatever reason}

If you've read the recipes, you might notice a few differences between the two recipes.\footnote{no, the fact that my filling has far more fat isn't the relevant portion here, despite the fact that it absolutely makes it taste better}
The biggest difference to me is in the dough.\footnote{it helps that every recipe for pierogi acknowledges that you should substitute the filling for your own favorite}
They use both sour cream and water in their dough, while I don't use either.
My reason for this comes from my formative years, when I learned to make pasta.
According to my teacher,\footnote{I don't remember who, but I assume one of my parents} Italian pasta dough\footnote{read: good pasta dough} only contains salt, egg, flour, and oil.

Since that dough has worked for everything I ever needed a dough for,\footnote{especially as a base for other additions to the dough when I feel creative} I never felt the need to use other pasta doughs.\footnote{with the exception of adding pepper, because I always put pepper in savory things with salt, which comes from an earlier formative experience}
So, when I made my pierogi, I just made dough the way I always do.\footnote{which really means just eyeballing everything and assuming it will work}

Now, by this point you might be asking what making dumplings has to do with the pronunciation of ukulele.\footnote{you also might be wondering if I'm getting hungry writing about this, and the answer is absolutely}
To me, they're two reflections of the same universal truth: people adapt everything they learn to fit into their prior knowledge and experiences.

In IPA\footnote{international phonetic alphabet} notation, the word ukulele is pronounced\footnote{in Hawaiian} /?uku?l\textepsilon{}l\textepsilon{}/, which contrasts to the SAE\footnote{Standard American English} pronunciation of /ju?k\textschwa?lejli/.
This is due to the nature of word movement through languages.
When a word is assumed into a new language, it\footnote{tends to} change its pronunciation to align to the new tongues phonological rules.
In Hawaiian, there is no schwa,\footnote{\textschwa{}} while SAE uses the schwa almost exclusively in unstressed syllables.
Additionally, SAE almost never begins words with vowel /u/\footnote{the ou in you}.
So, when the Hawaiian word entered the SAE lexicon, the /j/\footnote{like the y in you} was added, to make it align with the rules of the language.

Now, both of these changes to the word are fine and natural, at least to me.\footnote{and descriptivist users of language (read: most linguists)}
But, I've heard and seen complaints about the both how to pronounce ukulele, and, indirectly about how to make pierogi.
To be specific, I've been told that the way I pronounce the word \say{ukulele} is wrong, since the word is pronounced differently in the language of origin.
Additionally, there are many people who feel that changing a culture's recipe\footnote{especially when you don't belong to that culture (and no, I don't identify as Eastern European)} is wrong.\footnote{for reference: \href{https://www.npr.org/sections/thesalt/2016/03/22/471309991/when-chefs-become-famous-cooking-other-cultures-food}{this site} is one example, and \href{http://www.baystbull.com/in-food-culture-is-appropriation-actually-possible/}{this site} offer some perspectives}

To me, both of those erase the idea of positive change, which is makes the world beautiful and exciting.
Should I stop calling what I make pierogi because I don't use an \say{authentic} recipe?\footnote{side note: what defines authentic?}
Should I try to relearn the phonological system of my native tongue to accommodate a single word?\footnote{or pronounce everything as if it belongs to its original language}
I personally don't think so, and that's where today's title comes.
Both the way I make pierogi and the way I pronounce ukulele come from the background and mental connections I made before encountering the idea.

\section{Draft One}
At first glance, the title may seem like a non sequitur.\footnote{Lat. \say{it does not follow}}
What do pierogi\footnote{Eastern European potato dumplings} have to do with the ukulele?\footnote{A Hawaiian lute \href{https://en.wikipedia.org/wiki/Ukulele}{probably} based off of a Portuguese lute}
No, this title isn't\footnote{and wasn't} written as a non sequitur as a way to spark my creative energies.
Instead, while reflecting on my recipe for pierogi, especially in context with \say{authentic} recipes I've read, it occured to me that I feel similarly about pierogi and the pronunciation of the word ukulele.

To understand what I mean by that, below are two pierogi recipes.
One of them is the recipe I found online,\footnote{available \href{https://thestayathomechef.com/pierogi/}{here}} and the other is the recipe I used last time I made pierogi.\footnote{the recipe came from my spur of the moment actions}
As a bonus, I'll also put in the gluten free pierogi dough recipe I came up with.\footnote{for those of you who may want that for whatever reason}

So, you might notice a few differences between the two recipes.\footnote{no, the fact that my filling has far more fat isn't the relevant portion here}
The biggest difference to me is in the dough.
They use both sour cream and water in their dough, while I don't use either.
This mostly comes from when I first learned how to make pasta.
According to the person teaching me,\footnote{I honestly don't remember who, but I assume one of my parents} Italian pasta dough\footnote{read: correct pasta dough} only has salt, egg, flour, and oil.

Since that dough has worked for everything I ever needed a dough for,\footnote{especially as a base for other additions to the dough} I never felt the need to change it.
So, when I made my pierogi, I just made dough the way I always do.

Now, by this point you might be asking what that has to do with the pronunciation of ukulele.
To me, they're two reflections of the same universal truth: people adapt everything they learn to fit into their prior knowledge and experiences.

In IPA\footnote{international phonetic alphabet} notation, the word ukulele is pronounced\footnote{in Hawaiian} /?uku?l\textepsilon{}l\textepsilon{}/, which contrasts to the SAE\footnote{Standard American English} pronunciation of /ju?k\textschwa?lejli/.
This is due to the nature of word movement through languages.
When a word is assumed into a new language, it\footnote{tends to} change its pronunciation to align to the new tongues phonological rules.
In Hawaiian, there is no schwa,\footnote{\textschwa{}} while SAE uses the schwa almost exclusively in unstressed syllables.
Additionally, SAE almost never begins words with vowel /u/\footnote{the ou in you}.
So, when the Hawaiian word entered the SAE lexicon, the /j/\footnote{like the y in you} was added, to make it align with the rules of the language.

Now, both of these changes are fine and natural, at least to me.
But, I've heard and seen complaints about the second,\footnote{how to pronounce ukulele} and indirectly about the first.\footnote{making pierogi}
To be specific, I've been told that the way I pronounce the word \say{ukulele} is wrong.
Indirectly, there are many people who feel that changing a culture's recipe\footnote{especially when you don't belong to that culture} is wrong.\footnote{for reference: \href{https://www.npr.org/sections/thesalt/2016/03/22/471309991/when-chefs-become-famous-cooking-other-cultures-food}{this site} is one example, and \href{http://www.baystbull.com/in-food-culture-is-appropriation-actually-possible/}{this site} offer some perspectives}
To me, both of those erase the idea of change, which is at the heart of what makes the world beautiful and exciting.
Should I stop calling what I make pierogi because I don't use an \say{authentic}\footnote{side note: what defines authentic?} recipe?
Should I try to relearn the phonological system of my native tongue to accommodate a single word?
I personally don't think so, and that's where today's title comes.

\section{Recipes}
\subsection{From Internet}
For the dough:
\begin{itemize}
    \item 3-1/2 cups all-purpose flour, plus more for dusting
    \item 3 large eggs
    \item 2 tablespoons sour cream
    \item 3/4 cup water
\end{itemize}

Filling:
\begin{itemize}
    \item 2 cups mashed potatoes
    \item 1/2 teaspoon garlic powder
    \item 1/2 teaspoon onion powder
    \item 1/2 teaspoon salt
    \item 1/4 teaspoon black pepper
    \item 1 cup grated cheddar cheese
\end{itemize}

Since pierogi making instructions, like most dumplings, are effectively just \say{make dough,} \say{make filling,} \say{put filling in dough,} I elide them here.

\subsection{From Memory}
For the dough:
\begin{itemize}
    \item 1 egg
    \item olive oil in equal amount to the volume of egg
    \item sprinkle salt
    \item sprinkle black pepper
    \item flour to correct consistency
\end{itemize}

For the Gluten Free Dough:\footnote{as an aside, I find it odd that people say gluten free doughs require tons of extra effort.
Yes, with bread or things that need to trap gasses in them to rise, it's difficult to get the correct blend of protiens, but for something like a dumpling shell, as long as it's not water soluble and is neutrally flavored, it probably will work}
\begin{itemize}
    \item 1 egg
    \item olive oil in just under amount of egg volume
    \item sprinkle salt
    \item sprinkle black pepper
    \item white rice flour to correct consistency
\end{itemize}

Filling:
\begin{itemize}
    \item 5 pounds potato, peeled\footnote{that means 5 pounds before you peel them}
    \item 1 lb cream cheese
    \item 1/2 pound butter
    \item 2 red onions
    \item 5 cloves garlic
    \item olive oil as needed
    \item sharp cheddar to taste
    \item salt and pepper to taste
\end{itemize}

To make the filling, boil the potatoes, mince garlic and fry with onion in olive oil until caramalized.
Drain potatoes when soft, then add everything together, mixing well.

To make the dough, all ingredients but flour in a bowl, then slowly add flour until solid enough to handle.
Move onto a floured workspace and add more flour until the correct consistency.
\end{document}