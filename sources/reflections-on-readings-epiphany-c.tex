\documentclass[12pt]{article}[titlepage]
\newcommand{\say}[1]{``#1''}
\newcommand{\nsay}[1]{`#1'}
\usepackage{endnotes}
\newcommand{\1}{\={a}}
\newcommand{\2}{\={e}}
\newcommand{\3}{\={\i}}
\newcommand{\4}{\=o}
\newcommand{\5}{\=u}
\newcommand{\6}{\={A}}
\newcommand{\B}{\backslash{}}
\renewcommand{\,}{\textsuperscript{,}}
\usepackage{setspace}
\usepackage{tipa}
\usepackage{hyperref}
\begin{document}
\doublespacing
\section{\href{reflections-on-readings-epiphany-c.html}{Reflections on Today's Gospel}}
First Published: 2019 January 6

Isaiah 60:4A: \say{Raise your eyes and look about; they all gather and come to you.}

\section{Draft 1}
Today's readings focus on the first time that we see people recognize the Lord in his human incarnation.\footnote{I think? At least in the book of Matthew}
And, as the other readings point out, this is the answer to Isaiah's prophecy, that \say{[Jerusalem's] light has come, the glory of the Lord shines upon [it].}\footnote{Isaiah 60:1}
And, perhaps because I went to a Mass in Haitian Creole, where the priest made a point of mention that the Mass is the same in any language, other than the words themselves,\footnote{he phrased it better, but I am bad at turns of phrase} that's the part of the readings I've focused on.

The first reading tells us that all will gather and come to see the Lord.
And, as the Mass is the same in every language, we can truly be a catholic\footnote{little c because universal} church, one where all can join in the song.\footnote{yes, I did sing along. No, I did not know the words}
In this way, the message of today's readings is especially relevant to the random happenstance of life.
It's nice when that happens.
\end{document}