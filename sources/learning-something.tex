\documentclass[12pt]{article}[titlepage]
\newcommand{\say}[1]{``#1''}
\newcommand{\nsay}[1]{`#1'}
\usepackage{endnotes}
\newcommand{\1}{\={a}}
\newcommand{\2}{\={e}}
\newcommand{\3}{\={\i}}
\newcommand{\4}{\=o}
\newcommand{\5}{\=u}
\newcommand{\6}{\={A}}
\newcommand{\B}{\backslash{}}
\renewcommand{\,}{\textsuperscript{,}}
\usepackage{setspace}
\usepackage{tipa}
\usepackage{hyperref}
\begin{document}
\doublespacing
\section{\href{learning-something.html}{Learning Something}}
First Published: 2018 December 28
\section{Draft 1}
As some of you may know, as a way to kill time and attention, I sometimes draw celtic knotwork.
Today, I tried to make a shape that included some interesting breaks.
Unfortunately, I realized that the shape required more than one line.
To remedy this, I tried some different patterns before asking my father.

After a quick Google search, we learned that\footnote{for a simple rectangle} the number of lines required for a knot is the greatest common divisor of the length and width of the knot.
Unfortunately, that didn't end up being incredibly useful for my piece, but it's still cool to learn.
\end{document}