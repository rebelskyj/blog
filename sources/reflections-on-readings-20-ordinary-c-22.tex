\documentclass[12pt]{article}[titlepage]
\newcommand{\say}[1]{``#1''}
\newcommand{\nsay}[1]{`#1'}
\usepackage{endnotes}
\newcommand{\1}{\={a}}
\newcommand{\2}{\={e}}
\newcommand{\3}{\={\i}}
\newcommand{\4}{\=o}
\newcommand{\5}{\=u}
\newcommand{\6}{\={A}}
\newcommand{\B}{\backslash{}}
\renewcommand{\,}{\textsuperscript{,}}
\usepackage{setspace}
\usepackage{tipa}
\usepackage{hyperref}
\begin{document}
\doublespacing
\section{\href{reflections-on-readings-20-ordinary-c-22.html}{Reflections on Today's Gospel}}
First Published: 2022 August 14

Jeremiah 38:4 \say{Then the princes said to the king, \nsay{This man ought to be put to death. He is weakening the resolve of the soldiers left in this city and of all the people, by saying such things to them; he is not seeking the welfare of our people, but their ruin.}}

\section{Draft 1}
Again I start this reflection reflecting on the lack of cheer.
All three readings point to a single common element today: the Peace we are promised is not temporal.
That is, though living a Christian life will bring us peace, that does not mean we will live without conflict.
Each reading points to a different way in which our struggles to live as Christ commands will be redoubled by the world.

In the first reading, we hear Jeremiah being condemned by the nobility in the city.
In the context of today's readings, this reminds us that to live a holy life is to be persecuted for righteousness.
Now, this isn't to say that people being mad at your choices means they're holy, there are plenty of ways to be persecuted rightly.
But, to truly live a life centered around the Lord's will means going against the powers of the day.

The second reading reminds us that, though our spiritual struggles are painful, we are spared from the worst of it.
Indeed, there is no way that we could redeem ourselves through merit alone.
It is through the grace of our Lord alone that we have the strength to resist sin, and it is through His sacrifice that we are redeemed.
This reading reminds us that, though we can always strive more, we should strive always to model Christ.

The final reading is in many regards the hardest.
The Church has always had enemies with temporal power, and hopefully there's no need to be reminded that Christ suffered for us.
Instead, the Gospel today tells us that even the people we love most in this world may be at odds with us in following a holy life.
There's an expression I've heard, that all love on Earth is only good in so far as how it orients us to He who is Love, but having it expressed as Christ does showcases the pain.
There is a more joyful reading though: the love a parent has for their child is nothing compared to the love that Love has for us.
\end{document}