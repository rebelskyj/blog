\documentclass[12pt]{article}[titlepage]
\newcommand{\say}[1]{``#1''}
\newcommand{\nsay}[1]{`#1'}
\usepackage{endnotes}
\newcommand{\1}{\={a}}
\newcommand{\2}{\={e}}
\newcommand{\3}{\={\i}}
\newcommand{\4}{\=o}
\newcommand{\5}{\=u}
\newcommand{\6}{\={A}}
\newcommand{\B}{\backslash{}}
\renewcommand{\,}{\textsuperscript{,}}
\usepackage{setspace}
\usepackage{tipa}
\usepackage{hyperref}
\begin{document}
\doublespacing
\section{\href{conclusions.html}{On Conclusions}}
First Published: 2023 December 6
\section{Draft 2}
\href{science-mysticism.html}{Yesterday's musing} finally showed me what my greatest struggle has always been as a writer, especially a persuasive writer: I don't know how to write conclusions.
Let's look at the musing as a case study.

I had an idea: science as mysticism.
I explored what that could mean, at least in a few variations.
I would like to think that I even got so far as maybe showing that science and mysticism are, on some level, at least, reflections of the same thing.
I stopped before saying what that meant though.

As I've worked through this draft, I think that where the musing needed to go was one step further.
The point of the musing was that if science is mysticism, then we can approach truths we know about mysticism to science.
That is, in contrast to the post modern ideal of truth as fundamentally a human\footnote{I've had an argument with myself in the past about how objective morality or truth when you believe in an Almighty Creator is still subjective, just with a definitionally correct subject. For that reason, given that I do believe everything was created, it feels at least a little wrong to say that truth is not on any level constructed.} construction, I believe that truths are something we uncover that point to a greater truth.
I'm not sure whether the word I'm looking for is epistemology or metaphysics, but I know that there is a word for method of determining truth.
Whatever that word is, every one is fundamentally limited, because it relies on defining truth in a way that cannot work for all facts.

Mathematics, being applied philosophy, explains this well.
It can be demonstrated\footnote{Godel did it most famously, and others have followed} that there are always facts that cannot be solved for within a given set of axioms.
One of those facts is the validity of the set itself.\footnote{I think. Mathematicians and philosophers, please feel free to correct this take if it is incorrect.}
We can see that with science as a whole.

Science works because the universe behaves the same way from day to day.
Science cannot prove that this is true.\footnote{e.g. if I claim that everything is constantly doubling in size, there is no way to disprove it, since everything is measured based on a referent.
One level deeper, if we do not believe that memory is real, then there is nothing to say that whatever we record the speed of light as cannot change from day to day. I may remember it as 3e8 m/s, but that might just be because, in this split second, it is.
Whatever we measure it as next might retroactively have always been true.
Reality being, well, real, is a presupposition of science.
Ope I'm realizing now that this should have been maintext, not footnote.}

Because science can only answer questions about measurables, it can obscure just as much as it can illuminate.\footnote{I unfortunately walked away from this musing after the above line to do other stuff, and have now forgotten what my goal was in saying that. Let's try to reconstruct.
Ooh ok yes the whole scientism}
An issue that I've noticed, especially among non religious and non spiritual\footnote{initially this said friends, but I realized that's both inaccurate and a little offensive. the friends I have who hold any of these views have a much more nuanced understanding than I'm fighting here. Is this tilting at windmills and fighting strawmen? probably.} is a tendency towards scientism, solipsism, postmodernism,\footnote{I wish that postmodernism started with an s} or nihilism.
All four\footnote{initially three, but then I remembered that nihilism exists} are understandable, for all that I find each view point fundamentally wrong.\footnote{and unlike as a postmodernist, I can still make truth claims}

What's wrong with each of them, and how does tying science to mysticism help, though?\footnote{at what point does this musing cease to be a reflection on conclusions and start simply being the next draft of the musing?
I think as long as I conclude (hah) this musing with an overall reflection on concluding, I should be fine}
Great question, me.
Let's go through and find out.

Scientism is the claim that only things which are measurable and material are real.
How do concepts like justice work under this viewpoint?\footnote{immediate disclaimer: I will be proposing my own solution to the questions based on my own views and understandings.
As all three worldviews I hold are held by many people with their own diverse views and understandings, they may have different answers to the questions.
If they truly do believe in the worldview, their answers are more valid than mine for how they explain a concept}
As is pointed out in a Discworld book, if you grind the universe into a fine powder and run it through an atomic sieve, you will not find a single mote of justice.

Scientism has a fairly strong defense to justice, however.
We can demonstrate that whatever variable we seek to maximize\footnote{life expectancy, earning power, perceived happiness} is increased or decreased by varying certain conditions.
The visual that there are punishments to crimes leads to a reduction of crime.\footnote{is the argument, not necessarily true. I think I remember seeing somewhere that more intense punishments don't actually reduce crime.
How that works with the whole, \say{free will doesn't exist} that goes along with the concept, I'm not entirely sure.}

By tying science to mysticism, however, this worldview is untenable.
We cannot prove what happens after death with rigorous experimentation.
Spiritual truths cannot be measured with a mass spectrometer.

Second is solipsism.
I don't actually know many people who will verbally admit to this view, for all that it's fundamental to modern understandings of morality and really anything.
Solipsism tells us that, as Descartes points out, the only thing we can know is our own mind.
Everything else could be a figment of our imaginations or anything else.
I have no clue how justice works in solipsism.

Mysticism tells us that there is, in fact, fundamental truth outside of self.
At its core, mysticism is removal of self from the body, if only for a moment.

Third, postmodernism.
There are any number of definitions for postmodernism\footnote{I promise that some of my jokes are good}, but most boil down to a disbelief in absolute truth.
Reality is what we agree upon, and truth claims are fundamentally claims of power.\footnote{I think I mentioned yesterday in a footnote that I don't actually disagree with this claim, it's just that an omnipotent being making a claim of truth has, definitionally, the power needed to make that material reality}
Justice works because those in power convince the rest of us that society functions better when we follow the rules and conventions.

To be fair, mysticism isn't really needed to combat postmodernism.
The fact that science can measure objectively is itself an argument.
By tying it to mysticism, though, we are able to get to the root of the postmodern argument.
Mysticism is connection with the Divine.
Postmodernism fundamentally relies on the absence of anything divine.

Finally, we have nihilism.
Honestly, as a Christian, nihilism is a fairly compelling argument.
As far as everything we can measure suggests, the universe ultimately runs independently of any observation, and in time will die.
The grandest human accomplishments will die when our planet does, and in time every piece of every one of us will become iron, totally inert and dead in a universe as cold and static as the grave.\footnote{the metaphorical grave.}

Mysticism, which reminds that the universe is not an object of itself, but is an object lovingly and constantly sustained by its Creator, reminds us that the small and everyday actions we take do have cosmic importance.

Anyways, I think that going to here could have been a good thing in the previous musing, for all that I still think it doesn't go far enough.
So what, other worldviews are wrong? I already believed that before writing this musing, and I don't think that these arguments are any more convincing than the others against the views.
What is actionable about tying science to mysticism?

I think that what I really needed to do is have a \say{what does this mean for the reader.}\footnote{anyone who's ever been in charge of reading my writing, I apologize that it took me fully a quarter century to learn this fact}
There's an idea floating around in my head that I cannot quite get to crystallize.

I think that it's something about the importance of doing scientific work, as it brings us closer to G-d.
I think there's also something in it about the humility we need to have as researchers.\footnote{ope, yeah, there it is. Looks like I have a conclusion ready}
In most of the musing, I tried to convince that mysticism is like science.
Flipping the paradigm, though, gives the conclusion.

Mystics speak of how immensely small they are in relation to He who Created us.
As a scientist, I need to remember that any revelation I have in my work is just as much a function of the Divine as any holy apparition.\footnote{wow that feels maybe blasphemous. Find a better phrasing}
I am not creating knowledge, I am little more than a child who has been gently led to the smallest trickle of the truth.
I am led by someone who loves me and knows that the rushing torrents of Truth are too much for me right now, and I need to slowly be coaxed towards them.

What does this mean for concluding my musings in general, though?
As we pull back from the case study, I see that I really only do about half of the plotting for a musing that I need to.
I say what I have for an idea, but I don't connect it to a change in worldview.
How should my worldview change here?\footnote{I think that's an improvement, yeah.}

Edit: Upon reflecting after writing this but before posting, I think there's a major element of fear.
The more that I try to finalize an argument, the more of myself I put in it, and the more that I worry about someone disagreeing with my take.

Daily Reflection:
\begin{itemize}
\item Hobbies:
\begin{itemize}
\item Did I embroider today? Nope! Will hopefully remember to take it out of my car, though.
\item Did I play guitar today? I did! I need to tune out of DADGAD and replace strings soon, though.
\item Did I practice touch typing today? I did! Wow I cannot learn c I guess.
\end{itemize}
\item Reading
\begin{itemize}
\item Have I made progress on my Currently Reading Shelf? A little! I'm beginning to debate dropping one of the books, because I don't know if I'm really gaining anything for reading it. It's one of those three hundred or so page books that really just repeats the same idea for pages and chapters on end.
\item Did I read the book on craft? I did! Set aside fifteen minutes and read. It was really interesting, and right now I'm in the section on learning to read better.
\item Have I read the library books? Not yet. Will hopefully do so tonight.
\end{itemize}
\item Writing
\begin{itemize}
\item Did I write a sonnet? It was such an interesting sonnet. I should really explore it more
\item Did I revise a sonnet? I had to pick new rhymes a few times for this one, because as it turns out, neither voice nor farce have a plethora of rhymes.
\item Did I blog? Look at this metablog
\item Did I write ahead on Jeb? I'm most of the way through Friday's chapter, which is technically ahead.
\item Letter to friends? Nope.
\item Paper? I thought about what my paper is missing, which is kind of like work if you squint.
\end{itemize}
\item Wellness
\begin{itemize}
\item How well did I pray? Terribly.
\item Did I clean my space? I've picked all the low hanging fruit, which means that at this point cleaning becomes hard again.
\item Did I spend my time well? Eh, not really. I tried, though, and that's worth something.
\item Did I stretch? Forgot to stretch yesterday and this morning, did so.
\item Did I exercise? Forgot to do so this morning, did the minimum. Planking is starting to get hard.
\item Water? Better? I refilled my bottle once today, at least, which is good.
\end{itemize}
\end{itemize}


\section{Draft 1}
\href{science-mysticism.html}{Yesterday's musing} reminded me of something that I've struggled with in my writing.
I don't know if it's a recent change, or whether it's just something that I didn't care about in the past.
Regardless, I've realized that I struggle with conclusions.

Let's look at yesterday's musing as a sample case.
I had an idea: science is mysticism.
I was able to explore how I felt like the two could be connected.
That is, I worked backwards from the thesis, giving background.

Once I found the background, though, I gave up.
Or, rather, I found that I had nothing else to say.
What does it mean if mysticism and science are fundamentally the same?
One reader summed it up best \say{you stopped just when the musing was starting to get good.}

As I think about the issue, I realize it's something that I consider almost fundamental to the way that I view the world.
Justifications for research have always seemed strange to me.
After all, what reason do we really need for learning something more than \say{we didn't know the answer to this, and now we do}?
Most people do not agree with that take, however, and since I'm getting to the point in my life where I need to start putting out research for the broader public, it's probably worthwhile to think about what the overall goal of anything I do is.

So, let's return to the case study.\footnote{that's the word. I should replace sample case in the above if I redraft this musing}
What does linking science to mysticism do?
I find that I'm immediately struck with the fact that there's not one single answer\footnote{in that there's many answers that could work, not that an answer does not exist at all} that satisfies.
I think that I tend to take the easy way out when there are a plethora of potential options.
Rather than explore a single path, acknowledging that I will necessarily not travel down the other trails of answers, I simply point the reader and myself to a fork in the road and say \say{go forth and explore.}

So, let's try to combat that, at least a little.
What are some things that viewing science and mysticism as two manifestations of the same experience does or could do?
\begin{itemize}
\item Makes mysticism seem more legitimate. Modern society tends to be incredibly in favor of believing what science says\footnote{or at least people claim that science agrees with them. The fact that no political side seems to actually care about science except as a political cudgel is a far larger issue, and not really one at play here.}, and dismissive of the universality of spiritual truths.
\item Makes science seem more personal. There's a push in science these days to start accepting that, at the very least, the questions that scientists choose to answer are not always objectively chosen.
That is, even at the most conservative version of the argument, science has inherent bias because of the questions that we choose to answer and have chosen to answer in the past.
I don't know of anyone who thinks of mysticism as anything but a fundamentally personal experience, even when the knowledge is then spread to the wider world.
\item Clarify the fact that science is fundamentally limited as a method of revelation? There's something here, and it's bothering me like a loose tooth. I might need to fiddle with it a little more to figure out what I'm trying to say there.
\item Develop a metaphysics which works with both scientific and religious revelation? I think that I do really want to write this on some level, but I haven't read anywhere enough about it. Given how many great scientists have also written about the connection of faith and their research, I'm sure that someone has to have written about this, to say nothing of the fact that I don't even have a single metaphysics grasped well enough to explain it.\footnote{other than scientism, which says that there are only observables. I suppose that complete denial of the spiritual is a full metaphysics, but it's not an interesting one to me.}
\item Give the reader something which makes them view the world slightly differently.
As much as I wanted this to be the goal, I think that it might be best served as a meta goal.
That is, the best way to make readers view the world differently is with a convincing enough conclusion.
\end{itemize}

Taking a step back, I do think that's my fundamental issue.
Whenever I write something, at least part of my goal is changing my reader's worldview.
For all that I can and do often just walk away from hearing about an interesting argument and immediately start thinking of what the two sides could claim, I know that this is not a standard response.\footnote{some call it being a devil's advocate, with more or less respect for the choice.}

I think that my goal in the case study is explicitly having the reader treat science and mysticism as reflections of the same fact: that all truths point to Truth, and that we discover truth, rather than creating it.
That's a much larger argument than anything I proposed above, though, and I can see why I shied away, however internally from trying to write it.
For all that I personally know and believe it's better to do something badly than to not attempt something at all, I do find that I\footnote{like everyone, being fair to myself} instinctively believe that it's better to not fail than to try and fail.

I'm currently listening to a lot of BreadTube\footnote{a term often used to describe the left leaning parts of pseudo academic youtube. There's probably a better definition, for all that I don't want to look one up or think on it more.} and reading books that touch on modern theories of knowledge.
Most modern theories of knowledge, as best as I understand the political trends, follow from Foucault.
Now, as someone with no real formal training in philosophy, excuse me if this is wrong, but the crux of Foucault's arguments are that knowledge and power are fundamentally inseparable.
There is not Truth, only acceptance of ideas.
I think that the musing was a direct response to those claims, at least in part.
While a lot of what we do, think, believe, and say is socially constructed, I do still fundamentally believe in universal truth.

Once again, let's refocus on the case study.
I think that it might be worthwhile to look at it the same way that I've been trained to look at musical pieces.\footnote{I think that this might be generally how one is taught to read for bias, but I can't remember that, and that's a claim that's easier to disprove.
Since no single person taught me to read musical pieces, no one can disprove that these were the lessons I took, even if they're not the best.}
First, what is the explicitly stated goal of the work?

My goal for the musing was to talk about science as mysticism.

Second, what's the implicit goal of the work?

I'm still not sure, but I think that it was an affirmation of the universality of truth.
That feels somewhat true, but not fully.

I also don't know if I've ever been good at this method of looking at pieces, for all that I really enjoy seeing others do it.

So, twelve hundred words later, where am I?\footnote{Wow that's a deep philosophical question}
I think that I have issues writing conclusions in large part because I don't know what my implicit goals are for a work, and I'm not even always sure what my explicit goals are for a work.
Most of the time I content myself with a meta goal.\footnote{e.g. my goal for Jeb is primarily to have written it, rather than telling any particular story.
I'm sure that there is a pro somewhere to doing things this way, but there is clearly also a con}

Here comes the kicker, though.
What's the takeaway from this musing?\footnote{gosh that's a strong ending. Shame that it's not going to survive into future drafts, and wow I could phrase it better to make the irony clearer. Maybe it will survive after all}\end{document}