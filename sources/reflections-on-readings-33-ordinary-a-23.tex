\documentclass[12pt]{article}[titlepage]
\newcommand{\say}[1]{``#1''}
\newcommand{\nsay}[1]{`#1'}
\usepackage{endnotes}
\newcommand{\1}{\={a}}
\newcommand{\2}{\={e}}
\newcommand{\3}{\={\i}}
\newcommand{\4}{\=o}
\newcommand{\5}{\=u}
\newcommand{\6}{\={A}}
\newcommand{\B}{\backslash{}}
\renewcommand{\,}{\textsuperscript{,}}
\usepackage{setspace}
\usepackage{tipa}
\usepackage{hyperref}
\begin{document}
\doublespacing
\section{\href{reflections-on-readings-33-ordinary-b.html}{Reflections on Today's Gospel}}
First Published: 2023 November 19

\section{Draft 1}
Today's Gospel parable is the parable of the talents.
It's a parable I've always struggled with, for a few reasons.

The primary reason I've always struggled with it is that it doesn't account for failure.
The two options presented are either making more from what you are given, or not doing that.
There is no fourth servant who attempted to trade his master's property and failed.

I suppose that's probably mostly just because it would muddy the story.
The priest in today's homily pointed out that a talent is around 100 pounds of precious metal.
There's certainly something to be said about the fact that the rich man trusted even his lowest servant with so much money, and it does do a lot to explain why the servant was so sure he would be able to find it after burying it.

There's something else in the parable that sticks out to me right now, though.
Even though at first the parable appears as a straightforward endorsement of capitalism, there are limits.

The first servant doubles his five talents to ten.
That's a 100 percent increase.
However, he stops at that single doubling.

Some might argue that the market forces\footnote{I'm taking a completely non-religious view of the parable, trusting only the market} would not support someone creating so much more than five talents of wealth.
Ten talents might exhaust the field, creating a monopoly.
The second servant, however, starts with two.

When he doubles his starting value, he has nearly as much as the first servant was given.
If the message of the parable is that more is always better, then he should absolutely have doubled his investment again.
And yet, he didn't.

Now, as Catholics, we know a priori\footnote{wow the more that I accept that sometimes I just know an answer and so don't have to justify how we get to it the happier I am.
It's absolutely a dangerous trend that I need to make sure doesn't go to far.
After all, examining why things are true is often valuable, and especially when trying to self reflect.
However, in cases like this, it's nice to be able to skip the whole \say{but won't someone think of the poor conglomerate?}}
that unbridled capitalism is not good.\footnote{arguably the modern political ideal of capitalism, where not everyone has their basic needs met is bad on its own, even without the modifier of unbridled.
I'm using the classicalish take that it's the whole \say{individuals get to own things and set the value for their labor}}

In the context of what I assume we're supposed to read the parable as, treating talents in the modern sense, rather than as a concept of wealth, it is a relieving distinction.
Whatever gifts we are given are not needed to expand exponentially forever.
As someone who finds that they're constantly disappointed in lack of progress, that is a relieving proposition.

Of course, I mentioned at the beginning of this post that it bothers me that there is never an unsuccessful servant.
Within the context of the parable, that does not necessarily hold.
Those who try do not always succeed.

However, this is one of the places that\footnote{assuming my interpretation is correct, which I am generally unwilling to do.
In this specific case, however, I will, if only because I want to also see interpretations from real theologians, and I know that will permanently affect the way that I read the parable.
For all that I mused yesterday about how the way to judge a work is based on what it meant to do, not what it did, I do think that there's value in knowing what I think independently of what great minds think}
the fact that the story is a parable answers the question.\footnote{I am sure that it happens often, but most of the time the metaphor gets harder I find}
If our work is to use our talents for the building of the Kingdom, then there is no way for our work not to prosper.

They say that the road to hell is paved in good intentions.
That may be true, but intentions are not action.
To strive earnestly for a better world, with an open heart and mind, is to make the world a better place.
Even if we do not see the ways in which our work helps those around us, we can be assured that there is no good we do which does not multiply to infinity.

As much as I want to go and read the reflections of great thinkers of the Church on this parable, I know that I need to take the advice I said above and not push for the sake of pushing.
There are more upcoming hours in the day, and I hope that I will have the energy I need to do that reflection, but I will not count on it.

Update: I had my weekly bible study.
This week, not all of the students were available, so we joined with another group.
It is really interesting to see how much quieter my students are when surrounded by strangers.
Even on the first day, when we were just meeting for the first time, they seemed much more willing to speak up than they were today.

I really hope that the same is not true for the other class's students.
I would hate to think that my being there made them uncomfortable sharing what they thought.
Still, what little that each of the students said remained very well thought out.

Since I didn't fill this out yesterday, two days of Daily Reflections:
\begin{itemize}
\item Did I write 1700 words for NaNoWriMo? I did yesterday, I have not yet today. As I said, though, there are a number of hours, which I hope to have time to do.
Update: I have now written the words for the day.
I'm starting to fall behind my plotting, which isn't ideal, but is probably fine.
\item Did I write a chapter of Jeb? I wrote about 500 words yesterday.
I wanted to write more, but I felt like the pace I was trying to set was much faster than the pace that I needed to write for the scene, which really deserved and demanded a lot of focus.
As I said last week, I'm not going to write Jeb on Sundays, and I'm going to hold myself to that.
\item Did I blog? Technically I blogged yesterday, for all that it was a few words.
I also, as you can see, blogged here.
\item Did I stretch? I stretched my neck when stopped at stoplights.
\item Am I doing better at prayer than a rushed and thoughtless rosary? I think I fell asleep in the middle of my rosary on Friday, and I didn't do one last night.
\item Am I doing a good job writing letters to friends? No. I have carried an unwritten letter with me for more than a week at this point, at every stop planning sincerely to write it.
\end{itemize}
\end{document}