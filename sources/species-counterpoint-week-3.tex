\documentclass[12pt]{article}[titlepage]
\newcommand{\say}[1]{``#1''}
\newcommand{\nsay}[1]{`#1'}
\usepackage{endnotes}
\newcommand{\1}{\={a}}
\newcommand{\2}{\={e}}
\newcommand{\3}{\={\i}}
\newcommand{\4}{\=o}
\newcommand{\5}{\=u}
\newcommand{\6}{\={A}}
\newcommand{\B}{\backslash{}}
\renewcommand{\,}{\textsuperscript{,}}
\usepackage{setspace}
\usepackage{tipa}
\usepackage{hyperref}
\begin{document}
\doublespacing
\section{\href{species-counterpoint-week-3.html}{Species Counterpoint Third Week Reflection}}
First Published: 2023 January 7
\section{Draft 1}
Fourth Species counterpoint remains kind of strange, if only because most of the canti firmi\footnote{plural of cantus firums?} I write have mainly stepwise motion, meaning that you end up with a lot of chained suspensions.
I personally love chained suspensions, so that's fine with me, but I understand why others don't.

However, after a nice week of this, I feel ready for Fifth Species.
Fifth Species combines all the rules of the other species together kind of.
\begin{itemize}
\item quarter, half, and whole notes are allowed
\item dissonances resolve stepwise
\item dissonances allowed on the downbeat only when suspended
\item should be singable and pretty
\end{itemize}
After this, I start with three voice species counterpoint, which I'll need to look up the rules for.
\end{document}