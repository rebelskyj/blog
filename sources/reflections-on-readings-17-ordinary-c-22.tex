\documentclass[12pt]{article}[titlepage]
\newcommand{\say}[1]{``#1''}
\newcommand{\nsay}[1]{`#1'}
\usepackage{endnotes}
\newcommand{\1}{\={a}}
\newcommand{\2}{\={e}}
\newcommand{\3}{\={\i}}
\newcommand{\4}{\=o}
\newcommand{\5}{\=u}
\newcommand{\6}{\={A}}
\newcommand{\B}{\backslash{}}
\renewcommand{\,}{\textsuperscript{,}}
\usepackage{setspace}
\usepackage{tipa}
\usepackage{hyperref}
\begin{document}
\doublespacing
\section{\href{reflections-on-readings-17-ordinary-c-22.html}{Reflections on Today's Gospel}}
First Published: 2022 July 24

Genesis 18:27 \say{Abraham spoke up again: \nsay{See how I am presuming to speak to my Lord, though I am only dust and ashes!}}

\section{Draft 1}
As with my last reflection, part of this is stolen from the homily I listened to today.
In today's Gospel, our Lord teaches his disciples how to pray.

Now, I always took that as just \say{say the Our Father,} but the homily I listened to pointed out something very different.
The prayer that Christ teaches his disciples has the elements that our prayer should include.
It begins with the familiar\footnote{or familial} relationship that we have with the Lord, beginning with \say{Our Father}.
Our prayer should then turn towards blessing the Lord for His Goodness.\footnote{Hallowed be thy name...}
We move then into thanking the Lord for what He's done in our lives.\footnote{Give us this day our daily bread}
From there, we move to intercessory prayer.\footnote{Ibid?}
Then, we finish with praying for others.

The part that struck me most is the difference between how we are called to speak to the Almighty and how Abraham does in the first reading.
Throughout the reading, Abraham reiterates his fundamental unworthiness to speak to the Lord, something which\footnote{at least in my reading} is missing from the prayer Christ teaches.
I also see this reflected somewhat in the difference between how Jewish prayers\footnote{that I'm familiar with} and Christian prayers are structured.

More or less every Jewish prayer I've been taught begins with the formula \say{Blessed are you Lord our G-d, King of the Universe}, while most of the Christian prayers I can think of emphasize the fatherhood of the Almighty.\footnote{when prayer is focused on the Father of the Trinity.}
There are a lot of reasons for this difference, but the most fundamental to me is Baptism.
Under the Mosaic Covenant, Israel becomes the Lord's people.
In Baptism, we become His children.
I'm struggling to really articulate the difference, but I think it can be seen somewhat in the line I highlighted.

We are as dust and ash because of our sinful natures.
But we are washed clean in the blood of the Lamb through the Sacraments, and so can address the Lord more familiarly.
The concept of ritual also comes in here.

Under the Mosaic Covenant, communicating with the Lord happens in very ritualistic manners.
There are prescribed sins and remediations for the sin.
Blessings are ritualized and rote, which is something Modern Judaism keeps.\footnote{such as the list \href{https://en.wikipedia.org/wiki/List_of_Jewish_prayers_and_blessings}{here}}
In a quick scroll, there are formalized prayers for many of life's beauties, such as seeing rainbows or smelling flowers.

In contrast, the New Covenant places the emphasis of the law on our hearts.
Gone are individual sacrifices for individual sins.
Gone too is the concept of the priestly caste as wholly separate.
We are all called to be Priest, Prophet, and King in our Baptism.

I'm going to stop here before I ramble too much more. 
\end{document}