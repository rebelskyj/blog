\documentclass[12pt]{article}[titlepage]
\newcommand{\say}[1]{``#1''}
\newcommand{\nsay}[1]{`#1'}
\usepackage{endnotes}
\newcommand{\1}{\={a}}
\newcommand{\2}{\={e}}
\newcommand{\3}{\={\i}}
\newcommand{\4}{\=o}
\newcommand{\5}{\=u}
\newcommand{\6}{\={A}}
\newcommand{\B}{\backslash{}}
\renewcommand{\,}{\textsuperscript{,}}
\usepackage{setspace}
\usepackage{tipa}
\usepackage{hyperref}
\begin{document}
\doublespacing
\section{\href{music-planning.html}{Music Practice Planning}}
First Published: 2023 January 3

\section{Draft 1}
I talked about how I do better with doing things when they're scheduled, and how I do better with schedules when I know what's scheduled.
With that in mind, it seems useful to plan out what I'll practice on my instruments.

Much as I love accordion, it's more of an ordeal to practice, so I find that I don't do it as much, especially when I'm tired.
So, on days when I lack energy, my practice will be playing through a song on guitar and doing a scale.

If I have mental energy again,\footnote{which intellectually I know I will} I would like to try to learn at least a song every two weeks.\footnote{learn means arrange and memorize}
That way I can keep the open mics spiced up.
I'd ideally like to do most of these on accordion.
\end{document}