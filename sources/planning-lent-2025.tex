\documentclass[12pt]{article}[titlepage]
\newcommand{\say}[1]{``#1''}
\newcommand{\nsay}[1]{`#1'}
\usepackage{endnotes}
\newcommand{\B}{\backslash{}}
\renewcommand{\,}{\textsuperscript{,}}
\usepackage{setspace}
\usepackage{tipa}
\usepackage{hyperref}
\begin{document}
\doublespacing
\section{\href{planning-lent-2025.html}{Planning For Lent}}
First Published: 2025 March 1

\\section{Draft 1: 1 March 2025}  
As far as I can tell, I don't have any musings on Lent.\footnote{At least, not by name}  
Lent comes from the Old English word for spring, which I really appreciate, since so few liturgical words seem to come from English's roots.  
Every year I do something different for Lent, based on what I've done previous years and where I feel the most spiritual need for growth.  


The Church's default three things to focus on in Lent are: prayer, fasting, and almsgiving.  
I've often seen this expressed as giving something up, taking something on, and doing charity.  
These two triads are not identical, but they share some similarities.

A quick google search has any number of ideas for how to make Lent more meaningful.  
One of the sites has the very crucial piece of advice: the thing we give up for Lent shouldn't be inherently sinful, because we shouldn't be doing that in the first place.  
It's totally fine and good to give something up for Lent with the expressed purpose of returning to it after Lent.  
That's something I should keep in mind.  
Normally I tend to treat Lent as a test run for things which might be healthier for me long term.  
I'm going to give myself the freedom to also consider giving up or taking on things that I am pretty sure aren't going to serve me as a new practice.

In the past, I have tended to give up meat and alcohol, both because those are traditional Lenten things\footnote{the meat for sure, alcohol is less clear, given the fact that, you know, alcohol used to be an essential part of diets} and because they're parts of my diet that are nice but not necessarily essential.  
This year, however, I'm really trying to find a way to keep my body nourished, and so I think that anything that puts a block between me and food is probably not a healthy idea\footnote{fasting and avoiding meat on required days being the obvious exception}.

In the past I've also added on large prayer plans, but that doesn't feel as good right now, probably because of how little prayer I have right now.  
Still, I should add more prayer, both because it's a thing explicitly recommended in Lent, and also because it's a goal I've had external to that.  
I think that a decade of a rosary is absolutely a low bar, and something that is at least somewhat meditative, if I do it right.  
Chaplet of\footnote{to?} St. Michael is another good one, and I do find the intercessory prayers more powerful in that one, so I might say that as the thing.

Ok so then we have almsgiving.  
There are just so many places in the world right now that need help, and there are so many ways that I can give time, talent, and treasure\footnote{which is a common thing I see, I don't know if that's a Catholic only thing though?}.  
What causes are the nearest and dearest to me?

Honestly, I think that because so much of the focus of catholics around me is entirely on abortion, I find myself more and more looking at the ways that life is hard for mothers.  
There's the cheap answer of trying to campaign for maternity leave or better protections for mothers, but that's basically the same as doing nothing.  
There's a charity in town that gives away diapers and other supplies families might need, and that's probably a safe and good plan.  
It's controversial for having formerly\footnote{still?} been a crisis pregnancy center, and I understand some of the objections that many have to them.  
However, they are the major provider of aid for new mothers in the area, as far as I can tell.

I have a friend who volunteers at a house for young\footnote{I think} single mothers.  
It is also an organization that could always use more resources.  
For reasons which seem fair, they're less keen on having men volunteer, but I don't have to put all my eggs in a single basket.  
It is important to both give low level aid to many and high level aid to a few.  
Which is better is ultimately a meaningless question, because we are constantly reminded that anyone we see struggling is Christ.

There's also the more classic version of almsgiving, and the area certainly doesn't lack homeless people.  
One thing that I've seen suggested in the past is giving away as much money to others as I spend on myself.  
Since part of Lent is about giving up pleasures, the fact that it might lead me to spend less on myself is a benefit, as is the reminder of just how many blessings my life is filled with.\footnote{which is a sentence that still feels weird to say when I think about my mom}

It's hard for me to feel like giving someone money is not an effective form of charity, even if I understand economies of scale are sometimes helpful, especially for food banks and the like.  
However, charity is a virtue like all others, and it needs to grow from somewhere.  
I think that it makes most sense for me to send money to one of the two organizations for mothers and infants, in part because that was something dear to my own mother's heart.  
I'll also think more about volunteering, though I do truthfully feel like I don't have the time for it right now\footnote{which I know, is sort of the point. The woman who gave her single coin was worth far more than those who gave from their excess. However, I know that I need to give myself grace as well} between everything else that I do.

So we've taken on prayer: the chaplet of St. Michael and alms giving: donating at least as much as I spend on myself per week to one of the above organizations as well as giving to the homeless on the street.  
What am I giving up?

First: games. I spend a lot of time passively wasting on playing logic puzzles, and I don't think that's particularly healthy for me.  
I'll still play with friends or any non-digital game as it comes up, but solo games are out for the season.  
Should I give something else up?

I want to stop scrolling social media, so will try once again to stop scrolling.  
My friends value memes, it's true, but they value me more than the memes I give them.\footnote{hopefully}

I think that limited social media and no more games are probably two good things to give up.  
As I said, giving up meat doesn't seem like my best bet right now.  
All said, this seems reasonable to me!

  


N.B. I've decided to have the whole list of goals that I have for the month at the bottom of each posting, and I'll delete entries as is relevant. That way I can track everything each day!

\begin{itemize}   
\item Professional:   
\begin{itemize}   
\item Thesis Work:  
\begin{itemize}   
\item Detailed outline (chapters and section titles at the very minimum) by 18 March. Finished that last night! Can remove.  
\item Meet with Committee to decide a timeline. Before that, come up with my own idea for one.  Have a timeline of my own, the committee meeting is scheduled, so can remove.  
\item Work on the research within the thesis!   
\end{itemize}   
\item Be a better mentor: figure out how to take time to help underclassmen as they need help while still getting my own work done.   
\item Leave work at work. I've been only really thinking about my research this past month, and I can feel it wearing me down rapidly.  
I've started setting some boundaries, so we'll see if any of them happen  
\item Work towards future career:   
\begin{itemize}   
\item Read the recommended readings about science communication   
\item Do the reflections that were recommended to me (mostly focused around why I care about science communication)   
\item Work on the materials for the science outreach event in April: the handout while they work and a page for the families to take home   
\item Figure out the difference between my public-facing and field-facing presentation affects. As I focus on becoming a better presenter, I need to become aware of the difference and how to switch them   
\item I was reminded today that part of getting a job in the future requires, shockingly, searching and applying for them, so should start working on that as well  
\end{itemize}   
\end{itemize}   
\item Health:  
\begin{itemize}   
\item Spiritual:   
\begin{itemize}   
\item Figure out what I want to do for my Lent. In general, I think I want to give something up, take something on, and find a way to do charity.   
 Wow look, that's my musing today! Can remove!  
\item Do the Lenten goals.   
\item Be intentional about prayer. That means both making time for prayer and actually doing it.   
\end{itemize}   
\item Mental:   
\begin{itemize}   
\item Clean my Life:   
\begin{itemize}   
\item Remove dirt and clutter from physical spaces (standard definition of clean)   
\item Remove extraneous apps from phone.  
 Done, though I may increase the number of apps I delete. Right now I have the hard issue that some social media apps are my primary way of keeping in touch with some friends.  
Will have to see what happens  
\item Spend time each day thinking about the goals for the day, and getting them out of my head and onto the page  
\item Start reading and returning the library books I have.  
\item Start to separate the non-work time from work time. I think that, at the very least, I need to say that I will not do work in my apartment. Given that I generally avoid spending much time in there, these two goals should work together.   
As stated above, made my boundaries as a list, we'll see if I follow them. Will keep this entry with that in mind.  
\item Don't waste time, and in particular, be mindful about making sure to take breaks and rest.   
\item Clean sight lines. Is my space set up in a way that orients me towards my goals for the space? If not, how can I make it so?   
\end{itemize}   
\item Interpersonal Relationships:  
\begin{itemize}   
\item Figure out what belongs in a normal letter to a friend.  
\item Get back into writing letters.   
\item Upon feeling a sense of dread at receiving a message from someone, remember that my lived experience says that most interactions are positive. More to the point, if my friends didn't like me, they would tell me or at the very least would not continue to keep me in their life. If alone, speak something to that effect.   
\item Compile a list of people who are important to me. It does not need to be comprehensive, but ideally would approach that.   
Have done so! At least for people that I text or otherwise text message and don't know the timing that I should have with them.  
\item Figure out the method and frequency of communication I would like to have with them, be that texting, calling, visiting in person, etc.   
Asked them ! Can remove both of these entries.  
\item Work to begin doing so.   
\item Potentially start giving small gifts, though many people also dislike clutter, so think carefully about that one.   
I'm deleting this, because clutter ain't great.  
\end{itemize}  
\end{itemize}   
\item Physical:   
\begin{itemize}   
\item Go to group fitness classes more regularly and more often.   
\item Feed myself simply and healthily.   
\end{itemize}   
\end{itemize}   
\item Other:   
\begin{itemize}   
\item Music:   
\begin{itemize}   
\item Figure out something to work towards on guitar   
\item Work towards it   
\item Spend time making efforts to improve as a singer, not simply passively singing.   
\item Spend time making efforts to improve as a musician, not simply passively growing.   
\end{itemize}   
\item Writing:  
\begin{itemize}   
\item Find the mental block towards writing my web novel   
\item Write poetry more often, ideally nightly.   
\item Not only write blogs, but also post them. Ideas include:\footnote{as a living list!}  
\begin{itemize}   
\item 26 for 26   
\item Lenten goals. woo! Did it!  
\item Listening to an album and writing about my experience with it. Unsure if this is best done with one I have prior knowledge of or a new one, but regardless, sit and listen without other stimuli.   
\item The arts I've been doing lately   
\item Why I care about science and communicating it   
\item the block between me and my web novel   
\item My general disposition to authority. I'm not entirely sure what that meant, but it was a note I wrote to myself while half asleep, so I'll leave it here.  
\item How to feed myself.  
\end{itemize}   
\end{itemize}   
\end{itemize}   
\end{itemize}

  



\end{document}