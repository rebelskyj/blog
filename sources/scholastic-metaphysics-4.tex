\documentclass[12pt]{article}[titlepage]
\newcommand{\say}[1]{``#1''}
\newcommand{\nsay}[1]{`#1'}
\usepackage{endnotes}
\newcommand{\1}{\={a}}
\newcommand{\2}{\={e}}
\newcommand{\3}{\={\i}}
\newcommand{\4}{\=o}
\newcommand{\5}{\=u}
\newcommand{\6}{\={A}}
\newcommand{\B}{\backslash{}}
\renewcommand{\,}{\textsuperscript{,}}
\usepackage{setspace}
\usepackage{tipa}
\usepackage{hyperref}
\begin{document}
\doublespacing
\section{\href{scholastic-metaphysics-4.html}{Scholastic Metaphysics Chapter 3}}
First Published: 2022 March 17

\section{Draft 1}
Has it been two weeks since my last post in the series?
Yes.

Anyways.

First, my response to the questions pre-reading.
\begin{enumerate}
\item What is the point of stopping to discuss the special characteristics of the concepts we use in metaphysics?

It will lay a good foundation for us to actually learn things
\item \begin{itemize}
\item What does it mean that a concept is a \say{transcendental} one?

It exceeds cultural norms, and becomes true across all culture.
From a friend, truth and good are transcendental, I'm unsure if that's true.
\item Why is \emph{being} such?

Because the Almighty wishes it to be.
Wait.

I think the question means like, why is being a transcendental concept?
It's transcendental because being does not rely on a culture.
\item Why is it said to be at once the \say{poorest} and the \say{richest} of all concepts?

It is fundamental to all conceptions, but makes no conceptual claims in itself.
\end{itemize}
\item Most ordinary concepts we use are formed by abstracting a common essence and omitting particular details.
\begin{itemize}
\item Why is it that being cannot be thus formed?

Being lacks any particular details.
\item How then is it formed?

As above, through the will of the Divine, though I'm unsure if permissive or explicit.
\end{itemize}
\item \begin{itemize}
\item How do the three types of concepts differ from each other: univocal, equivocal, analogous?

Univocal is the same, equivocal is alike, analogous is made alike.
\item Why is it necessary that the main concepts used in metaphysics be analogous?

If they aren't able to made similar, there's no way to compare them.
However, if they are the same, there's no use in making them like each other.
\end{itemize}
\item \emph{Analogy of Attribution}:
\begin{itemize}
\item What does this mean?

No clue.
\item Examples?

Nope.
\item Does it express any intrinsic similarity between the prime analogate and the secondary anlogates?

Probably.
\item Why do we use it?

Has uses.
\end{itemize}
\item \emph{Analogy of Proportionality}:
\begin{itemize}
\item Meaning?

No idea.
\item Examples?

Nope.
\end{itemize}
\item What is the basis in reality for all uses of analogy of proportionality?

great question.
\item \begin{itemize}
\item What is happening in an analogy of \emph{Metaphorical or Improper Proportionality?}

ibid.
\item Examples?

ibid.
\item Why is it not literally true?

ibid.
\item What is the purpose of using it then?

ibid.
\end{itemize}
\item \emph{Analogy of Proper Proportionality}:
\begin{itemize}
\item Meaning?

ibid.
\item Examples?

ibid.
\item Explain how such an analogous term can be at once literally true of all its analogates, yet shifting somewhat in meaning for each, i.e., how can it be so flexible, as an univocal cannot?

i cannot.
\end{itemize}
\item Does an analogous concept contain two parts, one of which expresses just the similarity, the other just the diversity?

That seems reasonable.
\item \emph{Analogy Applied to Being}: How does this work, since being seems to be such a simple concept and not expressing any common action, as other proper proportionality concepts do?

Being is not a simple concept?
I have no idea.
\item \emph{Analogy Applied to God}: How does it avoid falling into anthropomorphism (making God too much like humans) on the one hand, or mere metaphor on the other?

Very carefully.
\item Why is William of Ockham forced to hold that the concept of \emph{being} applies univocally to both God and creatures, whereas St. Thomas is not?

St. Thomas makes a distinction between G-d and creatures.
\item How does St. Thomas ground the real similarity between God and creatures, making possible a proper analogy between them?

If I had to guess, the Incarnation, but that's a wild guess with no logic.
\end{enumerate}

Time to see how accurate they are as I read the text, which as always, I will rely to as I read:

\begin{enumerate}
\item What is the point of stopping to discuss the special characteristics of the concepts we use in metaphysics?

Otherwise we might misuse them is the gist as I understood it.
\item \begin{itemize}
\item What does it mean that a concept is a \say{transcendental} one?

It is all-inclusive to the things it describes.\footnote{I'm sure of the all-inclusive, the other part less so}
\item Why is \emph{being} such?

Because if something isn't then it doesn't exist.
\item Why is it said to be at once the \say{poorest} and the \say{richest} of all concepts?

Looks like I was right.
It's the poorest because it doesn't actually specify anything, but the richest because it describes everything.
\end{itemize}
\item Most ordinary concepts we use are formed by abstracting a common essence and omitting particular details.
\begin{itemize}
\item Why is it that being cannot be thus formed?

To omit a detail is to remove it from being, which you cannot, since being encompasses everything.
That is, while you can abstract sex and age from human beings, because there is a shared humanity without them, there is no way to abstract things away to get to being.
\item How then is it formed?

\say{Judgement of Separation,} so basically just saying the word is.
It feels a whole lot like plain abstraction to me, though.
\end{itemize}
\item \begin{itemize}
\item How do the three types of concepts differ from each other: univocal, equivocal, analogous?

Univocal concepts can be rigidly applied to multiple beings with the same meaning.
Ex: all hydrogen atoms can be described as such.

Equivocal concepts are homonyms, which is annoying.
Ex: he put his money in the \textbf{bank} which was on the river\textbf{bank}.

Analogous concepts are between the two, similar in meaning but not identical.
Ex: Strength of arm, muscle, and will.
\item Why is it necessary that the main concepts used in metaphysics be analogous?

Because we use the concept of being, and because we're generally talking about similar, not identical things.
\end{itemize}
\item \emph{Analogy of Attribution}:
\begin{itemize}
\item What does this mean?

Things get descriptions not because they are but because they are related to things which are.
\item Examples?

Healthy food.
Food is dead, so not healthy, but makes healthy people healthy, so is seen as healthy.
American cars lack citizenship but are made (in theory) in and by Americans.
\item Does it express any intrinsic similarity between the prime analogate and the secondary anlogates?

Yes, the secondary gains the term because of the prime.
\item Why do we use it?

We shouldn't.
\end{itemize}
\item \emph{Analogy of Proportionality}:
\begin{itemize}
\item Meaning?

Things are alike in how they do the same thing.
\item Examples?

Man and mice both know.
\end{itemize}
\item What is the basis in reality for all uses of analogy of proportionality?

Action is the thing which bonds beings, and so since proportionality is based on shared action...
\item \begin{itemize}
\item What is happening in an analogy of \emph{Metaphorical or Improper Proportionality?}

Use of metaphor.
\item Examples?

He's a snake.
\item Why is it not literally true?

Men aren't snakes.
\item What is the purpose of using it then?

Sounds better than \say{Like a snake, he is sneaky.}
Poetry is good.
\end{itemize}
\item \emph{Analogy of Proper Proportionality}:
\begin{itemize}
\item Meaning?

Seems like the general example from above with men and mice both knowing.
\item Examples?

Men and mice know differently, but both still know.
\item Explain how such an analogous term can be at once literally true of all its analogates, yet shifting somewhat in meaning for each, i.e., how can it be so flexible, as an univocal cannot?

Words have shifted meaning based on context.
\end{itemize}
\item Does an analogous concept contain two parts, one of which expresses just the similarity, the other just the diversity?

No, that would be a univocal and equivocal concept paired together.
Instead, it encompasses both at once, expressing the shared and not.
\item \emph{Analogy Applied to Being}: How does this work, since being seems to be such a simple concept and not expressing any common action, as other proper proportionality concepts do?

The common action is existence.
\item \emph{Analogy Applied to God}: How does it avoid falling into anthropomorphism (making God too much like humans) on the one hand, or mere metaphor on the other?

It says that since the common action is existence, rather than existence being separate from essence, since the Almighty exists, He has substance, and shares that with us.
However, sharing existence is all that is claimed, avoiding anthropomorphism.
\item Why is William of Ockham forced to hold that the concept of \emph{being} applies univocally to both God and creatures, whereas St. Thomas is not?

He focuses on being as essence, rather including existence.
\item How does St. Thomas ground the real similarity between God and creatures, making possible a proper analogy between them?

He says that they share existence.
\end{enumerate}

Wow I should reread this chapter, it went a lot over my head.

\end{document}