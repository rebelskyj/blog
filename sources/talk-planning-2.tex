\documentclass[12pt]{article}[titlepage]
\newcommand{\say}[1]{``#1''}
\newcommand{\nsay}[1]{`#1'}
\usepackage{endnotes}
\newcommand{\1}{\={a}}
\newcommand{\2}{\={e}}
\newcommand{\3}{\={\i}}
\newcommand{\4}{\=o}
\newcommand{\5}{\=u}
\newcommand{\6}{\={A}}
\newcommand{\B}{\backslash{}}
\renewcommand{\,}{\textsuperscript{,}}
\usepackage{setspace}
\usepackage{tipa}
\usepackage{hyperref}
\begin{document}
\doublespacing
\section{\href{talk-planning-2.html}{Talk Planning,Tuning Theory}}
First Published: 2022 August 9
\section{Draft 1.1 2022 August 22 (a typo was pointed out)}
\href{talk-planning-1.html}{Yesterday} I claimed that I have the ability to freestyle a cogent narrative about tuning theory.
To test\footnote{prove?} that claim,\footnote{and because I have no desire to actually do research on the topics I don't know today} I figured that I should do that here.
Halfway through writing this, I already realize that I can speak on the theory, though not in a cogent way.

I hope that I don't need to motivate what music is, at least informally.
Pitch is fairly fundamental to most of Western Music\footnote{and other music, but I'm only doing Western thoughts here}.
The concept of pitch as inherent to a specific body is an ancient thought.

Pythagoras is reported to have begun his theories on tuning when he observed blacksmiths.
Regardless of how hard the smith struck an anvil, each anvil rang at its own pitch.\footnote{there are connections I see to the photoelectric effect, but that's too much of a tangent (for now)}
He then allegedly measured them, and found what it was about each anvil that made it produce its specific pitch.

Moving now from the story to reality, tuning theory is based primarily on the harmonic series.
The harmonic series is produced\footnote{in theory} by any resonating body.
It's really easy to visualize harmonic series now that Fourier transforms are easy to perform.
Put simply\footnote{I think?}, it's every\footnote{positive?} integer multiplication of a base frequency.
That is, if your fundamental pitch is at 100 Hz, the first harmonic occurs at 200 Hz, the second at 300, and so on.

The ratio between any two\footnote{sequential usually} of these pitches in the series corresponds to a harmonic interval.
A one to two ratio is an octave.\footnote{I might use the piano in the room for these demonstrations}
A two to three ratio is a fifth.
A three to four ratio is like a 2:3:4 ratio, and so therefore a fourth, since stacking a fifth over a fourth is an octave.
A four to five ratio is a major third.

The human ear, of course, can't perfectly hear intervals.
If you have something close to a 2:3 ratio, instead of hearing 1999:3001, for instance, your ear is likely to hear a simple 2:3.
The smaller the numbers in the ratio, the more likely it is that you will hear when it's out of tune.

Pythagorean tuning is the earliest Western tuning we have.
It's based on the first three pitches of the harmonic series.
To construct it, start with a base frequency, then stack fifths until you get back to your starting note, dropping octaves as needed.

Of course, the mathematically inclined among you may notice that this doesn't mathematically work.
Other than the trivial case of n = m = 0, there is no m and n that satisfies:
\begin{equation}
\frac{1}{2^n} = \frac{2^m}{3^n}
\end{equation}
If you plot how close these numbers are, you will see that the closeness of fifths is optimized at 12 and 24.\footnote{and so on and so forth}

To balance having few pitches with good octave and fifths, they chose 12.
The final fifth is just incredibly out of tune, and called a wolf fifth, because it \say{sounds like a barking wolf}.
Thirds are also somewhat out of tune.
That's because a harmonically tuned major third is a 4:5 ratio, as mentioned above.
In Pythagorean tuning, each step is well tuned, being a 8:9 ratio.
However, two steps then becomes a 64:81 ratio, which is slightly different than 4:5.
That's not horribly out of tune.
A minor third in harmonic tuning is a 5:6 ratio.
A minor third in Pythagorean tuning is a a 32:27 ratio, which is different than 5:6.

As a result, as harmonies in the medieval time period started becoming based more around the third than the fourth and fifth, new tuning systems were derived.
Among these is quarter comma meantone, where you take the difference between 12 fifths and an octave, the comma, and distribute it to make four intervals slightly worse.
Other tuning systems come too, such as the well temperment of Bach fame.
One commonality among these systems is that each key sounds different, because the intervals are slightly different.

Now we come to the modern tuning systems of equal temper and stretched octave equal temper.
Why have intervals based on simple ratios, when you can instead say only the octave matters?
In equal temper, each half step is the twelfth root of two larger than the note below. 
To discuss the difference between different temperments' interval size, we use the term \say{cents}.
One hundred cents are an equal tempered half step.
Stretch octaves is what pianos are tuned to.
Scientists have found that octaves slightly larger than 1200 cents sound more like an octave than a real octave does, so piano intervals are even different.
Before this turns into a rant on pianos, I'll move on.

In retrospect, maybe I should start with the explanation of equal temper so that I can say what the difference between intervals in tuning systems is.
Anyways, this is at least some text that I can\footnote{should} clean up and throw onto a slide deck.
I think I cover most of what I want to cover here, though maybe I should tie it into the astronomy?
That ties with Kepler well.
We'll see I guess.

\section{Draft 1}
\href{talk-planning-1.html}{Yesterday} I claimed that I have the ability to freestyle a cogent narrative about tuning theory.
To test\footnote{prove?} that claim,\footnote{and because I have no desire to actually do research on the topics I don't know today} I figured that I should do that here.
Halfway through, I already realize that 

I hope that I don't need to motivate what music is, at least informally.
Pitch is fairly fundamental to most of Western Music\footnote{and other music, but I'm only doing Western thoughts here}.
The concept of pitch as inherent to a specific body is an ancient thought.

Pythagoras is reported to have begun his theories on tuning when he observed blacksmiths.
Regardless of how hard the smith struck an anvil, each anvil rang at its own pitch.\footnote{there are connections I see to the photoelectric effect, but that's too much of a tangent (for now)}
He then allegedly measured them, and found what it was about each anvil that made it produce its specific pitch.

Moving now from the story to reality, tuning theory is based primarily on the harmonic series.
The harmonic series is produced\footnote{in theory} by any resonating body.
It's really easy to visualize harmonic series now that Fourier transforms are easy to perform.
Put simply\footnote{I think?}, it's every\footnote{positive?} integer multiplication of a base frequency.
That is, if your fundamental pitch is at 100 Hz, the first harmonic occurs at 200 Hz, the second at 300, and so on.

The ratio between any two\footnote{sequential usually} of these pitches in the series corresponds to a harmonic interval.
A one to two ratio is an octave.\footnote{I might use the piano in the room for these demonstrations}
A two to three ratio is a fifth.
A three to four ratio is like a 2:3:4 ratio, and so therefore a fourth, since stacking a fifth over a fourth is an octave.
A four to five ratio is a major third.

The human ear, of course, can't perfectly hear intervals.
If you have something close to a 2:3 ratio, instead of hearing 1999:3001, for instance, your ear is likely to hear a simple 2:3.
The smaller the numbers in the ratio, the more likely it is that you will hear when it's out of tune.

Pythagorean tuning is the earliest Western tuning we have.
It's based on the first three pitches of the harmonic series.
To construct it, start with a base frequency, then stack fifths until you get back to your starting note, dropping octaves as needed.

Of course, the mathematically inclined among you may notice that this doesn't mathematically work.
Other than the trivial case of n = m = 0, there is no m and n that satisfies:
\begin{equation}
\frac{1}{2^n} = \frac{2^m}{3^n}
\end{equation}
If you plot how close these numbers are, you will see that the closeness of fifths is optimized at 12 and 24.\footnote{and so on and so forth}

To balance having few pitches with good octave and fifths, they chose 12.
The final fifth is just incredibly out of tune, and called a wolf fifth, because it \say{sounds like a barking wolf}.
Thirds are also somewhat out of tune.
That's because a harmonically tuned major third is a 4:5 ratio, as mentioned above.
In Pythagorean tuning, each step is well tuned, being a 8:9 ratio.
However, two steps then becomes a 64:81 ratio, which is slightly different than 4:5.
That's not horribly out of tune.
A minor third in harmonic tuning is a 5:6 ratio.
A minor third in Pythagorean tuning is a a 32:27 ratio, which is different than 5:6.

As a result, as harmonies in the medieval time period started becoming based more around the third than the fourth and fifth, new tuning systems were derived.
Among these is quarter comma meantone, where you take the difference between 12 fifths and an octave, the comma, and distribute it to make four intervals slightly worse.
Other tuning systems come too, such as the well temperment of Bach fame.
One commonality among these systems is that each key sounds different, because the intervals are slightly different.

Now we come to the modern tuning systems of equal temper and stretched octave equal temper.
Why have intervals based on simple ratios, when you can instead say only the octave matters?
In equal temper, each half step is the twelfth root of two larger than the note below. 
To discuss the difference between different temperments' interval size, we use the term \say{cents}.
One hundred cents are an equal tempered half step.
Stretch octaves is what pianos are tuned to.
Scientists have found that octaves slightly larger than 1200 cents sound more like an octave than a real octave does, so piano intervals are even different.
Before this turns into a rant on pianos, I'll move on.

In retrospect, maybe I should start with the explanation of equal temper so that I can say what the difference between intervals in tuning systems is.
Anyways, this is at least some text that I can\footnote{should} clean up and throw onto a slide deck.
I think I cover most of what I want to cover here, though maybe I should tie it into the astronomy?
That ties with Kepler well.
We'll see I guess.
\end{document}