\documentclass[12pt]{article}[titlepage]
\newcommand{\say}[1]{``#1''}
\newcommand{\nsay}[1]{`#1'}
\usepackage{endnotes}
\newcommand{\1}{\={a}}
\newcommand{\2}{\={e}}
\newcommand{\3}{\={\i}}
\newcommand{\4}{\=o}
\newcommand{\5}{\=u}
\newcommand{\6}{\={A}}
\newcommand{\B}{\backslash{}}
\renewcommand{\,}{\textsuperscript{,}}
\usepackage{setspace}
\usepackage{tipa}
\usepackage{hyperref}
\begin{document}
\doublespacing
\section{\href{poetry-forms-2.html}{Writing through Poetry Forms}}
First Published: 2022 April 14


\section{Draft 1}
As I mentioned \href{poetry-forms-1.html}{last time}, I'm writing through a book of poetic forms.
I wrote my first telestich, and it's a much different experience than writing a typical acrostic.
I can think of lots of words that start with almost any letter of the alphabet, but very few words ending in certain letters.
The first poem I worked on was based on \say{church}, which meant I needed two c-ended words and a u-ended word.
It took a lot of searching for both, so I think that to make a good telestich the word choice is far more important than in an acrostic.
Future poems will hopefully go better as I figure out good words.
Below is today's poem, rough as it is.

Begin and End\\
In watching water flow and ebB\\
Or daylight turn to evE\\
Unlike that last and bitter dreG\\
Which ends the alibI\\
I start and hope to wiN
\end{document}