\documentclass[12pt]{article}[titlepage]
\newcommand{\say}[1]{``#1''}
\newcommand{\nsay}[1]{`#1'}
\usepackage{endnotes}
\newcommand{\1}{\={a}}
\newcommand{\2}{\={e}}
\newcommand{\3}{\={\i}}
\newcommand{\4}{\=o}
\newcommand{\5}{\=u}
\newcommand{\6}{\={A}}
\newcommand{\B}{\backslash{}}
\renewcommand{\,}{\textsuperscript{,}}
\usepackage{setspace}
\usepackage{tipa}
\usepackage{hyperref}
\begin{document}
\doublespacing
\section{\href{lemon-wine-2.html}{Lemon Wine Starting and Ending}}
First Published: 2023 August 3


\section{Draft 1}
Huh, apparently my post in June about \href{lemon-wine-2.hmtl}{lemon wine} never posted.
Anyways, it's finally finished.
It was almost certainly ready well before I ended up finishing it off, but I was busy and so gave it some extra time.

In the initial recipe notes, I said that I would dry hop it, which I no longer plan to do.
I also said I would use three quarts of lemon juice, which ended up not being true.
I saw a video discussing oleo citrum\footnote{ok the video claimed I could get 8x (I think) as much juice from a citrus and I fall for clickbait}, and thought about how that could be nice in lemon wine.

So, I ended up peeling six lemons, getting about 4.5 ounces of lemon peel, which I mixed with around that much citric acid, let sit, then blended with the juice from the lemons.\footnote{read: I cut the pith off and blended the whole remaining lemon}
That mixture went into the reracked lemon wine along with stabilizers.
I then waited for a bit as I cleaned kegs and found access to CO2 again.

It is now safely inside of a keg, and has been receiving fantastic reviews.
I'm really happy with it, and used the dregs from reracking to pitch the next set of lemon wine, which is my laziest attempt yet.

In short, the new version is just the entire 10 pound bag of sugar that I normally use for both fermenting and backsweetening at the beginning with no inverting.
I then filled water, trusted that there would be sufficient nutrient, but repitched yeast just to be safe.
It's fermenting away, so I trust that it will go well as well.
If so, I plan to dry hop it.

\begin{itemize}
\item Made a few figures for the Pleiades\footnote{which I am terminally unable to spell correctly.} and added a few lines of text. It'll need to be done much faster moving forwards.
\item I spent ten minutes cleaning this morning and made nice amounts of progress.
\item Blogging going well.
\item Finished today's chapter of the book\footnote{just now} 2/4+
\item Wrote a sonnet yesterday, plan to write one after this.
\item Wrote two letters! And posted three\footnote{had a letter that was being saved for when a friend was back home}
\item No work on song, whoops. Should really get on this.
\item Stretched for a few minutes this morning, following part of my old diving routine.
\item So yeah I'm not prioritizing it as well today.
\item Did a nice chapel rosary. Then went and did a corporal work of mercy\footnote{visiting a friend in the hospital} and spent time praying at adoration and after.
The rosary was far less intentional than yesterday, but growth is almost never straightforward and continuous.
\end{itemize}

395/63
\end{document}