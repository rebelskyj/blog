\documentclass[12pt]{article}[titlepage]
\newcommand{\say}[1]{``#1''}
\newcommand{\nsay}[1]{`#1'}
\usepackage{endnotes}
\newcommand{\1}{\={a}}
\newcommand{\2}{\={e}}
\newcommand{\3}{\={\i}}
\newcommand{\4}{\=o}
\newcommand{\5}{\=u}
\newcommand{\6}{\={A}}
\newcommand{\B}{\backslash{}}
\renewcommand{\,}{\textsuperscript{,}}
\usepackage{setspace}
\usepackage{tipa}
\usepackage{hyperref}
\begin{document}
\doublespacing
\section{\href{lemon-wine-2.html}{Lemon Wine Redux}}
First Published: 2022 June 27


\section{Draft 1}
It's been a little less than a year since \href{lemon-wine.html}{the last time I mused about lemon wine,} and it seems like an appropriate time to do so again.
Last year I apparently made a single batch of lemon wine before writing the musing.
I ended up making one or two others, which was really fun.

This year, the recipe has changed slightly.
As before, I cooked approximately eight pounds of sugar in water.
This time, though, instead of lemon juice, I used pure citric acid.\footnote{because I own that now.}
I also took a note from my second batch last year and didn't add the lemon juice to my must.

There are a few reasons I think that's a good idea, most of which revolve around the fact that my yeast can't metabolize citric acid and I don't really want it to metabolize the ascorbic acid.\footnote{since that's vitamin c and anything to make my life healthier is great}
The other flavoring oils shouldn't be digestible, and again, I would like them to remain.
Also, adding lemon juice drastically drops the pH of the solution,\footnote{shockingly} and I don't want to stress my yeast out.

It's really strange to me how many flavor notes fermented sugar water has even without other additives.
I don't think I really believed that different yeasts actually produced ester profiles before this.
The fermentation seems to be stalling out a little bit, but I'm hopeful that it won't be true in the morning.
If it is, I'll do my usual lifehack of sprinkling a small amount of yeast nutrient to get rid of all the dissolved carbon dioxide\footnote{similar effect to sprinkling sugar in coke or any other carbonated drink. Small particles mean lots of nucleation sites, means bubbles go away}

It looks like I've historically used three quarts of lemon juice, which still seems like an appropriate amount.
Anyways, I'm excited to see how this batch comes out.
I think I'm going to do something that I've thought about since I started making lemon wine\footnote{my 21st birthday officer}, and dry hop it.
I'll go into more detail about what dry hopping is when that becomes relevant.


\begin{itemize}
\item I got an air filter and set it up.\footnote{because wow the air quality is terrible}
\item Blogging streak day three
\item Air quality got even worse, so I took a day off of exercise
\item I slept with my window open last night, and didn't realize the air quality was going to suffer so much. As a result, I slept poorly and in.
\item I listened to a lot of CCCiaY and prayed the chaplet of St. Michael the Archangel over breakfast. As yesterday, I'll also do a rosary in bed.
\item I wrote two chapters today! That technically means that I'm three ahead. As soon as tomorrow's chapter posts, though, it'll be back to two ahead. Still exciting!
\item Wrote a sonnet last night! Will not write one tonight because wow time flew.
\item Didn't write a letter. Should consider doing that tomorrow morning.
\end{itemize}

449/53
\end{document}