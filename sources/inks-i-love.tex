\documentclass[12pt]{article}  
\newcommand{\say}[1]{``#1''}  
\newcommand{\nsay}[1]{`#1'}  
\usepackage{endnotes}  
\newcommand{\B}{\backslash{}}  
\renewcommand{\,}{\textsuperscript{,}}  
\usepackage{setspace}   
\usepackage{tipa}  
\usepackage{hyperref}  
\begin{document}  
\doublespacing  
\section{\href{inks-i-love.html}{The Inks I've Loved}}  
First Published: 2025 November 28 (I keep forgetting to post)

\section{Draft 1: 24 November 2025}

I have been playing around with fountain pens for a little while now,and something I'm beginning to realize is that I want to have a record somewhere of what pens have worked well for me, what inks I love, and the like.  
With that in mind, today's post is just going to be the start of the list of inks I love.

Well, that's not entirely true.  
I also want to be doing some writing, so let's talk about ink going forward.

Every month, I'm taking in five five mL samples of ink.\footnote{before anyone suggests the obvious solution of stop doing that... no}  
That means that ideally I'd go through twenty five milliliters of ink monthly, if not a little more to start working through the backlog.  
I know that's not true, but I'm not entirely sure how untrue.

I also have the issue that by inking each pen with a different ink, I don't get the chance to see the way that it behaves with different nibs and feeds.  
With that in mind, I think that going forward I'd like to start using up each bottle of ink as it comes in, filling all my pens with the same color.  
Downsides of this are of course that I have only one color of ink at a time.  
However, I'm really not finding that there are many times in my day to day where I actually want to have multiple colors.

This also comes with the secondary benefit that I am going to be working through the number of inks that I have.  
Even if the rate of usage doesn't change, I'm still getting each bottle of ink done with one at a time, and so there's no chance that the ink can dry.

While I'm here, may as well give a quick review of the two new inks I have in pens.

Shimmer Lilac is a relatively translucent lilac color  
That is, with the fude pen, I can use it to highlight other text.  
That's not necessarily a bad thing, especially because it's still perfectly clear and readable.

It also nominally has a gold shimmer to it (little flecks).  
For whatever reason, I'm not getting a lot of that in the pen so far.  
Instead, the occasional word is just perfectly golden.  
If that was an effect I could control, I would absolutely adore it.  
As it is, though, I find it a little disheartening.  
Looking at a review of the ink, seems like the color palate I'm seeing is about what they saw as well.  
It's also got a fair amount of shading, at least on the notebook paper I'm using right now.  
Fair amount of what feels like feathering, even though the words themselves aren't smeared out.  
Moreso it seems like the ink is clumped around individual fibers?

Zeus Purple is the other color.  
It's definitely much closer to red than purple, at least to me.  
That being said, it does very much read as a sort of greco-roman imperial color.  
In the pen I have and writing only on printer paper, doesn't seem to have much by way of shading.  
Looking at the review, there's apparently a bronze shimmer/sheen that can show up as well.

It feels at least mostly professional, which is a plus in my book.  
The lilac, less so.

\section{Inks I've Loved}

Emerald of Chivor: it's been a while since I used this ink, but I remember absolutely adoring it every time that I've used it.

Monteverde Ocean Noir. I've only used this in the fude tip before, but it was just such a beautiful color.  
Definitely the sort of ink that someone who's \#professional might be using too, which is always nice.  
A kind of low saturation dark blue.

Current Pen List\footnote{for my own posterity, mostly}

\begin{itemize}  
\item Hongdian Black with Fude Nib: Esterbrook Shimmer Lilac 11/23  
\item Jinhao Shark: Diplomat Sepia Black. 10/6  
\item Pilot Preppy: Diamine Bilberry. 10/6  
\item Shaeffer (blue): Empty  
\item Diplomat: Laban Zeus Purple 11/23  
\item Kaweko: Stipela Sepia. 10/6  
\item Monteverde: empty  
\item Shaeffer Calligraphy: missing

\end{itemize}

\end{document}