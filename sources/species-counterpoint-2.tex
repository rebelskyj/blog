\documentclass[12pt]{article}[titlepage]
\newcommand{\say}[1]{``#1''}
\newcommand{\nsay}[1]{`#1'}
\usepackage{endnotes}
\newcommand{\1}{\={a}}
\newcommand{\2}{\={e}}
\newcommand{\3}{\={\i}}
\newcommand{\4}{\=o}
\newcommand{\5}{\=u}
\newcommand{\6}{\={A}}
\newcommand{\B}{\backslash{}}
\renewcommand{\,}{\textsuperscript{,}}
\usepackage{setspace}
\usepackage{tipa}
\usepackage{hyperref}
\begin{document}
\doublespacing
\section{\href{species-counterpoint-2.html}{Species Counterpoint Progress}}
First Published: 2022 December 21

\section{Draft 1}
I know that \href{species-counterpoint-planning.html}{I said} I would be doing one set a week, but I've done so much first species counterpoint with a single voice and cantus firmus that I already feel comfortable moving on.
So, time to start second species.

As a reminder, the rules are:
\begin{itemize}
\item Consonant downbeats\footnote{3,5,6,8}
\item Offbeat disonances must be approached and left stepwise\footnote{C-B-C or C-B-A, as two examples}
\item No hidden (parallel) perfect intervals\footnote{successive downbeats (or upbeats) need to not be the same perfect interval}
\item Two notes written per note in the cantus firmus
\item Generally prioritize small skips and step-wise movement
\end{itemize}
It's shocking to me how much harder this becomes.
I immediately forgot that 4ths are dissonant, and I also had so many hidden parallel perfect intervals.
However, I think this is also coming back to me quickly, so maybe I'll be able to get through it in less than a week also.
\end{document}