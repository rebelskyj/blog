\documentclass[12pt]{article}[titlepage]
\newcommand{\say}[1]{``#1''}
\newcommand{\nsay}[1]{`#1'}
\usepackage{endnotes}
\newcommand{\1}{\={a}}
\newcommand{\2}{\={e}}
\newcommand{\3}{\={\i}}
\newcommand{\4}{\=o}
\newcommand{\5}{\=u}
\newcommand{\6}{\={A}}
\newcommand{\B}{\backslash{}}
\renewcommand{\,}{\textsuperscript{,}}
\usepackage{setspace}
\usepackage{tipa}
\usepackage{hyperref}
\begin{document}
\doublespacing
\section{\href{letter-to-the-editor.html}{Letter to the Editor}}
First Published: 2019 February 5

Prereading note: this also appears in the \href{http://www.thesandb.com/opinion/letter-to-the-editor-opposing-brownells-and-supporting-the-union-is-hypocritical.html}{S&B}
\section{Draft 4:27 January}
I am a white, heterosexual man from a high socio-economic status background who grew up in rural Iowa.
More specifically, I grew up in Grinnell, Iowa.
While attending Grinnell College, one important task I'm frequently asked to accomplish is acknowledging my privilege.
This is due in large part, I believe, to the fact that Grinnell College prides itself on social justice.
Social justice requires understanding not only problems, but also the underlying inequities that cause the problems.
An underlying issue can come up in multiple, otherwise unrelated events.
One such underlying issue is elitism at Grinnell College.
Two examples of this are visible in student complaints towards the administration's stances on unionization efforts and the Redmond-Brownell family donating money.

A quick fact before I get going.
Fewer than 30\% of adults in Iowa have a bachelor's degree or higher.\footnote{https://www.iowadatacenter.org/quickfacts}
That is, because most\footnote{https://www.univstats.com/colleges/grinnell-college/graduation-rate} students at Grinnell College will receive bachelor's degree, we are in the top third of the state in at least one important measure of social status.
So, when thinking about how we interact with the community, the privilege we have as college students cannot be neglected.

Now then, growing up in Grinnell, one of my clearest recurring memories of the educational system is how socio-economic class affects both how students are treated and how they act.
For those readers who may be unaware, there are a number of wealthy families in Grinnell, many of whom have direct connections to the faculty of Grinnell College.
There are also a large number of students, more than 35\%, who qualify for free and reduced lunches.\footnote{http://db.desmoinesregister.com/iowa-free-reduced-meals/?searchterms\%5Bcol1\%5D=grinnell&searchterms\%5Bcol2\%5D=}

I vividly remember hearing students in the second category speak about how they could never belong at Grinnell College.
The statements tended to focus on how they felt difficulty connecting with peers from higher socio-economic families.
Since students from Grinnell, Iowa who tend to attend Grinnell College also tend to come from higher socio-economic classes, Grinnell College seems to many in the Grinnell educational system enrolled mainly by students from high socio-economic status families.
If they had difficulty relating to students who attended the same schools, lived in the same general area, and did many of the same activities, how could they relate to people who shared none of these.
Of course, we all know that there are a large number of students at Grinnell College who are not from high socio-economic class backgrounds.
But, there exists a problem in making young, potential first-generation college students in the area see this.

Here enters the Redmond-Brownell family.
For those new to the area, the Redmond-Brownell family is a local family that owns their own business.
The business is profitable, and they use the money they make to improve quality of life in the city of Grinnell and the surrounding area.
However, their business is based in selling parts and accessories to firearms, which many students and alumni find problematic.

Ignoring where the money goes for a moment, there are two key points to note.
First, Brownells\footnote{their company} employs many local workers.
Second, some members of the Grinnell community are sustenance hunters, people who rely on hunting to feed their families.
I hope that it isn't hard to see how protesting Brownells as a company, and gun companies as a concept could and almost should be taken as saying that the way that these people feed their families is wrong.
I also hope it isn't difficult to see how that could be alienating to a potential student.

If we don't ignore where the money goes, however, we see that the money donated does not benefit their family or business interests.
Instead, it was used to create the Ignite Program, which offered college students the chance to teach a one day class to local young students, which, to me, is one of the best examples of social justice at Grinnell College.
The Ignite Program is a free, one day workshop which includes food for the students.
It takes students in Grinnell and surrounding communities, many of whom could never see themselves at Grinnell College, and shows them that Grinnell College is a place they could belong.
Many of the students I knew who felt that they could never belong at Grinnell had never been in an academic building, because they already knew that they didn't belong, so saw no point in confirming that fact.

By protesting the Redmond-Brownells' funding of the Ignite Program, the College community was implicitly agreeing with these students.
Removing that program would do nothing except make it harder for students in the local community to feel that they could belong at Grinnell College.

Moving on to the union, I'll start with my own personal biases.
To me, a union should protect exploited or easily exploitable groups.
Unions that do not should not exist, as they weaken the idea of unions, and make it harder to immediately sympathize with them.
As I mentioned above, by definition, being at Grinnell College is a sign of privilege, especially in Grinnell, Iowa.
Regardless of the other identities Grinnell College students have, in their identity as a student at Grinnell College, they come from a place of privilege.
By striving for a union, College students are weakening the system of a union.

For instance, one of the claims of the Union was that students at Grinnell are underpaid.
Right now, the wage in the Dining Hall is \$9.78 for the workers, and \$10.24 for student leaders.\footnote{https://www.grinnell.edu/admission/financial-aid/affording-grinnell/student-employment}
For reference, minimum wage in Iowa is \$7.25.
If you're a student whose parents work minimum wage jobs, think of how alienating hearing complaints that \$10 an hour isn't enough to support a single student can be.
More importantly, Iowa is an \say{at-will} state, where employees can be terminated \say{at will} by an employer.
While expressing intent to join or create a union should be one of the protected groups, suing for wrongful termination costs more than families can afford.
For that reason, being able to push for a union, knowing that a job is secure while doing so, is in and of itself a sign of incredible privilege in Grinnell, Iowa.

So in conclusion, just as I, should be aware the privilege different identities I have bring, so too should the whole student body all reflect on how being a member of the Grinnell College student community grants us privilege, especially in conjunction with the city of Grinnell, Iowa. \end{document}