\documentclass[12pt]{article}[titlepage]
\newcommand{\say}[1]{``\#1''}
\newcommand{\nsay}[1]{`\#1'}
\usepackage{endnotes}
\newcommand{\1}{\={a}}
\newcommand{\2}{\={e}}
\newcommand{\3}{\={\i}}
\newcommand{\4}{\=o}
\newcommand{\5}{\=u}
\newcommand{\6}{\={A}}
\newcommand{\B}{\backslash{}}
\renewcommand{\,}{\textsuperscript{,}}
\usepackage{setspace}
\usepackage{tipa}
\usepackage{hyperref}
\begin{document}
\doublespacing
\section{\href{letter-to-the-editor.html}{Letter to the Editor}}
First Published: 2019 January
\section{Draft 4:27 January}
I am a tall, white, heterosexual man from a high socio-economic status background who grew up in rural Iowa.
More specifically, I grew up in Grinnell, Iowa.
While attending to Grinnell College, one important task I'm frequently asked to accomplish is acknowledging my privilege.
This is due in large part, I believe, to the fact that Grinnell College prides itself on social justice.
Social justice requires understanding not only problems, but also the underlying inequities that cause the problems.
An underlying issue can come up in multiple, otherwise unrelated events.
One such underlying issue is elitism at Grinnell College.
Two examples of this are visible in student complaints towards the administration's stances on unionization efforts and the Redmond-Brownell family donating money.

A quick fact before I get going.
Fewer than 30\% of adults in Iowa have a bachelor's degree or higher.\footnote{https://www.iowadatacenter.org/quickfacts}
That is, because most\footnote{https://www.univstats.com/colleges/grinnell-college/graduation-rate} students at Grinnell College will receive bachelor's degree, we are in the top third of the state in at least one important measure of social status.
So, when thinking about how we interact with the community, the privilege we have as college students cannot be neglected.

Now then, growing up in Grinnell, one of my clearest recurring memories of the educational system is how socio-economic class affects both how students are treated and how they act.
For those readers who may be unaware, there are a number of wealthy families in Grinnell, many of whom have direct connections to the faculty of Grinnell College.
There are also a large number of students, more than 35\%, who qualify for free and reduced lunches.\footnote{http://db.desmoinesregister.com/iowa-free-reduced-meals/?searchterms\%5Bcol1\%5D=grinnell&searchterms\%5Bcol2\%5D=}

I vividly remember hearing students in the second category speak about how they could never belong at Grinnell College.
The statements tended to focus on how they felt difficulty connecting with peers from higher socio-economic families.
Since students from Grinnell, Iowa who are tend to attend Grinnell College also tend to come from higher socio-economic classes, Grinnell College seems to many in the Grinnell educational system enrolled mainly by students from high socio-economic status families.
If they had difficulty relating to students who attended the same schools, lived in the same general area, and did many of the same activities, how could they relate to people who shared none of these.
Of course, we all know that there are a large number of students at Grinnell College who are not from high socio-economic class backgrounds.
But, there exists a problem in making young, potential first-generation college students in the area see this.

Here enters the Redmond-Brownell family.
For those new to the area, the Redmond-Brownell family is a local family that owns their own business.
The business is profitable, and they use the money they make to improve quality of life in the city of Grinnell and the surrounding area.
However, their business is based in selling parts and accessories to firearms, which many students and alumni find problematic.

Ignoring where the money goes for a moment, there are two key points to note.
First, Brownells\footnote{their company} employees many local workers.
Second, some members of the Grinnell community are sustenance hunters, people who rely on hunting to feed their families.
I hope that it isn't hard to see how protesting Brownells as a company, and gun companies as a concept could and almost should be taken as saying that the way that these people feed their families is wrong.
I also hope it isn't difficult to see how that could be alienating to a potential student.

If we don't ignore where the money goes, however, we see that the money donated does not benefit their family or business interests.
Instead, it was used to create the Ignite Program, which offered college students the chance to teach a one day class to local young students, which, to me, is one of the best examples of social justice at Grinnell College.
The Ignite Program is a free, one day workshop which includes food for the students.
It takes students in Grinnell and surrounding communities, many of whom could never see themselves at Grinnell College, and shows them that Grinnell College is a place they could belong.
Many of the students I knew who felt that they could never belong at Grinnell had never been in an academic building, because they already knew that they didn't belong, so saw no point in confirming that fact.

By protesting the Redmond-Brownells' funding of the Ignite Program, the College community was implicitly agreeing with these students.
Removing that program would do nothing except make it harder for students in the local community to feel that they could belong at Grinnell College.

Moving on to the union, I'll start with my own personal biases.
To me, a union should protect exploited or easily exploitable groups.
Unions that do not should not exist, as they weaken the idea of unions, and make it harder to immediately sympathize with them.
As I mentioned above, by definition, being at Grinnell College is a sign of privilege, especially in Grinnell, Iowa.
Regardless of the other identities Grinnell College students have, in their identity as a student at Grinnell College, they come from a place of privilege.
By striving for a union, College students are weakening the system of a union.

For instance, one of the claims of the Union was that students at Grinnell are underpaid.
Right now, the wage in the Dining Hall is \$9.78 for the workers, and \$10.24 for student leaders.\footnote{https://www.grinnell.edu/admission/financial-aid/affording-grinnell/student-employment}
For reference, minimum wage in Iowa is \$7.25.
If you're a student whose parents work minimum wage jobs,\footnote{and the plural is intentional, because very frequently multiple are needed} think of how alienating hearing complaints that \$10 an hour isn't enough to support a single student can be.
More importantly, Iowa is an \say{at-will} state, where employees can be terminated \say{at will} by an employer.
While expressing intent to join or create a union should be one of the protected groups, suing for wrongful termination costs more than families can afford.
For that reason, being able to push for a union, knowing that a job is secure while doing so, is in and of itself a sign of incredible privilege in Grinnell, Iowa.

So in conclusion, just as I, should be aware the privilege different identities I have bring, so too should the whole student body all reflect on how being a member of the Grinnell College student community grants us privilege, especially in conjunction with the city of Grinnell, Iowa. 
\section{Draft 3:27 January}
I am a tall, white, heterosexual man from a high socio-economic status background who grew up in rural Iowa.
More specifically, I grew up in Grinnell, Iowa.
While attending to Grinnell College, one important task I'm frequently asked to accomplish is acknowledging my privilege.
This is due in large part, I believe, to the fact that Grinnell College prides itself on social justice.
Social justice requires understanding not only problems, but underlying inequities and inequalities that manifest as problems.
The elitism shown in complaining about the College's response to unionization efforts and the Redmond-Brownell family donating money as social justice complaints quite frankly horrifies me. 

A quick fact before I get going.
Fewer than 30\% of adults in Iowa have a bachelor's degree or higher.\footnote{https://www.iowadatacenter.org/quickfacts}
That is, because most\footnote{https://www.univstats.com/colleges/grinnell-college/graduation-rate} students at Grinnell College will receive bachelor's degree, we are in the top third of the state in at least one, important measure of social status.
So, when thinking about how we interact with the community, the privilege we have as college students cannot be neglected.

Growing up in Grinnell, one of my clearest memories of the educational system is how socio-economic class affects student actions and reactions.
For those readers who may be unaware, there are a number of wealthy families in Grinnell, many of whom have direct connections to the faculty of Grinnell College.\footnote{staff is intentionally left off here, as many of the staff members for Grinnell College are not paid enough to be considered wealthy}
There are also a large number of students, more than 35\%, who qualify for free and reduced lunches.\footnote{http://db.desmoinesregister.com/iowa-free-reduced-meals/?searchterms\%5Bcol1\%5D=grinnell&searchterms\%5Bcol2\%5D=}

I vividly remember hearing students in the second category speak about how they could never belong at Grinnell College.
The statements tended to focus on how they felt unable to connect with peers from higher socio-economic families.
Since Grinnell IA students who are tend to attend Grinnell College also tend to come from the higher socio-economic classes in Grinnell, Grinnell College must be made up of students from high socio-economic .
 so they would be socially isolated.
Of course, we all know that there are a large number of students at Grinnell College who are not from high socio-economic class backgrounds.
But, there exists a problem in making young, potential first-generation students see this.

Here enters the Redmond-Brownell family.
In case you're new to the college, the Redmond-Brownell family is a local family that owns their own business.
The business is profitable, and they put much of the money into improving the quality of life in the city of Grinnell and the state of Iowa.
When asked why, they tend to give answers that relate to wanting their community to be the best it can be.
However, their business is based in selling parts and accessories to firearms, which many students and alumni find problematic.

Ignoring where the money goes for a moment, there are two key points to note.
First, Brownells employees many local workers.
Second, some members of the Grinnell community are sustenance hunters, people who rely on hunting to feed their families.
To both of these sets, I hope that it isn't hard to see how protesting Brownells as a company, and gun companies as a concept could and almost should be taken as saying that the way that they feed their families is wrong.
I also hope it isn't difficult to see how that could be alienating to a potential student.

If we don't ignore where the money goes, however, which we shouldn't, we see that the money donated does not benefit their family or business interests.
Instead, it was used to create the Ignite Program, which offered college students the chance to teach a one day class to local young students.
To me, that program is one of the best examples of social justice at Grinnell.
The Ignite Program is a free, one day workshop, which includes food for the youth.
It takes students in Grinnell and surrounding communities, many of whom could never see themselves at Grinnell College, and shows them that Grinnell College is a place they could belong.
Many of the students I knew who felt that they could never belong at Grinnell had never been in an academic building, because they already knew that they didn't belong. so saw no point in trying.

By protesting the Redmond-Brownells' funding of the Ignite Program, the College community was implicitly agreeing with these students.
Removing that program would do nothing except make it harder for students in the local community to feel that they could belong at Grinnell College.
Of course, many of the students at Grinnell College never had these chances, and still made it here.
But trying to argue that because a few have made it, anyone can, is a prime example of survivorship bias.

Moving on to the union, I'll start with my own personal biases.
To me, a union should protect exploited or easily exploitable groups.
Unions that do not should not exist, as they weaken the idea of unions, and make it harder to immediately sympathize with them.
As I mentioned above, by definition, being at Grinnell College is a sign of privilege, especially in Grinnell, Iowa.
Regardless of the other identities Grinnell College students have, in their identity as a student at Grinnell College, they come from a place of privilege.
By striving for a union, College students are weakening the system of a union.

For instance, one of the claims of the Union was that students at Grinnell are underpaid.
Right now, the wage in the Dining Hall is \$9.78 for the workers, and \$10.24 for student leaders.\footnote{https://www.grinnell.edu/admission/financial-aid/affording-grinnell/student-employment}
For reference, minimum wage in Iowa is \$7.25.
If you're a student whose parents work minimum wage jobs,\footnote{and the plural is intentional, because very frequently multiple are needed} think of how alienating hearing complaints that \$10 an hour isn't enough to support a single student can be.
More importantly, Iowa is an \say{at-will} state, where employees can be terminated \say{at will} by an employer.
While expressing intent to join or create a union should be one of the protected groups, suing for wrongful termination costs more than families can afford.
For that reason, being able to push for a union, knowing that a job is secure while doing so, is in and of itself a sign of incredible privilege in Grinnell, Iowa.

So in conclusion, just as I, should be aware the privilege different identities I have bring, so too should the whole student body all reflect on how being a member of the Grinnell College student community grants us privilege, especially in conjunction with the city of Grinnell, Iowa.
\section{Draft 2: 26/7 January}
I am a tall, white, heterosexual man from a high socio-economic status background who grew up in rural Iowa, in a little city most of you might know, Grinnell.
Since coming to Grinnell College, I've been asked on many occasions to reflect on and be aware of my privilege.
While I have, I feel often that Grinnellians fail to do so.
In particular, the unaware elitism that exists in Grinnell College students' treatment of the union issue and the Redmond-Brownell family's donations is horrid to me.

The easiest way to explain this elitism\footnote{at least to me} is through the statement that education improves people.
As a result of this statement, it's easily made clear that a lack of education makes someone worse.
So, as students at Grinnell College, a top-ranked post-secondary educational institution that prides itself on a dedication to social justice, it's our responsibility to be aware of the privilege that this education provides us.

Fewer than 30\% of adults in Iowa have a bachelor's degree or higher.\footnote{https://www.iowadatacenter.org/quickfacts}
A bachelor's degree already sets any student at Grinnell in the top third of the state in social standings.
So, when thinking about how we interact with the community, the privilege we have cannot be neglected.

So far as I can tell, nothing I've written so far\footnote{other than the fourth and potentially third sentences} is particularly controversial.
As a result, it's hard for me to understand where the lack of acknowledgement of this comes in when speaking about the two issues.

Growing up in Grinnell, one of my clearest memories of the educational system is how socio-economic class affects student actions and reactions.
For those readers who may be unaware, there are a number of wealthy families in Grinnell, many of whom have direct connections to the faculty of Grinnell College.
There are also a large number of students, with numbers I've heard ranging from 30-50\%, who qualify for free and reduced lunches.
I remember vividly hearing students in the second category speak about how they could never belong at Grinnell College.
These students tended to speak about how they felt unable to connect with peers from higher socio-economic families, and since the Grinnell Community School students who are tend to attend Grinnell College also tend to come from the higher socio-economic classes in Grinnell, Grinnell College must be made up of students like this, so they would be socially isolated.
Of course, we all know that there are a large number of students at Grinnell College who are not from high socio-economic class backgrounds.
But, there exists a problem in making young, potential first-generation students see this.

Here enters the Redmond-Brownell family.
In case you're new to the college, the Redmond-Brownell family is a local family that owns their own business.
The business is profitable, and they put much of the money into improving the quality of life in the city of Grinnell and the state of Iowa.
When asked why, they tend to give answers that relate to wanting their community to be the best it can be.
However, their business is based in selling parts and accessories to firearms, which many students and alumni find problematic.

Ignoring where the money goes for a moment, there are two key points to note.
First, Brownells employees many local workers.
Second, some members of the Grinnell community are sustenance hunters, people who rely on hunting to feed their families.
To both of these sets, I hope that it isn't hard to see how protesting Brownells as a company, and gun companies as a concept could and almost should be taken as saying that the way that they feed their families is wrong.
I also hope it isn't difficult to see how that could be alienating to a potential student.

If we don't ignore where the money goes, however, which we shouldn't, we see that the money donated does not benefit their family or business interests.
Instead, it was used to create the Ignite Program, which offered college students the chance to teach a one day class to local young students.
To me, that program is one of the best examples of social justice at Grinnell.
The Ignite Program is a free, one day workshop, which includes food for the youth.
It takes students in Grinnell and surrounding communities, many of whom could never see themselves at Grinnell College, and shows them that Grinnell College is a place they could belong.
Many of the students I knew who felt that they could never belong at Grinnell had never been in an academic building, because they already knew that they didn't belong. so saw no point in trying.

By protesting the Redmond-Brownells' funding of the Ignite Program, the College community was implicitly agreeing with these students.
Removing that program would do nothing except make it harder for students in the local community to feel that they could belong at Grinnell College.
Of course, many of the students at Grinnell College never had these chances, and still made it here.
But trying to argue that because a few have made it, anyone can, is a prime example of survivorship bias.

Moving on to the union, I'll start with my own personal biases.
To me, a union should protect exploited or easily exploitable groups.
Unions that do not should not exist, as they weaken the idea of unions, and make it harder to immediately sympathize with them.
As I mentioned above, by definition, being at Grinnell College is a sign of privilege, especially in Grinnell, Iowa.
Regardless of the other identities Grinnell College students have, in their identity as a student at Grinnell College, they come from a place of privilege.
By striving for a union, College students are weakening the system of a union.

For instance, one of the claims of the Union was that students at Grinnell are underpaid.
Right now, the wage in the Dining Hall is \$9.78 for the workers, and \$10.24 for student leaders.\footnote{https://www.grinnell.edu/admission/financial-aid/affording-grinnell/student-employment}
For reference, minimum wage in Iowa is \$7.25.
If you're a student whose parents work minimum wage jobs,\footnote{and the plural is intentional, because very frequently multiple are needed} think of how alienating hearing complaints that \$10 an hour isn't enough to support a single student can be.
More importantly, Iowa is an \say{at-will} state, where employees can be terminated \say{at will} by an employer.
While expressing intent to join or create a union should be one of the protected groups, suing for wrongful termination costs more than families can afford.
For that reason, being able to push for a union, knowing that a job is secure while doing so, is in and of itself a sign of incredible privilege in Grinnell, Iowa.

So in conclusion, just as I, should be aware the privilege different identities I have bring, so too should the whole student body all reflect on how being a member of the Grinnell College student community grants us privilege, especially in conjunction with the city of Grinnell, Iowa.
\section{Draft 1: 26 Jan}
Grinnell College prides itself on being an institution committed to social justice.
And, as one might expect, different institutions within the school interpret this differently, especially when it comes to priorities.
To some, the idea of social justice based on stock holdings is more important than social justice based on funding student educations.
In all of these cases, there are rational\footnote{and irrational} arguments on most all sides of the issues.

However, very frequently, the arguments that I've seen against College administrative positions seem incredibly horrible to me.
In particular, I'd like to focus on two events that felt incredibly elitist to me, as a lifelong resident of Grinnell: the protest of the Redmond-Brownell family giving money to Grinnell College\footnote{and the community as a whole} and the drive for a union.

Growing up in Grinnell, one of the most clear pieces of the educational system is socio-economic class affecting both how students are treated and act.
For those readers who may be unaware, there are a number of wealthy families in Grinnell, including many affiliated with the faculty.
There are also a large number of students\footnote{I think 50\% but should double check that} who qualify for free and reduced lunches.
In many cases, I've heard students who fit into the second category speak about how there's no way they could belong at Grinnell College.
The reasonings tend to fall into the feelings that, since the Grinnell Community School students who are likely to attend Grinnell College tend to come from the higher socio-economic classes in Grinnell, Grinnell College must be made up of students like this.
Of course, we all know that there are a large number of students at Grinnell College who are not from high socio-economic class backgrounds.
But, there exists a problem in making young, potential first-generation students see this.

So, enter the Redmond-Brownell family.
In case you're new to the college, the Redmond-Brownell family is a local family that owns their own business.
The business is profitable, and they put much of the money into improving the quality of life in the city of Grinnell and the state of Iowa.\footnote{add a why}
However, since their business is based in selling parts and accessories to firearms, many students and alumni see problems accepting money from the family.

What seldom gets reported, however, is where the money they donate to Grinnell College goes.
The money they donate was used to create a program called the Ignite Program, which let students teach a one day class to local young students.
To me, that program is one of the best examples of social justice at Grinnell.
The Ignite Program is a free, one day workshop, which includes food.
While the program is not without problems,\footnote{mostly relating to bringing the children to the program} at its core, it strives for social justice.
It takes students in Grinnell and surrounding communities, many of whom could never see themselves at Grinnell College, and shows them that Grinnell College is a place they could belong.
Many of the students I knew who felt that they could never belong at Grinnell had never been in an academic building, because they felt that they didn't belong.

By protesting the Redmond-Brownells' funding of the Ignite Program, the College community was implicitly agreeing with these students.
Removing that program would do nothing except make it harder for students in the local community to feel that they could belong at Grinnell College.
More to the point of the actual donations, however, some members of the Grinnell community are sustenance hunters.
For them, protesting Brownells as a company is saying that the way that they feed their families is wrong.
It's not hard to see how that could be alienating to a potential student, at least to me.

Moving on to the union, I'll start with my own personal biases.
To me, a union should protect exploited or exploitable groups.
By definition, a Grinnell degree is a sign of privilege, especially in rural Iowa.
Within the state, fewer than 29\% of adults have a bachelor's degree.\footnote{https://www.iowadatacenter.org/quickfacts}
So, regardless of the other identities Grinnell College students have, in their identity as a student at Grinnell College, they come from a place of privilege.
By striving for a union, College students are weakening the system of a union.

For instance, one of the claims of the Union was that students at Grinnell are underpaid.
Right now, the wage in the Dining Hall is \$9.78.\footnote{https://www.grinnell.edu/admission/financial-aid/affording-grinnell/student-employment}
For reference, minimum wage in Iowa is \$7.25.
If you're a student whose parents work minimum wage jobs,\footnote{and the plural is intentional, because very frequently multiple are needed} think of how alienating hearing complaints that \$10 an hour isn't enough to support a single student would be.
More importantly, Iowa is an \say{at-will} state, where employees can be terminated \say{at will} by an employer.
While wanting to join a union should be one of the protected groups, that requires the money to hire an attorney.
So, to these students, being able to push for a union while secure in the knowledge that a student's work will not be at risk is another incredible sign of privilege.

In conclusion, just as I, a large, white, man from a high socio-economic background should be aware of my privilege and how it intersects with my interactions with others, so too should we all reflect on how being a member of the Grinnell College student community grants us privilege, especially when we speak about social justice.
\end{document}