\documentclass[12pt]{article}[titlepage]
\newcommand{\say}[1]{``#1''}
\newcommand{\nsay}[1]{`#1'}
\usepackage{endnotes}
\newcommand{\B}{\backslash{}}
\renewcommand{\,}{\textsuperscript{,}}
\usepackage{setspace}
\usepackage{tipa}
\usepackage{hyperref}
\begin{document}
\doublespacing
\section{\href{schedules.html}{On Schedules}}
First Published: 2022 October 27

\section{Draft 1}
I find that my life runs better when it's more heavily scheduled.
Often, this means that I have activities planned for particular times, but it's more than that.
Lately I've been running into a fair number of friends who I realize I've lost contact with in the past few months, not through any malice on either of our ends, but simply because it's easy to forget people in graduate school.

I don't want to forget the people who mean a lot to me, so I need to figure out a way to get into the habit of messaging people I care about.
I was recently having a conversation with someone where I mentioned that I've been called really considerate before because I had written down a reminder, and then remembered the event from it.
I don't think of that as particularly speaking to my care/memory, because it was just me sending a note into the aether, where it eventually came back to me.
The person I was talking to is in the camp of extended memory as memory\footnote{if you don't know what this means, the tl;dr is basically research shows that when we read something we often remember where we read it, rather than what we read, because that's more efficient. Some consider the writing to count as memory}, and so assured me that not having the memory to remember important things wasn't important, especially since I had a system to figure it out.

Still, though, I know that I have systems which work. I really just need to use them.
Maybe I should just set a reminder to text people every few weeks.
\end{document}