\documentclass[12pt]{article}  
\newcommand{\say}[1]{``#1''}  
\newcommand{\nsay}[1]{`#1'}  
\usepackage{endnotes}  
\newcommand{\B}{\backslash{}}  
\renewcommand{\,}{\textsuperscript{,}}  
\usepackage{setspace}   
\usepackage{tipa}  
\usepackage{hyperref}  
\begin{document}  
\doublespacing  
\section{\href{dissertation-camp.html}{On Dissertation Writing Camp}}  
First Published: 2025 May 29

\section{Draft 1: 29 May 2025}

Last week I had the really nice opportunity to take part in a university-sponsored dissertation writing camp.  
Earlier iterations of this site spent much more time reflecting on the things I experienced, rather than just thinking about things, and so it feels reasonable to do so now.  
It ran from Monday to Friday, 9am to 430ish pm except for Friday which ended around 1300.

Each morning started with a fifteen to twenty minute presentation about some important part of the dissertation process, reminders for the camp in general, goal setting time.  
Because I respect the idea that copying wholesale is never appreciation, I'm not going to list every activity we did or every presentation's content.\footnote{no, it's not that I don't remember them all, that would be ridiculous}  
Goal setting was, despite how ridiculous it felt to me every day, really helpful, if only because it let me set my mind at ease knowing that I had things to do.  
It also helped me realize that what I set as a minimum goal is really almost always about what the actual amount of work I'll do in a day.  
I'm sure that there is a lot I could say about how\footnote{ughhh I hate that my default for h is always left index, not the right, as is appropriate} I consider my level of work always the minimum.  
I did slowly lower my expectations over the week, which was also probably good.

Something I found really interesting was just how much I not only knew a lot of the writing advice, but how much I had internalized it.  
I no longer write a first draft with any expectation of it having anything but potentially the content that I want in the final version.\footnote{wow yesterday's folly is such a good case study in what I do}  
I know that habit is important in writing, and I know that like passive voice generally bad, unless trying to distance self from the project.

I also found it really interesting how different my reality is from those around me.  
I mentioned that my goal each day was usually in the 2000 to 4000 range and got a lot of shocked looks.  
When I said that I absolutely hit that goal every day, they definitely did not look at me like I was fully human.  
For some, this made sense.  
Especially if I was working on a later draft of one of the chapters, there's no way that I could get that many words out.  
As someone who was literally just vomiting words on the paper because I can't edit an empty page, though, it felt strange to know that they were writing so little.\footnote{ok let's see, we were there 9-430, 9-930 and 16-1630 were both group times, and 1215-1300 was designated talk time, 1145-1215 was lunch, so 9:30-1145 is about 2.25 hours, and 1300-1600 is 3, so that's a little over 5 hours. 400 words an hour feels like such a small number. That's (breaking out calculator time) just over 6 words a minute. I cannot imagine staring at a page for ten seconds at a time, only to quickly write a single word}  
I do know a number of people were reading literature or otherwise gathering knowledge, and I'm sure that some are of the team that like it's important to get the final words on at a time, which is a totally valid take.\footnote{I remember reading the writing advice that some people edit as they go so that each word is great on the page and others edit draft after draft, neither is necessarily bad}

It was really interesting to see that so much of what I find hard about the thesis\footnote{still not totally sure what the difference is, found it absolutely incredible that they kept abbreviating dissertation as \say{diss}, because \say{I worked on my diss all week} feels like something that someone who loses a lot of ad hominem (I don't use italics for Latin because I am not a coward) arguments would do} process is shared not just between fields, but between like the entire graduate school.  
I'm very glad not to be a literary studies candidate, because I absolutely could not write hundreds of pages of text without many pictures, and the fact that they do is so impressive.  
In general, though, every time that I wanted to start playing a game instead of working, the fact that I would look around and see my explicit peers working hard meant that I did not.  
I did get up for a lot of walks, but like that's so fine.

The lunchtime seminars were also cool.  
On Monday we kind of just went over the most normative writing advice\footnote{do it, schedule it, make deadlines, etc}.  
Tuesday was something similar.  
Wednesday was an absolutely incredible seminar about mental health.\footnote{the professor giving that talk is just my favorite professor, and everything she teaches about always resonates hard}  
Thursday we learned about the absolute requirements from the graduate school about our thesis\footnote{and I am my father's son, asking questions that no one had ever considered. Highlights were absolutely \say{if you've removed maximum abstract length, is there a minimum?} and upon being told no, \say{so a blank abstract?}, and \say{if someone on the committee dies between approving the graduation and signing the e-form, what do we do?}}.

The closing sessions were often something similar, though tended to also have some element of interactivity.  
I liked them, in general.  
It was nice chatting with people and reflecting on the goals I had.

On Wednesday of camp, we were all hitting the midweek slump.  
I suggested a rousing \say{one, two, three, team!} to wake us up, and people semi-begrudgingly participated.  
As we closed for the night someone shyly raised their hand and asked if we could do it again.  
I felt so seen then.

Really, that reflected the broader takeaway I had from the camp experience: I wanted more camp.  
I, for one, think that an overnight writing lock-in would be really productive for me.  
More than that\footnote{oof I use this phrase far too much}, though, one of the explicit goals of the camp was that we bond with one another.  
Camp-like activities\footnote{and anything culty, in general} are a great way for people to bond.  
People seemed generally receptive to the idea, but no one really seemed to want to put in into practice.

The camp coordinators were also fantastic.  
All three are members of the writing center during the regular day, and they set up fifteen minute slots for us to talk about whatever was going on with our thesis.  
I'm not sure if I used those correctly, because in general I kind of just went and said \say{I think this} and got told \say{yeah, sure}.  
That reflects far more on me, though, because I did not have a clear goal for what I wanted in the conversations, and I do know the low hanging fruit that applies to all people.

All in all, I really loved the week, would absolutely do another one if ever given the chance, and recommend anyone else to do one as well.\footnote{wow look at this, I managed to do a full post in under an hour.}

\section{Daily Reflection 29 May 2025}

\begin{enumerate}

\item Top Priorities:

\begin{itemize}

\item Sleep:

\begin{itemize}

\item Keeping sleep time sacred?

Yeah! I'm trying a new sleep schedule, and today at least it felt really good!  
I'm not sure if it's just the thrill of trying something new, but despite the folly yesterday, I am absolutely currently setting my sleep schedule based on the idea of waking and sleeping on set time tables. There's a question I have about whether the real best sleep schedule might just be sleeping until well rested twice a day, but the big issue with that is that I do absolutely oversleep when given the chance.

\item Good sleep hygiene?

Decent!  
I moved my bed, like I have been thinking about doing for almost five years now\footnote{yes, one of the first things I considered when moving into my current apartment was that I put the bed in sideways from ideal. No I have not done anything to rectify that issue in the past five years}.  
I tried to not spend time in bed when I wasn't doing sleep tasks, and wow do I need a beanbag chair I think.\footnote{I'm trying to think what my ideal way of relaxing/lounging is when I'm trying to write or read or type on computer or draw or whatnot. Beanbag chairs have the nice part which is that their amorphous nature means that I'm always supported, but struggle because of the whole \say{no place to rest anything, there's something inherently weird about being a single person who owns a beanbag chair, }I don't know if I like the way that they sit in the room, and I cannot recline, which is a large goal.} Maybe I just need another mattress? Something to lie on on the ground that's also comfortable but not meant for sleep. I'm sure that there exist answers, but.

\item Sleeping enough?

I think so! New sleep schedule doing great things for me today.  
Part of the new schedule was not going to bed before a certain time, and it's really weird to me just how much time there is in a day. I ended up needing to make a list of just like \say{here are things that take time that you generally want to do. What if you tried that?} because I saw that I had an extra three hours.  
It was really cool to be able to just dedicate 45 minutes to reading a book, which is something I'm always wanting to do more and never actually doing!  
I also got a nice workout in, which was really cool and fun.

\item How well rested do I feel?

Honestly really good.  
Much as I loathe the linear clock, I do have to admit that saying \say{this is get up time and so I must arise} is much more effective in me than most of the other systems I've tried, at least for today.

\end{itemize}

\item Feed myself:

\begin{itemize}

\item Did I eat breakfast?

Yesterday I did! Today I had an apple.

\item Did I eat a second meal?

Yesterday I had two second meals! I ate something at 11ish and at 1400/1600\footnote{it was half finished before naptime}.

\item Did I eat dinner?

I had two dinners! Because first dinner was right after finishing the snack\footnote{is what I'll call an entire box of Kraft mac}, I think that I ended up being hungry again before it was time for bed.  
Staying up an extra few hours probably contributed to that as well, and so I had second dinner too!  
It's wild how good I feel when I eat real food, is really what I think that I'm getting at.

\item Water?

Generally! I found myself wanting to drink water again, which was really cool.

\end{itemize}

\item Family:

\begin{itemize}

\item Am I neglecting any familial obligations?

Not really. I could listen to the album more, but that's usually true. Might try to do that in the gym today, or else during a walk.

\end{itemize}

\item Movement:

\begin{itemize}

\item Am I stretching at least 5 minutes per hour of computer time?

I did not so much explicitly yesterday, and I did muse about that in the folly. In general, I think that it is probably better for me, even if worse for my immediate productivity, to do so, so I think we're going to shoot for it again.  
I did, however, stretch a bunch yesterday, and so might average out to once per hour.

\item Am I generally making efforts to be limber?

Kind of! I forgot to stretch before leaving for work today, but I can do that at the first hour break.\footnote{since I'm starting work at 700 today, I've also just started well before the usual stretch time.}

\end{itemize}

\item Spirituality:

\begin{itemize}

\item Time for prayer?

This is absolutely something that I can add in since I absolutely have been wanting to find more things to do.  
Prayer is important, even if silence is actively aggressive to me these days.

\item Prayer?

See above.

\item Time for sacred silence?

See above. Then again, I have been spending the last little bit before bed in silence and the first little bit in the morning, just because I leave the electronics in another room before sleeping, and it does really feel nice.

\item Deep breaths?

More than before, at least.

\end{itemize}

\end{itemize}

\item Secondary Priorities:

\begin{itemize}

\item Thesis/ Ph.D. work:

\begin{itemize}

\item Keeping up on the writing deadlines?

Eh. Yesterday I wrote a draft of the my conference presentation and finished rewriting some pieces of the code that had broken from a good idea I had.  
I'm trying to resubmit some jobs, but I apparently had a bunch of jobs half-fail at the start of the month and just completely forgot to check up on them, which I feel bad about.\footnote{both because I took up computing resources and because I will now have a lower priority for submitting jobs myself. Hmm we have computers in the lab that we aren't always using, could consider using those.}

Today's goals are revising the paper and making the plots for it, with some more thesis work later down.

\item Reading the necessary things?

Nothing is necessary, to this day.

\item Making graphs?

Nope, but we're going to do that now.  
Wild that I didn't ever pay attention to the majority of the jobs breaking down.

\begin{itemize}

\item Visual depiction of Latin Hypercube

\item Visual depiction of Loomis-Wood Diagrams

\item Visual depiction of Spectral Stacking

\item Visual depiction of how the fitness of the spectral stacks is really reliant on the graphs being the right height

\item Plots from the actual results of the runs, to make sure that it worked out.\footnote{SSC, AAT, if any vib states were good, what happened to the computations, etc}.

\end{itemize}

\begin{itemize}

\item Visual depiction of Grid Search

\item Visual depiction of random search

\item I guess that the stuff for intro to quantum video counts here.

\end{itemize}

\item Organizing citations?

I don't think that I need to do this.

\end{itemize}

\item Love:

\begin{itemize}

\item Taking risks?

Not a ton, but some very marginal ones.

\item Making efforts?

Kind of! At the absolute minimum I'm reading the book on relationships.

\item Showing affection?

Decently! I don't know if I was really interacting with others in person, so maybe it's important to do more digital shows of affection.

\item Being honest?

Yeah, I think so. At the very least not being dishonest!

\item Being open?

Generally! Luckily even if I'm an open book, that does not immediately mean people are going to read.

\item Being appropriately vulnerable?

Sure!

\end{itemize}

\end{itemize}

\item Adjacent to Primary and Secondary:

\begin{itemize}

\item Typing Practice?

Forgot yesterday, and today I'm planning to use it as a semi break if I don't feel like working through the list.

\item Applying to jobs?

Forgot to apply yesterday, have it on the list for today.

\item Reading the things I think could be good?

Eh, I gave up on one book because I realized that I didn't care about the remainder of its content, and wow that was such a freeing realization.\footnote{thanks dissertation camp}

\item Making manim videos?

No, but also on the list\footnote{if very low down}

\end{itemize}

\item Cleaning?

\begin{itemize}

\item Office

Nope

\item Home

A lot actually. I don't think that it's made a huge difference yet, but I think that it will soon show fruit.\footnote{there's the issue that at a certain point, mess and slightly more or less mess feel identical}

\item Car

No but doing that today

\item Computer

No

\item Other as needed

Generally tried to clean my mind yesterday. If the folly wasn't enough of a sign, I realized that I really need to be journaling more.

\end{itemize}

\item External Obligations:

\begin{itemize}

\item Guitar for wedding?

Not a ton.

\item Travel plans?

I realized that I haven't made them! That doesn't mean that I'm making them, though.

\item Talks for parks?

Honestly, I'm happy with the talk I have, so I'm just going to take this off the daily reflection

\item Other requested talks?

Nope! But I also don't think that it's the largest priority right now.

\item Talks for conferences?

Made a draft yesterday, and it's rough, but I think that it will rapidly take shape as I add back in more things.

\end{itemize}

\item Tertiary Goals:\footnote{mmmm off by N numbering}

\begin{itemize}

\item Blogging?

Yesterday.

 And!

 Today.

\item Reading?

I read for like 45 minutes yesterday and I spent another lot of time listening to audiobook, which felt nice.

\item Web Noveling?

Nope, which I just realized somehow didn't make it form unordered to ordered list of tasks.  
It's on there now.

\item Guitar?

Not a ton.

\item Other hobbies?

I realize that journal is a bit of a hobby, and I remembered that I have a fun fitness-adjacent set of games, so spent some time with them yesterday.

\end{itemize}

\item Quaternary Goals:

\begin{itemize}

\item Letter writing

It's on the list, even if it's at the bottom of the list.  
Hmm should I make a full list of things again somewhere accessible so that I can make ordered lists each day?  
Probably soon, even if not right now because that's absolutely a form of productive procrastination.

\item Handwriting/penmanship

I filled like five pages yesterday with lines and circles, and I think that I'm getting slightly better at making them good. It's definitely not a fast process, and I have to imagine I'm already well into diminishing returns.

\item Picking a new signature

Nope. Putting it on the list though.

\end{itemize}

\end{enumerate}

\end{document}