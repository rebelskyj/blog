\documentclass[12pt]{article}[titlepage]
\newcommand{\say}[1]{``#1''}
\newcommand{\nsay}[1]{`#1'}
\usepackage{endnotes}
\newcommand{\1}{\={a}}
\newcommand{\2}{\={e}}
\newcommand{\3}{\={\i}}
\newcommand{\4}{\=o}
\newcommand{\5}{\=u}
\newcommand{\6}{\={A}}
\newcommand{\B}{\backslash{}}
\renewcommand{\,}{\textsuperscript{,}}
\usepackage{setspace}
\usepackage{tipa}
\usepackage{hyperref}
\begin{document}
\doublespacing
\section{\href{writing-2.html}{On Writing (Continued)}}
First Published: 2022 February 9

\section{Draft 1}
A part of me feels like this isn't really a fair article to write, because I'm not totally sure how this essay will differ from yesterday's.
I guess one major difference is that yesterday was looking forward at what a writing habit might look like and internally at what's preventing me from having a good writing habit as I am currently.
Today's, on the other hand, will be moreso focused on what writing looked like yesterday and how I think I can continue it looking forward.
With this disclaimer out of the way, on to the meat of the document.

First,\footnote{Wow a part of me still really thinks about putting firstly whenever I make these lists, but that habit has been beaten (metaphorically of course) out of me} I was shocked at my writing rate yesterday.
I started just free writing a book idea I have had floating around for a little while,\footnote{also while generating the document, I realized that there were a few other books I'd started in the past and just immediately given up on} and set a ten minute timer.
At the end of ten minutes, I found that I had written around seven hundred words.
That was really shocking to me.
I didn't even know that my typing speed was seventy words a minute, let alone my creative writing\footnote{albeit at free flowing consciousness rather than well edited} speed.

Second, it was a nice reminder that some skills are like the metaphorical riding of bicycles.
I had a major struggle trying to fix the part of my voice that wants me to constantly go back and fix the grammar and spelling\footnote{and ideas, though that's less relevant} mistakes that I make while typing so quickly.
When I started writing yesterday, though, it was really easy for me to shut off the voice that said I should go back and fix the typos.
That made the writing time go by much more quickly, and it certainly helped me get more words on the page.

Third, it was a little harder than I thought to stop writing.
I read somewhere that it's really helpful to stop writing whenever your timer goes off, because if you were slogging to get through he minutes, it prevents some amount of burnout.
Conversely, if the words are flowing really easily for you, it makes it easier to want to write the next day.
It definitely worked, because I really want to write today, but I'm just not sure if I can find the time to write with the rest of my night being what it is.

In conclusion\footnote{wow look a five paragraph essay, habits do die hard}, I really liked restarting writing yesterday, even if I'm unsure how sustainable the quick high will be.
I'm wondering if I can do the additive, increasing my writing time by a minute a day until I get to a word count I'm happy with.
At some point I will also need to add some editing time, because wow the rough draft is incredibly rough.

439
78
\end{document}