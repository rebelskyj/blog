\documentclass[12pt]{article}[titlepage]
\newcommand{\say}[1]{``#1''}
\newcommand{\nsay}[1]{`#1'}
\usepackage{endnotes}
\newcommand{\1}{\={a}}
\newcommand{\2}{\={e}}
\newcommand{\3}{\={\i}}
\newcommand{\4}{\=o}
\newcommand{\5}{\=u}
\newcommand{\6}{\={A}}
\newcommand{\B}{\backslash{}}
\renewcommand{\,}{\textsuperscript{,}}
\usepackage{setspace}
\usepackage{tipa}
\usepackage{hyperref}
\begin{document}
\doublespacing
\section{\href{theatre-final.html}{Theatre Final}}
First Published: 2018 December 12
\section{Draft 5: 12 December}
One lens that is useful for viewing Shakespeare's Scottish tragedy, \textit{Macbeth}, is the lens of loyalty.
When viewed through that lens, the show seems to be a cautionary tale about the nature of disloyalty.
While disloyalty may have temporary gains, it will ultimately damn you and be the cause of your undoing.
But we may also learn about the show through his word choice.
Shakespeare, as every young student learns, invented many words.
And, as a poet, he is more deliberate than most in his repetitions of words.
More than that, however, Shakespeare uses words in conjunction with others.
So, we can see both how Shakespeare sees loyalty, and how his characters see it by looking at where in the show the four utterances of \say{loyal} occur.

The first utterance of loyalty occurs during the exposition of the show, when the audience learns about the thane of Cawdor's rebellion against the King of Scotland.
Here, however, this loyalty is negated by the prefix \say{dis.}
A wounded and vulnerable Ross explains that \say{Norway himself, / With terrible numbers, / Assisted by that most disloyal traitor / The thane of Cawdor, began a dismal conflict} (\textit{Macbeth}, 1.2).
Macdonwald, the \say{disloyal traitor}, is not seen to profit at all from his rebellion and disloyalty.
While he may have had fleeting increases to station or wealth under the \say{Norweyan banners} he fought under, he nonetheless quickly loses both these new advantages and his own life.
Here, Shakespeare first introduces the audience to loyalty through the consequences of its absence.
Although loyalty is rewarded later in the show, Shakespeare chooses to begin the show with the idea that disloyalty brings death.
However, Shakespeare quickly brings in this lighter side of loyalty, as King Duncan commands Ross to \say{with his [Macdonwald, ex-Thane of Cawdor's] former title greet Macbeth} (ibid).
Because \say{brave Macbeth} delivered victory to the Scots (ibid), he gains a title.

Sadly, this gain of title ultimately becomes Macbeth's downfall.
The witches hail him in turn as \say{thane of Glamis,} \say{thane of Cawdor,} and \say{king hereafter} (1.3).
While Macbeth is aware of being thane of Glamis, he finds the other propositions ridiculous.
Paul Ready's performance at the Sam Wanamaker Playhouse displays his shock clearly.
Despite being as disoriented as the audience from the shadows and music, he retorts that \say{to be king / Stands not within the prospect of belief, / No more than to be Cawdor} (ibid).
When Ross proclaims that \say{He (Duncan) bade me, from him, call thee thane of Cawdor,} Macbeth is confused, questioning why \say{Ross dress[es] [him] / In borrow'd robes} (1.3).
As he had just a few lines before noted that becoming king was as ridiculous a preposition as becoming thane of Cawdor, his appointment seems to justify the witches as truthful beings.
Banquo's remark: \say{can the devil speak true?} seems ignored by Macbeth.
In this way, Shakespeare also promotes the idea that even seemingly loyal followers may become disloyal if given the proper incentives.
Macbeth, who before now seems to have no reservations about living and dying as nothing more than thane of Glamis, now begins to set his sights higher.
He will no longer be satisfied with anything less than the title of King of Scotland, and will stop at nothing to reach this goal.

The second time we see loyalty mentioned is again in the presence of the king.
In this way, the idea that loyalty belongs to a king is again expressed through its use near Duncan.
Interestingly, this is also the last time that Duncan and Macbeth are seen on stage together.
Macbeth rebuts Duncan's assertion that \say{more is thy [Macbeth's] due than more than all can pay} with his own claim that \say{the service and the loyalty I [Macbeth] owe, / In doing it, pays itself} (1.4).
Macbeth follows this immediately with the claim that {Your highness' part / Is to receive our duties; and our duties / Are to your throne} (ibid).
In this way, we see an optimistic view of the monarchy.
The king and his servants both have a role.
The king is to support his sworn servants, giving them both labors to accomplish, but also the means to fulfill these decrees.
Once fulfilled, he is to take the fruit of the labors, a visible display of loyalty, raising their status simply through his acknowledgement of a job well done.
His servants, by comparison, are to bring glory to their family by honoring the king and helping their country.
When this covenant is fulfilled from both sides, we see the bountiful reign of Duncan, and the reign that we hope Malcolm will restore.
When the code is not followed, however, we see what happens in the reign of Macbeth.
Slowly but surely, every piece of the trust between a king and his people is eroded and destroyed, until his death goes unmourned.

Macbeth also makes the interesting claim here that loyalty and service are inherently tied.
Ready's Macbeth truly seems to believe this, as is clear in his next mention of loyalty, \say{who can be ... Loyal and neutral, in a moment?} (2.3).
However, for a character as heartless as Macbeth, the idea that service and loyalty are intrinsically tied makes an odd sort of sent.
As someone who has no morals of his own, being able to demonstrate potentially non-existent internal feelings through wholly external action would be highly desirable.
Conversely, as we see later in the show, faith without action seems unbelievable and false to Macbeth.

This scene of loyalty also plants the idea that Macbeth will betray Banquo in the minds of the audience.
Immediately after accepting Macbeth's assertions of the role of a king, Duncan turns to Banquo and praises him, saying that \say{Noble Banquo, / that hast no less deserved... let me hold thee to my heart} (ibid).
Here we see the idea planted in Macbeth's head that Banquo is a threat to the throne.
Paul Ready plays that scene well, looking at Philip Cumbus's Banquo as though he has become less of a faithful ally in his eyes.

As we see Duncan only briefly after this scene, it is not odd that we don't hear any mentions of loyalty.
But, once Duncan is dead, with Macbeth ready to claim the throne, we hear him again utter the word \say{loyal.}
When Macduff rebukes Macbeth for killing the murderers of the king, Paul Ready's Macbeth furiously replies: \say{Who can be wise, amazed, temperate and furious, / Loyal and neutral, in a moment? No man} (2.3).
The mention here is interesting, if only because it's again uttered by Macbeth.
Despite being the character who holds the least faith in the show, he's the one who literally professes loyalty the most.
The emotion Ready brings to the line makes the audience believe that Macbeth truly believes that you cannot be both \say{loyal and neutral} (ibid).
As we see in the rest of the show, that may be true.
Even though Macbeth kills his king, he still sees loyalty as a force that inherently drives someone to action.
From both of the mentions of loyalty Macbeth makes, it becomes somewhat clear that he sees that loyalty can only be demonstrated through external displays.

In the scenes that follow, especially the murder of Banquo, we see that Macbeth no longer understands the reciprocity needed for loyalty.
Just two short scenes after Macbeth notes that loyalty and neutrality are inherently opposed, we see Banquo acting fairly neutrally.
That is, while Banquo may be serving the commands that Macbeth is giving to him off-stage, he is not defending the realm from the rebellion brewing from Duncan's children in England and Ireland.

After giving Banquo leave to go ride with his child, Macbeth remarks that \say{My Genius is rebuked; as, it is said, / Mark Antony's was by Caesar} (3.1).
This comparison is interesting for a few reasons.
First, Macbeth references another of Shakespeare's tragedies, \textit{Antony and Cleopatra}.
Second, it describes himself as Mark Antony and Banquo as Caesar.
In Shakespeare's \textit{Antony and Cleopatra,} Antony is rebuked more than once by Caesar.
Caesars accusations can be contained with the idea of weakness, which is ultimately shown as true.
Caesar ultimately defeats Antony in battle.

The third reason this comparison is interesting is because of loyalty.
Caesar's loyalties never shift, while Mark Antony's loyalties shift towards Egypt.
This shift ultimately results in his fighting Rome, while Caesar begins and ends the play defending the Roman people.
For that reason, Macbeth seems to be aware that he has broken faith, while Banquo has not.
So, not only has Banquo been described as held \say{to [Duncan's] heart} while the former king still reigned (1.4), Macbeth acknowledges \say{that dauntless temper of [Banquo's] mind,} after he admits to Banquo that Macbeth will need his assistance with dealing with the rebellion, but Macbeth also has in his mind the words of the witches, that Banquo's children will inherit (3.1).
To me, Ready's Macbeth saw these both of these strands connecting and where they intersected.

If Banquo uncovered that Macbeth was a murderer, he would stand to claim the throne.
Macbeth cannot take this risk, so he orders the death of his companion.
After the death of one of the most faithful characters in the show, we never hear the word \say{loyalty} in Scotland again.

Instead, at the end of the show we hear the final mention of loyalty in England.
Macduff approaches Malcolm, hoping that he will reclaim the throne of his father.
Malcolm, however is wary of the Scotsman, and tests Macduff.
As one of the reasons Malcolm claims to be unworthy of the throne, he claims that he \say{should forge / quarrels unjust against the good and loyal, / Destroying them for wealth} (4.3).
His later recanting of these falsehoods, claiming that \say{[his] first false speaking / was this upon [himself]} (ibid).
However, before he recants, his placement of the word \say{loyal} still speaks volumes about what Shakespeare thinks of the concept.

For one, he claims that destroying people for wealth is unjust, which is a direct contrast to the reign of Macbeth, who seems to believe that force of arms justifies everything he does.
He also describes the subjects he would \say{forge quarrels} against as \say{good and loyal,} as these two concepts are intrinsically linked to each other (ibid).
Just as Macbeth cannot see a world where loyalty does not stir a heart to action, Kit Young's Malcolm seems unable to pull the concepts of goodness and loyalty from each other.
Potentially this is simply due to the horrors that he had felt.

Good and bad are opposites.
Killing his father, the king was bad and disloyal.
Therefore, in the mind of Young's Malcolm, it seems that goodness and loyalty must too be linked.
Malcolm confirms this during his recantation, saying that he \say{at no time broke [his] faith, would not betray / The devil to his fellow and delight / no less in truth than life} (ibid).
In these words, we see that Malcolm also ties his idea of loyalty to truth.
In saying that he both refuses to betray his people and that he loves truth as much as life, he posits that loyalty and truth are intrinsically linked.
Again, this could stem from the life he had lived.
Had he not had the nobility turned against him, believing that he had killed his father, he may never have made the connection between loyalty, truth and openness.
In direct contrast to Macbeth, it seems that Malcolm takes the view of loyalty as something that is tied to intrinsic factors, rather than extrinsic actions.
Of course, as the show's ending shows, like his father, he still rewards actions.
He names the men who served him \say{earls, the first that ever Scotland / In such an honour named} for their work in defeating Macbeth (5.8).

Through his four mentions of loyalty, Shakespeare grounds the show \textit{Macbeth} in four of its most pivotal scenes.
We see the scene that explains and justifies the entire plot, the scene where Macbeth becomes king, the scene where Macbeth is shown as irredeemable, and the scene where Malcolm accepts his destiny with the word \say{loyal.}
In such a way, Shakespeare draws the close listener in to further appreciate the message he is sending.
But, more than that, he uses loyalty in conjunction with other ideas to explain what it means to the different characters.

We see that to Ross, disloyalty comes from rebellion.
Therefore, it seems that he may take the view that loyalty is the default state, and that without acting disloyally, people inherently act loyally.
To Macbeth, who can't seem to comprehend the idea, loyalty seems to be wholly based on external actions.
In that way, it seems almost opposite Ross's view, in that disloyalty takes no effort, while loyalty is difficult.
Finally, we see the new king's view, which is that loyalty is good and honest, and goodness and honesty are needed for loyalty.
In his eyes, we see what may be the most honest view of loyalty, which is neither fully passive nor fully active.
Instead, like all ideals in life, it is difficult and multifaceted.
\section{Draft 4: 11 December}
Shakespeare, as every young student learns, invented many words.
And, as a poet, he is more deliberate than most in his repetitions of words.
One lens that is useful for viewing his Scottish tragedy \textit{Macbeth} is through the lens of loyalty.
When viewed through that lens, the show seems to be a cautionary tale about the nature of disloyalty.
While disloyalty may have temporary gains, it will ultimately damn you and be the cause of your undoing.
More than that, however, Shakespeare uses words in conjunction with others.
So, we can see both how Shakespeare sees loyalty, and how his characters see it by looking at where in the show the four utterances of \say{loyal} occur.

The first utterance of loyalty occurs during the exposition of the show, when the audience learns about the thane of Cawdor's rebellion against the King of Scotland.
Here, however, this loyalty is negated by the prefix \say{dis.}
A wounded and vulnerable Ross explains that \say{Norway himself, / With terrible numbers, / Assisted by that most disloyal traitor / The thane of Cawdor, began a dismal conflict} (\textit{Macbeth}, 1.2).
Macdonwald, the \say{disloyal traitor}, is not seen to profit at all from his rebellion and disloyalty.
However, he may have had fleeting increases to station or wealth under the \say{Norweyan banners} he fought under, but nonetheless, he quickly loses both these new advantages and his own life.
Here, Shakespeare first introduces the audience to loyalty with the consequences of its absence.
Unlike later in the show, where we see the idea of loyalty as rewarded, he chooses to begin the show with the idea that disloyalty brings death.
However, Shakespeare quickly brings in the lighter side of this coin as King Duncan commands Ross to \say{with his [Macdonwald, ex-Thane of Cawdor's] former title greet Macbeth} (ibid).
Because \say{brave Macbeth} delivered victory to the Scots (ibid), he gains a title.

Sadly, this gain of title ultimately becomes Macbeth's downfall.
The witches hail him in turn as \say{thane of Glamis,} \say{thane of Cawdor,} and \say{king hereafter} (1.3).
While Macbeth is aware of being thane of Glamis, he finds the other propositions ridiculous.
Paul Ready's performance at the Sam Wanamaker Playhouse displays this clearly.
Despite being as disoriented as the audience from the shadows and music, he retorts that \say{to be king / Stands not within the prospect of belief, / No more than to be Cawdor} (ibid).
When Ross proclaims that \say{He (Duncan) bade me, from him, call thee thane of Cawdor,} Macbeth is confused, questioning why \say{Ross dress[es] [him] / In borrow'd robes} (1.3).
As he had just a few lines before noted that becoming king was as ridiculous a preposition as becoming thane of Cawdor, his appointment seems to justify the witches as truthful beings.
Banquo's remark: \say{can the devil speak true?} seems ignored by Macbeth.
In this way, Shakespeare also promotes the idea that even seemingly loyal followers may become disloyal if given the proper incentives.
Macbeth, who before now seems to have no reservations about living and dying as nothing more than thane of Glamis, now begins to set his sights higher.
He will no longer be satisfied with anything less than the title of King of Scotland, and will stop at nothing to reach this goal.

The second time we see loyalty mentioned is again in the presence of the king.
In this way, the idea that loyalty belongs to a king is again expressed through its use near Duncan.
Interestingly, this is also the last time that Duncan and Macbeth are seen on stage together.
Duncan fears that he has failed as a liege lord and states that \say{more is thy due than more than all can pay,} as he feels that he has failed to give Macbeth what is owed to him as a faithful servant who defended the kingdom (1.4).
Macbeth rebuts this assertion with his own claim that \say{the service and the loyalty I [Macbeth] owe, / In doing it, pays itself} (ibid).
Macbeth follows this immediately with the claim that {Your highness' part / Is to receive our duties; and our duties / Are to your throne} (ibid).
In this way, we see an optimistic view of the monarchy.
The king and his servants both have a role.
The king is to support his sworn servants, giving them both labors to accomplish, but also the means to fulfill these decrees.
Once fulfilled, he is to take the fruit of the labors, raising their status simply through his acknowledgement of a job well done.
His servants, by comparison, are to bring glory to their family by honoring the king and helping their country.
When this covenant is fulfilled from both sides, we see the bountiful reign of Duncan, and the reign that we hope Malcolm will restore.
When the code is not followed, however, we see what happens in the reign of Macbeth.
Slowly but surely, every piece of the trust between a king and his people is eroded and destroyed, until his death goes unmourned.

Macbeth also makes the interesting claim here that loyalty and service are inherently tied.
Ready's Macbeth truly seems to believe this, as is clear in the next mention of loyalty.
However, for a character as heartless as Macbeth, the idea that service and loyalty are intrinsically tied makes an odd sort of sent.
As someone who has no morals of his own, being able to demonstrate potentially non-existent internal feelings through wholly external action would be highly desirable.
Conversely, as we see later in the show, faith without action seems unbelievable and false to Macbeth.

This scene of loyalty also plants the idea that Macbeth will betray Banquo in the minds of the audience.
Immediately after accepting Macbeth's assertions of the role of a king, Duncan turns to Banquo and praises him, saying that \say{Noble Banquo, / that hast no less deserved... let me hold thee to my heart} (ibid).
Here we see the idea planted in Macbeth's head that Banquo is a threat to the throne.
Paul Ready plays that scene well, looking at Philip Cumbus's Banquo as though he has become less of a faithful ally in his eyes.

As we see Duncan only briefly after this scene, it is not odd that we don't hear any mentions of loyalty.
But, once Duncan is dead, with Macbeth ready to claim the throne, we hear him again utter the word \say{loyal.}
The mention is interesting, if only because it's again uttered by Macbeth.
Despite being the character who holds the least faith in the show, he's the one who literally professes loyalty the most.
When Macduff rebukes Macbeth for killing the murderers of the king, Paul Ready's Macbeth furiously replies: \say{Who can be wise, amazed, temperate and furious, / Loyal and neutral, in a moment? No man} (2.3).
The emotion Ready brings to the line makes the audience believe that Macbeth truly believes that you cannot be both \say{loyal and neutral} (ibid).
As we see in the rest of the show, that may be true.
Even though Macbeth kills his king, he still sees loyalty as a force that inherently drives someone to action.
From both of the mentions of loyalty Macbeth makes, it becomes somewhat clear that he sees that loyalty can only be demonstrated through external displays.

And, in the scenes that follow, we see that Macbeth does not see loyalty as a two way path anymore, seen taken to the extreme in his killing of Banquo.
Just two short scenes after Macbeth notes that loyalty and neutrality are inherently opposed, we see Banquo, whose actions do not appear to be directly serving Macbeth.
That is, while Banquo may be serving the commands that Macbeth is giving to him off-stage, he is not defending the realm from the rebellion brewing from Duncan's children in England and Ireland.

After giving Banquo leave to go ride with his child, Macbeth remarks {Remove?} act that \say{My Genius is rebuked; as, it is said, / Mark Antony's was by Caesar} (3.1).
This comparison is interesting for a few reasons.
First, Macbeth references another of Shakespeare's tragedies, \textit{Antony and Cleopatra}.
Second, it describes himself as Mark Antony and Banquo as Caesar.

In Shakespeare's \textit{Antony and Cleopatra,} Antony is rebuked more than once by Caesar.
Caesars accusations can be contained with the idea of weakness, which is ultimately shown as true.
Caesar ultimately defeats Antony in battle.

The third reason this comparison is interesting is because of loyalty.
Caesar's loyalties never shift, while Mark Antony's loyalties shift towards Egypt.
This shift ultimately results in his fighting Rome, while Caesar begins and ends the play defending the Roman people.
For that reason, Macbeth seems to be aware that he has broken faith, while Banquo has not.
So, not only has Banquo been described as held \say{to [Duncan's] heart} while the former king still reigned (1.4), Macbeth acknowledges \say{that dauntless temper of [Banquo's] mind,} after he admits to Banquo that Macbeth will need his assistance with dealing with the rebellion, but Macbeth also has in his mind the words of the witches, that Banquo's children will inherit (3.1).
To me, Ready's Macbeth saw these different strands connecting as he saw where they intersected. \textbf{two saws?}

If Banquo uncovered that Macbeth was a murderer, he would stand to claim the throne.
Macbeth cannot take this risk, so he orders the death of his companion.
After the death of one of the most faithful characters in the show, we never hear the word in Scotland again.

Instead, at the end of the show we hear the final mention of loyalty in England.
Macduff approaches Malcolm, hoping that he will reclaim the throne of his father.
Malcolm, however is wary of the Scotsman, and tests Macduff.
As one of the reasons Malcolm claims to be unworthy of the throne, he claims that he \say{should forge / quarrels unjust against the good and loyal, / Destroying them for wealth} (4.3).
His later recanting of these falsehoods, claiming that \say{[his] first false speaking / was this upon [himself]} (ibid).
However, before he recants, his placement of the word \say{loyal} still speaks volumes about what Shakespeare thinks of the concept.

For one, he claims that destroying people for wealth is unjust, which is a direct contrast to the reign of Macbeth, who seems to believe that force of arms justifies everything he does.
He also describes the subjects he would \say{forge quarrels} against as \say{good and loyal,} as these two concepts are intrinsically linked to each other (ibid).
Just as Macbeth cannot see a world where loyalty does not stir a heart to action, Kit Young's Malcolm seems unable to pull the concepts of goodness and loyalty from each other.
Potentially this is simply due to the horrors that he had felt.

Good and bad are opposites.
Killing his father, the king was bad and disloyal.
Therefore, in the mind of Young's Malcolm, it seems that goodness and loyalty must too be linked.
Malcolm confirms this during his recantation, saying that he \say{at no time broke [his] faith, would not betray / The devil to his fellow and delight / no less in truth than life} (ibid).
In these words, we see that Malcolm also ties his idea of loyalty to truth.
In saying that he both refuses to betray his people and that he loves truth as much as life, he posits that loyalty and truth are intrinsically linked.
Again, this could stem from the life he had lived.
Had he not had the nobility turned against him, believing that he had killed his father, he may never have made the connection between loyalty, truth and openness.
In direct contrast to Macbeth, it seems that Malcolm takes the view of loyalty as something that is tied to intrinsic factors, rather than extrinsic actions.
Of course, as the show's ending shows, like his father, he still rewards actions.
He names the men who served him \say{earls, the first that ever Scotland / In such an honour named} for their work in defeating Macbeth (5.8).

Through his four mentions of loyalty, Shakespeare grounds the show \textit{Macbeth} in four of its most pivotal scenes.
We see the scene that explains and justifies the entire plot, the scene where Macbeth becomes king, the scene where Macbeth is shown as irredeemable, and the scene where Malcolm accepts his destiny with the word \say{loyal.}
In such a way, Shakespeare draws the close listener in to further appreciate the message he is sending.
But, more than that, he uses loyalty in conjunction with other ideas to explain what it means to the different characters.

We see that to Ross, disloyalty comes from rebellion.
Therefore, it seems that he may take the view that loyalty is the default state, and that without acting disloyally, people inherently act loyally.
To Macbeth, who can't seem to comprehend the idea, loyalty seems to be wholly based on external actions.
In that way, it seems almost opposite Ross's view, in that disloyalty takes no effort, while loyalty is difficult.
Finally, we see the new king's view, which is that loyalty is good and honest, and goodness and honesty are needed for loyalty.
In his eyes, we see what may be the most honest view of loyalty, which is neither fully passive nor fully active.
Instead, like all ideals in life, it is difficult and multifaceted.
\section{Draft 3: 11 December}
Shakespeare, as every young student learns, invented many words.
So, when he reuses words, there is certain to be a reason for that use.
One lens that is useful for viewing his Scottish tragedy \textit{Macbeth} is through the lens of loyalty.
When viewed through that lens, the show seems to be a cautionary tale about the nature of disloyalty.
While disloyalty may have temporary gains, it will ultimately damn you and be the cause of your undoing.
More than that, however, Shakespeare uses words in conjunction with others.
So, we can see both how Shakespeare sees loyalty, and how his characters see it by looking at where in the show the four utterances of \say{loyal} occur.

The first utterance of loyalty occurs during the exposition of the show, when the audience learns about the thane of Cawdor's rebellion against the King of Scotland.
Here, however, this loyalty is negated by the prefix \say{dis.}
A wounded and vulnerable Ross explains that \say{Norway himself, / With terrible numbers, / Assisted by that most disloyal traitor / The thane of Cawdor, began a dismal conflict} (\textit{Macbeth}, 1.2).
Macdonwald, the \say{disloyal traitor}, is not seen to profit at all from his rebellion and disloyalty.
However, he may have had fleeting increases to station or wealth under the \say{Norweyan banners} he fought under, but nonetheless, he quickly loses both these new advantages and his own life.
Here, Shakespeare first introduces the audience to loyalty with the consequences of its absence.
Unlike later in the show, where we see the idea of loyalty as rewarded, he chooses to begin the show with the idea that disloyalty brings death.
However, Shakespeare quickly brings in the lighter side of this coin as King Duncan commands Ross to \say{with his [Macdonwald, ex-Thane of Cawdor's] former title greet Macbeth} (ibid).
Because \say{brave Macbeth} delivered victory to the Scots (ibid), he gains a title.

Sadly, this gain of title ultimately becomes Macbeth's downfall.
The witches hail him in turn as \say{thane of Glamis,} \say{thane of Cawdor,} and \say{king hereafter} (1.3).
While Macbeth is aware of being thane of Glamis, he finds the other propositions ridiculous.
Paul Ready's performance at the Sam Wanamaker Playhouse display this clearly.
Despite being as disoriented as the audience from the shadows and music, he retorts that \say{to be king / Stands not within the prospect of belief, / No more than to be Cawdor} (ibid).
When Ross proclaims that \say{He (Duncan) bade me, from him, call thee thane of Cawdor,} Macbeth is confused, questioning why \say{Ross dress[es] [him] / In borrow'd robes} (1.3).
As he had just a few lines before noted that becoming king was as ridiculous a preposition as becoming thane of Cawdor, his appointment seems to justify the witches as truthful beings.
Banquo's remark: \say{can the devil speak true?} seems ignored by Macbeth.
In this way, Shakespeare also promotes the idea that even seemingly loyal followers may become disloyal if given the proper incentives.
Macbeth, who before now seems to have no reservations about living and dying as nothing more than thane of Glamis, now begins to set his sights higher.
He will no longer be satisfied with anything less than the title of King of Scotland, and will stop at nothing to reach this goal.

The second time we see loyalty mentioned is again in the presence of the king.
In this way, the idea that loyalty belongs to a king is again expressed through its use near Duncan.
Interestingly, this is also the last time that Duncan and Macbeth are seen on stage together.
As a conversation between a monarch and his successor, there are few that could be better.
Duncan fears that he has failed as a liege lord and states that \say{more is thy due than more than all can pay,} as he feels that he has failed to give Macbeth what is owed to him as a faithful servant who defended the kingdom (1.4).
Macbeth rebuts this assertion with his own claim that \say{the service and the loyalty I [Macbeth] owe, / In doing it, pays itself} (ibid).
Macbeth follows this immediately with the claim that {Your highness' part / Is to receive our duties; and our duties / Are to your throne} (ibid).
In this way, we see an optimistic view of the monarchy, as well as a good send off between a once and a future king.
The king and his servants both have a role.
The king is to support his sworn servants, giving them both labors to accomplish, but also the means to fulfill these decrees.
Once fulfilled, he is to take the fruit of the labors, raising their status simply through his acknowledgement of a job well done.
His servants, by comparison, are to bring glory to their family by honoring the king and helping their country.
When this covenant is fulfilled from both sides, we see the bountiful reign of Duncan, and the reign that we hope Malcolm will restore.
When the code is not followed, however, we see what happens in the reign of Macbeth.
Slowly but surely, every piece of the trust between a king and his people is eroded and destroyed, until his death goes unmourned.

Had the king died of another cause, these lines could have been seen as a call to the life as king that Macbeth should live.
But, as he did not, they almost seem a list of what Macbeth avoids doing while in command.

Macbeth also makes the interesting claim here that loyalty and service are inherently tied.
Ready's Macbeth truly seems to believe this, as is clear in the next mention of loyalty.
However, for a character as heartless as Macbeth, the idea that service and loyalty are intrinsically tied makes an odd sort of sent.
As someone who has no morals of his own, being able to demonstrate potentially non-existent internal feelings through wholly external action would be highly desirable.
Conversely, as we see later in the show, faith without action seems unbelievable and false to Macbeth.

This scene of loyalty also plants the idea that Macbeth will betray Banquo in the minds of the audience.
Immediately after accepting Macbeth's assertions of the role of a king, Duncan turns to Banquo and praises him, saying that \say{Noble Banquo, / that hast no less deserved... let me hold thee to my heart} (ibid).
Here we see the idea planted in Macbeth's head that Banquo is a threat to the throne.
Paul Ready plays that scene well, looking at Philip Cumbus's Banquo as though he has become less of a faithful ally in his eyes.

As we see Duncan only briefly after this scene, it is not odd that we don't hear any mentions of loyalty.
But, once Duncan is dead, with Macbeth ready to claim the throne, we hear him again utter the word \say{loyal.}
The mention is interesting, if only because it's again uttered by Macbeth.
Despite being the character who holds the least faith in the show, he's the one who literally professes loyalty the most.
When Macduff rebukes Macbeth for killing the murderers of the king, Paul Ready's Macbeth furiously replies: \say{Who can be wise, amazed, temperate and furious, / Loyal and neutral, in a moment? No man} (2.3).
The emotion Ready brings to the line makes the audience believe that Macbeth truly believes that you cannot be both \say{loyal and neutral} (ibid).
As we see in the rest of the show, that may be true.
Even though Macbeth kills his king, he still sees loyalty as a force that inherently drives someone to action.
From both of the mentions of loyalty Macbeth makes, it becomes somewhat clear that he sees that loyalty can only be demonstrated through external displays.

And, in the scenes that follow, we see that Macbeth does not see loyalty as a two way path anymore, seen taken to the extreme in his killing of Banquo.
Just two short scenes after Macbeth notes that loyalty and neutrality are inherently opposed, we see Banquo, whose actions do not appear to be directly serving Macbeth.
That is, while Banquo may be serving the commands that Macbeth is giving to him off-stage, he is not defending the realm from the rebellion brewing from Duncan's children in England and Ireland.

After giving Banquo leave to go ride with his child, Macbeth remarks act that \say{My Genius is rebuked; as, it is said, / Mark Antony's was by Caesar} (3.1).
This comparison is interesting for a few reasons.
First, Macbeth references another of Shakespeare's tragedies, \textit{Antony and Cleopatra}.
Second, it describes himself as Mark Antony and Banquo as Caesar.

In Shakespeare's \textit{Antony and Cleopatra,} Antony is rebuked more than once by Caesar.
Caesars accusations can be contained with the idea of weakness, which is ultimately shown as true.
Caesar ultimately defeats Antony in battle.

The third reason this comparison is interesting is because of loyalty.
Caesar's loyalties never shift, while Mark Antony's loyalties shift towards Egypt.
This shift ultimately results in his fighting Rome, while Caesar begins and ends the play defending the Roman people.
For that reason, Macbeth seems to be aware that he has broken faith, while Banquo has not.
So, not only has Banquo been described as held \say{to [Duncan's] heart} while the former king still reigned (1.4), Macbeth acknowledges \say{that dauntless temper of [Banquo's] mind,} after he admits to Banquo that Macbeth will need his assistance with dealing with the rebellion, but Macbeth also has in his mind the words of the witches, that Banquo's children will inherit (3.1).
To me, Ready's Macbeth saw these different strands connecting as he saw where they intersected.

If Banquo uncovered that Macbeth was a murderer, he would stand to claim the throne.
Macbeth cannot take this risk, so he orders the death of his companion.
After the death of one of the most faithful characters in the show, we never hear the word in Scotland again.

Instead, at the end of the show we hear the final mention of loyalty in England.
Macduff approaches Malcolm, hoping that he will reclaim the throne of his father.
Malcolm, however is wary of the Scotsman, and tests Macduff.
As one of the reasons Malcolm claims to be unworthy of the throne, he claims that he \say{should forge / quarrels unjust against the good and loyal, / Destroying them for wealth} (4.3).
His later recanting of these falsehoods, claiming that \say{[his] first false speaking / was this upon [himself]} (ibid).
However, before he recants, his placement of the word \say{loyal} still speaks volumes about what Shakespeare thinks of the concept.

For one, he claims that destroying people for wealth is unjust, which is a direct contrast to the reign of Macbeth, who seems to believe that force of arms justifies everything he does.
He also describes the subjects he would \say{forge quarrels} against as \say{good and loyal,} as these two concepts are intrinsically linked to each other (ibid).
Just as Macbeth cannot see a world where loyalty does not stir a heart to action, Kit Young's Malcolm seems unable to pull the concepts of goodness and loyalty from each other.
Potentially this is simply due to the horrors that he had felt.

Good and bad are opposites.
Killing his father, the king was bad and disloyal.
Therefore, in the mind of Young's Malcolm, it seems that goodness and loyalty must too be linked.
Malcolm confirms this during his recantation, saying that he \say{at no time broke [his] faith, would not betray / The devil to his fellow and delight / no less in truth than life} (ibid).
In these words, we see that Malcolm also ties his idea of loyalty to truth.
In saying that he both refuses to betray his people and that he loves truth as much as life, he posits that loyalty and truth are intrinsically linked.
Again, this could stem from the life he had lived.
Had he not had the nobility turned against him, believing that he had killed his father, he may never have made the connection between loyalty, truth and openness.
In direct contrast to Macbeth, it seems that Malcolm takes the view of loyalty as something that is tied to intrinsic factors, rather than extrinsic actions.
Of course, as the show's ending shows, like his father, he still rewards actions.
He names the men who served him \say{earls, the first that ever Scotland / In such an honour named} for their work in defeating Macbeth (5.8).

Through his four mentions of loyalty, Shakespeare grounds the show \textit{Macbeth} in four of its most pivotal scenes.
We see the scene that explains and justifies the entire plot, the scene where Macbeth becomes king, the scene where Macbeth is shown as unredeemable, and the scene where Malcolm accepts his destiny with the word \say{loyal.}
In such a way, Shakespeare draws the close listener in to further appreciate the message he is sending.
But, more than that, he uses loyalty in conjunction with other ideas to explain what it means to the different characters.

We see that to Ross, disloyalty comes from rebellion.
Therefore, it seems that he may take the view that loyalty is the default state, and that without acting disloyally, people inherently act loyally.
To Macbeth, who can't seem to comprehend the idea, loyalty seems to be wholly based on external actions.
In that way, it seems almost opposite Ross's view, in that disloyalty takes no effort, while loyalty is difficult.
Finally, we see the new king's view, which is that loyalty is good and honest, and goodness and honesty are needed for loyalty.
In his eyes, we see what may be the most honest view of loyalty, which is neither fully passive nor fully active.
Instead, like all ideals in life, it is difficult and multifaceted.

(2282)
\section{Draft 2.4: 11 December}
Loyal 4:
Near the end of the show, we see the final mention of loyalty.
Macduff approaches Malcolm, hoping that he will reclaim the throne of his father.
Malcolm, however is wary of the Scotsman, and tests Macduff.
As one of the reasons Malcolm claims to be unworthy of the throne, he claims that he \say{should forge / quarrels unjust against the good and loyal, / Destroying them for wealth} (4.3).
His later recanting of these falsehoods, claiming that \say{[his] first false speaking / was this upon [himself]} (ibid).
However, before he recants, his placement of the word \say{loyal} still speaks volumes about what Shakespeare thinks of the concept.

For one, he claims that destroying people for wealth is unjust, which is a direct contrast to the reign of Macbeth, who seems to believe that force of arms justifies everything he does.
He also describes the subjects he would \say{forge quarrels} against as \say{good and loyal,} as these two concepts are intrinsically linked to each other (ibid).
Just as Macbeth cannot see a world where loyalty does not stir a heart to action, Kit Young's Malcolm seems unable to pull the concepts of goodness and loyalty from each other.
Potentially this is simply due to the horrors that he had felt.

Good and bad are opposites.
Killing his father, the king was bad and disloyal.
Therefore, in the mind of Young's Malcolm, it seems that goodness and loyalty must too be linked.
Malcolm confirms this during his recantation, saying that he \say{at no time broke [his] faith, would not betray / The devil to his fellow and delight / no less in truth than life} (ibid).
In these words, we see that Malcolm also ties his idea of loyalty to truth.
In saying that he both refuses to betray his people and that he loves truth as much as life, he posits that loyalty and truth are intrinsically linked.
Again, this could stem from the life he had lived.
Had he not had the nobility turned against him, believing that he had killed his father, he may never have made the connection between loyalty, truth and openness.
In direct contrast to Macbeth, it seems that Malcolm takes the view of loyalty as something that is tied to intrinsic factors, rather than extrinsic actions.
Of course, as the show's ending shows, like his father, he still rewards actions.
He names the men who served him \say{earls, the first that ever Scotland / In such an honour named} for their work in defeating Macbeth (5.8).
(415) (total 1839)
\section{Draft 2.3: 11 December}
Loyal 3:
The third mention of loyalty, oddly enough, comes soon after Macbeth brutally murders Duncan, along with members of Duncan's staff to remove suspicion.
The mention is interesting, if only because it's again uttered by Macbeth.
Despite being the character who holds the least faith in the show, he's the one who literally professes loyalty.
When Macduff rebukes Macbeth for killing the murderers of the king, Paul Ready's Macbeth furiously replies: \say{Who can be wise, amazed, temperate and furious, / Loyal and neutral, in a moment? No man} (2.3).
The emotion Ready brings to the line makes the audience believe that Macbeth truly believes that you cannot be both \say{loyal and neutral} (ibid).
As we see in the rest of the show, that may be true.
Even though Macbeth kills his king, he still sees loyalty as a force that inherently drives someone to action.

In the scenes that follow, we see that Macbeth does not see loyalty as a two way path anymore, which can be seen taken to the extreme in his killing of Banquo.
Just two short scenes after Macbeth notes that loyalty and neutrality are inherently opposed, we see Banquo, whose actions do not appear to be directly serving Macbeth.
That is, while Banquo may be serving the commands that Macbeth is giving to him off-stage, he is not defending the realm from the rebellion brewing from Duncan's children in England and Ireland.

After giving Banquo leave to go ride with his child, Macbeth remarks act that \say{My Genius is rebuked; as, it is said, / Mark Antony's was by Caesar} (3.1).
This comparison is interesting for a few reasons.
First, Macbeth references another of Shakespeare's tragedies, \textit{Antony and Cleopatra}.
Second, it describes himself as Mark Antony and Banquo as Caesar.

In Shakespeare's \textit{Antony and Cleopatra,} Antony is rebuked more than once by Caesar.
Caesars accusations can be contained with the idea of weakness, which is ultimately shown as true.
Caesar ultimately defeats Antony in battle.

The third reason this comparison is interesting is because of loyalty.
Caesar's loyalties never shift, while Mark Antony's loyalties shift towards Egypt.
This shift ultimately results in his fighting Rome, while Caesar begins and ends the play defending the Roman people.
For that reason, Macbeth seems to be aware that he has broken faith, while Banquo has not.
So, not only has Banquo been described as held \say{to [Duncan's] heart} while the former king still reigned (1.4), Macbeth acknowledges \say{that dauntless temper of [Banquo's] mind,} after he admits to Banquo that Macbeth will need his assistance with dealing with the rebellion, but Macbeth also has in his mind the words of the witches, that Banquo's children will inherit (3.1).
To me, Ready's Macbeth saw these different strands connecting as he saw where they intersected.

If Banquo uncovered that Macbeth was a murderer, he would stand to claim the throne.
Macbeth cannot take this risk, so he orders the death of his companion.
(489) (total:1424)
\section{Draft 2.2: 11 December}
Loyal 2:
The second time we see loyalty mentioned is again in the presence of the king.
In this way, the idea that loyalty belongs to a king is again expressed through its mention near the king.
Interestingly, this is also the last time that Duncan and Macbeth are seen on stage together.
As a conversation between a monarch and his successor, there are few that could be better.
Duncan fears that he has failed as a liege lord and states that \say{more is thy due than more than all can pay,} as he feels that he has failed to give Macbeth what is owed to him as a faithful servant who defended the kingdom (1.4).
Macbeth rebuts this assertion with his own claim that \say{the service and the loyalty I [Macbeth] owe,/In doing it, pays itself} (ibid).
Macbeth follows this immediately with the claim that {Your highness' part / Is to receive our duties; and our duties / Are to your throne} (ibid).
In this way, we see an optimistic view of the monarchy.
The king and his servants both have a role.
The king is to support his sworn servants, giving them both labors to accomplish, but also the means to fulfill these decrees.
Once fulfilled, he is to take the fruit of the labors, raising their status simply through his acknowledgement of a job well done.
His servants, by comparison, are to bring glory to their family by honoring the king and helping their country.
When this covenant is fulfilled from both sides, we see the bountiful reign of Duncan, and the reign that we hope Malcolm will restore.
When the code is not followed, however, we see what happens in the reign of Macbeth.
Slowly but surely, every piece of the trust between a king and his people is eroded and destroyed, until his death goes unmourned.

Macbeth also makes the interesting claim here that loyalty and service are inherently tied.
Ready's Macbeth truly seems to believe this, as is clear in the next mention of loyalty.
However, for a character as heartless as Macbeth, the idea that service and loyalty are intrinsically tied almost makes a macabre sense.
As someone who has no morals of his own, being able to demonstrate internal feelings through action would be highly desirable.
Conversely, as we see later in the show, faith without action seems unbelievable and false to Macbeth.

This scene of loyalty also plants the idea that Macbeth will betray Banquo in the minds of the audience.
Immediately after accepting Macbeth's assertions of the role of a king, Duncan turns to Banquo and praises him, saying that \say{Noble Banquo, / that hast no less deserved... let me hold thee to my heart} (ibid).
Here we see the idea planted in Macbeth's head that Banquo is a threat to the throne.
Paul Ready plays that scene well, looking at Philip Cumbus's Banquo as though he has become less of a faithful ally in his eyes.
(493) (total = 935)
\section{Draft 2.1: 11 December}
Loyal 1:
The backstory to the show, which drives the entirety of the plot, is the Thane of Cawdor's rebellion against the King of Scotland.
The mention of this betrayal in Act 1 Scene 2, is also where we first see a character described with the word \say{loyal.}
Here, however, this loyalty is negated by the prefix \say{dis.}
Ross explains that \say{Norway himself, / With terrible numbers, / Assisted by that most disloyal traitor / The thane of Cawdor, began a dismal conflict} (\textit{Macbeth}, 1.2).
Macdonwald, the \say{disloyal traitor}, is not seen to profit at all from his rebellion and disloyalty.
While he may have had fleeting increases to station or wealth under the \say{Norweyan banners} he fought under, he quickly loses both these new advantages and his own life.
Here, Shakespeare begins with the dark side to the moral of loyalty.
Unlike later in the show, where we see the idea of loyalty as rewarded, he chooses to begin the show with the idea that disloyalty brings consequences.
However, Shakespeare quickly brings in the lighter side of this coin as King Duncan commands Ross to \say{with his [Macdonwald, ex-Thane of Cawdor's ] former title greet Macbeth} (ibid).
Because \say{brave Macbeth} delivered victory to the Scots (ibid), he gains a title.

But, this gain of title ultimately becomes Macbeth's downfall.
The witches hail him in turn as \say{thane of Glamis,} \say{thane of Cawdor,} and \say{king hereafter} (1.3).
While Macbeth is aware of being thane of Glamis, he finds the other propositions ridiculous.
Paul Ready's performance at the Sam Wanamaker Playhouse display this clearly.
Despite being as disoriented as the audience from the shadows and music, he retorts that \say{to be king / Stands not within the prospect of belief, / No more than to be Cawdor} (ibid).
When Ross proclaims that \say{He (Duncan) bade me, from him, call thee thane of Cawdor,} Macbeth is confused, questioning why \say{Ross dress[es] [him] / In borrow'd robes?} (1.3).
As he had just a few lines before noted that becoming king was as ridiculous a preposition as becoming thane of Cawdor, his appointment seems to justify the witches as truthful beings.
Banquo's remark: \say{can the devil speak true?} seems ignored by Macbeth, though we later see that this is not true.
In this way, Shakespeare also promotes the idea that even seemingly loyal followers may become disloyal if given the proper incentives.
Macbeth, who before now seems to have no reservations about living and dying as nothing more than thane of Glamis, now seems to have set his sights higher.
He is now no longer satisfied with anything less than the title of King of Scotland, and will stop at nothing to reach this goal.
(442)
\section{Notes: 11 December}
I need approx 500 words per mention of loyal. So let's just try that
\section{Draft 2: 11 December}
At the core of Shakespeare's famous Scottish tragedy \textit{Macbeth} is the struggle between loyalty and a desire for power.
Shakespeare carefully weaves a narrative that shows how being disloyal may be temporarily effective, but inherently leads to its own failure.
Despite only having the word \say{loyal} uttered four times in the show, two of which are modified versions of the word, the entire plot centers around these utterances.

The backstory to the show, which drives the entirety of the plot, is the Thane of Cawdor's rebellion against the King of Scotland. 
Act 1 Scene 2 has the first utterance of the word \say{loyal.}
Here, however, it is negated by the prexis \say{dis,} as Ross explains how \say{Norway himself, / With terrible numbers, / Assisted by that most disloyal traitor / The thane of Cawdor, began a dismal conflict} (\textit{Macbeth}, 1.2).
Macdonwald may have temporarily had gains under the rule of the \say{Norweyan banners} he fought under, but quickly Macbeth ends his life.
Here, we also see the other idea that Shakespeare introduces in the show, the idea that loyalty itself is something that brings reward.
Since Macbeth had been loyal to the king in defending the land, Ross is sent to \say{with his [Macdonwald, ex-Thane of Cawdor] former title greet Macbeth} (ibid).

Macbeth becoming Thane of Cawdor is the motivation that leads him to trust the witches, as Banquo remarks, \say{What, can the devil speak true?} after the witches' utterance of \say{hail to thee, thane of Cawdor!} was initially remarked by Macbeth are \say{borrowed robes} (1.3).
In this way, Shakespeare has the plot begin with Macbeth's loyalty being rewarded.
He reverses the idea in the final act, where Macbeth's disloyalty ultimately results in his death. 

The next time that we see Duncan, he confirms this idea of loyalty deserving reward.
Duncan states that he has failed as a liege lord, for \say{more is thy due than more than all can pay,} as he feels that he has failed to give Macbeth what is owed to him as a faithful servant who defended the kingdom.
Macbeth, however, gives the view of fealty that is common to traditional heroes, namely that \say{The service and the loyalty I [Macbeth] owe,/In doing it, pays itself} (1.4).
He follows this immediately with the idea that {Your highness' part / Is to receive our duties; and our duties / Are to your throne} (ibid).
That is, the king's role is to support his sworn servants, both by giving them work to do and in receiving the fruits of their labour.
They in turn are to support the king, both by following his orders, and in helping Scotland.
Shakespeare shows us how this can work mutually beneficially, as in the reign of Duncan and the assumed reign of Malcolm.
However, he also shows us how this can be destructive, when one party doesn't follow this code, as in the murderous reign of Macbeth.

Oddly, despite the \say{service and loyalty [he] owe[s]} to Malcolm (ibid), Macbeth very soon after that scene kills the king.
To hide his murder, he also kills members of Duncan's party to blame them.
Despite Macbeth being the most disloyal of characters, he utters two of the four mentions of the word.
Here, he remarks that say{Who can be wise, amazed, temperate and furious, / Loyal and neutral, in a moment? No man} (2.3).
This line can be performed in thousands of different ways.
An actor could perform the line as if Macbeth is afraid and trying to cover his tracks.
Alternatively, they could perform the line as though affronted at the accusations that are (rightly) leveled at him.
Paul Ready at the Sam Wanamaker Playhouse speaks this line incredulously, as if even Macbeth, a murderer, cannot understand how a person could see themselves as loyal and not act.
Even though he killed the king, he still sees loyalty as a force that inherently drives someone to action.

In fact, this idea of loyalty driving a soul to action is likely why he ends up killing Banquo.
Macbeth remarks in the third act that \say{My Genius is rebuked; as, it is said, / Mark Antony's was by Caesar} (3.1).
This comparison is interesting for a few reasons.
First, it references another of Shakespeare's tragedies.
Second, it describes himself as Mark Antony and Banquo as Caesar.
In Shakespeare's \textit{Antony and Cleopatra,} Antony, despite being the protagonist, is rightly rebuked by Caesar.
Caesar accuses him of becoming weak, and ultimately defeats Antony in battle.
Finally, Caesar's loyalties never shift.
While Mark Antony defects to Egypt, ultimately fighting Rome, Caesar begins and ends the play defending the Roman people.
For that reason, Macbeth, in addition to anger that Banquo's offspring would inherit the throne, is also afraid that Banquo may uncover his deceit.
Macbeth cannot risk being seen as a performer as regicide, so Banquo must die before \say{that dauntless temper of [Banquo's] mind} could uncover his murder (3.1).

Unfortunately for Macbeth, his reign is still cut short by murder.
Malcolm is approached by Macduff, who hopes to follow the king's son in restoring the throne.
Malcolm, wary at the idea of retaking the throne, especially as the call comes from a Scotsman, denies his abilities, stating that he \say{should forge / Quarrels unjust against the good and loyal, / Destroying them for wealth} (4.3).
This fear of taking the throne seems odd in the show, especially since Macbeth, who has done many of these actions, never once feared becoming a tyrant.
Quickly, Shakespeare illustrates to the audience that Malcolm would be a just ruler, as he claims that he \say{At no time broke [his] faith, would not betray / The devil to his fellow and delight / No less in truth than life} (ibid).
In this, we see that the idea of loyalty is tied directly to the idea of truth and faith.
Malcolm would not be disloyal to his people, including through deception.

978 words
\section{Draft 1: 11 December}
Shakespeare's \textit{Macbeth} is a show about loyalty and fealty, and the inevitable conflict when these two forces come into opposition.
First, we should examine what we mean by both of these words.
Loyalty is what is felt towards peers and lessers, a mutual understanding that you will offer protection.
Fealty, on the other hand, is offered to a superior as a promise that you will serve.
One crucial theme running through \textit{Macbeth} is that loyalty to Scotland and fealty to Scotland are not one and the same.

Act 1, Scene 2, has Ross first utter the word loyal, though here it's modified by the preposition dis.
He speaks about how \say{that most disloyal traitor,/The thane of Cawdor, began a dismal conflict,} then concludes that Macbeth, who earlier was named \say{the worthy thane of Ross,} slew the traitor.
Here we see loyalty and fealty to Scotland as one and the same.
Macbeth is loyal to Scotland, defending her from the invading \say{Norweyan banners.}\footnote{1.2}
However, he also exhibits his fealty to Scotland, as he puts his life on the line at the command of the King.
Macdonwald, the now-dead Thane of Cawdor, showed neither loyal nor fealty.
He rebelled against the crown, and pledged his allegiance to Norway.

Two scenes later, Macbeth and Duncan discourse on the difference between loyalty and fealty.
Duncan states that he has failed as a liege lord, for \say{more is thy due than more than all can pay,} as he feels that he has failed to give Macbeth what is owed to him as a faithful servant.
Macbeth, however, gives the view of fealty that is common to traditional heroes, namely that \say{The service and the loyalty I owe,/In doing it, pays itself,} then immediately follows with what fealty and loyalty mean.
Namely, that is {Your highness' part / Is to receive our duties; and our duties / Are to your throne.}
That is, the king's role is to accept and honor the works that his lessers do, and they in turn are to support the king.
When both the king and the people support the other, we have the reign of Duncan, and what is hoped for in the reign of Malcolm.
When this mutual growth doesn't exist, we see what happens in the reign of Macbeth: death and sorrow for all.

The third time the word loyalty is used is in Act 2 Scene 3, when Macbeth speaks of the people he's just murdered to hide his first regicide.
Here, the idea of loyalty is used as a counterpoint to the idea of neutrality, as Macbeth asks \say{Who can be wise, amazed, temperate and furious, / Loyal and neutral, in a moment? No man.}
This line could be done a variety of ways.
Paul Ready's Macbeth at the Sam Wanamaker Playhouse speaks this line incredulously.
Macbeth seems legitimately amazed at the idea that someone could hold these contradictory feelings in them at once.

The final time the word \say{loyal} is uttered is in Act 4, Scene 3.
Here, Malcolm speaks about how he would fail as a king.
He would \say{forge / Quarrels unjust against the good and loyal, / Destroying them for wealth.}
Of course, we quickly find out that he does not mean this.
Instead, he would be a good king for Scotland, and used these words to test whether Macduff truly had his and Scotland's best interests at heart.

\section{Draft 0: 10 December}
Shakespeare's \textit{Macbeth} is a show about loyalty and fealty, and the inevitable conflict when these two forces come into opposition.
First, we should examine what we mean by both of these words.
Loyalty is what is felt towards peers and lessers, a mutual understanding that you will offer protection.
Fealty, on the other hand, is offered to a superior as a promise that you will serve.
One crucial theme running through \textit{Macbeth} is that loyalty to Scotland and fealty to Scotland are not one and the same.

Act 1, Scene 2, has Ross first utter the word loyal, though here it's modified by the preposition dis.
He speaks about how \say{that most disloyal traitor,/The thane of Cawdor, began a dismal conflict,} then concludes that Macbeth, who earlier was named \say{the worthy thane of Ross,} slew the traitor.
Here we see loyalty and fealty to Scotland as one and the same.
Macbeth is loyal to Scotland, defending her from the invading \say{Norweyan banners.}\footnote{1.2}
However, he also exhibits his fealty to Scotland, as he puts his life on the line at the command of the King.
Macdonwald, the now-dead Thane of Cawdor, showed neither loyal nor fealty.
He rebelled against the crown, and pledged his allegiance to Norway.

Two scenes later, Macbeth and Duncan discourse on the difference between loyalty and fealty.
Duncan states that he has failed as a liege lord, for \say{more is thy due than more than all can pay,} as he feels that he has failed to give Macbeth what is owed to him as a faithful servant.
Macbeth, however, gives the view of fealty that is common to traditional heroes, namely that \say{The service and the loyalty I owe,/In doing it, pays itself,} then immediately follows with what fealty and loyalty mean.
Namely, that is {Your highness' part / Is to receive our duties; and our duties / Are to your throne.}
That is, the king's role is to accept and honor the works that his lessers do, and they in turn are to support the king.
When both the king and the people support the other, we have the reign of Duncan, and what is hoped for in the reign of Malcolm.
When this mutual growth doesn't exist, we see what happens in the reign of Macbeth: death and sorrow for all.


\section{Draft -1: 9 December}
Shakespeare, like all great writers for the stage, uses his staging notes as a way to inform his own views of characters and plot.
The clearest example of this is the fact that Shakespeare doesn't believe that Macbeth is truly the rightful ruler of Scotland.
To truly justify this claim, I will also look at two other Shakespearian tragedies: \textit{Antony and Cleopatra} and \textit{Othello}.

\textit{Antony and Cleopatra}, a show about a group of rightful monarchs exercising their rights, has a variety of flourishes.
Throughout the five acts of the show, Shakespeare calls for a grand total of nine flourishes.
\textit{Othello}, which has no royalty, has no flourishes.
\textit{Macbeth} itself has five flourishes.
However, these flourishes are never given to Macbeth, only his predecessor and successor.

In Act 1, Scene 4, Duncan receives two flourishes.
The first happens as the scene begins, and he enters the stage.
The second occurs as Duncan leaves.
Through these two flourishes, Shakespeare clearly states that Duncan is the rightful authority in the realm, as he calls for flourishes in his other shows at these similar moments.

Shakespeare does not dispute Macbeth's force of arms or leadership on the battlefield.
When Macbeth is first seen, in Act 1 Scene 3, Shakespeare calls for a \say{drum within,} as Macbeth enters.
The witches comment, \say{a drum! Macbeth doth come,} which echoes Iago in \textit{Othello}.

In Act 2, Scene 1 of \textit{Othello}, Iago proclaims \say{The Moor! I know his trumpet} after Shakespeare calls for a \say{trumpet within.}
By doing this, Shakespeare has two of his antagonists comment on the marching symbol of the titular protagonists.
In doing so, he also links his protagonists to martial calls, in \textit{Othello}, a trumpet, and in \textit{Macbeth} a drum.

Late in \textit{Macbeth}, during the siege on Castle \textbf{NAME}, Macbeth again is given drums as he prepares to battle the enemy forces in Act 5, Scene 4.
However, Malcolm, the late king's son and future king, receives \say{drums and colours} three times, twice while marching to kill Macbeth, and once when he has confirmation of Macbeth's death.

In fact, immediately after we watch Macbeth leave the stage to die, Malcolm enters in Act 5, Scene 8 to a flourish.
In case it wasn't immediately clear what Shakespeare meant to say, when Macbeth's severed head is brought on stage, the ensemble proclaims \say{Hail, King of Scotland!} and a flourish sounds.
As Malcolm leaves to be \say{crown'd at Scone,} the show ends, and the musician is instructed to end the show with a flourish.

We can contrast the death of a monarch in \textit{Macbeth} with Cleopatra's death scene at the end of \textit{Antony and Cleopatra.}
Immediately after Macbeth dies, we hear a flourish announcing the true king, yet Caesar, who before this point had received 5 flourishes announcing his entrances and exits, is not given a flourish when he enters to see the dead queen.
Through that, Shakespeare makes the point that the death of a rightful monarch is not a joyous occasion, even when the monarch is opposed to your own interests.
Macbeth, as a false king, deserves nothing but scorn.
\section{Draft -2: 9 December}
Antony and Cleo:
2.5 Cleo wants music
2.6 Pompey and Menas enter w instruments
2.7 Music plays
2.7 sennet sounded, egyptian bacchanals, 
1.1, 2.2x2, 2.6, 2.7 w drums, 4.4 \say{trumpets flourish}, 4.6, 5.2x2 flourishes
\href{http://www.lieder.net/lieder/get_text.html?TextId=18775}{Come thou monarch}  is a shakespeare original 

Argument: Shakespeare wants the audience to know that Macbeth isn't a rightful king, and makes it clear through the use of flourishes.
Othello, which doesn't concern any royalty, doesn't have any flourishes.
Antony and Cleopatra, on the other hand, has lots of flourishes.
1.1: Antony and Cleo enter,
2.2: Antony announcing he'll fight before being greeted by caesar,
2.2: Caesar, Antony, Lepidus exit,
2.6: Pompey and Menas enter w drum and trumpet, Caesar, Antony, Lepidus, Enobarbus, Mecaenas enter from other door,
2.7: flourish bc they're saying goodbye before exiting,
4.4: Captains and soldiers enter,
4.6: Caesar, Agripa, barbus enter,
5.2: Caesar, Gallos, Proculeius, Mecaenas, Seleucus, etc enter,
5.2: Caesar exits.

(2k words needed)
\section{Draft -3: 8 December}
I can't believe I deleted my draft again.
Oh well, I only had the first draft, so it's fine.
I'm comparing use of music in 3 Shakespeare tragedies: Macbeth, Othello, and Antony and Cleopatra.

Macbeth:
Flourishes called for five times: first time Duncan enters and leaves stage (1.4), when malcolm enters after macbeth dies (5.8), when macbeth head is brought out, and when he leaves to end the show.
Hautboy (1.7) to signal the horrible thing happening at castle Macbeth.
Hecate speaks in rhyme.
Song: come away, come away.
\href{https://doi.org/10.1093/mq/XLVII.1.22}{this guy} claims that come away come away (all of hecate really) doesn't count as Shakespeare, so we'll ignore it.
4.1 has more hautboys, during the scene with witches and ghosts.
Idk if that's included in the not real bit, so I'll assume it is since I don't remember those lines from the performance.

1.3 calls for a drum- witches prepare for macbeth
5.2 calls for drum and colours- marching to kill mac
5.4 again drum and colours, same thing
5.5 drum and colours as macbeth preps for war
5.6 as macduff enters to kill beth
5.8, with flourish is drum and colour

Othello:
\href{https://www.shakespeare.org.uk/explore-shakespeare/blogs/shakespeares-drinking-songs/}{canakin clink} (2.3) was likely a popular drinking song. \href{http://www.lieder.net/lieder/get_text.html?TextId=18061}{claims} original.
\href{http://www.lieder.net/lieder/get_text.html?TextId=31027} based off of folk song (2.3)
\href{http://www.lieder.net/lieder/get_text.html?TextId=14898}{poor} \href{https://www.shakespeare.org.uk/explore-shakespeare/blogs/shakespeares-saddest-song/}{soul} not Shakespeare original, but version by him 4.3
Iago noticing othello trumpet (2.1) \say{The Moor! I know his trumpet.}
3.1 cassio brings musicians, they get thrown away
5.2 emilia and willow song

\end{document}