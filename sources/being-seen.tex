\documentclass[12pt]{article}[titlepage]
\newcommand{\say}[1]{``#1''}
\newcommand{\nsay}[1]{`#1'}
\usepackage{endnotes}
\newcommand{\B}{\backslash{}}
\renewcommand{\,}{\textsuperscript{,}}
\usepackage{setspace}
\usepackage{tipa}
\usepackage{hyperref}
\begin{document}
\doublespacing
\section{\href{being-seen.html}{On Being Seen}}
First Published: 2023 December 19

\section{Draft 1}
I find that it's always interesting to learn how others see me.
I'd like to think that my own self image of myself hasn't changed over the years, for all that I know that it has.
Without delving too deeply into those, it's been a long and fun journey to really feel like myself in my own skin, which is something I'm glad for.

But, this musing isn't about how I see myself.
It's about how others see me.
I recently realized, after a few conversations, that the way I am perceived is far different than it once was.

Through high school, I think that I was seen mostly as a member of my family.
That's fair, and that hasn't really changed.\footnote{literally the other day a person approached me on the street and went \say{are you a (insert last name (not that any of you reading this blog couldn't immediately figure out my last name) here)?}, which was wild, given that I'm, as far as I know, the only member of my family to have spent more than a week in the city}
However, that was mostly among people who were, if not friends, then at least one link from a friend or family member.
When people saw me not as a member of my family, I know that one of the most common thoughts was that I looked like a football player.\footnote{to be fair, I was a football player from middle school through the end of high school, and it was absolutely the sport I felt the most connection to, for all that it's absolutely not the sport I was objectively best at}

Even through college, those were both common statements a common statement.
More than that, though, I know that I was often viewed as someone who enjoyed violent sports.\footnote{every time that I use that phrase I feel like it's the wrong one.}
The other day, though, I was at a rugby field with some graduate school friends, I made a comment about how I was considering trying rugby.\footnote{or something similar, I don't exactly remember}
The friends I was with both expressed surprise, which made sense to me, at first.

After all, I have never once expressed interest in rugby before, as far as they knew.\footnote{little do they know that one of my favorite babysitters (not the famous movie star) growing up was a rugby player}

However, that was not where the confusion lay.
My friends both expressed shock that I would be interested in a physical\footnote{is that the right word? I don't think so.} sport.\footnote{I think that I'm averaging about a footnote per sentence, which says a lot about what I'm thinking in this post}

That made me curious.
I realized that many of the people I've met since starting graduate school express surprise when I tell them that people used to think of me as physically intimidating.
I'm only now\footnote{literally as I write this blog post} realizing that this might have been less an expression of how friendly I am, and more a reflection of a change in the way that I'm generally viewed.

The next day, a friend was apparently talking about me to someone who didn't know me.
My friend described me with many of the common descriptors.\footnote{much as I wish that it was one, that did not include \say{incredibly attractive}. It did, however, include that I dove, like board games, am getting a graduate degree, and do music. It's a pretty fair summary}
Someone who knew both my friend and the new person commented that I have a nice smile.

I did smile when I heard that.

I don't disagree with the idea that I have a nice smile.\footnote{and not just because it's rude to disagree with people. I do actually like my smile}
However, I know that in high school, I was very uncomfortable with my smile.
I would do everything in my power to avoid smiling in photos.
It does make it a little awkward to show photos of high school me to friends now, though.

Returning to the point, I'm realizing that the way I am perceived frequently differs from the way that people in my life say that they perceive me.
I'm sure that there's some introspection I can do about that, and may in the future.\footnote{not tonight, though, because I'm tired}

Daily Reflection:
\begin{itemize}
\item Hobbies:
\begin{itemize}
\item Did I embroider today? I did a single stitch so I wouldn't lose the embroidery needle. I then remembered to take it with me.
\item Did I play guitar today? I packed it, which is nice.
\item Did I practice touch typing today? Maybe I should delete this goal.
\end{itemize}
\item Reading
\begin{itemize}
\item Have I made progress on my Currently Reading Shelf? I finished the book that I was listening to! And then I finished the first book in a series with a friend.
\item Did I read the book on craft? No, but I packed it!\footnote{I suppose there's an implicit thing that I'm dancing around}
\item Have I read the library books? I found a bunch more books that I want to read, which is sort of the opposite of progress.
\end{itemize}
\item Writing
\begin{itemize}
\item Did I write a sonnet? Yesterday's musing was well received. Today, hopefully I'll get one.
\item Did I revise a sonnet? Had so many false starts yesterday.
\item Did I blog? I mused, at least.
\item Did I write ahead on Jeb? I wrote about half a chapter. Hopefully I'll be awake enough to write a little more before bed. Otherwise, will finish the chapter in the morning.
\item Letter to friends? Very no. Visited a friend, though!
\item Paper? I realized that I never actually did the calculation to figure out how distortion constants play with the normal numbers, but wow I do not know how matrices work.
\end{itemize}
\item Wellness
\begin{itemize}
\item How well did I pray? Still bad.
\item Did I clean my space? I made so much progress!
\item Did I spend my time well? Yeah! I did good.
\item Did I stretch? Whoops! Maybe tomorrow.
\item Did I exercise? Nope!
\item Water? I tried talking a few times today, and then chugged a bottle of water because I was thirsty.
\end{itemize}
\end{itemize}
\end{document}