\documentclass[12pt]{article}[titlepage]
\newcommand{\say}[1]{``#1''}
\newcommand{\nsay}[1]{`#1'}
\usepackage{endnotes}
\newcommand{\1}{\={a}}
\newcommand{\2}{\={e}}
\newcommand{\3}{\={\i}}
\newcommand{\4}{\=o}
\newcommand{\5}{\=u}
\newcommand{\6}{\={A}}
\newcommand{\B}{\backslash{}}
\renewcommand{\,}{\textsuperscript{,}}
\usepackage{setspace}
\usepackage{tipa}
\usepackage{hyperref}
\begin{document}
\doublespacing
\section{\href{screaming-into-void.html}{Screaming Into the Void, or Why I'm (Still) Writing This}}
First Published: 2022 January 3
\section{Draft 1}
Something that I reflected on at least once in my initial blogging\footnote{though I cannot find the post now} is why I wrote it.
In part it was due to an assignment to journal more, and in part it was to keep people in my life updated on my whereabouts and doings.

Of course, a lot has changed since August 2018.
That class ended, for one.
I've finished my semester abroad.
I've graduated college and started my Ph.D. Program.
I've moved to a new city and state, where I will\footnote{hopefully} be for the next 3 or so years.
Most of my day-to-day isn't surprising to anyone who knows me.

So I have to think about what this blog is for anymore.
It's not enough for me to think of it as a way to force myself to write more\footnote{though that is certainly a large part of what I'm doing it for}, because I can write in other ways or in other places.
It's not enough for me to claim that this could be useful for my family, friends, and loved ones\footnote{these circles are not mutually exclusive in any way} to know what's happening in my mind, because I self censor far too much for that, and I talk to them besides.
It's not even enough for me to reflect on how great it was to see what I wrote while in London as a way of remembering my own past, which was incredibly enjoyable the other day.

Instead, I think my goal for writing this blog in 2022 will be to see it as\footnote{much like the title suggests} screaming into the void.
As a scientist, one metaphor that has stuck with me is the idea of the sum of human knowledge being represented as a sphere.\footnote{or sphere-like object, because soon we're deforming (though if I've learned anything about topology [or physics], it's that everything is basically a sphere}
Our job as scientists is to pick one infinitesimally small point on that sphere and push it just the slightest bit forward.

In time, our protrusion into the void of the unknown may be the starting point needed for others to push their own ideas forward and bring the world into a better way.
Or, the protrusion may end up simply remaining there, unused but known.

In this image, all of science\footnote{using a more historical meaning of science, moreso learning than STEM} is dedicated to looking away the safe and comfortable bubble of the known and used, and staring into the void of the unknown, trying to pin something down to add to the bubble of safety.

In some sense, that's one of my goals here.
Every way that I self-reflect, whether talking to others about myself, thinking about myself, writing about myself, writing as myself\footnote{because who else can I write as}, or even just existing pushes me into new truths about myself.
If my goal\footnote{which it is} is to know myself, how can I say that I'm working towards it if I refuse to expand my bubble of knowledge into the void.

More like music than chemistry, I expand into the void not by study and measurement, but by creation and action.
By writing this blog daily, staring into the void which is the genesis of my writings, I hope that in time I will be able to better see how and why I act the way I do.

Of course, there are other benefits I hope to enjoy as well.
For instance, my Sunday reflections on the Gospels I hope to use as a way of deepening my faith and my connection to the Lord.
My study posts I hope to use to pass my classes.\footnote{not actually, I am truly in the part of my life where I'm more concerned with the learning than the grade, which is horribly exciting}
My posts about concerts and events I see I hope to use to encourage myself to go to those events more, so I have an easy excuse not to self reflect for a day.

One final reason, though it's less cheerful than the rest, is similar to a reason my inspiration for this blog came from.
He wrote\footnote{writes?} his blog in part because of the knowledge of his own mortality.
When he is no longer here, he hopes that the blog he writes will be a source of fond memory and a way to connect with him for his family and loved ones.
If I'm being honest with myself, that has to be a part of what I'm doing too.
I hope that something that I write here, as I scream into the apparent void of unlisted urls, unpublicized blogs, and my own unknown mind, that at some point the signal will reach someone I love long past when I wrote it.

So, if you're here and want to drop me a line, feel free to message me and let me know you're reading.\footnote{if you don't know how to contact me, I'm unsure how you got onto this blog, but I'm sure you can find me somewhere}
\end{document}