\documentclass[12pt]{article}[titlepage]
\newcommand{\say}[1]{``#1''}
\newcommand{\nsay}[1]{`#1'}
\usepackage{endnotes}
\newcommand{\1}{\={a}}
\newcommand{\2}{\={e}}
\newcommand{\3}{\={\i}}
\newcommand{\4}{\=o}
\newcommand{\5}{\=u}
\newcommand{\6}{\={A}}
\newcommand{\B}{\backslash{}}
\renewcommand{\,}{\textsuperscript{,}}
\usepackage{setspace}
\usepackage{tipa}
\usepackage{hyperref}
\begin{document}
\doublespacing
\section{\href{writing-7.html}{On Writing Again}}
First Published: 2023 November 14

\section{Draft 1}
First, let's address the elephant in the room.
\href{dungeons-dragons-6.html}{Yesterday} I said I would be musing about my Pathfinder game tonight.
However, schedules ended up misaligning, and so we did not have pathfinder tonight.
Instead, I'm going to\footnote{in a meandering manner, as befits a first draft blog post (especially since I really don't want to write my book tonight but want to get more words on page, because it feels good to do so) of hopefully two} discuss some writing advice I saw today, and what my initial reactions are to it.

The last time I mused about writing was apparently \href{writing-6.html}{a few months ago.}
In that post, I talked about what was and wasn't working for me in writing during that period of my life.
Let's see how that's changed or stayed the same.

I was using an app called Stimuwrite at that point to motivate me.
It was a great app, and I would still really recommend using it, especially in places with less than high speed and reliable internet.
Thankfully, right now I have that constantly, so I have been using a web based app called 4thewords.
It's a really fun rpg style writing app that I may have spoken about before.

In case I haven't, I was very curious how they would make the fact that you fight monsters in this game work with the fact that it's an rpg.
For those who don't know, games in that genre tend to have different upgrades for your character as the game progresses.
4thewords has an interesting approach, where it takes fewer words typed to defeat monsters the higher your attack stat is.
It's fun for me, because it really does help the typing feel more meaningful.
Instead of a dopamine hit every one hundred words, for instance, right now I'm at the point where I get it about every seventy.

That, combined with the fact that they have a whole story line, including a side quest series for the month of November, makes it really motivating for me to write.
There are so many times that I look at the page, not wanting to write, but see the counting down of the clock and feel the words start to flow out.

I was struggling to write poetry then.
Now I am not.
Now, this is not because I suddenly got over the mental block which stopped me.
No, I took the easier route and stopped wanting to write poetry.
It at least aligns my actions to my goals, even if I don't necessarily want it to.\footnote{I forget if I've said on here, but I think that next month, instead of writing a novel, I might consider doing daily poetry attempts again.
I've set aside so much time for writing that it will hopefully be enough of a habit that I can continue.
Then again, I've recently started thinking about an idea for a series of stories or books that I'd like to write, but I'm unsure how workable the idea is.
The series would be Modern Myths, and I'll probably muse about it another time.}

I was in a good habit of writing letters to friends, which I'm trying to get back into.\footnote{it's shockingly hard to remember what to put in letters.
I feel like this summer I had a pattern down and knew what to write, but that doesn't feel true anymore for some reason.
Probably worth interrogating myself as to why that is (not that I don't already have ideas or anything)}

I wasn't journalling, which I kind of forgot was something I was trying to do.
I feel like this is a place that I spend a lot of time and words journaling, though.\footnote{neither of the two (different) spellings was accepted by the spell checker, so I have no idea what it should be.
Instead of being consistent, I'll probably switch between them in a pseudo-random fashion}

In writing that last post, I had just gotten a fountain pen.
Since then, I've written almost exclusively with it, and it really does make the writing process feel so much better.

I think that one major difference is the ritual involved.
I keep the pen in a small plastic/metal\footnote{it feels too cheap and flimsy to be metal but moves too easily to be plastic} case, and there's something really satisfying about opening the case to pull out my pen to start writing.
It's also, as I mentioned in the first musing, heavy.
The weight is really nice, because it reminds me that what I'm doing is an act of creation.

I think that I'm writing more again, though I don't know if it's all in the journals I keep.
I have been doing a lot of derivations for my research, and I've filled pages of my notebook in dense scratching text.

One benefit of using a pen with replaceable cartridges is that I get to change the color.
I apparently write through about one cartridge a month, which means I get to swap what I write with about once a month.
It's interesting to me how different my writing feels when I change up the ink even slightly.
I almost feel like it's the advice that is often given to guitarists: when it stops being fun to play the guitar, switch your strings.
Something about the new color just makes me want to write by hand a lot more.

It also flows so smoothly.
I didn't realize how much I would hate using ball point pens until I stopped having to.
When I use them now, it feels like I'm using a pen to write something down, rather than the pen feeling like a natural extension of my hands.

Anyways, one thousand words of rambling in, let's get to the meat of the discussion.
The writing advice I saw today came courtesy of a fantasy author I follow on tumblr, who shared a screencap with annotations of writing advice from Chuck Palahniuk.
It's framed as a six month process, wherein you are not allowed to write in certain ways.

Now, this is not a restriction like the sorts of constrained writing where you can't use certain letters, or you need to have certain cadences or rhyme schemes.
Instead, it is focused on the content of the writing, and making a point to show not tell.
It's filled with examples of how the process works, so I will do my best to synthesize the content here.\footnote{if you can't tell, one of the main purposes of this blog is to give me a space to write through how I feel about something.
I type far faster than I write (maybe I should learn shorthand to rectify that), so this works better for me by far than trying to journal.
Also, when I journal, I feel an urge to at least nominally work on my penmanship, which is a bit of an issue because I write even slower.}

First, no \say{thought} verbs are allowed.
That is, we are not allowed to say how a character feels.

Rather than \say{Jim wondered whether Jill was mad at him,} I instead would have to describe the scene.
\say{Jim had grown accustomed to Jill's bright wave every morning when they crossed paths.
Lately, though, she had given half hearted waves, if she bothered to acknowledge him at all.}

I still don't know if I did that correctly.
The author specifically says the goal is to show the reader through sensory cues exactly what the characters are feeling.
Characters cannot want or feel something, I must make the readers want or feel.\footnote{which is strange, and not something I necessarily think translates to all my writing particularly well.
I absolutely think that as an exercise it's worth doing, if only because I can see how prescriptive my writing style is.
I worry that I'll go to far with it, though, ending up with purple prose.
As one person I saw commented on a similar sentiment, not every sentence can be profound and breathtaking.
Sometimes the character just needs to be on the other side of the room.}

The author of the advice points out that many authors hamstring themselves by writing a paragraph that begins with how the character is feeling, killing the energy of the rest of the paragraph.

That is, a forbidden text might read something like:

Alex knew that he would be late to work today.
Five busses had passed him by without stopping, and another five just hadn't shown up.
Taxis seemed in shockingly short supply, and the wind precluded his running in any direction.

When rectified, the paragraph could begin Alex was late to work that day.\footnote{i guess I see where getting rid of think could help, it does read more snappily}
Or, at the very least, the think can be moved to the end of the paragraph.\footnote{which also does a lot to help.
Maybe there's something to this published and well renowned author's advice}

One particular piece of advice he gives is to minimize time that characters are alone.
Especially given the sort of writing that I'm prone to, that seems infeasible.
I really like the fact that my book has a deep exploration of a single person studying, for all that I can see, reading his advice and looking at my writing, exactly why my writing has been described as so impersonal.

Forgetting and remembering are also considered no nos, but he used it in the context of like remembering childhood.
Most of the time that I use remember, it is to spark something from a conversation that happened on screen a few chapters earlier.\footnote{now, is that itself good writing practice? maybe not.
I feel like a lot of advice is not written presuming serialized formats though.
By nature of the staggered releases, it is almost certain that the audience will not remember as well as if they had the entire book in front of them when they began to read}

He also rails against the verb to be, pointing out that most of the time, it's unneeded.
Rather than saying \say{Phillip had red hair,} I could do something more like \say{Phillip grinned and ran a finger through his scarlet locks.}
The homework is to go through my writing and get rid of any instance of thought words, then do the same to a published text.
I don't know if I'll actually do that this month, especially since it seems better used for an editing/revising phase of writing, at least at first.\footnote{as a friend put it, otherwise they'd spend too much time editing while they write, which is against the core ethos of NaNoWriMo}
Still, it is probably worthwhile to at least try drafting a chapter normally and then going through and seeing how differently it reads when following this advice.
I'm beginning to understand why some people treat learning the craft of fiction like they and others treat learning music theory.

For those who may not know, there's a common trend among amateur musicians\footnote{and, as I've recently learned, amateur writers} where they believe that learning too much about the craft will lessen their creative potential, making their writing like everyone else's.
As a musician, every time I hear that take, my response is either:
\begin{itemize}
\item by everyone else's you mean the people whose music you admire, listen to, and enjoy?
\item No, learning the rules explicitly keeps you from following them completely by rote because all you know is C, D, G on a guitar, which is following such basic rules that it's incredibly limiting to anything you could want to write
\end{itemize}
Of course, most of the time those responses are in my head.

As a former student of music\footnote{in that I formally studied music, not in that I think I know everything now}, I will admit that those people have a point.
During my first semester of learning theory explicitly in particular, my writing became much more stilted, following the conventions as I was taught them badly.\footnote{that is, I was bad at following the conventions. I was taught the conventions wonderfully and excellently}
After that initial growth period, though, I can see where my music really becomes my own.

It does bring an interesting issue, where you have to weight the audiences of trained and unskilled.
Let me try describing that better, using an analogy.\footnote{which just for this, I'll try to use the writing advice I'm nominally still reflecting on}

My first semester sophomore year, I took a composition class.
In that class, we were tasked with a number of compositional exercises which culminated in our final project: an original composition that we played live in front of the class and whoever came to watch.\footnote{our professor advertised, and I went to a small enough school that people actually went to their friend's class presentations, which doesn't really seem to happen where I am now. It's also possible I'm just out of the loop here, or that it didn't happen as much as I thought where I went to school.
Who can say for certain?}
Of course, writing a solo piece is incredibly difficult, because a single instrument playing tends to be boring unless it is being played by an expert.\footnote{now, we can get to the whole \say{wow that's incredibly elitist and western music centric of you, look at (insert any other solo tradition here) or think about all the singer songwriters out there.} That is a valid critique, and it is a sticking point for a lot of theorists that western music analysis more or less ignores words.
I'm now realizing that might be part of why so many people make such a big deal of madrigalisms (a lot of the madrigals (a style of polyphonic music) that still exist are sacred.
There are some conventions in writing for them, such as having either a unison or triad on the word G-d and Trinity (respectively for the first level and counterespectively for the spicy \say{look who understands theology} version) and rising on words like rise, falling on words like down), since they're really the only genre in western canon that has that (as with anything in music, exceptions apply) universally applicable)
As a result, artists like Woody Guthrie or Bob Dylan, though they uncontroversially write beautiful and wonderful music, are less interesting to study from a music perspective, as they intentionally rely on the stock nature of their arrangements and compositions to allow their words to shine through.
(Wow this is a lot of words, maybe worth not being a footnote, but too late)}
As a result, almost all\footnote{i think all of us, but I might be wrong (footnotes don't count, because they're not part of the narrative, so I don't have to write correctly according to the style)} of us chose to write a piece for an ensemble, which we were responsible for assembling.

My ensemble was two clarinets, a piano,\footnote{I forgot about this at first, and then remembered} and an oboe, and many of the other members of the class also wrote for strange voicings, based on what friends we had available.
One early draft of my piece had a series of chained 2-1 suspensions\footnote{for the non music nerds, a second is when you have the two notes next to each other, and suspensions are when you have a dissonance because one voice holds while the other moves.
In this case, they move to a second, then resolve to a unison, move to a second, on and on}
between, if my memory serves\footnote{see, that's the thing about this style guide.
Sometimes I do not know what I'm trying to remember, and the effort is worth mentioning.
Maybe there's a way to make it work, I should consider that as I think about this advice}
the two clarinets.
It was an intentional choice, because I really liked the way that they sounded.

My professor advised me that I should get rid of them.
He said something about them being a mark of someone who doesn't know how to compose.
At the time, I, like every other young creative,\footnote{I feel comfortable saying this, since I read a book that said it's common for writing teachers} bristled at this intrusion onto my creative vision.
However, I did really trust the professor and his advice, so I changed it to a series of 3-2 suspensions.
The entire piece clicked much better once I did, and it was a choice I'm really glad was made for me.
Also, now when I write I have a much better intuitive feeling for suspensions and the ways that our ears, living in the western canon as we all do, are trained.\footnote{there are a lot of arguments about how much of what we think about music is innate and how much is learned.
It's really hard to test these days, as basically every population we can interact with has been exposed to tons of western music, through the radio if nothing else.
Global hegemony has its downsides, and anthropologists are the ones to suffer.
That being said, it is uncontroversial to say that in general minor chords are sad and major chords are happy.
Whether this is because of something innate or learned does not matter, because it will work on your audiences.
(That's another way that learning is good.
I now know that when I want to showcase a sadder part of a text, I want to rely more on minor chords and motions to subdominant keys rather than dominant keys)}

Ok wait, no that's not a good example of how experts and non experts have different taste.
Shoot.
It does help illustrate the way that learning helps you to create the vision of what you want, though.
When I took the step back, I realized that my goal was not to create 2-1 suspensions.
That is, my goal was not any individual part of the creative process.
My goal was producing something beautiful at the end.
By learning what the typical standards of beauty were, I was able to channel my creative impulse more effectively.

I guess a better example of what I mean by expert and non expert ears is a lot of contemporary classical music.\footnote{oof those two words hurt to put in that order because there are so many contradictory meanings for them}
Or, if not that, then at least like twelve tone serialism from the early twentieth century.\footnote{to be fair, that is where most music theory classes stop.
After that, there's not a lot of canon, for all that there is absolutely development happening.
There's a lot to explore in the ways that mass production and dissemination of musical recordings changed the concept of musical eras, even before we get to the fractured space of modern music, where there's a lot of talk about there being no real cultural zeitgeist.}
Those who know what they're listening to\footnote{often because they're studying the score} get to see interesting and beautiful theory worked out on a page.
To the untrained ear, however, it just sounds like noise.

Of course, anyone writing music is aware\footnote{in my experience at least} of the audience they're writing to.
Or, rather, they should be.
I certainly write differently for myself on guitar, my band, choir, and other ensembles.
I had to learn how to do that, though, which is another point for formal learning: it gives you access to different registers.\footnote{shoot, need to focus on the thread, for all that I lost it far far ago}

What I mean to say is that I don't plan on writing literary fiction.\footnote{which I hate as a concept, much like I hate academic music as a concept, for all that I respect it more.
I'm positive that if I ever took courses on literary fiction that I would feel much better about it, because it is absolutely the anti intellectualism speaking when I say I don't like high brow stuff.
Of course, there's also explicit elitism in creating that's probably worth analyzing.
Certainly all the early pioneers into really weird music that I read about hoped that audiences would learn to like it, while it feels like a lot of writers disdain the common listener.
Who knows?
Might be worth musing about in depth later}
As a result, if getting rid of words like think and hope makes my writing read less like the fiction I want to write, I will probably stop doing it quickly.

Then again, as I've mentioned, I only know how to write music in different registers because I've tried and failed to write the same register for incompatible uses.\footnote{word to the wise, singer songwriter guitar songs should be basic in melody and chord progression}
I suppose that until I know what it means to write at different levels, my writing will always have that same issue.\footnote{my beta reader does often comment on the fact that the targeted reader reading level changes a lot based on the chapter and scene in my book.
That's probably bad.
Just as much as I don't want to have one scene of \say{this is bill.
See bill run,} in the middle of a high brow discussion, I also don't want to have something very high brow (I tried to write one, but realized I don't know how to do so consciously.
That's probably bad.
I think it involves like verb splitting (to boldy go) but don't know that for certain) in the middle of like an action scene where the goal is fast paced words.
I did also see something about how sentence length can be used to control a reader's reading.
That's absolutely something I should get better at.
Really what I'm saying is I want to read books on craft and learn to apply them}
So, in conclusion, I like the advice, for all that I don't know if it will be like classical counterpoint for me\footnote{something I'm aware of and make notice of when composing in style} or if it will be like learning melodic motion\footnote{something I use by reflex now, and then refine and double down on when revising a song}.
In either case, it will be helpful for me as I grow as a writer.

Ok, I said that this would be a two draft musing, but then I wrote continuously for twoish hours.
It's now bed time.

Daily Reflection:
\begin{itemize}
\item Did I write 1700 words for NaNoWriMo?
I did! Keeping on schedule, which is nice.
I also remembered to log that I wrote, which is annoying to remember to backdate.
\item Did I write a chapter of Jeb?
I wrote half a chapter.
It was a bit of a slog, but I hope that tomorrow is easier.
\item Did I blog? I rambled for almost four thousand words, at the very least.
\item Did I stretch? Oh gosh that's something I've fallen right out of the habit of.
It's hard to know what's a thing you do to recover from illness faster and what prolongs feeling bad.
Stretching is probably the former.
\item Am I doing better at prayer than a rushed and thoughtless rosary?
Eh. Honestly not really, but I don't know how much of that is not making time, and how much of it is that my mind is not in a good space right now.
I do know that I pray better when I'm doing better, which is kind of annoying.
\item Am I doing a good job writing letters to friends?
I posted the letter that I wrote yesterday, so that's a good enough job for right now I think.
\end{itemize}
\end{document}