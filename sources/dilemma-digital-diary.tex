\documentclass[12pt]{article}[titlepage]
\newcommand{\say}[1]{``#1''}
\newcommand{\nsay}[1]{`#1'}
\usepackage{endnotes}
\newcommand{\1}{\={a}}
\newcommand{\2}{\={e}}
\newcommand{\3}{\={\i}}
\newcommand{\4}{\=o}
\newcommand{\5}{\=u}
\newcommand{\6}{\={A}}
\newcommand{\B}{\backslash{}}
\renewcommand{\,}{\textsuperscript{,}}
\usepackage{setspace}
\usepackage{tipa}
\usepackage{hyperref}
\begin{document}
\doublespacing
\section{\href{dilemma-digital-diary.html}{The Dilemma of Digital Diaries}}
First Published: 2018 December 6
\section{Draft 12(?): 4 December 2018}
The internet forces change on almost every aspect of human life.
These changes can be controversial, especially in literature.
Particularly for diaries, the idea shifting from analog to digital formats seemed risk-filled in the early days of the internet.\footnote{\href{digital-diaries.html}{Rebelsky: Digital Diaries}}
But, while the shift in diaries from ink to bits is not inherently problematic, it can alter the form of both diary forms: those meant for the author's consumption and those meant for public consumption.
Many see that public diaries turn into social media posts and blogs, while private diaries turn into password-protected blogs or word documents.

The transition of private diary from analog to digital is a straightforward process.
Instead of penning out the words on a paper, the author types the words on a document or webpage.
Because of the nature of digital transcription, future archivists will be unable to note handwriting, margin notes, struck words or images drawn.
However, a variety of benefits are gained in the switch.
For example, if the journal is stored online, the author can access it from anywhere, not only in a single book.
It is also far easier to encrypt via software than to cypher by hand.
Furthermore, storing the diary digitally allows for easier search, as modern word processors has automated search capabilities.
Finally, digital media often save metadata such as location, when words are typed, and what edits are made.
These data are not the same as handwriting and margin notes, but can still inform the future reader about the author.
While the private digital diary may lose some of the forms that readers can use to learn about the author through how they write, alternate forms of gathering knowledge that are unique to digital media arise to replace them.

The digital public diary, on the other hand, is far more contentious.
When discussing the intersection of social media and diaries, it's important to define both terms.
A diary is a work that \say{focuses on expressing the reality of a contemporaneous account of the author's passage through time.}\footnote{\href{defining-diary.html}{Rebelsky: Defining Diary}}
Meanwhile, Wikipedia defines social media as \say{interactive computer-mediated technologies that facilitate the creation and sharing of information, ideas, career interests and other forms of expression via virtual communities and networks.}\footnote{\href{https://en.wikipedia.org/wiki/Social_media}{Wikipedia: Social Media}}
Within these two definitions, we see that the two appear to have some, but not total, overlap.
Each definition contains a component crucial to understanding its distinction.
For diaries, that component is \say{contemporaneous account of the author's passage through time;} for social media, it's the word \say{interactive.}
Obviously, not all social media are contemporaneous, nor even accounts of authors' passages through time.
One social media with no account of time is Wikipedia, as its social media entry mentions.
Because of this distinction, we see that claims of social media being modern diaries are not inherently true.
That is, while social media can be diaries, it does not need to be.

However, some social media, such as \href{instagram.com}{Instagram,} do function as a type of diary where people post photos with attached captions for public viewing.
These status updates are reminiscent of the letter-diaries of James Boswell and Frances Burney, which were read aloud and shared between friends and family.
However, Instagram-as-public-diary lacks the crucial component of social media: interaction.
People cannot enter into dialogue with the author as in a fully interactive Wikipedia page.
Rather, they can comment in ways that clearly come from other entities. 
So, while diaries themselves are not social media, they can inhabit a social medium.
That is, a diary on Instagram is not a social medium, while Instagram itself is.

But this doesn't address the question of how digitizing public diaries can affect their content.
Specifically, many users of Instagram have two profiles: a public profile with information to show that they live a fulfilling life filled with culturally normative behavior, and a private profile (known as a \say{finsta}) designed to show a small group of peers that they live an exciting life that defies cultural norms.
In such a way, the writer of a public diary is now able to appear both societally acceptable and rebellious to different parties, simply by choosing which profile to post on.
Likewise, a reader is able to see two disparate views of the author.

The platform itself can also affect the nature of the content.
Due to Instagram's image-centric nature, the focus of a Instagram-hosted diary will be photographic memories of  life.
In contrast, a Wordpress-diary will focus more on textual accounts of life, due to its text-centric nature.
This diversity of platform gives an unprecedented level of choice to authors in how they record their thoughts, especially since they are able to use multiple platforms.

The change of diaries to digital format does not occur without changes to the medium itself.
But some of these changes in form can aid future readers in understanding the person they read about, just as other changes can be hindrances.
Imagine if we not only had the personal thoughts of Burney written in letters, but photos that she found important coupled with captions explaining why.
And, by making diary production digital, authors are now able to choose exactly how to best express their thoughts, which reveals more about themselves.
\end{document}