\documentclass[12pt]{article}[titlepage]
\newcommand{\say}[1]{``#1''}
\newcommand{\nsay}[1]{`#1'}
\usepackage{endnotes}
\newcommand{\1}{\={a}}
\newcommand{\2}{\={e}}
\newcommand{\3}{\={\i}}
\newcommand{\4}{\=o}
\newcommand{\5}{\=u}
\newcommand{\6}{\={A}}
\newcommand{\B}{\backslash{}}
\renewcommand{\,}{\textsuperscript{,}}
\usepackage{setspace}
\usepackage{tipa}
\usepackage{hyperref}
\begin{document}
\doublespacing
\section{\href{meta-schedules.html}{Meta-Schedules}}
First Published: 2022 February 11

\section{Draft 1}
I've written a few posts about how I'm trying to reschedule my life so that time flows how I want it.
One aspect of this that I'm trying to work more on now is the meta-scheduling aspect of it.
That is, the way of scheduling my time so that it self schedules.

The concept stretches a bit further as well.
I think the concept of activation energy in chemistry or gravitational potential energy in physics is a fair example as well.
In removing bad habits, it's better to set up your day so that they become harder to do.
As an example, if I'm trying to eat less potato chips\footnote{less because I could never count how many I eat}, I could try to just say no every time that I see them on my counter while I'm hungry.
But, the path of least resistance is to have some chips.

Conversely, if I avoid buying them, I only have to have willpower one time, rather than every time.
Building better habits is a little harder, but is moreso about reducing the activation energy, rather than increasing.

For instance, I want to play guitar more, so I leave my guitar set up by my bed so I see it every morning when I wake up.
\end{document}