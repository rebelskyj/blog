\documentclass[12pt]{article}  
\newcommand{\say}[1]{``#1''}  
\newcommand{\nsay}[1]{`#1'}  
\usepackage{endnotes}  
\newcommand{\B}{\backslash{}}  
\renewcommand{\,}{\textsuperscript{,}}  
\usepackage{setspace}  
\usepackage{tipa}  
\usepackage{hyperref}  
\begin{document}  
\doublespacing  
\section{\href{revisiting-motivation.html}{Another Look On Motivation}}  
First Published: 2025 April 26

\section{Draft 4: 26 April 2025}

Today I watched a \href{https://www.youtube.com/watch?v=iy53s5b3xkA}{really interesting video} about many things, but especially the way we dehumanize children.  
As I then looked for a posting, something in me resonated with the idea of motivation.  
In the past three drafts, I've attempted to find how I feel\footnote{Draft 1}, explore what that means\footnote{Draft 2}, and frame the argument\footnote{Draft 3}.  
Hopefully this fourth draft will be enough that I can lay down my metaphorical pen and experience life again.\footnote{I yearn for inspiration and then when it comes it binds me tighter than any chain}

When I first started thinking about autotelic motivation, I thought it was a nifty idea.  
I even reframed my daily reflections in that form: what do I want to do for itself and what do I want to do as a means to some other purpose.  
In doing so, I realize that I have accepted one of the great lies in our society: intrinsic motivation.  
  
We glorify intrinsic motivation in modern society.  
If a child is honest and says that they are competing solely for a trophy, we scold them, saying that they should play for the love of the game.  
In a sense, I have been guilty of this.  
I decry schooling being thought of as job training rather than the end in itself.

However, every time that I have stopped to think about it, I realize that education is not an end in an of itself.  
Education lets you see the world more fully, allows you to express yourself, shows you the ways that you are within the web of humanity which reaches back to our first ancestors and into the infinite void of the future.  
Every action echoes into eternity.\footnote{this is not the place for \say{fields are just consequences of information having a travel speed}, but wow do I wish it was}

By saying that an action is intrinsically motivated, we say that the goal of the action is its completion.  
Not only do we not consider what the action will effect on the world, we actively ignore the effects that our actions will have on those around us.  
Here I must confess that I come into this argument with a fundamental worldview that may not be true for others.  
I believe and try to know\footnote{see footnotes in one of the earlier drafts} that every human life is infinitely valuable.

I mean this in a very literal sense: saying that an action is preferable to another because it will result in fewer deaths still assigns a value to human life.  
Infinity times 10 is still just infinity.

Maybe not as a consequence of this, but in a deep way connected to this, I also believe that we are all intrinsically bound to one another.  
I do not think of treating my sprained ankle as selfish, but I do occasionally worry that I only do the good I do because it makes me feel better.  
We are all sparks of the Divine, and we are all intimately connected to one another deeper than anything can or could ever sever.  
Try as society might, it cannot make us forget this fact forever.

Finally, I think that human life is uniquely priceless.  
Maybe this is a part of capitalism that I have yet to unlearn, but I am willing to say that many horrors can be inflicted on animals if it will save a human's life.\footnote{drug testing, etc.}  
I don't know how much I value a cow's life, but I know that it is some finite amount.  
Anything finite divided by infinity is 0, so on some level anything that benefits any human is worth any amount of harm.

I am not a computer, though, and can see slippery slopes.  
Treating things as though they have no value inherently makes you place a value on humans.  
This is bad, so we should be careful with the world around us, etc etc.\footnote{can you tell that I've circled around this for 4900 words already?}

When we treat a motivation as intrinsic, we explicitly say that the action it causes is an end.  
When I say that each human is an end, I mean that a human's value is entirely in being a human.  
Intrinsic motivations, then, say that what we want to do is as valuable as the people around us.

We forget the web that holds us together: I may love the sound of song, but that does not mean that my neighbor does too.

We ignore the consequences our actions will have: I may love playing soccer, but my loss fundamentally means my opponents lose. If one of them was relying on the win for a scholarship, my actions negatively impact his life's trajectory.

In short, we turn humans into tools.

While I think that this is probably too in the weeds to be a fight worth having with real people in real life, I do think that I will try to discourage intrinsic motivational speak.  
We should always have an idea of what our greatest goals are, and when we take conscious action, we must be able to connect them to that great work.

Judaism is incredible for this.

For a Christian, the fact that Judaism has effectively no theology about what happens after we die is almost unthinkable.  
For a Jew, the fact that Christianity's theology is almost entirely about what happens after we die is what makes it a death cult.  
To a Christian, the good in helping our neighbor is that when we die Christ will see that we did.  
To a Jew, the good in helping our neighbor is that it makes the world more ready for the Messiah.

In both cases, the mission is to do good.  
In both cases, a faithful adherent leaves the world better than they entered.  
In the Christian case, however, the motivation is fundamentally selfish.

\say{We are all one body} is a common refrain in Catholic social teaching.  
That is, harming others ultimately harms ourselves, and helping others helps ourselves.

Believing that every human life is infinitely valuable can almost immediately become an excuse for hedonism and self-centering.  
After all, \say{I} am infinitely valuable, so why am I limiting myself in XYZ way?

If we shouldn't value human lives relative to one another, then how do you deal with the fact that there legitimately is not and probably never will be enough medicine to treat all ailments?  
Any method of treating the ill fundamentally assigns one life as more valuable, because any medicine used is gone.

I have two gut reactions to this.

First, I will never deny that we live in an imperfect world.  
In a perfect world, would someone refuse medication, knowing that it could go to someone else?  
What if everyone did that?

I really don't know.

What I do know, though, is that we could absolutely restructure society in a way that reduces the number of illnesses and ailments people face.  
We know that there are so many things which cause chronic issues, and yet we boil the oceans making better autocomplete generators.  
The issue is not, in most cases, truly a lack of supplies.  
The issue is that those with power do not, have not, and refuse to see other humans.

Second, there's the Catholic teaching of double effect.  
Is it just word play to say that the reason we do something is what's important, not necessarily the outcomes?  
Why is it ok to give a patient a lethal dose of morphine for pain but not to just do euthanasia?

I really don't know how to justify double effect at this point.  
My mind is empty from realizing how much I try to exist independently of our world, and how radical a shift it would be for even a few people to embrace our net a little more.  
However, the principle of double effect is, fundamentally, what I think I've been getting at.

Intrinsic motivation tells us that the action is all that matters.  
Catholic teaching reminds us that actions have consequences, but that we can weight them.  
Personal moral discernment tells us how to weight what.

I think that I've still gotten a little lost here, but I find myself unchained from my muse.  
Maybe that means that the words here are sufficient, that they are enough.  
Are they an ending?

No, because this writing, like all else, sends its echoes into infinity.

I think this has changed my beliefs, we'll see if it changes what I know.

\section{Draft 3: 26 April 2025}

N.B. In the past I've said that I'm going to try not framing my follies and instead simply leap into them.  
I've tried that in the past two drafts, and I think that there is absolutely merit in doing so, especially in early drafts.  
In this folly, though, I am actively attempting to make an argument, and I can only think of the argument in a narrative structure.  
With that in mind, please bear with me as I set the scene for my views on motivation.

A common idea is that society revolves around individuals giving up some sense of autonomy in order to produce a better overall social order.  
The strong man may not be able to have everything he wants, but he will not lose everything when someone invents the club.  
At its core, this idea sees humans as individuals first and community as a production of humanity.

Instead, I want to imagine society as humanity's default state.  
After all, an infant child is completely helpless.  
A comment I like to make when picking up a wandering toddler or redirecting a small child is \say{sorry little one, there are limits to your autonomy.}  
In this way, I mirror the default state of society: as we grow, we gradually shed our dependencies on others and become an individual and atomized unit.  
That is, autonomy is something that we claim for ourselves, rather than where we begin.

I am far from the first to point out that society has atomized us more than ever before.

As with most things, though, the fruits of a tree are borne only after the tree has been nurtured.  
Our society would not be so atomized if we had not laid the bricks and walls which separated us.

I've said it before and I'll say it again here: every action resounds into infinity.  
There are no final consequences, only intermediate effects.  
Even the word consequence shows society's disconnect from this idea; it is intrinsically negative to believe that there is a connection between action and result.  
People will look at me strangely if I say that I was going to give my child consequences for their behavior, only to take them out to ice cream or otherwise reward them.

No, this framing is wrong.  
HMm.

\section{Draft 2: 26 April 2025}

N.B. I find that I'm much more willing to rewrite sentences in Draft 2. Just putting that out there so I remember if I ever go here again.

In my first post \href{motivation}{on motivation}, I was mostly complaining about the judgement I received for using extrinsic motivation as my primary source of reasoning.  
At the time, I felt like there was something unfulfilling in intrinsic motivation, but could not quite place my finger on what it was.  
With the wisdom of the past six years\footnote{oof time is ever sprinting onwards}, I think that I have a few potential answers.

Fundamentally, I think that there is not just an issue with the way that society and socialization act to try to prioritize intrinsic motivation.  
The issue is not just with actions which are intrinsically motivated being seen as good.  
My issue is the world view which makes intrinsically motivated action possible to be seen as anything but evil.\footnote{how's that for a strong start? Legitimate q}

Actions can only be their own ends in a worldview where we are a sole agent.  
If I do something only for how it will benefit me, then I fall into the deepest form of solipsism, not only ignoring the way that everything I do affects others, but actively trying to disconnect myself from the consequences of actions.  
Honestly, the fact that consequence is used almost exclusively as a pejorative is itself proof of this rot.

At my core, I truly believe\footnote{intellectually} and attempt to know\footnote{in the sense of my actions being actively guided} that every other human is just as impossibly valuable as myself.\footnote{I like the idea of treating belief and knowledge opposite of the standard usage, and might start trying to do so more. Knowledge must compel action, or else it is mere trivia. Belief should motivate action, but beliefs can be contradictory. Knowledge cannot contradict Knowledge, which means my worldview and every action I take is, by very nature of being an action I take, enacted by what I know. (oof this is going to be a whole series of follies itself isn't it?)}  
People are not means to whatever end I enact, people are themselves full ends in and of themselves.  
When I reduce someone to nothing more than the instrument of my will, I am killing them, in my own mind if not reality.\footnote{something something, horrors of war where generals send kids to die}

Actions, by contrast, are not alive.  
I'm not going to get into the deep dive of what I think an action is, because ultimately my goal today is very much not to play word games.  
Each human life is infinitely and incomparably valuable: priceless.  
Even by saying \say{this will kill X lives but save X+N}, you are assigning a price to the life.  
Infinity plus infinity is the same sized infinity.

Why am I going on so much about the value of humanity in a writing nominally about motivation?

Society today, modern capitalism more generally, the underpinnings of Protestantism\footnote{oof I do have to go here, but hate that this becomes attacking the faiths I don't like} even more broadly, and even Christianity itself\footnote{there we go} to some extent fundamentally atomize the human experience.  
Going from widest to narrowest, the Church claims to teach that we are to see Christ in everyone, which is why we do good.  
And yet, that statement is based entirely on the Gospel passage where Christ says that when we die, we will be judged by how we treat those lesser than us.  
Christianity as a faith is fundamentally focused on death, and the death of the individual more than anything else.  
We cannot save any other soul, only our own, after all.

Protestantism takes this a step further.  
The Church emphasizes that all theology needs to be connected to those who came before, and that personal interpretation and experience matters far less than historic doctrine.  
Protestantism, by contrast, explicitly centers the personal and lived realities of its believers.  
Is there something far more empowering in the idea that you have your own independent relationship with the Divine?

Absolutely.

Does that same empowerment also fundamentally divorce you from your fellow human?

Absolutely.

Protestantism leads to Calvinism leads to prosperity gospel leads to modern capitalism.  
I do not know that it had to, but this is not a \say{for want of a horseshoe} story.  
It did, and we can see the effects of this everywhere we look.

In modern capitalism, a common critique is that everything has value entirely based on what monetary worth can be assigned to it.  
That is not actually entirely accurate though.  
After all, who assigns worth?  
What is money?

Again, I'm not trying to fall into wordplay.\footnote{this is a reminder for myself to pull the reins back on the charging horse that is my thoughts right now}

People will speak of \say{the market} or \say{consensus} as though either of these exist in actuality.  
There are not markets like there are humans.  
There is not consensus without humanity.

If everything has value only in monetary worth, then we must profane the sacred.  
We must say that there is a level of financial hardship past which it is better to let a child die.  
There are the countless stories of companies accepting that a number of people will die, knowing that the cost of a recall is greater than the expected value of payouts they will need to give to their grieving families.  
At a deeper level, though, it infects all discourse.

It is not worth pumping money into a child whose life expectancy measures in weeks, because those resources could be better spent elsewhere.  
There are a limited number of donor organs, so those who are less likely to abuse their bodies deserve what few there are more.  
We pump billions of dollars into machine learning, actively burning the earth and sea.  
People will argue against this, saying \say{it's cheaper to just hire human labor.}

That is, everything in modern society is framed in costs and benefits.  
Society lionizes those who do whatever it takes to succeed.  
I alone matter.

In a sea of endless hordes, I alone have the spark of the Divine.

How does this relate to intrinsic motivation?\footnote{oof this is winding. Three drafter day for sure}

Intrinsic motivation tells us that our actions are their own end.  
I want to be clear, this is not me making an argument right now, this is just me stating what the term means.  
When we make actions an end, we make humans, the only true ends, means.  
This is my argument.

When I went to the wikipedia page for motivation, I was shown an image of two soccer players.  
One is thinking about all the things that winning the game might bring him, and the other is simply focused on the love of the sport.  
No one will support the first person, and rightly so.  
Personal glory is fundamentally a hollow and empty motivation.

The second, however, is doing something far more insidious.

\say{I want to win this so that the cute person will notice me} still gives the cute person a sense of agency.  
\say{I want to win this so that I get the trophy} assumes, on some level, that an external agent determines what it means to win and values that.

\say{I want to play because I enjoy soccer,} on the other hand, completely ignores everyone around.  
Soccer is a team sport; it requires coaches and referees, teammates and opponents, space and someone to maintain it, equipment and people to make it.  
Playing the sport for its own sake is saying, on a fundamental level, that every single one of those people and objects exists solely for your pleasure.

When I play a song for the beauty of the music of the moment, I ignore my connection to the rest of humanity.  
The neighbor who doesn't want to hear my music pumping, the energy that my speakers use, the disconnect of listening to a song alone, rather than live with my family: all of these are consequences that intrinsic actions ignore.

Rereading these last two paragraphs, even I find myself bristling slightly.  
Just because an end is directed, doesn't mean that there are no other considerations.  
However, I ask: what value does our enjoyment have?

I am not asking as an economist, who might find the exact dollar amount, pain you would endure, or harm you would be willing to inflict as a function of some arbitrary metric.  
I ask legitimately.

I said it at the beginning of this draft, and I will reiterate it here: every human life is infinitely precious and priceless.

Anything divided by infinity is 0.  
Any time that we assign value to something in a way that gives humans a value, we are explicitly saying that the humans are not priceless.  
The value of anything compared to a human life is nothing.

Here one might make a very fair point: we do not live in a vacuum.  
If I use all of my antibiotics on a chronically ill elderly man, then we will not have that medicine for the sick child.  
I agree, but even that framing belies the issue: we see ourselves as fundamentally separate.

Humans have a value intrinsic to our very nature.  
Humans also, though, exist as a social creature.  
In a very real sense, there is no \say{you} and \say{me}.

Many philosophical traditions find their way to this truth.  
The Church has the idea that \say{all sin is social sin because we are one body.}  
Hinduism has the belief in karma: all actions resound into infinity.  
Buddhism teaches that the idea of \say{I} as a distinct entity is fundamentally foolish.

Even common sense teaches us this, when we stop to think.

We would condemn someone who drove through a children's soccer game because they wanted to get to the parking lot across the field.  
I've seen many argue that this is our willingness to give up some of our own autonomy in order to enact order.  
I'd argue it's the opposite.  
We only claim autonomy, not have it on its own.

A baby is dependent, utterly and totally, on those around it.  
Is that a better way for me to frame the argument, maybe?

Start with \say{there is no autonomy}, then go to \say{every action is, by its very nature, consequential}, which brings us to \say{not considering consequences is choosing to dehumanize}?  
That seems reasonable, on to draft 3.

\section{Draft 1: 26 April 2025}

One of the initial goals of this site was that, by having noted drafts with dates, I would be able to revisit old musings\footnote{as I thought of them at the time} and add new drafts as my life changed or my views did.  
I'm still not sure how I feel about that concept, though I think that at the very least, the reminder that most of the time what I post here is a raw and unedited first draft serves me well when I want to cringe at the writing I do.\footnote{remember, don't kill the cringe, kill the part of you that cringes}

Still, even if I was going to write new drafts of topics in the same url as the old, I don't know if I would put this folly on top of \href{motivation.html}{the old one}.  
In that post, I focused\footnote{in what little I can call the few rambling words a focus} on my own internal versus external loci for motivation and why I thought that it was fine for me to have a mainly externalized locus of motivation.  
My goal today is somewhat different.

In one of the early posts of this iteration of the site\footnote{read: basically any time that I take more than a week or two off}, I made comments about motivation.  
Why do we do things, and what not.  
I got really into the idea of autotelic motivation, doing things as their own ends.

And, in general, I find that I'm thinking more and more about the ways that we as a society really do treat everything as a means and nothing as an end.  
I don't want that to be the case, and I'm finding myself aligning more and more with Catholic social morality\footnote{not that this is a bad thing, just that it's going from \say{I trust the people who formulated it and it seems vaguely good} to \say{wow I think that even someone who absolutely despised the Church should agree with this stuff}}, which constantly rails against this treatment.

However, the Church does, in many ways, still direct our actions to have a purpose outside of themselves.  
There's an image in C.S. Lewis's \say{The Great Divorce} that struck me the first time I read it and every time since.  
An artist is being given the chance to enter heaven, he just needs to remember the fact that his art started as a way to glorify G-d and only then became about the love of the art itself.  
Every time I think about it, I find a deep part of me recoiling from this thought.

I think part of it might be the Jewish morality that I've inherited.  
While the Christian goal is to bring people to heaven, the Jewish social mission is to make the world as heaven  
Catholics try to get to the Messiah, which Jews try to make a world fit for a Messiah.  
And so, the beauty of color and how it interfaces with light is good in an of itself.

This is a number of words to say that I don't really know how I feel about motivation.  
So much of what I do has the clear end that it is a means for.  
I play guitar so that I can be ready for a friend's wedding.  
I do try to do kindnesses in order to align myself more with the Divine Will.

And yet, as a friend's mother pointed out recently, there's something inherently selfish in her motivation to make people smile.  
She feels better when the people around her are happy,\footnote{as I would hope most everyone does} and so doing kindnesses benefits herself.  
Part of this disconnect has to do with the atomization of society, I more and more realize.  
When we see ourselves as independent agents, then helping someone else feels lessened if we benefit.  
If, instead, we see the web of interconnectedness and mutual obligation that we share with one another, then the benefit is in fact part of the help.  
I take antibiotics when I have infections, and I feed the hungry where they are.\footnote{aspirationally in both cases, of course}

I'm wondering if this atomization idea can help me think about my own motivation.  
What does it mean to do something for its own sake?

I realize that I might just have a philosophical and epistemological framework which precludes autotelic motivation from being a thing.  
Platonists certainly wouldn't believe in it, and while I don't know how much I agree with the idea of a hidden realm of forms, I do still think that we can reflect deeper truths with shallow works.

Instrumental versus Intrinsic value was coined to describe a sociologist's way of separating people's reasons for doing good.  
To him,\footnote{based on my understanding of reading a single paragraph of a Wikipedia article} and many thinkers after, intrinsic motivation is fundamentally ridiculous as a concept.  
Trying to divorce an action from its consequences is not just an effort in absurdism, it is itself wrong.

Why, then, do we spend so much energy as a society trying to convince especially our youth about the inherent superiority of judging actions as being their own ends?

This could quickly spiral into a diatribe about how capitalism as it exists now is fundamentally incapable of thinking about the future, and so everything's value is its value at this exact moment.  
It could also spiral into a commentary on the fact that treating things as their own ends closes us off from one another, reducing anyone we are with to mere instruments.  
Soccer, a prime example of what we are told to enjoy for its own sake, requires teammates and opponents, referees and coaches.  
If I do it solely for enjoyment, then the value of those around me is solely in how they can help me reach my aims.

Is society really so transactional that it has taken me until now to realize the fact that intrinsic motivation is itself a form of viewing the rest of the world as means rather than ends?

More so, why do I take issues with actions being ends but not people?

Well, as soon as I write that, I see the difference.  
I do fundamentally believe in the inherent value of all human life.  
A person has value by virtue of nothing more than the word vir applying.\footnote{no I am not being sexist here, just that hominue is not in the common parlance and I wanted to make the pun}  
Actions, on the other hand, have value by how they affect the world around us.

Cool, let's redraft this now that we know where we're going.

\section{Daily Notes}

\begin{itemize}

\item Obligations:

\begin{itemize}

\item Professional

\begin{itemize}

\item Leave work before 1900

Not yesterday, but I wasn't working at 1900, so I'm going to say that's ok

\item Write the thesis

Not anywhere near as much as I would have liked to have been, but I think that taking the breaks my mind and body demand is probably good

\item Revise the thesis

\item Edit the thesis

\item Research for the thesis

\item Read the books that might be useful for the thesis

\item Start citation tracking

\end{itemize}

\item Personal

\begin{itemize}

\item Learn the songs for to jam

\end{itemize}

\item Self:

\begin{itemize}

\item Silence

Generally decently, I think. Then again, I have gotten most of the way through the month long backlog of videos that I had wanted to consume, so maybe not so much.

\item Typing practice.

There's a tower defense game on Steam that is typing based, and I can definitely feel the way that all the typing I've been doing helped with there. Still, it's not the same  level of intense focus as actually focusing on a single letter.  
It was, however, much more fun, and I found myself working on typing for an hour.  
With that in mind, I'm going to try one or two more.

\item Keep the phone out of the room for bed

Nope!

\item Pray St. Michael Chaplet in the morning

...

\item Stretch in the morning

Not really at all this week, which is not great, especially since I also haven't really been stretching at night either.

\item Read at night

Not really, I read about half a book on Wednesday night, and then have been really tired upon bed time the past two nights.

\item Poetry at night

A few times! I feel like I need to systematize it again, because that's the only way that I've ever been able to stick to a routine. I think a sonnet a day can once again be fun, because they do kind of just flow off after a while.

\item Clean the home

Spent a little over an hour this morning, and that was great, and my home is markedly cleaner as a result.  
It is still far from what I'd like, but I'm doing everything in my power to not let perfect become the enemy of good.

\item Stretching, standing, drinking water

Not a ton for stretching or standing.  
Not enough for drinking water, which I don't love either.  
I have a water bottle sitting next to me, though, so this will help me to drink a little more.

\item Posture

I think generally decent.  
I still slouch far too much when sitting, but standing seems easier by the day.  
Of course, it is made leagues and fathoms\footnote{I don't know why old measurements are speaking to me right now but} easier when I am stretching appropriately.

\item No wasted time

I generally think that I'm doing ok here.  
As I see more and more often, though, I need to remember that I cannot optimize my life fully.  
I spent about half of yesterday playing a game while getting through the youtube backlog\footnote{yes, I'm aware there's something intrinsically not sane about listening to audio at 3x speed while playing a fast paced action game.}, but I was also feeling so worn down that it might not have been helpful to work harder.

\item Eat more than 2 meals a day

I think so?  
I have vague memories of eating food the past few days, even if that might be all that I have.

\end{itemize}

\end{itemize}

\item Goals and Growth:

\begin{itemize}

\item Ends:

\begin{itemize}

\item Letter writing, get into more

Delivered the one letter, second one I don't think has been received.

Also received a letter from a friend! That was really exciting.

\item Handwriting, pick and make the new one

Something about lower case letters in print really amplifies any imperfections in my writing, which I find really interesting.

\end{itemize}

\item Means:

\begin{itemize}

\item Typing speed, improve it.

I will now go to do my five minutes of assigned typing to see what the speed is today.  
All letters were above 4 per second which is kind of wild to me. Now the goal is 4.5.

I think that once I get all of the letters above 5 characters per second\footnote{I forget what the conversion from cps to wpm is, but let's double check.}

Oh gosh, my top is only 90 words per minute, and my average is around 60.  
I don't know why that feels horribly slow, except that I know so many people with over 100 wpm times.  
Then again, they tend to do them in more natural settings, where words flow from each other in some sensible manner.  
I'm happy with the progress, but going to keep it in CPS for the future.

Since they define one word as five characters\footnote{which is apparently standardized}, only when all of my letters are at or above 60 wpm\footnote{which feels weird, rifght??} will I start to worry about capitalization and punctuation.

\item Reading, do more of it

I got an ARC of a book that seems interesting, and have otherwise been trying to finish the audiobook I'm listening to.

\item Blogging, do it

Eh, haven't been great about this, which is probably a sign, at least somewhat, of the fact that I have been nearing burnout.  
Then again, the fact that I've also been struggling to come up with my daily list of five things also could have told me that.

\item Writing things that are not the blog and thesis, do

I wrote the short story for my web novel for the week,\footnote{I forget if I mentioned that here, but I'm trying to get back into it, but slowly, and so edging in by setting vignettes in the broader world} and it was generally well received.

\item Guitar, play it.

Not a ton, but I am at least strumming it most mornings and nights, which is still progress of a sort!

\end{itemize}

\end{itemize}

\end{itemize}

\end{document}