\documentclass[12pt]{article}  
\newcommand{\say}[1]{``#1''}  
\newcommand{\nsay}[1]{`#1'}  
\usepackage{endnotes}  
\newcommand{\B}{\backslash{}}  
\renewcommand{\,}{\textsuperscript{,}}  
\usepackage{setspace}   
\usepackage{tipa}  
\usepackage{hyperref}  
\begin{document}  
\doublespacing  
\section{\href{flection-november-25.html}{Monthly Flection}}  
First Published: 2025 November 10

\section{Draft 3: 10 November 2025}

I didn't love quite how messy the last draft was in the context of this nominally being a flection.\footnote{not a reflection because not pointing back}  
With that in mind, let's think about this month.  
My main goals are to become a functional adult.

August taught me to finish my degree.  
September taught me to stop being a grad student.  
October taught me to be a salary man.  
Now November lets me become my best self.

In general, the big goals I have for the month are a clean home and a song for the upcoming wedding.  
Next highest priority is the pattern and starting on the embroidery project, along with writing here.  
Following that, we have a decent sleep schedule and reading the\footnote{admittedly excessive number of} library books I checked out.

November is often a tough month, but here's hoping this one isn't.

\section{Draft 2: 10 November 2025}

Well, I missed the chance to post this as a true proflection, and so instead we have a flection.  
Month is not quite over, but we're about half way through it.  
So, what do I want to make of the month?

I really do want to write here more.  
I don't entirely know what the barrier is\footnote{clearly, as I am currently writing}

Unlike the previous draft, though, I'm not as interested here in exploring exactly how much time I can and will allocate to different tasks.  
Instead, I hope to use this as a way of figuring out what's important to me right now, and what I can and should be trying to push myself towards.

Clean home.  
Yesterday a friend expressed that they want to see my home before they finish graduate school.  
That should be a completely reasonable request, and so one goal I have is to have a home which is clean enough that I can have others visit.\footnote{of course, there's the secondary issue that I don't have seating in my home, but that's solvable by buying some cushions.}  
More than that, I like knowing where things are, I like having a home that doesn't cause me visual or mental pain when I look at it.  
I just recently got a new shelving unit, and I do really truly believe that the biggest issue I had was a lack of storage space.  
Of course, if I had too little storage space, that does also imply that I have too much stuff.  
Nevertheless.

Exercise.  
I recently signed up for a pilates gym.  
Both to get use out of my spending and because exercise is good for me, I want to go to that more.  
I like Pilates because it's done laying down a lot of the time.  
For whatever reason, the workout feels much better when I don't have to be standing or sitting up.

Music.  
I don't entirely know what I'm thinking here, but I guess there is a very limited amount of time left to write the piece for a friend's wedding.  
I should probably start that tomorrow.\footnote{because tonight is for this flection}  
Outside of that, I've been listening to a lot of music, and would like to keep doing that.

Celtic Knotwork.  
As I was thinking about how busy I feel despite having objectively more free time, I thought about how, while visiting college friends during my train trip, some of them asked if I was still drawing Celtic knots.  
They're certainly an activity that i can do for ages, as evidenced by my constantly getting lost in them.  
However, a dear friend is also graduating soon, and so I would like to make an embroidered piece for them.  
This requires crafting a pattern first, which right now means both drawing out the knot and figuring out a way to have breaks that I like.\footnote{in the previous Celtic embroidery, half the breaks look really pretty and the other half look ugggggly, at least to me}

Reading.  
I have checked out a number of books from the local library, and so I would very much like to read them so I can return them.  
That's something that leads well into the next point.

Less screen time.  
I don't like staring at my screen.  
I particularly don't like staring at my screen and feeling like I have nothing going on in it.  
Reading is fun, and so is experiencing\footnote{most of} the world around me.

There are countless other things, but now let's play the role of devil's advocate.

In order for me to do more things, I have to give up others.

I want to spend more time with friends, and that's time that I can't be writing here.

Oh!

Writing.

I realize that right now, my web novel feels less like something I want to do and more and more like a weight around my neck.  
I've had some ideas for more things I could write, but have felt like I can't write them because I should instead be writing this web novel.  
Given that it's something that I am doing entirely for fun, that's nonsense.  
I can and should write other things.

In particular, I want to try some flash fiction, poetry again, and ideally some sort of novel thing.

I want to sleep better.

Anyways, rather than continuing to make a wishlist, let's think about ways to make the schedule more doable for me.

First, I've learned that there is no point to me getting to work before about 710.  
I really enjoy starting my morning with a pastry, and they don't get put out until then.\footnote{no, I will not be accepting comments about not getting a pastry. They're delicious and brighten my morning}  
Since the Pilates morning class takes place at 6am, I need to be awake in time to reach that.  
It's generally better, so I'm told, to have a self-similar sleep schedule.  
With that in mind, waking at half five feels like the new goal.

On days without Pilates\footnote{read, at least two or three workdays and probably both weekends}, that leaves me with an hour and change before I can go to work.  
That time is easily and well spent cleaning, reading, writing, or drawing.  
Realistically, I think that it's probably best spent cleaning or reading because the drawing, embroidering, and writing can all suck my mind into them.

On days with Pilates, that's obviously not going to happen.  
However, getting my body moving during what's\footnote{with any luck} going to be a cold winter is definitely worth more than that.

As a consequence of getting to work earlier, I can then leave work earlier.  
That means that I have the time when others might be commuting home to instead work on the things which are important to me, such as writing this here site.  
Before bed is a great time for analog activities, such as embroidery and reading, or writing by hand, which is something that I'm enjoying.

What is standing in my way of getting my life on track like this?

I am eepy in the morning.\footnote{for the unknowing: eepy is slang for tired. I think it's a shortening of sleepy}  
This both makes it hard to get up and makes me feel less well rested.  
Solution: going to bed earlier doesn't really seem to help, because then I just wake up more in the night.  
I'm hoping that moving my electronics back out of the bedroom at night might help, and at worst may consider investing in an alarm clock that I keep outside of arm's reach.

I'm weary when I come home from work.  
If I schedule meditation time right away, or some other activity that gives me energy, then that's not as much of a problem, at least as I'm seeing right now.

I don't entirely know what a clean home means.  
That's probably not true, but it's a convenient fiction for myself, right now at least.  
Realistically, I only need my home to be clean to the point that others will see it.  
Is that doable?  
Maybe.

I don't know what to write.  
I think that might actually be one of the barriers, but realistically that's nonsense.  
I've never once typed more than a hundred words and not had more to write.\footnote{any previous evidence to the contrary is untrue, I've decided}

It's easier to not do things than to do them.  
Steam games are really easy.  
For now, I think that does mean that I should uninstall steam and all its games from my computer.  
That is no longer an easy thing to do.  
I will, once the embroidery pattern is made, leave the embroidery out.  
That will also make it easier to do.

Part of me is considering leaving the writing site I use open as a rule when I close my computer.  
That will certainly make it easier to use, but the only question is if I'll then just start getting rid of it immediately.  
Who can say?

Anyways, since spending time with friends is important to me, I'm off to do that.

\section{Draft 1: 2 November 2025 (working title: Monthly Proflection)}

Unlike most months, where my first post of the month is a reflection on the previous month, I figure that this month might better serve me as a forward facing.  
That is, rather than reflect where I was, I want to proflect\footnote{yes, I am taking advantage of my doctoral right to coin new terms} on where I want to be at the next reflection.  
Light takes time to travel between a mirror and my eyes, and my own goals can too.

So, what all feels meaningful to me when I think on the month?

This month I want to be better about writing.  
I have spent this past weekend unsure of how I feel, other than bad.  
In large part, that was only possible because I let myself get away from writing.  
Where can I make time for writing in the every day?

There are twenty four hours in a day.

I need to sleep at least eight.  
There are sixteen hours in each day.

In general, I'm told I should be working at least nine.  
That leaves seven hours.

If I drive to work, it's about a ninety minute round trip, leaving five and a half.  
If I bus, instead, it's about three total hours, but I do also get about two hours of reading time, leaving fourish.

I want to start doing cardio every day.  
Given that I should not do cardio just before bed\footnote{according to the sleep people, at least}, that probably means that it's a right after work activity.  
If I say half an hour for that, that leaves me with five or three and a half.

Reading is important to me, and I'd like to spend at least an hour a day reading.  
That is, five or still three and a half.

What else do I want to do?

This month I want to find what the minimal amount of clutter or mess or chaos is sustainable in my own home.  
I have every faith that it's much cleaner than I currently think, even if I don't entirely know where, exactly, the line is.

I want to ask myself what things that I own do I actually want to keep.  
There are a lot of things in my home which I have solely because it feels somehow wrong to dispose of them.  
That is far from a good reason to keep a cord for a device I no longer own, and so I'm hoping to start spending time thinking about that.

I want to do NaNoWriMo, even if I'm not going to participate in the official version or even necessarily write a novel.  
I do, in theory, at least, want to get back into the web novel, and this is a good month to start doing the writing for it.\footnote{side note, I have no idea why, but right now my fingers really feel like they're flying across the keys.}  
Otherwise, I want to be hand writing in my journal, and I want to be doing these sorts of things on my site.

If I leave for the day at 645, I can get to work by 710 or so.  
If I want to spend twenty minutes doing hand writing before that, I need to start writing by 625.  
Since it takes me at least tenish minutes to get ready, that means that I should try setting my alarm for six fifteen tonight.  
It's not quite 2100, and I did take a nap this afternoon, so that should be more than doable.

There is a big voice in my own head that thinks that I spend too much time in bed.  
That is, I think that if I gave myself less time to sleep, I would potentially not only sleep better, but feel more rested upon waking.  
Whether or not that's true, it is something I want to try to explore this month.

I've also just today finished\footnote{and more or less started} a book about how to organize my life given my neurotype.\footnote{or, at least, given a neurotype I could have been assigned at the time that the book was written}  
What I took from the book is that I should not be afraid to ask for help and that if there are services which can take care of tasks for me, I should follow through on doing them.  
Right now I feel like I'm generally eating ok, so probably not going to do grocery deliveries.

I don't like the idea of a stranger in my home, so for now at least I'm going to try to have my home organized.  
I do think that part of the issue is a lack of storage, and so I have made steps towards fixing that.  
I do also need to get rid of possessions, which November is a great month for.

Eight hundred words later, I think that this is a good place to call it for the night.  
Time to prep a space for me to write tomorrow morning, and then hopefully force myself into the task!

Current Pen List\footnote{for my own posterity, mostly}

\begin{itemize}  
\item Hongdian Black with Fude Nib: Monteverde Ocean Noir. 10/6  
\item Jinhao Shark: Diplomat Sepia Black. 10/6  
\item Pilot Preppy: Diamine Bilberry. 10/6  
\item Shaeffer: Private Reserve Ebony Green. 10/6  
\item Diplomat: Diplomat Caramel. 10/6  
\item Kaweko: Stipela Sepia. 10/6  
\item Monteverde: Diplomat Burgundy. 10/6

\end{itemize}

\end{document}