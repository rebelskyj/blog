\documentclass[12pt]{article}[titlepage]
\newcommand{\say}[1]{``#1''}
\newcommand{\nsay}[1]{`#1'}
\usepackage{endnotes}
\newcommand{\1}{\={a}}
\newcommand{\2}{\={e}}
\newcommand{\3}{\={\i}}
\newcommand{\4}{\=o}
\newcommand{\5}{\=u}
\newcommand{\6}{\={A}}
\newcommand{\B}{\backslash{}}
\renewcommand{\,}{\textsuperscript{,}}
\usepackage{setspace}
\usepackage{tipa}
\usepackage{hyperref}
\begin{document}
\doublespacing
\section{\href{timelapse.html}{Timelapses}}
First Published: 2019 January 11
\section{Draft 1}
In a class I took last semester, Creative Cartography, one of the ways we experimented with making maps was via timelapses.
I very much enjoyed the process, and repeated it later in the making of the growing of my beard.\footnote{referenced \href{growing-the-beard.html}{here}.}
When I shaved my beard, I again tried a timelapse.
And, recently, I tried doing so while working on a doodle that I enjoyed.

It's a really fun process, and it's a good way for me personally to do activities more mindfully.
If you haven't tried making them in the past, I'd recommend trying to make one, as it's very rewarding. 
\end{document}