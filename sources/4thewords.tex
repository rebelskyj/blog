\documentclass[12pt]{article}[titlepage]
\newcommand{\say}[1]{``#1''}
\newcommand{\nsay}[1]{`#1'}
\usepackage{endnotes}
\newcommand{\B}{\backslash{}}
\renewcommand{\,}{\textsuperscript{,}}
\usepackage{setspace}
\usepackage{tipa}
\usepackage{hyperref}
\begin{document}
\doublespacing
\section{\href{4thewords.html}{4thewords Review}}
First Published: 2023 December 21

\section{Draft 2}
I've talked a fair amount in these musings about the new site I'm using for writing.
As I've mentioned on more than a few occasions, I often have trouble motivating myself to write.\footnote{I don't know if that's quite the right statement, but it's more or less true}
You may have also noticed recently that I've more or less been able to put out a post basically every day.
There's a few reasons for that.

I've been writing every morning with a friend, for one.
I found an internal reason to write daily, for another.
Both of those, however, are why I am able to write a blog post every day, not how.
Today, I'd really like to spend some time discussing how I've been able to write so much more.

Since apparently mid august, I've been using a website called 4thewords.\footnote{I hate that there aren't any spaces in the title, but I really haven't seen it written with a space anywhere}
I guess that knowing I've been using it since August doesn't exactly lend credence to the idea that it helps me write.
However, it is working now.

So, what is 4thewords, and how does it motivate me to write more?

4thewords is a few things bundled into one.
It is a fairly lightweight word processor\footnote{as someone who writes almost exclusively uncompiled LaTeX (and soon Markdown), that's much more of a pro athan anything else}.
There's a fairly active and incredibly positive forum.
And, its real selling point, 4thewords is also an RPG.\footnote{role playing game (not to be confused with TTRPG, where the R stands for roll (not really but I wish))}

That is, you play as a character exploring through the world, completing quests and getting rewards.
Like many RPG\footnote{I never know how to pluralize initialisms, because games starts with G.}, 4thewords also lets you decorate your character, both with stat boosting and completely cosmetic items.
In what I think is a really brilliant choice, they have completely separated the two.
That is, you can choose to equip the best equipment for statistics without compromising on your artistic vision.

At this point, you might be wondering how they manage to combine a word processor with an RPG.
Most quests are completed by fighting monsters.
Monsters are fought by writing words against a time limit.
Unlike in a lot of RPG, there is generally no way for the monsters to strike back.

Most of the monsters require a certain number of words written within a given number of minutes.
They tend not to require a pace faster than ten words per minute, which is nice when doing something slower than free writing.
Even when I'm at my most type a few words and delete, I still average more than ten words typed a minute.
There are a few monsters which are called endurance monsters.
Rather than requiring a certain number of words, it requires a constant clip.
I generally dislike them, because I'm more motivated by \say{here's the number of words you have left} than \say{this is the amount of time you have left.}

However, fighting does not an RPG make.
Another important aspect of an RPG is items.
Most RPG have both decorative and functional items.
4thewords is no different.

Unlike more RPG, however, the form and function are completely separated.
You can choose to equip whatever items give you the stats you want along with whatever items give you the appearance you want.
Both sets of items are purchased in the in game stores for mostly in game available items, which you can get by fighting monsters and finishing quests.

Functional items feel a little strange in a game like this.
It seems odd that a game where the only combat is writing words within a timeline would have items that affect the gameplay.
And, thankfully, there are only three stats: attack, defense, and luck.

Attack follows a formula, where the higher your attack, the fewer words it takes to take down a monster.
It doesn't change the stated numbers, but two words might count for three.
Defense increases the amount of time that you have to fight a monster.
Unlike attack, it does change the number of seconds that you have to write.
A second still lasts exactly\footnote{i assume, I don't know if they've made it exact, but the ticks feel like around a second}

Finally, there is luck, which nominally affects the drops that a monster gives.
There's some variance to what monsters can drop, though there is usually a floor and a ceiling, and the two numbers are not always different.

So, why do I feel like this app helps me write?

I think that there are a few reasons.

First and foremost, I like the fact that there are a series of small goals that I can set to automatically start.
Writing 500 words is often a reasonable amount for me at once.
Especially if I'm writing quickly, I can generally get through one in about ten minutes.
Small monsters, with around 100 words to complete, give an almost constant source of dopamine. 

After writing two paragraphs, I tend to take a break for a second or so, because my mind needs a little bit of time to process what I said.
Four paragraphs is almost always at least 100 words, which means that after I write them, I get another monster.

The time also helps.
Without deadlines, I find that I procrastinate almost indefinitely.
Every so often, I'll look and see that there's only two minutes left to write 100 words, or twenty minutes left to write 1000.
When that happens, I put my nose to the metphorical grindstone and start working as hard as I can to get the words out.

Those two facts do a lot to help motivate me to write.
However, there are also a lot of small little features on the site that make it more friendly for me.
The overall design is fairly unobtrusive, which means that I'm not horribly distracted while I write.
However, the overall design is also very bright, which makes it fun, and I get pretty pictures to cheer me up.

Also, the site is explicit that their goal is helping users to meet their own writing goals.
Unlike some sites, which track incredibly hard to make sure that no one is doing anything that could even begin to think of cheating.
Even though there are some community aspects to the game, it's almost exclusively single player.
Because of that, the developers are very clear that you should use the site however works best for you.

For some people, that means copy pasting a few hundred words of lorem ipsum after they've done some editing or chores.
Personally, I tend not to copy paste, though I've started thinking of situations where I feel better about it.\footnote{mostly in situations where I hand write a letter to a friend, or writing out a derivation. I still feel a little icky about it, but whatever happens happens, I suppose}
There are also almost no negative consequences in the game.
Losing a monster doesn't affect you in any way except for a single tally somewhere fairly hidden in the game.\footnote{even that can be removed with an item that becomes very easy to gather by the mid to late game}
I find that negative consequences can make me want to avoid a situation, and so the absence makes it nice.

And, finally, the game is in active development!
I don't mean that in the way of \say{the site is full of bugs and major gameplay changes happen daily}, but in the \say{the developers clearly care about they're doing, and they add lots of new and exciting seasonal events.}
Right now, we're in the middle of the winter wonderland, which has a few time limited monsters and some special snow themed quests.
Just before that, however, was a really exciting quest series for NaNoWriMo!

Anyways, if this inspires you to try the application\footnote{I'm trying to stop saying app unless it's for mobile, since that was once the difference} at all, please reach out.

Daily Reflection:
\begin{itemize}
\item Hobbies:
\begin{itemize}
\item Did I embroider today? Shoot! Tomorrow for sure.
\item Did I play guitar today? Shoot! Tomorrow or tonight for sure.
\item Did I practice touch typing today? I made it past c for a second, and then got back to it. I think that right now my struggle letters are r, which I want to type with my ring finger, y, which I want to type with my left hand, and c, which I want to type with my index finger.
Since I haven't gotten all of the letters unlocked, it's more than likely that other letters (like p) will be an issue.
\end{itemize}
\item Reading
\begin{itemize}
\item Have I made progress on my Currently Reading Shelf? I finished the second book with a friend and started the third. However, I did read both of the books that I wanted to finish this month, I think.
\item Did I read the book on craft? I have it, and may remember to do tonight, will keep posted.
\item Have I read the library books? Still shoot. I have them with me, though, so may read tomorrow.
\end{itemize}
\item Writing
\begin{itemize}
\item Did I write a sonnet? I did not write one two days ago, and yesterday's was mediocre. Hopefully I remember to write today's before I'm too tired for it.
\item Did I revise a sonnet? Somehow when I'm tired I write sonnets thoughtlessly. That's probably because I don't care if the rhyme feels forced or if the story makes no sense.
\item Did I blog? Hooray!
\item Did I write ahead on Jeb? I'm about one third through tomorrow's chapter, so will try to do more in the next few hours.
\item Letter to friends? I played real life pathfinder with some friends\footnote{could have mused on that, I suppose}, and I got coffee with a dear friend!
\item Paper? I was grateful not to be an experimentalist today.
\end{itemize}
\item Wellness
\begin{itemize}
\item How well did I pray? Yeah oof remains accurate.
\item Did I spend my time well? Once again with family! Today also with friends, and that's great.
\item Did I stretch? Shoot.
\item Did I exercise? Darn
\item Water? I consumed more liquids today, and for that I am grateful
\end{itemize}
\end{itemize}

\section{Draft 1}
I know that I've talked a fair amount in my daily reflections about the site that I've started using.
Since I don't know if I've explicitly named the site, I use 4thewords, which is a site that tries to gamify writing.
The central concept of the game is that you choose to fight monsters, which come with a set number of words and time to defeat them.
By and large, this is not terribly difficult.
Most of the monsters require around 10 words per minute in order to defeat them.

Of course, like all good RPG\footnote{role playing games}, there are stat boosting items.
In the case of 4thewords, there are three stats: attack, defense, and luck.
I've listed them in that order in large part because that's the order that they're relevant to me.
Attack means that each word you write has a greater effect on the monster you're fighting\footnote{though, thankfully, not anywhere else that it actually tracks wordcount}.
The site gives vague approximations, but the fan site gives the explicit formula.

The words it takes to kill a monster are its default words times one hundred divided by one hundred plus your attack.
So, one hundred attack means that it takes half the words to defeat a monster that it claims it should. 
The part of me that remembers math tells me that n divided by n plus x shrinks very slowly. I think they call it exponential decay.
That makes sense, because in order to halve the words again, you would need three hundred attack, and then seven hundred, and so on.

Defense, unsurprisingly, does a similar thing.
Defense increases the amount of time that you have to fight the monster.
Although it will still say the same amount of time when its waiting in the queue,\footnote{you're allowed to prequeue ten monsters, though that number may increase} when the monster fight starts, the timer will show a greater amount of time.
I prefer that to seconds not ticking down at the same rate, if only because I like to be able to use the timer as an egg timer of sorts.\footnote{let me tell you, I'll write until the pasta finishes boiling gets a shocking amount of words finished}
There's a similar equation somewhere, though I have not looked it up.\footnote{unsurprisingly, if you know anything about me, the only think that I care about is damage output, because it means that I can get through monsters fast.
All that defense gives me is an excuse to procrastinate}

Finally, we have luck.
As this is an RPG, monsters drop loot when they die.
Luck affects the amount of loot that is dropped, though last time I checked, it was less clear exactly what the mechanism is for that.
Unlike with attack and defense, there is an optimal value for luck.\footnote{I cannot remember whether going over that just stops counting or whether it starts harming, but I do actually think that it's the latter.}

Now, I've been saying monster in the generic term.
There are a lot of unique, hand drawn monsters.\footnote{I am almost positive it says that the fan site says new monsters are hand drawn. The aesthetic on each monster does absolutely show a sense of care, and the fact that none of them are animated makes it feel even more likely}
One question that you might have is which monster to kill.

I'm actually realizing that I'm falling way too far into the weeds. Let's restart, this time focusing more on the fact that I want this to be a review.


\end{document}