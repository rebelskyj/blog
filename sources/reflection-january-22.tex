\documentclass[12pt]{article}[titlepage]
\newcommand{\say}[1]{``#1''}
\newcommand{\nsay}[1]{`#1'}
\usepackage{endnotes}
\newcommand{\B}{\backslash{}}
\renewcommand{\,}{\textsuperscript{,}}
\usepackage{setspace}
\usepackage{tipa}
\usepackage{hyperref}
\begin{document}
\doublespacing
\section{\href{reflection-january-22.html}{Monthly Reflection}}
First Published: 2022 January 31

\section{Draft 1}
I really thought that I had a monthly tradition of reflecting on the past month.
Looking through my history, however, I apparently only did this \href{reflection-january-19}{once}, which makes it somewhat funny that it was also a January reflection.

Since I set yearly goals, now seems like a good time to check in on them.

.

Huh,  I guess I didn't really write my goals on the blog, did I.
Welp, I guess let me look based on the goals I wrote down elsewhere.

I have a goal of blogging every day, I made it 28\footnote{including today} days out of 31, which is pretty good.
I'd like to do better next month, which might require more planning ahead.

I have a goal of writing a poem every day, I think I succeeded twice.
I might want to switch to an easier format than villanelles, maybe sonnets again, just to get back into it.

I have a goal of working out more.
I've been doing better on that regard, especially since I shifted from shooting for 30 minute workouts, which I cannot find the energy to do, into doing an additive thing.
Basically, every day I do one more pushup and squat.
I started on the 11th, so I do the number of days this year minus ten to see how many I should do each day.
I still haven't been great at keeping up with it, but hopefully next month will be better.

I've on average been good with Bible in a Year, though I've had to double up a few days.
I'd like to do better at that.

Looking ahead to next month, I think my goals will be:
\begin{enumerate}
\item Blog daily
\item Compose a poem and music daily
\item Do the pushup/squat thing daily
\item Listen to BiaY Daily
\item Track my daily doings better in my journal
\end{enumerate}
I guess I'll probably check in four weeks from now, to see how I did.

I do like that I've managed to add blogging back into my daily routine fairly easily, especially since I haven't found that it correlates too much with extra screen time\footnote{something I'm trying to limit}.
I still haven't found my right place for poetry.
Historically I would write them before bed I think, but I'm far too tired for that most nights.
Who knows, though, maybe taking those extra few minutes would serve me really well.
I guess that's something I can work towards in February\footnote{wow hard for me to believe that soon I'll have been writing in parts of the year I've never done before.}

My last post that was written the day it posted was on \href{diving-15-feb.html}{February 15}, and it was incredibly short.
My last post is dated \href{still-behind.html}{February 22} and was written March 3.
The last post I'm willing to claim as a fullish post is my reflection on the \href{reflections-on-readings-6-ordinary-c.html}{Sixth Week of Ordinary Time}, which was February 17 in 2019.\footnote{it will be February 13 this year}
It's weird to me that in a few weeks, between less than two and five, depending on which standard I'm using, I will officially be past anything I've written before.
The last post that was full and not backdated is the week before's \href{reflections-on-readings-5-ordinary-c.html}{Reflection on the Gospel}, which was the 10 of February.
It's a somewhat scary and exciting feeling.

Words: 522,35
\end{document}