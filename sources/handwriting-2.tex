\documentclass[12pt]{article}[titlepage]
\newcommand{\say}[1]{``#1''}
\newcommand{\nsay}[1]{`#1'}
\usepackage{endnotes}
\newcommand{\1}{\={a}}
\newcommand{\2}{\={e}}
\newcommand{\3}{\={\i}}
\newcommand{\4}{\=o}
\newcommand{\5}{\=u}
\newcommand{\6}{\={A}}
\newcommand{\B}{\backslash{}}
\renewcommand{\,}{\textsuperscript{,}}
\usepackage{setspace}
\usepackage{tipa}
\usepackage{hyperref}
\begin{document}
\doublespacing
\section{\href{handwriting-2.html}{On Handwriting Again}}
First Published: 2023 April 14

\section{Draft 1}
\href{handwriting.html}{Last time} I talked about my handwriting, I made the claim that the speed my handwriting changes was slowing down over time.
While I think that may have been true then, a large part of that had to do with the fact that I had been working in the same journal for nearly two years.\footnote{It might be nearly three, but my journals are far away, and I don't feel like finding one for certain right now.}

Unlike that journal, the journal I began around the time of my last post only lasted a few months.
Also unlike before, I had no issues finding a new way to change my penmanship.
It ended up being a two-fold change.

The first major change is that my handwriting has become much tighter than even before.
I attribute a large part of that to the fact that I got 0.3 mm mechanical pencils for Christmas, which give a significantly smaller line, especially how I use them.\footnote{rotating the pencil in my hand every few words so that the sharpest point is always what I write with.}

The other major change was the deliberate change that I made.
I apparently didn't talk about this last time, but I made a goal with myself to change my handwriting in some obvious and notable way each journal.\footnote{when asked why, I often respond with my joking (but truly and legitimately held belief) statement that people should set arbitrary goals and then strive to achieve them}
In my first journal, this meant writing in all caps.

In my second journal, it was all caps with the first letter of each word capitalized.
In the third, I think I added my double letters.\footnote{assuming that I have five. Otherwise ignore this one and treat all the numbers after it as one smaller}
In the fourth I stopped writing every other line, shrinking the vertical space on my page.

And, as I mentioned last time, in my fifth\footnote{again, assuming there were five before my newest} journal, I made overhanging letters.
I said then that I would do that with \say{t,i,s,c,g,j,f still unsure of b,p,r}.
The list is now slightly different.
I do it with B,C,E,F,G,I,J,P,R,S, and T.

Since I didn't alphabetize it last time, the differences are as follows.
I became sure of B, changed the way that I wrote my E so that it overhung better\footnote{i.e. made it less like an epsilon}.
I became sure of P and R.
All in all, not a huge difference between the first pages and the last.

In my new journal, all that remains, but now I write it with a slant.
I'm still not sure what angle to my vertical strokes is ideal, but I think I tend to be somewhere in the fifteen to twenty degree range.
Comments this week have included \footnote{looks like elvish}, \say{looks like a medieval scribe's handwriting}, and \say{it kind of looks like Arabic.}

I like how it looks, though I do think it's funny that I went from messy and illegible to pretty and illegible.
Maybe my next journal's handwriting change will be legibility to a broad audience.
\end{document}