\documentclass[12pt]{article}[titlepage]
\newcommand{\say}[1]{``#1''}
\newcommand{\nsay}[1]{`#1'}
\usepackage{endnotes}
\newcommand{\1}{\={a}}
\newcommand{\2}{\={e}}
\newcommand{\3}{\={\i}}
\newcommand{\4}{\=o}
\newcommand{\5}{\=u}
\newcommand{\6}{\={A}}
\newcommand{\B}{\backslash{}}
\renewcommand{\,}{\textsuperscript{,}}
\usepackage{setspace}
\usepackage{tipa}
\usepackage{hyperref}
\begin{document}
\doublespacing
\section{\href{candied-orange.html}{Candied Orange (Peel)}}
First Published: 2022 January 5
\section{Draft 1}
If you've seen some of my \href{boiling-water.html}{earlier post}s, you might notice that I sometimes used to mention recipes I'd cooked here.
Mostly this serves as a way for me to\footnote{in theory} recreate these dishes at a later time.

My brothers kindly gave me a\footnote{along with the better part of a second} 5 kilogram chocolate bar for Christmas.
Now, if you're anything like me, you immediately thought of the same thing to use the chocolate for!
That's right, candied orange peel.

Well, being fair, my mind went:
\begin{enumerate}
\item Chocolate and Christmas
\item Christmas = Orange\footnote{for some cultural reason with no direct impact on me}
\item Chocolate and Orange
\item Chocolate Oranges
\item ...
\end{enumerate}
I'm sure you can catch the rest. So, without further ado, I present my\footnote{yet unfinished} recipe for chocolate orange (peels).

\begin{enumerate}
\item Cut orange in half polewise
\item Remove Skin
\item Boil skin in water for 30 minutes
\item Remove skin and cool
\item Scrape pith\footnote{the white bit} off with a knife or something
\item Put orange slices and skin together with sugar and reserved boiling water
\item Cook until candied
\item Drain and Dry\footnote{where I am now}
\item Coat in chocolate??
\end{enumerate}
\end{document}