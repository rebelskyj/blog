\documentclass[12pt]{article}  
\newcommand{\say}[1]{``#1''}  
\newcommand{\nsay}[1]{`#1'}  
\usepackage{endnotes}  
\newcommand{\B}{\backslash{}}  
\renewcommand{\,}{\textsuperscript{,}}  
\usepackage{setspace}  
\usepackage{tipa}  
\usepackage{hyperref}  
\begin{document}  
\doublespacing  
\section{\href{out-through-forward.html}{The Only Way Out is Through. The Only Way Through is Forward}}  
First Published: 2025 April 11

\section{Draft 3: 11 April 2025}

One of the weirdest things to think about in physics, for me at least, is how important a frame of reference is.  
In general, we tend to treat the earth as our fixed reference.  
Of course, the earth, as we generally know from science, effectively moves around the sun.\footnote{yes, yes, all orbiting objects orbit each other}

A part of me wants to go off here about the way that I have lost my own fixed reference.  
However, the words are failing me, which is always a shame.  
The point of this framing was initially about how the earth is opposite the sun halfway through the year.  
In that respect, I'm as far away from my mom as I can ever be.

Anyways, I'd rather not focus on that, if only because I don't want to start crying.  
Why is that relevant to this musing in particular, though?

While my mother was suffering from cancer, she told me that she really found relief while reading my web serial.  
On days when I struggled to start writing the day's chapter, I would start by typing the title out.  
Getting the words on the page was almost always enough to get started with the chapter, and the reminder that what I was doing something not just for myself, but for someone I cared about helped me to finish it.  
As time has progressed, I've grown to use the phrase more and more, and the meaning it has for me has deepened and grown.

I have seen the first half of this expression at least somewhat often, and I feel like I've seen the second half as well.  
I don't know if I see them together, though.  
It's very important to me that the two are linked, at least in terms of motivating myself.

I often know that I want out of whatever place I am.  
Usually that's in a metaphorical or emotional sense, in which case it's far truer that I have to get through whatever I'm in.  
In a physical sense, I can usually simply run away, rather than going through whatever experience I have.

Why, then, is it important to remember that the only way through is forward?

To start, I cannot go backwards in time.  
Even though time doesn't feel particularly linear, the arrow does always face the same direction.  
More, though, I also find it important to remember that I cannot passively move through whatever I experience.  
If I want my situation to change, I must make efforts.

And, right now, I find myself realizing I should start explicitly saying this more often.  
It could be a fun way to practice or play with my penmanship.

\section{Draft 2: 9 April 2025}

Before the most recent total solar eclipse in the continental United States, I found myself giving a number of talks about how eclipses worked.\footnote{found myself is such a fun way to describe anything I've done, because it does so much to decenter my own agency}  
In the standard version of these talks, I explained how a day is relatively easy to describe based on the motion of the earth relative to the sun.\footnote{earth spins}  
A year, likewise, is relatively simple.\footnote{earth goes all the way around the sun}  
A month, by contrast, is far harder.

The standard Western calendar these days is almost completely decoupled from any historically informed meaning of a month.  
In general, months have historically\footnote{and in most non-Julian or Gregorian systems, still are} been defined by the moon.  
There are generally two ways of describing a month with the moon, one based on the time to return to a certain phase and one based on when the moon realigns with the far off stars.\footnote{because moon lighting is based on position of earth to sun, and so when moon gets back to same spot relative to earth as viewed from above, the earth has moved, and so it isn't the same phase}

Why am I talking about months and how arbitrary they are?

Yesterday, I was reminded that it had been six months since I lost my mother.

Six months, in my mind, should be exactly half a year, always and forever.  
In this case, it just about is.

When the earth was in the opposite point of its orbit, my mother passed from this life into the arms of the Lord.

In a very real sense, I am just about as far away from her as ever I can be.  
Space and time come together into a four dimensional reality, but what is time to man?  
As my memories fade, what will it mean to have lost my mother?  
Already, I find it harder and harder to remember the sound of her voice or the feeling of hugging her.

Even though she breathed her last half an orbit ago, I do really feel like I lost her slightly earlier.  
She had been growing steadily less lucid and wakeful in the weeks leading up to her death.  
I do not remember the last words we said to each other, but I am nearly positive the last words she would remember saying to me are that she loves me.  
I would hope that the last words she would remember me saying to her were that I love her.  
Even the morning that I woke up to my father's call, informing me that our vigil was over, I could not remember the exact time of our last words.

How, though, does this relate to the title of the post?

As my mother's cancer progressed, and she began dealing with the symptoms of both the disease and its treatment, I found myself in the same position that so many others have.  
I watched someone who had cared so deeply for me suffering, and could not find a way to take the suffering onto myself.  
However, she told me that she enjoyed reading the web novel that I had been writing.  
In retrospect, I think that she mostly just enjoyed being able to support me in my creative acts, but at the time I took her at face value.

Opening the blank document which would become the next chapter,\footnote{Yes, I do use a separate file for each chapter of the book, and yes, I tended to write each chapter without any real consideration for an overall plot} I would find myself staring at it.  
Like staring at a perfect marble block, knowing that anything can come from it, but that anything I made would preclude making something else, I found myself frozen with indecision.  
And, feeling generally down about things, I found it hard to write the fundamentally upbeat story that I had started.  
I knew that my mother enjoyed having something lighthearted, and so felt that I needed to keep the book positive.

And so, knowing that writing was a way that I could deal with my feelings and ameliorate suffering, the first words I would type would be \say{the only way out is through. The only way through is forward.}

As I think about the phrase and its usage, I realize that I cannot separate it from my mother.

How did\footnote{and to some extent does} this set of phrases or sentences or clauses or whatever else you want to call them\footnote{that is probably really meant to be a footnote} help me, and what did it mean?

I cannot separate the meaning I have with them now from the meaning I once had.  
However, I don't think that they're significantly different, and most of the difference is probably me better recognizing what it meant.\footnote{n.b. at this point I had to leave for an appointment, and planned to have continued the draft after returning home}

\section{Draft 1: 9 April 2025}

I have mentioned somewhere before\footnote{potentially only in the list of musings to write and the post about why I stopped with my web novel} that I have a phrase I use fairly often to motivate myself.  
Like so many others, a blank page terrifies some primal part of me.  
As a result,\footnote{and when generally life feels hard,} I found it really helpful to simply remind myself that \say{the only way out is through and the only way through is forward.}  
At first, I only used the beginning half, but I found that it did not do enough for me.  
So, what does it mean?

Honestly, I feel like the meaning is fairly self-evident.  
If I want to exit any situation I'm in, be it not having a draft finished or feeling a certain way, I have to work through whatever is keeping me from it.  
And, since I have yet to learn how to experience the arrow of time in more than one direction, that means that the only way to get through the issue is resolving it.  
That is, while it might have been better for me to not find myself falling from a plane, once in the air, that is not a useful thought.

How does it help me?

Truthfully, I find that it helps me most because it reminds me that, like all things, whatever issue I face is transient.\footnote{I have an idea for a reflection about how \say{remember you are dust and to dust you will return} can be, rather than simply humbling, also words of encouragement.}  
Also, since I tend to use the phrase mostly when facing something that I apparently feel conflicted about doing\footnote{else I either would just do it or wouldn't want to do it at all} something, it reminds me that I am the agent in control.  
One way forward is giving up a project, after all.

However, it is obviously not my preferred way forward.

The phrase is also a good reminder that, much as I might wish otherwise, I must deal with anything holding me back before I can be free.  
If there are chains of doubt weighing me down, I have to unlatch them.

\section{Daily Notes}

\begin{itemize}

\item Obligations:

\begin{itemize}

\item Professional

\begin{itemize}

\item Write the thesis

Doing so badly here. I have an idea for what happened to my motivation.\footnote{I write this before the text of the day, which is sort of opposite of how I assume it's consumed.}

\item Revise the thesis

\item Edit the thesis

\item Research for the thesis

Gave up on my new and cool and novel search algorithm for one who\footnote{which??} works. It hurts but I suppose that it's probably for the best.\footnote{honestly, I have no idea if that's true. Ran two samples last night, one with new and one with old}

\item Read the books that might be useful for the thesis

\item Start citation tracking

\end{itemize}

\item Personal

\begin{itemize}

\item Learn the songs for to jam

Remembered to bring the book home, but did not do much else.

\end{itemize}

\item Self:

\begin{itemize}

\item Silence

None at all.

\item Typing practice.

Unlocked another letter or two, which is great. Also sent it to a friend\footnote{idk how to describe some of the people in my life. Friend is clearly the wrong word because we've only interacted in professional environments but like we both smile at each other when we run into each other (in a way that is more happy/genuine than the standard greeting a known}

\item Keep the phone out of the room for bed

I did not, but I also did not scroll.  
I did, however, start catching up on videos, because, as mentioned before, I lack comfortable seating in my home.

\item Pray St. Michael Chaplet in the morning

...

\item Stretch in the morning

\item Read at night

\item Poetry at night

\item Clean the home

Did minimal yesterday, but minimal is not none!

\item Stretching, standing, drinking water

Ehhh. I did ok on drinking water, if not good. That's about it though.

\item Posture

Decent! Still weird to me how uncomfortable it is to have flat feet on the ground.  
Also like wow my legs only want to exist crossed when I lie down.

\item No wasted time

I did not have this, which is a bit of a shame.

\item Eat more than 2 meals a day

Made curry last night! And I had it over rice for dinner.  
It was really good, and wow I need to use cumin more.\footnote{if only because I bought a large bottle of it}

\end{itemize}

\end{itemize}

\item Goals and Growth:

\begin{itemize}

\item Ends:

\begin{itemize}

\item Letter writing, get into more

Eh I got new ink, which will hopefully help to motivate me

\item Handwriting, pick and make the new one

No real progress, but see above

\end{itemize}

\item Means:

\begin{itemize}

\item Typing speed, improve it.

My mind is empty today for some reason, and so I have spent much of the past hour working on my typing. Upon rereading the site, I saw that they recommend starting at a very low speed and only increasing the character per second goal when all letters reach the barrier.  
I decided that half character per second increments would be ideal, and then only after breaking three and a half characters per second on all letters did I realize that I should really be focusing on accuracy of both letter choice and finger use over speed right now.  
After all, when I practice my scales, what is important is getting the best possible tone, fingering, and note.  
After that, the goal is to get rhythm right, and I am not entirely sure how that connects here.

I guess that ideally, each letter would be typed at exactly the same pace, and I would sound perfect.

\item Reading, do more of it

I finished the book yesterday, and am making my way through the audiobook.  
Unfortunately, I also need to find time tonight to walk for an hour as I listen to the All Night Vigil again.

\item Blogging, do it

I didn't post yesterday's and I think there are two reasons.  
First, I want to make the post better before I release it.  
Second, I was out of energy.

\item Writing things that are not the blog and thesis, do

Practicing touch typing is like writing!

\end{itemize}

\end{itemize}

\end{itemize}

\section{Daily Notes 11 April 2025}

\begin{itemize}

\item Obligations:

\begin{itemize}

\item Professional

\begin{itemize}

\item Write the thesis: Making progress!

\item Revise the thesis

\item Edit the thesis

\item Research for the thesis.

Oh boy is it fun to deal with Government agencies. Also, wild that some people pick their phones up before a single ring. I didn't even know data sent that fast.

\item Read the books that might be useful for the thesis

I got another one today! Woo.

\item Start citation tracking

Slowly making a gigantic Zotero folder, which is close.

\end{itemize}

\item Personal

\begin{itemize}

\item Learn the songs for to jam

\end{itemize}

\item Self:

\begin{itemize}

\item Silence

Absolutely none. I want to finish this book.

\item Typing practice.

I didn't do it yesterday, but I would like to do it today.

\item Keep the phone out of the room for bed

Nope! Hopefully after this weekend.

\item Pray St. Michael Chaplet in the morning

\item Stretch in the morning

\item Read at night

\item Poetry at night

\item Clean the home

\item Stretching, standing, drinking water

\item Posture

Decently! Not stretching enough, which isn't great.

\item No wasted time

Yesterday was a blur, and I should not have gone on the field trip to the book sale.

\item Eat more than 2 meals a day

I think so! In that I ate a giant bowl of oats, a heart of lettuce,\footnote{why does lettuce only have a head and a heart? Where's the leg?} and some rice with curry. Was the curry rice a full meal? Great question, moving on.

\end{itemize}

\end{itemize}

\item Goals and Growth:

\begin{itemize}

\item Ends:

\begin{itemize}

\item Letter writing, get into more.

I just picked up all the etiquette books that I was considering reading, so now I can read them.  
Also! The pen set I'm ordering with a friend is being shipped, so soon I will be able to play with more inks.  
I do find it interesting how much nicer the pens that are more finicky seem like they're writing.  
I don't know if that is just a function of the ink that I'm using or what, but I do love how it looks.

The pen I'm using right now is a little high flowing, which is maybe not the best.  
I do love the way that the wet ink looks on the page.

\item Handwriting, pick and make the new one

I tried something to the extreme of print, with incredibly sharp lines.  
I realized how much I hate it, and have been doing some cursive now as I take notes.  
I remembered or was reminded\footnote{memory is strange} that cursive comes from running, and it is so nice just sliding my pen.  
Might just give up on the part of me that prefers print and commit to making more people learn cursive.

\end{itemize}

\item Means:

\begin{itemize}

\item Typing speed, improve it.

Whoop! Will do at some point, hopefully.

\item Reading, do more of it

Still listening to the book!

\item Blogging, do it

Oof, I haven't posted in a few days, which is not something I love.  
Time just completely escaped me, and last night I somehow lost the hours between 5 and 8 working on derivations.\footnote{see, as always \say{derivations are dangerous for me}}

\item Writing things that are not the blog and thesis, do

\end{itemize}

\end{itemize}

\end{itemize}

\end{document}