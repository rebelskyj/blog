\documentclass[12pt]{article}[titlepage]
\newcommand{\say}[1]{``#1''}
\newcommand{\nsay}[1]{`#1'}
\usepackage{endnotes}
\newcommand{\1}{\={a}}
\newcommand{\2}{\={e}}
\newcommand{\3}{\={\i}}
\newcommand{\4}{\=o}
\newcommand{\5}{\=u}
\newcommand{\6}{\={A}}
\newcommand{\B}{\backslash{}}
\renewcommand{\,}{\textsuperscript{,}}
\usepackage{setspace}
\usepackage{tipa}
\usepackage{hyperref}
\begin{document}
\doublespacing
\section{\href{learning-guitar.html}{Learning Guitar}}
First Published: 2022 April 30


\section{Draft 1}
Well, it's the end of the month, and I don't quite know what to talk about.
Tomorrow and Monday I can write about the readings and my reflections on this past month, but today has no such stipulation.
I was just playing guitar, so it seems reasonable to talk about that.

I've been playing guitar off and on for most of my life.
I don't know how old I was when I first started lessons, but I do remember that they were primarily focused on simple chords and strumming patterns.
I think that's called rhythm guitar by people in the know.

As I grew older, I eventually started studying music.
That happened to coincide with my breaks from guitar, which is a little sad.
It's made coming back to guitar really interesting every time I do, though.

Guitar, moreso than any other instrument I've seen, has learning materials which seem actively opposed to conventional music theory.
They don't use sheet music and barely use rhythm.
On the other hand, classical guitar has less of this issue.\footnote{as far as I can tell}
I just can't find any music easy enough to play that is interesting enough to keep me playing.

So, I decided that, as the adult I am now, I should really work on mostly doing exercises for my practice time.
For a while the skill I felt I lacked most was my right hand plucking, so I started working on Mauro Guiliani's right hand patterns.
It's 120 exercises that have you playing between a CM and a G7\footnote{I still hate how jazz presupposes that 7 means Mm, but that's another day's post} chord in your left hand while your right plucks out different patterns.
That's worked well for me, and I've enjoyed it.

These past few weeks, though, I've been saddened by my inability to play more melodic things on a guitar.
Partially that's because I don't plan on doing a lot of solo guitar, but partially I just never really built the skills for moving my left hand on the guitar.
As it turns out, Guiliani wrote some exercises for the right hand as well.
I've been working through the first bit of the first one for a week or so now, and it's weird.

I never realized how uncomfortable I was using my second and fourth fingers at once.\footnote{I will be using guitar naming for fingers in this post because I can't deal with the load of translating to real names}
The exercise has me doing a lot of that, which is really nice.
Primarily it just goes up and down the strings using the first through fifth frets playing thirds on adjacent strings.
It's not the most fun, but it is pretty cool to be able to do it with more ease.
As I look at the other exercises, it appears as though the next has me playing 6ths, the next octaves, the fourth is tenths which is wild to me, the fifth is thirds again for some reason, my minimal Italian suggesting that it helps with moving up the fretboard, the sixth again has me doing sixths\footnote{that's fun}, the seventh octaves, the eighth tenths, and repeating until we've gotten through sixteen exercises.
They each look to be about 20-24 measures of common time with moving sixteenth notes, so as you can imagine I am not going through it quickly.
I think the furthest I've gotten is six measures through the first, but that's progress still!

\end{document}