\documentclass[12pt]{article}[titlepage]
\newcommand{\say}[1]{``#1''}
\newcommand{\nsay}[1]{`#1'}
\usepackage{endnotes}
\newcommand{\1}{\={a}}
\newcommand{\2}{\={e}}
\newcommand{\3}{\={\i}}
\newcommand{\4}{\=o}
\newcommand{\5}{\=u}
\newcommand{\6}{\={A}}
\newcommand{\B}{\backslash{}}
\renewcommand{\,}{\textsuperscript{,}}
\usepackage{setspace}
\usepackage{tipa}
\usepackage{hyperref}
\begin{document}
\doublespacing
\section{\href{book-review-creative-survivor.html}{Book Review of The Creative Writer's Survival Guide}}
First Published: 2023 November 15
\section{Draft 1}
It's been a while since my last book review!
Looking at the site,\footnote{read: going to the git repo on my terminal and searching for all instances of book-review} I've reviewed seven books so far on this blog.
I've read far more than seven, so let's see if there's something in particular that links the books I have reviewed.

\begin{enumerate}
\item In April of 2022, I reviewed my first book, \href{book-review-grandmother-sorry.html}{My Grandmother Asked Me to Tell You She's Sorry by Fredrik Backman}.
It was a really good book, albeit one that I haven't thought about in ages.
I remember absolutely bawling as I finished the book, and I think it's the book that made me cry more than anything else I've ever read.
I should really read it again, because I loved it a lot and I'm curious how much I'll like it a second time through.
A lot of the book was interesting to me because of how little I knew about it.
\item \href{book-review-faces-lewis.html}{Till We Have Faces by C.S. Lewis} was apparently the second book I reviewed.\footnote{if this phrasing feels weird, that's because I misread dates, and thought it was the first.}
I read it as part of a book club in my parish, and I remember enjoying it a lot.
I don't think that I've really thought about it since I wrote the reflection, and I don't think I'd heard of it before the book club.
So, there's no personal connection I can find without reading it.\footnote{review or the book could be it, I suppose, but I meant review}
\item Next, a few months later, I reviewed \href{book-review-atomic-habits.html}{Atomic Habits by James Clear}.
I have a similar lack of connection to the book.
I enjoyed it, but I think I found it less impactful than the first book I reviewed.
\item One single day later, I reviewed \href{book-review-gideon-ninth.html}{Gideon the Ninth by Tamsyn Muir}.
I enjoyed that book a fair amount, though I think it's interesting that I ended my review saying that I was looking forward to the next book in the series.
If I remember correctly, the second book is written, appropriately enough, in second person.\footnote{appropriate here because if my memory serves, the first book is written in first person, and also I like the fact that book two is written in person two.
I guess that can't work if there's more than three books in the series, because fourth person isn't a thing.
(I wonder what fourth person could mean.
If we think about it, first person is referring to self, second person refers to the other participant in the conversation.
Third person refers to something not a part of the conversation.
I know that there are languages that divide the world differently, but I don't think linguistics needs more people.
Ope. I've looked it up, and there are evidences of persons after third.
Apparently some languages treat referring to someone nearby and somewhere far away differently.
When they do, the latter is sometimes considered fourth person.
There's also a reference to fifth person, so that's interesting.
Otherwise, fourth person is sometimes used in languages that treat generic referents differently than the standard third person (e.g. one should not).
That's kind of cool, and something worth thinking about for any future ConLang I work on.)
Ok, so it could work with more than three books when translated.}
\item About a month later, I reviewed \href{book-review-housekeeper-professor-ogawa.html}{The Housekeeper and the Professor by Yoko Ogawa}.
It was a really fantastic book, and one that I do have a long personal history with.
Before my grandmother died, I remember that she gave me the book to read.\footnote{when, before she died, I do not know.}
It is a bittersweet book, which is fitting for a lot of the memories that I have of times before my grandparents died.\footnote{though honestly, at this point they're almost all sweet.
There was a part of me that felt like I didn't value my grandparents enough while I had them, but that part of me knows things that I couldn't have known then.
I'm working on growing as a person, and part of that is letting go of regrets.}
Honestly, I'm choking up a little thinking about this book, which is kind of interesting.
I love it a lot, and I think that it holds up if anything better with having reread it.

It's interesting to me how that can be true.
I feel like normally when I read a book that I have fond memories of, the memories become tainted somehow.
Even if I appreciate the book as much, I still look at my past memories of the book with a different light.
With this book, however, my old memories are still there, if anything slightly better.
Regardless of if I actually read it in the upstairs bedroom of my grandmother's home or if that's just the room I have the clearest memories of, it is a warm thought.
\item Almost seven months later,\footnote{what a delay wow} I reviewed \href{book-review-after-dragons.html}{After the Dragons by Cynthia Zhang}.
It was a book that I picked up on impulse, enjoyed, and haven't really thought about since.
In that regard, it's similar to most of the items on this list.
It was a little more literary than I tend to enjoy, I do recall clearly.\footnote{though as mentioned yesterday, I recently read a book on being a novelist (more focused on the financial and professional consequences of the choice, rather than a style guide or anything, which meant it wasn't particularly relevant for my life.
For all that I do enjoy writing, and for all that I do intend to continue writing for the rest of my life, I don't know if I really want to try to make it my full time career.
I like it being a dedicated but amateur (in the sense of doing for love rather than money, which I think tends to be a newer distinction than I tend to use in my writing) hobby, and I worry about how little I would enjoy it if I were compensated.
That's probably an attitude I should investigate (I should read my old posts to see how often I say the word investigate, because I have a gut feeling I think that I should investigate a lot of my thoughts in the future.
Unfortunately, it's always the present).
Actually, that might be a good book to review here.
Anyways, the book talked about how there's a sense in which you should, upon reading something you do not enjoy, ask yourself what is wrong with you, rather than asking what's wrong with the text.
As he points out, if you read something for a course, that means a professor (ever since I made the choice to capitalize all professions in my book and then shift it to a school arc, where there are, obviously enough, a lot of Professors, I feel the urge to capitalize the word whenever I use it), who presumably has expertise you wish to acquire, thinks that it was worth reading.
Even if you were not assigned a book, the fact that it's published (the guide was written before the age of self publishing had really gotten started) means that a lot of people, all of whom are experts in their field, thought that the book was worth publishing.
It's an interesting thought, and one that I feel is useful.
I've realized that I have started to think of writing and words (there's a better way to phrase that, I'm sure, but I can't think of it) with the same part of me that thinks about music.

That is, I have two very distinct mindsets when I listen to music. (I think, let's see if two is the right number or if I should've said N and then fixed in post).
Most of the time, I treat music as an accompaniment to my life, something that I listen to because I enjoy, not trying to pick up anything in it.
Other times, though, I treat music as an academic exercise, looking and listening deeply to see what makes it do what it does.
When I listened to music that I did not find matched my aesthetic in undergraduate, I learned how to appreciate it for what it did well.
I should really get better at doing that for literature.
After all, anti intellectualism isn't cool anymore (it was never cool).
Wow that was a long diversion from the book I reviewed}
\item Finally, I reviewed \href{book-review-maus.html}{Art Spiegelman's Maus}./footnote{I don't know why it feels so important to me to give ownership of the book to him in that way, but I do.
It's something interesting about my thought process, at least.}
In the current world, it feels especially important for the book to be read, for all that this feels like the exact wrong time to suggest that others read it.
\end{enumerate}


Well, I can't find anything that directly links the books.
There are some I had a bit of a historic connection to, some that I read for external reasons, and some that I just chose to read on my own.
With that long introduction\footnote{about fifteen hundred words} out of the way, I've figured out what this post will be about.

I recently read The Creative Writer's Survival Guide: Advice from an Unrepentant Novelist by John McNally.
If I remember correctly, either my father or one of his colleagues was getting rid of a lot of his\footnote{the colleague was (is?) also a man, so technically his works regardless of which person was getting rid of books} books.
My father thought that my little brother and I might be interested in books on writing, and we each saved a few.

One of the books I saved was a book on how to do NaNoWriMo successfully, which I annotated and gave to a friend who's doing the event with me this year.\footnote{and I know that the friend reads the blog. Since I know I'll forget to ask you in real life, are you still reading the book? Do you find it at all helpful?}
In addition to the other books, which I have yet to read, I got the aforementioned survival guide.
I realized about halfway through reading that it was an ARC\footnote{Advanced reader's copy, which means a usually free copy sent ahead of the official release date.
I'm unsure why either of them would have had it, since neither seem to have a connection to the author, but I suppose that the author went to a writer's workshop at the same place that my father ended up going to a while later, so it's possible that he got it there.
We also have a shocking number of ARCs (ARC? because copies is still just C. Idk) in my home, I think at least.}.
I can't imagine too much of the content changed before final publication, but who can say.

Despite the fact that I have no intention to make a career out of writing,\footnote{see the footnote on After the Dragons} it had a fair amount of advice that seems relevant to me.
Most of it was given as asides to his main points, but I don't think that makes the advice any less useful.
For instance, he talked a lot about the resistance that many young authors have to being told that their writing can improve.
I know that there are strains of that still within me, and I should really get rid of it.
I know that I am not a great writer, for all that I think I'll start to defend myself as a good writer these days.

Other relevant advice was that you need to treat writing as a commitment, and that inspiration only strikes when it's made incredibly easy.\footnote{Ok, that's two very different things, but I think that they're kind of the same.}
That is, while sometimes a short story or poem comes out incredibly easily and polished from an early draft, it only comes because of a heavy grind.
This is one place where music is absolutely a great analogy, for all that it's also true for me in writing.

In music, if you just practice scales and tone every day for like fifteen minutes, any song you attempt after three weeks will suddenly become easier.
Small composition exercises on voice leading makes writing interesting and singable lines almost second nature.
Writing a sonnet a day starts out feeling impossible, but a few short weeks later, is incredibly easy.

Other good advice was generally remembering that everyone you interact with is a person.\footnote{having now moved past the paragraphs I discussed and come back to this footnote, I think that I explained it badly.
In part, that's because it's a summary of like a third of the book (if taken in very narrow slices).
It was given as equally important when submitting work to be published, speaking to professionals, and speaking to fans and publicists.}
That is, when you talk to an editor, for example, the editor will remember you after the conversation.
Even if that doesn't directly make it easier to publish with them, it does mean they're more likely to point you to a place you might be able to publish.
If you give a reading, anyone who comes can be great or terrible publicity for your book.

As someone who's steadily making a more public presence by giving talks, that's probably worth remembering.
Every time that someone leaves a talk of mine and thinks better of science or the University\footnote{hey look, a place where capitalizing was appropriate.
I want to bring back the early modern English trend of capitalizing any important words in a sentence, for all that I know that there are probably good and legitimate reasons that we discontinued the practice.
That does remind me, though, I read a fascinating legal brief today about double spacing.
Why, exactly, the definition of double space was relevant to a federal case, is still a little unclear to me, but it was a great read, as everything petty by lawyers tends to be.
Something about being highly educated in the art of argument and using that well honed skill to be petty is just very fun for me to spectate}, I've done a good thing.

Anyways, the book as a whole gave me a lot to think about, and it has a long list of books to read on improving my craft, so I'll probably try to read more of those in the future.
I don't know if I'll reread this book, but if any of my readers have been considering a career in writing\footnote{and, presumably, are pre-college somehow, since that kind of seems to be the focus group}, I would recommend it.
He did make it very clear that most writers end their careers at their best writing, which makes so much sense to me now that it's been pointed out.

Daily Reflection:
\begin{itemize}
\item Did I write 1700 words for NaNoWriMo?
I did, and the reminder from yesterday made me post the update.
\item Did I write a chapter of Jeb?
I finished the chapter, and it was a little easier.
\item Did I blog? I once again rambled, but I did write a post and explore some thoughts I'm having.
\item Did I stretch? Ope. Time to stretch.
\item Am I doing better at prayer than a rushed and thoughtless rosary? 
No.
I feel bad about it, but I did read\footnote{part of} a book on apologetics, which is somewhat prayerful, I suppose.
\item Am I doing a good job writing letters to friends?
Shoot! I meant to write a letter tonight. I did not and it's now too late for me to feel up to it.
I guess I'll do it tomorrow.
\end{itemize}


\end{document}