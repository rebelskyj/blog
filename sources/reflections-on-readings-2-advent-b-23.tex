\documentclass[12pt]{article}[titlepage]
\newcommand{\say}[1]{``#1''}
\newcommand{\nsay}[1]{`#1'}
\usepackage{endnotes}
\newcommand{\1}{\={a}}
\newcommand{\2}{\={e}}
\newcommand{\3}{\={\i}}
\newcommand{\4}{\=o}
\newcommand{\5}{\=u}
\newcommand{\6}{\={A}}
\newcommand{\B}{\backslash{}}
\renewcommand{\,}{\textsuperscript{,}}
\usepackage{setspace}
\usepackage{tipa}
\usepackage{hyperref}
\begin{document}
\doublespacing
\section{\href{reflections-on-readings-2-advent-b-23.html}{Reflections on Today's Gospel}}
First Published: 2023 December 10

\section{Draft 1}
I know that I try to have a deep and meaningful\footnote{at least to me} reflection about the Gospel each week, but I find that my mind isn't really catching on any of the many available hooks that the readings are filled with.
As seems to be my custom, there is a part of the reading that I struggle with.

In the first reading, we hear the Prophet Isaiah say that a voice cries out to prepare a way in the desert.\footnote{paraphrase of Is 40: 3a}
In the Gospel, however, we are told that the Prophet Isaiah says that a voice in the desert cries out.\footnote{paraphrase of Mk 1: 3a}
Interestingly, both happen in the third verse of the chapter, for all that the first happens in the fortieth chapter and the other in the first chapter.

Now, what about this bothers me, especially as a person who knows that ancient languages had a much different approach to punctuation than modern people do?
Primarily, it's the fact that if there was no punctuation, we\footnote{St. Jerome translating to the Vulgate, which I think does have punctuation, or any of the people translating to English} could have made the two line up.
But, we didn't.

I'm sure that there's some deep spiritual meaning for this, for all that I cannot find it.
The commentary on the USCCB website notes that, despite claiming to be a direct quote from Isaiah, there are references to other prophets' prophesies.\footnote{also, apparently Malachai was the penultimate prophet, since he was the last until John the Baptist. I should really have known that, since it seems like an important fact}

So, let's take the readings where I'm at today.
The first reading reminds us that struggles are both temporary and meant as purification.
Any hardship we endure now is either to cleanse us from some past wrong or\footnote{probably and, if we're being real} to prepare us for some future glory.

One part I find interesting is the idea that the geography of the world is something less than perfect.
There is a heresy that became popular near the Enlightenment\footnote{wow there's a lot of power in getting to choose the name of your movement} known as the Watchmaker G-d.
This heresy states that, much like a watchmaker, who carefully sets gears and winds a watch before stepping away, the Lord created the universe, along with all of its laws, and then stepped back to watch and see what happened.

Of course, as a scientist, I believe something similar.
As a Catholic, though, there is room for nuance, which I find very important.
So, where's the nuance?

First, an obvious contradiction to the Watchmaker\footnote{capitalized because it's still a way, however imperfect, of referring to the Divine} is the existence of miracles.
The water turning into wine or Our Lord's rising from the dead cannot happen without active intervention.
Even in the modern day, miracles are still performed wherever the Lord sees that it would most help the faithful.\footnote{I think? Honestly, why miracles happen is a mystery to me. If not for the fact that I think they might be a Mystery, I would probably be more bothered by that fact}
It goes deeper than that, though.

One thing that the Watchmaker heresy implies, however accidentally, is the preexistence of nature.
A watchmaker does not conjure the gears and hands of a watch from the formless void.
He may shape them from metal, but even the ore he gets comes from some other cause.
In contrast, the Lord created the universe from nothing.

So, it's an imperfect metaphor at best.
All metaphors are, which is why they're symbolic rather than literal.
What else is wrong with the Watchmaker?

The crucial issue in the Watchmaker heresy is the idea that the Lord stopped.
When a watchmaker finishes his work, the watch is done, and he can sell it or give it away.
Without the Lord constantly willing reality to remain, we would cease.
At every moment of every day, we are sustained and continued solely through Love.
Our Creator is not some dispassionate worker toiling for His pay.
Our Creator made us with Love, from Love, and for Love.

Returning to the readings, we see an oblique reference to this in a letter from the first Pope.
St. Peter reminds us that, though we experience life in something resembling a linear fashion\footnote{relativity is the reason for resembling, to say nothing of the whole field of cognitive science which tells us that our experience of passing seconds is not objective}, the Lord does not.
For Him who is outside of time, everything happens as it should.
It is a hard concept to imagine, for all that I know that it's essential for mathematicians, who often visualize high dimensional spaces.

I'm reminded of a movie\footnote{and the book the movie is based on, but I don't have as close a relationship or dear a memory with the book} called Flatland.
In it, a two dimensional creature is exposed to the concept of three dimensions, in part by seeing zero and one dimensional existence.
Thinking of it like that helps me resolve the whole \say{how does Free Will work when G-d knows every thought I've had before I have it?}
How, exactly, it works with miracles or the Divine Revelation, which explicitly happened to a specific people at a specific time, though, remains a mystery.
For all that I can trust that it is true, I would not be able to give a reasoned defense should that be someone's stumbling block to joining the Church.

So, where are we?
The readings appear to contradict themselves, for all that I know that any punctuation we add is a modern decision.
The universe is not a machine created by someone thoughtless, but a treasured love of Love Himself.
Time is an illusion, albeit one we are bound to.

I feel that this is as good a place as any to end.

Daily Reflection:
\begin{itemize}
\item Hobbies:
\begin{itemize}
\item Did I embroider today? Remains at work, and I'm ok with that fact.
\item Did I play guitar today? I replaced the strings! It's always wild to me how much of the upper end I lose when the strings get old, but also how much I kind of prefer my strings to not have particularly high overtones.
\item Did I practice touch typing today? A few lessons, and I managed to work my way back up to the letter c, for all that I didn't manage to beat or even improve it.
\end{itemize}
\item Reading
\begin{itemize}
\item Have I made progress on my Currently Reading Shelf? I realized that I'm listening to a lot of audiobooks recently, and there's no reason that I can't read some of the books I want to have read via audio means. Checked out the two that seemed available from my local libraries, and put a third on hold, I think.\footnote{maybe just checked out one and reserved one, we'll see.}
\item Did I read the book on craft? Tomorrow is always a day away.
\item Have I read the library books? see above.
\end{itemize}
\item Writing
\begin{itemize}
\item Did I write a sonnet? Whoops! Almost forgot.
\item Did I revise a sonnet? Have made ideas for what poem(s) I'll bring to the thing tomorrow.
\item Did I blog? I almost forgot to do the reflection, but yes!
\item Did I write ahead on Jeb? I broke the no Jeb on Sunday rule, writing about a third of a chapter.
\item Letter to friends? Considered the fact that I should write some, which is in a way kind of something resembling progress.
\item Paper? Without any guilt, I can say that I took the Lord's day to rest.
\end{itemize}
\item Wellness
\begin{itemize}
\item How well did I pray? Not the best but not the worst. I wrote most of this musing by the light of Hanukkah candles, which have only now extinguished themselves.
\item Did I clean my space? A tiny bit.
\item Did I spend my time well? Kind of? I took a nap which, while needed, always feels like it takes the entire day away.
\item Did I stretch? Nope! Got a shoulder massage, though, which is similar enough.
\item Did I exercise? Took the day off, because rest is important, probably.
\item Water? As happens on Sunday, I did much better in the morning because of singing, and then stopped.
\end{itemize}
\end{itemize}
\end{document}