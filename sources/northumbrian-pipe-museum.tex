\documentclass[12pt]{article}[titlepage]
\newcommand{\say}[1]{``#1''}
\newcommand{\nsay}[1]{`#1'}
\usepackage{endnotes}
\newcommand{\1}{\={a}}
\newcommand{\2}{\={e}}
\newcommand{\3}{\={\i}}
\newcommand{\4}{\=o}
\newcommand{\5}{\=u}
\newcommand{\6}{\={A}}
\newcommand{\B}{\backslash{}}
\renewcommand{\,}{\textsuperscript{,}}
\usepackage{setspace}
\usepackage{tipa}
\usepackage{hyperref}
\begin{document}
\doublespacing
\section{\href{northumbrian-pipe-museum.html}{Northumbrian Pipe Museum Review}}
First Published: 2018 December 7
\section{Draft 1}
Today, I had the wonderful opportunity to go to the Morpeth Chantry and Northumbrian Bagpipe Museum.
While I wouldn't recommend a trip to Morpeth just for the museum, the city was lovely, and I would highly recommend seeing it.

The museum itself was about the size of a typical exhibit in a full-sized museum, which was a little disappointing.
More disappointing was the lack of examples of how the instrument sounds or chances to make it sound myself.
They did have a cool thing where they'd rigged a set of the pipes to a piano keyboard so you could play a tune and make it sound, but it was broken.\footnote{which was sad}

However, they're a completely different beast from either Scottish war pipes or uilleann pipes.
They've got a closed chanter, and are bellows driven, and look really cute.
All this is to say: if anyone has\footnote{for whatever reason} a set of Northumbrian Pipes in need of a home, I promise to love and care for them.
\end{document}