\documentclass[12pt]{article}[titlepage]
\newcommand{\say}[1]{``#1''}
\newcommand{\nsay}[1]{`#1'}
\usepackage{endnotes}
\newcommand{\1}{\={a}}
\newcommand{\2}{\={e}}
\newcommand{\3}{\={\i}}
\newcommand{\4}{\=o}
\newcommand{\5}{\=u}
\newcommand{\6}{\={A}}
\newcommand{\B}{\backslash{}}
\renewcommand{\,}{\textsuperscript{,}}
\usepackage{setspace}
\usepackage{tipa}
\usepackage{hyperref}
\begin{document}
\doublespacing
\section{\href{poetry-planning.html}{Poetry Planning}}
First Published: 2022 December 26

Pre-reading note: Somewhat rambly
\section{Draft 1}
I talked about how I think that having schedules makes my life better.
Well, at least better in the sense of more in line with the goals that I set.
I mentioned \href{new-year-plans-23.html}{the other day} two things that will help, having preplanned ideas for blogging, and having a plan for the poems I'll write.

Anyways, something something about N stones and M birds.
There are 12 months in a year, which is one way I could arrange the poems I do.
There are 52 weeks, which is another way I could do it.

Maybe it would be helpful to start with the kind of poems I have written before and enjoy writing.
\begin{itemize}
\item Sonnets\foontote{There are a number of kinds of sonnets that I could do}
\item Villanelles
\item Ballads
\end{itemize}
In general, I like doing poems which are at least somewhat metric and constrained.
It's weird to me to realize that I've only really written four types of poems in recent memory.
It would be kind of nice to do more than that in the future.

So I think I'm going to do the same thing with poems that I'm doing with \href{species-counterpoint-planning.html}{counterpoint} and prose.\footnote{post to come}
Sundays will be a day for writing something out of pocket and planning the rest of the week.
Starting with the first week of the year, I'm going to shoot for the first six days to be sonnets. 
\end{document}