\documentclass[12pt]{article}[titlepage]
\newcommand{\say}[1]{``#1''}
\newcommand{\nsay}[1]{`#1'}
\usepackage{endnotes}
\newcommand{\1}{\={a}}
\newcommand{\2}{\={e}}
\newcommand{\3}{\={\i}}
\newcommand{\4}{\=o}
\newcommand{\5}{\=u}
\newcommand{\6}{\={A}}
\newcommand{\B}{\backslash{}}
\renewcommand{\,}{\textsuperscript{,}}
\usepackage{setspace}
\usepackage{tipa}
\usepackage{hyperref}
\begin{document}
\doublespacing
\section{\href{scholastic-metaphysics-1.html}{Scholastic Metaphysics 1, An Introduction}}
First Published: 2022 February 17

\section{Draft 1}
As I made my daily commute to work today, a few thoughts went through my head.
The first\footnote{irrelevant to this essay but legitimate nonetheless} was that the ice seemed maybe too crackly for me to have chosen to walk on the lake.
The second\footnote{relevant} thought was that I could use this platform as a way of keeping myself more accountable for the projects I want to do.

One project I think will fit really nicely here is my goal of learning Scholastic Metaphysics.

\say{But Jonathan}, you may be saying, \say{why would you care about Scholastic Metaphysics?}
Mostly I'm reading about it because one of the priests at the student center suggested it could be a fruitful discussion to think about the intersection of Scholastic Metaphysics and Chemistry.
I agreed, and he lent me a few books to read.

\say{But Jonathan,} you may be saying, \say{why now?}
The conversation happened in August\footnote{or the summer, I'm unsure, wait I have my log let me check. Hey nice it was 24 August}, but I kept not really reading the book, because things come up.
Yesterday night, though\footnote{isn't it weird how yesterday night feels wrong but last morning also does?}, he asked me if I'd had a chance to start them yet, so this seems like a good excuse to start again.

I'll be reading through \emph{The One and the Many: A Contemporary Thomistic Metaphysics} by W. Norris Clarke, S.J., if anyone wants to follow along.
The book has nineteen chapters, so assuming that each chapter is understandable in a week\footnote{One of those assumptions I have no clue how good it is}, I'll have twenty posts in this series, since today I'm reading the introduction.

Without further ado: my reflection on the Introduction to the book.\footnote{250/55 in}

The goal of this book is not to provide an insight into St. Thomas Aquinas's specific metaphysical system.
Instead, the goal is to \say{provide an advanced textbook of systematic metaphysics}\footnote{p1}.
One issue with Thomas's writings, according to the author, is that he uses Aristotle-like dense and technical writing, which is fairly inaccessible.

Apparently Kant and Hume and the like pushed for the idea that a systematic metaphysical understanding to all of creation is not workable, which is interesting.
This book will not be comparing Thomistic approaches to other schools of thought, which seems reasonable to me.
That's most of the introduction.

Come back next week to watch me try to piece my way through philosophy for the first time!
\end{document}