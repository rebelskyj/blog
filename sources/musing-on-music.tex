\documentclass[12pt]{article}[titlepage]
\newcommand{\say}[1]{``#1''}
\newcommand{\nsay}[1]{`#1'}
\usepackage{endnotes}
\newcommand{\1}{\={a}}
\newcommand{\2}{\={e}}
\newcommand{\3}{\={\i}}
\newcommand{\4}{\=o}
\newcommand{\5}{\=u}
\newcommand{\6}{\={A}}
\newcommand{\B}{\backslash{}}
\renewcommand{\,}{\textsuperscript{,}}
\usepackage{setspace}
\usepackage{hyperref}
\begin{document}
\doublespacing

\section{\href{musing_on_music.html}{Musing on Music}}
\section{Draft 3}
I'm almost always listening to music, even if only inside my head.
There's something sublime about music's ability to transport me from wherever I am and whatever I'm doing into a different place.
Although every song doesn't send me to the same place, or even the same place for each song, they always send me somewhere.

People ask me what kind of music I like\footnote{or prefer, or is my favorite. For my purposes, they're effectively the same}, and it's a fair question.
I'm a music major, I listen to a lot of music, and it's a normal question.
Nonetheless, I can only ever answer that I listen to almost everything.
It's both the most and least factual statement I can make.

The statement is filled with fact because there is no genre, artist, or even song that I completely refuse to see the merit of listening to.
Every song has its place, and even if I don't love every aspect of a piece\footnote{or even most aspects}, I still can recognize why its loved.
Even if I can't think why a song is loved, I can see its belonging in a situation, whether as ambiance or as transition.

For a very similar reason, the statement is totally devoid of fact.
That is, there is no song that I always love.
Songs that I love in most circumstances are unbearable in others.
As much as I love soft lilting pieces, I recognize that they don't belong in most people's pregame\footnote{or meet} playlists.
And, as much as I love heavy electronic beats, I understand that most people don't use them as lullabies\footnote{yes, both of those are real examples of music choices I make}.

Even If I don't like a song though, I always prefer music to its absence, which may be one of my defining traits.
If given the option between music or no music, there are nearly\footnote{I say nearly only because I'm sure there exists a situation, even if I can't think of one} no situations wherein I would not prefer music.
Music is the joy of life, and it's a joy that I refuse to do without.

\section{Draft 2}
There's something sublime about music's ability to transport me from wherever I am and whatever I'm doing into a different place.
Although every song doesn't send me to the same place, or even the same place for each song, they always send me somewhere.

People ask me what kind of music I like\footnote{or prefer, or is my favorite. For my purposes, they're effectively the same}, and it's a fair question.
I'm a music major, I listen to a lot of music, and it's a normal question.
Nonetheless, I can only ever answer that I listen to almost everything.
It's both the most and least factual statement I can make.

The statement is filled with fact because there is no genre, artist, or even song that I completely refuse to see the merit of listening to.
Every song has its place, and even if I don't love every aspect of a piece\footnote{or even most aspects}, I still can recognize why its loved.
Even if I can't think why a song is loved, I can see its belonging in a situation, whether as ambiance or as transition.

For a very similar reason, the statement is totally devoid of fact.
That is, there is no song that I always love.
Songs that I love in most circumstances are unbearable in others.
As much as I love soft lilting pieces, I recognize that they don't belong in most people's pregame\footnote{or meet} playlists.
And, as much as I love heavy electronic beats, I understand that most people don't use them as lullabies\footnote{yes, both of those are real examples of music choices I make}.

Despite the fact that I may hate a song though, I still won't refuse to listen to it.
I always prefer music to its absence.
That may be one of my defining traits.
If given the option between music or no music, there are nearly\footnote{I say nearly only because I'm sure there exists a situation, even if I can't think of one} no situations wherein I would not prefer music.
Even if I may prefer silence to the exact song being played, that's only because I'm always singing something to myself.

\section{Draft 1}
I love music.
There's something sublime about its ability to transport me from wherever I am, whatever I'm doing, into a different place.
Every song doesn't send me to the same place, and the same song may not send me to the same place twice, but they always send me somewhere different.

However, people ask me what kind of music I like\footnote{or prefer, or is my favorite. For my purposes, they're effectively the same}.
My usual answer of \say{anything} generally isn't well received.
Nonetheless, it's both the truest and least true statement I can make.

It's the truest because there is no genre, artist, or even song that I completely refuse to see the merit of listening to.
Every song has its place, and even if I don't love every aspect of a piece, I still can recognize why its loved.

It's the least true statement I can make for the same reason.
There is no song that I would always want to listen to.
Songs that I love in most circumstances are unbearable in others.
\end{document}