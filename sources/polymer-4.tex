\documentclass[12pt]{article}[titlepage]
\newcommand{\say}[1]{``#1''}
\newcommand{\nsay}[1]{`#1'}
\usepackage{endnotes}
\newcommand{\1}{\={a}}
\newcommand{\2}{\={e}}
\newcommand{\3}{\={\i}}
\newcommand{\4}{\=o}
\newcommand{\5}{\=u}
\newcommand{\6}{\={A}}
\newcommand{\B}{\backslash{}}
\newcommand{\sub}[1]{\textsubscript{#1}}
\renewcommand{\,}{\textsuperscript{,}}
\usepackage{setspace}
\usepackage{tipa}
\usepackage{hyperref}
\begin{document}
\doublespacing
\section{\href{polymer-4.html}{Polymer Review Part 4}}
First Published 2018 December 1

\section{Draft 1}
As I mentioned \href{polymer-3.html}{yesterday}, today we're talking about crystals.
Polymers are known as semi-crystalline, because they have both crystalline and amorphous sections.
Some might wonder why polymers can't be wholly crystalline.\footnote{I was one of those}
As I\footnote{hopefully} have mentioned before, polymers are weird because more or less all of their strength comes from the inter-molecular forces that hold them together, as opposed to normal matter, which I have no clue how it works apparently.
Anyways, if a polymer is fully crystalline, it doesn't have any entanglements with other polymer chains, and so will become a powder, because each polymer is wholly self contained.
So, in short, they can it just isn't useful.

Semi-crystalline\footnote{which may here-fore be referred to as crystalline} polymers are especially useful when used in applications where the Heat Distortion Temperature (HDT) is useful, because for most crystalline polymers, the HDT is far larger than the T\sub{g}.
Another way to increase the HDT of a polymer is through addition of glass particles.
Those increase stiffness, and effectively just shift the modulus of the material directly upwards.
Since amorphous polymers lose strength so quickly around T\sub{g}, the addition of glass does nearly nothing.
However, crystalline polymers lose strength more slowly, so the addition of particles is helpful.

Crystals in polymers form as lamellae, the ordered regions within a polymer.
Lamellae clump together into spherulites, which are what looks like the crystal in a polymer.
In order for crystals to form, the polymer must not be atactic.

Polymers can only crystallize when below T\sub{m}, because that's what T\sub{m} means.
When below T\sub{m}, polymer crystallization rate is affected by two factors: nucleation and growth.

Nucleation is the formation of the spherulites.
It occurs more quickly the further below T\sub{m} you get, until T\sub{g} is reached, because then the polymer can't move.
Other ways to increase the rate of nucleation include strain/shear stress.
In a wholly homogenous polymer, the polymer will spontaneously crystallize, but that is slow.
Most polymers have a nucleating agent added to them, commonly sorbitol.
But, almost anything in a polymer, including dyes, other polymers, and glass particles can also act as a nucleation point.
In general, the more nucleation points there are, the smaller the final size of a spherulite.\footnote{this is also why Chem lab told me to cool my solution very slowly to make collecting crystals easier. If I could do it slowly enough, only one crystal would form}
Small spherulite size can be beneficial, especially if it can be made smaller than the wavelength of light, so a bottle can be see through.\footnote{optically transparent}
The rate of nucleation increases as T drops until T\sub{g}.

The other factor, growth, is exactly what it sounds like.
Once nucleated, the spherulites grow.
Their growth rate is fastest somewhere between T\sub{m} and T\sub{g}, though closer to T\sub{m}.
This is because as T increases, the flexibility of the polymer chain increases.
That makes it easier to align, but also easier to unalign.
So, by plotting the two rates together, you can find the optimal temperature to cool to in order to let a polymer crystalize.

If, after cooling below T\sub{g}, you realize the polymer needs to be more crystalline, then you can anneal it.
That is, you can raise it above T\sub{g} so that the polymer can form into crystals.

Fun fact: although incredibly regular, the chain stiffness of polycarbonate is such that it is not crystalline in consumer uses because of the time required.
\end{document}