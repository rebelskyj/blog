\documentclass[12pt]{article}  
\newcommand{\say}[1]{``#1''}  
\newcommand{\nsay}[1]{`#1'}  
\usepackage{endnotes}  
\newcommand{\B}{\backslash{}}  
\renewcommand{\,}{\textsuperscript{,}}  
\usepackage{setspace}   
\usepackage{tipa}  
\usepackage{hyperref}  
\begin{document}  
\doublespacing  
\section{\href{moma.html}{On MoMA}}  
First Published: 25 Septmeber 2025
\section{Draft 1: 25 September 2025}
Lasterday I went to the New York Museum of Modern Art.
As I had known, but never articulated, I have a favorite kind of art: non-representational art with phyiscally meaningful texture.
That is, I want to be able tosee the paint as paint, even if it is also acting as something else.\footnote{yeah, of course I like rothko}
While looking at a Rothko, I was inspired to write a poem.\footnote{not a good one, to be fair.}
That seemed like as good of a way to consider the best arts I liked.

So, below are the poems I wrote, in the order I wrote them.

1:\footnote{On Rothko's 37 and 19}\\
In the soft and subtle shading, as blue fades to deeper blue\\
A source of light unseen, unmentioned, seems to frame the vicious shape\\
These blues unseen in nature seem primed to bleed off the deep red sea\\
consume and so be consumed, the painting draws me in.

2:\footnote{On Dorothea Tanning's \say{Dogs of Cythera}}\\
Grasping hands and wailing mouths\\
eyes as from a fog\\
Some heaven-seen corrupted\\
by mortal misery\\
That blood-pure blue cuts through the clouds,\\
Where Adam should be reaching out\\
But Adam is gone, Creator too\\
what's left but bloody blue?

3:\foonote{On Tomie Ohtake's Untitled, 1982}\\
If Rothko screams in single shades\\
What is this apple's song?\\
One stroke of green, on deepest red\\
A mirror symmetry

4:\footnote{On Frank Bowing's \say{Raining Down South}}\\
An oil slick of global south\\
Metallic and contrived\\
yet strokes of brush still visible\\
As textures slowly wind\\
Or is it more auroral?\\
The Dancing night-time light?\\
Which reminds us of the dance we're on.\\ 
When borders fin'ly cease

5:\footnote{On Sam Gilliam's \say{10/27/69} (bonus fact, this one is the only one I cried at)}\\
What weight can any color hold\\
When gravity exists?\\
What bloody martyrs' final cris\\
Would echo on this painted shroud?\\
There's something in the sunrise scheme (scene? screen?)\\
Which brings me to my sobbing, weakened knees.\\
3 necks are bound like shirts or murdered men.\\
3 times I must review this\\
3 times, 3 views, 3 forms

6:\footnote{On Kay WalkingStick's \say{Teepee Form Drawing}}\\
Soemthing in the martyr's red on black\\
The highlight it implies\\
Or in the white of careless folds\\
Which frame on further pause\\
I could not but stop and stare

7:\footnote{On Giacomo Balla's \say{Swifts: Paths of Movement + Dynamic Sequence}}\\
the concrete does not speak to me\\
too fixed in vapid form.\\
To swim, to dance, a gambling lifetime chance\\
A body broken, bent

8:\footnote{On Judith Lauland's \say{Concrete 61}}\\
What suffering countless centuries have seen\\
caused rudely (hah) by the sacrifice, which suffering was meant to cease\\

9:\footnote{On Christopher Cozier's \say{Tropical Night}}\\
Fifteen by seven, pinned neatly in a grid\\
Fifteen by seven sketches of man, bread,\footnote{new line break added here bc feels appropriate}\\
Fifteen by seven, so Fifty some left out\\
Fifteen by seven shades of life and living\\
Fifteen by seven by one\\
Fifteen by seven, once alive, now butterflies, hung.

10:\footnote{On Joan Miro's \say{The Birth of the World}}\\
The tears creation wept\\
The tans of toxic waste\\
the figure of a little boy\\
pulled to the cage of the sky

11:\footnote{On Jose Clemente Orozco's \say{Dive Bomber and Tank}}\\
The threnody of misery\\
THe sextet of dispair\\
The chains of dying broken men\\
The gears of war spin ever and forever on without cease\\
A plane may crash\\
And more will die

12:\footnote{On Morris Grave's \say{English Nightfall Piece}, \say{French Nightfall Piece}, \say{Roman Nightfall Piece}}\\
This tryptich soon will haunt my dreams\\
Forms caged in static line\\
Yet closer inspection quickly uncovers\\
The forms are broken fragments\\
held, preserved in gem and love

13:\footnote{On Rebecca Allen's \say{Girl Lifts Skirt}}\\
It's rudely crude, a relic from a long-held past\\
An animation of a woman forced\\
Art from numbers, tabbed and punched in cards into machine

14:\footnote{On Lotus L. Kang's \say{Molt (Toronto-Chicago-Woodridge-New York-Los Angeles-)}}\\
I want to come another time\\
To see the mark I've left\\
in uncured light consuming\\
In magnetic holding up

15:\footnote{On Richard Serra's \say{Equal}}\\
Monuments to a fallen king\\
Made by his own two hands\\
The shape, though fixed, is rough and mixed\\
No two of six the same\\
Is man unable to imagine\\
Some great work not made of iron?\\
Or like the blood that others shed\\
Or like the callused worker's hands\\
Or Or like the song of hammerblow?\\

16:\footnote{On Mel Bochner's \say{Measurement Room}}\\
I wish my life was just this perfect\\
Clean lines to be defined with popped out text

Well, fun to see that I did very much get caught on metaphor, and fun to see the way that they shifted as I kept writing.
Even though this is the order I wrote the poems, it's not the order that I approached the works.
I ceased returning to the initial works after 12, because I was very lost, so 13--16\footnote{happy, those who care about dashes?} are from the memory and terrible photo of the pieces I have.
12 was the first of the paintings written about that I saw.

Anyways, love modern art, I should go more.
\section{Daily Reflection: 25 September 2025}

\begin{itemize}

\item Did you journal by hand today?

No, but I think that the poetry yesterday counts.

\item Did you do a folly?

Shoot.
\item Did you in some way, shape, or form advance the web novel?

...

\item Did you work on music, whether education or creation?

I think I read some small.

\item Did you work on another hobby?

Poetry!

\item Did you stretch? Really?

Eh, some.

\item Prayer?

\item Meditation?

\item Reading?

Reread a web serial and started a dumb rr-ku book.

\item Minimizing screen time?

Honestly, kind of.
Art museum good for that.

\end{itemize}

Current Pen List\footnote{for my own posterity, mostly. I should really start noting which pen is which.....}

\begin{itemize}  
\item Hongdian Black with Fude Nib: Diplomat Caribbean (8/30ish)  
\item Jinhao Shark: Diplomat Caribbean (8/30ish)  
\item Pilot Preppy: Private Reserve Electric DC Blue I think (I think since late june. I think)  
\item Sheaffer: Private Reserve Spearmint (since 7/15) (I Think)
\end{itemize}

\end{document}