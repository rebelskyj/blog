\documentclass[12pt]{article}[titlepage]
\newcommand{\say}[1]{``#1''}
\newcommand{\nsay}[1]{`#1'}
\usepackage{endnotes}
\newcommand{\1}{\={a}}
\newcommand{\2}{\={e}}
\newcommand{\3}{\={\i}}
\newcommand{\4}{\=o}
\newcommand{\5}{\=u}
\newcommand{\6}{\={A}}
\newcommand{\B}{\backslash{}}
\renewcommand{\,}{\textsuperscript{,}}
\usepackage{setspace}
\usepackage{tipa}
\usepackage{hyperref}
\begin{document}
\doublespacing
\section{\href{spem-1.html}{Spem 1st Try}}
First Published: 2019 January 30

\section{Draft 1}
This year, the Grinnell Singers is performing Thomas Tallis'\footnote{I never know whether that gets another s} work \textit{Spem in Alium}.\footnote{which I recently learned was mentioned in 50 shades of grey}
This piece has 40 parts.
Grinnell Singers has\footnote{approximately} 40 people.
So, I'm on my own part.

We tried singing it.
It went ok.
Thankfully, the piece is old enough that it's mostly three note chords throughout, so I could sound not completely wrong.
C'est la vie.
\end{document}	