\documentclass[12pt]{article}[titlepage]
\newcommand{\say}[1]{``#1''}
\newcommand{\nsay}[1]{`#1'}
\usepackage{endnotes}
\newcommand{\1}{\={a}}
\newcommand{\2}{\={e}}
\newcommand{\3}{\={\i}}
\newcommand{\4}{\=o}
\newcommand{\5}{\=u}
\newcommand{\6}{\={A}}
\newcommand{\B}{\backslash{}}
\renewcommand{\,}{\textsuperscript{,}}
\usepackage{setspace}
\usepackage{tipa}
\usepackage{hyperref}
\begin{document}
\doublespacing
\section{\href{rolls.html}{Rolls}}
First Published: 2018 November 24
\section{Draft 1}
I've realized lately that one of the biggest problems I'm beginning to have in my life is that everything means more than one thing.
So, for instance, the title of this post, \href{rolls.html}{\say{rolls}}, could refer to one of two things, both of which are applicable to today, which is odd.

The first and most obvious of these is the food kind.
Small little loaves of bread.
Today, I decided to try making rolls out of normal bread dough.
But, since I measured exactly nothing, I can't give the recipe.

The oven was set to its hottest temperature, probably 250\footnote{in non-freedom units because I'm in the fallen empire} and I kneaded the dough around 5 times over 3ish hours.
It baked until crunchy, and was then drizzled with butter.

The other kind of roll is the musical roll.
Although wikipedia defines music roll differently, in the Celtic Folk tradition, a roll is a way to break up notes.
As you probably know about bagpipes, they can't play staccato or tongued notes.\footnote{as always, if I don't mention the type it's Scottish}
So, to differentiate between a half note and two quarter notes, supplementary notes are played between them.
A roll is where you play the note, the note above it, and the note below it, in a quick fashion.
There's more to it than that, especially in terms of spacing, timing, and etc. but that's a basic\footnote{and inaccurate} summary.
I worked on making my rolls on the penny whistle\footnote{because of course an instrument you can articulate also uses these} today.

\end{document}