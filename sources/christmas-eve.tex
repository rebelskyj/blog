\documentclass[12pt]{article}[titlepage]
\newcommand{\say}[1]{``\#1''}
\newcommand{\nsay}[1]{`\#1'}
\usepackage{endnotes}
\newcommand{\1}{\={a}}
\newcommand{\2}{\={e}}
\newcommand{\3}{\={\i}}
\newcommand{\4}{\=o}
\newcommand{\5}{\=u}
\newcommand{\6}{\={A}}
\newcommand{\B}{\backslash{}}
\renewcommand{\,}{\textsuperscript{,}}
\usepackage{setspace}
\usepackage{tipa}
\usepackage{hyperref}
\begin{document}
\doublespacing
\section{\href{christmas-eve.html}{Christmas Eve}}
First Published: 2018 December 24
\section{Draft 1}
Christmas Eve has always been a weird day to me.
Mostly, this is because I accept that\footnote{for liturgical purposes} days start and end at sunset.
That's the reason for Vigil Masses and such.
However, Eve means \href{https://en.wiktionary.org/w/index.php?title=Special:Search&search=eve}{the evening before.}
So, the idea of Christmas Eve just seems weird to me.
The evening before Christmas is technically the 23.
\end{document}