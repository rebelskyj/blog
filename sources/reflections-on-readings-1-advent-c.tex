\documentclass[12pt]{article}[titlepage]
\newcommand{\say}[1]{``#1''}
\newcommand{\nsay}[1]{`#1'}
\usepackage{endnotes}
\newcommand{\1}{\={a}}
\newcommand{\2}{\={e}}
\newcommand{\3}{\={\i}}
\newcommand{\4}{\=o}
\newcommand{\5}{\=u}
\newcommand{\6}{\={A}}
\newcommand{\B}{\backslash{}}
\renewcommand{\,}{\textsuperscript{,}}
\usepackage{setspace}
\usepackage{tipa}
\usepackage{hyperref}
\begin{document}
\doublespacing
\section{\href{reflections-on-readings-1-advent-c.html}{Reflections on Today's Gospel}}
First Published: 2018 November 4

Luke 21:27 \say{And then they will see the Son of Man coming in a cloud with power and great glory.}

\section{Draft 1}
Happy first Sunday in Advent, New Liturgical Year, and First Night of Hanukkah!
Advent has always been my favorite season in the Church.
It helps that I have really 5 options: Advent, Christmas, Lent, Easter, and Ordinary Time.
But, Advent has always been nice because it's the period where we wait for a wholly positive event to occur.
Unlike Easter, which saves our souls at the cost of the Lord's life, Christmas is nothing more or less than a celebration of the birth of our Savior.

And yet, the Gospel today is not about Christ the meek child being born to a virgin, surrounded by shepherds.
Instead, it's a reference and prophesy about the Lord's next coming.
It tells us that we must stay awake and vigilant, and not be snared by the problems of our life.
And really, that's what Advent is about.
It's a preparatory season where we assess ourselves and begin to hope that we could be worthy to welcome the baby Jesus into the world.

It's interesting reading today's readings, because they fall really nicely in a chronology.
The First Reading talks about the first coming of Jesus, the season we celebrate in just a few weeks.
The Psalm and Second Reading tell us the message Jesus often spoke, that we are to be holy and loving through all that we do.
Then, the Gospel talks about the second coming.
It's a great reminder of the way we are to view Advent, which is a celebration of the Lord's birth, life, and second coming.
\end{document}