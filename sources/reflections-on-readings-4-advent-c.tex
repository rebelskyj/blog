\documentclass[12pt]{article}[titlepage]
\newcommand{\say}[1]{``#1''}
\newcommand{\nsay}[1]{`#1'}
\usepackage{endnotes}
\newcommand{\1}{\={a}}
\newcommand{\2}{\={e}}
\newcommand{\3}{\={\i}}
\newcommand{\4}{\=o}
\newcommand{\5}{\=u}
\newcommand{\6}{\={A}}
\newcommand{\B}{\backslash{}}
\renewcommand{\,}{\textsuperscript{,}}
\usepackage{setspace}
\usepackage{tipa}
\usepackage{hyperref}
\begin{document}
\doublespacing
\section{\href{reflections-on-readings-4-advent-c.html}{Reflections on Today's Gospel}}
First Published: 2018 December 23

Luke 1:45 \say{Blessed are you who believed that what was spoken to you by the Lord would be fulfilled.}

\section{Draft 1}
Today's Gospel, as befits the eve of Christmas Eve, is very focused on the coming of the Lord.
The first reading tells how Bethlehem will be where \say{one who is to be ruler in Israel} comes from.\footnote{Michah 5:1}
The Gospel takes place right after Mary accepts the angel telling her that she will become pregnant.

Her cousin Elizabeth sees her and recognizes what child she carries.
The Gospel ends with the message I hear in different variations almost daily from friends and others who profess the Christian faith.
However, none speak it\footnote{in my opinion} as well as Elizabeth does, as befits one consumed by the Spirit.
We are told many things by the Lord.
We are told that we will be blessed if we believe.

Of course, we also know that believing requires doing.
Without this belief, Mary could have believed that the angel was from God and still said no.
But instead, because she believed and trusted in the Lord, she accepted the angel's request, saying \say{May it be done to me according to your word.}\footnote{Luke 1:38}
\end{document}