\documentclass[12pt]{article}  
\newcommand{\say}[1]{``#1''}  
\newcommand{\nsay}[1]{`#1'}  
\usepackage{endnotes}  
\newcommand{\B}{\backslash{}}  
\renewcommand{\,}{\textsuperscript{,}}  
\usepackage{setspace}   
\usepackage{tipa}  
\usepackage{hyperref}  
\begin{document}  
\doublespacing  
\section{\href{knowledge.html}{On Knowledge}}  
First Published: 

\section{Draft 1?: 16 September 2025}
I have every assumption that this folly will be a long time until publiction.
Recently\footnote{as of this draft, at least}, I wrote \href{words-rare-knowledge.html}{a folly about words for rare knowledge}.
In it, I suppose that I've implicitly accepted Cartesian views on reality; knowledge is a real thing that exists independent of knowers.
In reading Christopher Smalls \textit{Musicking}, however, I am reminded that the idea of a mind-body duality is not the only way to view the world.
One of the readings I did during the short-lived period where a friend and I were to read a Chemistry Education Course also mentions this.
Despite the fact that I am \href{mindfulness.html}{trying to live more mindfully}\footnote{it's weird hyperlinking to this page before posting it, but I suppose it will come up.}, I felt as though I needed to stop reading and begin writing about knowledge.
So, then, let's see whether or not the words for rare knowledge I used can be used outside of this dualistic framework.

No, that's a topic for once I have a good grasp on other ways of seeing knowledge.

For now, I mostly want to jot down the fact that Cartesian duality is only one way to view the world.
I don't know if I believe in the immaterial.
That is not to say that I reject the soul or the existence of heaven.
However, what I do know is that the Church teaches that we can come to revelation from human reason.
Within the laws of nature, a non-physical cannot interact with a physical; doing so means that the non-physical must be physical.
I know that this is at the heart of a lot of discourse, and so I want to read up on some ideas of how to solve the issue before I tackle it in short.

I do right now, however, have some belief that all knowledge is fundamentally eldritch.
That is, all knowledge makes it so you see the world differently.
Once one learns how to solve a Rubix cube, they can never go back to rediscovery.

\end{document}