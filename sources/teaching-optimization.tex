\documentclass[12pt]{article}  
\newcommand{\say}[1]{``#1''}  
\newcommand{\nsay}[1]{`#1'}  
\usepackage{endnotes}  
\newcommand{\B}{\backslash{}}  
\renewcommand{\,}{\textsuperscript{,}}  
\usepackage{setspace}   
\usepackage{tipa}  
\usepackage{hyperref}  
\begin{document}  
\doublespacing  
\section{\href{teaching-optimization.html}{On Teaching Optimization}}  
First Published: 18 September 2025

\section{Draft 2: 18 September 2025}
A beloved friend recently asked the question about learning pedagogy, \say{is it not enough to just teach well, as best you're able?}.
I think that it's really evocative question, for a variety of reasons.
Not least, what does it mean to teach well, and what does it mean to teach as best as I am able?

I am able to lift far more weight when I learn proper form.
I am able to swim far faster with proper form.
I can learn far more adeptly with better strategies.
It is perhaps unsurprisng to me, then, that I believe that I can teach better by learning more.

Of course, the connection to better learning is probably the most apt.
It's incredibly tempting to spend hours upon hours trying to design a perfectly optimized study plan.
A \say{less optimized} plan, by contrast, is likely to generate more learning if those same hours are used to study instead.
Similar tension exists in the realm of teaching.

In my view, there are a number of things that can make it easier or harder to teach.
For example, I once taught a chemistry course where we taught oxidation numbers five\footnote{I think} weeks before the electron.
When students asked what it meant that Oxygen is minus two\footnote{two minus, I was supposed to be persnickity} means, I had no other recourse but to say \say{wait a few weeks}.
In general, a bad curriculum makes it nearly impossible to be a good teacher.
The best I could teach there was far worse than the best I could teach in the astronomy course where I had more curricular flexibility.

Still, the research which exists tends to cover a single class or few classes of students.
Sometimes these demographics are easily mapped onto the demographics of a different classroom.
Just thinking about the difference in classroom dynamic between my undergraduate and graduate institutions, however, it is also clear that this mapping may not always be useful.

So, then, is it seeking optimization for its own sake to learn the current best practices of pedagogy?

I think especially at the point in career my beloved is at, where there is no expectation of designing a curriculum, let alone teaching a full course, it's generally better to know more than less.
When it comes time to create a course, however, I think that it's likely productive procrastination at best to look for the optimal ways to design a course rather than spending the mental and temporal effort on structuring the course day by day and overall.
I hope my views are coherent?

\section{Draft 1.5: 18 September 2025}
What does it mean to be a good teacher?

A good teacher has good curriculum, good in class presence, and forms good connections with students.
I don't think that any of these are contrary to each other, and in general I even think that the three are linked.\footnote{wild how I had an identical number of characters in the first half of each of these sentences. Fun things about using a FW font}
So, then, what does it mean to be good at each of these?

Being good at connection seems the easiest to define, if only because it's the hardest to quantify.\footnote{fun how definitions and quantification are inversely correlated to me}
Students should feel safe bringing up questions, should want to attend class, and should learn.
Anything else then falls into the individual professor's preference.
As a person who has never had their place in the Academy questioned, I find it important that students do not see me as someone inherently above them.
For the professors I've had with different identities which lead them to not have this same implicit level of authority, the reverse is often needed.
Whether or not students should feel comfortable bringing up their interpersonal extra-class conflicts for advice is entirely up to the professor's discretion, in my mind.

Great.

What does it mean to have good in class presence?
I think that it means that classes are led in a way that is best for student learning.
This is where a lot of the education research comes, I think.
I think that it's also important to be more than a machine, though.
In order to connect to students, there needs to be flexibility in the class plan.

Finally, what does it mean to have a good curriculum?
Students have clear ideas of what they should learn, an effective path to the knowledge, and ideally the professor doesn't need to think much in the day to day.

When building a course, then, how important is it to stay exactly up to date?

I'd generally argue not very.
In general, if professors are willing to place themselves in the role of a learner, then most decisions are fairly easy to make.
After all, I know no one who thinks that they learn best from pure lecture.
So, I do not need to read the literature to know that I should maybe not do pure lecture.

Of course, then we get to that fun dichotomy of perceived and real learning.
If I remember correctly, flipped classrooms have significantly lower perceived learning but higher real learning.
In my general view, it's more important for students to learn than to feel like they've learned.
In some regards, it is almost ideal that they leave the course feeling as if they've learned nothing, because that implies a deep level of integration for all the knowledge.

Flipped classrooms have the secondary benefit of reduced workload.
Once lectures are recorded a single time, the professor is free to use them and work on in-class exercises.

Hmm, I guess that I am agreeing with the initial consideration.

\section{Draft 1: 18 September 2025}

To be clear, this folly\footnote{I should find a different word for these. Folly is a little too self-denigrating to be best for me, I think. Something to do with the fact that I start each one with \say{on}?} is not about the teaching of optimization.
Rather, it is written in response to a dear friend's consideration after attending a lecture on chemistry education.
The main thrust of the argument, as I read, at least,\footnote{mmmm nested phrasing} is that there may be an over-emphasis on teaching optimally.
Is this just another example of the modern optimization grindset?

My initial response is that trying to change my teaching to better help students learn is not an example of optimization gone too far.
One quote in particular stands out to me right now, \say{is it not enough to just teach well, as best you're able?}

I've reflected on what it means to do something to my best ability before, likely multiple times.
In light of that, let's quickly summarize what I think that I came to as a conclusion.
Just as editing and revising prose or poetry does not make it less authentic, neither does learning better strategies preclude doing the best I can.
Much like with lifting, where the best I can do is easily measured as the heaviest.
Learning better form is likely to improve the highest I can lift.

Certainly this can go too far, however.
When I think about what teaching to the best of my ability means, I think that it's a balancing act between more than a few different sinks.
What are they? Not totally sure, but this is a great place to explore.\footnote{also I love music I should never go for days without listening}

Obviously subject mastery is part of it.
There are arguments I could see for not updating material, especially introductory material, in light of new discoveries.
That is, I have a vague memory of learning that SN1 and SN2 reactions are not as evenly split as we were taught in school.
However, there's also something to be said for a shared base of knowledge.
Especially in science, where all knowledge is definitionally approximate, is there something wrong with the lie to children we teach being one we ourselves were taught?
I'm not totally sure.

Nonetheless, it is still obviously important to know the material one teaches.
If students ask questions that are outside of the lesson plan, then it is good to be able to give them a true and accurate answer, at least to their current skill level.
Also, being able to point to current developments in the field gives students who are extra motivated places to explore their interest.

Course planning is another aspect.
It's generally good, in my view, at least, to have a coherent arc of teaching through the semester.
This may not be totally up to the individual instructor, since I know that many places require certain skills to be taught in certain orders.
Even if not that, though, it's expected that students leave courses within a track with a certain set of knowledge.
I'd like to hope that the knowledge students are expected to carry with them is clearly laid out somewhere, even if only visibly to the department instructors.

Decisions about grading, assessment, and assignment also fall here, in my view.
Decisions about format, be it virtual or in-person or flipped or mixed or workshop are also under course planning.

At some very quick point, however, it's likely that working on course planning can fall into the trap of hyper optimization.
I agree that in general it's best not to try to live life totally focused on optimization.
Knowing that time is limited, however, I do also think that there's something to be said for prioritizing time.
The gains students might have between slight differences in assignment types are likely overshadowed by any number of other considerations.

At the more micro level, there's the individual lesson plan for each class period.
I've never had a strong one of these, but the fact that nearly every person with degrees in education I know uses them implies that there is clearly merit.
It was nice having a course where the handout at the beginning of each period was the expected learning and plans for the day.

I think that it is within the lesson plans that I see most of the research, and where I think optimization can quickly spiral out.
Is it better to spend five hours thinking about exactly how well the assessment tracks the pyramid structure or the same five hours coming up with other possible questions?
I personally think that the latter is better.

I also think that there is something in modern education research in the idea of pre-thinking.
That is, there's a benefit to knowing exactly what is going to happen in the lesson ahead of time, rather than having to extemporize day of.
If teaching the same course year after year, doing this once up front saves the time for future terms.

So, keeping up on the education literature would seem to have merit.
However, student experience is going to be different than the experience of the research.
There's a joke my brothers and I bring up fairly often, \say{people misunderstand the Stanford Prison Experiment; it isn't that people are monsters, it's that Stanford students are monsters}.
That is, different campuses will have different demographics and cultures.
My undergraduate, for instance, prioritized teaching us the bare minimum to be able to delve deeply into our interest later.
Or, at least, that's how it felt, especially since each major was only allowed to require like 8 courses.

So talking to your students and treating them as not interchangable is important.
However, the more interchangably students are treated, the more able one is to avoid falling into implicit bias.
Hmm, that's a hard one.

I feel like I've gotten rambly here, so let's try again from the top. 
Draft 1.5 is just going to be thinking about what it means to be a good teacher, and then Draft 2 maybe bring the two together.

\section{Daily Reflection: 18 September 2025}

\begin{itemize}

\item Did you journal by hand today?

yeah.

\item Did you do a folly?

Not yesterday. I meant to hit post but then I was really tired upon entering the train and I had a seatmate.

\item Did you in some way, shape, or form advance the web novel?

Nope.

\item Did you work on music, whether education or creation?

I read a bit of Musicking.

\item Did you work on book binding?

Nope

\item Did you work on another hobby?

Nope.

\item Did you stretch? Really?

Nope.

\item Prayer?

Eh.

\item Meditation?

Kinda.

\item Reading?

Eh.

\item Minimizing screen time?

Eh.

\end{itemize}

Current Pen List\footnote{for my own posterity, mostly}

\begin{itemize}  
\item Hongdian Black with Fude Nib: Diplomat Caribbean (8/30ish)  
\item Jinhao Shark: Diplomat Caribbean (8/30ish)  
\item Pilot Preppy: Private Reserve Electric DC Blue I think (I think since late june. I think)  
\item Sheaffer: Private Reserve Spearmint (since 7/15) (I Think)
\end{itemize}

\end{document}