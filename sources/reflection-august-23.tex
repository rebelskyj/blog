\documentclass[12pt]{article}[titlepage]
\newcommand{\say}[1]{``#1''}
\newcommand{\nsay}[1]{`#1'}
\usepackage{endnotes}
\newcommand{\B}{\backslash{}}
\renewcommand{\,}{\textsuperscript{,}}
\usepackage{setspace}
\usepackage{tipa}
\usepackage{hyperref}
\begin{document}
\doublespacing
\section{\href{reflection-august-23.html}{Monthly Reflection}}
First Published: 2023 September 3

\section{Draft 1}
And, like sand through an hourglass, my own hours have passed silently by.
Whoops, that's a little maudlin.
Let's try opening again.

Like the moon which rises and falls each day, waxing and waning over the course of a month, my own month was filled with intermediate highs and lows.\footnote{still too needlessly poetic, but I suppose it's good enough.
No, on second thought, let's try a third time.}

August, like a summer's day, felt so long in the moment and yet so short upon reflection.\footnote{Yeah, that's great.}
I gave my \href{universe-4.html}{last} few\footnote{final reflection to come!} talks for the summer!
I went to a concert!\footnote{which I don't think that I've reviewed.
I may or may not.}
I \href{ms-holmes-ms-watson.hmtl}{went to a play!}

I turned \href{twenty-five}{twenty five}!
I gave my accordion away to be repaired.
I competed in a pinball tournament!
I went to a friend's thesis defense!

I wrote a few letters.
I swam in a Great Lake!\footnote{and was beside a different one last night}
I visited my family!

Ok, so in retrospect, I suppose that I did actually do a lot.
It still doesn't feel like it, but as they\footnote{I, every month, at least once} say, c'est la vie.

Let's see how that list compares with what I said I was excited for.
\begin{itemize}
\item My birthday!
Hey nice I did say that.
\item Giving five talks in state parks.
I at least alluded to this!
\item Sharing lemon wine with friends
I did! I didn't think it was notable, though, but I suppose that it was. 
Once again, generally positive reviews\footnote{to my face at least}
\item Finishing my stack of letters.
I also had a passive aggressive comment about getting a letter back.
I not only finished the stack, I got three!\footnote{wow!} letters back.
\item Starting to do open mics regularly again
I think I did one or two.
I don't know if I'm in an open mic frame of mind anymore.
Maybe it will come back to my life again at some point, but it certainly isn't there now.
\item Staying up on all of my book clubs.
I dropped one book club, and otherwise I think I did manage to do that.
\item Making progress on research.
I did! My instrument makes plasma now, and my computational project got approved as a real thing I can devote actual energy to.\footnote{for those not in academia, don't ask.
For those in academia, I'm sure you know what I'm talking about.}
\item Monetizing my story?
I did not, though I did just get a review of my book that explicitly asked me to, so I guess that gets to go on this month's?
\end{itemize}
Looking at last month's goals:\footnote{commentary removed}
\begin{itemize}
\item Finish my presentation on the Pleiades.
Hey nice! I did it.
\item Make my home clean again.
I made progress, which is like a success.
\item Blog more.
Ummm, let me check.
I blogged three more times!\footnote{so 11 instead of 8.
The actual goal was more words, but I did end up another 12 percent less on word count.
(it occurs to me that a motivated reader could, in theory, use the word counts and percent decreases I've given in my musings to figure out how many words I wrote last month.
If you're the first one to do so and tell me, I'll get you an ice cream, or something of equivalent value}
\item Stretch daily.
I stretched seven times.
That is much less than daily, especially since I know that at least two of them happened in the same day.
\item Sleep enough and have a sleep schedule based around waking up no later than 6.
I think I did more of the former and less of the latter.
I guess I don't even know if that's true, upon thinking.
I don't wake up to alarms, but I don't feel well rested.
Maybe I need more vitamin D.
\item Be more intentional about prayer.
Oops!
This is becoming a perennial goal that I never reach.
Maybe this month!
\item Get further ahead on the book.
In particular, I'd like to set the goal of more than four chapters a week.
Well, I did not do that at all.
I fell slightly behind, I think, because I was a little ahead at the start of the month and ended exactly at cue.
As you might expect from the lack of blogging during a two week period, there was minimal writing of the book going on then.
\item Write poetry every day.
I wrote some poetry, though not much at all.
\item Finish stack of letters, and maybe think of more people to write to? Otherwise, start journaling during my morning time?
I finished the stack, though I didn't think of anyone new, and I did not start writing in a journal.
I did use my blog much more as a journal, though, so I suppose that kind of counts, right?
\item Write a song.
Wow! A goal that I reached with no qualifiers.
Not only did I write a whole song, but I even have vocal drafts\footnote{read: I recorded a voicemail while driving} of a few more songs.
\end{itemize}

Things that I'm excited for this month:\footnote{assume that this was written on the first, so I'll talk about things that have happened since.}
\begin{itemize}
\item Running more than five miles again!\footnote{I did it on the first! I think I pr'd}
\item Swimming with friends!\footnote{did that this morning/early afternoon. (it was a long swim time}
\item Giving my final talk in the parks for the season\footnote{I am ninety percent sure I've used the program name in this blog before, but if I haven't, this will not be the first time. I did it, in fact, last night! It went really well. Without (many) spoilers, there were like seventy people again!}
Coincidentally, this is also the final of the talk series for any speaker.
\item Recording a song!
\item Being ready for my upcoming talks.
\item Maybe\footnote{depending on bureaucracy} give or at least hash out details for me to give my first invited talk at a college I do not attend.
\end{itemize}
Using last month's goals, things I am excited about,\footnote{which I retroactively added above this sentence}, and my goals for this year of life as inspiration, this month's goals:\footnote{excluding things I've already done}
\begin{itemize}
\item Finish/make my talk on the eclipses.
I have a lot of ideas, and I need to bang my head against the wall of learning how to animate until that wall breaks or I lose the motivation to do anything.
\item Make my home clean again.
Listen, if I have the goal long enough, it might set in finally.
\item Blog more!
I'm off to a terrible start, given that I'm writing this reflection on the third, but.
\item Stretch and exercise more.
I know that it is incredibly important to my mental health that I remain physically active.\footnote{to say nothing of my mental health.}
I also know that I hate how inflexible I am generally, and especially how inflexible I am now.
\item As before, sleep enough, and try to prioritize sleeping earlier.
I know that my best sleep happens earlier in the night.
I might try experimenting with biphasic sleeping\footnote{taking a nap} since I seem to remember that working for me at some point in my life.
\item As always, be more intentional about prayer.
Once more off to a bad start, but I really do feel better when I do.
I think that spending at least fifteen minutes in the chapel each day\footnote{from now on, at least} is not an unreasonable goal.
There isn't anything that I need to do that I can't push fifteen minutes.
\item Actually get ahead on my book.
I would like to be five chapters ahead and to have plotted out the rest of this arc.\footnote{which as of now, I have seventy two (I think, off by one errors are quintessential to my experience as a counter) chapters left, and I'm falling behind on the (completely minimal) plot that I've meant to write}
\item Write more poetry.
Daily may not be realistic, especially since I haven't written much this month.
\item Write or record a song.
I have the year's goal of recording an album.
If there are twelve songs on my album\footnote{which feels like a reasonable number}, I need to record on average a little more than a song a month\footnote{since I, you know, didn't record any last month}.
I also want to do almost exclusively, if not actually exclusively, originals on my album.\footnote{original arrangements of folk songs are maybe allowed.
I'm also going to need to decide how I feel about an instrumental song with CGS (Computer generated sound). 
I think positive, as long as I actually put some effort into mixing, especially since that's where almost all of the sounds are going to have to come from, since I don't own/know how to play many of the instruments that I would like to include on the album}
\item Write five\footnote{I initially wrote ten, but that's insane.
I'm not going to generate 280 (since it's the third) things I like about myself.
That feels a little too much.
190 seems much more reasonable.} things I like about myself every day\footnote{starting now}, three things I'm excited for, and ten things I'm grateful for.
I keep seeing things about cultivating positivity, and I really need to do that right now.
\end{itemize}
Given these goals, the questions that I will need to answer each day:
\begin{itemize}
\item Did I make progress on my eclipse talk(s)?
\item Did I fight against the entropy in my living space?
\item Have I been better about blogging?\footnote{honestly such a fun question because I can really only answer it in the positive}
\item Did I stretch?
\item Did I exercise?
\item Did I prioritize sleep?
\item Did I wake as early as I'd hoped?
\item Did I spend fifteen minutes in the chapel?
\item Am I writing more than the bare minimum for my book?
\item Am I making net progress on plotting my book?
\item Did I write poetry?
\item Did I work on my album?
\item Did I write five things I like about myself?
\item Did I write three things I'm excited for?
\item Did I write ten things I'm grateful for?
\item Generally, did I cultivate joy?
\end{itemize}


And, as a day in September\footnote{how is it already -ber months??}, today's responses:
\begin{itemize}
\item I did not make progress in my talks, though last night I did realize how badly I explain them without props or images, which is like progress.
\item I have not, and may or may not today, honestly.
\item I mean, at the current rate of 1/3 days, no. I will not exceed 11 posts this month.
\item I'll stretch after this post.
\item I swam with friends!!
\item Oof.
I mean, in my defense, I forgot an eye mask and was camping.
In my offense\footnote{I still have no idea what the right term for this is}, I could've remembered that or taken a nap at any point today.
\item I woke up a few times\footnote{see: camping}, and got up around six thirty I think.
That feels fine.
\item I'll go and do so after stretching/before I go to Mass tonight.
\item I wrote a chapter, which is arguably more.
I'll try to at least plot out my goal for the next chapter\footnote{and, importantly, actually follow that plotting, rather than this last chapter where I added fifteen hundred (maybe only a thousand) words about the MC's family because of a review.}
\item I mean, in the sense that I plotted four points to hit in the last chapter and hit one, yes.
\item No poetry. I'll see how I feel tonight.
\item I realized on the drive last night that at least one song needs to have a really driving piano beat.\footnote{like straight 8ths, parsimonious voice leading through the chords I'm going to use in the song} and I thought about what I might want, so kind of.
\item I typed, because I forgot my pen outside of my home and it's not worth getting up to go get it right now.
\item Apparently ibid does not, in fact, like I always thought, mean \say{see above.} 
It instead means previous citation.
I'll keep using it my way, though since it works.
Anyways, that's a long winded way of saying ibid.
\item ibid.
\item I don't know;
I think I tried, at least a little.
\end{itemize}
Oof, that was a bit of a marathon post.
Sorry.



\end{document}