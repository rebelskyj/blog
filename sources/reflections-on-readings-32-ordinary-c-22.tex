\documentclass[12pt]{article}[titlepage]
\newcommand{\say}[1]{``#1''}
\newcommand{\nsay}[1]{`#1'}
\usepackage{endnotes}
\newcommand{\1}{\={a}}
\newcommand{\2}{\={e}}
\newcommand{\3}{\={\i}}
\newcommand{\4}{\=o}
\newcommand{\5}{\=u}
\newcommand{\6}{\={A}}
\newcommand{\B}{\backslash{}}
\renewcommand{\,}{\textsuperscript{,}}
\usepackage{setspace}
\usepackage{tipa}
\usepackage{hyperref}
\begin{document}
\doublespacing
\section{\href{reflections-on-readings-32-ordinary-c-22.html}{Reflections on Today's Gospel}}
First Published: 2022 November 6

Luke 20:27 \say{Some Sadducees, those who deny that there is a resurrection, came forward and put this question to him,}

\section{Draft 1}
Today's readings strike home for me.
Mostly to me, the readings focus on legalism.

The Maccabees could be accused of dying on behalf of legalism.
After all, is the commandment not to eat pork really more important than the commandment to live?
Similarly, in homilies I've heard, I've generally seen malice attributed to the Sadducees.

The framing is always them trying to trap Our Lord in some sort of a logical quandary.
I guess I don't see an issue with that.
One thing that we know for certain about the Lord and His Promises is that they are consistent.
If we can reach different answers from different\footnote{and true} starting facts, that is a legitimate concern about the faith.

And so, I choose to see the Sadducees as truly wondering.
In the world view that they have, there is no way for this woman to be married to all seven men, but also no way to choose between them.
That's an obvious problem to the resurrection of the dead.

Our Lord's response is incredibly stunning.
Rather than rebuke them, He answers their questions in full.
More than that, though, He answers the questions that they were trying to ask as well.

It really reminds me of how one person's legalistic question is another's genuine stumbling block to belief.
I saw a post recently that said the incredible specificity with which all of our most dearly held beliefs have is a sign that there's something wrong with the Church.
I see it as the opposite: we know better and better what the true Truth is, because we know what is not it.
\end{document}