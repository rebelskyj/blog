\documentclass[12pt]{article}[titlepage]
\newcommand{\say}[1]{``#1''}
\newcommand{\nsay}[1]{`#1'}
\usepackage{endnotes}
\newcommand{\1}{\={a}}
\newcommand{\2}{\={e}}
\newcommand{\3}{\={\i}}
\newcommand{\4}{\=o}
\newcommand{\5}{\=u}
\newcommand{\6}{\={A}}
\newcommand{\B}{\backslash{}}
\renewcommand{\,}{\textsuperscript{,}}
\usepackage{setspace}
\usepackage{tipa}
\usepackage{hyperref}
\begin{document}
\doublespacing
\section{\href{defining-diaries.html}{Defining Diaries}}
\section{Draft 2: 19/20 October 2018}
Diaries, like many aspects of stereotypical female adolescence, are a nebulous concept.\footnote{what differentiates a horse from a pegasus from a unicorn? If a unicorn has wings grafted on, does it stay one?}
But, like any scientist, I hate undefined concepts.\footnote{well, hate's the wrong word, maybe \say{am uncomfortable with}}
So, I've been thinking about what a diary really is.

I tried defining it a few different ways.
First was through exclusion.
In \textit{A Brief History of Diaries,} \textit{Meditations} by Marcus Aurelius is considered a diary, as is a shopkeeper's logbook.
Both of those didn't seem to fit what I would consider a diary, but I couldn't figure out why.

Then came the idea of\footnote{like a good little linguistics child} splitting all types of writing into binary divisions.
It took a few tries, but I think I've gotten something that works.
Of course, writers hate dichotomies,\footnote{not that they're special, just that the two English people I said the idea to were immediately hostile, and I've never tried grouping music into categories, although the pushback to the Hornbostel--Sachs doesn't seem to be very major to me (although I'm also almost 60 years late to that fight [wow the 1960's are more than 50 years away] so what do I know)} so I needed to make sure that each category was a real dichotomy.

If we look at all writing that is, was, or will be,\footnote{as far as I can guess} there are only two kinds of writing.\footnote{ooh maybe this would make an opening line}
There is writing concerned with reality,\footnote{non-fictive writing} and writing that isn't concerned with reality.\footnote{fictive writing}
Of course, an immediate push back to this would be \say{what about writing that is partially concerned with reality?}
That's non-fictive.
It is concerned with reality.
It's like saying \say{a room either has some form of lighting, or it doesn't.}
Dimmer lighting, or lighting only in one corner still satisfies.

Obviously,\footnote{wow, the more I write the more I realize how dangerous that word is. Just because I'm finding something obvious as I write about it doesn't mean that everyone ever will} diaries have a focus on reality, so they are non-fictive.
Next, all non-fictive writing is either focused on a single entity, with the rest of creation\footnote{another word I'm realizing may be taken differently than intended} serving as a backdrop to it, or not doing that.

It may be helpful to give examples of writing in each category.
In the biographical category, there are diaries and biographies.
In the non-biographical category, we see things like epic poetry.\footnote{ok so this one may be a bad kind of dichotomy, because it seems completely arbitrary. I can't think of anything that I couldn't argue isn't concerned with a single entity unless I don't allow breaking, which doesn't work later. I also got into the whole internal dialogue of \say{if you believe that God is completely incomprehensible, and that we only see aspects, and every religion sees its own aspect, why do you believe you're in the \nsay{One, Holy ... Church}?}}

So if I ignore that, all writing concerned with reality is either narrative or not.

BREAK

Narrative writing is writing concerned with telling a story.
Think about a typical biography.
Conversely, there are non-narrative writings, such as instruction manuals.

Within narrative writing, there is autobiographical and not.
Autobiographical writing is writing concerning the author.
So, a member of an ethnic group writing an ethnography of the group, or authors writing about themselves.

Finally, all autobiographical writing is w

To me, a diary\footnote{in order from most to least specific} is a non-retrospective, autobiographical, narrative, biographical, non-fictive piece of writing.
In order, I'll explain each of these pieces, from least to most specific.

As a person who's taken Linguistics, I understand the importance of grouping things into dichotomies.
As an artist, I understand the frustration of false dichotomies.
So, I've attempted to group all writings into mutually exclusive categories.
First is fictive and null-fictive writing.

Confusingly, my definitions are based the opposite way here.
Null-fictive writing is writing concerned with portraying reality, while fictive writing is not.

All writing concerned with reality either has a singular entity at its focus: biographical, or not, null.

All biographical writing is either narrative: biographies and whatnot, or null: blurbs.

All biographies are either autobiographical: about self, or not: all other biographies.

Finally, all biographies are either retrospective: all the action takes place before writing, or not: diaries.

\section{Draft 1: 17 October 2018}
What's a diary?
To me, a diary\footnote{in order from most to least specific} is a non-retrospective, autobiographical, narrative, biographical, non-fictive piece of writing.
In order, I'll explain each of these pieces, from least to most specific.

As a person who's taken Linguistics, I understand the importance of grouping things into dichotomies.
As an artist, I understand the frustration of false dichotomies.
So, I've attempted to group all writings into mutually exclusive categories.
First is fictive and null-fictive writing.

Confusingly, my definitions are based the opposite way here.
Null-fictive writing is writing concerned with portraying reality, while fictive writing is not.

All writing concerned with reality either has a singular entity at its focus: biographical, or not, null.

All biographical writing is either narrative: biographies and whatnot, or null: blurbs.

All biographies are either autobiographical: about self, or not: all other biographies.

Finally, all biographies are either retrospective: all the action takes place before writing, or not: diaries.
\end{document}