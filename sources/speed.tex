\documentclass[12pt]{article}  
\newcommand{\say}[1]{``#1''}  
\newcommand{\nsay}[1]{`#1'}  
\usepackage{endnotes}  
\newcommand{\B}{\backslash{}}  
\renewcommand{\,}{\textsuperscript{,}}  
\usepackage{setspace}   
\usepackage{tipa}  
\usepackage{hyperref}  
\begin{document}  
\doublespacing  
\section{\href{speed.html}{On Speed}}  
First Published: 
\section{Draft 1: 23 September 2025}
Last night I went to dinner with some college friends.
It was a wonderful time, and as may be unsurprising, we eventually got ontot he topic of media we consume.
I mentioned that I consume my media at multiples of the nominally intended speed, and they expressed potential interest in it.
Of course, I have also received any number of objections to this method from others in my life.

The most salient objection I can think of is that the speed something plays does a lot to communicate.
Humor is reliant on timing, and so on.
For whatever reason, I find this only to be true for me in musical terms; a song or musical interlude played faster or slower feels fundamentally different in a way that speech faster or slower does not.
Perhaps this is because speed is relative in most vocalizations but absolute in music.

Regardless, I think that the point of creator's intent is a reasonable objection to my consumption.
Of course, there's much to be said for the fact that I consume almost nothing as it was originally intended.
All music written before the 1900s was intended to be played and enjoyed live.
Any music outside of the Western canon is and was meant to be enjoyed actively, whether via dance, participation, or verbal appreciation.\footnote{is it clear I'm reading about music right now?}

When I read a book, there is almost no chance that I read the book at the exact pace an author assumes.
Then again, as an author, I do not think that I have a pace that I intend my book to be read.
There's an inherent serial nature to much of this writing, and especially my web serial.
That being said, given that many serialized fiction is presented and resold as a single compendium, it's clear that's not important either.

Do directors and creators of film and television intend for shows to be at the speed they are?

Does what they intend reflect in what they create?

I am reminded of the finale of Game of Thrones, which is now infamously mocked for having absolutely atrocious lighting.
Audio in general these days is poorly mixed and recorded, and much of the art of the process has been lost.
If I am watching something where craft was not a major part of the process, how much should I respect the craft?

When we talk about the way that nothing lasts, part of it is due to intentionally planned obselescence.
However, there is also the inherently craftless nature of what we produce.
Few people I know can or would willingly create something intentionally subpar.
It is only in this current economic and industrial model, where creation is so fundamentally removed from human touch, that planned obselescence can be possible, or that clothing which disintigrates upon washing can exist.

This is not to say that I think premodern society was better than our current one.
This is to say, however, that consuming modern media at speeds outside of those the director may have designed does not feel wrong to me.

Taking it even one step further, I think about the fact that music has been almost completely subsumed in the craft of film.
Music does not exist for anything except for the emotional responses we have been trained to have to it.
Society\footnote{rightly, I'd argue} rails against Muzak, and so I see nothing wrong with railing against the emotionally manipulative nature of music in television and film.

Again, though, I can consier older works, where lighting and sound were considered.
They would never have imagined me watching on a computer screen.
Should I also not watch it like that?


\section{Daily Reflection: 23 September 2025}

\begin{itemize}

\item Did you journal by hand today?

No but yesterday.

\item Did you do a folly?

Woot!
\item Did you in some way, shape, or form advance the web novel?

No

\item Did you work on music, whether education or creation?

Read more small

\item Did you work on another hobby?

Drew some yesterday

\item Did you stretch? Really?

A lot!

\item Prayer?

\item Meditation?

\item Reading?

Just small and some Bluets.

\item Minimizing screen time?

Ehhhhhhhh. No

\end{itemize}

Current Pen List\footnote{for my own posterity, mostly}

\begin{itemize}  
\item Hongdian Black with Fude Nib: Diplomat Caribbean (8/30ish)  
\item Jinhao Shark: Diplomat Caribbean (8/30ish)  
\item Pilot Preppy: Private Reserve Electric DC Blue I think (I think since late june. I think)  
\item Sheaffer: Private Reserve Spearmint (since 7/15) (I Think)
\end{itemize}

\end{document}