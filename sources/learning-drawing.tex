\documentclass[12pt]{article}[titlepage]
\newcommand{\say}[1]{``#1''}
\newcommand{\nsay}[1]{`#1'}
\usepackage{endnotes}
\newcommand{\B}{\backslash{}}
\renewcommand{\,}{\textsuperscript{,}}
\usepackage{setspace}
\usepackage{tipa}
\usepackage{hyperref}
\begin{document}
\doublespacing
\section{\href{learning-drawing.html}{Learning to Draw}}
First Published: 2024 December 13
\section{Draft Two}

Visual art and music have always felt completely separate to me, in a way that I don't really think is true of most people.  
Given the fact that both are what people will point to as \say{the arts}, or \say{fine arts}, there's at least some level of correlation.  
I can understand part of it.  
After all, both are endeavors which fundamentally require conveying something within ourselves to the world outside, in a way that isn't as true for other activities.  
Both are seen as fundamental to humanity.

As I think about why I consider the two forms so different, I think that the artifacts of practice are incredibly important.  
When practicing music, there is no direct evidence once you have finished.  
The notes fade, sound waves dampened by everything around you.  
When finishing a practice of visual media, by contrast, the page is full of every mistake you've made.

The fear at the idea seeing my own mistakes, and the pain of actually witnessing the work I was proud of a day ago has certainly been a discouragement from my continuing the practice.  
I also don't know that I've ever really figured out a good way to practice the skill.  
With writing, for instance, I know that everything I write will, on some level, improve all of my writing.  
I also have relatively clear demarcations for what practice will most impact what part of my craft.\footnote{at least in theory. Whether or not those are borne out by reality is another matter entirely}  
The same is true of music.

In drawing, by contrast,\footnote{i keep wanting to just say art, which I know is wrong, and also I'm only really interested in pencil or pen drawing or digital drawing, so might as well just use that} I know that line and shape and form are important, but I still don't see how studying one thing inherently leads to improved ability to draw something else.  
On the same thread, I also do not have anything that has ever been a huge motivation as a relatively large project.  
Even to this day, a motivation for at least some of my music practice is upcoming public performances, which weigh heavily on me.  
I write as a part of my job.  
Drawing is not something I've done.

And yet, this musing is not about my historical struggles with drawing, it's about my current goals to learn how to draw.  
I've been working on the skill since about the day that my mother died.  
I think that it was the next day that I got a sketchbook and started to draw.  
Primarily I've been focusing on figure drawing.

I don't really know why that is.  
Certainly I like figures in art, and at least some of the art that I want to create has the human body as a part of it, but I think that a larger reason might just be that the media about learning to draw I've consumed lately has been focused on people's own goals to learn to draw form.

As I continue to draw the human form, though, I am more and more finding the ways that small gestures really do mean the difference between something completely discordant and something pleasant to look at.  
Simply thickening a line where there is shadow in a reference image adds a surprising amount of depth to the drawing.  
All this to say, I feel like I'm making progress, even though I don't really want to go back and see if it's true.  
I'm hoping to dedicate some time as I continue to move forward in my life to actively studying how to do it better.

  


\section{Draft One}  


Interestingly, it seems that I've only ever mused about drawing a single time, almost three years ago now.\footnote{I hate that 2022 is three years ago in just a few short weeks}  
As my goals right now hopefully indicate, I've decided that right now I want to learn to draw.  
I don't think that this is a new goal, but it's never been something that I have put for a sustained effort on.

I'm really not sure why I never really learned how to draw.  
For reference, when I say draw here, I'm using it in the sense of still life or figure, or generally of something vaguely resembling realism.  
I've done plenty of non\-representative art\footnote{representative was the word I was looking for}, and I have run up against the borders of insanity more than a few times constructing a knot.  
Despite this, I still don't really think that I can draw a guitar, even though there is one sitting in front of me at this exact moment and I can perfectly picture one in my mind.

When reading about learning to draw, a lot of the stories end up similar to those that I see in music spaces from those who enter later in life.  
Someone, either an authority figure or the general air of authority, convinced people that they were not musical at some fundamental level.  
Probably because I am a relatively competent musician\footnote{it's wild what things I do and don't feel comfortable claiming. I have never felt good about saying that I'm good at music, except when interfacing with someone who I think is wrong about their opinion}, I do not think that was ever the case for me.  
I also do not lack inspiration for things I would draw.  
My mind is filled with countless fantastical images that I wish I could convey to the world.

Especially since I have spent the past six\footnote{Oh gosh this blog is old} or so years working on my penmanship, it seems more than a little strange that I haven't really spent a lot of time with drawing.  
Of course, paging through my old notebooks does show a good number of pages with different amounts of pedagogically sound drawing practice.  
Unlike the music that runs through the books, however, there is no through line.

Goals:  
\begin{itemize}  
\item One offs:  
\begin{itemize}  
\item Talk to boss about Ph.D. timeline  
\item Pick a topic for a science communication article  
\item Make a list of the stretches I'll do each day  
\item Find a place to volunteer  
\item Paper hit list  
\item Compile a list of people I want to write letters to  
\item Muse about macros and micros  
\item Compile a list of 20 meals that I can make, with their ingredients (inc. shelf stable or lifetime), time, effort level, and nutrition info  
\item Figure out my motivation for each book and have it as the bookmark  
\item List of things that need to be cleaned and the frequency  
\item List of things in my life  
\item Make a list of musings to do  
\end{itemize}  
\item Weekly:  
\begin{itemize}  
\item Spend 30 minutes 2x a week working on writing the song \-\> will work tomorrow  
\item Ten minutes 4x a week on drawing \-\> Did extra today!  
\end{itemize}  
\item Daily:  
\begin{itemize}  
\item Define how I'm feeling each day at start and end \-\> So far so good  
\item Practice guitar daily (at least one scale and a chord progression) \-\> Working on it!  
\item Muse daily \-\> Wow a threepeat  
\item Stretch Twice a day \-\> Second stretch is becoming more and more abbreviated but still extant!  
\end{itemize}  
\end{itemize}

\end{document}