\documentclass[12pt]{article}[titlepage]
\newcommand{\say}[1]{``#1''}
\newcommand{\nsay}[1]{`#1'}
\usepackage{endnotes}
\newcommand{\B}{\backslash{}}
\renewcommand{\,}{\textsuperscript{,}}
\usepackage{setspace}
\usepackage{tipa}
\usepackage{hyperref}
\begin{document}
\doublespacing
\section{\href{reflection.html}{Reflecting on Reflections}}
First Published: 2023 July 30


\section{Draft 1}
Prereading note: this is incredibly rambly, because it's A: being written at the speed of thought, and B: intentionally longer than I generally make musings.\footnote{for reasons that are between me and myself}

So, I've realized that I'm feeling really self introspective right now.
There are a number of reasons, I'm sure.
I'm aware of at least a couple, though.

I took a long nap earlier today.
I don't know whether it's the exhaustion which requires a nap or the act of napping, but I find that my days spent after a nap are much more inward facing\footnote{emotionally, at least} than the norm.
More than that, though, when I woke from my nap, I realized that nothing on my to do list was actually something I needed to do.
That was really freeing, because it let me ask myself whether I actually wanted to do what I had said.

In a number of cases, I did.
I wanted to exercise, though I really did not\footnote{and almost never do} want to run.
Instead, I found an online yoga class, which was probably a better use of my time.\footnote{wow I'm way more stiff than I thought I was}
I wanted to eat, and so got to be mindful about what I cooked.
I did actually want to write a chapter of my book, though I did not want to write two.\footnote{Or, at least, the amount of time left in the day made me not want to.}
I wanted to write a letter to a friend, and now I've written it.

Of course, that is another reason for my introspection.
I tend to write my letters first thing in the morning,\footnote{or at least firstish thing after leaving my bed. Often that's an hour after waking, but that's neither here nor there} in part because I find that the walk to work helps to recenter me.
Without that centering, I was forced to sit and be with my thoughts.
I also wrote a short journal for the first time in far, far too long.
It was a short musing, but it was good for me to get back into the practice, at least a little.

Also, despite the fact that I went to Mass and then brunch with friends, the rest of my day has been spent in isolation.
Especially since the nap somewhat reset my sense of day, I really feel like I've been alone this whole day.
Interesting, unlike other times that has happened, I do not feel bad about it.\footnote{or as a result of it}
That's likely due to the yoga, since staying in is extremely correlated for me at least with not exercising.

And so, with the perfect storm of everything leading me into reflection and nothing preventing me from posting a musing, it seems appropriate to muse on reflecting.
This post could equally be about musing, or about the purpose of this blog.
I opened my journal entry with \say{Dear Future Me,} which I know I've talked about before as my primary intended audience.
By explicitly framing it, though, I realized that there are a number of questions I have for my future self.\footnote{which are in my journal, and will not be reproduced here.}

It really made me think about what changes I've gone through in my time being explicitly introspective.
I've come home from a time abroad.
For all that it was absolutely a real experience, it does often live in my memory as somewhat of a hazy dream.
Even when I read my explicitly chronological musings, I find that I disbelieve the chronology of my time there.

After that, I really began doing research far more legitimately.\footnote{I know that's the wrong word, but it's the one that I have right now.}
A summer spent bonding with other chemists really made me realize that my passions were best served going into chemistry, rather than\footnote{explicitly, at least} music.

I lived through a massive pandemic.\footnote{though, despite the fact that the world is over it, it has not really ended in any legitimate sense}
During that pandemic, I only just missed the second most destructive storm in American history.\footnote{and lived through a fair amount of the immediate rebuilding}
I moved to a new place, further from my family than I ever had been for an open ended period of time.\footnote{abroad, while further, was an explicitly predetermined length of time}

I started dating someone seriously for the first time in my life.
I started a band.
I got a Master's\footnote{I hate that it's possessive, but that's the nature of language I suppose} Degree.
I became a PhD candidate.

I have gone through half a dozen journals, changing my penmanship in each.
Within those journals are an interesting cross section of my life.
Entire courses of notes are contained within them.
Other courses are completely removed from them, because I decided to take notes on other platforms.

There are poems and songs aplenty, for all that I don't ever plan to revisit most of them.
I do often think about the fact that I used to be able to write a sonnet a night without too much thought.
What happened in my life that I feel busier than I ever have before, despite doing what feels like far less?

In college, I maintained a high GPA while taking overloads of courses, doing sports and music, and having a social life.
In graduate school, I no longer take classes, I don't do any explicit sport or music, and I feel like my social life is nearly nonexistent.
But, I do also write a book now, which is gradually becoming less weird for me to claim.

After all of the writing advice that my father tried to instill in me, I stopped writing in a five paragraph essay mode.
Of course, that meant that I began having far more paragraph breaks.
For a while, I think likely due to the book I was writing, that meant that I would write paragraphs all of three sentences each.
As I become\footnote{became? great question on tense there} more aware of that trend, I've tried to buck it.

And yet, what's the point of everything that I do?
In five years, I will likely never revisit my journals.
In ten years, I may forget that I ever had a blog.
If I have children, my thoughts at a time before they existed might be a comfort to them when I'm gone.
Of course, that presumes that my journals will even survive that long.\footnote{to say nothing of whether or not I'll have children}

What about the book that I'm writing, though?
I enjoy writing it, at least sometimes.
I like the nice comments that I get.\footnote{though wow the negative comments hurt way more than the nice comments make me feel better}
I know that there's the whole \say{not every hobby has to be money producing to be valuable,} but I still sometimes wonder about the value of my book.
My journaling and blogging, at least, help me reflect explicitly on what's going on in my life and hopefully\footnote{though I don't know how often it actually happens,} lead me closer to G-d and an eternity in His loving embrace.
Writing my book does not make me feel holier,\footnote{not feeling in the like \say{wow I feel so holy when I'm sleep deprived and then we have song time} way that camps do it, but in the intellectual way of being able to see where something is helpful to my sanctification. It also does not do the former} which probably isn't great.
On the other hand, writing my book does make me write more.
And, at least\footnote{as of writing this (part of this) post at 2033} 879 people enjoy the book enough to follow it.
I suppose that making other's days brighter is worth a lot.

When I look back at myself this time next year, what will I wish that I spent more time on?
I have to imagine that it will be similar things that I wish past me had spent time on now.
I wish that I would have practiced art, though I can never really explain why.
I wish that I would be better at organizing.
And, despite the fact that one of my most common compliments\footnote{complements? I can never keep the two separate} is that I am good at keeping in touch, I wish I did better at that.

So, where does that leave me?
I'm not totally sure.
What I do know is that, for all the choices past me made that I'm less than thrilled about, I can't deny that I ended up where I feel like I should be.
The people I've met throughout this journey remain important to me.



\end{document}