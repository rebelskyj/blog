\documentclass[12pt]{article}
\usepackage[left=1in, right=1in, top=1in, bottom=1in]{geometry}
\newcommand{\say}[1]{``#1''}
\newcommand{\nsay}[1]{`#1'}
\usepackage{endnotes}
\newcommand{\1}{\={a}}
\newcommand{\2}{\={e}}
\newcommand{\3}{\={\i}}
\newcommand{\4}{\=o}
\newcommand{\5}{\=u}
\newcommand{\6}{\={A}}
\newcommand{\B}{\backslash{}}
\renewcommand{\,}{\textsuperscript{,}}
\usepackage{setspace}
\usepackage{tipa}
\usepackage{hyperref}
\begin{document}
\section{\href{dance-nation-essay.html}{Dance Nation Essay}}
Prereading Note: This is being written for a class, and began as an adaptation from \href{dance-nation.html}{this post.}
Each draft will also end with a word count, as there is a requirement there.

\section{Draft 6}
Love, loss, dance, and werewolves all feature prominently in Clare Barron's \textit{Dance Nation} at the Almeida Theatre, which follows a children�s dance squad as they prepare for national tournaments. The show places a large emphasis on the sexualization and sexuality of young women, as well as their adolescence. Recommended for ages 16 and over, the show artfully weaves the lives of adults into their childhood recollections, creating a layered effect that both alienates and invites the audience.

Clearly, this dichotomy of alienation and invitation was a goal of both Barron and Director Bijan Sheibani, as every aspect of the show pushes the audience into this. From the cast's age, to Samal Blak's set design, to Lee Curran's lighting, the show leaves the audience in a constant state of confusion of time, space, and location.

The show begins with a blank set, and so the review follows. Before the show, the crowd sees a curtain lit to the nebulous phase between translucence and transparency. A mirrored wall behind can barely be made out, with lights running up the seams. As the show opens, the curtain drops, never to be seen again. Mirrors make up the entirety of the back wall, with lights strung between each. Throughout the show, these mirrors put the audience on stage, forcing them to see and therefore confront themselves in the scenes to follow.

These mirrored panels turn and twist about throughout the show, changing first into the black glitter of a dance backdrop. Blak is not content to merely use the panels as typical dance decoration, however, and later turns them into doors, a wolf head, a flowered field, and even a bathroom stall.

Blak artfully uses the panels as their own set of dichotomies: they serve as both entrances and exits, both bridging and breaking points between scenes, and as a way to both demarcate and join different sections of the stage.

In addition to the dichotomy of welcome and disaffection towards the audience, Blak plays with the dichotomy of set and prop. Throughout the show, actors bring in the set pieces and props, which are pushed to the side, rather than being removed from the stage. The stage begins to feel more and more cramped as the wings fill, and it's clear that the dancers feel the same way. Dance is slowly consuming their identities, and it's up to them to decide how to deal with it.
Curran's lighting artfully weaves into Blak's set. Every recurring source of 
lighting is used at different times to conversely bring the audience in or push them away. The most striking of these recurring lights is the portrait of the moon. The moon is used to both draw the audience into the story of Irfan Shamji's Luke, but also to push them away during the \say{girls�} rebellion dance.

As a final note about the set, all along the walls of the stage are trophies from dancers past. Brendan Cowell's Dance Teacher Pat chooses to use these trophies to symbolize the show's meaning of success. Not only does Cowell point out the one year with no trophy by saying no one remembers those dancers, he then mentions the next year, where a dancer was recruited to become a professional. One of the panels turns and shows us a shrine dedicated to the dancer, another of Blak's masterful uses of the panels. All of the dancers begin paying homage to the dancer. This is the first of many times throughout the show that the cast speak about how winning (here becoming noticed and recruited) is the only important aspect of dance.

To quote a common sports saying, \say{winning isn't everything, it's the only thing.} What the show quietly but insistently points out, however, is that there is a constant culling in this mindset. Despite being a championship team, no one remembers the rest of the star's team. Slowly but surely, in memory and in action, the weak are culled. Only the strong remain.

In the first scene, Miranda Foster's Vanessa gets injured and is never heard from again.
Near the end of the show, another dancer makes a mistake and quits dance forever as a result.
These actions, along with monologues, showcase a dangerous aspect of not only children's athletics, but also of society. Weakness and failure are not used as opportunities for growth, but rather as signs of inherent deficiency. And, since the weak get culled, any mistake means you no longer belong. But of course, weakness is not only physical. When Shamji's character tells us about his nostalgia for car rides home with his mother, we never again are given the chance to view him as anything but a surface for the other actors to push ideas off of.counterpoint for other characters.

In addition to the theme of growth, both physical and emotional, Barron adds a message about a woman's role. As the show culminates, Karla Crome's Amina is placed in the unenviable position of choosing between the sport she loves and the people she loves. She is vilified for making the choice that lauded and celebrated men make in other theatrical shows. Here, Barron finally outright states a woman's role: women are expected to exist solely for the benefit of those around them. The show had alluded to this before, with mentions of the actors bonding over shared trauma, where their growth comes from helping each other.

However, beautiful lights and sights, touching messages, and heartfelt acting isn't the measure of a successful show. There were countless times during the show where I felt myself forgetting that I was in a crowd, watching a group of grown adults acting as prepubescent children, and instead saw myself as a spectator at these competitions, or a person watching some children as they grow and learn. The separation from life, if only for a few moments, is truly what makes a show worth seeing. The Almeida's staging of \textit{Dance Nation} certainly suspended me in time.

\section{Draft 5}
Clare Barron's \textit{Dance Nation} at the Almeida Theatre follows a dance squad of ten to twelve year olds as they prepare for national tournaments.
The show places a large emphasis on the sexualization and sexuality of young women, as well as their adolescence.
Recommended for ages 16 and over, the show artfully weaves the lives of adults into their childhood recollections, creating a layered effect that both alienates and invites the audience.

Clearly, this dichotomy of alienation and invitation was a goal of Director Bijan Sheibani, as every aspect of the show pushes the audience into this.
From the cast's age, to Samal Blak's set design, to Lee Curran's lighting, the show leaves the audience in a constant state of confusion of time, space, and location.

The show begins with a blank set, and so the review follows.
Before the show, the crowd sees a curtain lit to the nebulous phase between translucence and transparency.
A mirrored wall behind can barely be made out, with lights running up the seams.
As the show opens, the curtain drops, never to be seen again.
Mirrors make up the entirety of the back wall, with lights strung between each.
Throughout the show, these mirrors put the audience on stage, forcing them to see and therefore confront themselves in the scenes to follow.

These mirrored panels turn and twist about throughout the show, changing first into the black glitter of a dance backdrop.
Blak is not content to merely use the panels as typical dance decoration, however, and later turns them into doors, a wolf head, and even a bathroom stall.

Blak artfully uses the panels as their own set of dichotomies: they serve as both entrances and exits, both bridging and breaking points between scenes, and as a way to both demarcate and join different sections of the stage.

In addition to the dichotomy of welcome and disaffection towards the audience, Blak plays with the dichotomy of set and prop.
Throughout the show, actors bring in the set pieces and props, which are pushed to the side, rather than being removed from the stage.
The stage begins to feel more and more cramped as the wings fill, and it's clear that the dancers feel the same way.
Dance is slowly consuming their identities, and it's up to them to decide how to deal with it.

Curran's lighting artfully weaves into Blak's set.
Every recurring source of lighting is used at different times to conversely bring the audience in or push them away.
The most striking of these recurring lights is the portrait of the moon.
The moon is used to both draw the audience into Irfan Shamji's Luke's story, but also to push them away during the \say{girl's} rebellion dance.

As a final note about the set, all along the walls of the stage are trophies from dancers past.
Brendan Cowell's Dance Teacher Pat chooses to use these trophies to symbolize the show's meaning of success.
Not only does the Cowell point out the one year with no trophy by saying no one remembers them, he then mentions the next year, where a dancer was recruited to become a professional.
One of the panels turns and shows us a shrine dedicated to the dancer. another of Blak's masterful uses of the panels.
All of the dancers begin paying homage to the dancer.
This is the first of many times throughout the show that the cast speak about how winning (here becoming noticed and recruited) is the only important aspect of dance.
To quote a common sports saying, \say{winning isn't everything, it's the only thing.}
What the show quietly but insistently points out, however, is that there is a constant culling in this mindset.
Despite being a championship team, no one remembers the rest of the star's team.
Slowly but surely, in memory and in action, the weak are culled.
Only the strong remain.

In the first scene, Miranda Foster's Amina gets injured and is never heard from again.
Near the end of the show, another dancer makes a mistake and quits dance forever as a result.
These actions, along with monologues, showcase a dangerous aspect of not only children's athletics, but also of society.
Weakness and failure are not used as opportunities for growth, but rather as signs of inherent deficiency.
And, since the weak get culled, any mistake means you no longer belong.
But of course, weakness is not only physical.
When Shamji's character tells us about his nostalgia for car rides home with his mother, we never again are given the chance to view him as anything but a counterpoint for other characters.

In addition to the theme of growth, both physical and emotional, Barron adds a message about a woman's role.
As the show culminates, Karla Crome's Amina is placed in the unenviable position of choosing between the sport she loves and the people she loves.
She makes the choice that is lauded and celebrated when men make in theatre, and is vilified for it.
Here, Barron finally outright states a woman's role: women are expected to exist solely for the benefit of those around them.
The show had alluded to this before, with mentions of the actors bonding over shared trauma, and growth from helping each other.

However, beautiful lights and sights, touching messages, and heartfelt acting isn't the measure of a successful show.
There were countless times during the show where I felt myself forgetting that I was in a crowd, watching a group of grown adults acting as prepubescent children, and instead saw myself as a spectator at these competitions, or a person watching some children as they grow and learn.
The separation from life, if only for a few moments, is truly what makes a show worth seeing.
The Almeida's staging of \textit{Dance Nation} certainly suspended me in time.
\section{Draft 4}
Clare Barron's \textit{Dance Nation} at the Almeida Theatre follows a dance squad of ten to twelve year olds as they prepare for national tournaments.
The show places a large emphasis on the sexualization and sexuality of young women, as well as their adolescence.
Recommended for ages 16 and over, the show artfully weaves the lives of adults into their childhood recollections, creating a layered effect that both alienates and invites the audience.

Clearly, this dichotomy of alienation and invitation was a goal of Director Bijan Sheibani, as every aspect of the show pushes the audience into this.
From the cast's age, to Samal Blak's set design, to Lee Curran's lighting, the show leaves the audience in a constant state of confusion of time, space, and location.

The show begins with a blank set, and so the review follows.
Before the show, the crowd sees a curtain lit to the nebulous phase between translucence and transparency.
A mirrored wall behind can barely be made out, with lights running up the seams.
As the show opens, the curtain drops, never to be seen again.
Mirrors make up the entirety of the back wall, with lights strung between each.
Throughout the show, these mirrors put the audience on stage, forcing them to see and therefore confront themselves in the scenes to follow.

These mirrored panels turn and twist about throughout the show, changing first into the black glitter of a dance backdrop.
Blak is not content to merely use the panels as typical dance decoration, however, and later turns them into doors, a wolf head, and even a bathroom stall.

Blak artfully uses the panels as their own set of dichotomies: they serve as both entrances and exits, both bridging and breaking points between scenes, and as a way to both demarcate and join different sections of the stage.

In addition to the dichotomy of welcome and disaffection towards the audience, Blak plays with the dichotomy of set and prop.
Throughout the show, actors bring in the set pieces and props, which are pushed to the side, rather than being removed from the stage.
The stage begins to feel more and more cramped as the wings fill, and it's clear that the dancers feel the same way.
Dance is slowly consuming their identities, and it's up to them to decide how to deal with it.

Curran's lighting artfully weaves into Blak's set.
Every recurring source of lighting is used at different times to conversely bring the audience in or push them away.
The most striking of these recurring lights is the portrait of the moon.
The moon is used to both draw the audience into Irfan Shamji's Luke's story, but also to push them away during the \say{girl's} rebellion dance.

As a final note about the set, all along the walls of the stage are trophies from dancers past.
Brendan Cowell's Dance Teacher Pat chooses to use these trophies to symbolize the show's meaning of success.
Not only does the Cowell point out the one year with no trophy by saying no one remembers them, he then mentions the next year, where a dancer was recruited to become a professional.
One of the panels turns and shows us a shrine dedicated to the dancer. another of Blak's masterful uses of the panels.
All of the dancers begin paying homage to the dancer.
This is the first of many times throughout the show that the cast speak about how winning (here becoming noticed and recruited) is the only important aspect of dance.
To quote a common sports saying, \say{winning isn't everything, it's the only thing.}
What the show quietly but insistently points out, however, is that there is a constant culling in this mindset.
Despite being a championship team, no one remembers the rest of the star's team.
Slowly but surely, in memory and in action, the weak are culled.
Only the strong remain.

In the first scene, Miranda Foster's Amina gets injured and is never heard from again.
Near the end of the show, another dancer makes a mistake and quits dance forever as a result.
These actions, along with monologues, showcase a dangerous aspect of not only children's athletics, but also of society.
Weakness and failure are not used as opportunities for growth, but rather as signs of inherent deficiency.
And, since the weak get culled, any mistake means you no longer belong.
But of course, weakness is not only physical.
When Shamji's character tells us about his nostalgia for car rides home with his mother, we never again are given the chance to view him as anything but a counterpoint for other characters.

In addition to the theme of growth, both physical and emotional, Barron adds a message about a woman's role.
As the show culminates, Karla Crome's Amina is placed in the unenviable position of choosing between the sport she loves and the people she loves.
She makes the choice that is lauded and celebrated when men make in theatre, and is vilified for it.
Here, Barron finally outright states a woman's role: women are expected to exist solely for the benefit of those around them.
The show had alluded to this before, with mentions of the actors bonding over shared trauma, and growth from helping each other.

However, beautiful lights and sights, touching messages, and heartfelt acting isn't the measure of a successful show.
There were countless times during the show where I felt myself forgetting that I was in a crowd, watching a group of grown adults acting as prepubescent children, and instead saw myself as a spectator at these competitions, or a person watching some children as they grow and learn.
The separation from life, if only for a few moments, is truly what makes a show worth seeing.
The Almeida's staging of \textit{Dance Nation} certainly suspended me in time.
973.
\section{Draft 3}
Clare Barron's \textit{Dance Nation} at the Almeida Theatre follows a U13 dance squad as it prepares for nationl tournaments.
The show places a large emphasis on the sexualization and sexuality of young women, as well as their adolescence.
Recommended for ages 16 and over, the show artfully weaves the lives of adults into their childhood recollections, creating a layered effect that both alienates and invites the audience.

Clearly, this dichotomy of alienation and invitation was a goal of Director Bijan Sheibani, as every aspect of the show pushes the audience into this.
From the cast's age, to Samal Blak's set design, to Lee Curran's lighting, the show leaves the audience in a constant state of confusion of time, space, and location.

Before the show, the crowd sees a curtain in the murky realm between translucence and transparency.
A mirrored wall behind can barely be made out, with lights running up the seams.
When the show opens, the curtain drops, never to be seen again.
Mirrors make up the entirity of the back wall, with lights strung between each.
Throughout the show, these mirrors put the audience on stage, forcing them to see and therefore confront themselves in the scenes to follow.

These mirrored panels turn and twist about throughout the show, changing first into the black glitter of a dance backdrop.
Blak is not content to merely use the panels as typical dance decoration, however, and later turns them into doors, a wolf head, and even a bathroom stall.

Blak artfully uses the panels as their own set of dichotomies: they serve as both entrances and exits, both bridging and breaking points between scenes, and as a way to both demarcate and join different sections of the stage.

Blak also did a wonderful job of bridging the dichotomy that is set and prop.
Actors carried in the set pieces, or pieces that they worked with remained a part of the show.
Throughout the show, the detritus of the dance studio slowly grows and takes over the wings of the stage.
The stage begins to feel more and more cramped as the wings fill, and it's clear that the dancers feel the same way.
Dance is slowly consuming their identities, and it's up to them to decide how to deal with it.

Curran's lighting artfully weaves into Blak's set, encouraging this dichotomy.
Every recurring source of lighting is used to bring the audience in, as well as push them away.
One of these is the recurring motif of a moon.
The moon is used to both draw the audience into Luke's story, but also to push them away during the \say{girl's} rebellion dance.

A final note about the set, as means to transition into the acting, all along what is meant to be the ceiling of the room are trophies from dance competitions past.
Sheibani and Brendan Cowell's Dance Teacher Pat choose to use these trophies as a way to symbolize the show's meaning of success.
Not only does the Cowell point out the one year with no trophy by saying no one remembers them, he then mentions the next year, where a dancer was recruited to become a professional.
One of the panels turns and shows us a shrine dedicated to the dancer.
All of the dancers begin paying homage to the dancer.
This is the first of many times throughout the show that the cast speak about how winning (becoming noticed and recruited) is the only important aspect of dance.
To quote a common sports saying, \say{winning isn't everything, it's the only thing.}
What the show quietly but insistently points out, however, is that there is a constant culling in this mindset.
Despite being a championship team, no one remembers the rest of the star's team.
Slowly but surely, in memory and in action, the weak are culled.
Only the strong remain.

In the first scene, Miranda Foster's Amina gets injured and is never heard from again.
Near the end of the show, another dancer makes a mistake and quits dance forever as a result.
These actions, along with monologues, showcase a dangerous aspect of not only children's athletics, but also of society.
Weakness and failure are not used as opportunities for growth, but rather as signs of inherent deficiency.
And, since the weak get culled, any mistake means you no longer belong.

As the show culminates, we see Karla Crome's Amina placed in the unenviable position of choosing between the sport she loves and the people she loves.
She makes the choice that is lauded and celebrated when men make in theatre, and is vilified for it.
Through this, Barron adds her final message to the show: women are expected to exist solely for the benefit of those around them.

However, beautiful lights and sights, touching messages, and heartfelt acting isn't the measure of a successful show.
There were countless times during the show where I felt myself forgetting that I was in a crowd, watching a group of grown adults acting as prepubescent children, and instead saw myself as a spectator at these competitions, or a person watching some children as they grow and learn.
The separation from life, if only for a few moments, is truly what makes a show worth seeing.
The Almeida's staging of \textit{Dance Nation} certainly suspended me in time.
887.
\section{Draft 2}
Clare Barron's \textit{Dance Nation} at the Almeida Theatre follows a U13 dance squad as it prepares for nationl tournaments.
The show places a large emphasis on the sexualization and sexuality of young women, as well as their adolescence.
Recommended for ages 16 and over, the show artfully weaves the lives of adults into their childhood recollections, creating a layered effect that both alienates and invites the audience.

Clearly, this dichotomy of alienation and invitation was a goal of Director Bijan Sheibani and Set Designer Samal Blak.

When first allowed on stage, the crowd sees a curtain in the murky realm between translucence and transparency.
A mirrored wall behind can barely be made out, with lights running up the seams.
When the show opens, the curtain drops, never to be seen again.
Mirrors made up the entire back wall, with lights strung between each.
Throughout the show, but especially in the beginning, these mirrors put the audience on stage, forcing them to confront themselves in the scenes to follow.

These mirrored panels turn and twist about throughout the show, changing first into the black glitter of a dance backdrop.
Blak is not content to merely use the panels as typical dance decoration, however, and later turns them into doors, a wolf head, and even a bathroom stall.

The use of the panels as entries and exits, to bridge and break scenes, and at the end as a way to recognize the disjointed nature of both the show and life is sublimely achieved.
Each of Blak's set choices felt masterfully laid out.

Lee Curran's lighting of the show is also incredibly helpful for setting and moving the scenes along, as well as making the minimal set feel larger.
When Irfan Shamji's Luke is on the car ride home with his mother, the moon in the upper right of the stage tells us of the long day he's had, and connects us to the other dancers, who each have their own memories of the night.
As the drive continues, orange lights flow by their faces, reminding the audience of their movement.
When Ria Zmitrowicz's Zuzu has a breakdown, the lights become harsher and harsher until a realization comes through.

But, these few examples are not truly representative of the lighting for the show.
There were countless times during the show where I felt myself forgetting that I was in a crowd, watching a group of grown adults acting as prepubescent children, and instead saw myself as a spectator at these competitions.
The acting was obviously a key piece of this, but Curran's lighting was certainly an aspect of this.
The lighting during each dance montage was as if pulled directly from a dance routine.
Aline David's choreography added to this effect, as did Moritz Junge's costuming.
Marc Teitler's sounds added to the dichotomy that Blak and Sheibani used.

Blak also did a wonderful job of using set as prop, and prop as set.
Actors carried in the set pieces, or pieces that they worked with remained a part of the show.
In the wings of the stage grew the detritus of a dance studio throughout the show, reminding the audience that dance is never far from a young dancer's mind.
As the show progresses, the stage becomes more and more cluttered, just as the depth of the characters' experiences do.

A final note about the set, as means to transition into the acting, all along what is meant to be the ceiling of the room are trophies from dance competitions past.
Sheibani and Brendan Cowell's Dance Teacher Pat choose to use these trophies as a way to symbolize the show's meaning of success.
Not only does the Cowell point out the one year with no trophy by saying no one remembers them, he then mentions the next year, where a dancer was recruited to become a professional.
One of the panels turns and shows us a shrine dedicated to the dancer.
All of the dancers begin paying homage to the dancer.
This is the first of many times throughout the show that the cast speak about how winning (becoming noticed and recruited) is the only important aspect of dance.
To quote a common sports saying, \say{winning isn't everything, it's the only thing.}
What the show quietly but insistently points out, however, is that there is a constant culling in this mindset.
Despite being a championship team, no one remembers the rest of the star's team.
Slowly but surely, in memory and in action, the weak are culled.
Only the strong remain.

In the first scene, Miranda Foster's Amina gets injured and is never heard from again.
Near the end of the show, another dancer makes a mistake and quits dance forever as a result.
These actions, along with monologues, showcase a dangerous aspect of not only children's athletics, but also of society.
Weakness and failure are not used as opportunities for growth, but rather as signs of inherent deficiency.
And, since the weak get culled, any mistake means you no longer belong.

As the show culminates, we see Karla Crome's Amina placed in the unenviable position of choosing between the sport she loves and the people she loves.
She makes the choice that is lauded and celebrated when men make in theatre, and is vilified for it.
Through this, Barron adds her final message to the show: women are expected to exist solely for the benefit of those around them.
907.
\section{Draft 1}
I had the opportunity to see Clare Barron's \textit{Dance Nation} at the Almeida Theatre, a show that follows a U13 dance squad as it prepares for national qualifying tournaments.
The show places a large emphasis on the sexualization and sexuality of young women, as well as the struggles of adolescence.

The stage opens with mirrors facing the audience.
These mirrors put the audience on stage, and force them to confront themselves in the scenes to follow.

When first allowed on stage, the crowd sees a curtain in the murky realm between translucence and transparency.
A mirrored wall behind can barely be made out, with lights running up the seams.
When the show opens, the curtain drops, never to be seen again.
Mirrors made up the entire back wall, with lights strung between each.

Throughout the show, the once mirrored panels turn and twist about, becoming anything from doors, to black glitter decoration, to a wolf's head, to a bathroom stall.
The use of the panels as entries and exits, as a way to bridge and break scenes, and, at the end, as a way to recognize the disjointed nature of both the show and life is fantastic.
Each of Samal Blak's set choices felt masterfully laid out.

Lee Curran's lighting of the show is also incredibly helpful for setting and moving the scenes along, as well as making the minimal set feel larger.
When Irfan Shamji's Luke is on the car ride home with his mother, the moon in the upper right of the stage tells us of the long day he's had, and connects us to the other dancers, who each have their own memories of the night.
As the drive continues, orange lights flow by their faces, reminding the audience of their movement.
When Ria Zmitrowicz's Zuzu has a breakdown, the lights become harsher and harsher until her realization comes through.

Continuing to mess with the set, the show did a wonderful job of playing with set as prop and prop as set.
Actors tended to carry in the set pieces, or pieces that they worked with remained a part of the show.
Always in the wings of the stage were the detritus of a dance studio, reminding that it's never far away, especially in the minds of the dancers.
As the show progresses, the stage becomes more and more cluttered, just as the depth of the characters' experiences do.

A final note about the set, as means to transition into the acting, all along what is meant to be the ceiling of the room are trophies from dance competitions past.
Director Bijan Sheibani and Brendan Cowell's Dance Teacher Pat choose to use these trophies as a way to symbolize the show's meaning of success.
Not only does the Cowell point out the one year with no trophy by saying no one remembers them, he then mentions the next year, where a dancer was recruited to become a professional.
One of the panels turns and shows us a shrine dedicated to the dancer.
All of the dancers begin paying homage to the dancer.
This is the first of many times throughout the show that the cast speak about how winning (becoming noticed and recruited) is the only important aspect of dance.
To quote a common sports saying, \say{winning isn't everything, it's the only thing.}
What the show quietly but insistently points out, however, is that there is a constant culling in this mindset.
Despite being a championship team, no one remembers the rest of the star's team.
Slowly but surely, in memory and in action, the weak are culled.
Only the strong remain.

In the first scene, Miranda Foster's Amina gets injured and is never heard from again.
Near the end of the show, another dancer makes a mistake and quits dance forever as a result.
These actions, along with monologues, showcase a dangerous aspect of not only children's athletics, but also of society.
Weakness and failure are not used as opportunities for growth, but rather as signs of inherent deficiency.
And, since the weak get culled, any mistake means you no longer belong.
686 words.
\end{document}