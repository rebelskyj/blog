\documentclass[12pt]{article}[titlepage]
\newcommand{\say}[1]{``#1''}
\newcommand{\nsay}[1]{`#1'}
\usepackage{endnotes}
\newcommand{\B}{\backslash{}}
\renewcommand{\,}{\textsuperscript{,}}
\usepackage{setspace}
\usepackage{tipa}
\usepackage{hyperref}
\usepackage{nested}
\begin{document}
\doublespacing
\section{\href{album-update-24-01-03.html}{Album Update}}
First Published: 2024 January 5 (because I forgot to post, not bc forgot to write.

\section{Draft 1}
Today is the first Wednesday of the year.
As a result, it is the first in my series\endnotemark[1]\endnotetext[1]{to be\endnotemark[2]}\endnotetext[2]{hopefully, at least, but I suppose that we'll see} of reflections on the album that I am going to have written, recorded, and released by my 26th birthday.\endnotemark[3]\endnotetext[3]{which wow now that I've written it out like that, it really sounds like a lot}
I spent half an hour or so this morning unplugged and thinking about what I might want to have an album be about.

Some of my readers might know that I have really only written a few songs in recent years.\endnotemark[4]\endnotetext[4]{I keep wanting to say that I've only written a few songs, and then I remember that I've written at least a dozen or so songs for solo voice. It's just that many of them reflect the fact that I am constantly learning and growing in my ability to create. In order to be where I am now, then, I needed to have written worse music.}
The song that's received the most praise is a song that I never really gave a title to.
Its working title is\endnotemark[5]\endnotetext[5]{was? will forever be because I'm not including it on the album} \say{Starfall,} which prompted an idea for how I could arrange the album.

As my readers may also know, I have been\endnotemark[6]\endnotetext[6]{have and will likely continue, which I guess means that present perfect is the right tense? I think} giving talks on space.
And so, like every scientist interested in art, I decided to mix the two.
My album will be 11 songs based around the birth, formation, life, and death of a star.
I've planned it to be the following songs:

\begin{enumerate}
\item Emptiness I
\item Diffuse Cloud
\item Dense Cloud
\item Pressure
\item Ignition
\item Emptiness II
\item Planet Formation
\item Homeostasis
\item Pressure Redux
\item Nova
\item Emptiness III
\end{enumerate}

Of course, all of these names are significantly less than finalized.
My idea is to write it as a triptych, with the first four, second four, and final three songs as smaller acts within the piece.
I'm hoping to have the Emptiness songs be in communication with each other.

There is, obviously, more to an album than what the album specifically is about.
The way that messages are conveyed and the metameaning\endnotemark[7]\endnotetext[7]{wow I should make sure that I stop relying on the prefix meta so much} are just as important as the text and music of the songs themselves.
I haven't fully decided what I want the album to be about in that sense, though I do think that I want it to be about a journey.

I've visualized the first act as the period(s) of a life where you become aware of the influences that have brought you to where you are today.
As Pressure breaks to Ignition, we reach the second act, where a person sees their own place in the world, distinct from those influences, though obviously still aware of their impact.
Finally, as Homeostasis gives way down to Pressure again, we see the impacts that we have on the world around us.

A star is born from the ashes of a dead star.
Its own ashes eventually become the cradle for another star to be born.
Between those, it exists as a bright point of a light, plasma, and potentially life.

Of course, I haven't really figured out to what extent I want the pieces themselves to be about stars and star formation.
I think that, on some level, at least, I want to reference the stages of the star's life in each song.
However, much like Starfall uses the idea of gravity to discuss a failed relationship, I think that the astronomical concepts may end up better used as framing metaphors for the individual songs.
A part of me wants there to be a binary star, which could work for the relationship aspect.

A healthy partnership, after all, is nothing like a planet orbiting a star.
There are probably relationships that I can think of to make the idea of planets around a star a metaphor for something, but I'm not sure what just yet.

So, now that I have the framework, I need to start working on the music.
I think that I'll work on the set points of the album: Emptiness, and see if that helps me with the goals?
That could be something good.

Anyways, I realize that I have also not been doing my daily reflections for the month.
Might as well do that here.\endnotemark[8]\endnotetext[8]{I.e. I will be doing it here}
Goals for January:
\begin{itemize}
\item Keep up on BiaY and CCCiaY.
\item Keep blogging daily
\item Get ahead on Jeb
\item Write and record at least a song
\item Exercise and stretch more
\item Finish the book on craft
\item Read and/or return ideally three of the library books that I have checked out.
\item Play guitar more
\item Pray better
\item Drink more water
\end{itemize}
That means that my daily reflection should be:
\begin{itemize}
\item Blog?
\item BiaY and CCCiaY?
\item Jeb?
\item Album?
\item Exercise?
\item Stretch?
\item Book on Craft?
\item Library Books?
\item Guitar?
\item Prayed?
\item Water?
\end{itemize}

I feel like most of the goals are best worked on after I return from my familial home.
That being said, my minimal exercise has been going well, I've been blogging, and today is the first day in a while that I have not written an entire chapter of the Jeb.
As demonstrated, I have made some progress on the Album\endnotemark[9]\endnotetext[9]{henceforth written with a capital A on this blog}, and tomorrow I plan to do more of the actual research work.
Reading needs to be more of a priority for me, but I don't quite know how to make it one.

\endnotes

\end{document}