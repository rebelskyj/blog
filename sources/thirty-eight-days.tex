
\documentclass[12pt]{article}  
\newcommand{\say}[1]{``#1''}  
\newcommand{\nsay}[1]{`#1'}  
\usepackage{endnotes}  
\newcommand{\B}{\backslash{}}  
\renewcommand{\,}{\textsuperscript{,}}  
\usepackage{setspace}   
\usepackage{tipa}  
\usepackage{hyperref}  
\begin{document}  
\doublespacing  
\section{\href{thirty-eight-days.html}{On the Next Thirty Eight Days}}  
First Published: 2025 July 18

\section{Draft 1: 18 July 2025}

As far as most of my time should be concerned, I have thirty eight days.  
In thirty eight days I defend my thesis, which means that in twenty four days my committee needs to have a full and polished version of my thesis.  
In ten days my advisor needs the same.

What do I know will be taking time in the next thirty eight days?  
Or, since hours are more what I care about, it's 1600 or so now.  
There are thirteen days left in July after this, 320 hours.  
I sleep for about ten hours a night\footnote{including like bed prep and such}, so that's really like 180 hours of real time.  
In August, I have 586 hours until the defense, but again we must sleep, and so 336 hours.

In the next 516 hours what do I need to accomplish?

I need to finish writing my thesis.  
I need to make the entire thesis presentation.  
I need to finish the research for the paper I'm trying to publish.

In the next 516 hours, what other commitments do I have?

I\footnote{in retrospect, perhaps stupidly} agreed to give talks every weekend from now until the defense.  
I'm going to assume an average travel time of five hours each direction\footnote{because intermediate travel and all}, so ten hours per weekend will be gone, on the road.  
Fifty five fewer hours, I have 461 hours.

Each day in the weekend I give an outreach talk that lasts an hour, and then have observing for between 0 and 2 hours.\footnote{depending on weather and desire of the attendees}  
I also like being there\footnote{read: have been requested to be} an hour early, so let's average that to three hours per talk.  
Fourteen talks means 42 hours, 419 hours.

Just over four hundred hours of time is what I have left.  
I had been picking up hobbies again, and I think that I might need to stop that, at least for now.\footnote{time to uninstall steam}  
I do know that I cannot work nonstop forever, though.  
If I think about when things are due, what deadlines do I need for myself to be successful?

Let's once again list out what the thesis is, needs, and contains:

\begin{enumerate}

\item Minimum Viable Thesis Document\footnote{what do I absolutely have to have, as opposed to what I'd like in the thesis}

\begin{enumerate}

\item Abstract

\item Introduction

\item Acknowledgements

\item Apparatus

\item RebelFit

\item Conclusions

\end{enumerate}

\item Desired Things:\footnote{in vaguely sorted order of importance}

\begin{enumerate}

\item Chapter on Outreach

\item Publicly accessible Chapter

\item Derivation of Watson T and $\rho$ values

\item Explanation of algorithms\footnote{added after writing some more, at RebelFit specifically}

\item Derivation of Rotational Spectroscopy

\item Philosophy of Science

\end{enumerate}

\item Things that I need for the presentation:

\begin{itemize}

\item I mean honestly, I hope that what I present will come entirely from the thesis paper, so that's probably fine for now. Since I turn the final thesis in on 8/11, I have the next two weeks to make the presentation.

\end{itemize}

\end{enumerate}

How long will each thing take? Great question.

In four hundred and nineteen hours, I need to write an entire abstract, which will take under three hours unless something goes catastrophically wrong.  
In four hundred and sixteen hours, I need to write an acknowledgement of at least some of the infinitely many things and people that have brought me to this point, which will take as long as I give it, as gratitude tends to.  
Grateful as I am, though, I cannot give it more than six.

With four hundred and ten remaining hours, I need to finish the introduction.  
I wanted to rephrase and rewrite a lot of it, which would likely take under five hours.  
I need to add some missing citations and add figures where needed.  
Citations should take under thirty more minutes at minimum\footnote{because I can always add other sources, especially for claims like \say{there are many instances of}}, and the needed figures probably take under three hours.  
Four hundred and one or four hundred and six.  
I want this thesis to be something I'm proud of.

With four hundred remaining hours, I need to write effectively the entire apparatus section, cite it appropriately, and add the needed figures.  
I think that I have most of the figures I would need.  
There are fourish\footnote{I think} instruments/apparatus that I'll be writing about.  
At most it should take an hour each to make the figures.  
I do really think that I can get a minimally viable text in under two hours.  
If I wanted to write a full description of each instrument, though, that will take much longer, and I don't care that much about any of them right now.

Three hundred and ninety four hours\footnote{if you notice time dropping more it's because I round up always} are left in my life.  
I need to have a full version of the conclusions and next steps.  
I sent in a draft of it that feels mostly finished, but I'll assume another four hours for that.

Three hundred and ninety hours to finish with RebelFit.  
I need to absolutely finish the paper, and I absolutely need to finish the paper.  
The only things left are adding in the proof that it works, I think, which requires the jobs to run and me to at least minimally analyze the output.  
I'm hopeful that the reviewer is ok with tentative, rather than explicit, assignment of the new states, but we'll see what happens.

Three hundred and eighty five hours to rewrite all of the RebelFit portion of my thesis.  
I think that I have all the content I would need, and I even think that I have most of the figures.\footnote{Thanks prior me for giving a presentation}  
Let's be generous to the productivity I am capable of and assume that it takes me three hours to get the current content into something sensible, and then under thirty to finish revisions.

Three hundred and fifty hours.  
That's honestly more time than I thought.  
Still, that's not enough time for me to really feel comfortable slacking off, so time to start work again.  
I have to head to the park at about six, and I do also need to eat\footnote{probably, at least}, so I have about one hundred minutes left to work today.  
What can I get done then?

Hopefully at the very least the daily reflection and getting RebelFit into something resembling a sensible thesis chapter.

\section{Draft 0: 18 July 2025 (blog title: on (the absence of) routine}

Routines are incredibly helpful and important to and for me.\footnote{helpful to? yeah. Helpful for? yeah, important to? not emotionally I guess but intellectually. Important for? absolutely}  
Despite that, as the absence of daily posts here illustrates, I have not generally done a great job of having a routine lately.  
Some of this is understandable: I had a conference and then it was the Fourth of July.  
However, in 10 days I need to have finished my thesis, and that feels like an optimistic goal, which it really shouldn't.\footnote{thrus (both but three?) because it is far away, I need to finish it, and there's realistically not that much work that needs to be done}

Since I know that I am more productive in times of routine, it felt like a good idea to use this folly to explore routine.  
However, that's no longer feeling right, because wow two weeks is much closer than it feels.  
I kind of do really feel like I need to be delivering something at least daily, if not moreso.\footnote{delivering here can mean internal, though most of what I have does need to go to the boss, so maybe not}

\section{Draft -1: 18 July 2025}

As the sporadic and mostly absent nature of this site probably attests,\footnote{attests to?} I do not currently have a good routine going.  
There are any number of reasons for that, but most of them boil down to the fact that, unlike what makes a good routine for me, every day is not like every other.  
However, I'm now in the final weeks before my thesis defense, which means that on some level the routine maybe should just be a baseline level of panic and stress.  
Given that I'm not feeling that, I can only conclude that I need to be more aware of what time I do and do not have.

So, let's quickly get the timeline established, and then we can start thinking about routine again.

My thesis defense is on August 25 at 10am.  
I am supposed to give my committee two weeks to read the thesis, which means that they need it by August 11.  
I want two weeks between the final rounds of my boss's edits and giving it to the committee, which means she needs the full thesis by July 28.  
That is, in the next ten days I need to be completely finished with my thesis.

Oh.

That's um, a lot sooner than it emotionally feels.  
So, what could a routine look like for me?

Right now I have outreach talks every weekend.  
I do find that I can be at least somewhat productive while traveling, both because driving helps me order my thoughts and just because I can kind of write from anywhere.  
Unfortunately, driving is also a fairly large time sink.  
This weekend is one of my shorter drives and is still five hours of driving.

Ok wow I'm falling out of the planned post, let's try this again.

\section{Daily Reflection}

\begin{enumerate}

\item Did you journal a full page today?

One sentence, because I thought while driving.

\item What ink/pen did you use, and what were your thoughts on it?

Fude with Neon yellow. I think that I don't like the PR\footnote{private reserve. Why did I footnote it? that just takes more time} neons are almost highlighter, and definitely need the thicker nib. However, even so I don't like how pale/translucent they are.

\item How's prayer?

Non-existent. I stopped by a coffee shop for caffeine earlier this afternoon, and heard two women discussing Catholicism, which did remind me to pray before eating.

\item How's focus?

Oof, this post is revitalizing it really well.

\item How's sleep?

Always I feel too tired.

\item How many meals, and how balanced?

Uhhh great question.  
Today I had two western omelet sandwiches and a flavored coffee drink for breakfast, and then cold brew and a crepe with strawberries and chocolate for late lunch.

\item How's the posture?

Generally ok!

\item How's the breath?

I remind myself to breathe sometimes even when I'm not doing this reflection.

\item How's the movement?

Not enough.

\item How's the physical flexibility?

Nowhere near enough. Oof.

\item How's keeping up with the family obligations?

Good!

\item How's the thesis?

As you can maybe tell from the above, great question.

\item How's the poetry?

Nonexistent.

\item How're the interpersonal relationships?

Honestly kind of great.  
I am more sociable these days, and have made friends and connections with people at a bar I like in town.  
I met up with friends multiple times this week, and I'm reaching out to future coworkers.

\item How's the music?

Honestly also good, I jammed with a friend earlier this week and we'd meant to do an open mic yesterday. (It was cancelled)

\item How's the other writing?

Nonexistent, which is where it should be. I do want to tell the readers of my web novel that I'm not in a mental space to write it right now, but even that feels too hard, which is sad.

\item How's the cleaning?

I need to do better at that

\item How's ordering the life?

I mean this post is decent at least!

\item Water?

Better than it has been, worse than it could be.

\end{enumerate}

Current Pen List\footnote{for my own posterity, mostly}

\begin{itemize}

\item Hongdian Black with Blade Nib: Missing  
\item Hongdian Black with Fude Nib: Private Reserve Neon Yellow (since pre-June I think)  
\item Shark: Purple of some sort, unsure which right now  
\item Kaweko: Chocolate Brown?  
\item Pilot Preppy: Private Reserve Electric DC Blue I think  
\item Conklin: Montverde Fire Opal  
\item Sheaffer: Private Reserve Spearmint

\end{itemize}

\end{document}