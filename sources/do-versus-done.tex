\documentclass[12pt]{article}[titlepage]
\newcommand{\say}[1]{``#1''}
\newcommand{\nsay}[1]{`#1'}
\usepackage{endnotes}
\newcommand{\1}{\={a}}
\newcommand{\2}{\={e}}
\newcommand{\3}{\={\i}}
\newcommand{\4}{\=o}
\newcommand{\5}{\=u}
\newcommand{\6}{\={A}}
\newcommand{\B}{\backslash{}}
\renewcommand{\,}{\textsuperscript{,}}
\usepackage{setspace}
\usepackage{tipa}
\usepackage{hyperref}
\begin{document}
\doublespacing
\section{\href{do-versus-done.html}{On Doing Versus Having Done}}
First Published: 2023 June 3

Prereading note: sorry, this one got really rambly. I should probably fix it up at some point
\section{Draft 1}
Like many people, I wish that I did more.\footnote{wow that's such a bold start to a blog post.}
At the very least, I wish I had better knowledge of where my hours went.

As a result, I have tried a number of things to get my life in order.\footnote{non-exhaustive list includes: scheduling events by the minute, saying what activities I want to do before and after certain milestones in the day (like going to work), making a list of activities I want to do, to do lists (which are different somehow), ranked priority list of all the things I need to do}
None of them have worked as well as I would like, but the struggle a few weeks ago did lead me to think more deeply about what my priorities really are.

I was making a list of activities that I wanted to complete.
Some of them were things that I enjoy doing in and of themselves\footnote{e.g. playing music}, and some are things that I enjoy the knowledge that I have done them.\footnote{running is really my best example for this. I hate running but love when I have run}
It made me think of a nice two dimensional way to rank different activities.

The two axis\footnote{axes?} titles that I settled on were do and done.
What does that mean?
Great question.

Do is tasks where doing them is itself the goal.
For instance, I like reading because it's a fun task and I enjoy doing it.
It may benefit me, but even without the growth it causes, it's something I'm glad to do.
As I write, I realize I should have started with done, so onward.

Done is what I described as tasks that I do not necessarily enjoy the process of completing them, but I enjoy the knowledge that I have done them.
As I mentioned in a footnote, running is really my best example of this.
Every time that I run, I find it an unpleasant experience.
Nevertheless, every time that I look back on a time that I have run, I am grateful to myself for having run.

Another way to think of it is future versus present enjoyment.
Things that I enjoy doing bring me enjoyment in the present, as I do them.
Things that I enjoy having done, by contrast, bring me enjoyment when I look back on the fact that I did them.

I have no clue if that made any sense, but it does mean that I can group most of the activities I do into four quadrants.
Ideally, I would like to spend all of my time doing things that I enjoy doing and having done.
Equally ideally, I would spend none of my time doing things that I neither enjoy doing nor having done.

As I made my list, though, it occurred to me that ranking the other two categories is an interesting values judgement.
If I prioritize things that I enjoy having done, then I will likely enjoy each day less.\footnote{at least according to first-level effects.}
On the other hand, if I prioritize things that I enjoy doing, I am likely to stagnate in most of my endeavors.

Returning once more to my realm of ideals, it would be fantastic if I was able to learn to enjoy in the moment all of the activities that I enjoy having done.
If running became fun, for instance, I am sure that I would run more.

As I think about my day, though, I realize that I spend a lot of time doing things that I neither enjoy doing nor having done.
That's probably not great for me.
Partially, it could be that I have the wrong framing for some tasks I accomplish.
Failing to do something, for instance, could be reframed as setting me up for a future success and therefore a thing that I enjoyed having done.

That is not most of what I find lives in that quadrant, though.
I am very easily sucked into time sinks.\footnote{specifically, useless time sinks}
I know while I am in them that I will not be glad for having spent my time on them in the future.
Even while I do them, I am not so much enjoying what I am doing as killing time.

I only have so much time, and it would be great if I stopped trying to kill it.

643/95
\end{document}