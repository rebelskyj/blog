\documentclass[12pt]{article}[titlepage]
\newcommand{\say}[1]{``#1''}
\newcommand{\nsay}[1]{`#1'}
\usepackage{endnotes}
\newcommand{\1}{\={a}}
\newcommand{\2}{\={e}}
\newcommand{\3}{\={\i}}
\newcommand{\4}{\=o}
\newcommand{\5}{\=u}
\newcommand{\6}{\={A}}
\newcommand{\B}{\backslash{}}
\renewcommand{\,}{\textsuperscript{,}}
\usepackage{setspace}
\usepackage{tipa}
\usepackage{hyperref}
\begin{document}
\doublespacing
\section{\href{toffee-recipe.html}{Toffee Recipe}}
\section{Draft 3}
There are some recipes I make mostly as an excuse to make others.
Among those is toffee.
Nowadays, I mostly\footnote{read: exclusively} make toffee when I'm making chocolate toffee cookies.\footnote{because sometimes I'll start making toffee and realize what I've done. The only solution is cookies.}
This is not a recipe for the cookies,\footnote{but one is probably forthcoming} but only for the toffee.

Now, you may notice the preface of \say{nowadays} for how I make toffee lately.
At first, I made toffee for its own sake.
The first time I made it, I made it for one of my older brother's friends.
What follows is the fruit of my painstaking\footnote{read: really delicious} trials into making toffee.

Ingredients:\\
1/2 cup salted butter\\
1/2 white granulated sugar\\
1 dash vanilla

Directions:\\
Put butter in large sauce pan (at least 6x the volume of the butter) over low heat.\\
When butter is mostly melted, add sugar and vanilla, and mix until well incorporated.\\
While cooking, it should expand to a much larger size.
When it does so, continue stirring until it reaches hard crack,\footnote{\href{https://en.wikipedia.org/wiki/Candy_making}{Wikipedia} suggests that means 295-309 F, or 146-154 C. If you drop it into a glass of water, it should form hard balls or threads that snap, rather than bending.} or, once you've made it a few times, until the aroma and color is correct.\\
Pour into a well greased sheet pan\\
Let cool until solid.\\
Break into pieces.\\
Serve however you intend to.

\section{Draft 2}
When I was younger, one of the first recipes I taught myself\footnote{as opposed to being shown} was toffee.
Now, this was not an unprompted decision.
A friend of my brother remarked on the expense of it, and after looking at the price, I thought it unreasonable, given how simple the recipe is.
And, after much experimentation with other recipes, I've found one that works for me.
Here it is:

1/2 cup salted butter\\
1/2 white granulated sugar\\
1 dash vanilla

Put butter in pan over low heat.\\
When butter is mostly melted, add sugar and vanilla.\\
Cook, stirring occasionally until it begins to grow in size.\\
Continue cooking until when you drop a little into a glass of room temperature water, it cools to a hard, crunchy ball,\footnote{hard crack} then pour onto a greased sheet pan.\\
Let cool until solid.\\
Break into pieces.

\section{Draft 1}
When I was a young lad,\footnote{my father took me into the city} I learned how to make toffee.
Now, this was not an unprompted decision.
One of my brother's friends really liked toffee, and I learned this when I received an email from a candy company.\footnote{I never know how much background information to give, but this feels like the wrong amount}
He asked me how much it would cost to purchase some, and I thought the price looked ridiculous.
When I looked up the recipe, it seemed simple enough to make.
And, it is!\footnote{Yay!}
So, a recipe for Toffee:

1/2 cup salted butter\\
1/2 white granulated sugar\\
1 dash vanilla

Put butter in pan over low heat.\\
When butter is mostly melted, add sugar and vanilla.\\
Cook, stirring occasionally until it begins to grow in size.\\
Continue cooking until the correct color, then pour onto a greased sheet pan.\\
Let cool until solid.\\
Break into pieces.
\end{document}