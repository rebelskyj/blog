\documentclass[12pt]{article}  
\newcommand{\say}[1]{``#1''}  
\newcommand{\nsay}[1]{`#1'}  
\usepackage{endnotes}  
\newcommand{\B}{\backslash{}}  
\renewcommand{\,}{\textsuperscript{,}}  
\usepackage{setspace}   
\usepackage{tipa}  
\usepackage{hyperref}  
\begin{document}  
\doublespacing  
\section{\href{self-help.html}{On Self Help}}  
First Published: 2025 May 28

\section{Draft 6: 28 May 2025}

Something I've noticed in my writing is that early drafts hardly matter at all.  
I don't know if those I interact with are the same way, but I need to write at least a full draft before I even know what I want to say in a given piece.  
This is true especially in this folly; I still don't entirely know what I want it to say, but I am going to declare here that this draft will entirely just be my relationship to the self-help genre.

Self improvement is a laudable ideal.  
No one is perfect, and we can always strive towards betterment.  
In a literate society, then, it is perhaps unsurprising that self-help is such a large and profitable area to write and publish.  
I have read more than my fair share of self help literature, even if many of the authors I read would actively object to the label.

This past year, and these months in particular, has really changed the way I interact with the world.  
Part of it is obvious; I lost the fundamental touchstone of my mother.  
I also think that age is part of it; I am older now, and so my thoughts are better.  
And, I'm coming to the end of my plans; I had never really thought about what I wanted to do with my life post-being a doctor.

Despite feeling like a swimmer adrift, I am not reaching out for the lifelines of self help literature.  
At this point, I do not know if any self-help book can address the problems I'm facing, and that really comes down to the fact that the issues I face right now are so intrinsically personal.  
I cannot imagine that there are enough twenty-six year old graduate students about to finish a degree who have just lost their mother, a first-generation student who encouraged them to study for passion.  
Among those, I don't know how many love to write so much, and love to love and interact so much as well.

Self-help as a concept is laudable, even though it often obscures the structural issues at play.  
Of course, very few people can directly change the structure of society, and so learning to work within the given system is understandable.  
Self-help books seem to me to struggle from the same issue as every form of education that does not take place in The Academy: impersonalization.  
We have known for almost 50 years now that a median student with a tutor will start to outperform ninety seven percent of their peers.  
I'm sure there's literature suggesting why, but fundamentally I think it is because a one-on-one interaction with a tutor makes it far harder to ignore what you don't know.

In short, I am grateful for the self-help books I have read, because many come with advice that, even if not helpful to me, is life-changing to someone I love.  
At a certain point, however, they stop being useful for self-improvement, because they cannot define the goals of my life.  
General career advice is great, but I would hope that me entering the job market with a Ph.D. is a very different process than someone entering the job market as a fresh college graduate.  
At a certain point in reading, the most helpful books shift from practical to philosophical.  
I've passed that point, and that's ok.

Nice! Good draft me.

\section{Draft 5: 28 May 2025}

Self-help and adjacent genres all suffer from the same fundamental issue as society at large; they treat each person as interchangeable pieces in a great machine.  
Books I've read about productivity always assume that certain events can be automated away: rather than cook a meal each day, meal prep for a week in advance.  
No attention is paid to the more structural questions: why are you living alone, is there no one else who you can divide the labor with and why not, does cooking not have intrinsic beauty to it, why do you feel as though you need to do more?  
Perhaps it is unsurprising, then, that a number of authors in the mindfulness and wellness spaces do not like being lumped in with self-help.  
After all, most of them focus on feeling enough.

And yet, even these books which can discuss social structures that limit the effectiveness of any given advice, like \say{you can't self-care yourself out of oppression}, do not speak to everyone's lived and living experience.  
No one is totally normative, and everyone interacts with the world differently.  
This is beautiful and true and good, but means that the more one sees the world differently, the more important it is that one ignore much self-help advice.

Awareness of the passage of time terrifies me at a fundamental level, and I do not hit my productive strides until I am able to fully dive into the water of a given problem or idea.  
The pomodoro method, which I have to assume can only work for those whose labor does not require holding large ideas while working, is therefore counter to what I do.

Obsession is my tendency: if I measure it, I will smother it in affection and attention.  
When coupled with the societal tendencies I've inherited towards certain forms of addiction and disorder, I should not track every calorie that I eat.

As someone who has read a lot of self help, I absolutely agree that it does a lot of good for a lot of people.  
As someone who is so different from so many others\footnote{if the number of hobbies I have, the Ph.D. I'm about to complete, and the fact that I've written 7000 words in this document alone today weren't enough of a sign}, I absolutely need to stop trusting the advice of strangers over my own lived experience.  
When authors cite studies, and do so correctly, their claims tend to be much more logic based: \say{assuming X, literature suggests Y.}  
If the assumption is wrong, it's easy to toss it aside.

I'm not saying throw the baby out with the bathwater.  
I think that even the cringiest and objectively awful self-help books I've read have something to teach me.  
I just also think that at a certain point, we need to acknowledge that the field is only helpful in so much as it helps.

Was this just another ramble? Let's read it and find out. If not, calling it here.  
If so, only allowed to revise.

\section{Draft 4: 28 May 2025}

Metaphor is powerful, and the superstructures of society impose metaphors on us.  
In today's day and age, almost everyone I know can tell you the time to the minute with almost no effort, and has the capability to show the time to any desired degree of precision.  
Some people function well as cogs: the rotation of the earth for a 24 hour day, regardless of season, can be broken into externalized blocks of time.  
This is not meant to denigrate the cog-people\footnote{even if the name feels intrinsically negative}.  
In the society we inhabit, being able to function according to an external clock makes you happier in a very real sense: life is happier when less frictive, and not rubbing against the constraints of our societal cages keeps the sores from forming.

If I could choose to be a coglet, I would.

Unfortunately, my relationship to time is not so mechanized.  
Knowing I have an appointment at 11, my day is blocked for at least 40 minutes leading up to it.  
Because I know that I do not relate well to time, I budget an extra half hour for any drive over fifteen minutes.  
Look at that, even when writing, I cannot escape the capitalist metaphor for time: budgeting implies spending implies that saving can happen.  
I arrive places early, because I am terrified of being late.

But, the cogless\footnote{I like coglet and cogless as descriptors} suffer in other ways as well.  
Nearly every self-betterment routine assumes that blocks of time exist as interchangeable units.  
If I have dance for an hour before lunch, I can move that to the hour just after lunch so that there's the extra morning hour of work, or so the claim goes.  
And, when dealing with external factors, this may well be true: the class lasts as long as the class-runner makes it.

When working alone, though, this is far from true.  
More than just my productivity at different hours of the day or days of the year, my need to use time changes as well.  
If I hold each stretch for thirty seconds, I don't stretch some muscles enough, but I stretch others more than enough; it all depends on what is tight on a given day at a given time.

If I construct a schedule for myself that implies tasks take fixed times, I am attempting to slot myself into being a cog.  
Advice for those who struggle with time blindness is often tasked around finding the level of scheduling where every minute that goes by was spent well.\footnote{capitalism intentional there}  
As someone who knows that some of my best memories come from the untracked hours I spent talking with friends, any linear awareness of time passing prevents me from fully experiencing the moment.

I don't quite know how to make this work for me, though.  
If I don't do things hourly, I will often not do them at all; I have been known to sit at a computer unmoving for hours on end.  
If I stop each hour to stretch, though, a part of me is looking at the clock, counting down how much more time I have until my next motion.  
The world around me conforms to linear time, and so if I want to interact with the world, I have to meet it on society's terms.  
When scheduling for myself, though, I think that what is most important is that I set a minimum bar of effort: if after X words or pages or stretches or scales or etc. I still don't feel like doing a task, then I can put it off for my next set.

It was suggested to me that I try a block scheduling for my life, affirming three periods of work with meals and movement after each.  
I think that this might work for me, especially if I am able to spend a few minutes each day determining what my priority list is.  
If I spend an entire day on one task, for instance, then I spent the entire day on it.  
If I cycle through every task and none feel good, then I know that I absolutely need to stop and interact with the world around me somehow, especially through movement.  
Will this new scheduling method work for me?  
Maybe.

Did this get completely off topic?  
Yeah...  
Final attempt?

\section{Draft 3: 28 May 2025}

Metaphor is powerful.

The 2008 financial crisis came because mortgages were compared to human lifetimes.  
When I say that there is a well I pull from to write, I tell myself that waiting to recover for days is the appropriate method to stall burnout.

Self-knowledge does not mean self-mastery.

I know that I struggle to start tasks; each new occasion marks another leap into an unknown and unlit chasm.  
I know that I struggle to finish tasks; a phantasm of death comes to collect each time I end something.  
Neither of these is a moral failing, or even really a failing at all.  
However, I need to schedule my life with the full knowledge that I will always have to force the first drops out.

If I had to distill the entirety of useful self-help literature I've read into just a few pieces, I think that these are the most resonant right now.  
However, in most of the other drafts, I've ignored another key question: what is the point of self-help literature?  
In short, I'd argue that self-help as a genre is about noticing areas of life dissatisfaction and removing that feeling.  
Depending on the book, that might present as learning to accept life as it is, or it might mean taking on new hobbies and routines.

Regardless of what advice one follows, though, there's a point that I know to be important and true even though nothing inside of me resonates with the fact: there are twenty four hours per calendar day.  
We cannot save the time, and it will pass regardless of how it is used.  
It is not spending in any real sense, as spending implies the option to save.

Every last moment of every last day is used.  
If I am trying to add anything to the day, that means by definition something else needs to be removed.  
Sometimes that's fairly easy: I don't like eating lunch as a break meal, and so can afford to spend an extra few minutes reading during the night.  
Other times it's incredibly difficult: I want to exercise more, but going to the gym alone is a ten to twenty minute round trip.  
Changing into clothes is another five or so\footnote{this includes the whole \say{I have to move from entrance to locker room}} in each direction, and showering is another impulse of time.  
At the very least then, if I want to work out in the gym, I need to budget, at the absolute minimum, thirty minutes more than the amount of time I want to work out.

So, how does this help me with self-help?

Realistically, it means that I more and more am in the camp of those who believe that the goal should be self-acceptance over self-improvement.  
I could log my time, it is true.  
However, logging my time is then a task of itself, and also like I refuse to see myself as solely a cog in a machine.  
Scheduling, even \say{practice for half an hour}, implicitly says that the linear relationship with time is the correct one and that there is some fundamental ordering to reality.

Oof this got away from me a little bit.  
Still, I'm curious where we're going!

I've written before about scheduling my life.  
Try as I might, I have yet to find a method that works for more than a few days.  
There are, however, a number of things that I know absolutely do not work for me.

I should not go days on end without just sitting down and hand-writing whatever is on my mind.  
In no way must what I write be even slightly coherent, the importance is solely in physicalizing the thought.  
I cannot ever rely on motivation: even going to a friend's party feels hard when in the midst of any other task.  
I should not have any games downloaded on my computer or phone: even when I promise myself it will just be a single game, ten minutes stretches into four hours far too easily.  
If there's a barrier to a task, it will go undone: if my guitar is not reachable without moving from my bed, I play it far less than if I can literally pluck it without needing to sit up.\footnote{do I need to sit up to play? yes, absolutely}

What else what else?  
I think that I might just try this again but in the form of rejecting time as authority.

\section{Draft 2: 28 May 2025}

Metaphors are powerful.

I often forget just how true this is for myself.  
I have spoken many times before about how my ideas and words come from a well of writing.  
Wells run dry, and the correct response to a well running dry is to let it refill.  
Therefore, when I feel as though writing is hard and the words are gone, I should stop and wait for new words to appear inside me.

This is perhaps true in some very very local sense for me, and only then.  
When I write more, I am more able to write even more.  
I refill myself by doing other forms of writing, not by avoiding writing altogether.

Metaphors are powerful, and the stories we tell are as well.

I believe that in general I am the biggest limiting factor to my success.  
In part, this is because I have managed to structure my life in such a way that it can tend to be true.  
I no longer do experimental work, and so if my calculations do not run or a paper is poorly written, it is on me.  
Buddhists talk about how the idea of self is an image, not reality.  
What does it mean for me to be my limiting factor?  
Looking non-religiously, what about the greater structures I find myself in?  
The world is more fragmented than it has ever been; labor has never been so divorced from results and compensation; workers are losing rights that were painfully clawed after for generations.

What stories do I tell that limit myself?  
One is always that I will ever want to do something.  
Even going to a friend's home for Memorial Day, something that they actively expressed interest in me attending, is difficult.

If I want to accomplish something, I must either find a way to claim that it's actually a continuation of some other task, or else grit my teeth, push through the thorns of complacency and fear, and find myself in a new grove of new tasks.  
If I want to finish something, I need to actively confront the spectre of death that lies in front of any accomplishment.  
Perhaps more importantly, though, I need to accept that neither of these are moral failings.  
Wanting to be better at starting and finishing tasks will not make it so, and even if there was some way for me to fix them, I do not know what it is.  
There's something to be said for wishing the world was a better place, doing what I can to effect that change.  
However, there is also something fundamentally important about working in the current world.

Do I wish that the electoral system would represent the people and that I could vote for a candidate I passionately support?  
Yes.  
Do I know that I must vote for the candidate which I find least objectionable in almost every circumstance?  
Yes.

If I am comfortable with accepting that I don't have a great political representative, why am I so opposed to accepting the parts of me that are not idealized?

At this point I must go to my meeting, but I do really feel like I've got a good post going, and I'm excited to finish it.\footnote{and also like I'm ok with the fact that this is not actively bringing me to my dissertation being completed. I am more than a worker and need to start reclaiming the activities which bring me life. Since I have been able to write 5000 words here without a throughline, I should accept that I have not been getting my words out}

\section{Draft 1: 28 May 2025}\footnote{this now makes the third draft I've started by calling Draft 1.}

What does it mean to be better?

This is a question I ask myself often.  
For many, it involves worldly success, which is often measured by money or influence.  
For others, it means that the boundaries they see in themselves are gradually eroded.  
For still others, it means learning to accept the boundaries which they have.

As someone who has consumed a large quantity of literature which could be described as self-help\footnote{many authors in related spaces do not like being grouped into that genre. However, they do not write my follies. I do}, the more that a book aims to effect specific changes in behaviors or patterns, the more that it tends to assume that the reader is \say{normal.}  
What any given author means by normal is very rarely explicitly stated, and it does differ slightly between different books, even those who share an author.  
In general, though, they assume someone who does not have a disability, has sufficient autonomy to markedly change their life if they so chose, and is dissatisfied with the state of their experience.  
Most tend towards a neo-liberal idea of self and betterment,\footnote{thanks to the seminar I went to on writing health which talked about how society does a lot to us, not least of which is saying that we are measured by individual productivity} which assumes that we are cogs that should be constantly producing and that we can improve our lot in life solely by our own efforts.

It isn't that I think they are intrinsically wrong.  
In fact, I do think that most people would benefit from knowing at least a little of the literature on how to live a life that is closer to what they want.  
At the very least, most of the books ask the reader to think about what they want, which I don't think many of us do anywhere as often as we maybe should.  
It's one thing to excel at the consequences of the choices we made, and it's something far different to make sure that we're choosing correctly.\footnote{can you tell that I'm born for academia?}  
However, the advice is, by nature of coming from a static book, one size for all.  
Every person is fundamentally and wholly unique\footnote{except maybe identical twins. The jury is out on that (joke, I know that they too have individual immortal souls, and so are their own being)}, and no advice will work for all people in all places.

Perhaps the most impactful piece of self-betterment advice I have encountered recently came during the dissertation writing camp I attended last week.  
During each lunch break, the organizers brought in a speaker to talk about some aspect of the writing experience.  
As someone who's consumed perhaps too much literature on producing writing, most of the advice was completely old to me: write daily, accept that early drafts are bad, make sure that things are appropriately formatted when submitting, etc.  
On Wednesday, however, they brought in a speaker to talk about mental health in the dissertation writing process.

I do not have a good grasp on linear time, and I have significant mental inertia: once I start on something, even and especially a break, I find it hard to stop.  
Standard productivity advice, like the pomodoro method, is actively harmful not just to my productivity, but to my overall sense of well-being.  
Despite knowing this at intuitive and intellectual levels, I still generally feel as though the fact that I cannot get through the many tasks I wish I could is a personal and therefore moral failing.

At this seminar, there were three key things that felt as though they were shining light onto a part of me that I didn't know existed, let alone was hidden in darkness.  
First, there is nothing wrong with not being able to follow any given writing advice; the speaker gave the example of someone with caretaking responsibilities being unable to consistently write at the same time every day, but quickly extended it.  
Second, the society we live in assumes that we can do literally everything on our own and that we should be able to do so.  
Even more than that, though, it tells us that we must judge our every action against not just our own goals, but also the accomplishments of those around us and those who could conceivably be called peers; once I finish my Ph.D., I am nominally on an even playing field with my advisor, so I should be able to output as much work in as short of a time frame.  
Of course, this idea is ridiculous; I hope the explanation alone conveys that.  
Finally, failing at something is not a moral failing.

I don't know why that final statement struck me so hard.  
It's not as though that's something that I've ever been explicitly taught, and in fact my family raised me with the opposite belief.  
And yet, the effects of society worm their way deep into our psyches.

Each day of camp, we were asked to make three goals: a dream goal, a reasonable but optimistic goal, and a minimum goal.  
In doing these, I realized that I have a fundamentally different understanding of my capabilities as the rest of the camp, and likely the normative person too.  
The minimum goal for me is what I know that I can accomplish given the lowest productivity that I have had over the past three weeks.  
The reasonable but optimistic goal assumes that I do not spiral out during the process, and the dream goal assumes that I can work without rest and at a pace where my hands are the sole limiting factor in production.

Perhaps because of this, the first two days I was barely able to meet the minimum goal.  
There's something to be said for setting lower goals, because that means that it's easier to exceed expectations.  
At a deep and primal level, though, I hate that idea.  
Lowering standards never feels like a good thing to me.

How does this relate to the self-help literature?

In general, books aimed towards improving your life assume that you do not know what your capabilities are.  
People tend to overestimate some aspects of themselves and underestimate their skill in other domains.  
I do not claim to be any different in general, but I do think that I know what I can do and what I want to do.

The issue for me is always starting and finishing tasks.  
Something deep inside of me sees every new action as a cliff that will lead to sharp rocks.  
Finishing any project is ending, which is a form of death, and I have a reasonable fear of causing death in others.\footnote{I hate that I have to write it like that. I would love if I felt as though I had a reasonable fear of death full stop}  
Once in a space where I am working, I perform best when I am able to remove every obstacle: water should be close at hand or preferably even just lean-overable, because the way time does not pass for me means that I will otherwise forget it.  
Hunger, which often gnaws at me, silences itself when I find myself \href{fugues-and-flow}{working deeply}.  
Rewards do not work for me because I understand that I am not really reward motivated, I am external praise motivated.

Ok this is good, I think that the focus is really better expressed as what normative advice is and what I would advise me about.  
Mostly the latter, in fact.

\section{Draft 0.5: 28 May 2025}

As a musician, I was well-schooled in the idea that practicing scales and chords\footnote{for polyphonic or homophonic (ooh a post about homophony in single instrumentation could be fun)} actively improves my ability to write, play, perform, and generally experience music.  
As a writer, then, it feels like the same should be true for typing practice; the goal of scales is to allow the instrument to become part of you\footnote{there's a psychology term for this and I remember reading the papers that showed that top musicians' minds literally treat the instrument as an extension of their body} by making the simple motions completely mindless.  
Typing practice lowers the barrier between thoughts existing in my mind and being put on the page.  
And yet, there's nothing I've seen really anywhere to suggest that aspiring authors should do typing courses.

Part of it is obviously a historical precedence.\footnote{I had a conversation with a friend today about whether that phrase is redundant. I settled on it meaning that the precedence is just age} 
Typewriters and computer keyboards are far, far younger than notation modern enough to make practicing scales a concept.  
Given how many writers still do so by hand, it is perhaps unsurprising that the advice I was given is that the analog to scales for writing is free-writing, where there is no self editing.  
And yet, I find that I'm better able to write now as I have done more scales.

Hmmm this is also not getting me where i want to go.  
DO I want to write about self-help?  
I don't know actually.  
Let's try one more time, and we can see where we get it.


\section{Draft 0: 28 May 2025}

Something that I realize more and more as I grow older and\footnote{presumably and hopefully} wiser is that I cannot live my life according to the standard set of optimization routines.  
This is not even in the \say{I am human not machine, and so cannot and should not be optimizing literally everything for the sake of optimization}, but also because I am not a standard human being.  
None of us are, which is another strike against the normative self-help literature\footnote{at this point I realized that the folly I had intended to write about paper and how I hold it was really better served as a thing about self help literature. Not sure if paper is going to be covered, so it's going in the unborn folly page}

Despite, or perhaps because of this, I have consumed a large number of books whose nominal goal is life and self improvement.  
Some of these have been explicitly self-help, and others are more theoretical or philosophical.  
The fact that my family reading group\footnote{which is currently on an indefinite hiatus/ is likely never coming back} picked this genre often probably also says something about the environment I occupy.  
We have a family reading group, and of every book which has ever been, we tended to pick those which claimed that they could fix some broken part of us.

This could very easily spiral into a theological folly about how we cannot fix what is broken, but that being broken is fundamental to human nature.  
Or, I could reflect on the Buddhist books I've been reading lately\footnote{does it say something that almost every author I can think of in the emotional intelligence/mindfulness space is Buddhist? Probably. I think I saw somewhere that there was a Catholic Buddhist monk, but I don't know if that's true, and that's too off topic for now}, and how viewing ourselves as broken is a bad story.  
For whatever reason, I'm instead reflecting on a book on rhetoric I read recently: \say{The Evolution of Mathematics}, along with the writing camp I attended last week.\footnote{which also absolutely needs to be a folly. I should do that tomorrow}  
The book argues that mathematics is best thought of as a type of rhetoric, and explores how Calculus required fundamentally shifting the rhetorical framework that people used when discussing numbers.  
It ended with a chapter discussing the paper which caused the 2008 global financial crisis through a rhetorical lens.  
Perhaps unsurprisingly, the derivations in the paper were mathematically valid, but made incorrect assumptions.

For some reason I'm thinking a lot about metaphor right now.  
There's a metaphor I use often about how writing is a well.  
Sometimes the well runs dry, and that means that I need to stop writing for a time.  
As I experience right now, though, I think that's fundamentally untrue.

So far as I can tell, the normal advice for self-care and self-growth involves trusting that you understand your bodies cues and listening to those cues.  
Recently, though, I gave someone advice which boiled down to \say{the only way to get through this is to just ignore the part of you that says you can't do it.}  
That's true for much, and I learned last week at writing camp that self-efficacy is better correlated with success than actual competency.

Ok wow this is spiraling fast.  
I don't entirely know what I'm trying to say today, and that's part of the issue.  
I haven't been journaling by hand in a while, which is almost certainly part of it.  
Take two.
\section{Daily Reflection 28 May 2025}

\begin{enumerate}

\item Top Priorities:

\begin{itemize}

\item Sleep:

\begin{itemize}

\item Keeping sleep time sacred?

Yeah!

\item Good sleep hygiene?

Eh! Last night I took a nap from 7-9, was then unable to sleep for a hot second, but did some stretching which was probably the best thing for me at the situation.

\item Sleeping enough?

Probably! I woke up this morning a few minutes before my alarm, changed it, and then basked in the joy of warm blanket for a while

\item How well rested do I feel?

Generally fairly! I think that I might now be caught up on sleep, which is good!

\end{itemize}

\item Feed myself:

\begin{itemize}

\item Did I eat breakfast?

Today? I had a pack of gushers

\item Did I eat a second meal?

Am planning to get a bagel sandwich for lunch. Yesterday, as discussed below, I went through a large bowl of oats

\item Did I eat dinner?

I ate jerky last night, and tonight is Wednesday, so it is burger night.

\item Water?

Not as much as I would like, but we're slowly edging back towards cells with water

\end{itemize}

\item Family:

\begin{itemize}

\item Am I neglecting any familial obligations?

Nope! Called the brothers and everything. We all have our tasks for the week, and mine is listening to the Red Album\footnote{which is so crunchy, wow. I love that deep fried sounds are either an intentional choice or evidence that the world has in fact made progress on recording technologies in the past few generations.}

\end{itemize}

\item Movement:

\begin{itemize}

\item Am I stretching at least 5 minutes per hour of computer time?

I am a monster and ignored all the alarms last night.

\item Am I generally making efforts to be limber?

I stretched last night and then again this morning.  
I still feel tight, so I don't know if it's great.

\end{itemize}

\item Spirituality:

\begin{itemize}

\item Time for prayer?

No

\item Prayer?

A morning prayer!

\item Time for sacred silence?

Bedtime last night

\item Deep breaths?

I kept telling myself to, and um it's hard to breathe deeply for some reason. Probably nothing worth thinking about

\end{itemize}

\end{itemize}

\item Secondary Priorities:

\begin{itemize}

\item Thesis/ Ph.D. work:

\begin{itemize}

\item Keeping up on the writing deadlines?

This week's whole plan is making plots and whatnot for the paper and for anything that I want for the presentations I'm going to be giving.  
I'm terrified of submitting the jobs to the cluster, even though there is no reason for me to have that fear.

\item Reading the necessary things?

I think so! I did a good lit search this past weekend.

\item Making graphs?

I'm going to re-itemize this into things that I need for RF and things that I just generally need for the thesis or whatnot.

\begin{itemize}

\item Visual depiction of Latin Hypercube

\item Visual depiction of Loomis-Wood Diagrams

\item Visual depiction of Spectral Stacking

\item Visual depiction of how the fitness of the spectral stacks is really reliant on the graphs being the right height

\item Plots from the actual results of the runs, to make sure that it worked out.\footnote{SSC, AAT, if any vib states were good, what happened to the computations, etc}.

\end{itemize}

\begin{itemize}

\item Visual depiction of Grid Search

\item Visual depiction of random search

\item I guess that the stuff for intro to quantum video counts here.

\end{itemize}

Oof that's basically all things that I need to do this week. That's fun and exciting, I guess.

\item Organizing citations?

Not so much, no. I did go through and clean some citation data yesterday, because I think that I don't trust every computational result that comes out of scientists.\footnote{See \href{what-we-dont-post}{my post about that}}

\end{itemize}

\item Love:

\begin{itemize}

\item Taking risks?

Nope, let's fix that now.

\item Making efforts?

Yeah! I reached out to a friend last night, and it was good to chat.

\item Showing affection?

I think so! It's hard to show affection to everyone at all times, especially when I'm as tired as I've been lately, but that's not really excuse.

\item Being honest?

I think so! Very few questions are being asked to me any more.

\item Being open?

Generally! I like showing the real me, and I think that I'm getting comfortable with my habit of loose papers in the backpack.\footnote{this is not a non sequitor I promise. Recently someone expressed horror at my current organization strategy (many folders with hole punched loose leaf pages, each folder having a cover page and otherwise by and large being hand-written), to which a friend responded that my previous life organization was loose pages in my backpack.  
I didn't think that they were right, but everyone else in my life agreed, and I have been realizing that, while it's never my only method, it is one I always have.  
It's nice being able to not have to damage or destroy books (codexes, I guess (codicies?)), and people in my life need a single piece of scratch paper fairly often (wow this is such a long footnote)}

\item Being appropriately vulnerable?

Generally! I'm sharing the struggles I'm having with people.  
I'm shocked at how positive the response to my latest song is, because \textbf{I} thought that it was a cry for help, but.

\end{itemize}

\end{itemize}

\item Adjacent to Primary and Secondary:

\begin{itemize}

\item Typing Practice?

I did yesterday, and will do once my writing buddy leaves for their day of work.

\item Applying to jobs?

I sent off two new job applications last night and will be starting another today.\footnote{I'm told this job is just a bunch of tests, and there's very little I excel at so much as standardized tests}

\item Reading the things I think could be good?

Nope.

\item Making manim videos?

Nope, I really need to get on this though.  
I feel like I'm always running about 30 percent behind these days, and I have no clue why.  
Might be good to just plot out my life again.\footnote{do I need to plot my life out forever every week in order to not go insane? maybe?  
I'll ask an expert about that today and see what they think about that. I'd love to think that there's another way for me to not feel panicked, or maybe the panic is healthy}

When do I have time for it, who can say?

\end{itemize}

\item Cleaning?

\begin{itemize}

\item Office

Not really, but it would probably be good for me to spend a little more time this week doing that.

\item Home

Not a ton, and I really feel behind on it.

\item Car

I still need to return the telescope. It's tempting to just do that today, but I'm also already in the office, and I like taking the bus sometimes.\footnote{have an appointment at 11 a few miles from here, and I think that it is actually about equal timing between walking the mile each way to my home and driving the car to the appointment versus taking the busses. That's wild, but I care about the earth a little bit}

\item Computer

Not really? I think that more and moreso\footnote{\href{https://english.stackexchange.com/questions/211458/more-so-or-moreso}{Even though spell check here hates it}, I think that it remains a valid word, even if I'm not necessarily using it correctly} it's getting to be clean.  
Then again, the number of files that I have on this writing site has once again ballooned, and I would like to start cleaning it out.

\item Other as needed

N/a.

\end{itemize}

\item External Obligations:

\begin{itemize}

\item Guitar for wedding?

I think that I'm happy with how well I can noodle around on the chords, so assuming that the groom-to-be is cool with what I have\footnote{and, more importantly, the bride-to-be, but I don't know her, so I have no idea if that's relevant or if she even wants music at the wedding}, it's just going to be about practicing so that my fingers don't bleed.

\item Travel plans?

No, but I really need to get on that.

\item Talks for parks?

I like the talk I have right now, and as I discussed with a friend yesterday, I can go on autopilot while delivering it, which is really nice.  
That being said, I think that there are other presentations I could do that might be better.

\item Other requested talks?

Not at all, but also need to get on that probably.

\item Talks for conferences?

No, but that's the thing for the week.

\end{itemize}

\item Tertiary Goals:\footnote{mmmm off by N numbering. No I'm never deleting this footnote, because it brings me a large spark of joy literally every time that I read it}

\begin{itemize}

\item Blogging?

Look at this!

\item Reading?

I restarted the \say{Full Murderhobo} series\footnote{trilogy?} on audiobook again.  
Is it at all high-brow? No.  
Is it enjoyable? Yeah!

\item Web Noveling?

Ughhhhhh this is another on the endless list of things that I fall short of.  
A dear friend just mentioned that they reread the entire series and are caught up, so there's even more motivation to continue writing.

\item Guitar?

I have the song, I don't know if I like the chords entirely as they are, and I know that I don't like the way my voice sounded on the recording I did.

\item Other hobbies?

Not so much, no.

\end{itemize}

\item Quaternary Goals:

\begin{itemize}

\item Letter writing

No, but that is a goal for today, I think.  
After my meeting, maybe I'll go to my cage and write a letter. A small friend\footnote{is this how to refer to a child of a dear friend?} of mine is about to have a birthday, and it would be nice to send a card, even if they cannot read it unaided.

\item Handwriting/penmanship

I did none yesterday, but I really want to be doing more.  
I saw a short tutorial on some mechanical drawing exercises, and I think that a lot of that will intersect nicely with the handwriting, because one goal is distinctive, but the other goal is reproducible.

\item Picking a new signature

No. I don't know if this is actually something that I want to be doing, though.

\end{itemize}

\end{enumerate}

\section{Daily Reflection 27 May 2025}

\begin{enumerate}

\item Top Priorities:

\begin{itemize}

\item Sleep:

\begin{itemize}

\item Keeping sleep time sacred?

Generally!

\item Good sleep hygiene?

Maybe?

\item Sleeping enough?

Maybe? I do feel like I'm needing to sleep more but might just be running myself to the bone.

\item How well rested do I feel?

Eh.

\end{itemize}

\item Feed myself:

\begin{itemize}

\item Did I eat breakfast?

In general I have not been doing well here. Today I've been working through a bowl of oats for like 4 hours.

\item Water?

Not well here either.

\end{itemize}

\item Family:

\begin{itemize}

\item Am I neglecting any familial obligations?

I almost forgot to listen to the album last week, but I made it through!

\end{itemize}

\item Movement:

\begin{itemize}

\item Am I stretching at least 5 minutes per hour of computer time?

No, shoot, time to add the alarms back to my schedule.

\item Am I generally making efforts to be limber?

Eh, I stretched for a little bit yesterday and Sunday, but generally no.

\end{itemize}

\item Spirituality:

\begin{itemize}

\item Time for prayer?

No.

\item Prayer?

No.

\item Time for sacred silence?

Kind of!

\item Deep breaths?

No.

\end{itemize}

\end{itemize}

\item Secondary Priorities:

\begin{itemize}

\item Thesis/ Ph.D. work:

\begin{itemize}

\item Keeping up on the writing deadlines?

No. This week's big goal is making all the plots I want and need for RebelFit.

\item Reading the necessary things?

N/A

\item Making graphs?

Not yet, but I'm hoping to get good at this going on.

\begin{itemize}

\item Visual depiction of Latin Hypercube

\item Visual depiction of Grid Search

\item Visual depiction of random search

\item Visual depiction of Loomis-Wood Diagrams

\item Visual depiction of Spectral Stacking

\item Visual depiction of how the fitness of the spectral stacks is really reliant on the graphs being the right height

\item I guess that the stuff for intro to quantum video counts here.

\item Plots from the actual results of the runs, to make sure that it worked out.\footnote{SSC, AAT, if any vib states were good, what happened to the computations, etc}.

\end{itemize}

\item Organizing citations?

Yeah! Over the weekend I think that I went through all the papers that compare theory and experiment going back to the start of 2021. I'd like to get more, though.

\end{itemize}

\item Love:

\begin{itemize}

\item Taking risks?

Eh, not a ton.

\item Making efforts?

No

\item Showing affection?

Yes!

\item Being honest?

I think so

\item Being open?

Maybe?

\item Being appropriately vulnerable?

Yes?

\end{itemize}

\end{itemize}

\item Adjacent to Primary and Secondary:

\begin{itemize}

\item Typing Practice?

Ope! Let's do that now. Generally doing ok.

\item Applying to jobs?

Submitted my first job application on Friday, hoping to get two more out tonight.

\item Reading the things I think could be good?

eh.

\item Making manim videos?

No, not at all

\end{itemize}

\item Cleaning?

\begin{itemize}

\item Office

It's starting to have entropy sickness again, but I have hope that I can fix it today.

\item Home

It's better than it was but still horrifying. I really need to get rid of things I think.

\item Car

I made it clean so that I could put a telescope inside it.  
Now I need to return the telescope.

\item Computer

Generally! I think that I'm making progress on having an ordered one.

\item Other as needed

\end{itemize}

\item External Obligations:

\begin{itemize}

\item Guitar for wedding?

Yeah! I think that it's coming nicely, and I was even recently inspired to write a whole new song with words and everything.

\item Travel plans?

\item Talks for parks?

I just reused the talk from last year, and I think that it went well?

\item Other requested talks?

Nope.

\item Talks for conferences?

This week's goal is getting all the stuff for the RebelFit presentation ready.

\end{itemize}

\item Tertiary Goals:\footnote{mmmm off by N numbering}

\begin{itemize}

\item Blogging?

Not at all.

\item Reading?

Not really, but I did mainline the Cradle series recently. I want to reread another fun series after that.

\item Web Noveling?

Nope, I want to be better

\item Guitar?

Yeah!

\item Other hobbies?

Wrote a song!

\end{itemize}

\item Quaternary Goals:

\begin{itemize}

\item Letter writing

Nope

\item Handwriting/penmanship

Fair amount of work, actually

\item Picking a new signature

\end{itemize}

\end{enumerate}

\end{document}