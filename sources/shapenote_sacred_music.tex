\documentclass[12pt]{article}  
\newcommand{\say}[1]{``#1''}  
\newcommand{\nsay}[1]{`#1'}  
\usepackage{endnotes}  
\newcommand{\B}{\backslash{}}  
\renewcommand{\,}{\textsuperscript{,}}  
\usepackage{setspace}  
\usepackage{tipa}  
\usepackage{hyperref}  
\begin{document}  
\doublespacing  
\section{\href{shapenote-sacred-music.html}{Shapenote, Sacred Music, and the Second Vatican Council}}  
First Published: 2025 April 22

\section{Draft 2: 22 April 2025}

One of my hotter liturgical takes is that we should not sing almost any hymns.  
The Church is clear that its preference is for, all else being equal, Gregorian chant over polyphony\footnote{especially from the original polyphonic era} over other Catholic music over the sacred music of the region.  
I note, at least, that nowhere in there is the singing of songs which are only ever Lutheran.

I've long been a believer that intent matters, especially in art.  
The fact that we sing hymns written by Martin Luther, who is uncontroversially not a Catholic in good standing, is always ridiculous to me.  
Even when the hymns aren't being written by active apostates\footnote{ooh gotta love a good word}, they are still written, especially the older ones, by a group fundamentally opposed to the Church.

I've also, like the Church, long been a believer that ends do not justify means.  
We cannot effect an evil now in order to cause some future good.  
Hymnody is, at the very heart of its creation, evil.\footnote{hmm I'm not feeling laid back today, am I?}

Many critique the Church for its art and beauty.  
Money spent on art, they argue, could be better spent on the needy.  
I agree with this take in some regards, though I do also absolutely agree with the argument that we should not have joyless existence.  
Also, much of the art being actively critiqued is old.  
The Church is not actively spending money on\say{at least much of. Preservation is its own thing, but I'm willing to say that preserving beauty is a worthwhile goal} the works, and selling them would only grant money once.  
More than that, though, the Church recognizes that when we see beauty, we are oriented towards Beauty.  
When we learn truth, we are oriented towards Truth.  
When we see light, we orient towards the Light of the World.

That is, having beautiful art, especially in the location where masses are held, is itself a way to help the congregation pray.  
Without getting into the debate about how literate the average person was in the age of Martin Luther, the church still taught that beauty is helpful, even and especially when it is hard to understand.  
Luther, among his many evils\footnote{he was incredibly Jew hating, even for people at the time}, introduced hymnody because he did not believe that peasants would understand the beauty of counterpoint.  
Instead, he took popular drinking songs and set them to sacred text.\footnote{and yes, I do get incredibly upset when anyone who advocates the use of hymns in Mass complains about music being too secular in its origins}  
Polyphony, in his eyes, was to be reserved for the elites.

Returning to the title of this folly, though, it is not enough to simply argue against hymnody.\footnote{though I do absolutely think we should be railing more against it. None of the ancestors I can point to would have sung them, so far as I know. It is just inarguably not a part of my own musical history}  
After all, I do not also argue in favor of Gregorian chant and polyphony.\footnote{that's a bit of a lie, I do generally think that polyphony is fun. I'm generally opposed to choral music at mass, though, but if we're going to do it, we should at least do it right. Also, again, polyphony is fun}  
Why should we sing shapenote?

The Church is clear in the Second Vatican Council that the musical traditions from outside the Latin Church, especially in mission areas, are to be encouraged.  
America is not and has never been a Catholic country.\footnote{the fact that the highest ranking nominal Catholics in the government were actively and publicly fighting with the Holy Father is its own thing}  
More than that, the Church is in crisis.  
We need to bring the lost sheep back into the fold, and bring the people who never had a home with Mother Church in.

The sacred musics of the United States are shapenote and spiritual/gospel.\footnote{I'm not going to argue for whether the two are one genre or two.}  
There are a number of considerations I have when advocating for music.  
Our country's long history of oppression towards the Black population, especially\footnote{only because I'm talking music here} in regards to music, should not be ignored.  
I sing in primarily if not exclusively white choirs, and suggesting that we take the sacred music of a group that has been explicitly othered since before American was an identity feels complicated.\footnote{why yes, I do also have feelings about all the Jewish tunes we stole}  
Shapenote, though, is not a wholly or even supermajority Black genre.

Shapenote singing from every level is designed to make it easy for the congregation to join in.  
Even outside of that, every shapenote song that are in hymnals are incredibly popular amongst the people.  
The goal is to get the congregation singing and to make the non-Catholic feel called home.

I do not know any choirs that sing Gregorian chant correctly, and I know few that do polyphony well.\footnote{yes, I do have a major chip on my shoulder about early music, how did you know?}  
All else is not equal.

\section{Draft 1: 22 April 2025}

Something I think about a lot is music.  
Something else I think a lot about is the place of music in the liturgy.  
In \href{what-we-dont-write}{one of my recent attempts at a post}\footnote{look at that gotcha moment}, I realized that song and prayer are intrinsically tied in my mind.

In thinking a little longer, I realize that there is also something almost contradictory in the way that I view music.  
On the one hand, I think that all music is, at least in part, a connection to the Divine.  
\say{The Lord of the Dance}, for all its apparent theological faults\footnote{I personally cannot tell where it blames all Jewish people for killing Christ, but I must defer to the bishopric (is that the word?, ah episcopacy)}, has always had a special place in my heart for that reason.  
From this general love of music, I think that it tracks that I hate the idea of banning any genre.  
I have friends whose parents would not let them listen to rock music growing up.  
It feels the same as any other form of censorship.\footnote{that is, I think that the government shouldn't be allowed to and that in general parents shouldn't}

In the document from the Second Vatican Council, \href{https://www.vatican.va/archive/hist\_councils/ii\_vatican\_council/documents/vat-ii\_instr\_19670305\_musicam-sacram\_en.html}{Musicam Sacram}, we hear the Church once again reaffirm its commitment to music and music in the liturgy.  
There have been any number of documents with at least some amount of Church authority behind them on how music should be performed in the Mass.  
I personally abhor Pope Saint Pius X's \href{https://www.papalencyclicals.net/pius10/tra-le-sollecitudini.htm}{tra le sollecitudini}\footnote{which, for some reason, is not available in English on the Vatican's site}, which is generally pointed to as the first modern Church writing on music.  
However, as the schismatic group implies, Pope St. Pius X has a devoted fan club.  
Anyways, not to get into the argument I'm having more and more with the normal Catholic part of me\footnote{read: the parts of me that don't overthink everything and generally tries to go with the flow} about so much of the things that we claim to be Church teachings actually being matters of rational inquiry, and therefore outside the purview of the Church, but I hate the disdain that practically drips from the document towards so called \say{popular music.}

Returning to Musicam Sacram, it is as far as I can tell, the last document from the Vatican that has the full weight of a Council behind it.  
There's a question deep within me about the places where it disagrees with the Council of Trent on music, but this is also not the place for that.  
Throughout the document, it is very clear that the Church wants the congregation to sing.  
I'm just going to go through the easiest to skim ones to make a list before restarting.

From paragraph 4 we learn that any music composed for the liturgy is sacred, as well as any sacred popular music.  
Paragraph 5 reminds us that the liturgy is better when sung.  
Paragraph 7 gives the advice of what to pick when adding music: start with the most important, which is usually dialogue between the priest and the congregation.\footnote{i.e. we should all be singing the Our Father more often}  
Paragraph 8 has the general Catholic take of \say{if you have a choice, pick the best singer, especially if it's being recorded}.\footnote{side note: love that they put that so early on in the document}  
Paragraph 9 says that any sacred music can be acceptable in the Mass, and that the capabilities of the people must be taken into account.\footnote{hold on to this one, we'll get back to shapenote}

It cites \href{https://www.vatican.va/archive/hist\_councils/ii\_vatican\_council/documents/vat-ii\_const\_19631204\_sacrosanctum-concilium\_en.html}{Sacrosantum Concilium p 116} which itself cites p60 of itself.  
In 116 we get the notorious \say{other things being equal, it (Gregorian chant) should be given pride of place in liturgical services.}  
Now, I have many feelings about Gregorian chant, and having gone to a Seder recently, they are only amplified.  
As far as any scholarly source I have seen claims, Gregorian chant is meant to be sung at the cadence we would read the words.  
I have never once been in a Catholic Mass where I heard chant done at that clip.

More than that, Gregorian chant is fundamentally not the music that people know.  
Paragraph 30 of SC\footnote{I will use abbreviations as I see fit} reminds us that the people should be encouraged to participate, especially through song.  
I'm going to quickly read through Chapter 6 of SC, since it also concerns sacred music.

Gotta love the opening, which says \say{The musical tradition of the universal Church is a treasure of inestimable value, greater even than that of any other art}\footnote{p112}.  
It quickly reminds us that language choice is important, and I have many feelings about the people who say that the documents say more of the mass should still be in Latin.  
There is no effort to make the layperson understand Latin.  
From 114: \say{bishops and other pastors of souls must be at pains to
ensure that, whenever the sacred action is to be celebrated with song,
the whole body of the faithful may be able to contribute that active
participation which is rightly theirs}.  
Composers are to be given special training, which I agree with, though don't necessarily agree when it comes to the \say{especially boys} line.  
As far as I know, there are an equal number of Doctors of the Church who are men and women  
Only other notable things are the love of the pipe organ\footnote{which is fine, in my eyes} and the fact that, especially in mission countries, we need to make special efforts to match our sacred music to the traditions of the people we speak to.

I've seen so much writing about how the world is post-Christian, and so can't help but feel like we need to take from the sacred traditions where we are, to get the people back.

Back to MS.

P11 reminds us that ornate music is not always better.  
P16: \say{One cannot find anything more religious and more joyful in sacred
celebrations than a whole congregation expressing its faith and devotion
in song.}  
In particular, 16B: \say{Through suitable instruction and practices, the people should be gradually led to a fuller—indeed, to a complete—participation in those parts of the singing which pertain to them.}  
That is, we should be making the choirs and people able to sing along with everything.  
Choirs should not sing alone.

Part C feels like it walks that back a little, but.

We are reminded that sacred silence is also important.\footnote{need to muse about how silence is not the default state of music, but static.}

18 once again reminds us that the Church needs to start teaching people to sing from a young age.  
Far less than half of my Catholic friends sing in Mass, almost all of whom because they do not think they are good enough at singing.  
If true, it is a gross failing on the side of Mother Church.

20 reminds us again that, even when you have a killer choir, you still have to let people in.

I find it fun that any choral voicing is acceptable, even \say{if there is a genuine case for it, of women only}.  
I see references to boys but not girls, which is a little confusing to me.

Now we get to the really fun parts: the order that music should be added!  
More important than the Kyrie, Gloria, Agnus Dei, Creed, Alleluia, or Psalm is that the Our Father be sung.  
The Gospel acclimation, entrance and exit rites, and prayer after communion are all also in this first degree.  
I don't know whether the enumerated items within each degree are sorted, but even if not, the Alleluia is in the third degree, among the least important of what is to be sung.  
It is in the same place as the standard places hymns go.

The people cannot be totally excluded from the Ordinary of the mass.\footnote{P34}  
I feel like every choir director I know treats this as a \say{in general} not \say{in every case} kind of rule.\footnote{i promise I'm not writing this to subtweet (rip) my current or any prior directors, I just want to get my thoughts on the page}

Oh interesting, we should be, when singing Gregorian chant, be singing the original settings, not random ones from people.

The line I have always found the most important: \say{Adapting sacred music for those regions which possess a musical
tradition of their own, especially mission areas,[42] will require a
very specialized preparation by the experts. It will be a question in
fact of how to harmonize the sense of the sacred with the spirit,
traditions and characteristic expressions proper to each of these
peoples. Those who work in this field should have a sufficient knowledge
both of the liturgy and musical tradition of the Church, and of the
language, popular songs and other characteristic expressions of the
people for whose benefit they are working.}\footnote{61}  
I do not think that America is generally considered a mission area, but we absolutely have a musical tradition outside of the Church.  
I'd argue America has two sacred music traditions unique to it: shapenote and gospel/spiritual.

Arguing that the primarily if not exclusively white choirs I sing in should sing more music which comes from Black oppression feels somewhat difficult, especially when remembering the way that so much of the popular music in America was actively stolen from Black composers and performers.  
However, shapenote singing lacks a lot of that cultural baggage.  
And, I'd argue just as importantly, every shapenote song in the hymnals I use is a song the congregation actively participates in.  
If the goal is to get the people to sing, we need to meet them where they are.

I find paragraph 63 fascinating.  
What does it mean for an instrument to only be appropriate for secular usage?  
Is not the great mission to go and make all Catholic?

67 reminds us that improv is cool and we should do it more.

Alrighty, let's restart.

\section{Daily Notes}

\begin{itemize}

\item Obligations:

\begin{itemize}

\item Professional

\begin{itemize}

\item Write the thesis

Tragically little progress

\item Revise the thesis

\item Edit the thesis

\item Research for the thesis

Decent amount! Right now I have some computations running and I'm relatively hopeful that they will come out well.  
In retrospect, I probably don't need to be using huge numbers of iterations right now when I'm really just trying to figure out if stuff works

\item Read the books that might be useful for the thesis

\item Start citation tracking

Working on it! For the paper I'm working on right now, I think that I have all the citations that I reference.

\end{itemize}

\item Personal

\begin{itemize}

\item Learn the songs for to jam

\end{itemize}

\item Self:

\begin{itemize}

\item Silence

Way too comfortable. I don't like not wanting to listen to things.

\item Typing practice.

I've done literally none for a long time, and I don't know if that is going to change. I guess it should, so let's go ahead and throw five minutes on.  
It happened! I'm starting to get into a bad\footnote{potentially} habit of restarting lessons at the first mistake I make.  
While this does mean I get more practice on trouble words, it does also distort my writing accuracy.  
Looking at the two photos of my average typing speeds, it does really look like I've made progress, which is both really cool and kind of shocking, because it doesn't really feel like I've gotten any better at this.

\item Keep the phone out of the room for bed

Not at all and WOW it needs to start living outside my room.  
I read a book for a few hours this morning\footnote{two am is morning right?} and that really screwed me up for the energy of the day.

\item Stretch in the morning

Nope! Let's get a little bit of this right now, at least like a front fold.

\item Read at night

I think so, the fun book, though, not any of the goal books. Still, reading is reading.

\item Poetry at night

Oh shoot I have completely spaced this.

\item Clean the home

I have fallen behind here, and that is a reflection on the way I am feeling.

\item Stretching, standing, drinking water

Eh, I suppose so, especially the water. Still not as much as I would like, but.

\item Posture

Nope! Wow I've been letting my shoulders slouch.  
Also wow I need to stretch my shoulders again, they are so tight.

\item No wasted time

I think so?  
I have been in the office until 9 pm every day this week, and that probably isn't the healthiest choice.  
I also have to believe that my productivity is lower for those last hours, especially given what it looks like in the morning.  
For the rest of the week, what would it look like to leave work before 6pm every night?  
That would give us sufficient evening time to do things!

Overall, though, I've felt like I'm not getting enough rest breaks.  
It kind of feels like I have been going from activity to activity without having time to stop.  
I know that's partially untrue, but an hour is truly an awkward amount of time when the two activities I have are both twenty minutes from home and less than four from each other.

\item Eat more than 2 meals a day

I have a very firm memory of doing really well at this yesterday!  
Today, I had a bagel for breakfast, and some carrots for lunch.\footnote{hmm that's not a full meal is it?}  
I'm planning to go grocery shopping tonight, though, so that should be a good place to get some more calories in.

\end{itemize}

\end{itemize}

\item Goals and Growth:

\begin{itemize}

\item Ends:

\begin{itemize}

\item Letter writing, get into more

I'm going to give myself some grace here, because it was literally Easter on Sunday.  
Tomorrow, I think that I might try to avoid the office for the morning and just work in a few different locations.  
If that makes me more productive, great! I know to avoid the office.  
If not, also great, and then I know that those locations are not winners for me.

All this to say, tomorrow I'm planning on spending some time in my cage, and I think that it could be nice to write another letter or two.  
Probably just one, but I would also like to read the etiquette books to know what goes in them.  
Looking at the website for one of the guides\footnote{wild, can I just say. It feels very like 2020s aesthetic and then I read the content and it's fresh out of the mid 1900s}, it seems like generally one is just supposed to know what to put in a letter?  
Eh, I'll try something out and see how it goes.

\item Handwriting, pick and make the new one

I'm enjoying working with lower case letters again! I do still kind of feel like I don't have enough control to make each letter look identical and be spaced the same, and I do catch myself often going into cursive for a few letters or back to all capitals on a difficult word.

\end{itemize}

\item Means:

\begin{itemize}

\item Typing speed, improve it.

I like having this in two places because it means that I get two chances to remind myself. I don't think that I want to spend another five minutes practicing, but I do have some typing games that could be fun when I finish with work.

\item Reading, do more of it

I finished the series I was reading. I forgot how it just never gets brighter.

\item Blogging, do it.

Well, in my defense, there has been a lot going on.

\item Writing things that are not the blog and thesis, do

I've been doing ok about keeping a hand written journal equivalent these past few days.  
Not entirely sure if it's the best use of my time but I enjoy it and I think that any time I spend on trying to figure myself and my priorities out is generally time well spent.

\end{itemize}

\end{itemize}

\end{itemize}

\end{document}