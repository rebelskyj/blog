\documentclass[12pt]{article}[titlepage]
\newcommand{\say}[1]{``#1''}
\newcommand{\nsay}[1]{`#1'}
\usepackage{endnotes}
\newcommand{\1}{\={a}}
\newcommand{\2}{\={e}}
\newcommand{\3}{\={\i}}
\newcommand{\4}{\=o}
\newcommand{\5}{\=u}
\newcommand{\6}{\={A}}
\newcommand{\B}{\backslash{}}
\renewcommand{\,}{\textsuperscript{,}}
\usepackage{setspace}
\usepackage{tipa}
\usepackage{hyperref}
\begin{document}
\doublespacing
\section{\href{thesis-prep.html}{Ib Thesis Prep}}
First Published: 2023 November 16

\section{Draft 1}
his morning, after my more or less daily writing session for NaNoWriMo\footnote{the more or less part is referencing the fact that a friend and I try to go together every day.
I do my NaNo every day, even when we don't work together}, I realized that I had no clue what I wanted to write about tonight.
It was suggested that I could muse on fan fiction, generally as a concept, and that seemed like a strongly potential idea.

However, my day ended up being far more eventful than I had expected.
I'll go through the particularly notable events as a series of vignettes, before seeing if I have something I can find a title for as a musing.
If not, hopefully I'll be inspired to find something to muse about from my journaling here.\footnote{the fact that I treat this site as a journal is certainly an interesting change.
Oh! I just remembered that one other idea I was given/ came to was reflecting on whatever I posted about five years ago today (the first year of the blog, and the most successful iteration).
It's a little strange to realize that I started this blog almost a fifth of my life ago.
It feels both like far more than a fifth of my life has passed me by and like I'm exactly the same person I was then.
I'm sure that's not too uncommon of a view, though.}

While walking to work, I realized that I haven't updated my yearly goals, which is a big oopsie.
It's a little late for that\footnote{wow is this foreshadowing? 
Why is author writing the post so late?
Will we ever find out?}, but I put it on my to do list for tomorrow, so hopefully I remember to do it.
Anyways, among the exciting things of the day was reading a really interesting old paper.

The paper itself was fun, but it cited a few textbooks for explanations of derivations.
As it turns out, the textbooks were simply early quantum texts, and just framed how the Born Oppenheimer approximation\footnote{you can treat the nuclei of a molecule as stationary when solving for most molecular transitions} is justified.
I realized how much I really appreciate older textbooks.
There are absolutely a number of reasons for that.
One major one, I realized today, is due to the limits of the system.

I read recently about how absolutely weird it really is that we can have so many images in all of our scientific papers these days.
In earlier times, it was a massive pain to get images into books, especially if you didn't want to dedicate an entire page to them.
As a result, rather than simply being able to go \say{as you can see from the graphs,} authors had to consider how to best\footnote{that phrasing felt wrong for a second.
I realize it's because I'm splitting the infinitive, which was frowned upon in early American grammatical conventions.
The reason, like many, comes from the fact that everyone agrees that Latin is the best language.
Since Latin has one word infinitives, you cannot split them.
As a result, people thought we should not do it in English either.
I hate that for a lot of reasons, not least that I like the fact that to go quickly and to quickly go have shades of different meaning.
By labeling one as inherently incorrect, assuming that usage doesn't change, all you do is pressure people into one framing of a situation} communicate information without visuals.

I think that there's also an element of novelty that modern textbook authors miss.
The textbook I read today, for instance, explained that it is clear that there is an optimal location for all the nuclei in a molecule that is neither totally compressed\footnote{because positive charges repel} nor infinitely separated\footnote{because we know a priori that molecules are real.
I have a whole rant brewing in my mind about how much of science is not really provable, but rather relies on \say{well yeah, of course this is right, look}, and how we diminish the importance of that in a lot of scientific discussions.
Of course, part of that rant belongs in the vignettes section of today's musing, so we'll wait for it}.
I said that to my group mates, and most of them rolled their eyes and said it was obvious.
It is obvious, to me, who has now taken three semesters of explicitly quantum chemistry and had the concept implicitly taught in any other number of courses.
If it was my first semester in quantum, though, I think that framing would be helpful.
That is, we know that there is an optimal structure, and we can constrain it slightly.

Honestly, I kind of just want to go into my rant right now, but I said I'd do a few vignettes, so we'll rush through them.
Probably because of the way I loved the textbook, a few group members started to joke about how long my thesis will be.
One phrase that came up more than once is \say{you should really be getting a degree in philosophy}, which makes me kind of sad.
The degree that I'm going for is explicitly a doctorate of philosophy.
Just because the modern world has moved past the idea that a Ph. D. is meant to represent philosophical reflections on the field, I don't see why I need to.

Anyways, I realized that there are a number of ways that I could write an absurdly long thesis, if I felt so called.\footnote{my group members tried to argue that I wouldn't have the time to write such a long thesis.
The fact that I can regularly put out five thousand words a day did not seem to sway them, for some reason.
I do think that they underestimate the quality of my sprint writing, but I will acknowledge that most of it is not directly ready to go into a thesis (maybe)}
There are a few ways that I could pad the length, which I will discuss after finishing vignettes.

I then found out that my choir rehearsal was cancelled for the night.
Almost immediately after that, someone in the choir asked if I wanted to sign up for a date auction with them.
I had no reason to say no, said yes, and then went to a young adult social night, which was really fun.

I also went to a cool seminar and got lunch with the seminar speaker today.
I think that's most of the vignettes.
Onto the meat of today's post: my path to an infinite thesis.

There are really two paths that can increase the length of my thesis: going deeper and going wider.
To go wider is easy, I just have to make a wider claim of what I did that is relevant to my degree.
Since I have given talks as part of my degree on a variety of subjects, there is an argument to be made that I should\footnote{or at least could} dedicate some space to them.
Of course, a long discourse on tuning theory may feel slightly off tone in a chemistry dissertation, which is where going deeper goes.

Going deeper can mean a lot of different things.
Especially since I'm doing a fairly heavy computational project, I feel like having a description of how we can start at a molecule and get to rotational numbers would not be bad.
However, since I like to take things to an extreme level, there are other ways we can frame it.

One joking idea was the history of the universe as seen through the lens of rotational spectroscopy.\footnote{or, later, through the lens of a bunch of different disciplines.}
In the beginning, there was the big bang.
There were no molecules, so it's not interesting.

The first stars formed.
They also didn't have any molecules with dipoles, at most having diatomic hydrogen, so are also not interesting.\footnote{ooh, I suppose D existed, so you could have like DHD+, which does have a dipole.
Shoot}

Then the stars exploded.
Those are all ions and atoms, so not interesting.\footnote{I think you can see where this is going.
In some regards it's a play on a semi famous quote in astronomy that things are either chemistry or interesting.}

As much as I think that could be a fun thing to do, and may try as a creative writing exercise, that is not quite as academic as many might want for a thesis.\footnote{hmmmm, I wonder if there's a way to phrase and structure the chapter so that it is interesting.
Cosmology as seen through rotational spectroscopy is limited, of course, but also hmmmm. I will consider more}
However, there's another way to go deeper.
As I said, I could go through how we go from a molecular structure to rotational constants.

There's no reason that I need to stop at molecular structure, though.
We could go one layer deeper, to how atoms create molecules.
Now, if I were a particle physicist, I could go the step lower to how atoms are constructed, but, again, the degree is philosophy.
Especially since the conception of an atom is an interesting historical trail, it could be fun to trace that.
One other cool thing could be to take some metaphysics.

I could\footnote{initially typed we, because I thought of this as a science thing, and as we know, science uses the royal we (a phrase not all my friends know about for some reason)} go into a whole discourse on what matter is, not as a matter of physics but as a philosophical thing.
I did say a few times that one goal I have for a thesis is that I can\footnote{changes in scientific understanding independent} use it as a text for any course I could likely teach in the future.
Now, does anyone really think that's an appropriate idea?
I sure kind of do, but I don't know if it will be approved.

Oh! Right, I was going to rant about modern textbook design.
I think another issue is that people today are too afraid of using words.
There are reasons for this, to be sure.

First and foremost, writing is no longer emphasized at any real level of education for a growing scientist.
You can see this in who gets recruited for technical writing.
Companies find it easier to teach English majors science than science majors how to write.

It goes deeper than that, though.
I think that most of the scientists I know only try to speak to other members of the field.\footnote{I know that it is a fairly common reaction when I say that I give a lot of public talks that scientists don't like to do it (this is phrased badly, and is a sign I should start wrapping up this musing before my brain goes too much more to mush)}
There are shades of this, obviously.

I'm only going to be writing to people fluent in English, for instance.
That does cut out a large portion of the world.
If I cut it down to chemists who understand basic chemical nomenclature, that cuts it down significantly again.
Most people would argue that a Ph. D. thesis can be targeted to at least the level of an upperclassman undergraduate in a field, though.

The issue I have is that people go even further than that.
They repeatedly use phrases that only have meaning to people within their own specific subfield, sometimes to the point that there are maybe twenty people in the world who can understand the jargon.
In a research paper, I can understand and even respect that choice.
After all, a research paper is meant to focus towards the field.
There are other avenues to share research with the broader public.\footnote{for all that no one really uses them ever}

I think that in a lot of people's views, a thesis is just a continuation of that thought.
Ok, think is probably a bit of an understatement.
Given that the prevailing way to construct a thesis that I've seen recently is just stapling papers together, it is exactly the thought process.

Ope I never said why I respect the choice.
Jargon has a purpose, and I will never deny it.
There are shades of meaning, and it is far more efficient and effective to use a word with shared meaning rather than dancing one's way around it for a paragraph.
If I want to talk about benzene, for instance, there is no reason that I should have to describe it as a structure where each of six carbons is joined in a ring with on average one and a half bonds to its two neighbors.\footnote{especially since that's slightly inaccurate.}
If I'm going to talk about benzene to a broader audience, though, I might need to do that.

Of course, if I'm talking to a broader audience, I might have to take the step back and explain what a bond is, and what exactly separates carbon from every other element.
I might also, depending on the context I'm speaking about benzene, not need to reference its structure at all.
If I'm describing the fact that there are many clear liquids, for instance, I can just say that benzene exists.

I do agree that I think a little more philosophically about my research than most of my peers, I'm realizing during this musing.
Much like a child that asks \say{why} to each successive answer, I'm dissatisfied with ending my field at something above first principles.
Of course, first principles means different things in different situations.

I cannot think of a way that I can meaningfully connect the Higgs-Boson to the rotational constant of a molecule, for instance.
There are reasons for this.

One major reason is emergent properties.
An emergent property is just when an ensemble of something behaves differently than you would expect from a single piece.
It's related to another concept that I can't remember the name of but that I'm sure I remember an XKCD about.
I can't find the comic right now, for all that I remember it clearly.\footnote{I found it! \href{https://www.explainxkcd.com/wiki/index.php/1734:_Reductionism}{Reductionism}}

Ok the concept is called reductionism, and it says that you can explain something by just explaining its component parts.
The comic points out the ridiculousness of this approach, by separating the word \say{reductionism} into its component letters and their specific meanings.
I don't really like reductionism as a concept, and particularly not when looking at my molecules.
They are fundamentally different than a collection of quarks, and I don't know how much learning about quarks helps you to understand the rotations.

Of course, the same is true for any piece of information.
On its own, it has no use.
It gains value almost by existence of any other piece of information.

Anyways, while I don't think that it's useful to describe first principles in Chemistry\footnote{generally.
I fully acknowledge the fuzzy edges of the discipline, but here I'm speaking for at the very least the chemistry I'm doing, which relies on molecules only barely being constructed of atoms (in that like we abstract away atomic information as quickly and fully as possible)} as anything smaller than an atom.
Of course, there are other orthogonal directions we can take.

Ensemble averages in statistical mechanics, for instance, are only really useful when you explain the quantum classical divide.
Rotational spectroscopy, when I take a step back and look at it, is kind of wild.
It's inherently a statistical mechanics problem, because we're probing millions and trillions of molecules and seeing how they respond in ensemble averages.
However, the way that they respond still correlates really neatly to discrete quantum transitions.
I'm sure there's something analogous between the way that pressure broadening makes multiple lines look like one line and the way that energy levels become bands in metals.

Oh gosh, energy levels are something I haven't even thought about or addressed.
In general chemistry, they're all over the place.
Despite the fact that I'm doing quantized transitions, we never really use them in rotational spectroscopy.\footnote{I'm realizing this isn't true.
We do when discussing coriolis coupling, which is where two rotational transitions of similar energy both shift slightly to have even more similar energy.
I'm a little unclear why that happens, but I think it's something about how vibrational and rotational energies aren't actually separable, we just pretend they are and add fudge factors}
I wonder why that is.

A few reasons pop to mind immediately.
For one, we often work with well over ten thousand energy levels.
In order to represent that on a page in any meaningful way, the page would need to be massive.

Ope ok so returning to the essay.
People today are afraid of language, which has a lot of reasons.
I'm not afraid of having a distinct voice in my thesis, which I think will help, and I'm not against making it broadly understandable.
I'm willing to put forth the extra energy to make it theoretically approachable.

Older textbooks come from a place of authors also not knowing as much, because the fields had so much less information.
As a result, if they had five pages to fill on quantum mechanics, there was less information they needed to cram, which meant there was more time that could be taken to explain.
Anyways, this feels like a good place to end for the night.
It's a rambly musing to be sure, but I plan to spend some time in the coming days and weeks revising some of these rambles.

Daily Reflection:
\begin{itemize}
\item Did I write 1700 words for NaNoWriMo?
I did! Just updated on the site too
\item Did I write a chapter of Jeb?
I did not. As mentioned, today was a busy day, which mean that I had less time and space to write.
Halfway through the month, I'm more ok with the fact that I can't write a long blog post and a chapter of Jeb a day.
Right now I'm valuing the long posts more than the Jeb.
\item Did I blog? This feels incredibly rambly, but I feel like I touched on a lot of things that I would otherwise forget and want to remember.
\item Did I stretch? I did stretch after last night, but not today.
\item Am I doing better at prayer than a rushed and thoughtless rosary? I spent time around Catholics and spent some time reading the book of essays, so kind of.
\item Am I doing a good job writing letters to friends? I really felt like I would have time for it today, but didn't.
The hours passed way too quickly.
I need to get back into writing things down physically.
For whatever reason, an inked page is easier for me to follow than a webpage.
I keep hoping that won't be true, but it keeps being true, and I need to learn to accept that fact.
\end{itemize}


\end{document}