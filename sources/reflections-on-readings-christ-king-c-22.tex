\documentclass[12pt]{article}[titlepage]
\newcommand{\say}[1]{``#1''}
\newcommand{\nsay}[1]{`#1'}
\usepackage{endnotes}
\newcommand{\1}{\={a}}
\newcommand{\2}{\={e}}
\newcommand{\3}{\={\i}}
\newcommand{\4}{\=o}
\newcommand{\5}{\=u}
\newcommand{\6}{\={A}}
\newcommand{\B}{\backslash{}}
\renewcommand{\,}{\textsuperscript{,}}
\usepackage{setspace}
\usepackage{tipa}
\usepackage{hyperref}
\begin{document}
\doublespacing
\section{\href{reflections-on-readings-christ-king-c-22.html}{Reflections on Today's Gospel}}
First Published: 2022 November 21

Luke 23:37 \say{they called out, \nsay{If you are King of the Jews, save yourself.}}
\section{Draft 2}
The liturgical year's last Sunday is always the Solemnity of Christ the King of the Universe.
In 2018, I noted that the year was the Year of Grace, though I don't know what the year is this time.

Something I found really powerful about the readings is how they connect to each other and to Christ's Kingship.
The first reading shows us David's entry into kingship, which is the line through which Christ claims His.
The second reminds us that everything was created through Christ for Christ.

And, the Gospel serves to remind us what the King's role is.
He is not some warlord, conquering nations under a sword.
Rather, he dies on behalf of those who revile and spite him.
\end{document}