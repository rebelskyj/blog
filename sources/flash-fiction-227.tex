\documentclass[12pt]{article}[titlepage]
\newcommand{\say}[1]{``#1''}
\newcommand{\nsay}[1]{`#1'}
\usepackage{endnotes}
\newcommand{\1}{\={a}}
\newcommand{\2}{\={e}}
\newcommand{\3}{\={\i}}
\newcommand{\4}{\=o}
\newcommand{\5}{\=u}
\newcommand{\6}{\={A}}
\newcommand{\B}{\backslash{}}
\renewcommand{\,}{\textsuperscript{,}}
\usepackage{setspace}
\usepackage{tipa}
\usepackage{hyperref}
\begin{document}
\doublespacing
\section{\href{flash-fiction-227.html}{Flash Fiction Friday}}
First Published: 2023 November 17

\section{Draft 2: 18 November 2023}
I find that I'm struggling to begin this second draft.
On the one hand, I want to simply reflect on what I wrote last night.
On the other, a reflection is its own post, not a new draft of a post.
So, let's rewrite my very scattered thoughts from last night into something a little more coherent, and then reflect?\footnote{that feels like a reasonable ordering}

As I mentioned \href{flash-fiction.html}{last Friday,} there is a weekly writing event called Flash Fiction Friday.
I really like it as a concept, especially since they limit submissions to 100-1000 words.
There's something really nice about having to tell my narrative in such a tight format.

This week is apparently the 227th that they've put out a prompt.
The prompt was \say{sands of time.}
My mind immediately leapt between a number of ways that I could respond to the prompt, and I think it could be interesting to reflect on some of how my creative process works, if only to see if it helps me to construct narratives better in the future.

I do not think that I am alone in first thinking about an hourglass when thinking about the sands of time.
There are so many ways that the concept is used in fiction that I've loved.
Right now,\footnote{and I think yesterday too,} the first usage that popped into my head was Death in Discworld.\footnote{Wow I really need to reread Discworld soon.
Now that I'm in the part of my life where I want to focus on craft, the fact that Pratchett manages to tell such deep, complex, and interesting stories in such a small number of pages.
Given that the majority of what I'm reading these days is serialized fiction, which, much like Dickens, uses hundreds of words where a single could suffice, I think it will be interesting to focus on how to tell a tighter story.
Anyways, back to the musing}

I've taken to rejecting the first idea that pops in my head reflexively.
The second idea that pops into my head, by contrast, gets an inordinate amount of consideration.
Yesterday, my second thought was how glass and sand relate.

As most people know, when you melt sand down, you get glass.
As a person who has taken some coursework in material science, though, I am fascinated by the way that they two substances are so fundamentally different.
Sand is crystalline, which in some circles of the world means that it is solid.\footnote{there are places where solid and crystalline are definitionally the same. I hate that as a concept, for all that I recognize the use of describing things like that.}
However, in bulk, sand behaves like a liquid.
You can pour it, and it does not hold together without something to bind it.

Glass, by contrast, is definitionally a glass material.\footnote{it bothers me more than I can express right now that the material science term for the entire phase of matter that melted sand becomes is glass.
Like I get it, the vast supermajority of the cases where we use that phase of matter are when we're dealing with actual glass, but it still bothers me.}
It behaves, on incredibly long timescales, almost as though a liquid.
This is because\footnote{I think, don't quote me.
Most of this information comes from vague recollections of the polymer class that I took the semester I started this blog.}
it is completely amorphous.

An interesting consequence of this fact is that glass, even when ground down to particles the same size as the sand it was initially formed from, does not return to being sand.
Instead, the tiny beads of glass remain completely amorphous.
Now, I'm certain that over some time scale they eventually reconvert to crystalline, if only because I know a priori that a lot of the sand is, in fact, crystalline.\footnote{the rant about a priori knowledge having value is going to come at some point, as soon as I can figure out how to make it cogent.}

With that fact in mind, I started to think about what the metaphor becomes with sand and glass.
Sand is a collection of completely discrete and stable particles.
Glass is something solid and yet not static at all.

From there, my mind immediately moved to glass as memory.
Memories feel and seem stable, but there countless studies that show how intrinsically fluid they are.
At that point, I had sand as moments of time, and glass as the meaning someone constructs.

I tried framing the story in a few ways, though I ultimately ended up with something that felt more like literature or philosophy than a short work of fiction.\footnote{the line between the three is incredibly hazy, I'll happily admit.
It just feels like there's something to the fact that there was only implicit story in my writing, rather than the explicit narrative that I shoot for normally.
Actually, as I think about it, that tends to separate the prose and poetry I write from a lot of what I think of when i think of literature or philosophy.
There tends to be explicit change in the writing I do, a way in which the world the story or poem is placed has changed as a result of the story being told.
(That phrasing works with my apparent goal for FFF, which is trying to find a way to describe the fact that the action of assigning meaning is, itself, meaningful.)}
I ended my musing by saying both that I would read the post in the morning, and that I thought I would dislike it.

The friend I'm writing with this morning and early afternoon read my post and asked me to expound on my claim that \say{here is no genuine way to express something complex.}
So, let's go through the post and see what we think.

I feel like it's a really rough post.
As I said, there was not really a narrative.
Reading it today, it really feels like what I have is the bones of a story, rather than the story itself.
That being said, I don't find it pretentious, which is a nice change.

Alright, let's explore why I feel like expressing complexity inherently relies on artifice.
I find that I'm immediately drawn to a conversation I had with a friend on Wednesday.
As it turns out, this friend has a degree in poetry\footnote{on some level, I didn't really ask exactly what the degree was, but I know that the friend wrote a senior thesis of poetry (I wonder if there's something inherently dehumanizing about the way that I try to anonymize everyone else that I refer to in these musings.
It certainly does something to remind me that this posting is, on some level, artificial (shoot that belongs outside of the footnote. Will bring up to explore))}, and he told me about an experience he had in one of his upper level classes.

One of his classmates refused to rewrite any poem.
The classmate tried to justify it by saying that the poem expressed what it tried to express.
To edit a poem was to compromise the artistic vision.

That anecdote came as we discussed the way that learning to revise is such a difficult skill.
Separating the created work from the inspiration of creation is a difficult endeavor, especially as the creator.

I'm struggling to articulate how that prior sentence connects to my feeling of artifice.
I think that, at first, everything is black and white, at least to me.\footnote{there are essays about how that phrase is problematic, and I've considered how I feel about it.
At this point, I don't know that I have a phrase which works better, and I find that policing language with the explicit goal of never saying anything even potentially harmful is itself a way to harm.
Shoot, that also belongs in the main text}
To add nuance is to explicitly reject the initial impression.

I suppose that takes us to the question of what genuine means.
Is my most genuine reaction the first impulse I have, or is it what I decide to believe when I have the chance to explore the consequences of each belief?
I don't have a good answer, for all that I have some more ways of musing about the difficulty of being genuine and nuanced.

I'll use an example from my own life.
When I write these posts, I often start by using identifying information about the people in my life.
However, I do not want to harm them in the real world, and I know that there is a chance that anything I say here, regardless of how innocuous it feels at the time, could later come back to haunt one of the two of us.
With that in mind, I scrub the identifying information as best as I can, until I'm sure that sometimes I've referenced people in a way that they would not even recognize as themselves if they were to read my blog.

As I keep writing each post, though, I fall into the mindset of anonymizing information.
By the end of each draft, I instinctively write about others in a way that makes them faceless beings who exist solely in the context of the narrative I'm telling.\footnote{hmm I wonder if that tendency is part of why people have commented so much on the characterization in my books.}

Which register of my writing is the real me?
Is it the one who uses friend's names, or the one who writes this blog?

I find that my gut reaction is to say, here, at least, the way that I construct the narrative as I write it is the genuine article.
At the beginning of the post, the real me refers to friends by names.
By the end of the post, the real me refers to friends only in how they intersect the story.

When I go back to add nuance, however, I have to choose which register to use for the entirety of the post.
Since I made the conscious decision long ago to have this blog not name others explicitly, that is the register I shift things to.
Given that I write each musing with the understanding that it will be posted, there's an easy argument to be made that what I meant to construct from the beginning was something that did not refer to people by name.

Abstracting back to the general question of what it means to be genuine, there is a tension between what is created and the vision.
Naively, the creation feels like the most genuine version of an act of creation.
After all, there is no way to perfectly recreate the conditions that lead to every decision we make.
No man can cross the same river twice, and everything we effect affects us.\footnote{ooh I love that ending phrase, for all that I also know I would hate having to read it}

However, like most naive conceptions, that idea does not hold up perfectly to scrutiny.
If I write a song about love, I almost always want there to be rhyme, because I feel like the sing songy nature of rhyming couplets fits well with my conception of what a love song should be.
If I end my first line with something difficult to rhyme, though, it does not compromise the idea to change that word.

After rambling far further than I meant to for a revised and trimmed draft, I think I do believe it's possible to discuss something complex in a genuine fashion.
I think that it requires a separation of intent from creation, though, which feels artificial in a lot of contexts.



\section{Draft 1}
It's another Friday, which means that it's time for another musing about the Flash Fiction I'm planning to write today.
It's far later than I'd like to be starting this musing, but I don't regret the time that I spent doing other things today.\footnote{more or less from 4:30 on tonight I've been celebrating a close friend's birthday, and that seems more important to me than an arbitrary writing goal.
I'm sure that there's something to be said about finding a balance for writing where it can be the main thing that I do, but still takes a backseat to the important parts of my life.
I should reflect on that further, and this footnote feels like as good of a place as any to do so. (I don't want the main content of this blog post to be my reflection on writing, because I decided arbitrarily that it would be planning for the flash fiction (as you might have seen from the opening line))

I do very much appreciate the fact that I write.
I love looking back at the content I've read, and there's a part of me that beams whenever someone tells me they've enjoyed anything I've written.
I can look at the writing I've done through the years and see a clear upward trend in quality, or at least quality per unit time.
I feel like, at a general level, at least, the second draft of anything non fiction I write is about twice as good as the first draft, and it often takes far less time.
However, all the content that goes into the first draft is important.

Last week's musing on the readings is a great example.
Before I had written the very long reflection, I don't think that I would have gotten to the musing that I had.
The second draft is strong, in large part, because the first draft is willing to take so many detours.

So, how do I square that reality with the fact that I find my writing improving?
There are two ways.
First, I find that the rabbit holes my mind explores are generally better the more I've written and read in my life.
I continue to grow as a person, and continue to find better ways to express a truth that I see or feel or believe.
Second, I get through each diversion more quickly.
For all that I am willing to ramble through these musings, I strongly prefer being able to do them in a single sitting.
The composite reality that the more I write, the more comfortable I am with spending a continuous amount of time writing and the fact that the more I write the physically faster I write means that I get through a lot more content in a first draft of a rambling musing.
There's the part I'm not addressing here which is that the quality of my first drafts is also increasing a fair amount, simply because my mind is getting better at plotting how a small narrative will end from a beginning.

All this to say, I think that the healthiest thing for me is to have writing be an important part of my life, but a part that is always able to be superseded by a friend or other social event.
I know myself well enough to know that I am healthier when I spend more time with friends.}
The prompt today is \say{sands of time.}

Probably because I have thought about sand for the book I'm writing\footnote{Jeb not NaNo}, I immediately thought about the fact that sand is crystalline, while glass is amorphous.
While walking home, I considered a little more in depth how I could use that to a narrative.

I think that there's something to the idea of each individual grain of sand being an explicit passage of time.
Regardless of how we feel about it, there exists an objective way to measure the time displacement that occurs on earth.
What's more meaningful by far, though, is the way that we turn these objective passings into a narrative.
I'm not entirely sure how to make fire or heat into the metaphor, but maybe there's something about lightning which forms fulgerites.

Ok actually, I love the idea of fulgerites.

So, let's reconstruct the narrative from there.
A fulgerite is a place where lightning has melted and reformed the sand into an interesting shape.
Sand, as sort of given in the prompt, is the passing of time.
Lightning is a person constructing meaning out of a series of events.
The result is a memory.

This feels like something that might be better expressed as a poem, for all that I refuse to do this as anything but prose.\footnote{entirely because I want to practice prose this month, and not at all because of any legitimate reason}

Ok, let's go through the same sets of questions I asked myself last week.
My FFF question set apparently is:
\begin{itemize}
\item Why am I writing it? Today, because I want to explore this idea of time as observation.
I'm realizing that a lot of how I view myself as a more literary writer is by emphasizing the importance of observation and human naming.
That's probably worth interrogating on another occasion.\footnote{the clock is rapidly ticking down to the end of the day, or I would be more tempted to do so now}
\item Person? Right now I'm leaning towards a third/fourth person telling.\footnote{fourth in the sense of like \say{one can imagine}}
\item Any other gimmicks?
I don't think so today.
\end{itemize}
Alright, time to try writing this.
I think that I'll let it be more poetic in phrasing than I tend to emphasize, in large part because I find that I wax poetic the later at night it is.\footnote{whether that's something intrinsic in me or the fact that I have spent more than a few months writing a poem as the last thing I do before sleeping is another conversation for me to have with myself.}

Well, I wrote it, and redrafted it at least four times.
I switched between first and third person accounts of the narrative as I went through the different drafts, which I think did a lot to help me tell the story that I wanted.
In my author's note, I admitted that I don't know if what I wrote really counts as prose or poetry, but I am as happy with it as I think I will be.
I'm sure that in the morning I'll read what I wrote and feel like it was too pretentious or trying too hard to be deep.\footnote{Then again, that's almost always true for me.
There's a part of me that does really believe that to express something well is to express it ingenuinely.
Or, at least, that part believes there is no genuine way to express something complex.
I don't think that's universally true.
Certainly when I read a book of theology, for instance, the depth of thought there feels more genuine.
I think that I don't trust myself enough to let myself make that, though.
We'll see.
Maybe I'll love the post tomorrow.}

It's now well past time for sleep, so I'll do my questions and head.
Daily Reflection:
\begin{itemize}
\item Did I write 1700 words for NaNoWriMo? I did! I'm now officially a full day ahead of schedule.
\item Did I write a chapter of Jeb? I wrote my ideas for the chapter and about 500 words of content. As I mentioned at the top, today was far longer than I planned on, for all that I do not regret it.
\item Did I blog? I'm realizing that I can use footnotes for all my long diversions, which lets my actual text be\footnote{if only slightly} more coherent.
\item Did I stretch? No.
\item Am I doing better at prayer than a rushed and thoughtless rosary? No? I feel like I've thought about my place in the universe a fair amount, which involves a certain level of prayerful intention for me, but I haven't done anything explicitly so.
\item Am I doing a good job writing letters to friends? Today passed too quickly. Yesterday I mentioned making a to do list, and that did absolutely help a lot.
Tomorrow I plan to spend a fair amount of time writing with my friend, which should give me the space I need to get through the chapter of Jeb I'd like to write, the NaNo writing I need to do, and the letter that I feel like I should write.\footnote{I'll be seeing a show with friends that I think I saw in London, so I'm excited to rereview it!}
Should is probably the wrong word.
At this point, I've written all the letters to people I felt an obligation to write. What's left is the letters I want to write.
Framing it that way is probably healthier.
\end{itemize}\end{document}