\documentclass[12pt]{article}[titlepage]
\newcommand{\say}[1]{``#1''}
\newcommand{\nsay}[1]{`#1'}
\usepackage{endnotes}
\newcommand{\1}{\={a}}
\newcommand{\2}{\={e}}
\newcommand{\3}{\={\i}}
\newcommand{\4}{\=o}
\newcommand{\5}{\=u}
\newcommand{\6}{\={A}}
\newcommand{\B}{\backslash{}}
\renewcommand{\,}{\textsuperscript{,}}
\usepackage{setspace}
\usepackage{tipa}
\usepackage{hyperref}
\begin{document}
\doublespacing
\section{\href{flash-fiction-227.html}{Flash Fiction Friday}}
First Published: 2023 November 17

\section{Draft 1}
It's another Friday, which means that it's time for another musing about the Flash Fiction I'm planning to write today.
It's far later than I'd like to be starting this musing, but I don't regret the time that I spent doing other things today.\footnote{more or less from 4:30 on tonight I've been celebrating a close friend's birthday, and that seems more important to me than an arbitrary writing goal.
I'm sure that there's something to be said about finding a balance for writing where it can be the main thing that I do, but still takes a backseat to the important parts of my life.
I should reflect on that further, and this footnote feels like as good of a place as any to do so. (I don't want the main content of this blog post to be my reflection on writing, because I decided arbitrarily that it would be planning for the flash fiction (as you might have seen from the opening line))

I do very much appreciate the fact that I write.
I love looking back at the content I've read, and there's a part of me that beams whenever someone tells me they've enjoyed anything I've written.
I can look at the writing I've done through the years and see a clear upward trend in quality, or at least quality per unit time.
I feel like, at a general level, at least, the second draft of anything non fiction I write is about twice as good as the first draft, and it often takes far less time.
However, all the content that goes into the first draft is important.

Last week's musing on the readings is a great example.
Before I had written the very long reflection, I don't think that I would have gotten to the musing that I had.
The second draft is strong, in large part, because the first draft is willing to take so many detours.

So, how do I square that reality with the fact that I find my writing improving?
There are two ways.
First, I find that the rabbit holes my mind explores are generally better the more I've written and read in my life.
I continue to grow as a person, and continue to find better ways to express a truth that I see or feel or believe.
Second, I get through each diversion more quickly.
For all that I am willing to ramble through these musings, I strongly prefer being able to do them in a single sitting.
The composite reality that the more I write, the more comfortable I am with spending a continuous amount of time writing and the fact that the more I write the physically faster I write means that I get through a lot more content in a first draft of a rambling musing.
There's the part I'm not addressing here which is that the quality of my first drafts is also increasing a fair amount, simply because my mind is getting better at plotting how a small narrative will end from a beginning.

All this to say, I think that the healthiest thing for me is to have writing be an important part of my life, but a part that is always able to be superseded by a friend or other social event.
I know myself well enough to know that I am healthier when I spend more time with friends.}
The prompt today is \say{sands of time.}

Probably because I have thought about sand for the book I'm writing\footnote{Jeb not NaNo}, I immediately thought about the fact that sand is crystalline, while glass is amorphous.
While walking home, I considered a little more in depth how I could use that to a narrative.

I think that there's something to the idea of each individual grain of sand being an explicit passage of time.
Regardless of how we feel about it, there exists an objective way to measure the time displacement that occurs on earth.
What's more meaningful by far, though, is the way that we turn these objective passings into a narrative.
I'm not entirely sure how to make fire or heat into the metaphor, but maybe there's something about lightning which forms fulgerites.

Ok actually, I love the idea of fulgerites.

So, let's reconstruct the narrative from there.
A fulgerite is a place where lightning has melted and reformed the sand into an interesting shape.
Sand, as sort of given in the prompt, is the passing of time.
Lightning is a person constructing meaning out of a series of events.
The result is a memory.

This feels like something that might be better expressed as a poem, for all that I refuse to do this as anything but prose.\footnote{entirely because I want to practice prose this month, and not at all because of any legitimate reason}

Ok, let's go through the same sets of questions I asked myself last week.
My FFF question set apparently is:
\begin{itemize}
\item Why am I writing it? Today, because I want to explore this idea of time as observation.
I'm realizing that a lot of how I view myself as a more literary writer is by emphasizing the importance of observation and human naming.
That's probably worth interrogating on another occasion.\footnote{the clock is rapidly ticking down to the end of the day, or I would be more tempted to do so now}
\item Person? Right now I'm leaning towards a third/fourth person telling.\footnote{fourth in the sense of like \say{one can imagine}}
\item Any other gimmicks?
I don't think so today.
\end{itemize}
Alright, time to try writing this.
I think that I'll let it be more poetic in phrasing than I tend to emphasize, in large part because I find that I wax poetic the later at night it is.\footnote{whether that's something intrinsic in me or the fact that I have spent more than a few months writing a poem as the last thing I do before sleeping is another conversation for me to have with myself.}

Well, I wrote it, and redrafted it at least four times.
I switched between first and third person accounts of the narrative as I went through the different drafts, which I think did a lot to help me tell the story that I wanted.
In my author's note, I admitted that I don't know if what I wrote really counts as prose or poetry, but I am as happy with it as I think I will be.
I'm sure that in the morning I'll read what I wrote and feel like it was too pretentious or trying too hard to be deep.\footnote{Then again, that's almost always true for me.
There's a part of me that does really believe that to express something well is to express it ingenuinely.
Or, at least, that part believes there is no genuine way to express something complex.
I don't think that's universally true.
Certainly when I read a book of theology, for instance, the depth of thought there feels more genuine.
I think that I don't trust myself enough to let myself make that, though.
We'll see.
Maybe I'll love the post tomorrow.}

It's now well past time for sleep, so I'll do my questions and head.
Daily Reflection:
\begin{itemize}
\item Did I write 1700 words for NaNoWriMo? I did! I'm now officially a full day ahead of schedule.
\item Did I write a chapter of Jeb? I wrote my ideas for the chapter and about 500 words of content. As I mentioned at the top, today was far longer than I planned on, for all that I do not regret it.
\item Did I blog? I'm realizing that I can use footnotes for all my long diversions, which lets my actual text be\footnote{if only slightly} more coherent.
\item Did I stretch? No.
\item Am I doing better at prayer than a rushed and thoughtless rosary? No? I feel like I've thought about my place in the universe a fair amount, which involves a certain level of prayerful intention for me, but I haven't done anything explicitly so.
\item Am I doing a good job writing letters to friends? Today passed too quickly. Yesterday I mentioned making a to do list, and that did absolutely help a lot.
Tomorrow I plan to spend a fair amount of time writing with my friend, which should give me the space I need to get through the chapter of Jeb I'd like to write, the NaNo writing I need to do, and the letter that I feel like I should write.\footnote{I'll be seeing a show with friends that I think I saw in London, so I'm excited to rereview it!}
Should is probably the wrong word.
At this point, I've written all the letters to people I felt an obligation to write. What's left is the letters I want to write.
Framing it that way is probably healthier.
\end{itemize}\end{document}