\documentclass[12pt]{article}[titlepage]
\newcommand{\say}[1]{``#1''}
\newcommand{\nsay}[1]{`#1'}
\usepackage{endnotes}
\newcommand{\1}{\={a}}
\newcommand{\2}{\={e}}
\newcommand{\3}{\={\i}}
\newcommand{\4}{\=o}
\newcommand{\5}{\=u}
\newcommand{\6}{\={A}}
\newcommand{\B}{\backslash{}}
\renewcommand{\,}{\textsuperscript{,}}
\usepackage{setspace}
\usepackage{tipa}
\usepackage{hyperref}
\begin{document}
\doublespacing
\section{\href{restarting-again-again-again.html}{Restarting (Again (again again))}}
First Published: 2023 July 18


\section{Draft 1}
I logged on this morning to see what my last post was, getting ready to have not written for about two weeks.
I suppose that seventeen days is around two weeks, but it's somehow far more disheartening to see that I last blogged my monthly reflection.
Some of the time I feel less bad about missing.

From the 9-15, for instance I was at a conference.\footnote{Musing to come}
On the 16 and 17, I was in rural parts of the state doing UitP events.\footnote{Musing to come}
And before the 9-15, I was volunteering at a service trip from the 5-7.\footnote{Musing (likely) to come}

However, I still feel bad about missing the postings during all of those days.
While I have the hazy blur of great times spent at all three of those events, that doesn't have the same element of memory that remembering to log my days each day tends to bring.
I also fell off of writing basically entirely.

At the start of the month, I was nearly three chapters ahead of writing.
By the end of the first week of July, I was no chapters ahead.
Since then, each chapter has been written scant hours and minutes before it needs to be posted, and I can see a clear decrease in quality as a result.
There were no poems, no songs, and no new journal entries written during this time either.

I think that the lack of writing really is\footnote{was? since I'm actively writing self reflection right now} getting to me.
I feel far less grounded without the structures I have intentionally installed in my life, and I think that writing is legitimately one of the bedrock ones somehow.
So, I hope that as I continue this month and future months, I will remember that the future version of me is inordinately grateful for the writing I do, especially compared to how much effort it really is to write.

In terms of goals for the month:
\begin{itemize}
\item Despite the fact that I've been home for a total of seven nights this month, which includes recovery crashes\footnote{when you get home from an event a few hours before normal bed but sleep the rest of the day anyways}, I have managed to improve the cleanliness of the home.
\item Well, I completely dropped the ball on blogging, but also all writing.
\item I'll defend myself from\footnote{internal} accusations of not exercising, because I have been away from home. I did swim for the first time in ages last Monday, which was really fun.
I need to get back into it.
\item While I haven't gotten my sleep schedule back to that, partially it is again outside of my locus of control.
When I'm at a conference that has pseudo-mandatory social time until eleven at night and nothing scheduled until half eight\footnote{a British construction I refuse to lose. For the non-Brits, half past eight}, it's hard to motivate myself to wake at six. I have been intentional about sleeping enough, though.
\item I was doing great on prayer during the first week of the month\footnote{though going to a church camp makes finding time to pray easy}. Since then, it's been a bit more of a struggle, and this does remind me to be more intentional about it.
\item Though I didn't get ahead, I'm still proud of myself for keeping up with the publishing schedule throughout all of the chaos I've endured.
Now that life\footnote{relatively} calms down, I think getting five chapters ahead is reasonable. Doing some quick math\footnote{maths? fun fact: in my book I avoid using either term with hopes of hiding where I'm from. I've taken to mostly using the term calculations}, I need six new chapters to keep up with the publishing schedule.
Being five chapters ahead plus publishing six means writing eleven.
There are fourteen days\footnote{twelve if we remove Sundays, which (see goal about prayer)}, so I should absolutely be able to get ahead.
It means less than a chapter a day, and I know that there will be days that I can write two.\footnote{though today may be a zero chapter day, since I drive back from the northern part of my state just in time to start volunteering.}
\item Though poetry felt like a reasonable goal, I have fallen completely off of that train.
Now that I'm going to be sleeping in my own bed again, though, I think it can come back.
\item Of all my goals, I think this is the one I've done best on. 
I've written three letters, though haven't addressed the last two\footnote{and might have forgotten who they were meant to be addressed to}. It's a really nice morning hobby, and I think that I will absolutely be bringing it back.
\end{itemize}

Well, I suppose losing half the month would make my daily reflection more into a half-month's reflection, so it makes sense that the goals took longer than expected.

682/140
\end{document}