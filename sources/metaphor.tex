\documentclass[12pt]{article}[titlepage]
\newcommand{\say}[1]{``#1''}
\newcommand{\nsay}[1]{`#1'}
\usepackage{endnotes}
\newcommand{\1}{\={a}}
\newcommand{\2}{\={e}}
\newcommand{\3}{\={\i}}
\newcommand{\4}{\=o}
\newcommand{\5}{\=u}
\newcommand{\6}{\={A}}
\newcommand{\B}{\backslash{}}
\renewcommand{\,}{\textsuperscript{,}}
\usepackage{setspace}
\usepackage{tipa}
\usepackage{hyperref}
\begin{document}
\doublespacing
\section{\href{metaphor.html}{On Metaphor}}
First Published: 2022 August 2
\section{Draft 1}
This past Sunday I remarked,\footnote{as I probably do too much} about how the work I do is entirely within the ivory tower.
In my eyes, at least, there is almost no chance that a hungry person will be fed or a homeless person housed as a result of my research.
A new friend commented that my research should instead be thought of as evangelizing wonder.
I like that.

Earlier in the conversation, I was explaining how an FTIR\footnote{Fourier-Transform Infrared (Spectrometers)} works.
One of the other people in the conversation commented on how I did a really good job explaining hard concepts, which I thought about for the rest of the day.
I have a lot of issues with the way we teach chemistry, and I guess hearing that was the inspiration I needed to start writing about chemistry.
As I mentioned \href{reflection-july-2022.html}{yesterday}, I really want to work on that.
I figure my blog\footnote{I still don't know whether the pedant in me should really be saying 'blog, but} is as good of a place as any.

So, what does any of this have to do with the title of today's musing \say{On Metaphors}?\footnote{I know that common practice has punctuation within quotes but I hate that practice}
Something that commonly bothers me about upper level chemistry and chemists is their denigration of metaphors.
\say{Electrons dont \emph{really} orbit around nuclei,} they say, \say{so why would we teach it that way?}

Maybe it's because I'm a Catholic, so I understand how metaphors, while fundamentally untrue, still have a lot of use in understanding difficult concepts.\footnote{see: every explanation for the Trinity and its associated heresy}
Maybe it's because I'm a musician, working in a field where we have to make the fundamentally abstract\footnote{feeling and emotion} tied to completely physical and arguably meaningless values.\footnote{the exact frequency of pitches, also the relative frequencies.}
Or, maybe it's just because I like metaphors.

I think a good example of how metaphors work well, despite being wrong as you extend, can be seen in a way to teach multiplication.\footnote{also powers but we'll get there later}
Imagine trying to teach someone multiplication.
In this metaphor, to teach the multiplication of two positive integers\footnote{which is where I, at least, started}, you could say that you make a row of blocks as long as the first number and a column as long as the second.
If you fill in the rectangle and count how many blocks you used, you get the answer for their product.

That is, if you are trying to multiply 2 and 3\footnote{I still don't know if I'm going to get into what numbers really are}, you make a rectangle 2 long and 3 wide.
You would end up with 6 blocks in this method.

At this point, I can see two objections.
The first is from the anti-metaphor people \say{that's not what multiplication means}.
Trying to teach what multiplication \href{https://web.archive.org/web/20220420205104/https://www.maa.org/external_archive/devlin/devlin_01_11.html}{actually is} is far too obtuse, in my opinion at least, to teach a child to love math.

I'm not going to go too far on a tangent, but I think that a primary goal of education should be instilling a love for the subject taught.
That, of course, needs to be balanced with teaching the subject correctly, but I feel that the balance should skew more towards love of subject the younger a student is.
A PhD student in Mathematics, for instance, probably\footnote{hopefully} doesn't need to be motivated to love math.
A second grader\footnote{google says when kids learn multiplication}, on the other hand, really does, especially given the culture we have.

Anyways, the anti-metaphor people are\footnote{as strawmen must be} going to be unconvinced by any argument for metaphor, so I'm going to move on to the other objection.
\say{Is that not just what multiplication is?} might be the other side of the argument.
There are a few reasons that placing blocks into rectangles doesn't accurately describe multiplication, which I'll go into below.

First, this only works in the case of, as mentioned above, positive integers.
Sure, you can weasel your way through some fractions through a few methods.
For instance, if you're multiplying by something and a half\footntoe{e.g. 1.5,15.5, etc}, you could just place a block every other row.
I hope you see where that's inherently wrong.
Or, you could make blocks that are a fractional width of the standard box.
That immediately stops working when we get to irrational numbers.
I would love to see any person who can make a box exactly pi wide.
There's also the issue where the child in question is no longer counting blocks, but rather making unit blocks.

Another issue is that there is no way to do a negative number, let alone an irrational number.
You can't have less than zero blocks.\footnote{Economics isn't reflective of physical reality}

Finally, multiplication isn't just a number, it can also have physical meaning.
In the post linked above, he discusses how three bags of five apples per bag multiplies through to get fifteen apples.
In order to teach that problem, you rely on students abstracting out the meaning.
As I learned teaching general chemistry, students have a lot of trouble with dimensional analysis.\footnote{multiplying units to cancel and change them.}

That being said, I still think the block metaphor is a good place for students to start with multiplication.
It teaches better than repeated addition, which is something I just really don't like as a concept.
Once students understand how the blocks move, they\footnote{I think, I really have no idea how children's visualization skills are} can abstract out the blocks and multiply in their heads.\footnote{or on a piece of paper once they learn their times tables (which I do see incredible value in)}
And, it's fun.
Kids like blocks.\footnote{in my experience}

I mentioned above that I might tie to powers, and I still have steam so I'm going to.
The first few powers come kind of well with this block metaphor.\footnote{the rest do too if you really want to get students thinking in many dimensions}
Three to the first power is a line three blocks long.
Three to the second\footnote{squared one might say...} is a square three by three.
Three to the third\footnote{cubed??} is a cube of side length three.

Sure, this doesn't expand into the fourth dimension by building blocks, but I honestly see that as more of a pro than a con.
Thinking in dimensions that aren't 0,1,2, or 3 is really vital to a lot of the work that even I, a non-mathematician do.
The sooner students start seeing that as an option, the better off they'll be, in my opinion.

Anyways, as I keep writing this, I keep thinking of more benefits of blocks\footnote{learning limits, approximating areas (kind of the same thing I guess), learning that (n+1)$^2$ = (n)$^2$+2n+1}, but I'm going to stop here.
\end{document}