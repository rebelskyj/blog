\documentclass[12pt]{article}[titlepage]
\newcommand{\say}[1]{``\#1''}
\newcommand{\nsay}[1]{`\#1'}
\usepackage{endnotes}
\newcommand{\1}{\={a}}
\newcommand{\2}{\={e}}
\newcommand{\3}{\={\i}}
\newcommand{\4}{\=o}
\newcommand{\5}{\=u}
\newcommand{\6}{\={A}}
\newcommand{\B}{\backslash{}}
\renewcommand{\,}{\textsuperscript{,}}
\usepackage{setspace}
\usepackage{tipa}
\usepackage{hyperref}
\begin{document}
\doublespacing
\section{\href{notes-on-counterpoint.html}{Notes on Counterpoint}}
First Published: 2019 February 05
\section{Draft 1}
Counterpoint comes from the fact that the notation was points, which would move counter to each other.
To make a two part improvisation based on a chant, it's fairly easy.
One voice sings the chant.
The other sings a third above or below the chant, ending cadences in unison. 
\end{document}