\documentclass[12pt]{article}[titlepage]
\newcommand{\say}[1]{``#1''}
\newcommand{\nsay}[1]{`#1'}
\usepackage{endnotes}
\newcommand{\1}{\={a}}
\newcommand{\2}{\={e}}
\newcommand{\3}{\={\i}}
\newcommand{\4}{\=o}
\newcommand{\5}{\=u}
\newcommand{\6}{\={A}}
\newcommand{\B}{\backslash{}}
\renewcommand{\,}{\textsuperscript{,}}
\usepackage{setspace}
\usepackage{tipa}
\usepackage{hyperref}
\begin{document}
\doublespacing
\section{\href{history-of-keyboards.html}{History of Keyboards}}
First Published: 2018 November 20
\section{Draft 1}
This afternoon and evening, I had the wonderful opportunity to attend a series of talks, workshops, and concerts for the \say{History of Keyboards} event at Queen Mary.

It began with a pairing of a cardiologist and pianist, who are each using the other for their research.
The cardiologist is using the pianist to see if music can affect cardiac pacing.
The musician is composing pieces based off of the irregular heartbeat recordings from the cardiologist.

Later, I got to see, hear, and play with a harpsichord, clavichord, and forte piano, as well as learn how they work.
Finally, I got to see some of the cool new developments in the piano world, the coolest of which in my opinion is the \href{http://instrumentslab.org/research/mrp.html}{Magnetic Resonator Piano}.
I also got to meet the creator,\footnote{inventor? maker?} of the instrument, and he seems amazing as well.

The Magnetic Resonator is an addition that can be placed onto an existing piano.
As far as I can tell and understand, it uses electromagnets to send vibrations into the strings\footnote{because they're ferrous and magnetically moveable} at the frequency the string vibrates, as well as the hypothetical harmonics.\footnote{because piano strings have a non-zero thickness, they slowly get sharp as you play through their harmonic progression. They're also not in tune with themselves but}

It was a great time, and I'm so glad that I had the opportunity to attend.
\end{document}