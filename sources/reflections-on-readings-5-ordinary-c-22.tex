\documentclass[12pt]{article}[titlepage]
\newcommand{\say}[1]{``#1''}
\newcommand{\nsay}[1]{`#1'}
\usepackage{endnotes}
\newcommand{\1}{\={a}}
\newcommand{\2}{\={e}}
\newcommand{\3}{\={\i}}
\newcommand{\4}{\=o}
\newcommand{\5}{\=u}
\newcommand{\6}{\={A}}
\newcommand{\B}{\backslash{}}
\renewcommand{\,}{\textsuperscript{,}}
\usepackage{setspace}
\usepackage{tipa}
\usepackage{hyperref}
\begin{document}
\doublespacing
\section{\href{reflections-on-readings-5-ordinary-c-22.html}{Reflections on Today's Gospel}}
First Published: 2022 February 6

Isaiah 6:5: \say{Then I said, \say{Woe is me, I am doomed! For I am a man of unclean lips, living among a people of unclean lips and my eyes have seen the King, the LORD of hosts!}}



\section{Draft 1}
As always, my relationship with the Gospel has changed a lot in the past three years.
In the past, I focused on the fact that we should be fearful and trusting.
Now I look at the readings and see that even the saints and prophets themselves were not perfect.

Isaiah is a massive figure in the theological development of the Jewish and Christian faiths.
And yet, he says that he is \say{a man of unclean lips}.
Paul too, whose readings I used to not like a lot, points out that he once persecuted the faithful.
The future first Pope himself says that he is a sinner.\footnote{Luke 5:8 \say{When Simon Peter saw this, he fell at the knees of Jesus and said, \nsay{Depart from me, Lord, for I am a sinful man.}}}

Something about that really speaks to me right now.
The Lord doesn't only speak to or call the perfect.
Rather, it is in being called and accepting our calls that we can turn away from our sinful natures.
\end{document}