\documentclass[12pt]{article}[titlepage]
\newcommand{\say}[1]{``#1''}
\newcommand{\nsay}[1]{`#1'}
\usepackage{endnotes}
\newcommand{\1}{\={a}}
\newcommand{\2}{\={e}}
\newcommand{\3}{\={\i}}
\newcommand{\4}{\=o}
\newcommand{\5}{\=u}
\newcommand{\6}{\={A}}
\newcommand{\B}{\backslash{}}
\renewcommand{\,}{\textsuperscript{,}}
\usepackage{setspace}
\usepackage{tipa}
\usepackage{hyperref}
\begin{document}
\doublespacing
\section{\href{poetry-forms-1.html}{Writing through Poetry Forms}}
First Published: 2022 April 12


\section{Draft 1}
As I mentioned \href{reflection-march-2022.html}{yesterday}, I'm writing through a book of poetic forms.
The first of these forms listed is the acrostic.
There are apparently four kinds of acrostic:
\begin{equation}
\item normal acrostic: the first letters of each line spell out a word, with stanzas forming the spaces
\item telestich: the last letter of each line spell out a word, otherwise the same
\item double acrostic: acrostic+telestich, but using the same words, I assume, because
\item coumpound acrostic: spells a different word in the first and last letters
\end{equation}
According to the book, acrostics tend to have meter and rhyme, but it specifies nothing further, which kind of makes sense.
Any other restrictions added probably just make it an acrostic and a rhyme or what not.

Yesterday I wrote a standard acrostic, which went easily enough.
Since my goal is proficiency\footnote{if that's a fair or valid term} in the different forms, I think the single one was enough.
Today is going to be a telestich, and hopefully that goes well as well.
\end{document}