\documentclass[12pt]{article}[titlepage]
\newcommand{\say}[1]{``#1''}
\newcommand{\nsay}[1]{`#1'}
\usepackage{endnotes}
\newcommand{\1}{\={a}}
\newcommand{\2}{\={e}}
\newcommand{\3}{\={\i}}
\newcommand{\4}{\=o}
\newcommand{\5}{\=u}
\newcommand{\6}{\={A}}
\newcommand{\B}{\backslash{}}
\renewcommand{\,}{\textsuperscript{,}}
\usepackage{setspace}
\usepackage{tipa}
\usepackage{hyperref}
\begin{document}
\doublespacing
\section{\href{reflections-on-readings-30-ordinary-b.html}{Reflections on Today's Gospel}}
First Published: 2018 October 28

Mark 10:47 \say{On hearing that it was Jesus of Nazareth, he began to cry out and say, \nsay{Jesus, son of David, have pity on me.}}
\section{Draft 1}
Today�s gospel shows us that even when the world is against us, Christ is for us. Bartimeaus\footnote{which the reading informed me means son of timeaus} calls to the Lord.
People around him mock him, and yet he persists.
But when the Lord answers him, he doesn�t ask for all his problems to be solved. 
Rather, he asks for divine intervention only in the part of his life which needs it, his loss of sight.

To me, that�s the other part of the message of today�s gospel.
We are told that we can ask God for help, and that He will answer, even if those around scorn us.
But, we should ask him only for what we cannot ourselves, or cannot via mortal means, accomplish.

\end{document}