\documentclass[a4paper]{article}
\newcommand{\say}[1]{``#1''}
\newcommand{\nsay}[1]{`#1'}
\usepackage{endnotes}
\newcommand{\1}{\={a}}
\newcommand{\2}{\={e}}
\newcommand{\3}{\={\i}}
\newcommand{\4}{\=o}
\newcommand{\5}{\=u}
\newcommand{\6}{\={A}}
\newcommand{\B}{\backslash{}}
\newcommand{\<}{\textless}
\renewcommand{\>}{\textgreater}
\renewcommand{\,}{\textsuperscript{,}}
\usepackage{setspace}
\usepackage{hyperref}
\begin{document}
\doublespacing
\section{\href{footnote-frenzy.html}{Footnote Frenzy}}
Prereading note: while writing a different post,\footnote{which will be posted (and rewritten) next time that the situation is valid} I ended up needing to deeply nest some statements.
I realized\footnote{about ten minutes in} that I had spent around equal time setting up the prereading note to actually writing the post, so I decided to just turn the note into its own post.
As a result, that's Draft 0.
Also, since I needed to make sure my\footnote{nested} footnotes parsed, I relabeled Draft 2 as 2/3, since it's unclear which is the correct term.\footnote{and I already have far too many footnotes}
Draft 4 remains as such.
\section{Draft 4}
Like many bad authors,\footnote{and hopefully some competent (since that's a bar I'm not sure I would consider this post at) authors (\href{https://archangelink.com/writer-versus-author/}{writers?)}} I rely a lot on gimmicks.\footnote{or at least one gimmick}
Also like bad authors,\footnote{but great artists (according to \href{https://quoteinvestigator.com/2013/03/06/artists-steal/}{someone})} I blatantly stole my gimmick from someone else.\footnote{as the title suggests, my father's \href{http://www.cs.grinnell.edu/~rebelsky/musings/}{\say{Daily Musings}}}
If it isn't clear from the eight\footnote{nine including this (assuming no more drafts)} footnotes I've already used, my gimmick is footnotes, and nested ones in particular.
As I mentioned in \href{https://j.rebelsky.com/arranging-for-bagpipe}{a previous post},\footnote{no, I have no internal consistency for which words are hyperlinked. In all honesty, it's what feels right as I type the command} I can't have nested footnotes.\footnote{i.e. a footnote that has a footnote as its referent (the thing that sends you to the note [I think?]) or its reference (the thing you get sent to [or switch this explanation with the one above, if needed])}
Instead, I\footnote{as mentioned in the linked post} use nested sets of punctuation.\footnote{which before today was limited to ([])}
So, when I had a chance to expand the list of nested parentheticals\footnote{which isn't really the right term, because I use more than parentheses} I use, I was happy.\footnote{yes, the nesting of strings (references? I'm not really sure what the right word is here) is actually something I feel joy about}
So, my list of nesting symbols\footnote{since I find a string of parentheses in in succession  hard to read (like the example here(see (if not do you see yet?) how hard it gets?)(hopefully) demonstrates), but different shapes in succession (like this [or this]) easier (still not always easy though) to read} now goes: footnotes,\footnote{Like this! (ooh meta)} parentheses,\footnote{seen in the footnote above's \say{ooh meta,} or in most of the prior (or the following [with some exceptions]) footnotes} square brackets,\footnote{I think they're called square brackets (although they aren't square [unless by square we mean \href{https://www.merriam-webster.com/dictionary/square}{Merriam Webster's} first definition \<which, oddly, refers to the tool, not the shape\>])} then angle brackets.\footnote{which makes no sense as a name (since all brackets have angles [other than parentheses I guess \<although it could be argued that they just have a lot of angles -but that feels like needless pedantry --although I guess all \href{https://en.wiktionary.org/wiki/pedantry}{pedantry} is supposed to be needless **because of the word \say{excessive}**-- that I don't know enough math for-\>, but they're not too important \<unless you actually follow the convention of parentheticals -but not the convention of avoiding their usage-\>]. \href{https://en.wikipedia.org/wiki/Bracket}{Wikipedia} calls them \say{pointy brackets}[which is kind of funny], so maybe I should too) in my opinion}
Unfortunately, after that, there are no more brackets\footnote{that I know (or at least strongly believe) are supported on the platforms I write and publish my work (if you can call it that)} that I can find,\footnote{maybe there's a reason for that} so\footnote{as you might have seen} I used short and long dashes,\footnote{I know one of them is an \say{em dash,} but I'm not sure which} then two asterisks\footnote{astereces? Given that it comes from Latin asteriscus, maybe not. CS people \href{https://en.wikipedia.org/wiki/Asterisk}{allegedly} call them stars, which is much easier} when I needed\footnote{read: wanted} to go one layer deeper in my nesting.\footnote{if I were a bird, I would be so warm}
I don't\footnote{didn't, and likely will not} like the way that they look,\footnote{mostly because I feel like two \href{https://en.wiktionary.org/wiki/asterisk}{asterisks} feel less like a divider and more like two arbitrary characters} so I hope I don't need to nest my footnotes more than five\footnote{not including the footnote itself} layers deep.\footnote{wow five feels so much more freeing than three}
And, as I read through this draft,\footnote{wow this essay is getting so \href{https://xkcd.com/917/}{meta}} I did find that the different punctuation helped me to parse the statements slightly more easily.
However, long and short dashes don't quite look different enough for me to parse at first glance, so it's a good thing I\footnote{hopefully} won't need to use them often.
Anyways, the 86 footnotes\footnote{that number was changed at the very end of the (writing of the) piece to reflect reality, and does not include nestings} of the piece contain a total\footnote{as above} of 1415 words within its footnotes.
That's nearly 70\% of the entirety of the words written.\footnote{ibid}
Whoops.
\section{Draft 2/3}
Like many bad authors,\footnote{and hopefully some competent (since that's a bar I'm not sure I would consider this post at) authors (\href{https://archangelink.com/writer-versus-author/}{writers?)}} I rely a lot on gimmicks.\footnote{or at least one gimmick}
Also like bad authors,\footnote{but great artists (according to \href{https://quoteinvestigator.com/2013/03/06/artists-steal/}{someone})} I blatantly stole my gimmick from someone else.\footnote{as the title suggests, my father's \href{http://www.cs.grinnell.edu/~rebelsky/musings/}{\say{Daily Musings}}}
If it isn't clear from the six\footnote{seven including this (assuming no more drafts[which was wrong])} footnotes I've already used, my gimmick is footnotes, and nested ones in particular.
As I mentioned in \href{https://j.rebelsky.com/arranging-for-bagpipe}{a previous post},\footnote{no, I have no internal consistency for which words are hyperlinked. In all honesty, it's what feels right as I type the command} I can't have nested footnotes.\footnote{i.e. a footnote that has a footnote as its referent (the thing that sends you to the note [I think?]) or its reference (the thing you get sent to [or switch this explanation with the one above, if needed])}
Instead, I\footnote{as mentioned in the linked post} use nested sets of punctuation.\footnote{which before today was limited to ([])}
So, when I had a chance to expand the list of nested parentheticals\footnote{which isn't really the right term, because I use more than parentheses} I use, I was happy.\footnote{yes, the nesting of strings (references? I'm not really sure what the right word is here) is actually something I feel joy about}
So, my list of nesting symbols\footnote{since I find a string of parentheses in in succession  hard to read (like the example here(see (if not do you see yet?) how hard it gets?)(hopefully) demonstrates), but different shapes in succession (like this [or this]) easier (still not always easy though) to read} now goes: footnotes,\footnote{Like this! (ooh meta)} parentheses,\footnote{seen in the footnote above's \say{ooh meta,} or in most of the prior (or the following [with some exceptions]) footnotes} square brackets,\footnote{I think they're called square brackets (although they aren't square [unless by square we mean \href{https://www.merriam-webster.com/dictionary/square}{Merriam Webster's} first definition \<which, oddly, refers to the tool, not the shape\>])} then angle brackets.\footnote{which makes no sense as a name (since all brackets have angles [other than parentheses I guess \<although it could be argued that they just have a lot of angles **but that feels like needless pedantry**\>, but they're not too important \<unless you actually follow the convention of parentheticals **but not the convention of avoiding their usage**\>]. \href{https://en.wikipedia.org/wiki/Bracket}{Wikipedia} calls them \say{pointy brackets}[which is kind of funny], so maybe I should too) in my opinion}
Unfortunately, after that, there are no more brackets\footnote{that I know (or at least strongly believe) are supported on the platforms I write and publish my work (if you can call it that)} that I can find,\footnote{maybe there's a reason for that} so\footnote{as you might have seen} I used two asterisks\footnote{astereces? Given that it comes from Latin asteriscus, maybe not. CS people \href{https://en.wikipedia.org/wiki/Asterisk}{allegedly} call them stars, which is much easier} when I needed\footnote{read: wanted} to go one layer deeper in my nesting.\footnote{if I were a bird, I would be so warm}\,\footnote{and no, I will not use em dashes, since I still don't know whether ems are the long or short dash (- or --), or how long and short dashes differ. If I ever learn, I may incorporate them (whoops, the draft above disproves this)}
I don't\footnote{didn't, and likely will not} like the way that they look,\footnote{mostly because I feel like two \href{https://en.wiktionary.org/wiki/asterisk}{asterisks} feel less like a divider and more like two arbitrary characters} so I hope I don't need to nest my footnotes more than three\footnote{not including the footnote itself} layers deep.\footnote{or I can learn to use dashes and em dashes (ooh I could use both of those to get two more layers free [shoot I'm writing another draft])}
And, as I read through this draft,\footnote{wow this essay is getting so \href{https://xkcd.com/917/}{meta}} I did find that the different punctuation helped me to parse the statements slightly more easily.
\section{Draft 1}
Like many bad authors,\footnote{and hopefully some good ones} I rely a lot on gimmicks.
Also like bad authors,\footnote{but great artists (according to \href{https://quoteinvestigator.com/2013/03/06/artists-steal/}{someone})} I copy my gimmick from someone else.\footnote{as the title suggests, my father's \say{Daily Musings}}
So, when I had a chance to expand the list of nested parentheticals\footnote{which isn't really the right term, because I use more than parentheses} I need, I was happy.\footnote{yes, that is actually something I feel joy about}
So, my list of nesting\footnote{since I find a string of parentheses in order hard to read, but different shapes (like this [or this]) easier to read} now goes: footnotes,\footnote{Like this! (ooh meta)} parentheses,\footnote{seen in the footnote above \say{ooh meta,} or in most of the prior footnotes (or the following [with some exceptions])} square brackets,\footnote{I think they're called square brackets (although they aren't square [unless by square we mean \href{https://www.merriam-webster.com/dictionary/square}{Merriam Webster's} first definition \<which, oddly, refers to the tool, not the shape\>])} then angle brackets.\footnote{which makes no sense as a name (since all brackets have angles [other than parentheses I guess \<although it could be argued that they just have a lot of angles **but that feels like needless pedantry**\> but they're not too important]. \href{https://en.wikipedia.org/wiki/Bracket}{Wikipedia} calls them \say{pointy brackets}[which is kind of funny] so maybe I should too) in my opinion}
Unfortunately, after that, there are no more brackets\footnote{that I know (or at least strongly believe) are supported on the platforms I write and publish my work (if you can call it that)} that I can find,\footnote{maybe there's a reason for that} so\footnote{as you might have seen} I switched to using two asterisks.\footnote{astereces? Given that it comes from Latin asteriscus, maybe not. CS people \href{https://en.wikipedia.org/wiki/Asterisk}{allegedly} call them stars, which is much easier}
I don't really like the way that they look,\footnote{mostly because it feels less like a divider, and more of just two random characters} so I hope I don't need to nest my footnotes more than three\footnote{not including the footnote itself} deep.
\section{Draft 0}
Prereading note: Yay! I finally used more nesting.\footnote{yes, that is actually something I feel joy about}
It now goes: Footnote,\footnote{like this! (ooh meta)} parentheses,\footnote{like the footnote above's line \say{ooh meta,} (or like this [or any of the following explanatory footnotes])} square brackets,\footnote{I think they're called square brackets (although, they aren't square [unless by square we mean \href{https://www.merriam-webster.com/dictionary/square}{Merriam Webster's} first definition \<which, oddly, refers to the tool, not the shape\>])} then angle brackets.\footnote{which makes no sense as a name (since all brackets have angles [other than parentheses I guess \<although it could be argued that they just have a lot of angles\> but they're not too important]. \href{https://en.wikipedia.org/wiki/Bracket}{Wikipedia} calls them \say{pointy brackets}[which is kind of funny] so maybe I should too)}
Unfortunately, after that, there are no more brackets that I can find.\footnote{maybe there's a reason for that}
\end{document}
