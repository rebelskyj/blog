\documentclass[12pt]{article}[titlepage]
\newcommand{\say}[1]{``#1''}
\newcommand{\nsay}[1]{`#1'}
\usepackage{endnotes}
\newcommand{\1}{\={a}}
\newcommand{\2}{\={e}}
\newcommand{\3}{\={\i}}
\newcommand{\4}{\=o}
\newcommand{\5}{\=u}
\newcommand{\6}{\={A}}
\newcommand{\B}{\backslash{}}
\renewcommand{\,}{\textsuperscript{,}}
\usepackage{setspace}
\usepackage{tipa}
\usepackage{hyperref}
\begin{document}
\doublespacing
\section{\href{first-day-of-class-S2022.html}{First Day of Class}}
First Published: 2022 January 25

\section{Draft 1}
Today is the first day of my new Spring semester.
It was a fun day!

Out of the blue, I happened to see someone I care deeply about.
I got to write a letter to that same person, which was claimed to be appreciated.
I got a letter in turn from the exact same person!\footnote{Exciting!!}
In my choir class I got to see a bunch of friends from last semester.
In the same choir class, I got to see\footnote{the first set of} the new music that I'm going to be singing this semester, and found out that we're performing at an early music festival\footnote{do I need to capitalize that?}, which is incredibly fun and exciting.
My ISM class\footnote{Interstellar Medium, not a class on the concept of ism's} met, and I'm learning interesting things in it/sure I'll learn many more.

On the research side, a piece I needed machined in the shop was finished, so I got to spend some time playing with electronic-adjacent pieces\footnote{read: I drilled some holes and screwed some screws}.
I got to see my group and have long and productive conversations with them all about research.

In productivity, I got through more of my items today than I did yesterday, which is incredible and a little bit disappointing in how I did yesterday.
I started a massive revision of a small choral piece I'm working on, which\footnote{since I'm running short on words} I will write about below.

The piece is designed to be performed during the Ash Wednesday ashing\footnote{wrong word, but when the ashes are put on peoples' foreheads}.
As a result, I want to have a piece which can be performed for some arbitrary amount of time, so that everyone who wants ashes can do so to the music, without it necessarily needing to repeat in a perfect loop.
The easiest answer to that problem is to simply have aleatoric music, but simple answers aren't always good answers.
The biggest problem with aleatoric music is that I am almost positive the performers would hate it.

So, we then move to aleatoric music lite.
My goal is a four part melody that can be sung as a round or as four part harmony, where each of the four voices can take each of the four parts.
I am not going to be so extra so as to plan for the whole 4 factorial\footnote{12} combinations possible, especially because most of them should be the same-ish sounding.
There are some problems with that, however.
The biggest problem is that there is a very small range that\footnote{octave transposed as needed} can be sung by all four voice parts.
After that, figuring out the rules for inverting counterpoint are relatively easier.\footnote{I hope}
In the choir I'm writing for, the range I think that I can have all four parts sing is G-G\footnote{4-5 for sopranos, 3-4 for altos, and 2-3 for baritone/tenor and bass}.
Since it's for Ash Wednesday, I want a sad tune, and since it's liturgical adjacent I would like something modal.
As a result, the piece is in Phrygian.\footnote{E for white note scales and mi for solfeggio scales}

Without this footer I'm at 461 no footnotes and 77 footnote words, which meets my goal. Maybe tomorrow I should keep writing about this, since it seems to be easy to come out
\end{document}