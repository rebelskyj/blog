\documentclass[12pt]{article}[titlepage]
\newcommand{\say}[1]{``#1''}
\newcommand{\nsay}[1]{`#1'}
\usepackage{endnotes}
\newcommand{\1}{\={a}}
\newcommand{\2}{\={e}}
\newcommand{\3}{\={\i}}
\newcommand{\4}{\=o}
\newcommand{\5}{\=u}
\newcommand{\6}{\={A}}
\newcommand{\B}{\backslash{}}
\renewcommand{\,}{\textsuperscript{,}}
\usepackage{setspace}
\usepackage{tipa}
\usepackage{hyperref}
\begin{document}
\doublespacing
\section{\href{reading-3.html}{On Reading Continued}}
First Published: 2022 July 6

\section{Draft 1}
Yesterday the newest book in a series I\footnote{most of my family, and some of my friends} really enjoy was released.
It's book eleven in the series, and so I prepared in the way I tend to for new additions to a series: I reread the prior books.
I found that I again completely misestimated the amount of time it would take me.

It feels like every time I reread a series to prepare for a release I panic at first.
\say{Oh no,} I think, \say{I've started too late. I'm not going to make it to the new book in time.}
Then, inevitably, I get to the last book and have to stretch it to the release of the new one.

This pattern was repeated, but it was really fun.
I got to see plot threads that I'd missed on prior read-throughs that were picked up and woven in to the latest two books, and I just enjoy the series generally.
I tore through the newest book, but I'm listening through it now because I really want to make sure I got all of the content.
\end{document}