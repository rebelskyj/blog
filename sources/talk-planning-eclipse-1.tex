\documentclass[12pt]{article}[titlepage]
\newcommand{\say}[1]{``#1''}
\newcommand{\nsay}[1]{`#1'}
\usepackage{endnotes}
\newcommand{\1}{\={a}}
\newcommand{\2}{\={e}}
\newcommand{\3}{\={\i}}
\newcommand{\4}{\=o}
\newcommand{\5}{\=u}
\newcommand{\6}{\={A}}
\newcommand{\B}{\backslash{}}
\renewcommand{\,}{\textsuperscript{,}}
\usepackage{setspace}
\usepackage{tipa}
\usepackage{hyperref}
\begin{document}
\doublespacing
\section{\href{talk-planning-eclipse-1.html}{Talk Planning (Eclipse)}}
First Published: 2023 September 12
\section{Draft 1}
First, sorry about the sudden radio silence.
If I had to blame anything, I would blame the combination of:
\begin{itemize}
\item Getting suddenly busy\footnote{I went home for jury duty, served (technically), drove back to school, went to my first day of choir for the year, had first group meeting of the year, went to an event with friends, went to a baby shower, went to a euchre tournament, went to intro catechesis, and went to a thank you dinner to donors (I was part of the thanking not the giving). On the bright side, I now have a lot of content I can blog about for the next while, assuming that I find the energy}
\item \href{spoons-2}{getting out of mental space}.\footnote{running? I feel like I only ever think of running out of, not getting out of, but I suppose that either works in this case? maybe}
\end{itemize}

Without further ado, though, I do have another talk in\footnote{oh gosh that's soon} just over three weeks.
With twenty something days to go, it feels like an appropriate time to shift my planning from completely analog to something with a bit more structure.
I'll be giving a talk on the eclipses that are coming up.
I think I've mentioned on here before\footnote{maybe}, but there are four upcoming eclipses on the continental US.\footnote{I guess this is technically doxxing myself, but I don't think anyone will be surprised to learn that's where I am, especially since I've made comments about living here before, I'm almost positive.}
In order, these are a partial solar eclipse on October 14, a partial lunar eclipse on October 28, a penumbral lunar eclipse on March 25, and a total solar eclipse on April 8 (the first since 2017, and the last until\footnote{I think} 2044).\footnote{why yes, this last sentence was directly copied from one of the abstracts I submitted.
How could you tell?}

There are a lot of ways that I could frame an eclipse talk.
Having now talked to a number of people who have materials prepared, I realize that I don't love most of the ways that they do.
Being, as I am\footnote{by need, growth, and choice}, myself, the way that I'm choosing to frame the talk is by answering the following questions\footnote{probably in this order, but I haven't decided if that's true for certain.
Thankfully, in order of how much time I feel the need to spend, making figures is an order of magnitude more than making slides is an order of magnitude more than editing slides is an order of magnitude more than preparation.
Since I have no intention of spending the rest of my waking hours on preparing figures, I don't really need to worry about anything else}
\begin{itemize}
\item What is an eclipse?\footnote{that's something I feel like I should be able to readily define. Right now I just have \say{when heavenly bodies get in each other's way}
\item How many kinds of eclipse are there?\footnote{ok so there's solar and lunar, which is two. There's partial and total, which applies to both, so four. But then there's also the fact that the interaction of the sun's light and the earth and moon results in the earth casting not just an umbral shadow but a penumbral shadow. I don't know if a partial penumbral eclipse is a meaningful kind, but I still see at least five kinds of eclipse when I quickly google (partial and total solar, partial, total, and penumbral lunar.}
\item Why do eclipses happen?\footnote{lunar eclipses: Earth gets in the way of the sun's light to the moon. Solar: oh gosh, so many lucky pieces of orbital dynamics that resulted in the moon being the same orbital size as the sun! wild.}
\item Why don't eclipses happen?\footnote{ok so that's not really the question, but it like why don't we have them constantly? Especially once you learn that solar eclipses can only happen during new moons and how new moons form, it is more than a little strange to think about.}
\item How do we know when eclipses will happen?\footnote{because we, like ancient mesopetamia (babylonia? I forget and should absolutely learn by then) are able to track patterns}
\end{itemize}

So, really, only one of those questions actually requires any effort from me.
My \say{What is an eclipse} slide\footnote{section?} will just be a picture of a pretty eclipse with the text definition of one.
My \say{How many kinds} slides or section will be a picture of each kind with a description of how they work.\footnote{oof that means I'll have to explain the penumbra/umbra difference in earth shadow. That's probably fine though? I mean it's a pretty easy thing to demonstrate/explain I hope}
My \say{Why do they happen} slide will just be \say{wow we're truly blessed.}
My \say{How do we know} is also pretty easy.
Like, the answer is really just pattern matching,\footnote{had to delete the next line, which did absolutely read \say{which will be a fun diversion into harmonics. It's true, and I do still actively believe in the harmony of the spheres and think more people need to,} but this is not the place (something I'll need to repeat to myself over and over as I prepare these slides)} and the hardest part will be not going off on a rant about harmonics (see footnote).

Now, of course, you'll notice that I did not talk about \say{Why don't they happen?}.
And that's because the answer is really hard.

Ok, so, without graphics.
The sun exists.
That feels like a good place to start.
Since we live in a vaguely heliocentric world\footnote{I refuse to get into the fact that the sun is moving, because I don't want it to/don't think that it should affect eclipses}, everything we care about exists in reference to the sun.

Now, we also know that there is an earth.
The Earth, in fact.\footnote{hmmm capitalization is going to be an issue for anything written. I'm so grateful that my talks are almost exclusively oral and that I don't live in a fantasy world where capital and lower case letters are actually audible (also I hate remembering the fact that lower case letters are so named because in early typesetting they existed in the physically lower cases.)}
The Earth moves around the sun.

Now, there are some fun things about that.
Imagine if the Earth didn't rotate at all.\footnote{from an observer staring down on the solar system}
We would have one day every year.
That would be really bad, I think.\footnote{I should probably find an actual citation for that. My imagining that each part of the Earth would boil then freeze may not be entirely accurate.}
Even worse, though, the Earth could be tidally locked, which would mean that, at least in respect to the sun, the earth never rotated.
Ope, actually, since we're saying that the sun is the point of reference, we can thankfully ignore the Earth being stationary from an above inertial reference frame.\footnote{a phrase I did, in fact, say to a four year old}

Ok so we have the sun.
The sun is orbited by the Earth.
If the Earth was tidally locked\footnote{spoiler, like the moon to the Earth} we would have half the earth bathed in flame and half bathed in ice.\footnote{ok realistically not really, but it's at least poetically evocative, which is my goal for right now. Facts can always follow from a story, the reverse is not always true.}
Instead, we are lucky that the Earth rotates about four hundred times per rotation around the sun.\footnote{the fact that it isn't totally in line with the orbit around the sun used to mystify me. Now the fact that it lines up so close to nicely and so nicely with the moon shocks me more}

Now, if the Earth rotated around the sun without a tilt, there would be no seasons.\footnote{Strictly speaking, the earth's orbit is not a circle, and so maybe we still would. Then again, the Earth is closest to the sun during northern winter, and I don't think that the equatorial regions really have any temperature swings, though, again, that's something I should consider. Ughhhhhh I don't want to get into the fact that planets have elliptical orbits, but that's at least something that I should consider for the future video series (which I'll talk about at the end of this post, because it's a great idea that a friend had, especially since more and more it looks like that's where I want my job to end up}
Of course, as we know, we have seasons, which means that we orbit at a tilt.
That does then raise the question of what a day is.
After all, if you're on either of the poles, there would be entire calendar days with no apparent change in the location of the sun in the sky.\footnote{at least during the winter months.
Less sure what the sun does in the sky over summer}
Also, anywhere you are, a fun thing to do is to find that the sun sets later on the days following the longest day of the year than it did on that day.
That's because the day is more than 24 hours during that point.
Oh cool, the length of a day can vary by as much as 8 seconds during the course of the earth's orbit.
Hmm, that doesn't explain why sun sets later.

Oh gosh, it has to do with incident tilt of the earth and therefore apparent amounts of sunlight, along with the fact that the amount of sunshine we get is shifted slightly later, I guess.
That is way more than I ever wanted to know and learn, at least for this talk, so I guess I have to leave out the fun fact about how the sun sets later.

Diversions aside, we now have the sun and the earth, which rotates at its tilt across the sun.
It's at this point I want to use blender to animate, because I do not want to hand animate this, and also my ability to animate in python is very limited.\footnote{shockingly, the programming language is not the best tool for making pictures}
But, more importantly, we then add in the moon.

So, first of all, if the moon was the wrong angular size in the sky, we would have much different numbers of eclipses.
Shoot, I'll have to discuss angular size here.
Actually, is that the first place to go?
I think that the first place I should do it is by going \say{ok so the moon is actually bound to the earth, rather than the sun, which feels weird.}

The moon is tidally locked to the earth, so we only see one face.
If you think of it like a mirror, we can see a full moon when it's opposite the earth from the sun, and we see a new moon when the moon is between the earth and the sun.
This is where the question of why not all eclipses becomes relevant.

So first, we have size on sky.
If moon big, more solar eclipses.
If moon very big, no chance for total lunar.

But, of course, assumption will start with everything in the same plane, because of course it will.
Instead, moon orbits earth at an angle.
Because of that, we can only have the intersection at certain times of year.
This is really what I want the animation for, because the two dimensional static drawings aren't totally helpful.

Anyways, a friend also said that if I'm interested in public outreach generally, I should start a youtube channel and post my talks there.
It's not a bad idea, for all that I know it would make me incredibly neurotic.

\begin{itemize}
\item This is progress!
\item Kind of fought entropy. I meal prepped this past Sunday, which also did a lot to give me food for the week.
\item Terrible at blogging. Whoopse
\item I've been stretching off and on as parts of me hurt.
\item I have been as active as someone rushing from point a to b thoughtlessly is.
\item I've been trying to prioritize sleep, though I know that I did not on Friday or Saturday.
\item I have been needing my alarms to wake up every morning.
\item I have not prayed really at all.
\item Book chapters are coming out at the last possible moment.
\item I mean the net progress is still there I guess.
\item Poetry doesn't happen.
\item I played around with the song a little bit, but nothing explicitly for the album.
\item I did not do the things I like, excite, or grateful since the last blog post. Tomorrow morning, I guess.
\item It's been a struggle to cultivate being, let alone being joyful.
\end{itemize}

\end{document}