\documentclass[12pt]{article}[titlepage]
\newcommand{\say}[1]{``#1''}
\newcommand{\nsay}[1]{`#1'}
\usepackage{endnotes}
\newcommand{\1}{\={a}}
\newcommand{\2}{\={e}}
\newcommand{\3}{\={\i}}
\newcommand{\4}{\=o}
\newcommand{\5}{\=u}
\newcommand{\6}{\={A}}
\newcommand{\B}{\backslash{}}
\renewcommand{\,}{\textsuperscript{,}}
\usepackage{setspace}
\usepackage{tipa}
\usepackage{hyperref}
\begin{document}
\doublespacing
\section{\href{twelfth-night.html}{Twelfth Night Review}}
First Posted: 2018 October 15
\section{Draft 2}
Today, I had the incredible opportunity to see Kwame Kwei-Armah and Shaina Taub's adaptation of \textit{Twelfth Night} at the Young Vic Theatre.
It compresses the entire show into 90 minutes, including all of the\footnote{many} musical interludes
From the opening until the curtain call, I was almost overwhelmed with joy.

Now, onto a description of the show itself.
The show opens with an old man handing out jerk chicken.\footnote{which was delicious}
At least I hope that was part of the show, and it wasn't just a random stranger giving me food.\footnote{which, now that I think about it, wouldn't have affected me taking it}
The set is a beautiful one point perspective feeling city street.
In the middle of the thrust, a van is parked.
The show begins with the van slowly moving upstage while mourners come out and soulful saxophone plays.
However, that is the first and last time the show is anything less than overwhelmingly energized.

The biggest uniting thread in the show, other than the comic ridiculousness, is the women's chorus.
After every major plot point, they come out, singing \say{What's the word on the street?} before explaining what just happened, in case we had somehow missed it.

Then, when Malvolio sings his song about becoming Count, it felt like the quintessential musical number, but something I couldn't name was missing.
All of a sudden, he does a magic trick and has a cane in his hand, and the chorus comes out in top hats.
The tap dancing interlude that follows was exactly as fun and cheesy as it\footnote{hopefully} sounds.
That was a hallmark of the show, honestly: the cheesy, over the top fun.

Another musical highlight for me was the scene where Viola\footnote{spoiler?} and Andrew are preparing for the fight.
I had equal parts Rocky and Meatloaf coursing through my head in the song, where the cast starts air punching and jump-roping to fast paced electrical guitar and strong drums.

I left the show the happiest I've been after a theatre production in a while.
And really, what more can I say?
If you get a chance, I would highly recommend watching it

\section{Draft 1}
Today, I had the incredible opportunity to see Kwame Kwei-Armah and Shaina Taub's adaptation of \textit{Twelfth Night} at the Young Vic Theatre.
It, like all good adaptations, takes the best parts of the original show and leaves out the bad.
From the opening until the curtain call, I was star struck for the entire show.

Now, for those of you who don't know, I sometimes just have a great time.
As the diving coach \href{diving.html}{mentioned}, I have an expression of joy when I'm doing things I enjoy, there diving.
This weekend, I had the chance to climb around Arthur's Seat, and had a great goofy grin on.
And, tonight, my cheeks were almost sore from how much I was smiling.

Now, onto a description of the show itself.
The show opens with an old man handing out jerk chicken.\footnote{which was delicious}
At least I hope that was part of the show, and it wasn't just a random stranger giving me food.\footnote{which, now that I think about it, wouldn't have affected me taking it}
The set is a beautiful one point perspective feeling city street.
In the middle of the thrust, a van is parked.
The show begins with the van slowly moving upstage while mourners come out and soulful saxophone plays.
However, that is the first and last time the show felt anything but energized to me.

When Malvolio sings his song about becoming Count, it felt like the quintessential musical number, but something I couldn't name was missing.
All of a sudden, he does a magic trick and has a cane in his hand, and the chorus comes out in top hats.
The tap dancing interlude that follows was exactly as fun and cheesy as it\footnote{hopefully} sounds.
That was a hallmark of the show, honestly.
It was a cheery, fun, lively show.

Speaking of the chorus, they were another highlight of the show.
After every major plot point, a women's chorus came out, singing \say{What's the word on the street?} before explaining what just happened, in case we had missed it.

But, throughout the show, every actor was as over the top as they could possibly be without making it into a farce.

Another musical highlight for me was the scene where Viola\footnote{spoiler?} and Andrew are preparing for the fight.
I had equal parts Rocky and Meatloaf coursing through my head in the song, where the cast starts air punching and jump-roping to fast paced electrical guitar and strong drums.

I left the show the happiest I've been after a theatre production in a while.
And really, what more can I say?
\end{document}