\documentclass[12pt]{article}[titlepage]
\newcommand{\say}[1]{``#1''}
\newcommand{\nsay}[1]{`#1'}
\usepackage{endnotes}
\newcommand{\1}{\={a}}
\newcommand{\2}{\={e}}
\newcommand{\3}{\={\i}}
\newcommand{\4}{\=o}
\newcommand{\5}{\=u}
\newcommand{\6}{\={A}}
\newcommand{\B}{\backslash{}}
\renewcommand{\,}{\textsuperscript{,}}
\usepackage{setspace}
\usepackage{tipa}
\usepackage{hyperref}
\begin{document}
\doublespacing
\section{\href{reflections-on-readings-2-advent-c.html}{Reflections on Today's Gospel}}
First Published: 2018 December 9

Luke 3:4 \say{He went throughout (the) whole region of the Jordan, proclaiming a baptism of repentance for the forgiveness of sins.}

\section{Draft 1}
Today marks the second week of Year C in the Catholic Church.
Year C is nice because it focuses on Luke's Gospel.
For those of you who don't know, each of the Gospels is written for a particular group.
There's a cute mnemonic I use to remember it.

Matthew is for the Jews,\footnote{yay rhyme!}
John focused on no one,\footnote{yes it's forced}
Mark and Luke both exist, and Greeks and Romans both existed.
That's where it breaks down a little, and I have to remember that Luke is happy, and Mark is power-focused, and then I remember again.

The focus of each of the Gospels is on a different aspect of Jesus' life.
Matthew is nice because he focuses on the dogmatic aspects and how the Lord fulfills the prophecies in the Old Testament, John is nice because he's the least focused, Mark is nice because you see the miracles Jesus performs/
But Luke is my favorite, because he focuses on truth.

Anyways, today's line was picked because, at the end of the day, that's the job we're all called to do.
We are called to go and forgive, and let those who don't know the Word of God hear it.
We are called to be a force of repentance, helping \say{all flesh ... see the salvation of God.}\footnote{Luke 3.6}
\end{document}