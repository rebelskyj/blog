\documentclass[12pt]{article}[titlepage]
\newcommand{\say}[1]{``#1''}
\newcommand{\nsay}[1]{`#1'}
\usepackage{endnotes}
\newcommand{\1}{\={a}}
\newcommand{\2}{\={e}}
\newcommand{\3}{\={\i}}
\newcommand{\4}{\=o}
\newcommand{\5}{\=u}
\newcommand{\6}{\={A}}
\newcommand{\B}{\backslash{}}
\renewcommand{\,}{\textsuperscript{,}}
\usepackage{setspace}
\usepackage{tipa}
\usepackage{hyperref}
\begin{document}
\doublespacing
\section{\href{christmas-2022.html}{Christmas 2022}}
First Published: 2022 December 25
\section{Draft 1}
I don't know why I was surprised that the only other Christmas post I made was in 2018, but I was.
Anyways, since it's Sunday, I'm going to reflect at least a little on the readings.

On one hand, it's somewhat difficult to reflect on the Christmas readings.
After all, despite the fact that there is just the one set,\footnote{Not A,B,C or 1,2} there are four readings which belong to the celebration.
The Vigil Mass has the same Gospel as last Sunday's with verses on either side added.

I think there's something beautiful in what part of the Nativity story each of the four readings focuses on.
At the Vigil, which we celebrate because we take the custom of counting days starting at sunset from the Jewish tradition, we get the lineage of Christ.
After the lineage, we are simply told that Mary was found with child, and that Joseph was a righteous man who did not want to put her to shame.
We are reminded of the fact that Christ is the fulfillment of the promises to Abraham and his descendants.

At the Night Mass, we read from Luke.
We are reminded that the Lord was born in the night amidst shepherds.
In the darkest time of year, we are reminded of the Joy that came into the world, bringing light to our darkness.

As night turns to dawn, we get the readings for the Dawn Mass.
The Gospel picks up exactly where the Night Mass ended.
Just as we go from darkness and fear into praise and joy in the light, we watch the shepherds go and spread the glory of the Almighty.

Finally, there is the Day Mass.
In this Gospel, we go through the beginning of John.
We are not awaiting the Christchild anymore, we have received him.
Now begins the season of preparations for Lent and Easter.
And so, we first are reminded that Christ and the Father have been together since eternity.
We then are told of John, who paves the way for Christ's ministry.

Outside of the readings, I remain incredibly grateful for the time that I get to spend with my family.
I am also incredibly grateful for the fact that I can communicate with my friends around the world.
Merry Christmas to all.
\end{document}