\documentclass[12pt]{article}[titlepage]
\newcommand{\say}[1]{``#1''}
\newcommand{\nsay}[1]{`#1'}
\usepackage{endnotes}
\newcommand{\1}{\={a}}
\newcommand{\2}{\={e}}
\newcommand{\3}{\={\i}}
\newcommand{\4}{\=o}
\newcommand{\5}{\=u}
\newcommand{\6}{\={A}}
\newcommand{\B}{\backslash{}}
\renewcommand{\,}{\textsuperscript{,}}
\usepackage{setspace}
\usepackage{tipa}
\usepackage{hyperref}
\begin{document}
\doublespacing
\section{\href{reflections-on-readings-8-ordinary-c-22.html}{Reflections on Today's Gospel}}
First Published: 2022 February 27

Lule 6:44A \say{For every tree is known by its own fruit.}

\section{Draft 1}
Today is the last Sunday before Lent.
I find the readings, and especially their connection, really powerful as we approach this next Liturgical phase.

The Gospel tells us that \say{A good tree does not bear rotten fruit, nor does a rotten tree bear good fruit. For every tree is known by its own fruit. For people do not pick figs from thornbushes, nor do they gather grapes from brambles.}\footnote{Luke 6:43-44}
We are also told in the First Reading that \say{The fruit of a tree shows the care it has had,}\footnote{Sirach 27:6A}
From these two lines, I at least see that we see the care of a tree making it good or rotten.

When taking care of a tree, it feels tempting to give it more and constant sunshine, since that's where its energy comes from.
But, the nighttime is also essential to the growth of a tree.
Night is when a plant does much of its growing apparently\footnote{so says a random internet source}, and it is essential to their growth and flowering.
So, too, are the different periods in the Liturgical Year essential to our growth in faith.

A tree grows in the sunshine because it has the energy there and immediately available to grow.
So too, do we grow in our days of feasting, seeing the Lord in his joy and goodness.
But, at night time a tree grows to find the sunlight again.
So too, do we grow in our days of fasting as we think on the many blessings we have been given, far in excess of what we deserve.

The homily I heard tonight made the claim that focusing on fixing temporal ills is ultimately meaningless, because the world will end.\footnote{I think, it was a little unclear to me}
In the example, a man brings about world peace and dies, feeling satisfied.
Two years later, an asteroid destroys the earth.
The priest's claim was that this shows the meaningless of earthly accomplishments, because everything he did was undone in a few years.
I took the opposite message away, though.
For the two years after his death, there was no hunger pulling people from Christ, no war ending lives needlessly.
Even one moment free of needless hardship for one person makes temporal good worth it to me.

\end{document}