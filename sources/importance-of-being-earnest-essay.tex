\documentclass[12pt]{article}[titlepage]
\newcommand{\say}[1]{``#1''}
\newcommand{\nsay}[1]{`#1'}
\usepackage{endnotes}
\newcommand{\1}{\={a}}
\newcommand{\2}{\={e}}
\newcommand{\3}{\={\i}}
\newcommand{\4}{\=o}
\newcommand{\5}{\=u}
\newcommand{\6}{\={A}}
\newcommand{\B}{\backslash{}}
\renewcommand{\,}{\textsuperscript{,}}
\usepackage{setspace}
\usepackage{tipa}
\usepackage{hyperref}
\begin{document}
\doublespacing
\section{\href{importance-of-being-earnest-essay.html}{Essay About Importance of Being Earnest}}
\section{Draft 4: 21 October 2018}
The final show that Oscar Wilde wrote before being imprisoned for homosexuality was \textit{The Importance of Being Earnest}.
In part because of this, a lot of critical analysis has been done on analyzing the show through the lens of understanding how Wilde might have felt about his own homoerotic tendencies.
Since the analysis is ultimately on the theatrical work, much of this discussion enters into the realm of the performance.
As a result, a controversial aspect of performing Oscar Wilde's \textit{The Importance of Being Earnest} is the extent to which homoerotic subtext should or should not be performed and turned into action on-stage.
And, as the Vaudeville Theatre's recent staging suggests, the consensus of theatre critics in London is that the potential homoerotic subtext in the show should remain as such, and not be elevated to the role of staging.
Nonetheless, one aspect that seems neglected is Wilde's apparent disdain for love and marriage throughout the show.
Regardless of the existence or validity of seemingly homoerotic subtext, \textit{The Importance of Being Earnest} contains significant amounts of anti-marriage sentiment, or at least anti-contemporary British marriage.
This attack seems focused on two fronts: one generally on the subject of marriage, and the other on young people's view of marriage.

The initial offensive on the general idea of marriage by characters begins almost immediately when the show begins.
Lane (Algernon's butler) mentions that \say{I have often observed that in married households the champagne is rarely of first-rate brand} (Wilde 6).
After this, Lane makes a comment about how he had been married in his youth, but is no longer.
Algernon internally admonishes him, saying that \say{Lane's views on marriage seem somewhat lax,} before questioning, \say{if the lower classes don't set us a good example, what on earth is the use of them?} (6).
But, based on his comments towards Jack about the nature of divorce only a few pages later, namely that \say{divorces are made in heaven,} (8), it seems more likely that he was admonishing Lane's view that marriage is \say{is a very pleasant state,} rather than his other comment about having been married previously.
In this conversation, Wilde makes two points clear.
First, being a bachelor is a preferable position to being married.
Second, as the conversation begins in discussion of a \say{bachelor's establishment} (6), the conversation makes the claim that being around other unmarried men is the preferable state for a man.
Not content to simply disparage the institution of marriage, Wilde also quickly attacks the idea of monogamy, having Algernon comment that \say{in married life, three is company and two is none,} and that the idea has been \say{proved in half the time (25 years)} (12).
However, lest the audience take Wilde's bitter words as angrily as they could be interpreted, Jack admonishes Algernon, \say{don't try to be cynical} (12).

The idea of mocking young people's expectations of marriage is, like the idea of mocking marriage as an institution, an idea that is expressed from the early parts of the show.
When Jack arrives early in the first scene, he and Algernon discuss the potential for Jack and Gwendolen (Algernon's cousin) to get married.
Algernon claims that he \say{doesn't see anything romantic in proposing,} and that there is \say{nothing romantic about a definite engagement,} (7).
He immediately expands on his thought, explaining that \say{the very essence of romance is uncertainty,} (7).
This line of thinking stems from the idea that marriage naturally entails a lack of surprise, and therefore romance, and therefore love.
In rebuttal to Jack's claim that \say{the Divorce Court was specially invented for people} like Algernon (7), Algernon replies that \say{divorces are made in heaven,} (8) a clear reversal of the typical admonition that \say{marriage is made in heaven.}
This conversation continues its disparagement of marriage, with Algernon commenting \say{girls never marry the boys they flirt with,} (8).
His disparagement seems to be focused on the same line of thought as his belief in romance.
In Algernon's view, a girl flirts with someone she feels uncertain about.
So, either she feels uncertain about the man, and therefore wouldn't be willing to marry him, or she feels certain, and so has no need to flirt.
Algernon lends more credence to that belief when he states that \say{the amount of women in London who flirt with their own husbands is perfectly scandalous} (11).

However, Wilde is not content to simply mock a bachelor's view of marriage, and extends his battle to that of young, eligible women.
Both Gwendolen and Cecily comment that they could not marry their love if he was not named Ernest.
Gwendolen comments that in the name Jack, her love's true name, \say{there is very little music in the name,} and that she \say{pit[ies] any woman who is married to a man named John,} and since \say{Jack is a notorious domesticity for John,} she clearly could not marry a man named Jack.
Cecily is blunter, and outright tells Algernon that \say{[she] might respect you, Ernest, I might admire your character, but I fear that I should not be able to give you my undivided attention} (36).
That plays into a larger goal of the show, that of making the trivial serious and the serious trivial, but still serves to highlight Wilde's mockery of married British life.

There is a sly joke about the nature of proposals of marriage as Jack attempts to propose to Gwendolen and is mocked for his poor performance, as Gwendolen comments that \say{men often propose for practice. I know my brother Gerald does,} as a way of noting to Jack that he should have been better prepared for his own proposal.
However, it also has the effect of telling the audience that a proposal is not a weighty action, preparing two people to be bound together.
Rather, proposing is simply asking a question to pass the time, or playing a game as if to practice for reality.
Interestingly, Cecily seems to take a similar but disparate view of engagements to Gwendolen, in that she believes that it should take multiple attempts to make it to the altar.
Where Gwendolen seems to think you should propose to multiple people, who (one assumes) would reject you, before meeting your love, Cecily believes that \say{it would hardly have been a serious engagement if it hadn't been broken off at least once} (36).

Another place where Wilde mocks the young women's perspective of love is when Gwendolen and Cecily first meet.
During their meeting, they argue about which of them is engaged to be married to Ernest.
As evidence for each of their proposals, they produce their diaries.
Of course, the audience is made to laugh, as earlier Cecily had commented often about how she had fabricated her diary, with lines such as \say{I was forced to write your letters for you} (35).
Gwendolen continues this statement of falsified diaries, mentioning that \say{I never travel without my diary. One should always have something sensational to read in the train} (39).
They proceed to argue about whether the initial proposal is more valid, as it was an earlier contract, or whether the new proposal should be seen as the more valid one, because it clearly demonstrates a change of heart.
Through this mockery of diaries, and emphasis on the diary as the most important record of young women's emotion, Wilde again diminishes the importance of proposing.

Wilde, in addition to mocking a young women's belief in marriage, plays with the beliefs of marriage through his only older character, Lady Bracknell.
Lady Bracknell seems to play a different role than every other character in the show.
Rather than being a young, unaware person like the other leads, she is an old, married woman.
Through her, Wilde is given the opportunity to play with ideas of marriage even more.
From her first entrance, she clashes with the standard agreement of marriage.
She says that Lady Harburry, whose husband's death is alluded to, has become \say{quite golden with grief,} (13) a subversion of the typical golden with joy.
Later in the show, she mocks the idea of a marriage of equals, claiming that \say{I have never undeceived him (her husband) on any question. I would consider it wrong} (48).

Wilde was undoubtedly struggling with his own self image in the time he wrote \textit{The Importance of Being Earnest}, but seems to also have been struggling with his own married life.
If he, in writing the show, could have audiences laugh at the absurdity of marriage, maybe his own married life might seem less gloomy.
And so, despite his claim of writing \say{art for art's sake,} a clear case can be made that he was writing the show as a way to deal with his own marital issues.
Throughout the show, the idea of marriage is mocked again and again, to the point where it seems as absurd as the idea that a child and a manuscript could accidentally be switched at a railway station.

\section{Draft 3.5: 21 October 2018}
The final show that Oscar Wilde wrote before being imprisoned for homosexuality was \textit{The Importance of Being Earnest}.
In part because of this, a lot of critical analysis has been done on analyzing the show through the lens of understanding how Wilde might have felt about his own homoerotic tendencies.
As the analysis is ultimately on a theatrical work, much of this discussion enters into the realm of performing this show.
As a result, a controversial aspect of performing Oscar Wilde's \textit{The Importance of Being Earnest} is the amount to which homoerotic subtext should or should not be performed and turned into action on-stage.
And, as the Vaudeville Theatre's recent staging suggests, the consensus of theatre critics in London is that the potential homoerotic subtext in the show should remain as subtext, and not be elevated to the role of staging.
Nonetheless, one aspect that seems neglected is Wilde's disdain for love and marriage throughout the show.
Regardless of existence or validity of seeming homoerotic subtext, \textit{The Importance of Being Earnest} contains significant amounts of anti-marriage sentiment, at least in the way that marriage was viewed in his society.
This attack seems focused on two fronts.
The first of these is the set of attacks made by characters in reference to marriage and married people.
The second of these is the more misguided attack, wherein the unmarried characters make declarations about the nature of marriage that are laughably naive.

The direct attacks on marriage by characters begins almost immediately when the show begins.
 Lane (Algernon's butler) mentions that \say{I have often observed that in married households the champagne is rarely of first-rate brand,} (p. 6).
When Jack arrives, Algernon claims that he \say{doesn't see anything romantic in proposing,} and that there is \say{nothing romantic about a definite engagement,} (p. 7).
He immediately expands on his thought, explaining that \say{the very essence of romance is uncertainty,} (p. 7 ).
In rebuttal to Jack's claim that \say{the Divorce Court was specially invented for people} like Algernon (p. 7), Algernon replies that \say{divorces are made in heaven,} (p. 8) a clear reversal of the typical admonition that \say{marriage is made in heaven.}
This conversation continues its disparagement of marriage, with Algernon commenting \say{girls never marry the boys they flirt with,} (p. 8).
Not content to simply disparage the institution of marriage, Wilde also attacks the idea of monogamy, having Algernon comment that \say{in married life, three is company and two is none,} and that the idea has been \say{proved in half the time (25 years)} (p. 12).
However, lest the audience take Wilde's bitter words as angrily as they could be interpreted, Jack admonishes Algernon, \say{don't try to be cynical} (p. 12).

After this, Lane makes a comment about how he had been married in his youth, but is no longer.
Algernon internally admonishes him, saying that \say{Lane's views on marriage seem somewhat lax,} before questioning, \say{if the lower classes don't set us a good example, what on earth is the use of them?} (p. 6).
But, based on his comments towards Jack about the nature of divorce, namely that \say{divorces are made in heaven,} (p. 8), it seems more likely that he was admonishing Lane's view that marriage is \say{is a very pleasant state,} rather than his other comment about having been married previously.
In this conversation, Wilde makes two points clear.
First, being a bachelor is a preferable position to being married.
Second, as the conversation begins in discussion of a \say{bachelor's establishment} (p. 6), it makes the claim that being around other unmarried men is the preferable state for a man.

Not content to simply mock a bachelor's view of marriage, Wilde extends his attack to that of a young, eligible woman.
Both Gwendolen and Cecily comment that they could not marry their love if he was not named Ernest.
In Gwendolen's place, she comments about Jack, her loves true name, \say{there is very little music in the name,} and that she \say{pit(ies) any woman who is married to a man named John,} and since \say{Jack is a notorious domesticity for John,} she clearly could not marry a man named Jack.
Cecily is blunter, and outright tells Algernon that \say{I (she) might respect you, Ernest (Algernon), I might admire your character, but I fear that I should not be able to give you my undivided attention,} (p. 36)
That plays into a larger goal of the show, that of making the trivial serious and the serious trivial, but still serves to highlight Wilde's distaste of married life.

Wilde continues his attack on marriage during Algernon's conversation with Jack.
Algernon states that \say{the amount of women in London who flirt with their own husbands is perfectly scandalous,} (p. 11).
When Lady Bracknell comes in, she mentions another subversion.
Lady Harburry, whose husband's death is alluded to, is said to have become \say{quite golden with grief,} (p. 13) a subversion of the typical golden with joy.
In these subversions, Wilde tells the audience even though he is aware of the stereotypes, he just simply ignores them.

There is a sly joke about the nature of proposals for marriage on page 16, where Gwendolen comments that \say{men often propose for practice. I know my brother Gerald does,} as a way of noting to Jack that he should have been better prepared for his own proposal.
However, it also has the effect of telling the audience that a proposal is not a weighty action, preparing two people to be bound together.
Rather, proposing is simply asking a question to pass the time, or playing a game as if to practice for reality.
Interestingly, Cecily seems to take the opposite view of engagements and proposals to Gwendolen.
Where Gwendolen seems to think you should propose to multiple people, who (one assumes) would reject you before meeting your love, Cecily believes that \say{it would hardly have been a serious engagement if it hadn't been broken off at least once,} (p. 36).

Another place where Wilde mocks the young women's perspective of love is when Gwendolen and Cecily first meet.
During their meeting, they argue about which of them is engaged to be married to Ernest.

As evidence for each of their proposals, they produce their diaries.
Of course, the audience is made to laugh, as earlier Cecily had commented often about how she had fabricated her diary, with lines such as \say{I was forced to write your letters for you} (p. 35).
They argue about whether the initial proposal is more valid, as it was an earlier contract, or whether the new proposal should be seen as the more valid one, because it clearly demonstrates a change of heart.

Lady Bracknell mocks the idea of a marriage of equals, claiming that \say{I have never undeceived him (her husband) on any question. I would consider it wrong,} (p. 48).

Wilde was undoubtedly struggling with his own self image in the time he wrote \textit{The Importance of Being Earnest}, but seems to also have been struggling with his own married life.
If he, in writing the show, could have audiences laugh at the absurdity of marriage, maybe his own married life might seem less gloomy.
And so, despite his claim of writing \say{art for art's sake,} a clear case can be made that he was writing the show as a way to deal with his own marital issues.

\section{Draft 3: 21 October 2018}
One controversial aspect of performing Oscar Wilde's \textit{The Importance of Being Earnest} is the amount to which homoerotic subtext should or should not be performed and turned into action on-stage.
Wilde was arrested for homosexuality shortly after the premier of the show, and so much imagery is plausibly available.
However, the most interesting arc that Wilde shows in \textit{The Importance of Being Earnest} is his own disdain for love and marriage.
Regardless of the homoerotic subtext, there is a significant of anti-heteronormative in-text verbiage in the show.

This anti-heterosexual sentiment is not a slyly alluded to metaphor.
Rather, even from the first few lines, Lane (Algernon's butler) mentions that \say{I have often observed that in married households the champagne is rarely of first-rate brand,} (p. 6).
After this, Lane makes a comment about how he had been married in his youth, but is no longer.
Algernon internally admonishes him, saying that \say{Lane's views on marriage seem somewhat lax,} before questioning, \say{if the lower classes don't set us a good example, what on earth is the use of them?} (p. 6).
But, based on his comments towards Jack about the nature of divorce, namely that \say{divorces are made in heaven,} (p. 8), it seems more likely that he was admonishing Lane's view that marriage is \say{is a very pleasant state,} rather than his other comment about having been married previously.
In this conversation, Wilde makes two points clear.
First, being a bachelor is a preferable position to being married.
Second, as the conversation begins in discussion of a \say{bachelor's establishment} (p. 6), it makes the claim that being around other unmarried men is the preferable state for a man.

Immediately after this, Jack arrives.
While discussing their cousin Gwendolen, Algernon makes the claim that he \say{doesn't see anything romantic in proposing,} and that there is \say{nothing romantic about a definite engagement,} (p. 7).
He immediately expands on his thought, explaining that \say{the very essence of romance is uncertainty,} (p. 7 ).
In rebuttal to Jack's claim that \say{the Divorce Court was specially invented for people} like Algernon (p. 7), Algernon replies that \say{divorces are made in heaven,} (p. 8) a clear reversal of the typical admonition that \say{marriage is made in heaven.}
This conversation continues its disparagement of marriage, with Algernon commenting \say{girls never marry the boys they flirt with,} (p. 8).
Not content to simply disparage the institution of marriage, Wilde also attacks the idea of monogamy, having Algernon comment that \say{in married life, three is company and two is none,} and that the idea has been \say{proved in half the time (25 years)} (p. 12).
However, lest the audience take Wilde's bitter words as angrily as they could be interpreted, Jack admonishes Algernon, \say{don't try to be cynical} (p. 12).

Not content to simply mock a bachelor's view of marriage, Wilde extends his attack to that of a young, eligible woman.
Both Gwendolen and Cecily comment that they could not marry their love if he was not named Ernest.
In Gwendolen's place, she comments about Jack, her loves true name, \say{there is very little music in the name,} and that she \say{pit(ies) any woman who is married to a man named John,} and since \say{Jack is a notorious domesticity for John,} she clearly could not marry a man named Jack.
Cecily is blunter, and outright tells Algernon that \say{I (she) might respect you, Ernest (Algernon), I might admire your character, but I fear that I should not be able to give you my undivided attention,} (p. 36)
That plays into a larger goal of the show, that of making the trivial serious and the serious trivial, but still serves to highlight Wilde's distaste of married life.

Wilde continues his attack on marriage during Algernon's conversation with Jack.
Algernon states that \say{the amount of women in London who flirt with their own husbands is perfectly scandalous,} (p. 11).
When Lady Bracknell comes in, she mentions another subversion.
Lady Harburry, whose husband's death is alluded to, is said to have become \say{quite golden with grief,} (p. 13) a subversion of the typical golden with joy.
In these subversions, Wilde tells the audience even though he is aware of the stereotypes, he just simply ignores them.

There is a sly joke about the nature of proposals for marriage on page 16, where Gwendolen comments that \say{men often propose for practice. I know my brother Gerald does,} as a way of noting to Jack that he should have been better prepared for his own proposal.
However, it also has the effect of telling the audience that a proposal is not a weighty action, preparing two people to be bound together.
Rather, proposing is simply asking a question to pass the time, or playing a game as if to practice for reality.
Interestingly, Cecily seems to take the opposite view of engagements and proposals to Gwendolen.
Where Gwendolen seems to think you should propose to multiple people, who (one assumes) would reject you before meeting your love, Cecily believes that \say{it would hardly have been a serious engagement if it hadn't been broken off at least once,} (p. 36).

Another place where Wilde mocks the young women's perspective of love is when Gwendolen and Cecily first meet.
During their meeting, they argue about which of them is engaged to be married to Ernest.

As evidence for each of their proposals, they produce their diaries.
Of course, the audience is made to laugh, as earlier Cecily had commented often about how she had fabricated her diary, with lines such as \say{I was forced to write your letters for you} (p. 35).
They argue about whether the initial proposal is more valid, as it was an earlier contract, or whether the new proposal should be seen as the more valid one, because it clearly demonstrates a change of heart.

Lady Bracknell mocks the idea of a marriage of equals, claiming that \say{I have never undeceived him (her husband) on any question. I would consider it wrong,} (p. 48).

Throughout Oscar Wilde's \textit{The Importance of Being Earnest}, he consistently and regularly mocks the idea of marriage.
Rather than focusing on the extent and nature of the homoerotic subtext within the show, an analysis based simply on dissatisfaction with his own marriage may yield more information on the mind of the author who wrote it.
But, as he was known to be a quick wit, it may simply be that he was capable of mocking that which he held dear.
1081 words.

\section{Draft 2: 21 October 2018}
Oscar Wilde's \textit{The Importance of Being Earnest} mocks contemporary British society by making caricatures of both nobility and the nouveau-riche.
Although he claims that his goal is \say{art for art's sake,} and so there is no point to be raised, the humor of the show comes from his mockery of the culture.
That is, the humor of the show comes from its close relation to the real world that the audience would be occupying.
Oscar Wilde drives his plot in \textit{The Importance of Being Earnest} solely through the use of conversation, though only a specific subset of conversation, namely the small-talk that aristocrats use as a way of demonstrating their level of prestige and make a mockery of his contemporary culture.

Interestingly, small-talk seems to play the same role in this show that it typically does in reality.
The goal of the speech is to be inoffensive, so-as to prevent any discord or strife, and to prevent making an enemy of someone you are speaking to, whether for personal gain, or simply to avoid personal loss.

Speaking of small-talk, Wilde chooses to open the show with small-talk between Algernon and his butler, Lane, having them discuss the preparations for Lady Bracknell's visit.
This sort of dialogue is the stereotypical dialogue between a bachelor and his butler, focusing on drinking, such as Algernon's query \say{Why is it... the servants invariably drink the champagne?} (p 6).
As Jack enters, the dialogue shifts into the sort of small-talk proffered between two people of similar social stature.
When Algernon asks \say{What brings you up to town?}, Jack esponds \say{Oh, pleasure pleasure! What else should bring anyone anywhere?} before responding with an attack on Algernon, \say{Eating as usual, I see, Algy!} (p. 7).
However, even that attack falls within the stereotypical domain of small-talk, as there is a focus on the diet of people in his contemporary society.

Algernon and Jack's small talk continues until Lady Bracknell arrives.
Here, the conversation changes as one would expect when an older aristocratic woman joins, with her beginning \say{good afternoon, dear Algernon, I hope you are behaving well.} (p. 12).
Algernon's reply that he is \say{feeling very well} (ibid) is another of Wilde's uses of quip where no quip would be expected in a real conversation.
These immediate responses serve as a way to show to the audience that the characters on stage are not, as the Naturalists and Realists would have you believe, people, but are rather caricatures.
These caricatures could not belong in real life, and so Wilde puts them on the stage to be laughed at.

Wilde continues his minimization of the important, and his grand-sizing of the trivial on page 18, where Lady Bracknell says \say{now onto minor matters. Are your parents living?} immediately after questioning him on whether he lived on the \say{fashionable} side of Belgrave Square.
This treatment of important as small-talk and small talk as important is an interesting theme in the show.
It makes the analysis of small-talk much more difficult, as the topics addressed with the (lack of) gravitas normally expected for small talk are generally matters of the utmost import, while topics that would be seen as trivial and unimportant are, as demonstrated by the comparisons of diaries between Cecily and Gwendolen in Act Two, treated as utterly important.

Lady Bracknell notes the importance of being seen well by your peers and betters, and admonishes Algernon to \say{never speak disrespectfully of Society, Algernon. Only people who can't get into it do that,} (p. 50).
But Wilde is not content to let the small note of not discussing anything of real importance remain small.
An example of this is Lady Bracknell's diatribe about eduction on page 17, where she says \say{I do not approve of anything that tampers with natural ignorance. ... Fortunately in England, at any rate, education produces no effect whatsoever.}
This idea leads into a later goal of the show, namely that of mocking society.

However, obviously some of the text of the show does not deal with small-talk.
Yet, paradoxically, the treatment of this speech is treated the same way that small-talk is in a typical show.
Groundbreaking lines about a man who doesn't know his family history are treated as small throw-away gags (though that does end up becoming a plot point later in the show), while the small-talk that normally would be ignored (such as Algernon finding Jack's cigarette case) ends up being crucial to the plot.

In \textit{Earnest,} more-so than many other shows, the small-talk and seemingly-random dialogue plays a large role in driving the plot.
The off handed comment about how Jack had left a cigar case somehow becomes the entire rest of the plot, as Algernon discovers that Jack is leading a second life.
That second life leads Algernon to meet his future bride in Cecily.

Another interesting theme in the show, especially given Wilde's own sexual preferences, is that of marriage.
To begin, in the first few lines of the show, Lane (Algernon's butler) mentions that \say{I have often observed that in married households the champagne is rarely of first-rate brand,} (p. 6).
Algernon internally admonishes him, saying that \say{Lane's views on marriage seem somewhat lax,} before questioning, \say{if the lower classes don't set us a good example, what on earth is the use of them?} (p. 6).

Later, when Jack arrives, Algernon makes the claim that he \say{doesn't see anything romantic in proposing,} and that there is \say{nothing romantic about a definite engagement,} (p. 7).
He immediately expands on his thought, explaining that \say{the very essence of romance is uncertainty,} (p. 7 ).
In rebuttal to Jack's claim that \say{the Divorce Court was specially invented for people} like Algernon (p. 7), Algernon replies that \say{divorces are made in heaven,} (p. 8) a clear reversal of the typical admonition that \say{marriage is made in heaven.}

This confusion about marriage continues through the rest of show, with the next example coming in Algernon's view that \say{girls never marry the boys they flirt with,} (p. 8).
Wilde is aware of his bitterness towards marriage, and has Jack comment, \say{don't try to be cynical} after Algernon comments that the statement \say{in married life, three is company and two is none} has been \say{proved in half the time (25 years)} (p. 12).

A twice recurred comment in the show, which both Algernon and Jack experience, is that their lady love couldn't love them if they weren't named Ernest.

And, there is the long diversion about the nature of engagement, with the evidence coming from each of the two women's diaries.
 
No real conversation flows as smoothly as each of the conversations between the characters.
And yet, this flow of conversation doesn't play the role that smooth dialogue does in a show, making a character seem eloquent or humorous.
Rather, it is used to make the characters seem inhuman and grotesque.

\section{Draft 1: 20 October 2018}
Oscar Wilde's \textit{The Importance of Being Earnest} mocks contemporary British society by making caricatures of both nobility and the nouveau-riche.
Although he claims that his goal is \say{art for art's sake,} and so there is no point to be raised, the humor of the show comes from his mockery of the culture.
But, the humor of the show comes from its close relation to the real world that the audience would be occupying.
Oscar Wilde drives his plot in \textit{The Importance of Being Earnest} solely through the use of conversation, though only a specific subset of conversation, namely the small talk that aristocrats use as a way of demonstrating their level of prestige and make a mockery of his contemporary culture.

Speaking of small-talk, Wilde chooses to open the show with small-talk between Algernon and his butler, Lane, having them discuss the preparations for Lady Bracknell's visit.
As Jack enters, the dialogue shifts into the sort of small-talk proffered between two people of similar social stature.
When asked what brings Jack to Algernon's abode, he (Jack) responds \say{Oh, pleasure pleasure! What else should bring anyone anywhere?} (p. 7) before responding with an attack on Algernon, \say{Eating as usual, I see, Algy!}
However, even that attack falls within the stereotypical domain of small-talk.

When Lady Bracknell arrives, the conversation begins as one would expect an aristocratic conversation to begin, with her commenting \say{good afternoon, dear Algernon, I hope you are behaving well.} (p. 12)
Algernon's reply that he is \say{feeling very well} (ibid) is another of Wilde's uses of quip where no quip would be expected in a real conversation.
These immediate responses serve as a way to show to the audience that the characters on stage are not, as the Naturalists and \textbf{WORD} would have you believe, people, but are rather caricatures.
These caricatures could not belong in real life, and so Wilde puts them on the stage to be laughed at.

Interestingly, small-talk seems to play the same role in this show that it typically does.
Its goal is to be inoffensive, so-as to prevent any discord or strife, and to prevent making an enemy of someone near you.

Wilde continues his minimization of the important, and his grandsizing of the trivial on page 18, where Lady Bracknell says \say{now onto minor matters. Are your parents living?} immediately after questioning him on whether he lived on the \say{fashionable} side of Belgrave Square.

Lady Bracknell notes the importance of being seen well by your peers and betters, and notes that \say{Some quote about how only trashy people believe things}
But Wilde is not content to let the small note of not discussing anything of real importance remain small.
Instead, Gwendoline tells that she was \say{educated not at all or something}
This idea leads into a later role, that of mocking society.

However, obviously some of the text of the show does not deal with small-talk.
Yet, paradoxically, the treatment of this speech is treated the same way that small-talk is in a typical show.
Groundbreaking lines about a man who doesn't know his family history are treated as small throw-away gags (though that does end up becoming a plot point later in the show), while the small-talk that normally would be ignored ends up being crucial to the plot.

In \textit{Earnest,} more-so than many other shows, the small-talk and seemingly-random dialogue plays a large role in driving the plot.
The off handed comment about how Jack had left a cigar case somehow becomes the entire rest of the plot, as Algernon discovers that Jack is leading a second life.
That second life leads Algernon to meet his future bride.
And, there is the long diversion about the nature of engagement, with the evidence coming from each of the two women's diaries.

Wilde uses the overemphasized nature of the characters' small-talk as a way to mock his society.
Most of the barbs he throws are oblique, and mention British society only tangentially.
However, some of his statements directly target the reigning society.
An example of this is Lady Bracknell's diatribe about eduction on page 17, where she says \say{I do not approve of anything that tampers with natural ignorance. ... Fortunately in England, at any rate, education produces no effect whatsoever.}
 
No real conversation flows as smoothly as each of the conversations between the characters.
And yet, this flow of conversation doesn't play the role that smooth dialogue does in a show, making a character seem eloquent or humorous.
Rather, it is used to make the characters seem inhuman and grotesque.

\section{Draft 0.1: 20 October 2018}
Oscar Wilde's \textit{The Importance of Being Earnest} mocks contemporary British society by making caricatures of both nobility and the nouveau-riche.
In doing so, he claims that his goal is \say{art for art's sake,} and so there is no point to be raised.
But, the humor of the show comes from its close relation to the real world that the audience would be occupying.
Oscar Wilde drives his plot in \textit{The Importance of Being Earnest} solely through the use of conversation, though only a specific subset of conversation, namely the small talk that aristocrats use as a way of demonstrating their level of prestige and make a mockery of his contemporary culture.

Wilde's show is made up of high amounts of aristocratic smalltalk.
Lady Bracknell notes the importance of being seen well by your peers and betters, and notes that \say{Some quote about how only trashy people believe things}
But Wilde is not content to let the small note of not discussing anything of real importance remain small.
Instead, Gwendoline tells that she was \say{educated not at all or something}
This idea leads into a later role, that of mocking society.

However, obviously some of the text of the show does not deal with small-talk.
Yet, paradoxically, the treatment of this speech is treated the same way that small-talk is in a typical show.
Groundbreaking lines about a man who doesn't know his family history are treated as small throw-away gags (though that does end up becoming a plot point later in the show), while the small-talk that normally would be ignored ends up being crucial to the plot.

In \textit{Earnest,} more-so than many other shows, the small-talk and seemingly-random dialogue plays a large role in driving the plot.
The off handed comment about how Jack had left a cigar case somehow becomes the entire rest of the plot, as Algernon discovers that Jack is leading a second life.
That second life leads Algernon to meet his future bride.
And, there is the long diversion about the nature of engagement, with the evidence coming from each of the two women's diaries.

Wilde uses the overemphasized nature of the characters' small-talk as a way to mock his society.
No real conversation flows as smoothly as each of the conversations between the characters.

\section{Draft 0: 20 October 2018}
Note: I'm workshopping the beginning with a friend. I forgot how much I hate doing this sort of essay prompt.

Two friends whose lies and stories spiral out of control.

Oscar Wilde's \textit{The Importance of Being Earnest} is only wholly character driven show we've seen.
In every other show, some of the action is produced and done by characters offstage.

Wilde puts the frame around this action as a way of mocking his culture.

The idea explored is that there is nothing more in reality than what you make over smalltalk.

Don't say most character driven, only say character driven.

Oscar Wilde drives his plot in \textit{The Importance of Being Earnest} solely through the use of conversation, though only a specific subset of conversation, namely the small talk that aristocrats use as a way of demonstrating their level of prestige and make a mockery of his contemporary culture.

\begin{enumerate}
\item Defining aristocratic small talk using text
\item Short piece accepting not all of it is small talk
\item Small talk = prestige
\item connect small talk to plot
\item mockery of culture - make the biggest
\item so what?
\end{enumerate}
\end{document}