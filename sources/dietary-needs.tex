\documentclass[12pt]{article}[titlepage]
\newcommand{\say}[1]{``#1''}
\newcommand{\nsay}[1]{`#1'}
\usepackage{endnotes}
\newcommand{\B}{\backslash{}}
\renewcommand{\,}{\textsuperscript{,}}
\usepackage{setspace}
\usepackage{tipa}
\usepackage{hyperref}
\begin{document}
\doublespacing
\section{\href{dietary-needs.html}{Initial Reflection on Dietary Needs}}

First Published: 8 January 2025

\section{Draft 2: 8 January 2025}

As I've mentioned in a fair number of musings, I want to be better at fueling my body.  
That means that I'm going to aim for a few goals:

\begin{itemize}  
\item Macronutrients all at acceptable levels  
\item Unprocessed or less processed foods where possible  
\item More fruits and vegetables, generally aiming for dark green, red, orange, legumes, starches to be each represented.  
\item Easy, in mental, physical, and temporal meanings.  
\end{itemize}

\href{https://www.nal.usda.gov/human-nutrition-and-food-safety/dri-calculator/}{both lets me} get some estimates and gives me the actual links to the book where the USDA lays out their explicit recommendations. 
Unfortunately, those PDFs do not give citations, but I will trust that the data are at least generally good.  
With that in mind, the goals I should be hitting:

\begin{itemize}  
\item At least a gallon of total water per day. I generally feel like I should be drinking around that much, if I go by the whole \say{drink when you're thirsty} mantra.  
\item An absolute floor of \href{https://nap.nationalacademies.org/read/11537/chapter/10}{130 grams} of carbohydrates a day.\footnote{though, I could easily believe that my brain is in the top 2 to 3 percent of calorie usage, in which case would need to be higher.}  
\item If I trust the AMDR, which is at the very least a decent starting place, I should be getting between 45 and 65 percent of calories from carbs, 25 to 35 percent fat, and 10 to 30 protein. I can hit the thresholds of at least one of them without breaking the system, which is good.  
\item I should be getting \href{https://nap.nationalacademies.org/read/11537/chapter/14}{around 0.8} grams of protein per kilogram of body mass per day, bringing us to around 90 grams of protein.  
\item In that, I apparently absolutely have to have: histidine, isoleucine, leucine, lysine, methionine, phenylalinine, threonine, tryptophan, and valine. I should make sure that I have the prerequisites, or take in: arginine, cysteine, glutamine, glycine, proline, and tyrosine. \href{https://nap.nationalacademies.org/read/11537/chapter/53#464}{daily, that means}:\footnote{all in mg/kg, all RDA}  
\begin{itemize}  
\item 19 histidine  
\item 30 Isoleucine  
\item 62 leucine  
\item 52 lysine  
\item 26 methionine and cysteine (combined, I assume)  
\item 51 phenylalanine and tyrosine (combined, I assume)  
\item 30 threonine  
\item 9 tryptophan  
\item 35 valine  
\end{itemize}
Now, because I am not completely insane, I probably won't be very much harping on all of these points, but it's at least generally good to keep in mind. Protein makes body run, and generally there don't seem to be side effects from too much protein or of any amino acid, so floors are almost certainly sufficient in themselves.  
\item Fat, unfortunately, lacks RDA, and instead only has Adequate Intake (AI). It's broken down into Linoleic and alpha Linoleic acids, which need 17 and 1.6 g/kg/day, respectively. Apparently outside of that, there's not evidence to suggest any other recommendations for fat, though I assume that percents are generally at least somewhat supported as normally decent for each piece. LDL is linked to increased saturated\footnote{more solid} fat content, and LDL is generally not great. Linoleic acid is an essential fatty acid, and it produces the n-6 fatty acids. It is not immediately clear to me what it does. Alpha is also essential, produces n-3, and is needed to balance n3 and n6 acids, as well as helping structural nerves especially in the nerves and retina. AIs were set from the median in the US, because there's no common deficit. In general, no American or Canadian is lacking in either of those fats, which is a nice, if unsurprising result.  
\item So, in general, fat can make up the rest of my diet, I should aim for less saturated fats, and also make sure I consume the proper fatty acids. My body can make all of them except for linoleic and alpha linoleic acids. They're important for growth, but I am no longer a child, and so no longer really grow.  
\item Higher fat diets (relative to carbs) tend to be better the more sedentary the population is. Given how sedentary I am, probably good to err on the side of more fat. Athletes may need higher carb diets, especially in the short term,   
\item Added sugar bad.  
\end{itemize}

Ok, so that's interesting and good enough. It's always nice to confirm that there's a large range of completely acceptable values for every macronutrient. The micros I will assume I can hit if I get a diverse diet in, at least for now.  
Now comes the much harder part: generating an actual diet.

I think that having the list of colors is a thing I need to do when plotting literally any meal, and especially while shopping.  
I think that it's also probably best to have as much of what I'm cooking as things that I can generally do in batches that requires minimal in person effort.  
The absolute ideal, is of course, something like the classic crock pot meal, which is a turn on whenever and come back to whenever, with no real time considerations other than a floor.

I know myself well enough to know that I do, legitimately, enjoy cooking when I can, and that I hate following recipes.  
Those two conditions make a lot of meal planning hard, as does my tendency to forget things when they're out of sight and to fall into chaos at the slightest provocation.  
However!   


Hope is not lost.  
I do know myself well enough to know that when I have a list, even if it's vague, I can often follow along with them, especially when I have a reason to do so.\footnote{like, for instance, this whole blog post and my general desire to eat better}

I think that this may be more than a single blog post is able to have, but I will definetly try to start making vague measurements in my recipes, so that I can both recreate them and see what the health information in them is. Additionally, that can let me start experimenting with different changes to each recipe.  
For instance, yesterday I baked a loaf of sandwich bread where I substituted yogurt for milk and added some almond flour.  


\section{Draft 1: 8 January 2025}  


As I've mentioned a fair amount in the recent musings, I want to be better about feeding myself.  
There are any number of reasons for this, but if I'm being honest, it's mostly due to the loss of my mother.\footnote{as so much of what I do is these days}  
With that in mind, there are some general guidelines that I'm going to try to keep in mind going forward.\footnote{will be looking things up really quickly to make sure that I don't miss anything super important.}


\begin{itemize}  
\item Get sufficient macros in. I know that I, personally, don't feel great when I don't get enough protein in.\footnote{Yes, I'm well aware that the average American gets far more than enough protein in their diet, but I'm not average}  
\item Focus on more whole grain and less processed food generally  
\item I often see fiber as a point that needs to be addressed. I'm not sure how true that is in my case, especially given the above item, but I'll at least ball park it to make sure I'm there.  
\item The CDC \href{https://www.cdc.gov/nccdphp/dnpao/features/healthy-eating-tips/index.html}{that I need more potassium}, likely. That's interesting, at least.  
\item Saturated fats are allegedly bad, though I keep wanting to look up why that is.\footnote{\href{https://www.health.harvard.edu/staying-healthy/the-truth-about-fats-bad-and-good}{says that trans fats are banned, interestingly}. Looking at \href{https://pubmed.ncbi.nlm.nih.gov/20071648/}{the abstract of this meta analysis}, seems like that's not necessarily true.} Looking at the Harvard link from the footnote, seems mostly like polyunsaturated fats, like corn and sunflower oil, are essential for bodily function. Other than that, more liquid at room temperature probably better, but not something to hugely stress about.  
\item Increase the variety and quantity of fruits and vegetables in my diet. I know that I probably don't get enough pretty colors, especially because I tend to go through phases of a single plant.\href{https://www.health.harvard.edu/healthbeat/how-the-dietary-guidelines-define-a-healthy-eating-pattern}{A Harvard page} claims that the groups I need to hit are: dark green, red, and orange legumes, and starchy.  
\item Don't be super fussy in the day to day. The goal is that this is a background process, both because I want to not spend time and energy on this, and also because literature generally says good enough is, in fact, good enough.  
\item Make it something that I can do easily. That can mean different things to me at different periods of my life, but I think that it's just a good thing to keep in mind in general.  
\end{itemize}

Alrighty, that's a pretty doable set of items. With that in mind, let's start being explicit with some numbers\footnote{for the macros, in particular}. Thankfully, \href{https://www.nal.usda.gov/human-nutrition-and-food-safety/dri-calculator}{Uncle Sam} has us covered with that. These days, I do unfortunately think that I am Inactive to Low Active based on their terms, though that is something I'd like to change:\footnote{N.B. I am rounding everything, because I don't like unnecessary precision, and don't really believe that my body is a bomb calorimeter (unfortunately).}  


\begin{itemize}  
\item 16 cups of water, including everything in the foods I consume. That's one gallon, and does change when I play with numbers, so that's fun.  
\item About 3500 calories a day. Oof that's going to be fun and difficult. Of course, if I'm going for a slight deficit\footnote{which I kind of am}, that's ok too.  
\item 90 grams of protein. Some quick googling suggests approx. 4 calories per gram, so 360 cals from protein\footnote{which, wildly enough, is on the low end. The USDA recommends 10-25 percent from protein}  
\item Somewhere from 75 to 135 grams of fat per day. 9 Cal/g, so 675 to 1215.\footnote{USDA wants 20 to 35 percent}  
\item Carbs are essential for brain function, or so claims \href{https://nap.nationalacademies.org/read/11537/chapter/10#103}{the USDA}. The recommended amount is listed by the amount expected to be needed for brain function, and explicitly prevents ketosis. Because all brains are basically the same size, carbohydrate recommendations plateau at the age of 1, and remain at around 130 grams after that. However, if we look at the numbers, that doesn't work out   
\end{itemize}

I'm realizing that I'm losing the thread, so let's restart.  


\end{document}