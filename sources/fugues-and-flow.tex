
\documentclass[12pt]{article}  
\newcommand{\say}[1]{``#1''}  
\newcommand{\nsay}[1]{`#1'}  
\usepackage{endnotes}  
\newcommand{\B}{\backslash{}}  
\renewcommand{\,}{\textsuperscript{,}}  
\usepackage{setspace}  
\usepackage{tipa}  
\usepackage{hyperref}  
\begin{document}  
\doublespacing  
\section{\href{fugues-and-flow.html}{On Fugue and Flow States}}  
First Published: 2025 April 23

\section{Draft 1}

Today I'd like to explore the difference between fugue and flow.  
Flow states are apparently great, but a quick google search also tells me that ADHD and autism havers often have trouble differentiating flow from hyperfocus.  
I assume that I use fugue like they use hyperfocus, and even if neither explicit diagnosis applies to me, I think some of the experiences are shared.

One article seems to say that the difference is in intensity, where a flow state has you still aware of the outside world.  
However, the author then goes on to describe the difference in a cleaning example.  
The hyperfocused person will clean long past the normal level of cleanliness.  
It did have the great advice of only picking rest activities which are short term.\footnote{such as spending time in the winter air}

It seems also as though the site is opposed to hyperfocus because it is directed towards something unhelpful, like youtube.  
Apparently I should set break times and follow the mandatory breaks.

Oh cool! Next article then says that there's academic research claiming that the difference is all in framing.

So, now that we've done our small literature review, let's get on to what I think about them.

Flow is something that's allegedly really desireable.  
My family immediately fell in love with the concept, which I think is only in part because the originator of the concept is a professor at my parents' alma mater.  
In general, it seems like I've been running into the term more and more often in the years since the pandemic took over.\footnote{It's so strange to me that our book club existed before the pandemic, because it feels like it just started, even realizing that we haven't done it in more than a calendar year. Anyways}  
Many are now writing and speaking about how one consequence of our low focus world is that we are less and less able to fall into flow states.

And so, it's interesting to me that states of fugue, or hyperfocus, depending on the nomenclature one uses, are considered negative while flow states are positive.  
There's tons of literature\footnote{if you print it out, I assume that's literally true} on flow, and it's been the object of a lot of research.  
Fugue, on the other hand, has a number of meanings.  
Most often, it refers to a very intricate and rigid musical form.  
However, it can also be used to refer to a state where one forgets who and where they are and sets out wandering.  
Finally, we have the definition I tend to imply, which is an almost dreamlike state of consciousness.

What do I mean by that?

When I am in the midst of a dream, I cannot tell you the passage of linear time.  
What few clocks I remember are never in sync with the real world.\footnote{even though I'm getting better at guessing the time when I wake up, which is weird}  
When I start working on something, I also lose track of time.

In the peak of my fuguing, It was difficult for someone to get my attention.  
For better and worse, I am now distractable enough that I can be broken from the task at hand.

When is it for the better?

I have many obligations in life, and being called to them is good.  
Sometimes my body has needs that I'm ignoring, and being forced out of the state makes me aware of the fact.  
And, finally, sometimes the fugue is not helpful.

Many can relate to the experience of scrolling for hours, not out of any real interest, but simply because that's what's happening.  
In what feels similar, I can fall down the hole of fixing issues of issues of issues.  
As an example, I think that I spent forty minutes one day trying to figure out how to download an unlisted tex package.  
Why?

It had a way of formatting a specific equation that was unique to it.  
Why did I need that?

I was trying to copy a derivation from a paper.  
Why was I doing that?

I was trying to see if my project could leverage some other mathematical concepts.  
Why?

Honestly at this point we just get to the perfectionism inherent to me.

So, I guess the moral here is that I will call it flow when it serves me and fugue when it does not.  
I have to imagine that there's a connection between the fugue state and musical form, because it does truly feel like I would need to be in one to write one.  
Who can say, though?

\section{Daily Notes}

\begin{itemize}

\item Obligations:

\begin{itemize}

\item Professional

\begin{itemize}

\item Leave work before 1900

Woo! I left at approximately 1730, but am now at a journal club, where I've signed up to write an article for the summer.

\item Write the thesis

Spent a good few hours on it today! Woo.

Since my boss told me today that she would like to see something as soon as it's ready, I do think that I should focus on a lower hanging fruit, so writing part of the thesis which doesn't actively require me to do large swaths of derivations is probably best for me.  
Or, I suppose that equally valid would simply be not doing the derivations right now and having them come into a later draft.

Let's stet that as the next writing goal: any time that there is something which requires me to pause for math, I skip it, until I have at least a solid and working set of text that I can add the math to later.\footnote{Ah gotta love how effective reflection is}

\item Revise the thesis

Also somewhat! Realized that some of the things I do in my program are different now and so changed them to be more accurate. Also realized that I have some math that is not necessarily accurate, which I don't so much love.

\item Edit the thesis

\item Research for the thesis

Woo! Spent a good few hours on this today, and made a presentation to give to the group on Friday, where I will hopefully be able to better highlight what things I do and do not know right now.

\item Read the books that might be useful for the thesis

\item Start citation tracking

Document continues to grow, I continue to not annotate it.

\end{itemize}

\item Personal

\begin{itemize}

\item Learn the songs for to jam

Strummed my guitar for a few moments this morning.  
I really need to get back into things sooner than later.

\end{itemize}

\item Self:

\begin{itemize}

\item Silence

I found the silence painful at points today, so good on me!

\item Typing practice.

Did some yesterday, will be doing more now.

Woo! Just got all of my letters above 3.5 characters per second, which means that now the goal is to get all of them above 4 per second.  
In time I will become a good typist.  
Unfortunately, the part of me which is opposed to cheating by just restarting lessons just lost me the one hundred and fifty seven lesson streak of over 95 percent accuracy.

\item Keep the phone out of the room for bed

I did that today! It was great, and I actually felt rested in the morning.

\item Pray St. Michael Chaplet in the morning

No, sadly.

\item Stretch in the morning

I did this morning! It was great, and I felt better today for having done so.

\item Read at night

I did not yesterday, I don't think?  
I don't entirely recall, though.

\item Poetry at night

SHOOT! Tonight it will! I will remind myself.

\item Clean the home

Small steps forward yesterday and today!

\item Stretching, standing, drinking water

I got notified multiple times today that I had been stationary and sedentary for an entire hour, so clearly not.  
I also don't think that I needed to refill my water bottle, which also suggests that I did not drink water appropriately.

\item Posture

Generally decent, though I did see myself in a window with incredibly slouched shoulders.

\item No wasted time

I think generally doing an ok job of this!

\item Eat more than 2 meals a day

I did that yesterday I think!  
After posting the post I ate more.

This day, however, I had breakfast? I think? Don't entirely recall.

I did eat lunch, and was invited somewhere for dinner, and so had to eat.  
Tragically this means that I did miss my weekly Wednesday appointment, but it's probably good for me.

\end{itemize}

\end{itemize}

\item Goals and Growth:

\begin{itemize}

\item Ends:

\begin{itemize}

\item Letter writing, get into more

I wrote two letters yesterday! I posted one of them and forgot about the other.\footnote{sorry friend}

\item Handwriting, pick and make the new one

I still don't entirely know how I feel about lower case letters, but wow.

\end{itemize}

\item Means:

\begin{itemize}

\item Typing speed, improve it.

Finally broke 3.5 on everything!!

\item Reading, do more of it

I've been listening to the audiobook, and so will be moving on from that hopefully.

\item Blogging, do it

Look at this! Twoish for two! I'm going to post this one before night's end, even if it does end up being half baked, because I hope that it will encourage me to do more returning to older posts.

\item Writing things that are not the blog and thesis, do

I don't think that I've done really any of this!  
I guess the letters, and the morning journalling that I do also probably helps.

\end{itemize}

\end{itemize}

\end{itemize}

\end{document}