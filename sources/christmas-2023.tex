\documentclass[12pt]{article}[titlepage]
\newcommand{\say}[1]{``#1''}
\newcommand{\nsay}[1]{`#1'}
\usepackage{endnotes}
\newcommand{\1}{\={a}}
\newcommand{\2}{\={e}}
\newcommand{\3}{\={\i}}
\newcommand{\4}{\=o}
\newcommand{\5}{\=u}
\newcommand{\6}{\={A}}
\newcommand{\B}{\backslash{}}
\renewcommand{\,}{\textsuperscript{,}}
\usepackage{setspace}
\usepackage{tipa}
\usepackage{hyperref}
\begin{document}
\doublespacing
\section{\href{christmas-2023.html}{Christmas 2023}}
First Published: 2023 December 25
\section{Draft 2}
This is now \href{christmas-2018.html}{my} \href{christmas-2022.html}{third} musing on Christmas.
As with the other two, I'm publishing it on Christmas Day.\footnote{well, the 25 of December, at least. Whether I wrote them before the morning or after night began is up to consideration}

My previous musings about Christmas have been fairly short.
The first\footnote{2018} was four sentences.
Half of those were less than four words a piece.\footnote{fewer than? in theory I know that words can be counted}
The second was a little longer, but focused almost exclusively on the liturgical readings.

Today, I'd like to focus a little on the traditions that my family has kept.
As I grow older, I find that I've been thinking a lot more about what parts of my life I'd like to keep forever, what parts I wish never happened, and what I think served its purpose and is no longer needed.

Our Christmas mornings tend to start with cinnamon rolls.
This past Thanksgiving, my little brother\footnote{hmm is this too much information to have publicly available? the existence of presumably two brothers} and I realized that we both had fond memories of orange rolls as children.
We still are not entirely certain which family member we had the rolls for,\footnote{though we have some good ideas} but we do know why we stopped having them.
In short, none of the extant adults in my family like them.\footnote{adults means generations older than me. Yes, I do know that I'm an adult in basically any sense that people use. However}
I did not remember that, however, and got orange rolls for the morning after Thanksgiving.\footnote{another tradition we have as a family}
My little brother and I loved them, and so this year's Christmas morning breakfast also included orange rolls.\footnote{which I did promise to bring to a friend. I need to make sure to do that}

Moving slightly back in the chronology, we get matching pajamas as a family every year. 
We open them on Christmas Eve, and then we all wear them on Christmas day.
A related tradition: the children take a photo with Santa every year.
Once we reached an age that it stopped seeming reasonable to go to the Santa in town,\footnote{ok, if you look at the photos, a few years after that point} we started taking them as a family on Christmas morning.
We've now merged the two traditions, and it's really fun getting to look back and see the siblings in matching clothes.

We do a gift exchange after the photos\footnote{there's a few more photos that we take. I enjoy them, but would not be heart broken if they ceased}, and then we make a quick breakfast and eat it.
After that, we play a board game.\footnote{most years, there is a new board game in someone's gifts. In the rest, we play a family favorites}
By that point, we're all tired of each other\footnote{in the \say{everyone is an introvert} way, not the other ways}, so we tend to be free until dinner.
For some reason, I have no memories of what we historically eat for Christmas dinner.

All in all, I love the fact that my family's Christmas is what it is.
I'm sure that at least some portion of my attachment to the traditions comes from the fact that they're the traditions that we have.
If I had grown with others, I am sure that I would have loved them just as much.\footnote{though with the family that I have, I feel like some of the traditions we have are the carcinization equivalent}

Daily Reflection:
\begin{itemize}
\item Hobbies:
\begin{itemize}
\item Did I embroider today? I plotted out the next embroidery project I want to work on. Unfortunately, I did not start on it until 2100, and that is far too late to start embroidering for the day. Maybe tomorrow.
\item Did I play guitar today? Just a touch! But I did enjoy the amount that I did.
\item Did I practice touch typing today? I made it all the way to Q! It bothers me more than a little that the last letter we get is J, but as soon as I make it through the Q, I'm through all the letters at at least 50 wpm, which means that it becomes time to reset the goal with much higher wpm. Probably closer to 70, since that really is my goal speed. Then again, it isn't really like I need to be able to type q at 70 wpm, since most of the time I won't need it. B is the big issue letter today\footnote{allegedly}, so at least that's progress, since C no longer is a struggle for me I guess. 

Update a few minutes later, I've officially gotten the entire alphabet at 50 words per minute. Now it's time to up the goal to 75 and see what happens.
Ooh cool! It saves my progress, so now I just had to restart with A, and then go to Y. That's pretty fun.
I am wondering whether it might be in my best interest to set the intermediate goal a little lower.
Idk.
\end{itemize}
\item Reading
\begin{itemize}
\item Have I made progress on my Currently Reading Shelf? Nope! But that's ok! I spent the day with my family instead.
\item Did I read the book on craft? Shoot! Hopefully tomorrow.
\item Have I read the library books? Not even a little. Let's see if we cannot do more of that next month in particular. I feel like my schedule doesn't really exist when I'm at home.
\end{itemize}
\item Writing
\begin{itemize}
\item Did I write a sonnet? I wrote one yesterday, and I'm going to take today off from writing one, since it's very late out, and I would rather sleep.
\item Did I blog? Another day where I'm not thrilled with the quality, but that's life, I suppose.
\item Did I write ahead on Jeb? It's Christmas, so that's obviously not happening either.
\item Letter to friends? I sent all the Christmas notes, and I had some good conversations with friends.
\item Paper? It remains a holy day.
\end{itemize}
\item Wellness
\begin{itemize}
\item How well did I pray? Not great.
\item Did I spend my time well? I think so!
\item Did I stretch? No. Sad.
\item Did I exercise? Shockingly! A little, if only because there were some activities today.
\item Water? I drank water to the point that I no longer wanted it today! I'm very proud of myself.
\end{itemize}
\end{itemize}


\section{Draft 1}
This is now \href{christmas-2018.html}{my} \href{christmas-2022.html}{third} musing on Christmas.
As with the other two, I'm publishing it on Christmas Day.\footnote{well, the 25 of December, at least. Whether I wrote them before the morning or after night began is up to consideration}
Apparently both of my musings on Christmas have been fairly short.

My initial musing, published in the initial iteration of this blog, was four sentences, half of which were under four words.
My musing last year, being a combined Christmas and Reflection on Readings, was a little longer, and focused more on the liturgical aspect of the season.

This year, however, I am writing this musing early enough that I have time to actually think,\footnote{unlike 2018} and not on a Sunday.\footnote{unlike last year}
As a result, I find that I have more time to consider what I want to muse about.

So, what do I want to muse about?
The book on the craft of writing suggests that it's best to plan to spend half the time you have allotted to an essay on planning and half on actually writing.
I don't think that will be a part of my life for a while, but let's try some of what they suggest.

One piece of advice was to do some free association.
Christmas to Christmas movies to Scrooge.
Christmas, advent, awaiting, 
Christmas, stockings, family, loved ones.

OK my mind doesn't do free word association well right now.\footnote{might be something that's worth thinking about sometime in the future}

What are things that I did today?

Texted wishes for a happy Christmas to friends\footnote{initially kith, which is obsolete except for kith and kin, which is like aid and abet. It means those familiar, and is same as couth, where we get uncouth from. Wow I love linguistic drift. I wonder how much the spelling versions of linguistic drift will stop in this literate, digital, and prescriptive era. Then again, we've had radio and audio recordings for a while and we still have shift of pronunciations and meanings, so it probably won't prevent it. I can't think of any words that have changed spelling in my lifetime, though. Maybe a few double words got broken, but that's about it (a la alright versus all right)}, and had some nice digital conversations with them.
I spent time with family, which was lovely as always!\footnote{we got the coziest set of matching pajamas ever! (note that coziness is comfort related, and so is actually somewhat negatively correlated with how warm the pajamas are, at least in this household}
In particular, we played games as a family, had some meals together, and did a small gift exchange.

I've started writing this musing, which is something.
I honestly think that may be literally everything that I've done today.
Ok so maybe musing about today as a thing isn't a great idea.
My family's reaction to me saying who all I was texting today makes me think a musing about that might not be welcome either.

That's really it for today.
So, let's think about what Christmas in general means to me?
Let's see what that sparks.
I guess the question is if I want to focus on the cultural or religious connotations that Christmas has for me.\footnote{yes, I know, I should not have divided parts of me, just me. However, I live in 2023, which means that the secular and the sacred are fundamentally separated for nearly everyone}

Having now made, eaten, and cleaned up from dinner, there's probably something I can muse about combining my post about family recipes, the fact that we only use red potatoes for mashed potatoes, and the fact that no one on the internet will say they're meant for that.
However, I don't really know if I can connect that to Christmas, especially today.
It might be a good idea to think more about it for future musings about family recipes.

Ok so let's try to make a coherent draft.
I still haven't figured out what I want to muse about.
The writing book says that you should spend a few days on this process, but acknowledges that you may not always have the option to do so.

Christmas.
What's it about?
No that feels too trite and over done.

Christmas.
Fears that I have related to it?
Feels too personal to this blog.

Christmas.
Why is it so hot out today?
Could be good.

Christmas.
Changes?

Hmmm.
What can we talk about for changes?

There's the obvious, that the weather is changing as the climate gets worse.
There's the way that family traditions change.
There's the voice in my head that says this might be the last Christmas we have together.\footnote{which will not be included, because still too personal}
There's the fact that life hasn't really returned to a stasis since Covid?
Idk let's see what happens if we just go like that.
\end{document}