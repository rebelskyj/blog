\documentclass[12pt]{article}[titlepage]
\newcommand{\say}[1]{``#1''}
\newcommand{\nsay}[1]{`#1'}
\usepackage{endnotes}
\newcommand{\B}{\backslash{}}
\renewcommand{\,}{\textsuperscript{,}}
\usepackage{setspace}
\usepackage{tipa}
\usepackage{hyperref}
\begin{document}
\doublespacing
\section{\href{pathfinder-0-1.html}{Pathfinder Character Backstory}}
First Published: 2022 March 16

\section{Draft 3: Short story for Salvia 17 March 2022 edited for consistency with the campagin}
Salvia finished dressing for the day and double checked that her band was still well-woven and on her wrist.
Nothing had come loose since she last checked, but it would not do to be anything less than perfect today.
It was the day of her thesis defense, when she could finally call herself a Wizard and graduate of Eriontha Academy.

She had walked out the door before realizing she forgot her Thesis project, the first staff carved in Gnomic.
When she went back in to take the Staff, she realized that her familiar Burrus had already grabbed it for her.
Of all the seminars she took, Familiar Magic was definitely the one that was the most fun.
She'd always had a connection to burrowing creatures, so it was far easier for her to bond to a fox than most of her classmates.
Some people called her crazy, but she knew that the beasts talked to her.

As she looked at the sun, she realized that she was running late.
She started to run to the lecture hall, where she hoped that her presentation would go well.

\say{Thank you for coming today.
I will be presenting my Thesis: \nsay{Advances in Bacula Magorum through integration with Gnomic Lineae Tradition}.
Following my presentation, I will be accepting questions}
It was a needlessly opaque title.
Truly all she'd done was use the Gnomes' written language of woven thread to enforce the language of Magic.
While she loved her staff, she knew that the day-to-day spell weaving was a much more useful thing to the general community.
Rather than carry around scrolls upon scrolls of Spells, which needed to be prepared daily, she had meta-knots.
She knew how to stabilize the knots so that she could make a Spell Scroll, but the meta-stable versions were far easier.
Of course, it did require the caster to know Gnomish, but that seemed reasonable to her.

Thankfully, she was able to field every question her committee threw at her.
She'd chosen them carefully, since a lot of the faculty was very opposed to doing any sort of Arcane Magic in any language but Draconic.
Now came the fun part: the party.
Her entire community had pitched in to throw a party for her to celebrate her graduation.

The non-Gnomes started to leave quickly, and she started seeing Gnomes she'd never met before.

\say{Hello cousin,} she said, reaching out to shake hands with someone her age that she didn't know.
They read the lineages strung around the other's wrist.
She noticed that he was single, but shared a sixth generation ancestor with her.

\say{Hello, Cousin,} he replied.
Truthfully, that was the most difficult part of translating between Gnomish and Draconic.
Draconic had words for every relation you might have, sure, but no one used them.
The inflection used on \say{cousin} in Gnomish could mean anything from first cousin to unrelated Gnome, and it was hard to teach that differentiation to non-Gnomes.

Really, that went back to the Weaving, though.
The pride of every Gnomish household was the tapestry showing their ancestry.
Each sibling older than their ancestor was recorded, going back to the founders of the community.
Each community maintained a tapestry which went from the founders of the community back to the founders of the community they had come from.

It was said that Gnomes could once trace their ancestry back to the First Fourteen Gnomes, but the tapestry, along with their temple, had burned down a thousand years ago.
Now Gnomes knew they were related, though often not how.
Most families maintained the tradition of a bracelet which had the names of your parents, the opposite sexed parent's parents, and so on back six generations.
It did mean that her brother's lineage would look far different than hers, if she had one, but that was just how Gnomish Culture worked.

Still, since Gnomes recognized that the Sorus Empire didn't understand that naming convention, and since it was unwilling to have siblings not share last names, the Gnomes and the Empire came to an agreement.
All Gnomes officially had the last name Gnomeson.
Since there weren't many of them, it didn't make non-Gnomes' lives much harder, since they tended not to reuse names.
And, Gnomes didn't need to speak to each other to understand what their real names were, since they literally wore them on their sleeves.

\say{So what are you going to do now that you're an official Wizard?} her family asked.
Her family had warned her about the Fading since she was a child, and right now she was expected to finally tell the community what her Quest would be.

Gnomish Quests were very different than the Empire's idea of a quest.
They were simply a way for the idealistic youth to go and explore.
Her parents, Chithara and Chitharus, had both tried to walk the full border of the Empire, trading songs while they did.
They met along the path, and the Quest was quickly abandoned in favor of them learning to make music with each other.

Salvia had chosen the standard Gnomish Quest, and was quite happy about it.
\say{I'm going to learn unknown truths.}
After all, Arcane Magic in Gnomish was a form of unknown truth.
The more that Salvia had reflected on it, the more she realized that it was what her heart called her to do.



\section{Draft 2: Short story for Salvia 17 March 2022}
Salvia finished dressing for the day and double checked that her band was still well-woven and on her wrist.
Nothing had come loose since she last checked, but it would not do to be anything less than perfect today.
It was the day of her thesis defense, when she could finally call herself a Wizard and graduate of Eriontha Academy, the first Gnome to do so.

She had walked out the door before realizing she forgot her Thesis project, the first staff carved in Gnomic.
When she went back in to take the Staff, she realized that her familiar Burrus had already grabbed it for her.
Of all the seminars she took, Familiar Magic was definitely the one that was the most fun.
She'd always had a connection to burrowing creatures, so it was far easier for her to bond to a fox than most of her classmates.
Some people called her crazy, but she knew that the beasts talked to her.

As she looked at the sun, she realized that she was running late.
She started to run to the lecture hall, where she hoped that her presentation would go well.

\say{Thank you for coming today.
I will be presenting my Thesis: \nsay{Advances in Bacula Magorum through integration with Gnomic Lineae Tradition}.
Following my presentation, I will be accepting questions}
It was a needlessly opaque title.
Truly all she'd done was use the Gnomes' written language of woven thread to enforce the language of Magic.
While she loved her staff, she knew that the day-to-day spell weaving was a much more useful thing to the general community.
Rather than carry around scrolls upon scrolls of Spells, which needed to be prepared daily, she had meta-knots.
She knew how to stabilize the knots so that she could make a Spell Scroll, but the meta-stable versions were far easier.
Of course, it did require the caster to know Gnomish, but that seemed reasonable to her.

Thankfully, she was able to field every question her committee threw at her.
She'd chosen them carefully, since a lot of the faculty was very opposed to doing any sort of Arcane Magic in any language but Draconic.
Now came the fun part: the party.
Her entire community had pitched in to throw a party for her to celebrate her graduation.

The non-Gnomes started to leave quickly, and she started seeing Gnomes she'd never met before.

\say{Hello cousin,} she said, reaching out to shake hands with someone her age that she didn't know.
They read the lineages strung around the other's wrist.
She noticed that he was single, but shared a sixth generation ancestor with her.

\say{Hello, Cousin,} he replied.
Truthfully, that was the most difficult part of translating between Gnomish and Draconic.
Draconic had words for every relation you might have, sure, but no one used them.
The inflection used on \say{cousin} in Gnomish could mean anything from first cousin to unrelated Gnome, and it was hard to teach that differentiation to non-Gnomes.

Really, that went back to the Weaving, though.
The pride of every Gnomish household was the tapestry showing their ancestry.
Each sibling older than their ancestor was recorded, going back to the founders of the community.
Each community maintained a tapestry which went from the founders of the community back to the founders of the community they had come from.

It was said that Gnomes could once trace their ancestry back to the First Fourteen Gnomes, but the tapestry, along with their temple, had burned down millennia ago.
Now Gnomes knew they were related, though often not how.
Most families maintained the tradition of a bracelet which had the names of your parents, the opposite sexed parent's parents, and so on back six generations.
It did mean that her brother's lineage would look far different than hers, if she had one, but that was just how Gnomish Culture worked.

Still, since Gnomes recognized that the Sorus Empire didn't understand that naming convention, and since it was unwilling to have siblings not share last names, the Gnomes and the Empire came to an agreement.
All Gnomes officially had the last name Gnomeson.
Since there weren't many of them, it didn't make non-Gnomes' lives much harder, since they tended not to reuse names.
And, Gnomes didn't need to speak to each other to understand what their real names were, since they literally wore them on their sleeves.

\say{So what are you going to do now that you're an official Wizard?} her family asked.
Her family had warned her about the Fading since she was a child, and right now she was expected to finally tell the community what her Quest would be.

Gnomish Quests were very different than the Empire's idea of a quest.
They were simply a way for the idealistic youth to go and explore.
Her parents, Chithara and Chitharus, had both tried to walk the full border of the Empire, trading songs while they did.
They met along the path, and the Quest was quickly abandoned in favor of them learning to make music with each other.

Salvia had chosen the standard Gnomish Quest, and was quite happy about it.
\say{I'm going to learn unknown truths.}
After all, Arcane Magic in Gnomish was a form of unknown truth.
The more that Salvia had reflected on it, the more she realized that it was what her heart called her to do.


\section{Draft 1}
I wrote a basic background history for my character, which is below.

Salvia Gnomeson is a gnome.
Gnomes don't believe in last names, so whenever you need one they just all go by Gnomeson because they're children of gnomes.
She grew up in an expatriated community in a poorer part of the capital.
Spent most of her childhood translating the traditional woven stories of Gnomish into Common and Draconic and legal documents from Draconic into Gnomish.

She was always touched by the wild, but never was able to fully turn it into a true sorcerous or druidic talent.
Instead, was always studious, so went to Wizard school.

While in school, refused to specialize, since her goal was moreso converting spells into woven expressions in Gnomish than actually learning magic.
Managed to learn how to save a spell as something you can cast once (scrolls), and a temporary version which lasts a day or so (vancian preparations).
Instead of remembering how to do that in particular, realized you can write directions for weaving the spells as a knot, though that takes some valuable threads to prevent fraying/because magic fiat. (costs 1 cp per level of spell).
Since 10 cantrips and 6 spells, 91 cp of fine thread left.

While researching, had a thesis idea: permanently woven spells that can be activated at will.
She was reminded of staves, so worked to make staff using Gnomish language rather than the traditional one.
Currently doesn't work great, but has a permanent cantrip and a semi-permanent lvl1 spell that can be refreshed.

\end{document}