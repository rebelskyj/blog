\documentclass[12pt]{article}[titlepage]
\newcommand{\say}[1]{``#1''}
\newcommand{\nsay}[1]{`#1'}
\usepackage{endnotes}
\newcommand{\1}{\={a}}
\newcommand{\2}{\={e}}
\newcommand{\3}{\={\i}}
\newcommand{\4}{\=o}
\newcommand{\5}{\=u}
\newcommand{\6}{\={A}}
\newcommand{\B}{\backslash{}}
\renewcommand{\,}{\textsuperscript{,}}
\usepackage{setspace}
\usepackage{tipa}
\usepackage{hyperref}
\begin{document}
\doublespacing
\section{\href{basic-flatbread-recipe.html}{Basic Flatbread Recipe}}
\section{Draft 2}
As I mentioned in \href{basic-bread-recipe.html}{a previous post}, I bake bread.\footnote{or at least, I used to bake bread and still know how to}
In response to that post, a reader asked about other kinds of bread I bake.\footnote{the actual comment was \say{Didn't you start making flatbread? Is that worth mentioning?}}
So, I thought today I would write about my flatbread recipe.\footnote{I also didn't have anything else on my mind}

All things considered, it's not much different than the basic bread recipe.\footnote{I don't really feel like I should link the same post multiple times(?)}\,\footnote{although arguably all things considered almost every (food) recipe isn't that different, since they alI give directions for how to prepare food, and compared to say, how to write a novel, they give more similar information. But I digress}
It also contains between 3 and four ingredients.\footnote{because yeast is optional after batch one (and even at batch one if you like to be risky)}
It follows below.
As before, it is adapted from \href{https://artisanbreadinfive.com}{this website's book}.

Basic Flatbread Recipe:

Ingredients:\\
5 pounds flour.\footnote{the reasoning is given in the first bread post}\\
10 cups water.\footnote{astute readers may notice that there is more water here}\\
2 heapingish tablespoons salt.\footnote{I like my flatbread a little saltier than my regular bread, but that's just me}
2 heaping tablespoons yeast.\footnote{or not}

Dough Directions:\\
Mix all ingredients together.\\
Cover, but not airtight.\\
Let sit until at least doubled.\\
Optionally: refrigerate at any point after mixing.
This delays souring.\\
Once risen, I tend to pick up and drop the container holding the dough to deflate it, but that's mostly just because it's fun to watch.
I have no clue what the effects are on the dough.\\
Cover airtight.

Baking Basics:\\
Flour the dough.\\
Pull out volume of dough desired.\\
Using plenty of flour, stretch until at desired thinness.\\
Note: If it begins shrinking, let stand for a few minutes then try again.\\
Put into oven preheated to hottest temperature for a while.\footnote{I generally hope for thirty minutes, but any amount of time (including while it's heating) works}\\
Pull out when dry.\\
Serve.

\section{Draft 1}
As I mentioned in \href{basic-bread-recipe.html}{a previous post}, I bake bread.
Now, in response to that post, a reader asked about other kinds of bread I bake.\footnote{the actual comment was \say{Didn't you start making flatbread? Is that worth mentioning?}}
So, I thought today I would write about my flatbread recipe.

All things considered, it's not much different than the basic bread recipe.\footnote{I don't really feel like I should link the same post multiple times(?)}
It also contains between 3 and four ingredients.\footnote{because yeast is optional after batch one (and even at batch one if you like to be risky)}
It follows below.
As before, it is adapted.

Basic Flatbread Recipe:

Ingredients:\\
5 pounds flour.\footnote{the reasoning is given in the first bread post}\\
10 cups water.\\
2 heapingish tablespoons salt.\footnote{I like my flatbread a little saltier than my regular bread, but that's just me}
2 heaping tablespoons yeast.\footnote{or not}

Dough Directions:\\
Mix all ingredients together.\\
Cover, but not airtight.\\
Let sit until at least doubled.\\
Optionally: refrigerate at any point after mixing.
This delays souring.

Baking Basics:\\
Flour the dough.\\
Pull out volume of dough desired.\\
Using plenty of flour, stretch until at desired thinness.\\
Note: If it begins shrinking, let stand for a few minutes then try again.\\
Put into oven preheated to hottest temperature for a while.\footnote{I generally hope for thirty minutes, but any amount of time (including while it's heating) works}\\
Pull out when dry.\\
Serve immediately\footnote{or not}


\end{document}