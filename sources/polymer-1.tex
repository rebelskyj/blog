\documentclass[12pt]{article}[titlepage]
\newcommand{\say}[1]{``#1''}
\newcommand{\nsay}[1]{`#1'}
\usepackage{endnotes}
\newcommand{\1}{\={a}}
\newcommand{\2}{\={e}}
\newcommand{\3}{\={\i}}
\newcommand{\4}{\=o}
\newcommand{\5}{\=u}
\newcommand{\6}{\={A}}
\newcommand{\B}{\backslash{}}
\newcommand{\sub}[1]{\textsubscript{#1}}
\renewcommand{\,}{\textsuperscript{,}}
\usepackage{setspace}
\usepackage{tipa}
\usepackage{hyperref}
\begin{document}
\doublespacing
\section{\href{polymer-1.html}{Polymer Review Part 1}}
First Published 2018 November 28

Prereading Note: I'll apologize in advance for what will likely be most of my posts until the end of the semester.
Of all the classes I'm in, only one has a final, my class on polymers.
So as a way to review, I'll be describing different concepts here, if only so that I know that I know them.
\section{Draft 1}
Today, I'll be discussing the different ways of measuring the mass of a polymer.
As far as I remember, there are only four that are relevant to me for the final: M\sub{n}, M\sub{w}, M\sub{z}, and M\sub{e}.
I'll discuss each one in turn, first with how you calculate it, and then what it is used for.

So, first is M\sub{n}.
M\sub{n} is the number average molecular mass of a polymer.
As such, like M\sub{w} and M\sub{z}, it's only useful for thermoplastics.
To calculate M\sub{n}, you divide the mass of each polymer chain by its numerical representation, then sum all of those.
So, if you have 10 chains with a mass of 10, and 10 with a mass of 100, it's 10/2+100/2 = 55 for M\sub{n}.
Since M\sub{n}\footnote{as you'll learn later} is the molecular weight most concerned with the number of short chains, it's good for telling you about how a polymer will yield, especially once it begins to crack.

Next is M\sub{w}.
M\sub{w} is the weight average molecular mass, and is the most common/useful of them.
To calculate M\sub{w}, you divide the mass of each polymer chain by its mass representation, then sum all of those.
So, in the initial example, 10 chains of 10 and 10 chains of 100, you would get 100+1000=1100, and then 100/1100+1000/1100 = around 92.
So, we immediately see that M\sub{w} is more easily measured, because it's easier to measure the mass of a polymer than the number of chains.
M\sub{w} is most relevant for the processability of the polymer, especially its flow in the molten state.

M\sub{z} is weird.
Like M\sub{w}, it weights the mass, but it takes it the next step, and does the square of the mass.
M\sub{w} is relevant mostly in governing die swell, also known as melt-elasticity.

Finally, we have M\sub{e}.
M\sub{e} is the measure of the mass between entanglements.
Off the top of my head\footnote{and at a cursory glance at the textbook} I don't know how this is measured.
But, it's mostly useful for knowing how strong the polymer will be in the rubber phase.\footnote{the rubber phase I'll talk about probably tomorrow}

So, yeah.
There's four major ways to say the mass of a polymer.
Oh! 
Also, the difference between M\sub{n} and M\sub{w} is known as the Molecular Weight Distribution, MWD.
That tells you how much strain softening will occur.
A larger (broader) MWD will lead to more strain softening.

In conclusion, polymers are weird.
Hopefully by the end of these posts you'll understand why if you don't already.
\end{document}