\documentclass[12pt]{article}[titlepage]
\newcommand{\say}[1]{``#1''}
\newcommand{\nsay}[1]{`#1'}
\usepackage{endnotes}
\newcommand{\1}{\={a}}
\newcommand{\2}{\={e}}
\newcommand{\3}{\={\i}}
\newcommand{\4}{\=o}
\newcommand{\5}{\=u}
\newcommand{\6}{\={A}}
\newcommand{\B}{\backslash{}}
\renewcommand{\,}{\textsuperscript{,}}
\usepackage{setspace}
\usepackage{tipa}
\usepackage{hyperref}
\begin{document}
\doublespacing
\section{\href{creative-hobbies.html}{Creative Hobbies}}
First Published: 2022 January 17

\section{Draft 1.1}\footnote{a friend read and caught a bad turn of phrase, corrected}
I recently read a post that claimed everyone should have at least one hobby which creates.
Something about that really stuck with me over the past week or so, and I'm using today's post to try to explore why.\footnote{I've often found that simply writing things helps me to come to a conclusion}

First, I guess I should give the claim.
I believe that every person should have at least one hobby which involves two\footnote{2} components:
\begin{enumerate}
\item Creating something
\item Sharing that creation with other people
\end{enumerate}
Ideally, I would rather that the creation be something physical, though that bias may come from other sources,\footnote{e.g. you learn better when physically writing as opposed to typing, so maybe physically writing someone a letter is better than emailing??} but I'm not including it because I'm still working through how I feel about it.

Now I guess I should support the claim.
Disclaimer for readers\footnote{hypothetical readers of course}, I'm going to be entering into a lot of speculative theology here, please feel free to correct me if I am in error anywhere.
In the beginning of the Bible, we see the Lord make man in His image.
There are\footnote{I'm sure} countless tomes written about what it means that man exists in the image of the Most High.
Today, I'd like to posit that a\footnote{here's the place} way we are like Him is in our inherent drive to creation.

Before He creates man, the Lord begins by creating the universe.
He creates all the living things which populate the earth, the stars in their courses, and the laws which govern nature.
And it is all beautiful.

While we cannot create planets and stars, we can still create beauty, and so show ourselves in reflection of the Almighty.
Now we must move a few verses ahead to the creation of man.
The Lord makes man for\footnote{based on either I've been taught, think, or is true, I am unsure at this moment} at least\footnote{weasel words am I right} one reason, to share in the beauty of His creation.
So too, we act in the likeness of the Lord when we share things we have created with each other.

Moving from the Genesis/religious argument, I think hobbies which create have an inherent goodness to them due to their adding of the sum beauty in the world.
The worst written poem still has some joy within it, and a burnt pie may still taste ok.\footnote{especially if you avoid the burnt bits}
Too, giving to others means that you don't end up creating only for yourself, as our lives aren't meant for ourselves.
We're meant to share all that we are to lead all people to the Lord.\footnote{oops, guess religious wasn't over}
Our Lord is the G-d of beauty and of love, and so any time we help people to experience either we are helping them to know the Most High.

I guess I should give examples of creation-hobbies.
There are the obvious ones in hobbies which create\footnote{semi} permanent physical goods such as:
\begin{itemize}
\item Woodworking
\item Crochet
\item Knitting
\end{itemize}
There are hobbies which create\footnote{by design} impermanent physical goods such a:
\begin{itemize}
\item Cooking
\item Soap making
\item Baking
\end{itemize}
And there are hobbies which create intangibles, such as:
\begin{itemize}
\item Writing
\item Drawing
\item Music
\end{itemize}
To me, all of these are creation-hobbies.
Obviously, this is a non-exhaustive list, and I would not even stake a claim that my three categories are themselves exhaustive.
But, in the first category, you can create goods which others look at time and time again, reminding them that they are loved each time they do.
The second, though transient, can show that you were thinking of them and, in the first and third, share the most vital of human connections, food.
The third shares our mind and heart with others in the most direct way, which can make it the hardest to truly share.
But, all of them are wonderful and I obviously think sharing and creating more beauty is a good thing
\section{Draft 1}
I recently read a post that claimed everyone should have at least one hobby which creates.
Something about that really stuck with me over the past week or so, and I'm using today's post to try to explore why.\footnote{I've often found that simply writing things helps me to come to a conclusion}

First, I guess I should give the claim.
I believe that every person should have at least one hobby which involves two\footnote{2} components:
\begin{enumerate}
\item Creating something
\item Sharing that creation with other people
\end{enumerate}
Ideally, I would rather that the creation be something physical, though that bias may come from other sources,\footnote{e.g. you learn better when physically writing as opposed to typing, so maybe physically writing someone a letter is better than emailing??} but I'm not including it because I'm still working through how I feel about it.

Now I guess I should support the claim.
Disclaimer for readers\footnote{hypothetical readers of course}, I'm going to be entering into a lot of speculative theology here, please feel free to correct me if I am in error anywhere.
In the beginning of the Bible, we see the Lord make man in His image.
There are\footnote{I'm sure} countless tomes written about what it means that man exists in the image of the Most High.
Today, I'd like to posit that the way we are like Him is in our inherent drive to creation.

Before He creates man, the Lord begins by creating the universe.
He creates all the living things which populate the earth, the stars in their courses, and the laws which govern nature.
And it is all beautiful.

While we cannot create planets and stars, we can still create beauty, and so show ourselves in reflection of the Almighty.
Now we must move a few verses ahead to the creation of man.
The Lord makes man for\footnote{based on either I've been taught, think, or is true, I am unsure at this moment} at least\footnote{weasel words am I right} one reason, to share in the beauty of His creation.
So too, we act in the likeness of the Lord when we share things we have created with each other.

Moving from the Genesis/religious argument, I think hobbies which create have an inherent goodness to them due to their adding of the sum beauty in the world.
The worst written poem still has some joy within it, and a burnt pie may still taste ok.\footnote{especially if you avoid the burnt bits}
Too, giving to others means that you don't end up creating only for yourself, as our lives aren't meant for ourselves.
We're meant to share all that we are to lead all people to the Lord.\footnote{oops, guess religious wasn't over}
Our Lord is the G-d of beauty and of love, and so any time we help people to experience either we are helping them to know the Most High.

I guess I should give examples of creation-hobbies.
There are the obvious ones in hobbies which create\footnote{semi} permanent physical goods such as:
\begin{itemize}
\item Woodworking
\item Crochet
\item Knitting
\end{itemize}
There are hobbies which create\footnote{by design} impermanent physical goods such a:
\begin{itemize}
\item Cooking
\item Soap making
\item Baking
\end{itemize}
And there are hobbies which create intangibles, such as:
\begin{itemize}
\item Writing
\item Drawing
\item Music
\end{itemize}
To me, all of these are creation-hobbies.
Obviously, this is a non-exhaustive list, and I would not even stake a claim that my three categories are themselves exhaustive.
But, in the first category, you can create goods which others look at time and time again, reminding them that they are loved each time they do.
The second, though transient, can show that you were thinking of them and, in the first and third, share the most vital of human connections, food.
The third shares our mind and heart with others in the most direct way, which can make it the hardest to truly share.
But, all of them are wonderful and I obviously think sharing and creating more beauty is a good thing
\end{document}