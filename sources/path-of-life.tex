\documentclass[12pt]{article}  
\newcommand{\say}[1]{``#1''}  
\newcommand{\nsay}[1]{`#1'}  
\usepackage{endnotes}  
\newcommand{\B}{\backslash{}}  
\renewcommand{\,}{\textsuperscript{,}}  
\usepackage{setspace}  
\usepackage{tipa}  
\usepackage{hyperref}  
\begin{document}  
\doublespacing  
\section{\href{path-of-life.html}{On the Path of a Life}}  
First Published: 2025 April 7

\section{Draft 1: 7 April 2025}

I think a lot about the way that my life is and will be, especially these past few months.  
In part, I am absolutely sure that this is because dealing with a death, even one which is far from sudden, makes anyone think about the way that they're spending their own time on the earth.  
In part, I think it has something to do with the fact that I'm closing a chapter of my life.  
Before I turn to the next section, I would like to think that it will be a better one.\footnote{not that this stage (other than the obvious) was bad, just that I always want better for those I care about, and I care about myself.}

I remember watching a video a few years ago that really changed the way that I view a lot of the goals I set.  
I am well aware of the SMART\footnote{specific, measurable, achievable, relevant, and time bound} method, but I find that a lot of the things I want to do are fundamentally not either of the first, and as a result, hard to know the achievability or timeline for.  
When life doesn't work with a plan, one can either change the plan or the life.  
Since goal setting is to help me live, I want the goals to reflect my life, and not the other way around.

I think that the video was initially set about new year's resolutions, and it talked about how, at the end of the day\footnote{end of the year?}, most of these goals, even when framed as SMART ones, are not really the actual goal.  
A goal to stop eating sweets, for instance, is more likely the method to meet some real goal, such as improving health or managing blood sugar.  
As someone pointed out once\footnote{I don't remember where}, a measure stops being effective when it becomes a goal.  
We see this all the time, such as how standardized tests effectively became the curriculum taught in public schools.  
The goal of a standardized test was to see children's general knowledge.  
Instead, it becomes about how well a student has been prepared for the specific exam.

Also, especially with New Year's Resolutions\footnote{note the capitalization! Fancy.}, life circumstances change rapidly in a way that makes SMART goals bad.  
If I twist my ankle on the fifteenth with a goal of running daily, then I'll have to stop.  
If, instead, the goal was something broader, like improving my physical fitness, I can instead start rowing or swimming or anything else.  
In general, the author stated confidently, themes can be the answer to your life.

As I've grown, I find myself trusting common wisdom about change less and less.\footnote{partially because survivorship bias, partially because I am not the average person, partially because I find myself trusting other's knowledge less and less (wow I really need to do reason and revelation already}  
While I have not had much luck with themes, something he said was very impactful.

Butchering the metaphor, as is my right as an author, think of your life as a road.  
We cannot change the path we have taken.  
And, in general, any choice we make does not truly diverge us that much.\footnote{obviously like jumping out of a plane will impact the shape much more than like red tie versus blue tie for a day}  
However, each choice we make turns us slightly more into the kind of person who makes those kinds of choices.  
Every time we eat a salad, our brains are a little more tuned to eating salads.  
Every time that we do a kindness for no reason save itself, we become slightly kinder.

Is that a particularly deep or profound metaphor? Maybe not to the average reader.  
However, I find it far more helpful to me than thinking about goals.  
I do not really want to be a runner, so practicing running doesn't make a lot of sense.  
I do, however, want to be an older person who is described as \say{shockingly, almost horrifyingly spry}, and so want to practice the exercises that will both lead me to being spry today as well as lay a good foundation for me to age.

A habit is, by most definitions, something we do without conscious thought.  
That is, it is when we follow the twists in the road without consideration.  
A metaphor I used in my web novel was the river of fate.\footnote{I think fate is not real, but the normative Catholic stance is that only humanity (and some non-corporeal beings) have free will. It's not a large leap to say that this means the universe is predictable for any system without human connection, and I think that I've seen talks with that exact thesis. I don't know where I stand on the issue, generally thinking that we should err on the side of treating animals with compassion, as they too are G-d's creations, but}  
We are all riding rafts down a stream which never ends.  
Following the path set for you is not just easy, it is effortless.

Like any other source of flowing water, though, the more you wish to diverge from the main flow, the more effort it takes.  
Rerouting the river itself is nearly impossible, especially the wider and stronger the river is.  
The metaphor gets a bit strained here, because each action we take is making the riverbed ahead of us, so I suppose that a better analogy might be a river shrouded in mist.  
We get to decide what comes next, but rivers tend to flow a specific way.

All this to say, not making choices is the default way people go through the world.  
Better habits lead to better lives because choosing the good is difficult, not because of the word good, but because of the word choose.  
All choices take effort, and so the goal for me is less any individual good habit or virtuous thing and much more about shifting who I am to become the sort of person who does not have to think about doing good.  
Everything yearns to rest, and so making the good easier is the best thing that I can do.

Hmm, I'm not super happy with this, because I don't really think that I said anything that meaningful, and what shreds of meaning may have made their way in are obscured by the filler of my stream of consciousness.  
Still, it is also important to accept what is, rather than what I wish would be.

\section{Daily Notes}

The astute reader might note that this has changed. I don't think that spending twenty minutes a day on the daily goals is necessarily serving me well right now, so I tried to pare it down. I will likely pare down even further before the time is through

\begin{itemize}

\item Obligations:

\begin{itemize}

\item Professional

\begin{itemize}

\item Write the thesis

I tried to spend some time on this yesterday, but got far too into the musing.

\item Revise the thesis

See above.

\item Edit the thesis

\item Research for the thesis

See above

\item Read the books that might be useful for the thesis

\item Start citation tracking

\end{itemize}

\item Personal

\begin{itemize}

\item Learn the songs for to jam

\end{itemize}

\item Self:

\begin{itemize}

\item Typing practice.

Did my time today!

Averaged just under 6.2 characters a second, unlocked letters through \say{u}, and generally felt like I was doing ok.  
Interestingly, my average speed decreased through the lessons on almost every letter. Wonder what that's about. I don't get new letters until I'm at 5 characters per second on the previous letters, which I really like as a system, because 5 is just about where it stops feeling like something that I need to think about at all.  
There's something to be said for more incremental changes to be sure, especially since the default goal is less than three characters per second.  
Still, once I'm at 5 per second in all letters, I think that I'll start working on punctuation and capitalization.  
If I can get all of that together up to 5 per second, then it's on to six, and so on until I either give up, die, or max out the program.

\item Keep the phone out of the room for bed

Did not do this, and started a new audio book, which I am enjoying.  
I'm listening to it at single speed, and I'm more and more thinking that this publisher\footnote{is it a publisher for audio? I think so} records books with the assumption that they'll be listened to at greater than single times speed.  
Still, there's nothing that I really gain from listening faster, especially since there are no other books that I'm trying to consume right now.  
Learning to go slower is almost never a bad thing.

\item Pray St. Michael Chaplet in the morning

I did not do this today or yesterday, which isn't great.

\item Stretch in the morning

I did that today, though certainly not for an hour.  
I think that I should have time tonight to do so, though, which will hopefully help me.  
I found that I was incredibly tight in the front of my chest today, so I have confidence that stretching is doing something.\footnote{even if it's making other parts of me tighter rather than the parts I think I'm stretching lighter.}  
My back also seized when I bent over suddenly, which I don't like.  
Might be good to start some lower back stretches.\footnote{lifts?}

\item Read at night

Nope.

\item Poetry at night

Nope.

\item Clean the home

Spent time this morning! Realized, as I tend to, that one of the big issues in my life is storage.  
I've been keeping spices above the stove, but\footnote{obviously} that area gets fairly hot, which is not good for spice storage.  
It's really not good for most storage, but I think that I have enough things that are non-consumable that I can make it work.\footnote{oh duh, if I move dishes into there, at worst I end up with nice warm plates to serve my meals. Nice how thinking through things by typing helps me come to answers}

\item Stretching, standing, drinking water

I did kind of ok with this yesterday, other than the time period where I left my water at primary location and moved to secondary.  
Stretching, generally not as much as I'd like.

\item Posture

So I have realized that I instinctively cross my legs, which is weird, and a habit I think that I would like to break.  
I catch myself doing it so often, though.  
Even lying in bed, my right foot goes over the left.

\item No wasted time

Generally decently, although I did spend a while yesterday looking for pen ink, I enjoy writing enough that I'm willing to call that time decently, if not well, spent.

\item Eat more than 2 meals a day\footnote{meaning a snack and 2 meals is good or three meals, etc.}

I ate two and a bit yesterday!\footnote{rice and bratwurst and gravy for dinner, which was shockingly good. Gravy entirely because Japanese curry starts from a roux and I forgot how much I love both making and consuming roux based sauces. Lunch was a pot pie from the local gas station chain. Breakfast was a rice krispie treat and a protein bar.}

\end{itemize}

\end{itemize}

\item Goals and Growth:

\begin{itemize}

\item Ends:

\begin{itemize}

\item Letter writing, get into more

Did Not write yesterday, will not write today or likely tomorrow or the rest of the week.  
I need to lower the barrier to going to my jail cell in the library, or else I should just be honest with myself and take the letters elsewhere.  
Hard to decide either way.

\item Handwriting, pick and make the new one

In reading the book about gentlemanly etiquette, I was reminded that the primary goal of any hand\footnote{which I think isn't used as a term any longer. Maybe script} should be legibility.  
A secondary consideration is normally speed, but I think that my personal priority list puts that at the end of the list.  
For my new hand, I'm prioritizing legibility first, as anything I write I want to be read, and read easily.  
Next I have uniqueness.

I know that it's ridiculous to want my handwriting to be different just for the sake of being different, but I have a few justifications.  
First, it makes it slightly harder for someone to falsify something as having been written by me, even if that is unlikely.  
Second, it makes life more fun to look at something fun.  
It doesn't have to be unique in the sense of never seen, just in the sense of not commonly seen today.

The next priority is beauty.  
Beauty is not necessarily quantifiable, but for anything practical, it does have elements of both form and function.  
Anything which passes the first criterion\footnote{right? because it's the singular one there?} will satisfy the functional element.  
If it is also unique, I find it unlikely that I will not find the writing beautiful.  
However, if I have to choose any specific character's beauty or originality, it is good to remember the priorities I have set.

Finally, I want to consider speed.  
Although I do not ever really need to write quickly by hand, I am not hoping for calligraphy.  
My writing needs to be done at a writing pace, which means that, by and large, extraneous strokes and flourishes don't get to exist.  
Finding a hand that satisfies all of these criteria is difficult, to be sure, but most of the fun in getting a script is playing with options.  
I think I might go letter by letter, if only because that way I can really focus to make sure I'm able to reproduce the exact thing I want.

Also, cursive script is sadly out, because so many people these days cannot read it.  
Even though I can and think that more people should be able to, I want my writing to be able to be read by anyone.

Oh, I guess it's an unstated thing before this, but I want the handwriting to scale.  
That is, I want to be able to draw the same shapes on a small note, a notebook, and a whiteboard or blackboard and have it look fine, if not good.  
Why? Because why go through the effort of making beauty if I do not commit to it?

\end{itemize}

\item Means:

\begin{itemize}

\item Typing speed, improve it.

As mentioned above, I have been working on that.

\item Reading, do more of it

Yesterday, while spending time with the friend, we had a fair amount of downtime where conversation did not need to occur.  
I kept reading the book I started a few months ago when I was reading more, and I forgot how much I enjoy it.  
I do also want to reread one series, which I'm not letting myself do at least until I finish this book.\footnote{it's a really dystopian progression fantasy series, where the main character is poor and wants to cultivate, and ends up joining a horribly abusive fighting ring just to be able to pay off his debts.  
Not that I want to write more dystopic fiction (if anything, I want the reverse. I want writing to return to an era where we have hope. Society influences art and art influences society. I want to do my part to make both better), but it was enjoyable, and I'm curious if I'll still enjoy the themes on a second read through.}

\item Blogging, do it

Look at this! Of the seven days of the month, I think that I have done at least 5!

\item Writing things that are not the blog and thesis, do

I wrote some reflections about the reading before blogging yesterday, which somewhat counts, and I've done a bit of scheduling for things.  
I also spent most of the late morning and early afternoon helping my undergrad to write a script to do some data analysis.\footnote{would it have been faster by a lot and also easier for everyone if I had just written it? yes. Did the undergrad necessarily want to write the script? no. Does the undergrad need to know python? no, but I think that the more tools one has, generally the better.}

\end{itemize}

\end{itemize}

\end{itemize}
\end{document}