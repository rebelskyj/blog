\documentclass[12pt]{article}[titlepage]
\newcommand{\say}[1]{``#1''}
\newcommand{\nsay}[1]{`#1'}
\usepackage{endnotes}
\newcommand{\1}{\={a}}
\newcommand{\2}{\={e}}
\newcommand{\3}{\={\i}}
\newcommand{\4}{\=o}
\newcommand{\5}{\=u}
\newcommand{\6}{\={A}}
\newcommand{\B}{\backslash{}}
\renewcommand{\,}{\textsuperscript{,}}
\usepackage{setspace}
\usepackage{tipa}
\usepackage{hyperref}
\begin{document}
\doublespacing
\section{\href{finishing-a-blanket.html}{Finishing a Blanket}}
First Published: 2022 July 11

\section{Draft 1}
I was at a craft circle when I started a bag that I made last week.
Apparently my\footnote{incredibly meager} ability to crochet got people's attention, because one of the other members asked if I could help crochet one of their projects shut.

Another, later, messaged me to ask if I would be willing to finish a blanket that a dead friend of theirs started.
I still don't know how to respond to messages like that, but I went with \say{Yeah! I'd be happy to.} which may be too chipper in retrospect.

But, as it turns out, all the crocheting is done on the blanket.
The person who asked knew that the pieces were all done and just needed to be assembled, but thought that also required crochet.
Not so.
It requires stitching, whip stitches to be exact.

Finishing a join that the friend\footnote{presumably} had started was terrifying at the end, because the strip I joined was longer than the one I joined it too.
When I finished my second join and found that it was even longer I will admit to a slight panic.
But, I realized it's not totally my fault.

Of the five remaining panels to join\footnote{six already being joined}, each is a different length, with up to 6 inches difference.
Anyways, I'm having fun practicing my whip stitch and getting better at it!
\end{document}