\documentclass[12pt]{article}  
\newcommand{\say}[1]{``#1''}  
\newcommand{\nsay}[1]{`#1'}  
\usepackage{endnotes}  
\newcommand{\B}{\backslash{}}  
\renewcommand{\,}{\textsuperscript{,}}  
\usepackage{setspace}   
\usepackage{tipa}  
\usepackage{hyperref}  
\begin{document}  
\doublespacing  
\section{\href{kindness.html}{On Kindness}}  
First Published: 2025 September 15

\section{Draft 1: 15 September 2025}
Today in the Uber from my brother's apartment to the train station, the driver asked why I was in town.\footnote{side note: the number of people who just immediately assume business when they see me is wild.
The one time I think someone assumed I wasn't traveling for business lately was when I was doing an outreach talk (so actually doing business)}
When I said that I was visiting my brother, he replied \say{Oh, that's so kind of you.}
That made me think about the word kind.

It was clear pretty quickly that English was not his only mother tongue.\footnote{whether or not English was his mother tongue, not totally sure}
It's totally possible that kind is just the word he has for general sense of positive action.
However, I then started thinking about one of the conversations I had with my mother and family.
She expressed that she did not think of herself as nice, and the rest of my family disagreed.
I generally agreed with her, since and kind are different\footnote{or, as Andrew Lloyd Weber reminds us \say{nice is different than good}}.

In reading Brene\footnote{I don't believe in diacritics in English language writing, because I'm a monster in some regards} Brown's \say{Atlas of the Heart}, I was glad to see that she had the same opinion.
While I did not agree with all of her emotional definitions\footnote{especially the fact that she found anger to be a fundamentally negative emotion, while I've mostly seen it as neutral bordering on positive}, I did greatly agree with her distinction between kind and nice.
Kindness is helping others, not just by doing what you think that they need, but doing what they actually need.\footnote{which is what separates it from like pity or charity}
Niceness, by contrast, is not rocking the boat.

My mother was a small first-generation college-educated woman.
It is perhaps unsurprising, then, that she learned not to be nice very early on in her career.
If she was unwilling to rock the boat or be a nuisance, then there was no way that she would accomplish any of her professional goals.
She was more than willing to tell people when they were wrong, and was more than happy to stand up and disagree with others.

However, she tempered this general lack of niceness with the greatest amount of kindness I think I have ever seen in someone.
My entire life, old farmers would come up to us when we were in public to ask my mother for medical help.
She always made time for them.
One day I asked her why, and she explained that for many of them, not only is the act of going to the doctor something incredibly time consuming, but it is also an uncomfortable environment.\footnote{as I think a little more, unsure if this is a real memory or just something I synthesized}
By helping them with their questions when they came up, however, she was able to monitor them and make sure that they were doing ok.

Whenever someone collapsed in church,\footnote{which I'm told is not a normative experience for everyone} she was the first to respond.
At one of my football games, a player hurt his neck, and she leapt over the fence and onto the field almost before anyone else noticed.

So, then, other than gushing about how great my mother was, what's the point of these past paragraphs?

Honestly, not totally sure!

I've written a fair bit lately about the meaning of words, and so spending some time distinguishing kindness and niceness feels reasonable enough.
I think that this can honestly be a reasonable enough short musing.
Maybe it's just that I want to make sure that this post gets posted, and I know that I won't if I keep waiting.
Still, I'm starting a sixty\footnote{I think?} hour train ride, which means now might very well be a great time to start binding books again!

\section{Daily Reflection: 15 September 2025}

\begin{itemize}

\item Did you journal by hand today?

Not so much, if only because spending time with my brother often precludes that.

\item Did you do a folly?

Couple more days of failing to write. Sad enough.

\item Did you in some way, shape, or form advance the web novel?

As always, no.

\item Did you work on music, whether education or creation?

As always, no.
I did talk about music a bit yesterday, which was nice.

\item Did you work on book binding?

Still no.

\item Did you work on another hobby?

Spent time with my brother, went for some walks!

\item Did you stretch? Really?

Not really, a little in the shower.

\item Prayer?

Nope.

\item Meditation?

More or less no.

\item Reading?

Some, but not a lot.

\item Minimizing screen time?

Yes! 
Benefit of time with brother is that I spend less on the phone.
\end{itemize}

Current Pen List\footnote{for my own posterity, mostly}

\begin{itemize}  
\item Hongdian Black with Fude Nib: Diplomat Caribbean (8/30ish)  
\item Jinhao Shark: Diplomat Caribbean (8/30ish)  
\item Pilot Preppy: Private Reserve Electric DC Blue I think (I think since late june. I think)  
\item Sheaffer: Private Reserve Spearmint (since 7/15) (I Think)
\end{itemize}

\end{document}