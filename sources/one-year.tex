\documentclass[12pt]{article}  
\newcommand{\say}[1]{``#1''}  
\newcommand{\nsay}[1]{`#1'}  
\usepackage{endnotes}  
\newcommand{\B}{\backslash{}}  
\renewcommand{\,}{\textsuperscript{,}}  
\usepackage{setspace}   
\usepackage{tipa}  
\usepackage{hyperref}  
\begin{document}  
\doublespacing  
\section{\href{one-year.html}{One Year}}  
First Published: 2025 October 8

Content warning: horribly sad.

\section{Draft 1}

It's been one year.

What am I supposed to do with that?

I miss you.

I'm sad that you couldn't be there to watch me graduate.

I'm sad that I can't talk to you about my first few days at work.  
You would have laughed at how I forgot what meaningful use is.

The sunset on the lake was beautiful tonight; I'm sorry that you never got to sit on the dock and watch it like you wanted.

I've been generally doing well; I don't tear up when the fact that you died comes up any more.

I'm not doing well right now; tears are flowing.

There's a poem and accompanying textpost on tumblr about the way that men love.  
One that stuck out to me was the man who became a cobbler because his own mother was buried without shoes.

I am not an oncologist, and I could never be one; watching even one mother die like you did would break me forever.  
I will work with oncology; even though better charting wouldn't have saved you, it still feels like what I can do.

It's been one year.

It feels like it's been forever.

It feels like yesterday.

It's been one year.  
One fucking year.\footnote{I try really hard not to swear on this blog, and I think that this might actually be the first profanity}

So much still hurts.  
I don't use purple or pink any more.  
I can't listen to the Bible or Catechism in a Year; that was supposed to be something we did together.  
I can't write my web novel; that was what you said helped during chemo.

The death certificate says that it's been a year.  
Of course, the death certificate is right; the world exists as the bureaucracy says and no other way.  
Even though it's been a year, it's been more than a year since the last time you held me.  
It's been more than a year since the last words you spoke.

I want to think that the last thing you heard me say was \say{I love you}.  
For my sake, I hope that I got to say goodbye, even if I only meant for the night.

I think that part of me will always be stuck in bed, waking up to the sound of my brother knocking on my door.  
Even without him saying anything, I knew.

Part of me will always be stuck at the side of your deathbed, saying the Lux Aeterna, since I was the only one who had a prayer ready.

Part of me will be at your deathbed before you died, when you said that you didn't care about the funerary arrangements except that you wanted them to be Catholic.  
I haven't been able to sing City of God in almost a year.

It's been a year.

One friend who you never got to meet texted me today.  
I think you would have loved her.

It's been a year.

I didn't do as you asked; I graduated before [].\footnote{I never use identifying names for my family, but it feels wrong to say my brother here, when I'm referring to one in particular}  
I hope that you wouldn't be mad at me.  
It sucks so much that not only did I have a graduation that felt rushed, not only did I have to do it without you, but all the while I was doing it I knew deep in my heart and keenly in my mind that I was disobeying your last wishes.

Part of me is mad.

You gave me a task related to both of my brothers.  
I don't think that you gave them one related to me.

It's been a year.

I haven't had these great wracking sobs in months.  
I almost forgot what it felt like to have tears leaking down my face as each breath catches, regardless of what else is going on.

It's been a year.

I haven't had anyone message me or stop me in the street to say how much you meant to them in what feels like a decade; rationally I know it can't have been more than seven months.

It's been a year.

You told us not to stay at your bedside day in and day out.  
Does it make me a bad son to have listened?

I look at the last photos we took as a family, and I hardly recognize you.  
When the friend asked me if I remembered your face, I could only think of the photo you used for everything; I think that it was at least fifteen years old by the time that you retired it.

At the wake, I had to walk away and sit alone.  
Despite everyone knowing that I was doing that to be alone, I still had a number of people stop in.  
I scrolled through my voice messages, knowing that I had never deleted any.

It took five years of scrolling to find one from you.

I guess that I'm glad I always picked up.

I'm sad that, in the over a year we had planning for your death, I never had you send me a message simply saying \say{I love you}.

It's been a year.

The weekend I came back to Madison, or at least what feels like it, I went to a club with a friend for her birthday.  
While waiting outside the bathroom for a friend, I met another mother dying of cancer; she had no idea how to tell her children.  
I was able to tell her that I appreciated everything you did for us, and that I made it through.

It's been a year, and yet part of me still feels like I should have covered mirrors.  
It feels wrong to realize that I wore a white shirt today of all days.  
Today was too beautiful of a day to be a reminder that you're gone forever.

Today was far too lacking in beauty to be a reminder of the day your pain stopped.

It's been a year.

Anything else I have to say belongs to you, and you alone.

It's been a year.

I love you.

Goodbye again.

Goodbye forever.

It's been a year.

How do I still have smiles inside of me?

How do I tell the people around me that the reason I don't mention you when talking about what my family does isn't estrangement?  
I guess it's the most permanent form of estrangement, so I suppose they understand.

It's been a year.

maybe if I say that one more time, or one time more, it will tell my heart that it can heal.

it's been a year

three hundred and sixty five days

thirteen moons, I think.

one year

\end{document}