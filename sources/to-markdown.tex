\documentclass[12pt]{article}[titlepage]
\newcommand{\say}[1]{``#1''}
\newcommand{\nsay}[1]{`#1'}
\usepackage{endnotes}
\newcommand{\1}{\={a}}
\newcommand{\2}{\={e}}
\newcommand{\3}{\={\i}}
\newcommand{\4}{\=o}
\newcommand{\5}{\=u}
\newcommand{\6}{\={A}}
\newcommand{\B}{\backslash{}}
\renewcommand{\,}{\textsuperscript{,}}
\usepackage{setspace}
\usepackage{tipa}
\usepackage{hyperref}
\begin{document}
\doublespacing
\section{\href{to-markdown.html}{Moving to Markdown}}
First Published: 2023 November 28

\section{Draft 1}
Prereading note: Not only is this rambling, but it more or less is three blog posts stapled to each other, each of which was already rambling.
I apologize in advance to my readers.

I do not remember exactly why,\footnote{looking through the text conversations, I think that it was because a friend suggested that I could consider having a substack instead of my current blogging platform, which is, as best as I know, a heartily kludged together mess of so many different things (end result is LaTeX to HTML, with markdown and pandoc in the middle? I think? I don't really know how to make a makefile, which is such a shame, because wow it would make so much of my life easier. I only really know how to read the highest level of coding languages, and even then only vaguely. As I continue to do more and more computational research, that becomes more and more of an issue). I don't think I want a SubStack, though} but I was considering revisiting \href{footnote-frenzy.html}{my post about footnotes} today.
I spent a fairly long time revisiting many of the same paths that I trod a few years ago when I first set up this blog.\footnote{for those who don't know, my father has a blog of his own, which is apparently well enough known (a member of an affiliate department of mine with no relationship to him mentioned that they found his blog and started following it). One of the most famous parts of that blog (if you're reading, father, I said one) is its relationships with footnotes, which are nested aggressively and sometimes recursively}
In the end, I was unable to find any solutions to the issue of creating footnotes within footnotes.

Now, my readers may be interested in having nested footnotes themselves in LaTeX.\footnote{if you're confused about why this post is clearly a .html file when I said it's LaTeX, please read the first footnote (which I can't reference here, because the need to communicate between all the different programs means that you only get the options which work well between all of them. Nested footnotes, as you might be able to tell, are not considered essential to everyone, for all that House of Leaves and Terry Pratchett's works rely on them. Apparently translating editorial editions also requires them for actual academic purposes, which is fun and interesting), where I explain (not for the first time, I do not believe) that these musings are initially written in Tex before being compiled into HTML}
If so, there are a few options.
\begin{itemize}
\item Be better at writing, and find a solution that does not require having nested footnotes.
That seems to be the critical consensus among TeX users, which I find dissatisfying.
\item Use the bigfoot package.
\begin{itemize}
\item Pros: it lets you have arbitrarily deep footnote nests
\item Cons: Each nested footnote needs to be a layer deeper, which means that you can't have like footnote 1 ref 2 ref 3 all on the same level.
Given that the only legitimate\footnote{read: academic} use for nesting is editorializing on translations, that's probably fine for its use case.
\end{itemize}
\item use \textbackslash{}footnotemark within the footnote, and then \textbackslash{}footnotetext out of the footnote, which allows you to nest the footnotes.
If you use the hyperref package, the footnotes even reference each other like they should.
\item Write your own solution.
I'm sure that those who know how to use actual computer science would be able to write their own package, should they so choose.
\end{itemize}

Now, of these options, I personally find the penultimate to be the best for my use case.
I hopefully will not need to have nested footnotes in my actual thesis\footnote{which is likely to be the next major TeX file I use. I wonder if I can write my thesis in markdown (spoiler I guess). Hmm, there's a markdown package which allows for writing in markdown in LaTeX. That's fun and cool to learn}, but if I do, using footnotetext and footnotemark works well enough, and fixes most of my issues pretty quickly.
I don't like having layers of footnotes.

That's something that's worth investigating.
At a gut level, I feel like multiple layers claim that the point of a footnote is that it's less important than the main text.
I don't generally agree with that line of thought.
I'm bad at expressing thoughts right now,\footnote{this thought, at least, given that I've tried rewriting it at least four times (hmm that does sort of get rid of my whole \say{I do not edit while I write} claim that I had in the first iteration of the blog} but I generally think of the point of most prose, even academic writing, is to convey a narrative.
In an academic sense, then, footnotes are used to explain information which is essential to understand a claim but which is not directly essential to the narrative.
For instance, if I was discussing how advances in rotational spectroscopy\footnote{a technique where low energy quantized transitions of a molecule are measured} allow for better determination of molecular structure, the footnote there could have been nice for someone who might not know what rotational spectroscopy is.\footnote{this is getting meta, and that's kinda fun}
However, there might be information which is needed to understand the footnote.
In the case of my footnote above, it could be useful to explain what quantized transitions are.\footnote{normally reading my blog does not require the footnotes. That is really not the case here wow}
Since the footnote level is already for context clues, context to context is still a context clue.\footnote{I don't know if people agree with this take, but it resonates with me, which makes it true enough}

Anyways, since I cannot have nested footnotes in the current way I'm writing, what can I do instead?

The obvious answer is to ask my father.
I did, and learned\footnote{as an earlier footnote suggests} that he writes his posts in markdown.\footnote{I knew that at some point, but shoved the knowledge to the side of my mind, where knowledge goes to be lost}
Now, I was initially ready to immediately switch over to writing in markdown.

However, there is an issue.
As you might have guessed from the last item in the ways to have nested footnotes, I do not have a great grasp of coding best practices, or practices at all.
Readers may wonder how, exactly, I was able to go from TeX to HTML with an automated script, let alone set up an entire blog.
To that, I answer, my dear sibling was willing to make the blog set up for me.
Since the way I was writing 5 years ago\footnote{and, frankly, now} is primarily in TeX, that made the most sense.
To switch to markdown,\footnote{which does seem legitimately better for most of my daily use cases, if I'm being totally honest} I would need to change up some code in the make file.
I don't know how to do that, so I will need to beg and barter time from a family member who knows how to code, with hope that they will teach me.\footnote{beg and barter is an aggressive term. I just had to ask}

So, one thousand words into what was supposed to be a quick diversion, I suppose that I should start the actual text of this musing.
Maybe I'll just talk about markdown, which I'd like to learn, I think.
It seems to have the benefits of TeX\footnote{not being WYSIWYG, which more and more is something I prefer in my writing. I like the fact that all the text looks the same as I edit it, but looks nice when others see it} along with being more scalable\footnote{or whatever the word is for not having anything restricting me}.

Both are markup languages, which is apparently where the term markdown comes from.\footnote{computer scientists love puns too much}
Markup languages are just ways of turning plaintext into pretty text.
Plain text is\footnote{to a first order approximation} anything that you can actually type on your computer.
You will notice that there is no bold key on your computer keyboard.\footnote{if you have a keyboard that has a bold key, I assume you already know what plaintext is. As such, you shouldn't be reading this footnote or the actual text, and so I will not accept complaints}
In most word processors that my friends interact with,\footnote{I say friends not family, because my family all use various versions of markup languages} you have to click a button that bolds the text.
In a markup language, you would instead do something like (bold) this text (not bold).\footnote{I don't know of any that use this format, but wow that would be fun, if incredibly annoying}

Anyways, that is not where the similarities stop.
The main differences between the languages come from the use cases they were created for.
LaTeX was built to make TeX more useful.
TeX was built because Donald Knuth\footnote{the father of modern computer science} saw a reprint of one of his books and was fundamentally dissatisfied with how the equations were set on the page.
Rather than any of the other solutions one could come up with, he wrote a processor which engraves pretty pages.\footnote{it's this kind of dedication which explains a lot of modern computer science}
Given that he was a computer scientist\footnote{the computer scientist?} he made it Turing Complete.

What, exactly Turing Completeness is, is not relevant here.
The short answer is that if something is Turing Complete, you can make any program in it that you can make in another Turing Complete language.
In theory, this does mean that you can program Doom in LaTeX.\footnote{I'm so glad I'm not \href{https://tex.stackexchange.com/questions/649019/could-tex-run-doom}{the only one to ask this question} (tl;dr yes)}

Markdown, by contrast, was written to make writing HTML faster.
I vaguely remember hearing that HTML is not Turing Complete, which implies to me that Markdown would not be either.
However, as I have repeated constantly, I am not a coder.
This is not a relevant issue to me.

Since LaTeX was made for academic publishing usage, it is an incredibly powerful language.
By default, it auto-numbers footnotes, has support to automatically make citations in more or less any citation style you like, and so on.
When writing an academic paper, which often has a ridiculous number of revisions, number and locations of footnotes do not remain constant.
As such, Markdown, which requires directly specifying what footnote a reference refers to, does not satisfice\footnote{that's a fun word that my spellchecker doesn't like} for academic purposes.

Still, for a majority of the writing I do, I do not actually need to cite, or to have autorenumbering footnotes.
I also don't really use equations\footnote{though Markdown now supports LaTeX equations, which is wildly fun}
For the academic writing I do, LaTeX is still superior, so that's going to be what I stick with.
For this blog, and most of the fun writing I do, however\footnote{and, given how easy it is to interconvert file formats, even first drafts of the academic writing I do} Markdown might be the thing for me to switch to.

Eighteen hundred words in, let's learn Markdown.
It's a very simple language, which will be nice.

To make italicized text, markdown uses either asterisks or underscores on either side of the text. 

To make a link, I put the text I want to display in square brackets, followed by the link in parentheses.

To make an ordered list, I have a series of lines where the first begins with 1.\footnote{interestingly, after the number one, any number is allowed, and the list will automatically change to be proper ordering. That's another way that TeX is technically better, because one could, should they so desire, skip numbers in a tex list}
Indention makes nested lists.

To make unordered lists, it's the same as an ordered list, though with dashes, asterisks, or plusses.
Best practice is apparently to keep them the same, though that's not always required.

You can apparently denote code by using quote marks.
I'm not totally sure how to make quotes, but that's hardish in latex.\footnote{I defined a macro to do it for me}

Ooh! I can strikethrough text semi easily in markdown, simply by using two tildes on either side.
That's pretty nifty.

Footnotes!\footnote{wow it took me shockingly long to get to this}
To create a footnote, open bracket, caret, the number of the footnote, close bracket.
To create the text for the footnote, the same, but then a colon and the text.

Highlight is double equals sign on either side.

Subscript is single tilde on either side.

Welp, as soon as I have the requisite makefile, and maybe before, I'll switch over to our good friend markdown!

Daily Reflection:
\begin{itemize}
\item Did I write 1700 words for NaNoWriMo? Almost at 50K! 
\item Did I write a chapter of Jeb? I will finish after this.
\item Did I blog? I mean at some point I should stop letting myself just put random words on a page and call it a musing.
\item Did I stretch? I'm sore at this point, which isn't great, but I think that I'm less sore than I would have been if I didn't stretch, so I should do it again today.
\item Am I doing better at prayer than a rushed and thoughtless rosary? I made it through the whole rosary yesterday. It was a little rushed.
\item Am I doing a good job writing letters to friends? I think I'm good getting rid of this goal starting tomorrow, since I don't really care about it for the next, um, day.
\end{itemize}

\end{document}