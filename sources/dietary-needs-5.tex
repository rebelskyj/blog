\documentclass[12pt]{article}[titlepage]
\newcommand{\say}[1]{``#1''}
\newcommand{\nsay}[1]{`#1'}
\usepackage{endnotes}
\newcommand{\B}{\backslash{}}
\renewcommand{\,}{\textsuperscript{,}}
\usepackage{setspace}
\usepackage{tipa}
\usepackage{hyperref}
\begin{document}
\doublespacing
\section{\href{dietary-needs-5.html}{The End of a Series}}

First Published: 18 January 2025

\section{Draft 2}  
After writing \href{dietary-needs-4}{yesterday's musing}, a friend asked what the daily needs of each phytonutrient are.  
Unfortunately, that doesn't seem like an easy to answer question, since the textbook \href{https://nap.nationalacademies.org/read/11537/chapter/1}{I've been relying on} lists vitamins, not phytonutrients.  
It has, however, been \href{https://nap.nationalacademies.org/read/26818/chapter/1}{updated in part}, though tragically only for total energy expenditure.  
Time to see what Big Food\footnote{one of the rare times I use this phrase where it's accurate} thinks that I need.  


\begin{itemize}  
\item \href{https://nap.nationalacademies.org/read/26818/chapter/2\#3}{Ooh! Equations with the actual R2 and RMSE}\footnote{about 340 cals}. Apparently men tend to need 10.83 fewer calories a year after the age of 19\footnote{and women require 7 fewer}, regardless of physical activity.  
I do find it really interesting how very different the four levels of activity look for the caloric needs.  
Height is least important for Active men, and almost twice as important for very active men as anyone else.  
Weight also increases caloric needs as activity rises, which is fun.   
Anyways, I should apparently be shooting\footnote{I'll look at all four values, knowing that the very active will likely never be relevant again, even if the top three might be} for between 3200 and 4200, depending on how active I am. It's interesting how much of a jump there is between active and very active, though. That's all from this book, so now to go through the 2006 book.  
\item Exercise\footnote{which feels weird to put in dietary reference but}: 60 minutes daily briskly walking or jogging at 3-4 miles per hour. I don't consider that a brisk walking pace, since it's a 15 to 20 minute mile but.

OOH! They go into detail about how to figure out your activity level, which is only designed for normal BMI\footnote{which does explain why they updated it}:  
\begin{itemize}  
\item Sedentary: PAL between 1 and 1.4\footnote{so curious what sub 1 is}  
\item low active: 1.4 to 1.6  
\item active: 1.6 to 1.9  
\item very active 1.9 to 2.5\footnote{what about above that!}  
\end{itemize}  
For PAL defined as the caloric intake divided by basal energy expenditure\footnote{I think that's the assume you don't move at all}  
They go through a list of activities with the amount it is expected to change PAL per hour performed\footnote{which I love as a concept, thinking about the fact that your basal needs are the entire basis for the increased costs}. Billiards is apparently worth between 0.05 and 0.1 PAL per hour, meaning that 4 hours of billiards is almost enough to bring someone out of a sedentary life if that's their only movement.  


Walking at the pace they mention above is worth between .13 and .22 PAL per hour\footnote{this is a dangerous concept for me to be aware of, if only because I can start actually counting my movement in a variety of ways}  
\item Carbs: have done before: 130 g/day for everyone over the age of 1 who is not pregnant. Needed to prevent ketosis and keep brain happy.  
\item Fiber: Intake is based on caloric intake, which is kind of fun. If under one, no data exists.\footnote{which makes sense, because I don't tend to think of milk as fibrous} Otherwise, 14 grams per 1000 calories consumed is the expected need, though it's not well proven.  
Dietary fiber is plant matter we can't digest, and Functional fiber is additives.

Because we can't measure things that aren't absorbed, difficult to get good data. Ooh they break down the different forms of fiber by their effect. I'll list the ones I find most interesting:  
\begin{itemize}  
\item Guar gum can reduce blood cholesterol by 11-16 percent, along with significant reduction in glycemic response\footnote{which I assume is good}  
\item Oat products are more easily fermented.  
\end{itemize}  
The value comes from optimal intake to prevent heart disease. 42 to 56 grams a day isn't that hard to hit at all  
\item Fat, as before, pretty easy to hit  
\item Cholesterol: your body can make it, so don't need to eat it. Eating any makes you more likely to have chronic heart disease, and is found in animal products.  
\item Protein as before, and is still broken down into the various amino acids.  
\item Water: 3.7 L total consumption is apparently the average for men 19 to 70. That's not super helpful for me, and I think I should shoot higher.  
\item Woo Vitamins!  
\begin{itemize}  
\item Vitamin A: 900 micrograms, and under 3000 micrograms. It's measured in retinol activity, and one gram retinol is worth 12 grams of beta carotene or 24 grams of alpha carotene or beta cryptothanxin. Retinol is found in animal products, so the precursors are mostly needed for vegetarians. Recommended amount based on liver stores.  
\item B6: 1.3 mg, under 100.\footnote{wow that's a range} Upper range causes neuropathy. Minimum is set by some fancy biology word. It's about 75 percent bioavailable from a mixed diet, and comes from starchy vegetables, organ meat, non citrus fruit, and fortified cereal.  
\item B12: 2.4 mg.  
\item Biotin 30 ug, apparently we don't know how the body processes it when bound to protein. Needed for carboxylation  
\item Vitamin C: 90 to 2000 mg\footnote{I have absolutely taken more than that}. Only issue with too much is diarrhea, and smokers need more.  
\item Carotenoids: Natural pigments, like beta carotene. Wildly, raw carrot absorption of beta carotene can be as bad as 5 percent, fruits are better, and supplements can go up to 70 percent. Steaming improves bioavailability, but prolonged cooking can make it worse.  
Needs to be consumed with fat for optimal intake.  
\item Choline: 550 to 3500 mg. Mostly comes from membranes in foods: milk, liver, egg, peanut. Lethicin is apparently a supplement added for it.  
\item D: 5 to 50 ug. Too much apparently causes too much calcium in blood  
\item E: 15 to 1000 mg. Prevents free radicals, occurs in 8 forms, but plasma only maintains one. Upper limit because of hemorrhages. Vegetable oils have it, no one in America tends to be deficient unless body can't intake.  
Apparently no specific metabolic roles.  
\item Folate: 400 to 1000 ug. Comes from dark green vegetables, beans, legumes. Metabolizes\footnote{originally had an s (a s?) instead of z} nucleic and amino acids. UL defined because it can mask a B12 deficiency.  
\item K: 120 ug. Blood coagulation and bone metabolism. Green vegetables\footnote{tragically, mostly brassicas} and vegetable oils are the normal sources.  
\item Niacin: 16 to 35 mg.  Too high, flushing. Comes from meat or fortification. Insufficient B6, riboflavin, or iron may increase needs. If insufficient, causes pellagra.  
\item Pantothenic Acid: 5 mg. Unknown how we get it, looks like less processed food better.\footnote{the latter half of the sentence is commentary}  
\item Riboflavin: 1.3 mg. B2\footnote{so many b vitamins}. Found in milk, bread, and\footnote{as is becoming a trend,} fortified foods. Needed for redox reactions. Apparently no adverse upper level effects have been found.  
General malaise symptoms\footnote{commentary} are deficiency signs.  
\item Thiamin: 1.2 mg. B1! Metabolizes carbs and branched amino acids. We get it from fortified foods and ham. Anorexia and weight loss are symptoms of not enough. Also causes beriberi.  
\item Elements: It goes through: Calcium\footnote{1000 to 2500 mg}, the bone maker, Chromium\footnote{35 ug, estimated bc IR}, which lets insulin work, Copper\footnote{900 to 10000 ug}, for molecular oxygen reduction, Fluoride\footnote{IR, 4-10 mg} for teeth and bone, Iodine\footnote{150 to 1100 ug} for the thyroid\footnote{I knew this one! Comes from the soil which is why the Great Lakes are not healthy}, Magnesium\footnote{410 mg, UL of 350, which is apparently only from pharmacological agents. Kind of funny still} shows up all over the place in enzymatic things, Iron\footnote{8 to 45 mg, one of the rare cases where women have a higher one, and a significant one at that (18 mg)} for hemoglobin, Manganese\footnote{I will always be mad that this and Magnesium are both elements}\textsuperscript{,}\footnote{IR 2.3 to 11 mg} helps with bone making, Molybdenum\footnote{45 to 2000 ug} as a cofactor,\footnote{upper limits cause reproductive issues in mice, not people, it comes from the soil}, Phosphorus\footnote{mmmm matches}\textsuperscript{,}\footnote{700 to 4000 mg} in Phosphate, for bone construction, Potassium\footnote{IR, 4.7 g} for proper signaling, effects dependent on its anion, and it helps prevent NaCl from hurting us, Selenium\footnote{I never would have thought that I needed this, 55 to 400 ug} prevents oxidative stress and helps redox C, too much causes brittle hair and nails, NaCl\footnote{IR, 1.5 to 2.3 g Na, 2.3 to 3.6 g Cl} to keep our body like the sea, Sulfate\footnote{I do have to wonder why Phosphate wasn't listed but} comes from amino acid, Zinc\footnote{11 to 40 mg} which makes us grow, and (Arsenic, Boron, Nickel, Silicon, Vanadium)\footnote{all UL and mg/day, B: 20, Ni: 1, V: 1.8}, made a note that these elements might be needed for something, but not usually.   
\end{itemize}  
\end{itemize}

I do also have a new favorite quote from the day: \say{data were insufficient to set a UL\footnote{upper limit} for arsenic. Although a UL was not determined for arsenic, there is no justification fro adding it to food or supplements}  
Interestingly: organic arsenic is usually fine. Inorganic arsenic (arsenite III or arsenate V), by contrast, \say{is an established human poison}.

So, what have we learned?

Dietary science is a lot of \say{people who eat like this tend to do better}, and I remember seeing a thing about how the way food is presented affects absorption of nutrients.  
It makes me feel better that I cannot fall into the trap of living my life by evidence based practice to the detriment of my day to day living, because there truly is not a lot of evidence.\footnote{in the sense that like if I do the generally normally accepted advice of \say{eat less processed food, lots of plant, and move around more}, I'll probably be fine}  
Well, this has been a fun journey.  
Tune in next time to see my recipe for bread\footnote{and a review!}

  


\section{Draft 1}  
A friend, upon reading yesterday's musing, asked a very poignant question: how much of each phytonutrient do I actually need?  
Using \href{https://nap.nationalacademies.org/read/11537/chapter/59}{our favorite textbook}, we find that the actual chemicals are not listed, instead the vitamins are.  


\begin{itemize}  
\item lycopene is apparently \href{https://lpi.oregonstate.edu/mic/dietary-factors/phytochemicals/carotenoids}{not essential}.  
\item beta cryptothanxin converts to Vitamin A at a 24 to 1 efficiency, so in order to get the \href{https://nap.nationalacademies.org/read/11537/chapter/17}{900 micrograms a day I need, I need to consume 21.6 milligrams. Interestingly, \https://tools.myfooddata.com/nutrient-ranking-tool/beta-carotene/all/highest/household/common/no}{myfooddata tells me} that carrots contain no beta cryptothanxin. One carrot does, however, contain about 6 milligrams of beta carotene, which is equivalent to half a mig\footnote{since I say mig I type mig, instead of mg} of Retinol\footnote{read: Vitamin A}, so two carrots a day\footnote{assuming a seven to eight inch carrot} is large enough, even if I have no other sources.  
\item sulforaphane, isothiocyanates, and indoles  
\item anthocyanins  
\item alicin  
\end{itemize}

Instead of trusting the website that told me I need pretty colors, I'm actually just going to take this time and see what nutrients the government thinks that I need daily.\footnote{these numbers are for males aged 19-30, but don't really change a lot if you change that}  


\begin{itemize}  
\item Vitamin A: 900 micrograms\footnote{no weight}  
\item Vitamin C: 90 milligrams  
\item Vitamin D: insufficient research, recommendation is 5 micrograms  
\item Vitamin E: 15 milligrams  
\item Vitamin K: insufficient research, recommendation 120 micrograms  
\item Thiamin: 1.2 mg  
\item Riboflavin: 1.3 mg  
\item Niacin: 16 mg  
\item Vitamin B6: 1.3 mg\footnote{the first one I've noticed increases as I grow older after 30. I wonder why}  
\item Folate: 400 micrograms  
\item Vitamin B12: 2.4 micrograms  
\item Pantothenic Acid: insufficient research, recommend 5 mg  
\item Biotin: Insufficient research, recommend 30 micrograms  
\item Choline: IR\footnote{typing insufficient research is hard, so I'm just abbreviating here on out}, 550 mg  
\item Calcium IR: 1000 mg\footnote{how that differs from a gram, unsure}  
\item Chromium\footnote{ Wild, I had no idea I was actually supposed to have any}: IR, 35 micrograms  
\item Copper: 900 ug\footnote{I hate typing microgram, and mu is not a latin letter, so I use its closest analog}  
\item Fluoride: IR, 4 mg  
\item Iodine: 150 ug  
\item Magnesium: 410 mg  
\item Manganese: IR 2.3 mg  
\item Molybdenum: 45 ug  
\item Phosphorus: 700 mg  
\item Selenium: 55 ug  
\item Zinc: 11 mg  
\item Potassium: IR 4.7 g\footnote{wow such a big number}  
\item Sodium: IR 1.5 g  
\item Chloride:\footnote{I wonder if there's really any dietary source for chlorine outside of salt} IR 2.3 g  
\item Water: IR\footnote{OH NO! I've learned that this is just normal data. I'll have to look at each chapter for the weight based things}, interestingly is not weight based. If I consume more than 0.7 to 1 \href{https://nap.nationalacademies.org/read/11537/chapter/15}{liter per hour} of it, I might get toxicity.  
\end{itemize}

Welp, time to go back and see what the USDA cares enough about to write a chapter on. 
\end{document}