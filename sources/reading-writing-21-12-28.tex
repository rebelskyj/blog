\documentclass[12pt]{article}[titlepage]
\newcommand{\say}[1]{``#1''}
\newcommand{\nsay}[1]{`#1'}
\usepackage{endnotes}
\newcommand{\1}{\={a}}
\newcommand{\2}{\={e}}
\newcommand{\3}{\={\i}}
\newcommand{\4}{\=o}
\newcommand{\5}{\=u}
\newcommand{\6}{\={A}}
\newcommand{\B}{\backslash{}}
\renewcommand{\,}{\textsuperscript{,}}
\usepackage{setspace}
\usepackage{tipa}
\usepackage{hyperref}
\begin{document}
\doublespacing
\section{\href{reading-writing-23-12-28.html}{What I Read and Wrote}}
First Published: 2023 December 28

\section{Draft 1}
As I mentioned \href{schedules-3.html}{yesterday}, Thursdays are going to be reserved for what I'm reading and writing lately.
It is currently a form of vacation for me right now, which means that I am doing far less of both than normal.\footnote{or at least reading. Oddly enough, I do tend to read more when I'm busier, probably because I have more downtime and less time with people, which makes it easier not to be social.}

In terms of reading, this past month\footnote{i.e. December 2023, according to my reading logs} I finished:
\begin{itemize}
\item Genesis of Gender by Abigail Favale. The book came recommended to me, and it's a deconstruction of modern gender theory from a former gender theory professor who converted to Catholicism.
I found it better the more I read the book, probably because the focus shifted towards the latter parts.
I may or may not discuss it in greater depth in the future, depending on how starved for content I become.
\item There's a serialized version of C.S. Lewis's Screwtape Letters\footnote{re-serialized? since it did initially appear as a serial. Nowadays it gets sent as a weekly email, which was nice} that ended this year's run in December.
It was a much different experience to read the book in short weekly snippets.
\item I listened through C. Mantis's Path of Ascension\footnote{or at least the books available in audio format currently}. I think that it's my third or fourth read through, since I've been following the serial since early on in its career on the internet.
\item I also listened through Casualfarmer's Beware of Chicken, which is another former web serial turned book available from an indie publisher.
\item I read through Peter Kreeft's Fundamentals of the Faith. It was a really beautiful book, but I unfortunately\footnote{in terms of being able to reference it in the future, not in terms of any desire I have to actually continue owning it} gave it away to a friend.
\item I finished Dr. Bessel van der Kolk's The Body Keeps the Score, which I've been working on in theory since last year.
I found that it was much easier to get through in audio format, but that may have just been where I was at for that week.
\item I listened to one of the recommended reads for my entire campus, Clint Smith's How the Word is Passed. It was an interesting reflection on the way that America has or hasn't dealt with slavery and the consequences of chattel slavery.
I know that it changed my views of the Confederacy at least a little, if only because I tend to be sympathetic to foot soldiers.\footnote{and he did make the very fair point that records (apparently, I'm not going to fact check) show poor white Southerners still enjoyed the benefit of not being the lowest class}
\item I listened to Travis Baldree's Legends and Lattes. I know the author primarily as an audio book reader, so I was very interested in his book when it came out. It's apparently become a fairly well known book, at least in the sense that I've seen it more places than any of the books I've heard of his.
\item After watching CGP Grey's video about how to be miserable\footnote{OH! That YouTube channel might be the reason that I spell it grey not gray. I hadn't considered that}, I read the book that he based it on, How to Be Miserable: 40 Strategies you already use by Randy Paterson. 
It was really eye opening to me how half of the items or so I read and went \say{who would ever do that?} and the other half were habits I hardly realized I had, they were so ingrained.
\item I generally kept up on the web serials that I'm reading. Path of Ascension and Beware of Chicken both post fairly consistent updates, which is fun. It's wild how far ahead the freely available content is from the nominally published.\footnote{I can discuss other web novels I've been reading if anyone actually wants to know. Feel free to shoot me a message if you want any recommendations.}
\end{itemize}

That more or less sums up the reading\footnote{and audiobook listening, I guess,} that I've been doing.
In terms of writing, I've written a fair amount of code, though that's been a lot of writing long functions just to delete them when I realize that I don't want the functionality they provide.\footnote{do functions provide functionality? I think so, at least.}
I've recently been taking a break from Jeb, which has been really restful, though I am excited to get back to writing it.
I wrote a fair number of sonnets this month, and I finally got to the point that writing a sonnet is hardly an effort, even if it isn't good.

Oh! I suppose that I've also been reading the book on writing, Writing Well by Sven Birkerts and Donald Hall. It's a fantastic book, and I'm really enjoying the way that it's making me think about the way that I write and consume language.
I just finished the section of words, and one piece of advice they kept repeating in that section is that \say{a change in style, however slight, is a change in meaning, however slight}.
I feel like often, at least in my experience, people don't treat rephrased sentences as different meaning.
As someone who does legitimately believe that no distinguishable wordings are fungible,\footnote{oof that's a jargon filled sentence. I should change that, according to them. Then again, I am trying to be formal, so distinguishable, and I'm trying to push against the commodification of language, which fungibility brings to mind, at least to me. Maybe it did work} it was nice to have that take explicitly stated.
Even the difference between \say{I read a book} and \say{a book was read by me} has a shade of difference, even outside of the taste.
Of course, the taste of words is not something that they underestimate.
One piece of advice they stress almost as much as not mixing metaphors or using dead cliches is having sentences flow.
I've just started the sentence section, so I haven't heard exactly what they mean by it, but I am excited to find out.

I also just started the first of my Saturday Musings, which is focused about copyright. I don't know if it will end up being the first Saturday Musing I post, because it's a lot to write about, but I hope to do the subject justice.
I'd also like to start plotting out the next few chapters of my web serial, and now is as good of a time as any, I suppose.
Plotted out the next three and a half chapters, and I'm excited to write them again.
If I was smart, I would find a writing guide that actually teaches how plotting works, since right now I just kind of free associate what I want to see and then write in details.

Daily Reflection:
\begin{itemize}
\item Hobbies:
\begin{itemize}
\item Did I embroider today? Shoot! I should do that tomorrow. It's too late o clock right now.
\item Did I play guitar today? No, sadly today was more focused on real work.
\item Did I practice touch typing today? Once again, a few lessons. I think that it might not be the best choice ever to go straight from 50 wpm goal to 75 wpm. Might be a smarter choice to do smaller, more iterative versions, where I go up by like 5 wpm at a time. Then again, it isn't like I need to worry about optimizing this. Any typing practice is probably ninety percent as effective as any other.
\end{itemize}
\item Reading
\begin{itemize}
\item Have I made progress on my Currently Reading Shelf? I am!
\item Did I read the book on craft? I read it a little last night and then a fair amount this morning!
\end{itemize}
\item Writing
\begin{itemize}
\item Did I blog? Kind of! More brain dump than anything else, but that's life sometimes.
\item Did I write ahead on Jeb? I plotted out a few chapters, which is nice.
\item Letter to friends? I met with one and set up a meeting with another!
\item Paper? I once again proved to myself and the smart people\footnote{smart here meaning knowledgeable in the specific domain and nearby enough for me to pester} in my life that, not only can I not use derivative based methods for most of the work, I cannot even use most derivative free methods for optimization.
\end{itemize}
\item Wellness
\begin{itemize}
\item How well did I pray? Remaining less than great. This is something I'd like to work on in the future.
\item Did I spend my time well? Kind of! I spent a lot of time with family again, caught up with a close friend, and did some research work. Between that, pretty well.
\item Did I stretch? sadness.
\item Did I exercise? sorrow.
\item Water? Remembered I was thirsty at one occasion and then drank water! That counts for something!
\end{itemize}
\end{itemize}
\end{document}