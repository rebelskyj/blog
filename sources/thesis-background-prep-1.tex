\documentclass[12pt]{article}[titlepage]
\newcommand{\say}[1]{``#1''}
\newcommand{\nsay}[1]{`#1'}
\usepackage{endnotes}
\newcommand{\1}{\={a}}
\newcommand{\2}{\={e}}
\newcommand{\3}{\={\i}}
\newcommand{\4}{\=o}
\newcommand{\5}{\=u}
\newcommand{\6}{\={A}}
\newcommand{\B}{\backslash{}}
\renewcommand{\,}{\textsuperscript{,}}
\usepackage{setspace}
\usepackage{tipa}
\usepackage{hyperref}
\begin{document}
\doublespacing
\section{\href{thesis-background-prep-1.html}{Thesis Background Exam Prep}}
First Published: 2022 January 10

Prereading note: This is hopefully going to be my abstract
\section{Draft 0.5: 18 January 2022}
The major motivation of my research is bridging the gap between known terrestrial life, composed of complex chains of amino acids, and the assumed lack of life extraterrestrially.
In the heat of stars, where heavier elements are formed, the intense radiation will prevent any sort of molecular bonds from occurring, leaving us with ionized monatomic species
On the other end of the question of life, prior research has shown that a terrestrial and aqueous collection of amino acids will spontaneously self assemble into repeating patterns (life) when exposed to electricity. 

The question then becomes how we can move from these single-atom ion into amino acids.
Prior research has found many molecules which could be amino acid precursors, but there are yet to be complete models which produce these amino acids.
As a result, we search for ions and radicals which would form these atoms.

\section{Draft 0: A text I sent}
So yeah our big motivation is that like life exists (allegedly) and we're curious how close to life you can get without a rocky planet. We know if you put a bucket of amino acids together and shock them you get life, but most historic research suggests amino acids only form on planets. Bc space there's lots of radicals and ions (bc nothing to react w and radiation), so we look at those as precursors
\end{document}