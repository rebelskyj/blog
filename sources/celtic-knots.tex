\documentclass[12pt]{article}[titlepage]
\newcommand{\say}[1]{``#1''}
\newcommand{\nsay}[1]{`#1'}
\usepackage{endnotes}
\newcommand{\B}{\backslash{}}
\renewcommand{\,}{\textsuperscript{,}}
\usepackage{setspace}
\usepackage{tipa}
\usepackage{hyperref}
\begin{document}
\doublespacing
\section{\href{celtic-knots.html}{On Celtic Knots}}
First Published: 2022 October 26

\section{Draft 1}
As I've mentioned \href{celtic-crochet.html}{before}, I like crocheting Celtic knots.
I also like drawing them.
One issue I often have with them, though, is that extant knotwork often has motifs other than just an interwoven strand.

A very common motif is some sort of animal shape, which I've always struggled to draw.
So, recently I decided it would be fun to learn how to draw a dragon,\footnote{or at least a dragon head} so that I could make better\footnote{read: more interesting} knotwork.

It went really slowly at first.
Every dragon I drew looked somewhat goofy, which I only realized later was because of a lack of details.\footnote{at least the way I was drawing them}
Eventually I figured out how to draw nice\footnote{read: looks mean} dragon heads, and even made a convincing knot or two with them.
Then I decided to learn to draw wings, for some reason.

I've always struggled with wings, probably because of the inherent need for a 3D perspective.
For whatever reason, though, this time I drew good enough wings within like 15 minutes of trying, which is fun.
Anyways, I'm enjoying the Celtic knot resurgence in my life.
\end{document}