\documentclass[12pt]{article}[titlepage]
\newcommand{\say}[1]{``#1''}
\newcommand{\nsay}[1]{`#1'}
\usepackage{endnotes}
\newcommand{\1}{\={a}}
\newcommand{\2}{\={e}}
\newcommand{\3}{\={\i}}
\newcommand{\4}{\=o}
\newcommand{\5}{\=u}
\newcommand{\6}{\={A}}
\newcommand{\B}{\backslash{}}
\renewcommand{\,}{\textsuperscript{,}}
\usepackage{setspace}
\usepackage{tipa}
\usepackage{hyperref}
\begin{document}
\doublespacing
\section{\href{reflections-on-readings-5-ordinary-c.html}{Reflections on Today's Gospel}}
First Published: 2019 February 10

Isaiah 6:8: \say{Then I heard the voice of the Lord saying, \nsay{Whom shall I send? Who will go for us?} \nsay{Here I am,} I said; \nsay{send me!}}

\section{Draft 1}
Today's readings speak about the nature of God's relationship with us.
As Isaiah was, we are to be awed and fearing.\footnote{Isaiah 6:5 \say{Then I said, \nsay{Woe is me, I am doomed! For I am a man of unclean lips, living among a people of unclean lips, and my eyes have seen the King, the LORD of hosts!}}}
As Simon was, we are to be unquestioning,\footnote{Luke 5:5 \say{Simon said in reply, \nsay{Master, we have worked hard all night and have caught nothing, but at your command I will lower the nets.}}} and again, fearful.\footnote{When Simon Peter saw this, he fell at the knees of Jesus and said, \nsay{Depart from me, Lord, for I am a sinful man.}}

But this is not all that the relationship is.
For just after Isaiah speaks about his unworthiness, he is made pure.\footnote{Isaiah 6:7 \say{Then one of the seraphim flew to me, holding an ember which he had taken with tongs from the altar. He touched my mouth with it. \nsay{See,} he said, \nsay{now that this has touched your lips, your wickedness is removed, your sin purged.}}}
And when Simon and his partners too followed God, they are exhorted, \say{Do not be afraid; from now on you will be catching men.}\footnote{Luke 5:10B}
All in all, that's nice to think about.
We can trust wholly in the Lord our God, and he will make us pure and worthy.

\end{document}