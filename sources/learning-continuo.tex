\documentclass[12pt]{article}[titlepage]
\newcommand{\say}[1]{``#1''}
\newcommand{\nsay}[1]{`#1'}
\usepackage{endnotes}
\newcommand{\1}{\={a}}
\newcommand{\2}{\={e}}
\newcommand{\3}{\={\i}}
\newcommand{\4}{\=o}
\newcommand{\5}{\=u}
\newcommand{\6}{\={A}}
\newcommand{\B}{\backslash{}}
\renewcommand{\,}{\textsuperscript{,}}
\usepackage{setspace}
\usepackage{tipa}
\usepackage{hyperref}
\begin{document}
\doublespacing
\section{\href{learning-continuo.html}{Learning Continuo}}
First Published: 2019 January 26
\section{Draft 1}
I learned that for a class I am taking about historical improvisation, I need to learn continuo.\footnote{the bass part}
As is to be expected, most of this is scarcely notated, with the assumption that it'll be composed to fit the performer that day.
Unfortunately, as I mentioned to the professor, keyboards and I are not friends.

So, I proposed that I could do continuo on the gothic harp, as I know how to play it.\footnote{and because it's fun to play}
But, the gothic harp, as one might expect, is not a chromatic instrument.
It has seven notes to an octave, which can be tuned.
However, that's really hard to do in the moment.
Hypothetically, I could use my opposite hand to raise a note, but that's apparently not allowed, as continuo playing needs both.
This means that I'll be learning continuo this semester.
\end{document}