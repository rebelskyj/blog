\documentclass[12pt]{article}[titlepage]
\newcommand{\say}[1]{``#1''}
\newcommand{\nsay}[1]{`#1'}
\usepackage{endnotes}
\newcommand{\1}{\={a}}
\newcommand{\2}{\={e}}
\newcommand{\3}{\={\i}}
\newcommand{\4}{\=o}
\newcommand{\5}{\=u}
\newcommand{\6}{\={A}}
\newcommand{\B}{\backslash{}}
\renewcommand{\,}{\textsuperscript{,}}
\usepackage{setspace}
\usepackage{tipa}
\usepackage{hyperref}
\begin{document}
\doublespacing
\section{\href{reflections-on-readings-26-ordinary-b.html}{Reflections on Today's Gospel}}
Prereading Note: Sundays will probably be reading responses, and I'll start each one with the line that stands out most to me.
\section{Draft 2}
Mark 9:40 \say{For he that is not against us is for us.}

Today is the 26 Sunday of Ordinary Time in Year B.

The first reading has Joshua\footnote{one of Moses' (Moses's?) \say{chosen men}} telling Moses to rebuke the men who the spirit is speaking through.
Moses tells him that he wishes that everyone could be so chosen.

Jesus continues this in the gospel, where the line that stuck out most to me occurs.
While it paints the world in the exact same set of two groups that toxic personalities do,\footnote{if you're not with us you're against us} he does so in the opposite way.
Not to let this be confused for assuming that no harm will befall you, or that we should trust everyone and everything, Jesus then tells what to do with identified enemies.
In short, remove them.
Whether a body part or a fellow human, it's our duty to remove them when they lead us to sin.

However, the point of everyone who isn't against you being for you is certainly an important piece of the Gospel.
I'm as guilty as anyone else\footnote{if not more so} in grouping people who aren't clearly on my side as enemies.
This comes out most clearly\footnote{at least to me} in my faith.
I often group think of\footnote{and often tell} people who have different faiths than me as being wrong, and even evil, even though their faith is doing for them what faith is supposed to do for all of us: bringing us closer to God, and helping us to help the world.
Today's Gospel called me to be better, which is all that I can ever hope for

\section{Draft 1}
Mark 9:40 \say{For he that is not against us is for us.}

Today is the 26 Sunday of Ordinary Time in Year B.
I really enjoyed the readings, but of course, felt a call from them.

The first reading has Joshua\footnote{one of Moses' (Moses's?) \say{chosen men}} telling Moses to rebuke the men who the spirit is speaking through.
Moses tells him that he wishes that everyone could be so chosen.

Jesus continues this in the gospel, where the line that stuck out most to me occurs.
While it paints the world in the exact same set of two groups that toxic personalities do,\footnote{if you're not with us you're against us} he does so in the opposite way.
Not to let this be confused for assuming that no harm will befall you, Jesus then tells us what to do with identified enemies.
In short, remove them.
Whether a body part or a fellow human, it's our duty to make the world the best it can be.

I'm as guilty as anyone else in grouping people who aren't clearly on my side as enemies.
This comes out most clearly\footnote{at least to me} in my faith.
I very often do group people who have different faiths than me as wrong, even though their faith is doing for them what faith is supposed to do for all of us: bringing us closer to God, and helping us to help the world.

\end{document}