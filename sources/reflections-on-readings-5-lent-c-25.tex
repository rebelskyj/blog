\documentclass[12pt]{article}[titlepage]
\newcommand{\say}[1]{``#1''}
\newcommand{\nsay}[1]{`#1'}
\usepackage{endnotes}
\newcommand{\B}{\backslash{}}
\renewcommand{\,}{\textsuperscript{,}}
\usepackage{setspace}
\usepackage{tipa}
\usepackage{hyperref}
\begin{document}
\doublespacing
\section{\href{reflections-on-readings-5-lent-c-25.html}{Reflections on the Gospel}}
First Published: 2025 April 6

I guess it shouldn't surprise me that I have only one other reflection on a Lenten gospel, since I've never been good at this blog in the springtime.
\\Note 7 April 2025: By mistake I uploaded the monthly reflection again.
Whoops!


N.B. Oof this one got rambly and also weirdly aggressive and uses second person more than usual.  
To any and all of my readers, I promise that the you in the writing is the generic you, not you specifically.

\section{Draft 2: 6 April 2025}

\say{Has no one condemned you?}

She replied, \say{No one, sir.}

Then Jesus said, \say{Neither do I condemn you. Go, and from now on do not sin any more.}

Of the lines in the Bible, I don't really know why this one doesn't get the same sort of, if you'll pardon the pun, religious fanatacism that others like John 3:16 gets.  
What better encapsulates Church teaching than Christ's final line there?

We are a faith of hope, of belief, of love.  
G-d does not condemn anyone to hell, nor does he condemn anyone generally.  
We choose hell.

However, this is not free license to act how we please.  
We are also commanded to sin no further.

Who has the authority to make such statements?

The Lord, the G-d of Abraham and Isaac, the voice of the burning bush, the soft call on the wind.

I cannot understand why the beauty of this line is not commented on more often.

I do, however, understand why so many non-and non-practicing\footnote{Because once baptized always baptized, etc} Christians use this as a cudgel.  
I have primarily seen it used in reference to homosexuality and abortion, the two hot button topics of the era for believers.

When the crowd is gone, and the woman is standing alone in front of her Creator, as we all will do, He asks if anyone condemns her.  
If she had been wracked with guilt over her actions, she could still have condemned herself.  
She does not, however.  
Looking around, and seeing the space empty of herself and Love Himself, she responds that no one does.

While the crowd arrives and ultimately leaves, Christ is writing in the sand.  
Folk legend says that he was writing the sins of everyone in the crowd, which, even if I do not agree with on principle, points so something we too often forget.  
In the parables that find themselves recreated in our lives, we are not Him.  
We are not the father from last week, gladly welcoming his son back.  
Most often, we are not even the prodigal son.  
We are the other son, the one feeling jilted because his father is celebrating the return of his wayward son.

I think that we should all take part in a thought exercise.  
Imagine the worst person who has ever lived.  
Pray for them.

That's not the exercise, but it is a good reminder.  
The exercise works just as well with a fictional person, assuming that you can still bring yourself to feel the same level of disgust and horror at the actions of the person.  
Imagine that when you die, and the Heavenly Father receives you into his warm embrace, you find yourself face to face with this person.  
I am not asking how the you of then would feel, I am asking how the you of right now would feel.

Or, take it a few steps closer to home.  
Think of everyone in your life, repentant sinner trying their best, person flaunting their sin, the annoying person who always takes forever to order their coffee.  
Would you be happy spending an eternity in heaven with all of them?  
Would you be happy spending an eternity in heaven if the two of you alone were saved?  
If the answer to any of these questions is no, then you do not have the heart that Christ demands.

Too often, people who claim to be Christian will speak of justice when they mean retribution.  
We need to remember, all sin is public sin, because we are all one body.  
You cannot harm one part of yourself without harming the whole.  
If, in the gym, one arm gave out before the other, you\footnote{hopefully} would not shame the arm.  
You would give it the extra attention that it needs in order to be as strong as the other.

The Church does not teach equality, but equity.  
\say{To whom much has been given, much is expected}, reminds us that those who have been blessed with more or seemingly greater\footnote{remembering that the only greatness in heaven is He Who Is. All that we do on earth is just in attempt to model ourselves after Him} gifts are expected to give more to the world around them.  
In direct opposition to Calvinism, the Church teaches that those who have received more from the Lord are actively sinful if they notice someone suffering.  
St. Basil, one of the early Church Fathers\footnote{since for some reason there are people who think that those born centuries ago have wisdom that we lack by virtue of coming later} put it so strikingly.  
We would condemn someone stealing the coat from a homeless man's back.  
What then, is the coat that we do not wear?

As with every Gospel passage, it can return to love of neighbor.  
However, this passage is specifically about justice.

One argument people use for retributive \say{justice} is that it works as a deterrent.  
If you cut of thieves' hands, then you will have less theft.  
If you kill the murderer, others will not lash out in anger.  
This is not, however, a matter of faith and morals.  
This is a matter of fact and evidence.

The research is resoundingly clear that retributive justice does not work for anything except causing suffering.\footnote{which, I will admit, is some people's stated goal}  
All punishment should be oriented towards the salvation of a person's soul and the souls of all in the world.

An argument for life in prison or capital punishment is that a country may be unable to keep someone effectively detained.  
If a murderer escapes and kills a dozen people, wouldn't it have been better to have killed him to prevent that?  
The Church is incredibly clear on that point.

No.

The Church rejects any way of quantifying the value of human life, as to put a price on something of infinite glory is to inherently devalue it.  
We are crafted, lovingly and individually, in the image and likeness of the source of all good.  
By saying that twenty lives is worth more than a life, or a thousand, or any number, you are inherently saying that there is a worth to human life.  
One life is worth one life.

Something mathematics has right is the idea of infinity.  
What is infinity plus one?  
Or, getting to the direct thrust, what is infinity times twenty, or a million?

Infinity remains infinity.

There is more to be said.  
One could speak about how jury nullification, at its best, deals with the fact that laws are black and white but circumstances rarely are.  
One could speak about how we can never know the state of another's soul.  
One could speak about how kings had, or at least claimed, a divine right to rule, which gave them the power of choosing death.  
One could speak about how, if the goal is to cow a populace into not murdering, it is truly less relevant who is hung, as long as a body swings from the rope.

This Lenten season, I invite you and myself to truly think on the meaning of Christ's passion.  
He willingly died for each and every one of us and our sins.  
My sin and my sin alone condemns Christ to the Cross.

But, as Lent becomes Easter, so too does Christ rise from the dead, showing the path forward.  
We should not condemn those around us, but in all things should strive to love everyone as Christ loves us.  
After all, just as two infinities is the same as one, half an infinity is as much too.  
Loving any fraction of the amount that Christ does means having an infinite well of love.

\section{Draft 1: 6 April 2025}

Wow, sing of an age past but not forgotten Sunday!\footnote{unlike wayback wednesday or throwback thursday or flashback friday (I'm going to start playing with capitalization, I think. I'm thinking about how to change my handwriting again, and capitalization is a thing that I play with a lot there so may as well do here), I don't know of any pithy ways to say that for satur or sundays. If you know of one, please let me know}  
It has been ages since the Gospel really spoke to me like it did today.  
With that in mind, I'm going to reflect on it.

Today's Gospel is probably one of the more commonly\footnote{intentionally. I know that we all have tons of Biblical allusions in everyday speech that go completely unnoted} cited passages of the Bible, especially by non believers.  
In it, a woman is caught in the act of adultery.  
The crowd,\footnote{wow I realize what Jewish authors mean by the New Testament being really Jew Hatey (I'm no longer using antisemitism, because we should call it what it is) more and more lately} seeing an opportunity to entrap The Lord, ask Him what they should do.  
Obviously Mosaic Law says that she is to be stoned, but something something, Roman government said that Jews can't do their own capital punishment or something.

Christ does not respond at first, instead writing\footnote{I swear I've seen versions that say drawing} on the ground.  
They repeat he question, and he says that the one without sin should cast the first stone.

In the Catholic joke version of this parable, a stone flies through the air and Christ exclaims \say{Mom! I told you not to come today!}

In the Biblical\footnote{read, more likely to be what happened} account, the crowd leaves one by one, until it is only the woman and Christ standing there.  
He asks if any condemn her, and she says no.  
Christ says that he, then, will also not, and tells her to go forth and sin no further.

Now, there's a lot that I want to unpack here.

First, something I don't know if I have ever seen is that Christ implicitly is asking the woman if she condemns herself or her actions.  
I feel like there's something really important there, especially in the context of how people use the passage.  
However, I'm not sure what it is yet, so I'll try to figure out by the end of this reflection.\footnote{something something, if I had more time I would have written less}

Second, there is a popular Catholic belief that Christ was writing the sins of the crowd on the ground, and that is what caused them to leave.  
I think that this is a bad faith reading.  
I do not know a single person who would be able to confidently state that they were perfect and without blame, especially in the context of explicit entrapment.  
Too, there is the point to remember that they asked Him this to cause trouble.  
I do not think that they would have stoned her even if He had said that all must follow the Mosaic Law.

In this reading, it is simply a waiting game.  
At some point, the provocateurs get bored or have other obligations, and are so forced to leave.  
Christ, waiting patiently\footnote{and who among us hasn't whittled a day away writing on the nearest available form?}, does not address the woman until the entirety of the crowd leaves.  
I also don't know what to make of this, except that we should never make choices about judgement and punishment rashly.

If I had been attempting to goad someone I thought of as a false prophet into choosing between two terrible choices and he just didn't give a real answer, I think that I would be relatively patient.  
I surely would believe that he would become uncomfortable with a crowd looming over him, anxiously awaiting an answer, far before I would have reason to leave.  
I don't know if that's true, but, if the leaders had that belief, they would have been proven entirely wrong.

Third,\footnote{both the first and third points here are things that I didn't think about until I started writing here. Wild how consciously taking time to consider something makes it easier to consider} there is something really interesting to me about the way that the Law is fundamentally changed in that instance.  
Nothing in the Mosaic Law, so far as I am aware, requires the dispensers of judgement to themselves be pure.

In Catholic teaching, sacraments are effective \say{ex opere operandi}; the holiness of the one providing the sacrament has no effect on the efficacy of the sacrament.  
There's something really interesting to me about the fact that we have decided that expecting perfection from those who have been called and chosen to lead the people of G-d\footnote{who, remember, are told it would be better that they have a millstone tied around their neck and be cast into the sea than to mislead}, and yet Christ seems to be demanding that we be perfect in order to enact justice.  
Regardless of whether we feel it is merited, the crowd was correct that the explicit punishment laid out in the Law was death by stoning.  
Again, I don't know what to do with this thought, so I will leave it for a little as well.

Moving to the more immediately applicable interpreting, this passage is often thrown by non-believers\footnote{or believers in something else. I'm not totally sure that it's a fair way to characterize someone. Non-Christians? Hmm} when Christians attempt to enact policy or seek punishment for violations, especially moral violations.  
The common response is that Christ said that she was to sin no more, and so telling people to stop bad actions is still ok or something.  
My most charitable reading of the Christian response is that the Christian legitimately believed that the other person was unaware that they were violating moral law, and out of perfect love and charity were simply informing them of a new Truth.\footnote{see my future musing on truth, reason, and revelation}

Of course, we know that is not the case.  
Especially in the context of homosexuality, where I feel like the verse was most popularly used, no one is unaware that the standard belief in Christian society has been that homosexuality is inherently sinful.\footnote{Though I did recently encounter an interesting argument that says more or less that the anti-homosexual stance the Church has taken is relatively modern. The Bible is more than okay with slavery, but we banned it when society said that we must. Even in America, the Jesuits (ugh I need to muse about my issues feeling called to Ignatian spirituality) owned, bought, and sold enslaved people as chattel. Something something, cultural norms aren't always right, especially arguments which are entirely \say{well we kind of always felt like this}. Like yes there are a few Biblical passages which explicitly decry homosexual actions, but those are also generally surrounded by any number of admonitions that we no longer follow. I should think on this longer}  
I think that, truly, this is an example of just how much Christian morality on justice is nonexistent among nominal believers.\footnote{nominal here because there's the whole Catholic thing of \say{faith without works is dead}, so if you don't change how you act based on information, do you really believe it?}

What is the goal of a punishment?  
To me, the answer is simple: as with literally everything that we do, the end goal of a punishment should be to bring the entire world to G-d and Christ.  
What does that mean in this context, though?

Generally, it means that reform based punishments should be seen as objectively better than retribution based punishment.\footnote{I say punishment not justice here, because starting here I mean Justice as a fundamental and independent of human belief thing, not simply what we have agreed as a society}  
The Church, in \say{Fides et Ratio}, was very clear that religious and scientific truth do not disagree.  
The idea that anyone has the right to end another's life, whether explicitly by capital punishment or implicitly by life sentences, comes from the belief that Church Fathers had that deterrence was an effective way to reduce crime.

That is not a question of faith and morals.

Deterrence's effect on crime rates lies exactly in the camp of a question that rational\footnote{since it's faith and reason, I'm using reason and rationality} thought can answer.  
Even the federal government, whose recent administration is explicitly in favor of retribution based punishment, \href{https://nij.ojp.gov/topics/articles/five-things-about-deterrence}{still acknowledges that harsh punishments do not reduce crime.}  
So, then, how can a Catholic support retribution based punishment?

This is not a rhetorical question, for all that its presence in my writing makes it inherently rhetorical.\footnote{because there's no way for another to respond in the moment.}  
I legitimately do not know of an argument for retribution based punishment that does not rely on the argument of \say{the Church Fathers said it was ok.}  
The Church Fathers are often wrong.  
Aquinas, the argumentative Catholic's favorite cudgel, famously did not believe in the immaculate conception.  
We don't question the validity of this.

We don't\footnote{or, at least, I hope that we don't. Given how much people hate the ratio part of fides et ratio lately, I'm not confident} question that daytime is bright because of the sun, as Augustine did.  
So then why is it that we question the Holy Father, who, seeing this, recently reminded all Catholics that we cannot support the death penalty.  
I have seen that some say that we still should allow it because we may not be able to keep someone imprisoned.  
Why is someone being imprisoned?

In the Christian ethical framework, we are punished only in so far as it is beneficial towards leading us to the Gospel.  
If we believe that crime is bad for getting all souls to heaven\footnote{which I don't think anyone explicitly states, but it is frequently assumed}, then we should work to stop crime in whatever way is most effective.  
The data show that means catching people, and giving them ways to rehabilitate.

If a murderer escapes custody and kills again, this is not somehow worse than killing them.  
If a murderer escapes and kills a thousand people, it is still no worse than killing them.  
All lives are infinitely precious.

Despite what we may have thought as children, infinity times one thousand is still infinity.

Of course, this line of thought quickly leads to pure pacifism.

Unfortunately, I now have a meeting with a friend, and so cannot finish the thought.  
That's probably fine, since I am on a huge tangent right now.

\section{Daily Reflection: 6 April 2025)}

I'm realizing that the way that these goals is structured is not entirely what I want right now.  
It kind of came up in yesterday's musing, but, while I don't think that today is going to be the day that I work to fix it\footnote{because I have far more obligations than I thought that I did}, I think that it's still probably worth thinking about.  
Right now I'm kind of leaning towards obligations\footnote{things that some external source expects from me. E.g. my thesis, the guitar with a friend}, growth, both autotelic\footnote{things that I want to get better at for their own sake. (I say this having yet to write the post about intrinsic motivation), but like composing, I want to be better at for its own sake, and I guess singing or guitar, etc} and means\footnote{things that I want to get better at to help with something else. E.g. I want to be faster at typing mostly so that I can get back to having my brain be the rate limiting step in my writing, rather than my keys, as it is right now. The site I'm training on helpfully lets you express speed in characters per second or minute, which makes far more sense to me than words per minute for a specific letter, even though I intellectually know that they're just linear conversions}, and how to improve in some other regard.\footnote{e.g. eat better, write letters, etc. I guess those go in self improvement, but... (you see why I have this issue)}

Anyways

\begin{itemize}   
\item Intentionality:  
\begin{itemize}  
\item At least hourly, stand up, drink water, take two deep breaths, and do a stretch of some sort

Absolutely did not do this yesterday.  
I was writing for a solid four hours and stood maybe three times, with absolutely no stretching.  
That being said, though, I do think that I was much better about drinking water yesterday, which is really good!  
Unfortunately, my bottle is just barely too low to drink from with good posture with my current desk layout.\footnote{wait that's such an easy fix since I have some 3 inch binders now. Woo gerbil hour is slightly more back. Issue is now that in order to have water bottle in good place, I also have to have hands in bad place. I really need a split keyboard, and then water can just go in the middle.}

\item Be proactive about avoiding overwhelm and when feeling overwhelmed, stop and figure out why.

Generally doing ok! I do find that I'm getting migraines more often, so it might be worthwhile to just start taking caffeine in the morning with the rest of my pills.  
\item Light a candle and read by candlelight each night. Along with this, leave all electronics outside of the bedroom and/or move them away at least an hour before bed time.

I lit the candle but did not do candlelight or electronics free.  
\item Candle time in the morning before electronics. Use the time for prayer  
  
See above.  
\item Focus on good posture, especially straight back and making sure that neck isn't awkwardly positioned.

Did decently during choir retreat, realized why so many doctors hate laptops while working yesterday.

\item Don't waste time, and in particular, be mindful about making sure to take breaks and rest. Especially make sure to do rest which revitalizes the me of tomorrow, rather than rest which simply keeps me in stasis.

Eh. I think that I was generally ok with the revitalizing aspect, because I feel well rested.  
Then again, I also slept in effectively two more hours today.  
I have to decide between feeling rested and feeling tired at night's end, I think.  
If I just force my way through this week, hopefully I can reset the schedule?

\item Interpersonal Relationships:  
\begin{itemize}   
\item Figure out what belongs in a normal letter to a friend.

Darn libraries being closed on the weekend.  
\item Get back into writing letters.

I woke up to a message from a friend today asking if I was still willing and able to help with a project today.  
Since my general rule of thumb is that in person events almost always trump unexpected things to a friend,\footnote{I realize now, and am very comfortable with} I agreed.  
That does mean I don't have a super free day, though.  
\item Work to message friends at desired intervals.  
\end{itemize}

\end{itemize}  
\item Professional:   
\begin{itemize}   
\item Do the Thesis and other research requirements. Upcoming deadlines:  
\begin{itemize}  
\item Brain dump about science communication (Overdue)  
\item Brain dump a publicly accessible chapter (Overdue)  
\item Have final convergences for the results I'm trying to reproduce (due 4/4)  
\item Draft of the first paper (due end of month, but I want to make sure that I've reupdated it sooner than later)  
\item Finish revising and editing the overview of a program chapter (due 4/7) and send it to the boss  
\item Revise the Science communication and publicly readable chapters (due 4/7)  
\item Send the science communication chapter to the boss (4/14)  
\item Brain Dump the background to the program (4/14)  
\end{itemize}  
\item Only do the work I feel called to when I've finished the tasks set to me for the day or outside of normal working hours (post 1725)  
\item Start making the giant citation document so that I don't have to search for citations later.  
\item Work towards future career:   
\begin{itemize}   
\item Figure out the difference between my public-facing and field-facing presentation affects. As I focus on becoming a better presenter, I need to become aware of the difference and how to switch them.  
\item Need to look for jobs  
\end{itemize}   
\end{itemize}   
\item Health:  
\begin{itemize}   
\item Spiritual:   
\begin{itemize}   
\item Get back into the Lenten goals (pray chaplet of St. Michael, give money equal to amount I'm spending on myself, stop scrolling social media, stop playing video games)  
\item Be intentional about prayer. That means both making time for prayer and actually doing it.  
\end{itemize}   
\item Physical:   
\begin{itemize}   
\item Start focusing on posture again, especially while sitting.  
\item Go to group fitness classes more regularly and more often. If not, do workout at home  
\item Feed myself simply and healthily. Healthy here means trying to generally avoid processing.  
\end{itemize}

\item Mental:   
\begin{itemize}  
\item Clean Life:   
\item Remove dirt and clutter from physical spaces (standard definition of clean):   
\begin{itemize}  
\item At least once a week, each room has nothing on the floor  
\item At least once a week, all surfaces which are not inherently storage are cleared off  
\item At least once every two weeks, each room is vacuumed  
\item At least once every month, all non-storage surfaces are explicitly washed/cleaned  
\item At least once a week, I get rid of at least one item that I notice (meaning throw away or in rare circumstances gift or donate)  
\item Clean sight lines. Is my space set up in a way that orients me towards my goals for the space? If not, how can I make it so?  
\end{itemize}  
\item Spend time each day thinking about the goals for the day, and getting them out of my head and onto the page.  
\item Continue to explicitly confront the voice in my head that says that people hate me.  
\end{itemize}  
\end{itemize}   
\item Hobbies:   
\begin{itemize}   
\item Reading  
\begin{itemize}  
\item Start reading and returning the library books I have.  
\item Finish the book on mindfulness I started. (also make a list of the exercises in the book and try them out)  
\item Read more poetry  
\end{itemize}

\item Music:   
\begin{itemize}   
\item Work on guitar  
\item Learn the songs that jam partner suggested and/or requested I learn  
\item Get back into the album.  
\end{itemize}   
\item Writing:  
\begin{itemize}   
\item Write poetry more often, ideally nightly.  
\item Find a way to add meta data to my blog posts and then add the meta data  
\item Not only write blogs, but also post them.  
\item Get back into writing the web novel  
\item I talked about this yesterday, but typing practice. Right now my goal is 5 characters per second, and sadly all my data just got lost between trying to log in through a few sources.  
That's ok, data are data  
\end{itemize}   
\item Other hobbies, do them.  
\end{itemize}   
\end{itemize}
\end{document}