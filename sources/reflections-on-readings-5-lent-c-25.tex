\documentclass[12pt]{article}[titlepage]
\newcommand{\say}[1]{``#1''}
\newcommand{\nsay}[1]{`#1'}
\usepackage{endnotes}
\newcommand{\B}{\backslash{}}
\renewcommand{\,}{\textsuperscript{,}}
\usepackage{setspace}
\usepackage{tipa}
\usepackage{hyperref}
\begin{document}
\doublespacing
\section{\href{reflections-on-readings-5-lent-c-25.html}{Reflections on the Gospel}}
First Published: 2025 April 6

I guess it shouldn't surprise me that I have only one other reflection on a Lenten gospel, since I've never been good at this blog in the springtime.
\section{Draft 2: 1 April 2025}

N.B. the two drafts are markedly different, and they cover somewhat very different things

Time has continued its ceaseless march ever forward, and marching, passed March by.  
My experience has been detaching from time again, which isn't great, but is true.  
With that in mind, I will really reaffirm here my goal to write on this site more often.  
I do not know what happened this past month, or really this year so far, and I do not like that.

Still, in the interest of giving myself grace, it's good to remember why I wasn't blogging.  
The past few days in particular, I was really focused on the derivation for some equations that I just realized might be helpful for me.  
Looking with a little more intention, I realize that I let myself be consumed by that project, and the consumption is not particularly healthy.  
Moving forward, I want to be better at keeping myself outside of obsession.

Five great things from March:  
\begin{itemize}  
\item Tentative approval for my defense timeline  
\item Received an award for outstanding service to the university  
\item Got a massage  
\item Had friends respond positively when I asked them about contact rates  
\item Started writing choral music  
\end{itemize}

This coming April, five things I'm looking forward to are:  
\begin{itemize}  
\item Leading the science experiment this coming Saturday.  
\item Going to a friend's thesis defense  
\item Finishing and submitting my first first author paper  
\item Writing more with a close friend  
\item Becoming more of the person that I would like to be, both by changing and by changing my goals.  
\end{itemize}

Particular areas I want to focus on this month:\footnote{since this is a reflection document, I feel much more ok with using lists, for some reason. I don't think that's an impulse I need to delve too deeply on, though}  
\begin{itemize}  
\item Prayer. This has been at the bottom of my priorities for months now, and I am finally starting to notice the consequences. I want to reconnect with the Almighty, and that requires prayer.  
\item Physical health. I know that I have more energy and feel better when I work out more often, and so I really need to make that an emphasis  
\item Avoiding obsession. I know that I can let myself get stuck on a problem until I solve it, and I also know that doing so means that I fall behind on everything else that I want to be doing. When I catch myself obsessing, stop.  
\item Intentionality in general. I want to make sure that what I do and plan to do is what I want to be doing. In particular, I want to make the choices I want to do the path of least resistance, which I think means making more schedules and forcing myself to follow them.\footnote{see: avoid obsession}  
\item Working on hobbies. I know that I feel more complete as a human when I do things other than research. More than that, the time I spend on research is improved by taking time off.  
\end{itemize}

While that is five items, many of them intersect in a variety of ways.  
All in all, I know that I am better when I don't rely on my internal memory, but instead have it extended into a readily available source.  
I need that source, however, to not be as messy as my own mind, which is its own problem.  
Still, the better I become at setting and enforcing boundaries, the better all of this will become as well.

I look forward to seeing the person I become, and I look forward to seeing the person I was and am as I continue to reflect.

\section{Draft 1: 1 April 2025}  
It's been somehow another month.  
Time continues its aggressive march forward, and it has left March behind.  
Despite the fact that I had exactly three (3)\footnote{I always forget which way of writing the number is supposed to be in parentheses} blog posts for the month, it seems good for me to do my usual monthly reflection.

I really haven't been doing much blogging at all this year, which is a bit of a shame, though tracks with the fact that I have no real sense of how time has moved this year.  
Still, it's always good for me to have some highlights from the previous month:\footnote{this is a thing that I've always done as a list, so I'm going to keep it as one!}

\begin{itemize}  
\item Met with my committee and set a tentative thesis defense date! It's so wild to me that they all think that I can be done in just a few more months.  
\item Got an award for dedication to giving talks to the broader state-wide community  
\item Got positive responses from my friends when I asked them all about the rate that they wanted me to message them.  
\item Got a massage!  
\item Started writing choral music again.  
\end{itemize}

It's interesting to me that a full majority of these are things that happened to me\footnote{as defined by \say{got}}, but that's probably fine.

What are five things that I'm looking forward to in the coming month?  
\begin{itemize}  
\item Leading a science exploration day this Saturday!  
\item Going to a friend's thesis defense  
\item Finishing and submitting my first first author paper  
\item Going to a farmer's\footnote{farmers'? farmers? great question} market  
\item Writing more with a close friend!  
\end{itemize}

Normally I would now go through the goals that I had last month and see how I did, and then create the goals for the coming month.  
However, my goals list is currently a living document, so that's kind of taken care of.  
Still, probably good to at least reflect on them as a macro level.

I still like the division between Professional, Health, and Other, and I think that the upcoming deadlines will be nice as well.  
It might make more sense for me to move upcoming deadlines into the Professional, especially since effectively all of the deadlines that I have are in relation to the thesis I'm writing.  
Professional otherwise just reminds me that wow I am not doing a good job of actually working towards a future career.

Health being broken into mental, physical, and spiritual still seems good, and the order seems reasonable to me still.  
I think that the cleaning goals will become reasonable once I've achieved them a single time.  
Still, that does mean that I need to start prioritizing them more.  
Today, much as I want to work on the idea for my research that I had as I was getting ready to leave work on Friday, I should\footnote{here being used in the sense of I'm realizing this might be better for me, not in a judgemental way (if you're reading this and confused why I've started explicitly tone tagging certain words, it's because I was implicitly advised that doing so might be good for me)} probably instead go home and clean, especially since I'll be busy tomorrow night.

Cleaning my life belonging in mental health remains kind of odd to me, but I can't really say that I disagree with it, since I do feel like it's my mental health that suffers the most from not having a clean life.  
I've added the goal of candlelight time each night, and I'm realizing that part of my problem is not feeling like I have a comfortable place to sit in my apartment outside of my bed.  
I don't know if that means that I should get a new couch, a new chair, make a bundle of blankets and call it a sitting location, or what, but it is certainly something to consider.  
Goals for the day remains a good thing for me to do.  
Doing it this morning certainly helped me feel far less stressed and frantic than yesterday, where I did not take the time.  
In general, I think that I should probably just make more of an effort to be intentional, in the senses of:  
\begin{itemize}  
\item Do the things I know prevent me from feeling overwhelmed. (e.g. fill my water pitcher in the morning when entering work, check my list of tasks and make sure they're still accurate)  
\item When I feel overwhelmed, take the time to reflect and figure out why, and then write down what's making me feel so.  
\item Take at least a little time every hour, at the very least, to stand, stretch, drink water, and take two deep breaths.\footnote{I'm not entirely sure how this is going to happen while I TA, but it's something to try}  
\end{itemize}

I don't really know where intention should go, but I think that I might like it to be its own category.  
This month, I think that physical health is more important to me than mental health\footnote{That came out wrong, but like I think that I need to focus more on my physical well being than my mental well being, since the latter is in what feels like a better space than the former}  
Hmm, only having three physical health goals is a little lacking, for all that they do really make space for how I want to improve.  
I suppose that making an actual diet plan could be good\footnote{i.e. when I'll cook how much so that I have meals for so long}, and that might be a worthwhile activity for tonight while cleaning.  
Honestly, posture probably belongs in intentionality, so I'll have it double listed.

Spiritual health remains the highest priority, so it remains at the top of the list, and I'm trying to figure out where to add the time.  
Candles in the morning for the chaplet could be a good starting spot.  
I'm realizing that the candle is better placed in intentionality, so have moved it there.

Interpersonal relationships I think should be moved into intentionality, since the goal of the interpersonal relationships is to be more intentional.  
I also think the bit about rest can be there as well!  
Wow this is getting revamped a ton.

I'd like to have a schedule of blog posts, which probably deserves to live somewhere outside of the daily posts. Then again, it could be a fun way for readers to see what they're going to get.  
Eh, I think that it's better to let me decide what needs to come on each day, for all that there are a number of posts that I do really feel like I need to have on certain days.  
I should make a list of the lists I'm trying to make right now\footnote{which I'm going to do in a different document}, because I know that otherwise I'll forget it all.

Other is a category that's grown just so much since the start of the document, and it's probably worthwhile to consider breaking it apart.  
Right now I basically have the reading goal of getting through the library books, which is arguably an intentionality thing, doing music, writing things that could be fun, and other artistic endeavors I'd like to do.  
I have forgotten what embroidery physical relic means, and that's a bit of a shame.  
Oh, duh, I literally just meant that I should embroider something.

What can I call these things? Hobbies? That's probably a better term, and then I can move the other creative ideas that I generally have into their own, again separate, living document, since they're not going to happen on the blog.  
I don't really think that I need to be actively working on all of the hobbies at once, and so it could be good to treat them instead as potential options, which I put in my schedule as items for restorative rest.

Great, that's really the whole new set of daily goals (which I've put in the modified Daily Reflection section, which I did not leave as a record of the changes I made, because that goal of the blog has been abandoned.  
Let's clean up these thoughts in the next draft and then call it good to post.

\section{Daily Reflection: 1 April 2025)}  
N.B. Since I've realized that I will often give up halfway through a post, I've decided that I'm going to start each post now with the daily set of reflections.

\begin{itemize}   
\item Intentionality:  
\begin{itemize}  
\item At least hourly, stand up, drink water, take two deep breaths, and do a stretch of some sort  
\item Be proactive about avoiding overwhelm and when feeling overwhelmed, stop and figure out why  
\item Light a candle and read by candlelight each night. Along with this, leave all electronics outside of the bedroom and/or move them away at least an hour before bed time.  
\item Candle time in the morning before electronics. Use the time for prayer  
\item Focus on good posture, especially straight back and making sure that neck isn't awkwardly positioned.\footnote{I can't remember the word right now}

\item Don't waste time, and in particular, be mindful about making sure to take breaks and rest. Especially make sure to do rest which revitalizes the me of tomorrow, rather than rest which simply keeps me in stasis.

This weekend wasn't great for non-stasis activities, but I've got ideas for how to do so moving forward.

\item Interpersonal Relationships:  
\begin{itemize}   
\item Figure out what belongs in a normal letter to a friend.  
\item Get back into writing letters.

Wrote one on Sunday!  
\item Work to message friends at desired intervals.

I need to make a list of the friends who responded and their preferred intervals.  
\end{itemize}

\end{itemize}  
\item Professional:   
\begin{itemize}   
\item Do the Thesis and other research requirements. Upcoming deadlines:  
\begin{itemize}  
\item Brain dump about science communication (Overdue)\footnote{brain dump meaning I sit down at my computer and start typing until the well runs dry in regards to the idea. Avoid significant editing or revising wherever possible}  
\item Brain dump a publicly accessible chapter (Overdue)  
\item Revise the overview of my program, which also means finishing writing it up\footnote{something I just realized is that I never added time to make graphics, or a stage where I would explicitly put them in. I think that I plan to do so after the first time my advisor looks at the drafts, because no point making something she thinks is pointless} (Overdue)  
\item Have final convergences for the results I'm trying to reproduce (due 4/4)\footnote{I know this is more than a week away, leave me alone}  
\item Draft of the first paper (due end of month, but I want to make sure that I've reupdated it sooner than later)  
\end{itemize}  
\item Only do the work I feel called to when I've finished the tasks set to me for the day or outside of normal working hours (post 1725)  
\item Start making the giant citation document so that I don't have to search for citations later.  
\item Work towards future career:   
\begin{itemize}   
\item Read the recommended readings about science communication. Just did.  
\item Do the reflections that were recommended to me (mostly focused around why I care about science communication)

Tomorrow.  
\item Figure out the difference between my public-facing and field-facing presentation affects. As I focus on becoming a better presenter, I need to become aware of the difference and how to switch them.  
\item Need to look for jobs

Have started! Now I need to start actually filling out applications, etc.  
\end{itemize}   
\end{itemize}   
\item Health:  
\begin{itemize}   
\item Spiritual:   
\begin{itemize}   
\item Get back into the Lenten goals (pray chaplet of St. Michael, give money equal to amount I'm spending on myself, stop scrolling social media, stop playing video games)  
\item Be intentional about prayer. That means both making time for prayer and actually doing it.  
\end{itemize}   
\item Physical:   
\begin{itemize}   
\item Start focusing on posture again, especially while sitting.  
\item Go to group fitness classes more regularly and more often.

I think that a secondary goal could be that if I don't go in the morning and there's not a class I can make it to in the evening that I'll just do one myself at home. I have a yoga mat and a family subscription to some yoga classes.  
\item Feed myself simply and healthily. Healthy here means trying to generally avoid processing.

Honestly, I didn't realize how much I actively crave artificial dyes. I still have some candy that was gifted to me, so I'm also going to keep eating it.  
\end{itemize}

\item Mental:   
\begin{itemize}  
\item Clean Life:   
\begin{itemize}   
\item Remove dirt and clutter from physical spaces (standard definition of clean):   
\begin{itemize}  
\item At least once a week, each room has nothing on the floor  
\item At least once a week, all surfaces which are not inherently storage are cleared off  
\item At least once every two weeks, each room is vacuumed  
\item At least once every month, all non-storage surfaces are explicitly washed/cleaned  
\item At least once a week, I get rid of at least one item that I notice (meaning throw away or in rare circumstances gift or donate)  
\item Clean sight lines. Is my space set up in a way that orients me towards my goals for the space? If not, how can I make it so?  
\end{itemize}

So far I'm behind, but the goal right now is really getting to a baseline where that's possible  
\item Spend time each day thinking about the goals for the day, and getting them out of my head and onto the page.

Yesterday I started by just working and wow I felt horribly overwhelmed right away. Today I spent like 10 minutes looking over the schedule I had previously set for myself, and I feel much more ready for the month  
\item Continue to explicitly confront the voice in my head that says that people hate me.

Have been, and will continue to do so.  
\end{itemize}   
\end{itemize}  
\end{itemize}   
\item Hobbies:   
\begin{itemize}   
\item Reading  
\begin{itemize}  
\item Start reading and returning the library books I have.

I want to set the goal of at least one book a week, but that may not be possible, given so much stuff.  
\item Finish the book on mindfulness I started. (also make a list of the exercises in the book and try them out)  
\item Read more poetry  
\end{itemize}

\item Music:   
\begin{itemize}   
\item Work on guitar  
\item Learn the songs that jam partner suggested and/or requested I learn  
\item Get back into the album.  
\end{itemize}   
\item Writing:  
\begin{itemize}   
\item Write poetry more often, ideally nightly.

I should really start doing this again, and also on Sunday I lit a candle and it made my night better. If I do those together I think that it will be better for me.  
\item Find a way to add meta data to my blog posts and then add the meta data\footnote{at least moving forwards and hopefully also working backwards through the many posts that I have}  
\item Not only write blogs, but also post them.  
\item Get back into writing the web novel  
\item Write other fiction. Ideas include:  
\begin{itemize}  
\item The book I made a document about in late January  
\item A story told through bullet points  
\end{itemize}  
\end{itemize}   
\item Embroidery:  
\begin{itemize}  
\item Make pattern  
\item Make embroidery from pattern  
\item Explore stitches other than the single stitch I've used exclusively for all of the projects I've finished.  
\end{itemize}  
\end{itemize}   
\end{itemize}
\end{document}