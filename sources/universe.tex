\documentclass[12pt]{article}[titlepage]
\newcommand{\say}[1]{``#1''}
\newcommand{\nsay}[1]{`#1'}
\usepackage{endnotes}
\newcommand{\1}{\={a}}
\newcommand{\2}{\={e}}
\newcommand{\3}{\={\i}}
\newcommand{\4}{\=o}
\newcommand{\5}{\=u}
\newcommand{\6}{\={A}}
\newcommand{\B}{\backslash{}}
\renewcommand{\,}{\textsuperscript{,}}
\usepackage{setspace}
\usepackage{tipa}
\usepackage{hyperref}
\begin{document}
\doublespacing
\section{\href{universe.html}{Universe in the Park (Part 3? 1?)}}
First Published: 2023 June 2
\section{Draft 1}
As I mentioned \href{reflection-may-23.html}{yesterday}, I gave a talk this past Saturday.
It was a part of my university's\footnote{or the state, or the astronomy department, depending on where we draw the line (or the donor who funds the talks I suppose even)} \say{Universe in the Park} series.
I went to a State Wildlife Area\footnote{which is like a state park except focused on research not tourism} and gave the first of what is currently\footnote{very actively debating signing up for two more} scheduled to be six\footnote{UitP (as I abbreviate it), I'm giving at least two other formal talks outside of this program} talks this summer.

My talk changed a lot from what I gave last year.
Last year, I had titled my talk \say{Prebiotic Chemistry in the Interstellar Medium,} which is not, in retrospect, as good of a title as I had thought.
With the added wisdom of almost ten additional months, I revised the slide deck almost completely.
It is now titled \say{Searching for the Origins of Life in the Coldest Regions of Space}, a much more evocative title.\footnote{the clever amongst you may notice that the two talk titles are reducible to each other. That is somewhat intentional. (for those not in the loop, searching for the origins of life is just a less (differently?) jargoned way of saying prebiotic chemistry, and the coldest (relevant) regions of space are in the interstellar medium)}

My first talk went really well I think, all things considered.
The general format of a Universe in the Park presentation is giving a short-ish lecture\footnote{which I claimed would be forty five minutes despite knowing in my heart that it would be more like 25}, answering audience questions until they run out,\footnote{favorites from this past one include \say{so what do you think of astrology?} (asked by a 20 something), \say{why do stars look blue and yellow?} (asked by a youngish schoolchild [side note, I do love that schoolchild is a great way to say \say{I know that they're older than four, but that's all I can commit to}]), and \say{so does space really smell like raspberries?} (asked by a father, to which I responded something along the lines of \say{it's the same ester that makes rum smell, so they could have had a much more fun title with \nsay{space smells like rum}})} and then using a telescope until either the park is closed or everyone is bored.
Despite now being trained to use the telescope's GPS features, I used very little of them.

Mostly, people wanted to look at the moon, which is fair.
Unfortunately the viewfinder\footnote{read: weak telescope coupled to powerful telescope to allow for rough aligning} had gotten misaligned from the viewing scope, and I didn't want people to have to wait for the\footnote{optimistically} five minutes it would take to align them together.
Instead, everyone got to see the moon through the viewfinder, and then I coupled the two.

It was then that I learned just how quickly the moon moves through the sky.
I found that I needed to adjust the sight every four or five people.\footnote{interestingly, I've realized that in my writing I still naturally default to objects first, but I then immediately change it (here I initially began the sentence \say{every four or five people}, but changed it before finishing the sentence. I'm sure musing about how my writing has changed almost 220,000 words published over two web serials later would be interesting, but that's probably best left to another time)}
Every time that I did, at least three mosquitoes took their chance to bite my face, leading to more than a few jerks.
Thankfully, the moon is\footnote{was?} bright enough that I was able to quickly adjust it back by following its aura.

After everyone viewed the moon and was impressed by its many craters\footnote{I promise that sentence is not sarcastic at all. Maybe hyperbolic, but I think to at least a first order approximation everyone was}, I moved the view to Venus, both to use the GPS alignment and because it's the only other object I think is cool to look at in the sky.
Switching out the optic for a more powerful one, people were able to see that Venus, like the moon,\footnote{Moon? I forget how capitalization rules work with celestial bodies} has shaded regions.

At that point, the number of people there was dwindling.
I pointed the telescope to a few stars\footnote{spoiler, stars look the same}, and people all filed out by ten pm or so.
It still wasn't even fully dark, not that I'm complaining.

I even got feedback from the park person.\footnote{i forget his actual title}
It's mostly irrelevant, except for a quote which I want to exemplify more daily, \say{It covered the science behind space much more in depth than I thought it would, which was fine. But he also, related it well to what people would know about, including kids. He also was not thrown off by any questions that were asked}.
I choose to interpret this as meaning I went in depth but still kept people engaged.
What more could I hope for as a scientist communicating to the public?

(508/362)
\end{document}