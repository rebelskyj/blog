\documentclass[12pt]{article}[titlepage]
\newcommand{\say}[1]{``#1''}
\newcommand{\nsay}[1]{`#1'}
\usepackage{endnotes}
\newcommand{\1}{\={a}}
\newcommand{\2}{\={e}}
\newcommand{\3}{\={\i}}
\newcommand{\4}{\=o}
\newcommand{\5}{\=u}
\newcommand{\6}{\={A}}
\newcommand{\B}{\backslash{}}
\renewcommand{\,}{\textsuperscript{,}}
\usepackage{setspace}
\usepackage{tipa}
\usepackage{hyperref}
\begin{document}
\doublespacing
\section{\href{silly-putty-and-pitch.html}{Silly Putty and Pitch, or What I've Been Learning}}
First Posted: 2018 October 18
\section{Draft 1}
So, as I've mentioned before, I'm taking a class on polymers.
And, so far I've learned\footnote{among other things} that the most important property of a polymer is its \say{glass transition temperature.}
That is, the temperature when it changes from being stiff to being not stiff.

Different polymers behave differently after the glass transition temperature.
Some become rubbery.
Some, the more crystalline ones, stay stiff.
Some turn into a liquid.\footnote{also apparently only crystals melt}
Some probably do something else.

But!
Apparently that temperature is also time dependent.
That's weird to me, since melting doesn't feel like the sort of thing that is affected by time.
But, the longer that the time frame is, the more like a liquid a polymer acts, and the shorter, the more like a solid.
So, for short enough time frames, things with a glass transition of very low can remain undeformed at high temperatures.
Contrastingly, pitch, which my professor describes as \say{hard and brittle,} flows over long enough time frames.

So, that's why silly putty is stretchy when you stretch it, breaky when you smash it quickly, and pancakes if you let it sit.
I thought it was cool.

Post Script: As I reread the title, I now am thinking of a musical piece featuring silly putty.
I'll keep y'all updated on the future of that.
\end{document}