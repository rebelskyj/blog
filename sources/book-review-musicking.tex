\documentclass[12pt]{article}  
\newcommand{\say}[1]{``#1''}  
\newcommand{\nsay}[1]{`#1'}  
\usepackage{endnotes}  
\newcommand{\B}{\backslash{}}  
\renewcommand{\,}{\textsuperscript{,}}  
\usepackage{setspace}   
\usepackage{tipa}  
\usepackage{hyperref}  
\begin{document}  
\doublespacing  
\section{\href{book-review-musicking.html}{Review of Musicking}}  
First Published: 2025 October 6

\section{Draft 1: 6 October 2025}

I am not irregularly asked what my favorite book is.  
There are any number of ways to answer it.  
I tend to lean on one of a few answers: the books I can read the most often, the books I have read the most, the books that have most inspired me, the books that most changed the way I move through the world, the books that I think will most impress the asker.  
It is the penultimate of these which concerns Christopher Small's \textit{Musicking}.

I've written here before about the semester I spent on an independent study focused on what it means to listen to music.  
I was assigned any number of readings and listenings, and many of them were great.  
The one that changed me most, however, was an article by Small also called Musicking.  
My biggest takeaway from that article was that anyone involved in the creation of music, down to the person selling tickets at the window or the movers who put the piano on stage, is a musician.

The book takes a similar, if more in depth\footnote{as one might hope, given that it is a full book} look at the question: what is music?

Small's major argument is that music is not a noun, it is a verb.  
Musicking is the act we do, which is full of the different relationships we have to each other, the repertoire, and so on.  
This book also makes clear just how strange the modern orchestral concert is in the grand scheme of the world.  
Small takes aim at Platonism, pointing out the ridiculousness of an abstract perfect form.  
His take on music is best summed up on page 218: \say{just as there is no such thing as music, neither is there such a thing as beauty.}  
That is, although things can be beautiful, beauty is not a thing which exists independent of the experience.  
Likewise, music exists solely as performance and interrelation.

While reading the book, I was reminded of just how much I truly am unlearned in musical literature.  
I had vague notions of many of the points he hit about masculine and feminine subjects, semiotics, and the like.  
However, I would not have anywhere near the ability to write about the subjects\footnote{hah, get it, because subject is a thing in music but it was also a subject of the book} that he treated as mere set dressing.

Also in the book were some beautiful lines and thought provoking ideas.  
I have scattered notes, which I'm tempted to present as is\footnote{the reader is welcome to request them}.  
I'm also tempted to simply say \say{read it yourself}.  
I'll take a middle ground, however, and point to some portions I found particularly moving or thought provoking.

One of the chief points Small comes back to again and again is the idea of ritual.  
All music, he argues, has ritual significance.  
The ritual of listening to a walkman\footnote{dating the author}, though different than the ritual of going to a theatre, is still one in itself.  
Similarly, there is an argument that the modern orchestral hall is sacred in the classic meaning of the word; it cannot be used for the day to day practical, and is only useful for its dedicated purpose.

Small, like I love to, attacks equal temperament briefly.  
Equal temperament is mathematical and abstract, completely divorced from what a human would naturally come to, much like modern society.  
I tend to dislike equal temperament for how it makes each key the same, which is part of his argument as well.  
It's only a short portion of the book, almost an aside.

At least twice he brings up the fact that no pre-modern music was written with the idea that we would listen to it again and again.  
Orchestras were by and large written to be performed at a specific event, and then never again.  
Even those which were composed for replaying require an orchestra.  
Before modern recording technology, that put a hard limit on the number of times we could listen to Eroica.

Finally, I'd like to list a few quotations that for whatever reason I felt compelled to stop and write down.  
There are some quotes I said \say{find on page X}, but those are clearly less powerful to me.

124: \say{Those who talk of delayed gratification ought to be made to sit through all of the \textit{Terminator} movies, followed perhaps by the \textit{Die Hard} series. No delayed gratification there either; they grip, as they are meant to grip, from the first frame.}  
This quote comes from a section about how the development of the Western canon is about delaying gratification more and more.

197: \say{relationships between performers and listeners may be close, intimate, and even loving, as when the lover or the suitor sings or plays to the beloved or the sought.}  
I don't know what about this quote struck me so hard, but something in it is just so beautiful to me that I cannot but rewrite it.  
I think it might be the repetition of love in loving, lover beloved?

202: \say{like all wind instruments it is animated into life by the breath from his body, the most intimate relationship one can have with a musical instrument}. I disagreed with this take, because I find that something like a cello can be more intimate, even if I cannot express quite why.  
Something about holding it and embracing, maybe?

also 202: \say{Simple it may be in its construction, but primitive it is not}. This, like the above quote, is about a hand-carved flute.  
He points out that the modern conveniences like slides and valves make playing far easier.  
That's such a great point, and I am honestly kind of surprised that he didn't tie in the whole \say{because the shepherd can make more microadjustments, his music is more free},\footnote{not direct quote, to be clear} or something about the constraints the modern instrument cause for the modern composer.

212: \say{in my opinion any music teacher caught ... using the epithet tone-deaf of a pupil should be sacked on the spot.}  
I too agree that there are few who are tone deaf, and that the teacher should serve as an encourager, not a discourager.

213: \say{all musicking is ultimately a political act}.  
It's important to remember that everything is political.  
By definition, the relationships we create have political meaning.

220: John Cage used to respond to interview questions he didn't like \say{I don’t find that a very interesting question}.  
There's a kind of power in that.

Also, Mozart apparently stopped practicing at the age of 7, because his day to day music work was enough for him to stay in musical shape.

All in all, this book was incredibly fun to read, powerful, and relatively easy, though very deep.  
I would highly recommend that others read it, and I have every intention of reading it again.

\section{Draft 0: 17 September 2025}  
Before I start this review, I just have to say that it is wild to me how much books from the late nineteen hundreds feel both like they just happened and happened centuries ago.  
Small, for instance,

\section{Draft -1: 16 September 2025}  
Ok so technically I haven't finished \textit{Musicking} yet, but I'm optimistic.  
I do find it interesting that Small frames the entire book through the lens of an orchestral concert.  
I was assigned an article about it back when I was doing an independent study on listening to music.  
It was absolutely one of the most influential writings that I've ever encountered, if only for how it makes me think of music.

Small's key argument is that music is not a thing but an action.  
When we music, we are listeners or performers or dancers or stage hands.

I love this take because it helps me to deal with that most fundamental urge of the scientist in me: labeling what is and isn't music and who is and isn't a musician.  
Music is what we make of it, and musicians are those that make it.

I enjoyed greatly the aside on Cartesian dualism and especially the way that Small makes a point that so much of the industrialization we live and think under is like the water a fish has no name for.  
I hadn't considered the mind matter dualism really at all, especially in the context of music.

\section{Daily Reflection: 6 October 2025}

\begin{itemize}

\item Did you stretch?

No.

\item Did you attempt to pray something rote?

Whoops!

\item What's going on in writing?

... look at this at least.

\item How's work?

Day one down! Lots of meetings, lots more tomorrow.

\item Reading?

Bus ride was nice.  
Finished Musicking, made progress on another book.

\item Sleep?

Woke up fine today!

\item Water?

Not enough.

\item Food?

I think enough

\item Cleaning?\footnote{new today}

...

\end{itemize}

Current Pen List\footnote{for my own posterity, mostly}

\begin{itemize}  
\item Hongdian Black with Fude Nib: Monteverde Ocean Noir. 10/6

Adore the color, almost feels low saturation but in a very saturated way.  
Maybe low vibrancy?  
\item Jinhao Shark: Diplomat Sepia Black. 10/6

This pen now lives at home, rather than traveling.  
I don't know how to feel about sepia black.  
Kind of weird color.  
\item Pilot Preppy: Diamine Bilberry. 10/6

Nice solid blue.  
\item Shaeffer: Private Reserve Ebony Green. 10/6

Absolutely adore the green wow.  
\item Diplomat: Diplomat Caramel. 10/6

Fun reddish-orange color.  
Does read very caramel.  
\item Kaweko: Stipela Sepia. 10/6

Nice bookish color.  
\item Monteverde: Diplomat Burgundy. 10/6

Mmmm red. This one will live at the office.

\end{itemize}

\end{document}