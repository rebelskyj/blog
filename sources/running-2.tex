\documentclass[12pt]{article}[titlepage]
\newcommand{\say}[1]{``#1''}
\newcommand{\nsay}[1]{`#1'}
\usepackage{endnotes}
\newcommand{\1}{\={a}}
\newcommand{\2}{\={e}}
\newcommand{\3}{\={\i}}
\newcommand{\4}{\=o}
\newcommand{\5}{\=u}
\newcommand{\6}{\={A}}
\newcommand{\B}{\backslash{}}
\renewcommand{\,}{\textsuperscript{,}}
\usepackage{setspace}
\usepackage{tipa}
\usepackage{hyperref}
\begin{document}
\doublespacing
\section{\href{running-2.html}{On Running Another Year}}
First Published: 2023 April 13
\section{Draft 1}
\href{running-1.html}{Last time} that I mused about running, I talked about the Doctoral Degree Dash that I did last summer.
In it, I ran further than I thought I could.\footnote{To be fair, I didn't think that I could run 5 miles, which was (is?) apparently untrue}

Since then, I've tried running off and on.
Lately I've been more off than on, but even when I try, I find that I can only really run around a mile in the indoor track I'm using before I lose the ability to keep going.

This week has been shockingly warm here, and it's made me want to get outside and run.
So, tonight I decided to run while praying a rosary,\footnote{something I did a fair amount last summer} to see if that could help motivate me to run longer.

In total, I ended up running just under three miles.
I think that there were likely N causes to that.\footnote{I'm going to try to actually write these posts without going back more going forward, and when I'm not sure how many causes or reasons or items in general lists I have, I normally either pick a number arbitrarily or put N.}

First, I actually felt like the distance measurement was accurate.
When I run on the indoor track, it claims that 6.6 laps equal a mile, but my watch seems to think it's closer to eight.

Second, the constant variety in an outdoor run as compared to indoors.
When I run on the track, I'm running in a small loop over and over, so it's harder to want to keep going.
On the other hand, when I run outside, especially in the latter half of the run, the faster I go, the sooner I get home.

Third, I actually did something with my mind to stop me from thinking about how miserable it is to run.\footnote{So here N=3}

I found out my watch now gives me a lot of statistics about my runs.
For instance, my heart rate basically never dropped below 175 while I ran, which is an incredibly high feeling number.
My pace also seems to be fairly well correlated to my stride length, which makes sense to me.
When I get tired my steps certainly get shorter.

Anyways, I just thought it was interesting how different it feels to run outside than inside.
I'm hoping to get in better physical shape this summer, but summer kind of starts now.
\end{document}