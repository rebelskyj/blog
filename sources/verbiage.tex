\documentclass[12pt]{article}  
\newcommand{\say}[1]{``#1''}  
\newcommand{\nsay}[1]{`#1'}  
\usepackage{endnotes}  
\newcommand{\B}{\backslash{}}  
\renewcommand{\,}{\textsuperscript{,}}  
\usepackage{setspace}  
\usepackage{tipa}  
\usepackage{hyperref}  
\begin{document}  
\doublespacing  
\section{\href{verbiage.html}{On What I Call This Writing}}  
First Published: 2025 April 16

\section{Draft 1}

I don't know what to refer to these writings as.

The obvious answer is to call it a blog, but since blog comes from web log, meaning a log on the internet, I'm not sure that it's the best term.  
After all, these posts are, at least hopefully, less a factual recounting of elapsed time and more a series of explorations into ideas.

My father, who I copy so much of this site from, calls his writings musings.  
Trawling through his site, I eventually saw that, as I see in the early posts, he initially called the posts essays.  
A commenter pointed out that not all of his posts are, technically speaking, essays.\footnote{I don't know if I agree with that commenter, but, lacking context, I'll trust taht the common usage at least might differ}  
Since he frequently refers to his muse in the writings, calling the writings musings became a next step.  
As far as I can tell, that's the extent of his reasoning.\footnote{is this bait to see if he still reads this? maybe}  
To muse is to contemplate or think deeply, and so there's something to be said for the idea that, if I am thinking deeply, then I am musing.

Essay, being his initial title, is another way to refer to what I'm writing.  
This has the benefit of sounding a lot more pretentious\footnote{for example, \say{Oh, yeah, I'm working on a series of essays} versus \say{Oh, yeah, I have a blog}}.  
It comes from the French for \say{to attempt}, and initially were\footnote{nominally, at least} used as a way to \say{attempt} to put thoughts into writing.\footnote{it was here I took a forced three hour break}  
I don't really know if this is so much about me attempting to put thoughts into writing as it is developing thoughts through writing, but it's still something to consider.

Post is another easy option, since it's sort of the default thing that most social media\footnote{I hate that we've turned this plural into both the plural and singular} uses.  
It, as far as I can guess and tell from three seconds of research, refers to the fact that when you wanted to distribute information broadly,\footnote{I really hope that broadside comes from this and not vice versa} you could affix writing to a physical post.  
I don't love it, and I think that most of it is just that I don't like the way that the word feels in my mouth.  
It's also vague and not impressive sounding, so that doesn't help its case.

Experiments could be a fun name, especially since I am a scientist.  
I'm experimenting to figure out answers, even if I'm not using the classical scientific method.  
That just feels overly pretentious though.

Attempts?  
  
Let's see how that feels, \say{in today's attempt, I want to think about how I feel about}.  
I don't hate it, but I don't love it enough to be comfortable forcing those in my life to accept it as part of my idiolect.\footnote{which is, unfortunately, a big detterent I realize upon writing that}

Interestingly, it appears that the Latin word for \say{to try}\footnote{tentare, though interestingly, seems like Wikipedia lists words and forms from first person singular, not infinitive, weird.} comes from the word for either stretching or having.  
I'm going to guess it's the grasping one.  
So, how do I feel about words like grasp or containing?  
Eh.

If we return to musing, we get contemplations and reflections.  
My word processor doesn't like making contemplation plural, so reflections is probably good.  
There's an argument to be made that I am not reflecting, I'm emitting,\footnote{don't boo me} since the information almost always basically goes out, rather than going in both directions.  
Still, it is about looking for the after effects of thoughts that I've had and encountered.  
In that regard, ripples could be good.\footnote{waves? currents?}  
I'm reading a book right now which says fields are just a consequence of information having a limited speed.  
I don't know how that relates here, though.

I think that I'm happiest with reflections for now, even as I solicit more feedback.

Post Script:

Unlike my father, I do not feel the need to dedicate part of the naming to rants, because few enough of my writings are based in anger

\section{Daily Notes}

\begin{itemize}

\item Obligations:

\begin{itemize}

\item Professional

\begin{itemize}

\item Write the thesis

Realized I've been really slacking on this, and so made some efforts to try to figure out how it might be better for me if I set up my schedule differently.  
Unfortunately, it does more and more seem like I can only use each system for a few days or weeks before it stops working for me.  
I guess that I have also only today woken up early enough to do my ideal morning routine, so that might also have something to do with it.

\item Revise the thesis

\item Edit the thesis

\item Research for the thesis

It took me until I was lying\footnote{laying? I guess in this case either way works because I can act on myself or just act} in bed last night to realize one of the very silly mistakes I've been making.  
I have a bunch of systems of equations which interrelate variables.  
I'd been solving them by hand.

As it turns out, and as I had literally used during an early version of this project, Python has a package which will solve symbolic systems of equations.  
When fully reduced, it turns out that I was not doing the best of jobs in terms of algebra.\footnote{read: the numbers disagreed depending on the system I used}  
Then again, given the fact that the numbers are very small, it's entirely possible that I'm just hitting up against the floating point limits at times.\footnote{fixed point arithmetic (feels like it should be arithmatic because mat is math right?) might be useful, but I refuse on principle}  
So, I'm now spending the much shorter amount of time that it takes to resolve the different numbers in terms of each other.

\item Read the books that might be useful for the thesis

One of these days I'll get better about this. That day, however, is not today.

\item Start citation tracking

I'm at least citing within my notes, though that does mean that I'll have to page through the book, reading through the only partially coherent ramblings of a madman.

\end{itemize}

\item Personal

\begin{itemize}

\item Learn the songs for to jam

Really need to get on this one, especially since I also really need to learn and write some guitar music.

I do also have a wedding coming up early 2026\footnote{that should not be next year, ew} that I have a commission for, so should look at this.

\end{itemize}

\item Self:

\begin{itemize}

\item Silence

Doing really well at this one, but I don't know if it's silence in a good way, if that makes any sense at all.

\item Typing practice.

Shoot! Anyways, I'm unsure if I'll get to it today, since wow the time flies.

\item Keep the phone out of the room for bed

Nope! But I did generally find that it wasn't too horrible in the morning, and I set it down without much of an issue at all, so the end goal was still accomplished.

\item Pray St. Michael Chaplet in the morning

I did! It went well, and I think helped me set myself up for success.

\item Stretch in the morning

I did! Wild that I can almost touch my palms to the ground, and equally wild that the same parts of me continue to feel the tightest.

\item Read at night

\item Poetry at night

\item Clean the home

Woo!

\item Stretching, standing, drinking water

Nope! OOf I am perpetually dehydrated right now. That's a good goal for today: get through the water bottle.

\item Posture

Decently, again, I continue to catch myself more and more frequently.  
The shoulders do have a tendency to slump in, but I also don't want to go too far in the opposite direction.  
Unfortunately, most advice about posture assumes it's the spine that's the issue, not the shoulders, so I'm not sure which part of the shoulder should touch the wall.

\item No wasted time

Only existentially! I did not, as it turns out, need to solve for the different commutation relationships that were derived in the paper.  
I'm glad that I did, just because it means that I can trust parts of my code more, but.

\item Eat more than 2 meals a day

I think so! I ate oats for breakfast, an apple, and curry rice!  
Woo, go me.

\end{itemize}

\end{itemize}

\item Goals and Growth:

\begin{itemize}

\item Ends:

\begin{itemize}

\item Letter writing, get into more

Playing with inks has been nice.

\item Handwriting, pick and make the new one

I've decided that, since some amount of ink stays in the nib\footnote{honestly, a shocking amount. I think that I can write three full pages} when I empty it, journaling in an analog sense could be useful each morning

\end{itemize}

\item Means:

\begin{itemize}

\item Typing speed, improve it.

\item Reading, do more of it

\item Blogging, do it

Woo!

\item Writing things that are not the blog and thesis, do

Journal in the morning!

\end{itemize}

\end{itemize}

\end{itemize}

\end{document}