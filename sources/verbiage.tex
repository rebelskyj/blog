\documentclass[12pt]{article}  
\newcommand{\say}[1]{``#1''}  
\newcommand{\nsay}[1]{`#1'}  
\usepackage{endnotes}  
\newcommand{\B}{\backslash{}}  
\renewcommand{\,}{\textsuperscript{,}}  
\usepackage{setspace}  
\usepackage{tipa}  
\usepackage{hyperref}  
\begin{document}  
\doublespacing  
\section{\href{verbiage.html}{On What I Call This Writing}}  
First Published: 2025 April 16

\section{Draft 2: 16 April 2025}

Immediately after posting the first draft, one of my readers brainstormed some ideas with me, so I'm going to run through all of them and say what I think of them.

Explorations.  
I don't really love this one, because I don't think of this site as being focused around discovery, which I consider a large part of exploration.  
It also has a weird feeling associated when I think of describing it to others.  
Sadly, I think reflections has this one beat.

Expressions.  
I like expressions as a word, and there is the whole fun bit about like genes get expressed, emotions get expressed, etc. that I'm having trouble connecting to right now.  
In terms of relating it to these writings, though, I think that it feels a little off.  
I think that I want to think of this as a form of expression, rather than expression itself.\footnote{yeah that resonates}

Impressions.  
I do so much love the inner outer divide.  
There could be something fun about having impressions and expressions as the concept for what I write, because it does point out that I, rightly, am not just responding to material (impression) but that the response itself motivates action (expression).  
However, \say{want to read my expressions and impressions} feels wrong, and more importantly, ordering them would cause me so much pain.

Screaming into the void.  
I think that I've made comments here about my writing going into the void when I felt like no one read.  
There's something kind of fun about asking someone if they've read my screams.\footnote{yes, I do ask my friends if they've read my blog, and no, I do not feel ashamed of that fact}  
However, reading screams parses badly to me.

Sending into the ether.  
In the post I'm still working on, I talk about the ether in what I understand to be a premodern sense.  
Since I haven't finished it yet, though, I don't think that it would be good for me to attach the label just yet.

Messages.  
A message is a unit of information, which most of the dictionaries describe as short.  
While that may have been historically accurate to me, these writings\footnote{no I will not let them be called writings at the end, because that feels wrong. Scripts? Maybe, I'll consider that one next} are consistently getting into the multiple thousand words.  
That is, over the course of a month, I am approaching the length of a full novel.\footnote{based on NaNoWriMo standards.}\,\footnote{oh, I wonder if this might be part of why I'm struggling to write a paper}  
I have trouble calling 4 percent of a novel a short unit of information.  
I am also not going to address whether there is, in fact, information in these digital pages.

Page is an interesting option.\footnote{new while writing here}  
I like that it refers to both a leaf in a book as well as an errandboy.  
Then again, I don't know quite what that means, and so will move on from it as well.

Scripts.\footnote{new while writing here}  
There's the joke that the greatest perk of a Ph.D. is that any time a package arrives, you get to say \say{ah, just what the doctor ordered.}  
Scripts also imply, at least to me, the exact font or hand that people write in.  
It's long been known and believed\footnote{depending on which side of the debate you fall under} that handwriting is informative.  
I also cannot help but feel like the tone of my writing says things about me.\footnote{hopefully complimentary, though I don't know if that is always true}  
Scripts also have the connection to actors, and in general have some sort of a prescriptive\footnote{wow how have I never noticed that prescriptive is root word script} element to them.  
I fear and love the idea that what exists here effects change on the world.

Scripts lack a little bit of immediate ease in understanding, though.  
While musings and essays and even reflections seem like something I could say to a stranger and expect them to parse my meaning, I have to imagine that if someone heard I wrote a script a day that I was either a really slow pharmacist or an incredibly prolific stagewriter.  
Still, all words only get meaning in as much as they are used, and so scripts are definitely up there.

On a similar note, manuscript.\footnote{new while writing here}  
Manuscripts, as the breakdown of its parts might imply, are generally implied to be hand written.  
However, there is an argument to be made for this work as not being printed, and therefor\footnote{oh, journaling isn't a word but therefor is??} belonging as one, along with the idea that the original version is a manuscript.  
This site is certainly the original version of most of these thoughts in my mind, though the fact that there are drafts might make that untrue in some valid and fundamental sense.  
After all, a draft is, by its very meaning, derivative.\footnote{and no, I will not be entertaining the idea of any calculus based names}

Opus and opera.\footnote{new while writing here}  
I mean these are, by most definitions, works.  
However, I think that, again, people might misunderstand when they say I add to my opera daily.  
I also really hate that we call musical pieces by work and then number, even if I can't quite justify why.  
I think that it has something to do with the fact that they're both functionally just numbering schemes, and there doesn't seem to be consistent which is the super and sub heading.

Works\footnote{new here}.  
I do kind of love this.  
\say{look at what I have wrought}, I might say.\footnote{wild, it's the past tense generally of work}  
However, this is not really labor, in most senses of the word.  
I receive no compensation, and I do not struggle to do this.\footnote{most of the time}  
It also feels intellectually dishonest, because I do still associate work with labor with physical exertion.  
That's something to consider.

Labors\footnote{new here while writing}.  
I love this, because writings are often described as children, the output of labor in the childrearing sense is a child, and it hearkens to epic times.  
However, laborious is what we turn the word into, and I don't really love that, because I want this writing to remain fun.  
Labors of love are, after all, labors.

Folios.\footnote{new here}  
Folio can mean a few interrelated things, which is great, because what's one more.  
It does generally imply that the document is folded, though, and there aren't any folds in digital scrolling.  
It also comes from the Latin for leaf, and I do like thinking of my writing as life-oriented, just like leaves.  
I think this might be the current winner, because it's also a fun word to say.

Paper.\footnote{new here}  
White papers are a common way to quickly represent information, often in an informal context.  
In general, I think that no one really has issues abstracting papers into something digital, though I feel like most still imply some sort of pagination in a printer friendly way.  
It also feels somewhat confusing to the eavesdropper, since I exist near academia.  
Still, another close one.

Illuminations.\footnote{new here}  
Wow I'm really digging deep into book lore.  
I would like to think that my writings shine light on something, but it feels pretentious to assume that they would.  
In an ideal world, I would be writing illuminations, but that does not mean I would refer to them as such.

Incunable or cradle.  
The first refers to early printings in England, and the second is the English of the term.  
Honestly, I do kind of like calling this a cradle.  
This is a place for new ideas to be birthed, or at least cared for.  
What does that make each writing, though?  
Or, is each writing a cradle, and the overall effect is the nursery?  
I don't know if I like cradle, because it feels too abstract.  
Incunable is fun, though, and does refer to the fact that\footnote{hopefully} these writings are the beginnings of my career.  
It's a little difficult for me to say and spell though.

Type.  
Eh, doesn't resonate.  
Moving on.

Parchment.  
I do love the idea of taking a word which has a very specific (if often misused) meaning and using it intentionally in a different context.  
Also, I do kind of love the visceral nature of parchment.  
Something about turning an animal into the substrate for ideas resonates within a deep part of me.  
Parchment is winning out for now, I guess.

Vellum is indistinguishable from parchment, and so modern scholars use the term membrane.  
Ooh membrane is almost better.  
It points out that there is an inherent barrier, both between my mind and the keys, but also between the words and the reader.  
Still, both are again a little too obscure.\footnote{these have all been new}

Returning to the actual list, we have letters.  
This makes me think of the biblical\footnote{does bible really just mean books? yes. I hate language} books, which are at least somewhat public facing, and generally directed towards an explicit end.  
That's true here.  
Letters are winning.

Characters is where my mind went from letters, since I do find it strange that we have a word which refers to both the individual glyph and the string of them together.  
Character, number, and glyph, however, are all a little too far from common usage for my tastes.

Writing glyphs just now made me think of the word arcana, and its singular arcanum.\footnote{new here}  
I would argue that this is specialized, and accessible only to a select few.  
I do like \say{I'm working on my arcane website}, and I like thinking of the writings as self contained.

Arcana brings me to esoterica and eldritch.\footnote{both new here}  
Esoterica is a great word, and generally refers to the impractical or at least obscure.  
Obscura, though, implies some level of intentional obscurement.  
Eldritch originally comes from elf, and is therefore bad.  
Esoterica is a fun winner right now, though there isn't a singular form of the word, I don't think?  
I am ok referring to my work each day as an esoteric, though.

From esoteric, one can easily move to follies.\footnote{new here}  
I feel like I remember watching a show called follies as a child, and I have to assume it was looney tunes, giving an inverse Nimrod effect.\footnote{Nimrod is a mighty hunter in the Bible}  
It feels a little self denigrating to refer to these works as foolish, though it isn't necessarily inaccurate.  
Time spent here, after all, is time not spent elsewhere.

Email.  
Eh, it's accurate but.

Moving to the more abstract, I have yarns, spilled ink, thread, and stitch.  
Spilled ink is a fun one, but might be better as the title for the blog than a specific post.  
It feels strange at a deep level to message someone \say{what'd you think of my latest spilled ink?}

Rants or Ragings both imply more anger than I want to bring.

Echoes could be fun, but feel too abstract.

My mind, like the well it is, has run dry now, and so I present a list of the remaining options so one might peruse at their leisure.

\begin{itemize}  
\item Broadsides. Because like the old printing  
\item Impulse driver. Because I'm causing things, and science  
\item Waveform. Because, like a reflection, has a specific shape  
\item Wave generator. General idea of make waves.  
\item Genesis. Creation.  
\item Creation, because I made it.  
\item Things I made  
\item Mades  
\item Crafts  
\item Things I found  
\item Relics  
\item Unburying myself  
\item Clawing my way out of the earth  
\item Ideas which take me hostage   
\item Fixations  
\item Negotiation  
\item Argument  
\item Agreement  
\item Chat  
\item Fireside  
\item Recipe  
\item Algorithm  
\end{itemize}

\section{Draft 1}

I don't know what to refer to these writings as.

The obvious answer is to call it a blog, but since blog comes from web log, meaning a log on the internet, I'm not sure that it's the best term.  
After all, these posts are, at least hopefully, less a factual recounting of elapsed time and more a series of explorations into ideas.

My father, who I copy so much of this site from, calls his writings musings.  
Trawling through his site, I eventually saw that, as I see in the early posts, he initially called the posts essays.  
A commenter pointed out that not all of his posts are, technically speaking, essays.\footnote{I don't know if I agree with that commenter, but, lacking context, I'll trust taht the common usage at least might differ}  
Since he frequently refers to his muse in the writings, calling the writings musings became a next step.  
As far as I can tell, that's the extent of his reasoning.\footnote{is this bait to see if he still reads this? maybe}  
To muse is to contemplate or think deeply, and so there's something to be said for the idea that, if I am thinking deeply, then I am musing.

Essay, being his initial title, is another way to refer to what I'm writing.  
This has the benefit of sounding a lot more pretentious\footnote{for example, \say{Oh, yeah, I'm working on a series of essays} versus \say{Oh, yeah, I have a blog}}.  
It comes from the French for \say{to attempt}, and initially were\footnote{nominally, at least} used as a way to \say{attempt} to put thoughts into writing.\footnote{it was here I took a forced three hour break}  
I don't really know if this is so much about me attempting to put thoughts into writing as it is developing thoughts through writing, but it's still something to consider.

Post is another easy option, since it's sort of the default thing that most social media\footnote{I hate that we've turned this plural into both the plural and singular} uses.  
It, as far as I can guess and tell from three seconds of research, refers to the fact that when you wanted to distribute information broadly,\footnote{I really hope that broadside comes from this and not vice versa} you could affix writing to a physical post.  
I don't love it, and I think that most of it is just that I don't like the way that the word feels in my mouth.  
It's also vague and not impressive sounding, so that doesn't help its case.

Experiments could be a fun name, especially since I am a scientist.  
I'm experimenting to figure out answers, even if I'm not using the classical scientific method.  
That just feels overly pretentious though.

Attempts?  
  
Let's see how that feels, \say{in today's attempt, I want to think about how I feel about}.  
I don't hate it, but I don't love it enough to be comfortable forcing those in my life to accept it as part of my idiolect.\footnote{which is, unfortunately, a big detterent I realize upon writing that}

Interestingly, it appears that the Latin word for \say{to try}\footnote{tentare, though interestingly, seems like Wikipedia lists words and forms from first person singular, not infinitive, weird.} comes from the word for either stretching or having.  
I'm going to guess it's the grasping one.  
So, how do I feel about words like grasp or containing?  
Eh.

If we return to musing, we get contemplations and reflections.  
My word processor doesn't like making contemplation plural, so reflections is probably good.  
There's an argument to be made that I am not reflecting, I'm emitting,\footnote{don't boo me} since the information almost always basically goes out, rather than going in both directions.  
Still, it is about looking for the after effects of thoughts that I've had and encountered.  
In that regard, ripples could be good.\footnote{waves? currents?}  
I'm reading a book right now which says fields are just a consequence of information having a limited speed.  
I don't know how that relates here, though.

I think that I'm happiest with reflections for now, even as I solicit more feedback.

Post Script:

Unlike my father, I do not feel the need to dedicate part of the naming to rants, because few enough of my writings are based in anger

\section{Daily Notes}

\begin{itemize}

\item Obligations:

\begin{itemize}

\item Professional

\begin{itemize}

\item Write the thesis

Realized I've been really slacking on this, and so made some efforts to try to figure out how it might be better for me if I set up my schedule differently.  
Unfortunately, it does more and more seem like I can only use each system for a few days or weeks before it stops working for me.  
I guess that I have also only today woken up early enough to do my ideal morning routine, so that might also have something to do with it.

\item Revise the thesis

\item Edit the thesis

\item Research for the thesis

It took me until I was lying\footnote{laying? I guess in this case either way works because I can act on myself or just act} in bed last night to realize one of the very silly mistakes I've been making.  
I have a bunch of systems of equations which interrelate variables.  
I'd been solving them by hand.

As it turns out, and as I had literally used during an early version of this project, Python has a package which will solve symbolic systems of equations.  
When fully reduced, it turns out that I was not doing the best of jobs in terms of algebra.\footnote{read: the numbers disagreed depending on the system I used}  
Then again, given the fact that the numbers are very small, it's entirely possible that I'm just hitting up against the floating point limits at times.\footnote{fixed point arithmetic (feels like it should be arithmatic because mat is math right?) might be useful, but I refuse on principle}  
So, I'm now spending the much shorter amount of time that it takes to resolve the different numbers in terms of each other.

\item Read the books that might be useful for the thesis

One of these days I'll get better about this. That day, however, is not today.

\item Start citation tracking

I'm at least citing within my notes, though that does mean that I'll have to page through the book, reading through the only partially coherent ramblings of a madman.

\end{itemize}

\item Personal

\begin{itemize}

\item Learn the songs for to jam

Really need to get on this one, especially since I also really need to learn and write some guitar music.

I do also have a wedding coming up early 2026\footnote{that should not be next year, ew} that I have a commission for, so should look at this.

\end{itemize}

\item Self:

\begin{itemize}

\item Silence

Doing really well at this one, but I don't know if it's silence in a good way, if that makes any sense at all.

\item Typing practice.

Shoot! Anyways, I'm unsure if I'll get to it today, since wow the time flies.

\item Keep the phone out of the room for bed

Nope! But I did generally find that it wasn't too horrible in the morning, and I set it down without much of an issue at all, so the end goal was still accomplished.

\item Pray St. Michael Chaplet in the morning

I did! It went well, and I think helped me set myself up for success.

\item Stretch in the morning

I did! Wild that I can almost touch my palms to the ground, and equally wild that the same parts of me continue to feel the tightest.

\item Read at night

\item Poetry at night

\item Clean the home

Woo!

\item Stretching, standing, drinking water

Nope! OOf I am perpetually dehydrated right now. That's a good goal for today: get through the water bottle.

\item Posture

Decently, again, I continue to catch myself more and more frequently.  
The shoulders do have a tendency to slump in, but I also don't want to go too far in the opposite direction.  
Unfortunately, most advice about posture assumes it's the spine that's the issue, not the shoulders, so I'm not sure which part of the shoulder should touch the wall.

\item No wasted time

Only existentially! I did not, as it turns out, need to solve for the different commutation relationships that were derived in the paper.  
I'm glad that I did, just because it means that I can trust parts of my code more, but.

\item Eat more than 2 meals a day

I think so! I ate oats for breakfast, an apple, and curry rice!  
Woo, go me.

\end{itemize}

\end{itemize}

\item Goals and Growth:

\begin{itemize}

\item Ends:

\begin{itemize}

\item Letter writing, get into more

Playing with inks has been nice.

\item Handwriting, pick and make the new one

I've decided that, since some amount of ink stays in the nib\footnote{honestly, a shocking amount. I think that I can write three full pages} when I empty it, journaling in an analog sense could be useful each morning

\end{itemize}

\item Means:

\begin{itemize}

\item Typing speed, improve it.

\item Reading, do more of it

\item Blogging, do it

Woo!

\item Writing things that are not the blog and thesis, do

Journal in the morning!

\end{itemize}

\end{itemize}

\end{itemize}

\end{document}