\documentclass[12pt]{article}[titlepage]
\newcommand{\say}[1]{``#1''}
\newcommand{\nsay}[1]{`#1'}
\usepackage{endnotes}
\newcommand{\B}{\backslash{}}
\renewcommand{\,}{\textsuperscript{,}}
\usepackage{setspace}
\usepackage{tipa}
\usepackage{hyperref}
\begin{document}
\doublespacing
\section{\href{reflection-february-2025html}{Reflection on the Start to the Year}}
First Published: 2025 February 28

\section{Draft 2: 28 February}  
As the second month of the year comes to a close, I was reminded by a dear friend that it has been a while since last I updated my blog.  
The first draft of this post contains more ramblings and ideas, but my ideal blog post\footnote{truthfully, I'd like all my writing to be so, but as someone (I think Twain) said, something something if I had more time I'd have written less} is much tighter and cleaner.\footnote{hmm the choice of adjectives I made comes with a lot of connotations. Why is a more concise (the word I forgot) writing cleaner?}  
Where my writing, like my mind, is meant to ramble, I prefer to have asides in the footnotes.  
With that in mind\footnote{wow a lot of w sentences}, let's reflect and, like all interactions, plot towards the future.\footnote{that metaphor failed, but that's fine. In general I was thinking how like ripples from a pond you can trace time in both directions. Idk}

Since the year began, I've managed to make progress on a fair number of my goals for the year, and possibly more importantly, I've also revised my goals for the year, incorporating both more time to think and the lived experiences that I have from attempting them.  
The only major difference is that I no longer have a goal relating to drawing or art of that kind.  
More than that, I now know that when I feel as though I should learn to draw, I simply have a short lull in my life, and some activity is soon to require my attention.

As we look towards the next month and period of the year\footnote{since Lent is about to begin}, it seems worthwhile to explicitly state some goals, so that I can reflect on them.  
Given my inability to write those reflections without finding any number of sidebars\footnote{which I think is a boating reference? should look that up}, and given that they didn't appear until at least a few thousand words in, let's rewrite them.  
My goals can broadly be grouped into: Professional, Health, Other.\footnote{wild how all categorizations work when you add an other}  
Other is wrong, but I can't find the word right now to describe the connecting thread for them that doesn't also mean professional or health related goals.\footnote{personal, for instance, doesn't work, because health is an incredibly personal goal}

\begin{itemize}  
\item Professional:  
\begin{itemize}  
\item Thesis Work:\footnote{I do love nested lists, and nested things generally, as a group member pointed out (about my code)}  
\begin{itemize}  
\item Detailed outline (chapters and section titles at the very minimum) by 18 March  
\item Meet with Committee to decide a timeline. Before that, come up with my own idea for one.  
\item Work on the research within the thesis!  
\end{itemize}  
\item Be a better mentor: figure out how to take time to help underclassmen as they need help while still getting my own work done.  
\item Leave work at work. I've been only really thinking about my research this past month, and I can feel it wearing me down rapidly.  
\item Work towards future career:  
\begin{itemize}  
\item Read the recommended readings about science communication  
\item Do the reflections that were recommended to me (mostly focused around why I care about science communication)  
\item Work on the materials for the science outreach event in April: the handout while they work and a page for the families to take home  
\item Figure out the difference between my public-facing and field-facing presentation affects. As I focus on becoming a better presenter, I need to become aware of the difference and how to switch them  
\end{itemize}  
\end{itemize}  
\item Health:\footnote{I break into physical, mental, and spiritual, not because I think that I am these three discrete things, but because I think most of my goals primarily target one of the three aspects of me, and the effects that they have on the rest of me are harder to quantify (not that all goals need to be quantifiable)}  
\begin{itemize}  
\item Spiritual:  
\begin{itemize}  
\item Figure out what I want to do for my Lent. In general, I think I want to give something up, take something on, and find a way to do charity.  
\item Do the Lenten goals.  
\item Be intentional about prayer. That means both making time for prayer and actually doing it.  
\end{itemize}  
\item Mental:  
\begin{itemize}  
\item Clean my Life:  
\begin{itemize}  
\item Remove dirt and clutter from physical spaces (standard definition of clean)  
\item Remove extraneous apps from phone  
\item Spend time each day thinking about the goals for the day, and getting them out of my head and onto the page\footnote{be it physical or digital}  
\item Start reading and returning the library books I have.\footnote{elements of this hit professional, because it is potentially part of the thesis work}  
\item Start to separate the non-work time from work time. I think that, at the very least, I need to say that I will not do work in my apartment. Given that I generally avoid spending much time in there, these two goals should work together.  
\item Don't waste time, and in particular, be mindful about making sure to take breaks and rest.  
\item Clean sight lines. Is my space set up in a way that orients me towards my goals for the space? If not, how can I make it so?  
\end{itemize}  
\item Interpersonal Relationships\footnote{are essential to my mental health, and I know this}  
\begin{itemize}  
\item Figure out what belongs in a normal letter to a friend.\footnote{even ignoring that not everyone and I had a deep connection over my mom dying, that's only good for a single letter I think. Also, I want them to be potentially light, rather than always heavy. \say{Hi Friend, I love you and hope you're doing well} is not a heavy statement, but feels lacking to me for a letter. Whether that's a personal issue or actually advisable, who knows? not me yet!}  
\item Get back into writing letters.  
\item Upon feeling a sense of dread at receiving a message from someone, remember that my lived experience says that most interactions are positive. More to the point, if my friends didn't like me, they would tell me or at the very least would not continue to keep me in their life. If alone, speak something to that effect.  
\item Compile a list of people who are important to me. It does not need to be comprehensive, but ideally would approach that.  
\item Figure out the method and frequency of communication I would like to have with them, be that texting, calling, visiting in person, etc.  
\item Work to begin doing so.  
\item Potentially start giving small gifts, though many people also dislike clutter, so think carefully about that one.  
\end{itemize}
\end{itemize}  
\item Physical:  
\begin{itemize}  
\item Go to group fitness classes more regularly and more often.  
\item Feed myself simply and healthily.  
\end{itemize}  
\end{itemize}  
\item Other:  
\begin{itemize}  
\item Music:  
\begin{itemize}  
\item Figure out something to work towards on guitar  
\item Work towards it  
\item Spend time making efforts to improve as a singer, not simply passively singing.  
\item Spend time making efforts to improve as a musician, not simply passively growing.  
\end{itemize}  
\item Writing:\footnote{other than letters and daily notes, as in health or the professional ones in professional}  
\begin{itemize}  
\item Find the mental block towards writing my web novel  
\item Write poetry more often, ideally nightly.  
\item Not only write blogs, but also post them. Ideas include:  
\begin{itemize}  
\item 26 for 26  
\item Lenten goals  
\item Listening to an album and writing about my experience with it. Unsure if this is best done with one I have prior knowledge of or a new one, but regardless, sit and listen without other stimuli.  
\item The arts I've been doing lately  
\item Why I care about science and communicating it  
\item the block between me and my web novel  
\end{itemize}  
\end{itemize}  
\end{itemize}  
\end{itemize}  
Well, when spelled out, that's both a lot and not many goals at all!

I think that it's also probably smart for me to break the goals into one-offs and continuing goals:  
\begin{itemize}  
\item Discrete Goals:  
\begin{itemize}  
\item Thesis outline by March 18 for committee meeting.  
\item Committee Meeting  
\item Recommended readings about science communication  
\item Make Lenten goals  
\item Remove extraneous apps from phone.  
\item List of people  
\item Learn how to write a letter  
\item Find a guitar project  
\item Find web novel block  
\item Materials for science expo  
\end{itemize}  
\item Continuous Goals:\footnote{meaning, not planning to finish in March, even if they can be explicitly finished}  
\begin{itemize}  
\item Thesis research  
\item Writing my thesis  
\item Being a better mentor  
\item Separating work from not-work  
\item Think about presentation affect  
\item Pray better  
\item Eat better  
\item Write letters  
\item Reach out to friends  
\item quiet the voice in me that is nervous  
\item grow as a musician (guitar, voice, general)  
\item Write poetry  
\item Write blogs  
\item Daily goals/to dos\footnote{Hmm I don't have a good mental distinction between the two. Should I?}  
\item Clean home (std meaning)  
\item Improve aesthetics of spaces  
\item Be mindful of my time  
\end{itemize}  
\end{itemize}

Despite how long this list appears, it's really a very discrete number of things.  
More importantly, most all of the continuous goals are me attempting to orient myself.  
That is, rather than trying to get to writing daily poetry, I just want daily poetry to be on my mind going forward.  
Well, more than 6000 words later, I think that I should call this reflection here.

\section{Draft 1: 28 February}

It's officially the end of the second month of the year.  
It has been a little over a month since the last time I posted here, and that's not great, especially given the goals that I've had.  
Let's use this space\footnote{digital, and also the time that I have right now (spacetime is a thing! That means space is interchangeable with time)} and look through what \href{planning-2025.html}{our goals were} for the year\footnote{and also January} and see how resonant they still are, along with how much I have made progress on them.

Let's start with the things that I'm excited for this year.

\begin{itemize}  
\item I have a timeline of sorts for my Ph.D, and I'm meeting with my committee soon to see whether they agree with it.  
\item I think that I'm continuing to grow in my relationships, so that's really nice.  
\item I think that there was a typo in the old document, which is kind of fun. I do feel like my faith has continued to be a little touch and go, but hopefully as days get longer and brighter, so too will my outlook.  
\item I've been doing better about feeding myself, though I'm starting to let it slide again. Last weekend's adventures of traveling to watch a close friend defend his thesis and then returning to help with a chaotic graduate recruiting weekend, while fun, were certainly not conducive to my schedule.  
\item Finding a way to live with my grief remains a weird thing.  
I certainly feel like I'm dealing with it better than I was in early January, and like with any pain I have, I find myself compulsively touching it.  
It's so weird to realize that I can no longer and never again just text my mom a dumb question that I've forgotten the answer to and don't feel like looking up.  
When something goes wrong in my life, I can no longer go to her for guidance or comfort.  
No longer will I hear her tell stories of her childhood and early adulthood.

Despite all that, the pain continues to be duller when I think on these things.  
I am slowly coming to terms with the reality, much as it really hurts to do so.  
\end{itemize}

Moving on to my January goals, which I'm also going to treat as February goals:  
\begin{itemize}  
\item I don't think that I've done a single piece of album work or touched my accordion since writing that post.  
Music has been really hard for me lately, both because it's always an emotional experience and I've been trying to repress my emotions and because I can't separate my memories of sharing music with my mother from any music that I do.  
Still, this past week or two, I have been playing \say{I walk the Line} most mornings, so that's at least some progress back in the direcion of making music.  
\item I finished the embroidery project! That's really exciting and great.  
\item Working on art is something that I very quickly gave up on. Much as I love art, I more and more realize that it is so far down my list of priorities.  
\item I stopped stretching in the morning and evening, because I've started going to yoga\footnote{conveniently hitting the second point}. This week in particular, though, I've been exhausted and trying to recover from that\footnote{not eating well probably doesn't help that fact, but}.  
It's been really nice to spend 45 minutes in the morning and/or evening working on my flexibility and breath, and while I still absolutely cannot do the movements as naturally as the instructors, I can even more definitely say that I'm getting more flexible.  
\item I think that I did a personal posting about my food needs? I don't really remember when the series of posts was, but if it was after January 5, then I guess that I did that! Yay, a goal I finished.  
\item I wrote four or so letters in four weeks, and then fell off of it. I had set it as a thing I did on the way home from choir, and choir stopped happening for a little bit, so I guess the writing did too. I should really get back into it.  
\item Journal in the morning. I've stopped doing this, along with basically all journalling. I don't know why, except that I've been feeling really driven in my work and nothing else lately. How much of that is repression, how much of that is a genuine drive, and how much is a fear of falling behind, I don't know.  
\item I stopped writing poetry, and that's a shame. I think that I'm going to try to get back into that. I've been spending some time most nights crocheting while catching up on my soaps\footnote{read: the youtubers I enjoy or whatever audiobook I'm listening to}, and finishing the night off with a little writing would probably serve me well.  
\item I haven't been able to touch my webnovel at all. I looked at the comments on the post where I said that I would be taking a hiatus, and the support there was really great.  
Even thinking about the novel this little bit has made my breath catch\footnote{not in a good way}, and that's something I need to figure out how to work through, because I do think that I still want to finish it? As I typed that, I realized that it wasn't necessarily true, hence the question mark.  
That's something I can and probably should spend some time considering tomorrow.  
\item Clearly I haven't been blogging daily.   
I have written a few posts that I didn't post though, because they went in directions I didn't really want to share with the broader world.  
Still, it's also good for me to remember how to sanitize my thoughts for broader distribution, especially as we start to head into public talk season.  
\end{itemize}

Finally, with my yearly goals:  
\begin{itemize}  
\item I have made basically no progress on the thesis, though I have done some vague writing on the work I'm doing, which will hopefully make the thesis writing process easier.  
Still, that's something I should spend some time on, and that might be a good thing to do while calculations run. At the very least, I want to spend some time making an outline of a thesis.  
\item Coming to terms with my grief is, in fact, something that I'm doing as best a job on as I think that I can! Woo, finally something else I can unreservedly say that I've done well on so far.  
\item Still working on the sustainability, but I think that there's significant progress for sure. I'm right now struggling with the fact that I don't know what I like eating, and my \say{fifteenth century girl dinner} of a roll, some hard cheese, apples, carrots, and some meat isn't feeling great any more. I think that a large part of that is as simple as my body's cues being different right now, and so I need to really figure out what my body needs, regardless of what I hear it saying.  
\item Not writing the web novel has been very sustainable, unfortunately. I want that to stop being the case, though, which probably means that I would need to take a step back from my research. Given that my groupmates thought that I needed to go home at 2pm yesterday because I seemed completely burnt out, that might be a good idea on its own merits.  
\item Wasting time has been ebbing and flowing. Generally, I find myself mindlessly browsing the internet far less, and I tend to be doing something with my hands while I watch all the videos that I enjoy consuming.  
I've even started unsubscribing from a number of the content creators that I followed while I needed constant audio stimulation, and will probably continue to winnow down the list.  
Outside of that, though, I do find that I spend a fair amount of time letting time pass, even if that's just taking up logic puzzles. With Lent upcoming that could be a great thing to focus on. Instead of spending time playing mindless games, spend time praying.  
\item I wrote a few letters, but have kind of fallen off of it lately. I would like to get back into that, if only because I love my cage\footnote{the space I have reserved in the library is very cage-like}.  
\item I fell off of journalling, and I don't know if it really is something that I feel good about lately.\footnote{as I've been writing this, I've also been catching up on text with friends, and remembering the correct usage of full stops is always a fun journey when swapping between the two back and forth}  
\item I thought that I wanted to learn to draw, but more and more I'm realizing I prefer other forms of artistic expression, mostly involving string and yarn. I'm not totally sure why that is, but I'm sure that there are reasons I can consider. Maybe it's just as simple as the fact that crochet and embroidery\footnote{at least the way I do them} are both so repetitive that I can treat them as meditative in a way that drawing isn't. The fact that I know far fewer people markedly better at yarn craft than me compared to drawing certainly doesn't hurt either.  
\item Finishing the album is something that I've made literally no progress on lately. That's probably ok, though, given how little time I've felt like I have had in general lately. I've been able to shift my sleep schedule earlier and earlier, which is actually really nice. I love waking up before 6 because my body is rested.\footnote{wow this reflection is getting rambly faster than I would have expected (rambley? spell checker dislikes both and googling (I had a moment of \say{I don't use google or to support it}., then remembered brand dilution is a thing) doesn't seem to immediately treat either as a word}  
\item I joined a new choir,\footnote{TTBB! My first ever I realize} and I did immediately start writing a piece for it, only to immediately lose energy.  
\item I've been crocheting a lot lately, because I learned that making the flower part of a rose\footnote{not the stem or calyx (which is a term crochet embroidery pattern makers feel wayyy too loose in using, imo)} takes between ten and twenty minutes depending on how well my hands are moving. That's like the perfect amount of time, and it means that I don't feel bad if I have to restart, in addition to making it far less likely that I do need to restart.  
\item I feel like I have dropped accordion from my list of instruments, but am getting more proficiency for sure in guitar. I find myself picking differently to get genre-specific effects, and I can do scales far faster now, for all that I rarely use them.  
Singing has also been fun. The new choir expects me to sing a G4, so I've been once again forced to rely on my head voice/falsetto, and especially the transition between the two. It's a skill I had developed really well during my time as an undergraduate, where I sang tenor in the early music choir, so it's been fun to start to get it back. Yesterday, the church choir I sang in moved me down to bass 2, which is only notable because one of our songs this coming Sunday requires a D2. Even though I really struggle with E2 and F2, D2 comes out really well. I think that I'm probably just actively using subharmonics there, but who can say for certain?\footnote{me, if I bothered to pay attention}.

More than anything, though, all the singing in different styles I'm doing is forcing me to once again remember the way that I have to set my entire vocal structure in order to have different ranges feel singable. I tend to reset on a breath, which makes sense to me, even if I do also wonder if it might be better for me to start trying to be able to do the changes more on the fly.

\item My life is absolutely cleaner! My home, while still a mess, is miles better than it has been, and I just find that I'm more and more able to keep my home relatively clean. That is, progress is well ongoing, even though I recognize I have a ways to go.  
\item Despite it being two of my goals\footnote{whoops}, I will still not be working on art right now.  
\item I should really start compiling my list of 26 for 26\footnote{which is a blog post I should do}. I feel like I'm making progress on it, but without tracking, it can be hard to know for certain.  
\end{itemize}

So, two thousand words in\footnote{hey cool, my writing pace is still around 2k an hour, which is right around 30 a minute. Given that I have to think about what I'm going to write, along with the fact that I've been multitasking and correcting all my errors, I'm really happy with that}, what is the summary of my reflection?

I'm generally doing better than I thought I was on my goals, even if I'm doing far worse than the me of early January had hoped.

Before I answer \say{What do I want to work on in the month of March?}, let's get some highlights of the past two months out of the way, because focusing on the good is better when framing my future.

\begin{itemize}  
\item Got my talk accepted for a conference!  
\item Learned how to crochet roses, and got a ton of compliments on them.  
\item Visited a friend to watch their dance recital\footnote{they were, unsurprisingly, fantastic} and gift them a hat\footnote{which I did, in fact finish after the recital but before I left}  
\item Had a wild interaction with a woman in a club, where she told me that she was dying of cancer and scared for her kids, and I was able to give advice as to what was helpful for me on the other end of the process.  
\item Started some new medications, which have been really helpful  
\item Found out what kind of group fitness classes I enjoy  
\item Helped out with the science bowl again!  
\item Taught the kids I teach how to binary search\footnote{no, that was not connected to the theology in any way, shape, or form}  
\item Read a book that's been on my TBR since just about as long as I've had one\footnote{and, wildly, it was really great. One review put it nicely \say{it shouldn't have been as enjoyable as it was}, since it did really have a pretty predictable and generic plot, caricatures of characters (ooh that's a great line, I need to do something with that in the future), and the outdated sexual morals of the 1970s (consent is much different now)}  
\item Learned how to use the high throughput computing cluster at my school, and started using it.  
\item Baked a lot of bread!  
\item Made a pork shoulder that was absolutely delicious. Half of it fed me for weeks.  
\item Went to a close friend's thesis defense, and was in the acknowledgements!!! That's quite possibly the first time I've been actually cited as a scientific source  
\item Found out that the research I did in undergrad is not publishable\footnote{not so much we got scooped as the field came to the knowledge as a whole}  
\item Helped out at a recruiting weekend and ended up being looked to to put out a number of\footnote{thankfully metaphorical} fires.  
\item Met with my advisor and got a vague timeline for finishing my degree  
\item Started watching a show weekly with a group of friends  
\item Started leading my subgroup in weekly meetings and in general  
\item Met with quite possibly the world's best chemistry outreach\footnote{he'd say public engagement, and I agree with his points} person and got advice about structuring a talk and really just conversation in general.  
\end{itemize}

Wow, that's way more than I thought, and I had to go back multiple times to add more and more to the list.  
Honestly, I feel way better about a lot now that I have that all down there.  
It is wild to me how much just sitting and reflecting does to make me feel centered, and I do absolutely need to make more of a point of doing so.

Looking into March, what's on the docket?  
\begin{itemize}  
\item Progress on thesis. At the very least, I want to have a detailed outline, where I have the chapters I'm going to write and the major sections within them as titles. Since I'm meeting with my committee in mid March, I should aim to do that sooner than later\footnote{i.e. probably today in between reviewing documents with the group}  
\item Work on the food situation. With Lent coming up, I usually give up alcohol and meat. I don't know if I'm going to give up either or both this year, but regardless, I want to focus on food which is more contemplative than mindless, and really make an effort to focus on eating food which is less ultraprocessed.\footnote{Deep down, this feels like something related to spirituality, but for the life of me, I cannot find the words to describe how} That is, I want to make sure that I'm feeding myself simply and healthily\footnote{ah yeah simple is probably what I meant}.  
\item Figure out what the block is on my web novel. I don't need to get back into it, but if I realize that it's not going to come back, then I should accept that. If it is going to be something I pick back up, then I need to figure out what's stopping me.  
\item Don't waste time. I think that I'm going to give up playing games for lent\footnote{in the sense of video games to pass the time, not in the sense of any shared experience with people}, though I should really reflect on my plans for lent before it starts. I also think that not wasting time can be thought of as not scrolling by social media, and so much more. I know that rest is vital, and so not a waste, but there are ways to rest that are not as brain rotting\footnote{I think that's a new phrase but} as social media. I want to focus on those.  
\item Writing. I want to, in addition to the thesis and novel, get back into blogging, poetry, and letter writing. The letters I've been sending have been really heavy, but letters don't need to be so. Maybe read an old etiquette\footnote{wow I had no idea where the t's in that word went} guide on letters to see what used to be normative to include. The letter itself is probably the important part, not the depth of the contents.\footnote{that feels wrong to say, but I am in general struggling to find the words to express myself. That's part of why I want to get back into journaling and poetry}  
\item Be better about group fitness classes. Twice a day during the work week is likely too many, but twice per week is definitely too few. As much as I have been enjoying the productivity of morning work, it does leave me crashing by early afternoon, so going to the morning yoga is likely a good option.  
\item Be better about taking chances on interpersonal relationships. I have a deep seated and frankly irrational fear that many people in my life despise me. The best way to get that part of me to reduce is by forcing it to confront reality. Also, rejection rarely hurts as much as it feels like it will, especially if I don't irrationally psych myself in any direction.  
\item Music. I want to make efforts to grow in guitar and voice, not simply passively have them. I also want to get back into listening to music, and think that spending time reflecting on albums could be a good way to do that.  
\item Continue to clean my life. I want the spaces to be cleaner, but I also want the content to be less cluttered. I do best when I know what's coming and feel prepared\footnote{once again the wrong word, but like I feel better when I have a journal with me, regardless of my intention to write}.  
\item Work on 26 for 26. At the very least, figure out how behind I am.  
\item Healthy work life balance. I need to not spend every waking moment thinking about work, and I also need to be available as the senior member of my research group when younger students need my help.  
\item Taking the advice about science communication, revise last year's outreach talks and prepare for the weekend talk I have coming up.  
\item Spend a little time each morning either virtually or physically thinking about goals for the day. I don't know if looking at the goals again at night really helps that much, but if it's something I have the space to do\footnote{mentally and emotionally}, then I might as well try that too.  
\end{itemize}

This set of goals is markedly different than the one at year beginning, and I feel comfortable with the changes.  
Mostly, they come from me realizing that my priorities are starting to focus on excelling in the areas I care about, rather than trying to become competent at even more areas that I have no true need for.\footnote{Drawing is really the big one here. I don't care about it when I don't have time, and when I do have time, I do. That's interesting enough, and is probably something I can keep in mind as I move forward in life. The more space I have, the more I care about learning to draw}

In a slightly more coherent manner, the goals I have for March:  
\begin{itemize}  
\item Professional:  
\begin{itemize}  
\item Get a working outline of my thesis before committee meeting  
\item Work on leaving work at work  
\item Work on being a better mentor  
\item Work on science communication:  
\begin{itemize}  
\item Draft the materials for the science exploration I'm helping to lead  
\item Read the recommended readings from the meeting and look to how I can incorporate that into the materials I already have  
\item figure out how to separate my public facing presentation from intra-field communication. The talks need to be different, and I want to make sure I have the affects in my head.  
\end{itemize}  
\end{itemize}  
\item Heath:\footnote{retroactively placed above artistic after first point there because I was (am as of right now) unsure whether I should put writing there. Why I don't think of writing as artistic is a question for another time}  
\begin{itemize}  
\item Physical:\footnote{yes, I realize that I am not a mind, body, and soul as three distinct parts, but it does help me to think of the driving reason behind each. Secondary effects are not primary}  
\begin{itemize}  
\item Feed myself simply and healthily. I think that I should probably start tracking what I eat, at least vaguely, because right now I feel like I don't feed myself enough, and the fact that my weight is stable means that the calories in and out are equaling each other, and I know that I'm not getting great calories in. I am in a healthy enough mind space right now that I don't have to settle for simply making sure that I have any fuel, I can work on bettering the fuel.\footnote{there's a post that I've seen that means a lot to me talking about how there's no bad calories, and also that potatoes are not great in spite of caloric density, but in part because of it} I know that nutrition is aided by cooking, so figure out what foods I need to cook in order to efficiently extract the nutrition, or other options\footnote{acids??}\footnote{oof this is getting rambly}  
\item Go to Group Fitness more regularly  
\end{itemize}  
\item Mental:  
\begin{itemize}  
\item Spend time each day thinking about what I want to have happen in the course of the day.  
\item Work to keep professional and personal things separate\footnote{in that like I want to be able to mindfully take breaks}  
\item Don't waste time, and in particular, be mindful about making sure to take breaks and rest.  
\item Clean my spaces:  
\begin{itemize}  
\item Normal Physical things (kitchen, bathroom, messes on tables)  
\item Sight lines: are the visuals I have at any given point the ones that I want? Are spaces set up to be conducive to what I want?  
\item Start reading and returning books\footnote{this is kinda new buttttt}  
\item Honestly, I think that the first two points, if not all three belong in here\footnote{if we redraft, move them down here}  
\end{itemize}  
\item Be more intentional about reaching out to others:  
\begin{itemize}  
\item Both figure out what to put in a normal letter and get back into writing them  
\item When I feel the sudden gut wrenching fear of an unexpected message, stop and verbally\footnote{if alone} remind myself that I have\footnote{probably, though that's something I shouldn't say there. Hmm what is something that I can say that feels true and is helpful (this was initially going to be followed by wronged, but the new one is better)} good and honest friends who value our relationship like I do,\footnote{like not necessarily meaning to the same degree, because that's not important} and much as I like randomly reaching out, some of them do as well.  
\item Once again compile a list of people who are important to me, and work to start figuring out times and ways to connect with them\footnote{obviously make sure they agree with the goals, but for now make it aspirational}\footnote{ooof helping one of the fellow students in my group with a coding issue really took me out of this. The next like 600 words I spent describing the difference between how I feel also did}  
\item I think those three are good! Maybe start giving small gifts, but that's something I can do.  
\end{itemize}  
\end{itemize}  
\item Spiritual:  
\begin{itemize}  
\item Be more intentional about prayer  
\item Figure out what I want to do with my lent, and start doing that.\footnote{taking a break here for work with the group. Returning: let's keep going}  
\end{itemize}  
\end{itemize}  
\item Artistic:  
\begin{itemize}  
\item Work on guitar and voice, not just passively.  
\item I guess the rest of writing belongs here  
\item Writing:  
\begin{itemize}  
\item Find the web novel mental block  
\item Start writing poetry again  
\item Start blogging (and also posting the blogs more)  
\item 26 for 26? This should maybe go somewhere else? idk. Artistic might be the wrong goal. Maybe just \say{other}  
\end{itemize}  
\end{itemize}  
\end{itemize}

Woo! We did it! Only 4500 words to vaguely get my point across.   
Let's revise this so that i can make it a little easier to follow


\end{document}