\documentclass[12pt]{article}[titlepage]
\newcommand{\say}[1]{``#1''}
\newcommand{\nsay}[1]{`#1'}
\usepackage{endnotes}
\newcommand{\B}{\backslash{}}
\renewcommand{\,}{\textsuperscript{,}}
\usepackage{setspace}
\usepackage{tipa}
\usepackage{hyperref}
\begin{document}
\doublespacing
\section{\href{quick-happy-memories.html}{A Few Happy Memories}}
First Published: 2024 December 9

\section{Draft One:}
As I mentioned \href{reflection\-2024b.html}{in my last musing}, I've run into the issue of trying to do highly emotionally charged musings at times and places where that is not necessarily conducive to finishing them.  
With that in mind, and keeping in mind the two desires that these musings result from: wanting to feel close to my mother still and wanting to muse more often, I thought that it might be good to just go through a few quick and happy memories I have of my mother while they're as fresh as they'll ever be again.

There is a slight issue in doing this, though, because so much of my relationship with her was about the constancy, so the many conversations we have blur into one cohesive sense of warmth.  
One memory that comes to mind right now, though, is from Hanukkah one of the years that I was in high school.

It was during one of the phases that either I or my father\footnote{or quite possibly both of us} were in the space of connecting to our Jewish heritage, and so we were lighting the candles.  
On Wednesday, though, we\footnote{I and my mother} had Religious Education, and for some reason my father was not available to watch the candles burn.  
So, she had me take the menorah to the church basement, and I lit them before class.  
It was shocking to me how many Catholics, especially confirmed ones, were so unaware of the Jewish roots of our faith.

I think that it ended up derailing the planned class discussion, because rather than talking about whatever the planned lesson was, we talked about Hanukkah and the Church's relationship to Jewry.

The other memory coming to me right now is when she said something which more and more I've come to realize is both absolutely fundamental to my own faith and also not incredibly common: all theology needs to be rooted entirely in love.  
That is, when speaking against something, love has to be central to the entire argument.  
Rather than using fear of damnation and hellfire, using the knowledge that the Lord is Love and that love is our highest calling is not just the key to good apologetics.

Honestly, the more I think about those words, the more that I find them benefiting my life.  
The more I can act and speak from a place of love, and the more that I try to ascribe good intentions to others, the better the world seems and the better my relationships become.

I also remember the first time that I crocheted a basic hat for my mother.  
She was bemoaning losing hair from her cancer treatments, and asked me to make her a hat.  
It took me a few hours of half attention, but I saw her wearing it so much afterwards.  
I know that it's so trite to talk about how the smallest things we do can have the largest impacts, but I do truly find that to be the case.

In short, I miss you mom, but I wouldn't trade my memories for anything.  

\end{document}