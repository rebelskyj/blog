\documentclass[12pt]{article}[titlepage]
\newcommand{\say}[1]{``#1''}
\newcommand{\nsay}[1]{`#1'}
\usepackage{endnotes}
\newcommand{\1}{\={a}}
\newcommand{\2}{\={e}}
\newcommand{\3}{\={\i}}
\newcommand{\4}{\=o}
\newcommand{\5}{\=u}
\newcommand{\6}{\={A}}
\newcommand{\B}{\backslash{}}
\renewcommand{\,}{\textsuperscript{,}}
\usepackage{setspace}
\usepackage{tipa}
\usepackage{hyperref}
\begin{document}
\doublespacing
\section{\href{lock-ins-0.html}{Lock-ins}}
First Published: 2022 March 9


\section{Draft 1}
In preparation for my Second Year Exam, I've realized that I should probably learn some actual science.
Thus, this meta-series begins.
I am nearly positive that a lot of the data I collect will require the use of lock-in amplifiers.
If I can get one of the instruments I'm planning to work on done, I'm sure that the number will increase to all of the in-lab spectroscopy that I will do.
As a result, I need to learn about lock-in amplifiers.
My plan is to work from three sources to learn this topic:
\begin{enumerate}
\item A paper that describes the lock-in amplifier I think I want to build
\item The book published in 1983 on lock-in amplifiers
\item Prior theses from group members on lock-in amplifiers
\end{enumerate}
Realistically, I think I'm going to be learning these in reverse order, because the theses will elucidate what we need and use lock-ins for, the textbook will teach me how they actually work, and then I will be able to figure out whether or not the paper's amplifier will work for what I need.

I guess I should probably stop talking about what I'll read and start reading.
I'm also not sure how anonymous to be with the details I'm giving, but I'll err on the side of fewer identifying informations.
I would feel worse about describing our current limitations if the theses I'm reading weren't published on the internet.

\end{document}