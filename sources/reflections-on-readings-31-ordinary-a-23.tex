\documentclass[12pt]{article}[titlepage]
\newcommand{\say}[1]{``#1''}
\newcommand{\nsay}[1]{`#1'}
\usepackage{endnotes}
\newcommand{\1}{\={a}}
\newcommand{\2}{\={e}}
\newcommand{\3}{\={\i}}
\newcommand{\4}{\=o}
\newcommand{\5}{\=u}
\newcommand{\6}{\={A}}
\newcommand{\B}{\backslash{}}
\renewcommand{\,}{\textsuperscript{,}}
\usepackage{setspace}
\usepackage{tipa}
\usepackage{hyperref}
\begin{document}
\doublespacing
\section{\href{reflections-on-readings-31-ordinary-a-23.html}{Reflections on Today's Gospel}}
First Published: 2023 November 5


\section{Draft 1}
It's been quite a while since my last reflection on the Sunday Mass readings.
There are a number of reasons for that, but especially in light of \href{christian-writing.html}{my recent musing where I said I wanted to do more religious writing,} spending a musing a week explicitly reflecting on the Bible feels like a good way to practice that.
Also, since the last time that I've written a reflection on the readings, I finally learned at least one reason I tend to have trouble connecting the first reading and Gospel to the second reading.
According to the notes in the edition of the Bible I use most, the Gospel passages were chosen to try to give a relatively full accounting of the different Gospels over the three year sequence.
Each of the first readings is chosen to match the Gospel, whether by showing something prophetic Christ did, or simply just by focusing on similar themes.
The second readings, however, are not chosen to match the other two.
Instead, they are just supposed to trace through the letters, with the goal of giving a good summary of each letter.\footnote{I'm less positive of this claim, but it's what I remember interpreting the answer as.
At the very least, I am positive that it said the second reading is not inherently connected to the other two readings, but that the other two readings are intrinsically designed to work together.}

Anyways, nearly two hundred and fifty words later, let's talk about today's readings.
The first reading comes from the book of Malachi, which I've now learned is the final of the twelve minor prophets in the Bible, and is the final book before Matthew in almost every translation of the Christian Bible.
Malachi, as it turns out, means messenger, which leads some to believe that the name of the book is referencing a title, rather than the given name of the author.

The first reading is a warning to the priests of the Jewish people.
It can be read a number of ways, I am certain.
However, the choice of Gospel passage makes it clear to me what interpretation we're expected to take from the reading.

In the Gospel, Christ talks about the way that the Pharisees and scribes of the people are not living in accord with the covenant the Almighty established over His people.
Despite that, Christ acknowledges, along with, it seems, every Gospel and New Testament Letter writer, that the Pharisees still have authority over the people of Israel.
He says \say{do and observe all things whatsoever they tell you,}\footnote{Matthew 23:3a (ish)} before immediately warning the listeners not to act as the Pharisees do.

That is an interesting admonition for me to reflect on.
I feel like I tend to have difficulty accepting truths that come from people who I know do not act according to them.
If someone tells me that something is healthy and will make me a better person, but does not do it themselves, there is a level of disconnect that I personally struggle to connect through.
And yet, as the Gospel points out, this is not a unique or novel situation that I find myself in.

There are two ways that I can read the instruction.\footnote{standard disclaimer: I am not a theologian or a consecrated.
This is me thinking and reflecting as I write the musing, not a guarantee of a normative theological opinion, let alone the absolute truth}
First, people in authority over us making orders that are within the standards of their authority should generally be obeyed.\footnote{even with all of these disclaimers, I feel uncomfortable with the sentence, which says a lot about me}
That is, if a boss tells his employees to clean the building, that is his right.\footnote{interesting that I A, assume that the boss is a man, B, call it his right.}
If the employees do not clean, even if the boss does not, then they are in the wrong.\footnote{again, in this entirely hypothetical situation I'm constructing}

The other way that I can understand this is that people can be flawed and recognize that about themselves.
I, for instance, know that exercising regularly makes me feel better physically and emotionally.
I know the same is true for me about having a regular prayer life.
Despite this, I do not exercise regularly, and my prayer life\footnote{as I've discussed numerous times} needs plenty of work.
If someone was feeling generally down about things, I would\footnote{of course assuming so many things here} advise them to try praying and exercising more.
Even though I do not do it, it's still good advice.

How much more true can that be for people who have dedicated their lives to studying and interpreting the Word of the Almighty?
When a priest or bishop today makes a theological point that I disagree with, I have to recognize that he has gone through significantly more formal theological training than me.
Even when what they say is clearly wrong to me, it is still worth the time it takes to understand where they are getting their thoughts from.

In the time that Christ was preaching, the Pharisees were the voice of the people, speaking with Mosaic authority, rather than the priestly power that the Sadducees spoke with.\footnote{apparently, if I trust Wikipedia}
It's interesting that we start today with a reading discussing how the priestly caste will be ignored, and then we see Christ disagreeing with the major opposition to the priestly caste.
I'm sure that there's something deep and profound in the framing, but I can't find it right now.

The second reading, in stark contrast, is simply a message of evangelization.
We, as Catholics, are called to spread the faith to the whole world.
I know that I'm not great about doing that, both because I am afraid to evangelize and because I do not live a good Catholic life.
There is nothing I can do but try better tomorrow.

Daily Reflection:
\begin{itemize}
\item Did I write 1700 words for NaNoWriMo? I did, but I don't really like where the story is going right\footnote{typed write at first} now.
I hope that'll change tomorrow, but I'm worried it won't.
\item Did I write a chapter of Jeb? I finished the chapter that I started last night.
It felt either like a really strong chapter or a really weak chapter.
My beta reader thought it was an incredibly strong chapter, which is nice.
I think I'm a little in my head about the book because I got a review which glowed about how different my writing style is from the normal.
I didn't think I had a distinctive tone, so that was a bit of a shock to me.
\item Did I blog? See that? It's been almost a full week of daily posts, all of which are quadruple digits.
\item Did I stretch? Oops! I'll stretch when I post this.
\item Am I doing better at prayer than a rushed and thoughtless rosary? My rosary was incredibly rushed last night, I did a rushed Angelus today, and I struggled to pay attention during Mass. So, all in all, prayer is not going great.
\item Am I doing a good job writing letters to friends? Shoot! I knew I forgot something.
Ok, before I stretch, I'll address the letter that I wrote last night so that I can post it in the morning.
Depending how I feel after a few minutes away from the computer, I may also write another letter or at least address an envelope.
\end{itemize}

\end{document}