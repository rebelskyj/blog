\documentclass[12pt]{article}[titlepage]
\newcommand{\say}[1]{``#1''}
\newcommand{\nsay}[1]{`#1'}
\usepackage{endnotes}
\newcommand{\B}{\backslash{}}
\renewcommand{\,}{\textsuperscript{,}}
\usepackage{setspace}
\usepackage{tipa}
\usepackage{hyperref}
\begin{document}
\doublespacing
\section{\href{celtic-crochet.html}{Introduction to Celtic Crochet}}
First Published: 2022 February 1

\section{Draft 1}
Wow, it's hard to believe that it's already February.
As is so often the case, the day after I make some great resolution to be more creative, I hit a wall of writer's block.
Since I refuse to be beholden to inspiration, I tried to think of a way to work through this issue.
Today, that took the form of scrolling through old post names while looking around at my desk.
In doing so, I saw the shawl that I've been working on for a few weeks now, and it inspired me to write about the crocheting I've been doing.

As I mentioned, I prefer crochet to knitting because of the flexibility of patterns that you can make.
The first thing I crocheted was a hat, as I think was the second.
Very soon after making hats, however, I started working on what I've begun calling \say{Celtic Crochet}.
I call it that for the very simple reason that the patterns end up looking like a Celtic knot.\footnote{when I both make the design and follow it correctly, which is not as common as I'd like}
The style I've developed for this is a three row pattern, which has some advantages and disadvantages.

One major advantage is that it allows for really nice two-color patterns with a major and a minor color.
By using the same color on the first and third rows, loops don't end up looking odd, because the same color is always on the outside.
Another big advantage is that it's fairly easy to algorithm.\footnote{Is algorithm a verb? I sure hope it is because algorithize feels wrong}

One big disadvantage of three row is that the foundation row doesn't count, which means that if you look very carefully, you may notice where the first row is versus the second and third rows.
Since many people don't even notice that the design is woven, that's really not a huge issue.

Speaking of weaving, however, there are two massive disadvantages with the pattern-style I use.
The first is that you have to make a foundation row with as many stitches as the pattern is large, which is approximately 16 per hole in a Celtic knot.\footnote{if I remember my math correctly}
The other issue is that, once you've crocheted the whole length, you're mostly just left with a slightly kinky rope, which you then have to weave totally around itself in order to finish the design.
So often I have struggled a lot with that final step, because weaving is hard.

That being said, I really like the way that the designs come out.
I liked them so much in late 2019 that I decided that I would make a blanket that is totally one length of cord then woven together.
Over two years later, I still have more than half of the blanket to go.\footnote{To be fair I've taken many multiple month long breaks}
When I finish the blanket, it will be far too long to weave, but that's\footnote{hopefully!} an issue for future me.

While designing the blanket, a friend asked me a fair question, though one which belies a misunderstanding about Celtic knotwork.
They asked why I wouldn't just make many small pieces and then sew them together.

One crucial element about well-made\footnote{drawn usually} knots is that they are a single strand.
I've tried a few times to crochet the whole celtic knot already woven, but I haven't been able to get it running any time I've tried.
If I could, I think that might really be fantastic, because there would be no loose ends to weave together.

559
50
\end{document}