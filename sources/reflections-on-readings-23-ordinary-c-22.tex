\documentclass[12pt]{article}[titlepage]
\newcommand{\say}[1]{``#1''}
\newcommand{\nsay}[1]{`#1'}
\usepackage{endnotes}
\newcommand{\1}{\={a}}
\newcommand{\2}{\={e}}
\newcommand{\3}{\={\i}}
\newcommand{\4}{\=o}
\newcommand{\5}{\=u}
\newcommand{\6}{\={A}}
\newcommand{\B}{\backslash{}}
\renewcommand{\,}{\textsuperscript{,}}
\usepackage{setspace}
\usepackage{tipa}
\usepackage{hyperref}
\begin{document}
\doublespacing
\section{\href{reflections-on-readings-22-ordinary-c-22.html}{Reflections on Today's Gospel}}
First Published: 2022 September 4

Wisdom 9:13 \say{For who knows God's counsel,

or who can conceive what the Lord intends}

\section{Draft 1}
The Gospel today tells us that in order to follow Christ, we must hate those closest to us, including ourselves.
The priest's homily focused on the idiomatic meaning of hate in this context.
Hate is meant instead as a less-preferred option, rather than our general usage of an actually unwanted thing.

I'm reminded of two lessons that have been taught to me, however.
The first is that anything earthly is only good only in how it orients us to He who is Good.
That is, if love of our family or life keeps us from following Christ, we do need to actively choose to not love them in that way in order to follow Him.
I personally cannot think of a better example of hatred than actively choosing not to love.

Of course, this takes us to the second lesson: Christ asks us to be willing to give up everything, not to give up everything.
That's a bit of a narrow distinction for many to understand, but the examples are easy.
Imagine two monks, both with a vow of poverty.
One owns a single book, which he treasures and refuses to share.

The other owns dozens, at least in theory.
In practice, he lends them out constantly, never expecting the book returned.
Which of these two monks is living free from material attachments better?

More than that, we are not all called to a life of physical poverty and celibacy.
But, if the Lord should demand it of us, we are told that we need to be willing to give anything of ours to Him.
\end{document}