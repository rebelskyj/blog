\documentclass[12pt]{article}[titlepage]
\newcommand{\say}[1]{``#1''}
\newcommand{\nsay}[1]{`#1'}
\usepackage{endnotes}
\newcommand{\1}{\={a}}
\newcommand{\2}{\={e}}
\newcommand{\3}{\={\i}}
\newcommand{\4}{\=o}
\newcommand{\5}{\=u}
\newcommand{\6}{\={A}}
\newcommand{\B}{\backslash{}}
\renewcommand{\,}{\textsuperscript{,}}
\usepackage{setspace}
\usepackage{tipa}
\usepackage{hyperref}
\begin{document}
\doublespacing
\section{\href{reflections-on-readings-3-ordinary-c.html}{Reflections on Today's Gospel}}
First Published: 2019 January 27

Nehemiah 8:9B: \say{Today is holy to the LORD your God. Do not lament, do not weep!}

\section{Draft 1}
Today's Gospel reading speaks about Jesus' return to Galilee, where he reads a passage from the Prophet Isaiah.
Here, we see the Lord telling the congregation that, in the words of the Niceno-Constantinopolitan Creed, \say{in accordance with the Scriptures} he has come and fulfilled the words.
All in all, it's one of the more straightforward Gospels, at least to me.

But, as the first reading points out, \say{today is holy to the Lord.}
We should live each day knowing that the Lord loved us so that he took the form of a man and suffered death on a cross for us.
\end{document}