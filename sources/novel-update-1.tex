\documentclass[12pt]{article}[titlepage]
\newcommand{\say}[1]{``#1''}
\newcommand{\nsay}[1]{`#1'}
\usepackage{endnotes}
\newcommand{\1}{\={a}}
\newcommand{\2}{\={e}}
\newcommand{\3}{\={\i}}
\newcommand{\4}{\=o}
\newcommand{\5}{\=u}
\newcommand{\6}{\={A}}
\newcommand{\B}{\backslash{}}
\renewcommand{\,}{\textsuperscript{,}}
\usepackage{setspace}
\usepackage{tipa}
\usepackage{hyperref}
\begin{document}
\doublespacing
\section{\href{novel-update-1.html}{Novel Update}}
First Published: 2022 February 15

\section{Draft 1}
This is the first\footnote{by definition novel (new)} post in what I'm expecting to be a fairly repetitive series of posts.
More or less, I've worked on the novel\footnote{book thing, since I get to call it what I want} I'm writing for too much time today for me to have the urge to create a whole new writing prompt and write from it.
Instead, I get to write about how I'm writing that novel.

Currently, I have been doing a timed\footnote{I set a timer and write for that amount of time, both to monitor writing speed and so that I have external control on my time} mostly linear\footnote{I try to just continue the plot forward, though occasionally make notes about what I should edit later} word vomit\footnote{I just spew words onto the page, planning on them being really awful, because I'm going to edit them later} to generate content, then going back through on later days and revising that text into something vaguely in the shape I want.
An analogy I'm creating right now is that the first pass is throwing clay onto a block, and the second pass shapes it into vaguely what I'm looking for.
Of course, since I'm adding more clay every day, I'll likely need to add more and more refinement so that the new clump of clay doesn't destroy the shape I'm building.
What that means extra-metaphorically\footnote{meaning outside of the metaphor, not making the metaphor moreso} is that I'll likely try to add another set of passes before I start publishing it.

Publishing is another point I should discuss here.
I'm currently planning to publish it on an online web-serial site, for a variety of reasons, most of which being that I like the site and would like the flexibility of not having to finish the book before I put it out for people to see.
Of course, that does limit how much I can work on the revisions, but that's ok with me, because at some point you have to present a sculpture, regardless of if you want to keep editing it.

I've had a couple of people read the approximately 2000 words I've done the first pass of the editing through.
The word vomit right now consists of about another 3000 words, but I've noticed that I add far more words on the first revision of each section, which probably says a lot about the way that I write in general.

As I think about that last sentence, it is 100 percent true.
Whenever I write papers I find that I have to use the first few drafts as practically outlines, making paragraphs out of each sentence.\footnote{sometimes literally more than one paragraph per sentence, those are the fun ones}
Today I wrote about 1700 words, which would put me solidly in NaNoWriMo writing pace.\footnote{NaNoWriMo stands for National Novel Writer's Month, and has a goal of 50000 words in November, which averages out to just under 1700 words a day}
In the word vomit period of my writing, I find that I'm generally putting between 60 and 70 words on the page per minute, which would imply a time of 25 to 30 minutes of writing to get to my 1700 today.

However, I don't really time the editing portion, for the reasons of:
\begin{itemize}
\item I don't like being timed for things I want to do
\item My goal isn't speed but quality
\item idk I just haven't been yet\footnote{I hate that I have this need to have all lists be in groupings of three}
\end{itemize}
But I think I probably spent around one and a half hours on the editing today, which may not be the most sustainable for me longterm.
Who knows, though, maybe I'll become more efficient with both, especially since I think a day of word vomiting gets me one or two days of editing, so eventually I'll just have a story to edit, rather than also generate.

523
144
\end{document}