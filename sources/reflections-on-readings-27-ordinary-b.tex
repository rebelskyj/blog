\documentclass[12pt]{article}[titlepage]
\newcommand{\say}[1]{``#1''}
\newcommand{\nsay}[1]{`#1'}
\usepackage{endnotes}
\newcommand{\1}{\={a}}
\newcommand{\2}{\={e}}
\newcommand{\3}{\={\i}}
\newcommand{\4}{\=o}
\newcommand{\5}{\=u}
\newcommand{\6}{\={A}}
\newcommand{\B}{\backslash{}}
\renewcommand{\,}{\textsuperscript{,}}
\usepackage{setspace}
\usepackage{tipa}
\usepackage{hyperref}
\begin{document}
\doublespacing
\section{\href{reflections-on-readings-27-ordinary-b.html}{Reflections on Today's Gospel}}
\section{Draft 2}
Mark 10:15 \say{Amen, I say to you, whoever does not accept the kingdom of God like a child will not enter it.}

Today is the 27 Sunday of Ordinary Time in Year B.

The main focus of today's readings is marriage,\footnote{at least, I thought so, and the priest's homily and the bulletin both said so} but the line that stuck out to me was in the optional section of the Gospel.
After exhorting the Pharisees that divorce is not from God, he talks to his disciples. 
Jesus tells the disciples that they must have faith like a child.
For most of my life, this was one of the pieces of the Gospel I had the hardest time believing.
As I look back, I realize it's mostly that I have a problem with unquestioned faith.
To be more precise, I have a problem with being told not to question my faith.

More importantly, I remember being a child.
I questioned everything.
And I do mean everything.
I asked the normal questions, \say{why is the grass green,} \say{why is the sky blue,} and so on.
I also\footnote{assuming I remember correctly} asked slightly odder questions, including many about my faith.
I don't think that I was not a normal child in that regard.
But, I did trust that whoever I was asking would know the answer, and tell me the truth.

That's what having the faith of a child means to me.
Children don't blindly follow, but they do follow.
They question, but they trust in answers.
There's a sense of joy that comes with every question, and a sense of peace that comes with every answer.

\section{Draft 1}
Mark 10:15 \say{Amen, I say to you, whoever does not accept the kingdom of God like a child will not enter it.}

Today is the 27 Sunday of Ordinary Time in Year B.

The main focus of today's readings is marriage, but the line that stuck out to me was in the optional section of the Gospel.
Jesus tells the disciples that they must have faith like a child.
Now, this is a belief that I've had a lot of trouble accepting the explanations for in my life.
The way I tend to hear it expressed is that we should believe unquestioningly.

But, I remember being a child.
I questioned everything.
Questions like \say{Why is the sky blue? Why is grass green?} and so on were a constant feature in my life.
And to me, that's what child-like faith is about.
It's not blindly following, it's questioning everything, to really learn about it.
There's a blind trust that the person answering your questions knows the answer,\footnote{and since the Church believes that God knows everything, the faith isn't misplaced} and that the person answering will tell the truth.\footnote{which hopefully isn't theologically controversial to say God does}
And, there's a sense of joy and wonder when we learn something new.
I still remember the constant sense of excitement I used to get whenever I learned something new.
So, faith like a child means questioning everything, not nothing.
\end{document}