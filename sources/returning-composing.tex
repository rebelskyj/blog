
\documentclass[12pt]{article}  
\newcommand{\say}[1]{``#1''}  
\newcommand{\nsay}[1]{`#1'}  
\usepackage{endnotes}  
\newcommand{\B}{\backslash{}}  
\renewcommand{\,}{\textsuperscript{,}}  
\usepackage{setspace}   
\usepackage{tipa}  
\usepackage{hyperref}  
\begin{document}  
\doublespacing  
\section{\href{returning-composing.html}{On Returning to Composing}}  
First Published: 2025 November 23

\section{Draft 1: 23 November 2025}

I'm not entirely sure why, but for some reason I cannot seem to remember how to compose.  
Looking at the timelines, I don't think that it's been so long since the last time that I wrote something relatively large.  
I suppose that I have not been writing any other music lately, so that could be part of it.  
Still, it seems like there's got to be more to it.

Since I have no idea what's going on with composing and the absence of it, I figure now is as good a time as any to consider other composition and analysis tricks I have heard of\footnote{or, at least, think that I have heard of}.  
Luckily, the commissioner for the piece has a few pieces that she wanted as inspiration, so I can look at them to hopefully get ideas.  
Even better, she sent me the text that she wants me to set.

Perhaps I'm struggling because I'm not writing for unaccompanied choir or for myself with whatever instrumentation I feel like on the given day.  
Instead, I'm writing for solo female voice over pipe organ.  
I've never been good at writing for pipe organ, or really any keyboard.  
That's another thing that this score study might help with.

So, what score study am I planning to do?

First, I want to try doing a visualization of the vocal line.  
My main idea is to have two lines, one for the highest and lowest note that the singer sings.  
From there, vaguely trace the melodic contour, with space for breath and whatnot.  
At the end, I'd love to go through and figure out if there is some middle point that the singer hangs out near.

I'm going to do the normal chordal analysis.  
Luckily, because the pieces I was told to use as reference were originally scored for a full orchestra, I have the advantage that someone else has already done a chord analysis, so I just have to look at the chords they've named, and then use that to figure out what the overall structure is.  
Tbh, I'm kind of hopeful that it won't be too painful, because I would love to be able to figure out exactly what the chord structure is during this writing time right now.

Given how overwhelmed I felt during mass today, maybe the reason I feel so tired is not just because I generally feel tired or anything like that.  
It's entirely possible that I do actually need to be resting more than I have, and that for whatever reason I have not been recovering from overwhelm before going forth again.  
I have a whole book about that, and I should really like to finish it.

So, when we consider the list of things to do\footnote{which maybe should go somewhere not archived into eternity}, I have the draft of the piece that I nominally promised a draft of today.  
That's priority one.

I have the book I would like to read about dealing with overwhelm.  
That realistically belongs as priority two.

I have the general desire to do some more creative writing again.  
That may be something worth considering, but that's also probably a low down the priority list item.

I'm going home soon to visit family, so worth making sure my home is in a good state for that.  
Otherwise, I think that I'm really just good to keep reading through the stack of books I checked out from the library.

...

I was planning to write with a friend, but they're running very late.  
I've already journaled almost an entire hand written page, and this is far from useful about the concept of composing, which I don't really feel deserves much more exploration.

Oh!

I have been thinking a lot about this site.

Especially over the summer and early fall, I did a fair amount of relatively in depth/intellectual posts.  
I don't think that's sustainable, both because I don't have the mental time or energy to devote to considering deep issues daily, and also because there aren't that many things to consider.

If my current readers (sound off) don't hate this format, I might consider doing more blog style posts, and then hopefully the constant writing will encourage me to explore the deeper thoughts\footnote{I promise that I have} that I want to work through.

Oh!\footnote{I do appreciate that I jump scare myself when I page back and forth between this post and other pages}

I want to generally get into composing again.  
I've been listening to a lot of minimalism lately, especially Reich and different covers\footnote{versions? recordings? great q} of Piano Phase.  
In the weekly album club my family is now doing, I have commented a few times on the fact that older music\footnote{in specific, David Bowie's Rise and Fall of Ziggy Stardust} is often much more willing to sit with an idea for a long time than newer music.  
I don't generally think that either approach is better, but it's making me think about Musicking again.\footnote{I will become insufferable about this book, I can already tell. If only I was sorry}

Small points out that the modern experience of being able to listen to a piece countless times is an anachronism.  
If I'm going to listen to the same song 100 times, I think that it makes sense to explore any given idea less.  
In a way, listening to the same song over and over is itself a form of exploring the idea more?  
Hmm, thinking about listening to a song on repeat as a form of minimalism is something worth thinking about.  
Especially when I consider the embodied nature of listening to music, the fact that definitionally an environment is changing around me as i listen to a song on loop is something interesting.

also, i'm considering whether doing no caps is a form of writing that appeals to me.  
it comes up because I didn't capitalize the i in the last sentence of the above paragraph, and something about it felt right.  
great question whether or not it will continue as a phase  
i suppose the other question comes in the form of punctuation

just listend\footnote{and spelling, i guess} to a video this morning about what it means to be a successful writer.  
in general the answer was writing is successful when it affects others.  
having a more unique affect is certainly one way to effect things.

Ok but going back to Small

Listening to pieces on loop is a relatively modern idea.  
Listening to disparate pieces together is incredibly modern.  
I think about the stories that my dad tells\footnote{told? it's been a while since the last time I recall him mentioning them} about making mix tapes.  
There's something fundamentally different, in my experience, in even burning a playlist to a CD compared to just having the playlist on a digital medium.  
I can't imagine how much more different it felt before music was all digital and easily shifted onto a mix.

The fact that it's almost easier\footnote{most streaming services default into some variation of a shuffle with other artists} to listen to different parts of an album separated than together these days perhaps does something to explain why the album is dying.  
By that, I mean that like rock operas are clearly a story told through an album.  
Even non-rock opera, though, tended to have a clear thematic through line.  
A video essayist I really appreciate\footnote{lmk if you want the link, I think that I mostly have him from some social commentary, but he does a lot with rap and R&B and the general Black music scene. He's old enough to have lived through these changes that I write about, and he also comments on some of them, from the obviously older perspective} somewhat bemoaned the death of the album.

Something I'm thinking about, too, is the soccer web novel I'm reading.\footnote{hmmm maybe sitting down for hours at a time to just let my thoughts percolate is actually a good thing for me. Who could have ever imagined??}  
One of the chapters focuses on our MC, a brash twenty-something, interacting with an older musician.  
They meet in a record shop, and there are a few pages\footnote{note: the fact that the way I read this book is as text posts on Patreon doesn't affect me calling them pages.  
Embodiment is real, but so is analogy} of discussion about the way that listening to a vinyl is a fundamentally different experience to streaming.  
The embodied nature of putting a record into the player, and then having to carefully place the needle, is used very briefly as a metaphor.

Sorry, where was I going with this?

Uhhhh...

Records don't exist anymore.  
Don't listen to songs how they were composed.  
Exploring an idea.

Oh!

So yeah, as I'm listening to a lot of early minimalism and really loving the slow nature of the change they use, I'm considering what that could mean for me as a person composing my own music.  
The VST for choral words I got seems like it's going to be a bit of a pain to figure out, but part of that may also be the composition workflow I use and that they assume I'll use.  
My issue yesterday\footnote{post blog} centered around the fact that if a single syllable was wrong, it's really hard to fix it in context.  
I'm sure that as I use the software more, I'll be better at knowing how what I type translates into what I hear.  
It's also possible that they expect me to be generating the piece word by word, rather than having all the notes and words, and only then putting them together.

I'm also now thinking about the ways that I can sample with these.  
It's trivial, as it turns out, to have a virtual choir sing at any arbitrary tempo.  
Something that's fun to explore in minimalism is two voices with slightly different tempi.  
The way that they phase in and out of sync with each other causes some fun things to appear.

Ok so that's a thing I can do.

I can also try finally getting a recording of the pieces I wrote.  
If the VST ends up any good, I can recommend it to my college director, who a few years ago now sent me the best bet he had for doing auto-singing.  
If it becomes something I can do with any real speed, it's worth considering doing it as a side hustle; I know that at least one of the choirs I sing for spends money to get practice tapes\footnote{though, I doubt it's ever instantiated onto physical media} for people to practice with on their own.  
It feels rational that I would be able to undercut whatever price they pay, but then we get into the fun conversation about morality and ethics and taking work from performers by using machines.

Anyways.

Friend has arrived, so this draft is done.

Current Pen List\footnote{for my own posterity, mostly}

\begin{itemize}  
\item Hongdian Black with Fude Nib: Esterbrook Shimmer Lilac 11/23  
\item Jinhao Shark: Diplomat Sepia Black. 10/6  
\item Pilot Preppy: Diamine Bilberry. 10/6  
\item Shaeffer (blue): Empty  
\item Diplomat: Laban Zeus Purple 11/23  
\item Kaweko: Stipela Sepia. 10/6  
\item Monteverde: empty  
\item Shaeffer Calligraphy: missing

\end{itemize}

\end{document}