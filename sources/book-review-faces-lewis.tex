\documentclass[12pt]{article}[titlepage]
\newcommand{\say}[1]{``#1''}
\newcommand{\nsay}[1]{`#1'}
\usepackage{endnotes}
\newcommand{\1}{\={a}}
\newcommand{\2}{\={e}}
\newcommand{\3}{\={\i}}
\newcommand{\4}{\=o}
\newcommand{\5}{\=u}
\newcommand{\6}{\={A}}
\newcommand{\B}{\backslash{}}
\renewcommand{\,}{\textsuperscript{,}}
\usepackage{setspace}
\usepackage{tipa}
\usepackage{hyperref}
\begin{document}
\doublespacing
\section{\href{book-review-faces-lewis.html}{Book Review of Till We Have Faces}}
First Published: 2022 August 10
\section{Draft 1}
This summer I joined a book club at my parish.
The book the parish chose to read was C.S. Lewis's\footnote{Lewis'?} \texti{Till We Have Faces.}
The story is a retelling of Eros and Psyche focusing on one of Psyche's sisters.

I really enjoyed the book, especially looking back on it.
At the beginning, I found that there was a lot of anger and world-building that I couldn't quite see the point of.
Unlike most of the Lewis books I've read, there was no immediate didactic idea I could find.

I really don't know how to write a review, but I did really enjoy the book.
Lewis does a great job taking an unreliable narrator and using the unreliability in the text.
Most of what is beautiful about the book to me entirely spoils the reveal of the book, so I'll leave it there.

The general premise of the book is that in the kingdom of Glome, which is far from Ancient Greece, there lives a king and his three daughters.
One of them is beautiful, one shrewd, and one doesn't really matter.\footnote{to the story at least}
The first two\footnote{who are the younger, because I'm bad at framing} are mentored by a Greek, who they lovingly refer to as the Fox.

Overall, I'd give the book a 4 out of 5.
It was really captivating as it continued, and it has a lot of great places for me to see how I can improve as a person.
However, the beginning was really slow for me.
\end{document}