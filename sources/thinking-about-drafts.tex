\documentclass[12pt]{article}[titlepage]
\newcommand{\say}[1]{``#1''}
\newcommand{\nsay}[1]{`#1'}
\usepackage{endnotes}
\newcommand{\1}{\={a}}
\newcommand{\2}{\={e}}
\newcommand{\3}{\={\i}}
\newcommand{\4}{\=o}
\newcommand{\5}{\=u}
\newcommand{\6}{\={A}}
\newcommand{\B}{\backslash{}}
\renewcommand{\,}{\textsuperscript{,}}
\usepackage{setspace}
\usepackage{tipa}
\usepackage{hyperref}
\begin{document}
\doublespacing
\section{\href{thinking-about-drafts.html}{Thinking About Drafts}}
First Posted: 2018 October 10\footnote{the date format that makes the most sense}
\section{Draft 3}
As I mentioned in \href{disclaimer.html}{an earlier post},\footnote{it makes the sense that disclaimer is the first post with drafts, but I'm still surprised} I plan to leave every draft of every post visible for all eternity.\footnote{barring any unforeseen circumstances, or the foreseen ones listed in the post}
All of those drafts have come from a\footnote{nominally} continuous writing,\footnote{i.e. I get distracted while writing, but the computer never closes and the document is never saved because I like to live dangerously} except in the case of the assignments I have posted, which have taken a few days.\footnote{because I am good student} 

But, just because I've published\footnote{for a liberal definition of publishing} a post doesn't mean that it's finished.
I've left posts before I was satisfied because I was too tired to continue.\footnote{sometimes I start too tired. Those tend to be the really short posts}
Since my self defined obligation is writing daily, not writing well daily, it's fine.

Nonetheless, a certain part of my personality\footnote{the Type A part} demands satisfactory writing out of me.
So, I may go back and update old posts.
If so, I'll make a note about the fact that I've edited them, and also add in the date of first publication, since that will become relevant afterwards.\footnote{I'm not going to do so ahead of time because that takes energy, and isn't worth the effort right now, and I'll probably forget to in the morning}

\section{Draft 3}
As I mentioned in \href{disclaimer.html}{an earlier post},\footnote{hmm I guess it makes the most sense for my disclaimer to be my first post that has drafts, but it was still a surprise} I plan to leave every draft of every one of my posts visible for all eternity.\footnote{barring any unforeseen circumstances, or the foreseen ones listed in the post}
All of those drafts have come from a single writing stint, except in the case of the assignments I have posted, which have taken a few days.\footnote{I think} 

But, just because I've published\footnote{for a generous definition of publishing} a post doesn't mean that it's finished.
I've more than once left a post not because I was satisfied, but because I was too tired to continue.\footnote{sometimes I start too tired. Those tend to be the really short posts}
Since my self defined obligation is writing daily, not writing well daily, it's fine.

Nonetheless, a certain part of my personality\footnote{the type a part} demands satisfactory writing out of me.
So, I may go back and update old posts.
If so, I'll make a note about the fact that I've edited them, and also add in the date of first publication, since that will become relevant afterwards.\footnote{I'm not going to do so ahead of time because that takes energy, and isn't worth the effort right now, and I'll probably forget to in the morning}

New question: what about the drafts of my written assignments that I wrote earlier than the day that they were posted?
I think I'm just going to ignore that in the past, and make an effort to remember in the future.
Great.
\section{Draft 2}
As I mentioned in \href{disclaimer.html}{an earlier post},\footnote{hmm I guess it makes the most sense for my disclaimer to be my first post that has drafts, but it was still a surprise} I'm leaving each draft of my posts up for all eternity.\footnote{barring any unforeseen circumstances, or the foreseen ones listed in the post}
But, just because I've published\footnote{for a generous definition of publishing} a post doesn't mean that it's finished.
I've more than once left a post not because I was satisfied, but because I was too tired to continue.\footnote{sometimes I start too tired. Those tend to be the really short posts}
Since my self defined obligation is writing daily, not writing well daily, it's fine.

Nonetheless, a certain part of my personality\footnote{the type a part} demands satisfactory writing out of me.
So, I may go back and update old posts.
If so, I'll make a note about the fact that I've edited them, and also add in the date of first publication, since that will become relevant afterwards.\footnote{I'm not going to do so ahead of time because that takes energy, and isn't worth the effort right now, and I'll probably forget to in the morning}
Now the question becomes: do I alert my\footnote{alleged} readers about the changes.
Advantages: looks like I'm writing even more, lets people see what polished me looks like.
Disadvantages: spams other people, I'm not even sure if I want to remember.
So, I think I'm not going to make notices when I revise a post, unless there's a significant reason to.
Woo\footnote{Whoo? I have no clue how to spell that sound and refuse to look it up} good job writing by the seat of my pants\footnote{I swear that's a metaphor I've heard} take two.

\section{Draft 1}
As I mentioned in an earlier post,\footnote{I'm not going to look up which one} I'm leaving each draft of my posts up for all eternity.\footnote{barring any unforeseen circumstances}
But, just because I've published\footnote{for a generous definition of publishing} a post doesn't mean that it's finished.
Very often, I've left the post unsatisfied, but more tired than able to continue writing.
So, I publish it and call it good enough.

But what happens when I decide to revise a post later?
Should I just add another draft number to the top?
Do I make a note after the draft, like \say{revised this date}?
Ooh, does that mean I should start writing when each draft is written?
I think I'm actually going to start doing that.
Whoops, instead of solving an issue I just made another.
Actually, this works.
Now I can just add to new drafts, or even at the top of the page, updated this date.
Great job writing at the speed of thought.
\end{document}