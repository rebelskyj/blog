\documentclass[12pt]{article}  
\newcommand{\say}[1]{``#1''}  
\newcommand{\nsay}[1]{`#1'}  
\usepackage{endnotes}  
\newcommand{\B}{\backslash{}}  
\renewcommand{\,}{\textsuperscript{,}}  
\usepackage{setspace}   
\usepackage{tipa}  
\usepackage{hyperref}  
\begin{document}  
\doublespacing  
\section{\href{trusting-myself.html}{On Trusting Myself}}  
First Published: 2025 June 2

\section{Draft 2: 2 June 2025}

I'd like to think that I trust myself.  
More than that, though, I would like to trust that I can trust myself.  
That is, my concern with trusting my body, mind, past and future self is that I don't know if I believe them.

My mind frequently tells me incorrect things, as a number of musings here have demonstrated.  
The fact that I can do a task knowing that it will trick my mined into moving in a productive direction shows how ridiculous the concept is.  
My mind knows what can trick itself and is able to pull a trick there.  
I've often heard that you cannot tickle yourself,\footnote{though I guess I do have friends that claim the opposite} so it's strange to me that this would work.

Ok, but just because I cannot trust every single thing does not mean that I should not trust my mind in general.  
I have a lot of people that I trust on some issues but not every issue.  
I guess when I think of trusting myself, I don't so much think of learning what parts of me tend to be trustworthy and what parts tend to deceive me.

Trusting the me of the past and future, on the other hand, feels so utterly impossible.  
I should know what I am going to want tomorrow, and whenever I make a prediction, I tend to be correct.  
However, my own motivation is always far lower than I think that it will be; there must be a way for me to recognize that motivation is always retroactive.

Finally, and completely unrelatedly, recently I talked about trying to divorce myself from concepts of linear time.  
I realize that's not true in many regards: I'm now actively scheduling my sleep, and I set up appointments with people at set times.  
Today I'm even trying to chronicle where my time goes.\footnote{spoiler: lots of blocks of flow work}

But, it still feels emotionally true that I'm not letting linear time define me.  
Why?

First, even though I'm scheduling the wake time in the morning, I'm not scheduling the bed time quite as heavily.\footnote{though I am going to start tracking it}  
I'm also taking a two hour nap in the middle of the day, which is not at all culturally normative.\footnote{for those over the age of four, at least}  
I used to have the boundary of no work at home, before 7 am, or after 8 pm.  
I no longer have these boundaries, because I have found that it's healthier for me to instead set such reasonable boundaries as \say{when I say I'm done, I am.}

Huh, I guess that I do trust myself, at least in terms of when I should and should not be working.

The final reason that I'm feeling less constrained by linear time is that the work schedule I'm attempting right now simply has me generate potential action items.  
Each day I get three attempts to go through them, and in theory I will cut myself off at a certain point.  
Is this just how normal people structure their day, except with the layer of three blocks? Possibly.

So, do I trust myself?

I think that I trust myself more than I thought that I did.  
Horrifying, I can't even trust my own sense of trust in self.\footnote{this is a joke, if it wasn't clear}

\section{Draft 1: 2 June 2025}

This post went to \href{what-we-dont-post}{the wrong places}

\section{Daily Reflection}

\begin{enumerate}

\item Did you journal by hand, and do you feel like the stormy questions in your mind got on the page?

I think so! I unfortunately did spend about two hours at work trying to solve a twenty minute question, so that wasn't great.\footnote{not because my mind wasn't working correctly, just because I'm feeling unproductive}  
I don't entirely know what's going on in my head right now, but there is something for sure.

\item Did you do your best to sit in still silence?

Not really so far. I did turn off my audiobook when I started cleaning my pen and working.  
In general, though, this morning hasn't had still time.   
When can I schedule that? Maybe after I finish this folly I'll take five to just sit down. There are some nice chairs here that look really good for a meditation sentence.

\item Are you making sure that each task is given your full attention, not just because the task deserves it, but because you deserve the luxury of doing a single thing at a time?

I was listening to the new audiobook on my walk to work, and at one point I did start one handed juggling. Other than that, generally! Though I am just now finishing a pair of conversations that were happening concurrent with this writing, which is not fair to any of the tasks.

\item Are you focusing on your posture and breath?

Yeah! It feels really nice to be sitting marginally better, and I do find that I feel more attractive, present, and emotionally comfortable when I stand and sit with my shoulders held back.  
In general breathing deeply, though it is taking some extra thought.\footnote{this new keyboard is really fun, but wow my typing accuracy is currently going horribly downhill, which I think is mostly because the keys are just slightly differently positioned than on my normal keyboard, or maybe are just slightly smaller. Going to go back to old keyboard for now}

\item What in your body is holding tension right now? How can you fix it? When will you fix it?

Right now the tension is mostly in my neck and upper shoulders, and i think that neck rolls would help to fix it. I'll do that during the five minutes of silence.\footnote{ugh that does mean I should add five minutes of listening to my body. Ten minutes is still an appropriate amount of time to spend on tasks}

Oh also I keep wanting to cross my legs, which is not great. I don't know if that's habit or hips.

\item Comments on sleep?

Not really. I had my alarm fifteen minutes earlier this morning, woke up eight minutes before the alarm, and then struggled to leave the bed for five minutes.  
Now that I'm up, I feel slightly less awake, though I wonder if that might be resolved by naptime.  
I also only woke up once that I remember, but it did last much longer than normal.  
Two hour nap for today.

\item How's eating going? In particular, how are you doing with eating plants and unprocessed food?

Eating is not really going.  
I remembered to have something before leaving home this morning, but something was just a protein shake.\footnote{which is not nothing, and is in fact good for me}  
When I get back to the office\footnote{so, after I take my ten minutes of silence, probably}, I'll make sure to make my bowl of oats.  
There are also cookies in the lounge near me, so I might go for that.

\item Are you neglecting any of your familial obligations? If so, how can you rectify this?

Nope! Helping my brothers where needed.

\item Cleaning: what is the biggest priority you have right now, and what is the next action item for it?

Biggest priority remains the front of house.  
Next task is still just throw away and hide. I'll try to do that after nap time today.

\item Thesis: current task. What's preventing you from finishing it? How will you remove that obstacle?

Currently I need to get the many jobs running.  
What's preventing me from finishing is that the code to send it on the cluster isn't working, and I'm not entirely sure that I am putting in the right options.  
I'll take a step back, and look at the full input file and set of input file creation to ensure that I'm getting things done correctly.  
Once the jobs are set up, I will monitor them at least a few times, just to make sure that nothing catastrophic is happening.

\item Thesis: next task. What will you need to be able to do it?

Next task will be analyzing the data from the time checks I did last week. I have the data on a jump drive\footnote{thumb drive? usb drive? flash drive is usually what I've called it in the past}, and I think that I've ported it onto my computer.  
The goal of this set of trials was to see whether RebelFit runs at different speeds depending on the kind of data going in, so I put in an experimental spectrum, blank spectrum, noise, and noise times a large number all with the same steps.  
I then had it call the catalog loader a bunch of times for each version.

I'll need to do the following:

\begin{itemize}

\item Find median, mean, and standard deviation of times for each run

\item Make a table of the information and put it in the paper, at least for now

\end{itemize}

\item What's the next job you're applying to?\footnote{note that this might be a \say{things we don't post} but}

Right now my goal is going to be getting my resume on USjobs.

\item Are you intentionally trying to spend time with others?

Yeah! I just finished my morning writing pair, and I intentionally put us where I was told there was another group of people I knew writing. They are not, however, here.  
Maybe I misread or misheard them.

\item Are you doing your absolute best to ensure that you and those you interact with view the interactions in the same light? Are you sure?

Not an issue so far! Thankfully I have clearly communicated my intentions.  
I am meeting with my advisor today to make sure we're not talking past each other, and I'm about to try to schedule a meeting with someone to see if we're interacting the way I think that we are.

\item Are you keeping up on this daily set of reflection questions?

Look at this! Two in a row.

\item Are you keeping up on writing the follies? If not, what's in the way?

Presumably I'll be able to write and then post this one today. If not, it's because I'm struggling to get thoughts.

\item How are the long form follies coming? Do you feel like they're weighing you down right now?

Haven't started. I do absolutely feel like the faith one is weighing me down, but I think that spending time on it before end of day will consume the rest of my day, so I want to make sure that, at the very least, the computational jobs I have are submitted and ready to run.

\item Are you writing poetry? When, and what were your takeaways from the previous day's writing?

I wrote poetry last night.  
In general, I write a lot of sad lyric, especially about feeling as though I wear a mask.  
Last night, though, I got called to write some love lyric, which is always nice.  
I'm drifting further away from strict meter and rhyme with every passing day, but that's not necessarily an issue.

Sing-songy words aren't actually a plus in modern music writing, and it's really difficult to get highly metered prose to sound natural.  
Still, metered and natural sounding prose is absolutely stunning, and it did bother me an inordinate amount when the album I was listening to absolutely mangled word emphasis and sentence cadence.

\item Are you making music? If not, what is in the way?

No, I think that the guitar might be in the wrong place. I'm also not writing music because it's been less than twelve hours since my last musing, and most of the remaining time has been spent sleeping or working.

\item Web novel?

Nope. I don't think that I'm even going to plan on it today, because I really want to try to unload the long-form folly.

\end{enumerate}

\end{document}