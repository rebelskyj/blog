\documentclass[12pt]{article}[titlepage]
\newcommand{\say}[1]{``#1''}
\newcommand{\nsay}[1]{`#1'}
\usepackage{endnotes}
\newcommand{\1}{\={a}}
\newcommand{\2}{\={e}}
\newcommand{\3}{\={\i}}
\newcommand{\4}{\=o}
\newcommand{\5}{\=u}
\newcommand{\6}{\={A}}
\newcommand{\B}{\backslash{}}
\renewcommand{\,}{\textsuperscript{,}}
\usepackage{setspace}
\usepackage{tipa}
\usepackage{hyperref}
\begin{document}
\doublespacing
\section{\href{reflections-on-readings-holy-family-c.html}{Reflections on Today's Gospel}}
First Published: 2018 December 30

Luke 2:52: \say{He went down with them and came to Nazareth, and was obedient to them; and his mother kept all these things in her heart.}

\section{Draft 1}
Today's readings, as the name of the day would suggest, all focus on family.
The first reading focuses on one family, the prophet Samuel's.
We learn that he is offered as a \say{perpetual nazirite,} or holy one.\footnote{1 Samuel 22}

The second reading explains what Paul sees as an ideal family.

Finally, the Gospel speaks of one of the only instances of the Holy Family where all is not perfect.
The young Jesus goes missing after Passover, and is found at the Temple.
This reading comes from Luke, and like most of Luke, I really enjoy it.

The reading tells us that children are inquisitive, and that is a good trait.
It tells us that mothers should show concern for their children.
It doesn't say much about a father's role, but the reading can't do everything.
And, like much of the dialogue in the Bible, it contains some dialogue that seems hard to picture actually happening.
\end{document}