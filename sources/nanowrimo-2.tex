\documentclass[12pt]{article}[titlepage]
\newcommand{\say}[1]{``#1''}
\newcommand{\nsay}[1]{`#1'}
\usepackage{endnotes}
\newcommand{\1}{\={a}}
\newcommand{\2}{\={e}}
\newcommand{\3}{\={\i}}
\newcommand{\4}{\=o}
\newcommand{\5}{\=u}
\newcommand{\6}{\={A}}
\newcommand{\B}{\backslash{}}
\renewcommand{\,}{\textsuperscript{,}}
\usepackage{setspace}
\usepackage{tipa}
\usepackage{hyperref}
\begin{document}
\doublespacing
\section{\href{nanowrimo-2.html}{NaNoWriMo Redux}}
First Published: 2022 October 28
\section{Draft 1}
Way back \href{nanowrimo.tex}{in the initial iteration of this blog}, I attempted the National Novel Writer's Month (NaNoWriMo).
I think\footnote{because the past is a mystery shrouded in darkness and memory} that I mostly just tried to hit the wordcount\footnote{50000 words} in these blog posts.
My logic then was that I had far more free time, as I was lacking the obligations of \say{sport, ensembles, (most of) my instruments, much of the studying, a job, etc}, and so should be able to do it.
Today much of that is true as well, and I would really like to see if I can write a novel in a month.

Of course, it's not going to be good, but I've started actively scheduling time in my day to write\footnote{hence more of these posts lately}, and it's worked well for creating ideas.
I also really have a lot of ideas I want to read, and the only realistic way that will happen is if I become the writer for them.
Back then I was apparently writing around 750 words a day, which is far more than my posts lately\footnote{which makes sense, since back then I was writing it with the express purpose of chronicling my time in a far-off land}.
But, having now written a book\footnote{which I just now realize I don't know if I ever posted about? Ah \href{reflection-july-2022.html}{I did}}, I feel more able to write a long\footnote{ish}-form piece of creative prose.
\end{document}