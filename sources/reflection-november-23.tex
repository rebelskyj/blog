\documentclass[12pt]{article}[titlepage]
\newcommand{\say}[1]{``#1''}
\newcommand{\nsay}[1]{`#1'}
\usepackage{endnotes}
\newcommand{\B}{\backslash{}}
\renewcommand{\,}{\textsuperscript{,}}
\usepackage{setspace}
\usepackage{tipa}
\usepackage{hyperref}
\begin{document}
\doublespacing
\section{\href{reflection-november-23.html}{Monthly Reflection}}
First Written: 2023 November 30

\section{Draft 1}
Wow!
Another month has passed me by.
This month didn't feel like it disappeared in a blink like the past few have, which is really nice.
 
I feel like I've grown as a person and in some of my interpersonal relationships, which is always really nice.
It was a really chill month in many regards, if only because the only travel I did was for family reasons, rather than anything even slightly professional.
 
Relevant/notable things I did this month:
\begin{itemize}
\item Started writing my first first author paper.\footnote{kind of. I wrote one in UG but that's kind of just sat around for a bunch of years, and I'm unsure if anything will ever actually come of it}
\item Started playing DnD again
\item Got sick
\item Ran into friends from the past and present at random and unexpected places.\footnote{to be fair, I suppose that the existence of coffee shop AU's as such a large genre does do a lot to make them seem like a place you might run into people}
\item Got auctioned off!\footnote{I don't think I've talked about this. The reason for the auction is upcoming and feels like it might deserve a post all its own whenever it happens}
\item Saw 12th night again!
\item Solo taught my first class
\item Had Thanksgiving with my family!
\item \href{nanowrimo-6.html}{Did NaNoWriMo for the second time!}
\end{itemize}
 
Let's see how that lines up with my predictions for the month.
 
\begin{itemize}
\item Going home for Thanksgiving. I did that, and I'm glad I knew it would be notable. 
\item Doing NaNoWriMo with a friend. It's been really fun to write with the friend, and it sounds like we're planning to continue our morning writing sessions!\footnote{where I'm currently writing this, interestingly enough} 
\item I'm guest teaching a class, which is fun. It was fun! I don't know if I did a great job, but I didn't get an angry email or letter about it, so I assume that I did well enough. I suppose we'll find out in the future. 
\item I'm going to a cool discussion about Pascal. I ended up being too sick on the day of the talk, which is sad. I guess I did put that I got sick, so at least I have the event that ended up replacing it. 
\end{itemize}
 
It's always nice when I actually do the things that I planned on doing\footnote{mostly, at least}.
It's even nicer when I also do things that are fun that I did not know I would be doing.
Now that we've gone through the discrete events, let's see how we did on the monthly goals for October.
I always feel a little like I've let myself down when I go through my monthly goals, which is a fun thing to reflect on.
It's interesting that I continually think that I will suddenly be able to do things that I demonstrably have not been able to do.
With that disclaimer\footnote{for myself, not the readers,} out of the way, let's see what we wanted to do, did, and how that overlapped.
I'll delete commentary in the initial goals, except where I want to comment on them in reflection.
 
\begin{itemize}
\item Write 1700 words a day for my NaNo project.
\item Write a chapter of Jeb every day.
\item Blog every day.
\item Stretch every day.
\item Improve the cleanliness of my home.
\item Pray better
\item Write letters to friends.
At a minimum, there are three people I owe a letter to and at least one other person I would like to send a letter to.
We're almost at the time of year that holiday cards become appropriate, and I could consider doing those this year. 
\end{itemize}
 
Honestly, I did much better than I thought I would.
I'm not sure if I suddenly learned how to set reasonable goals or if I just had an unusually good month, but either way:
 
\begin{itemize}
\item Until yesterday, I did this with flying colors. I averaged about 1765 words a day, with minimal variance. I never went above 1800 and never went below 1745 words.
Yesterday, I made the intentional choice to finish the book, rather than write an extra 3000 words, and I'm comfortable with intentionally changing my goals.
\item I didn't do great on this. I managed to pull really far ahead in the early days of the month, and then immediately stopped writing more or less at all, and ended up with as much of a backlog as I began.
That's a little bit of a shame, but I do think that I'll have an easier time writing in the upcoming months, since I now have the book plotted out at least a little.
\item With one exception, I wrote a blog every day this month. It's nice to have a record of what I was thinking about for at least a little bit of each day this past month. 
\item I stretched more than I did in October, but still not daily.
I do feel like I've started to be more aware of my posture when I write, though, and I do sometimes even catch myself slouching too far over and sit up a little straighter.
\item I don't quite remember what my home looked like at the start of the month. I think it's a little better than then, but not markedly so.
That's probably a goal that's worth having again, if only because I do want to start to clean my home. I'm a little surprised that I didn't put it in my daily reflections. I'm sure there's a reason for that, but I can't remember why right now.
\item I don't know if I would say that I have started praying better.
I think maybe a little.
It's definitely a goal I want to focus on even more this month.
\item I wrote the three letters to friends, I think.
Holiday cards still remain an option, and have the benefit of me being able to write the same letter, more or less by rote.
Honestly, I could just type and print out the letters, but that feels kind of sad.
I know that I greatly prefer hand written letters to typed form letters. 
\end{itemize}
 
Let's now shift from facing backwards to looking forwards.
This month, I am excited for:
\begin{itemize}
\item Consequences of the auction\footnote{this sounds much more ominous than it really is}
\item Getting better at prose
\item Going home for Christmas
\end{itemize}
 
Welp, the list is shorter once again.
This is not because I do not plan on doing anything, but rather because I have much of the month open, which I'm excited for.
It'll be fun to see what ends up filling my hours.
 
Goals for the upcoming month:
\begin{itemize}
\item Finish at least three of the books that have been sitting maligned on my \say{Currently Reading} shelf.
There's an app I use to keep track of my reading.
This past summer I set the admirable goal of cleaning it out.
Of course, that did not happen.
Moving forward, though, I would like to start doing that.
\item On a related note, read two of the books that I've checked out from the library.
If I divide the number of books I have checked out by the number months I expect to have left in the graduate program, I get a number I don't like.
There are absolutely some books that I've checked out to use as the occasional reference, rather than to read fully through.
They are, however, in the minority.
I would like to start spending more of my time intentionally, and part of that means reading more books to grow as a person, rather than just fluff to fill the hours.
\item Following that line of thought, I want to spend my time more intentionally. I know that I have a bad habit of just letting life pass me by, getting distracted by random things on my computer or phone.
I would like to not do that  
\item I want to write a sonnet every day.
My poetry friend suggested that writing lots of sonnets was the best way that he knew to improve at writing meter.
There was an implication within it that I should, or at least might want to consider revising the poems as well.
I still don't really know how to workshop and revise poetry, especially fixed forms like the sonnet.
That could be a fun learning experience.
\item I want to read a book on the craft of writing start to finish.
I know that this really bumps up the number of books that I'm going to try reading, but that's probably ok. I like to read, and I'm getting faster at reading again.
Then again, most of the books I have as reading goals this month are not beach reads, which might mean that trying to read them too quickly is a bad idea.
Still, we'll do whatever feels right.
\item Get better at prayer. Reading the book on apologetics is making me realize just how much I do still think of prayer as an emotional, rather than will-based experience. I think that shifting my mental conception might be really helpful
\item Clean my home. It's at a point right now that I think I could leave it almost perfect when I go to visit family for the holidays.
It would be really nice to be able to come back to something clean, and that's a great motivator.
\item Stretch and do some other form of exercise every day.
Even if it's just a few pushups and a few situps, I want to get in shape. I have friends who I can, in theory at least, lift with.
And, I do really value stretching.
\item Get better at touch typing.
There's a website I found which slowly introduces you to more letters as you learn them, so that you can start to rebuild the muscle memory of where fingers go and what letters belong to each of them.
\item Write letters to friends.
I have a bunch of people I'd like to write a letter to.\footnote{this month I'll try to be smart and write down exactly what friends I want to write a letter to and who I've written a letter to so that I can keep track and not accidentally forget anyone or spam them with repeat letters.
As nice as a handwritten letter is, getting two identical ones probably wouldn't feel great.}
\item Get ahead on Jeb. Ideally, I would like to be five chapters ahead at the end of the year.\footnote{because five chapters is a common promotion that authors run to entice people to pay for their writing on the site I use. It could be fun to start monetizing my work.}
\item Get better at embroidery.
I've allegedly started to learn, but I want to spend at least a little time every few days.
Then again, the best way to build a habit is daily.
Maybe doing like a strand a day would be a good idea, at least until I go home for break 
\item Play guitar every day.
I want to start recording an album soon.
I need to be able to play the music on it.
Even if it's just a few chords and a scale, I want to get back into guitar.
Replacing the strings will probably help with that.
\item I want to keep up the daily blogging. That seems like an easy enough goal.
\item I want to finish a full draft of the paper I'm working on. Without having finished that, I feel like I'll have no way of knowing when, exactly, I've finished the research for the paper.
\item I want to drink more water. I've been pretty dehydrated lately. 
\end{itemize} 
 
So, that's a very ambitious set of goals.
Let's see what we can do for a daily reflection in a way that isn't going to be a full musing in and of itself every day.
I'm going to move some items around so that like elements are closer in daily reflection, as opposed to being in the more or less stream of consciousness order that they are right now.
I think that a part of me thinks that higher items on the daily reflection reflect my own internal belief that it's more important to hit the goals.
I don't know if that's true.
I'm changing enough this month that I don't know if I really want to also add priorities.
It might be fun to do nested lists.
I'll see if that works.
Items this month will be in the order that I remember/find them in the list. 
\begin{itemize}
\item Hobbies:
\begin{itemize}
\item Did I embroider today?
\item Did I play guitar today?
\item Did I practice touch typing today? 
\end{itemize}
\item Reading
\begin{itemize}
\item Have I made progress on my Currently Reading Shelf?
\item Did I read the book on craft?
\item Have I read the library books?
\end{itemize}
\item Writing
\begin{itemize}
\item Did I write a sonnet?
\item Did I revise a sonnet?
\item Did I blog?
\item Did I write ahead on Jeb?
\item Letter to friends?
\item Paper? 
\end{itemize}
\item Wellness
\begin{itemize}
\item How well did I pray?
\item Did I clean my space?
\item Did I spend my time well?
\item Did I stretch?
\item Did I exercise?
\item Water? 
\end{itemize} 
\end{itemize}
 
Oof, that's enough different items that I had to double check my list repeatedly to make sure that I didn't miss anything.
Still, I think that I'm at a place right now where I'd rather aim high and fall short than try to meet small goals.
We'll see if I still feel that way in the middle of the month.
 
Looking forward, I do want to start composing again, as I have mentioned.
In January, I think that I would like to start doing a small composition exercise every day, but I suppose I'll see how I feel at the end of December.\footnote{it's wild to me that it's basically December now. That's terrifying to think about.}
Since I'm currently writing this post at the beginning of the day, I will wait to do the daily reflection until the end of the day.
That also means that I'll wait to post this until the end of the day, which might mean that I end up coming up with a second draft as I have different ideas for what, exactly, I want to spend next month doing.
Just sitting here, I've found that the list has grown a lot in the past hour, which is nice and a little terrifying.
 
Coming back to the post at the end of the night, I didn't really have much to add, which is nice.
Time for my last November daily reflection!
 
Daily Reflection:
\begin{itemize}
\item Did I write a chapter of Jeb? I wrote about half of one, which I think is enough for tonight. I'm ok with going to bed a little earlier instead of writing a little more. 
\item Did I blog? I like today's blog. It does what the musings are supposed to do, which is refocus me. 
\item Did I stretch? I did end up stretching after posting last night's musing. Will do so after this one. 
\item Am I doing better at prayer than a rushed and thoughtless rosary? I did not make it very well focused last night. Maybe tonight.
If not, I guess I'm still trying, which is worth something. 
\end{itemize}
\end{document}