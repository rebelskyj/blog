\documentclass[12pt]{article}[titlepage]
\newcommand{\say}[1]{``#1''}
\newcommand{\nsay}[1]{`#1'}
\usepackage{endnotes}
\newcommand{\1}{\={a}}
\newcommand{\2}{\={e}}
\newcommand{\3}{\={\i}}
\newcommand{\4}{\=o}
\newcommand{\5}{\=u}
\newcommand{\6}{\={A}}
\newcommand{\B}{\backslash{}}
\renewcommand{\,}{\textsuperscript{,}}
\usepackage{setspace}
\usepackage{tipa}
\usepackage{hyperref}
\begin{document}
\doublespacing
\section{\href{editing.html}{On Editing}}
First Published: 2023 December 9

\section{Draft 3}
Irish legend says that the airs\footnote{melodies} to the most striking and memorable Irish folk songs were not composed, but discovered.
Considering it that way, editing makes far more sense.

If what I write is not, in fact, a creation but a discovery, then editing makes complete sense.
The words on the page are not the discovery, but my pale imitation of them.
Like a sculptor with clay, I can remove or add material, shifting it as I need until I reach the shape that I see in my vision.

For all that I can imagine my writing as that, I tend not to.
It feels a little too pretentious to say that the fun web serial I write about a boy wandering through life is some profound gift from the cosmos.\footnote{as I write that sentence though, I do have to remember my historic takes, which do include anything creative as being a small mirror of the Creator, and then it feels wrong to say that anything I craft does not, on some level, come to me as a gift}
I tend to think of my writing as the words that I put on a page.

When I think of writing in that way, editing becomes something hard.
Deleting words is exactly destroying something I've created.
Adding words is grafting something onto a creation.

As I've mused about editing over these past drafts, my own feelings on the art have changed.
I think that my issue with editing stems directly from the fact that I consider the writing to be the action, rather than the mode of transportation.
My goal is not to put specific words on a page, but to convey a narrative.

Of course, that becomes an issue of its own.
If I do not know what narrative I'm trying to convey, how can I know which words belong?
There are some obvious choices, such as avoiding cliche.
Even if I know what I'm trying to convey, I also need to take a step even further back to consider why I'm trying to convey what I'm trying to say, and why via writing.
Doing that takes far more mental effort than simply dumping words on a page and seeing where the story takes me.

And, I do not dislike the places that the story and my mind, less strictly filtered, tend to roam.
I suppose that there's something to be said for shaping what I have.
When I see a piece of wood that kind of resembles a tiger, carving it to make it more closely resemble that shape is an obvious choice.

And so, I suppose that the answer has been in front of me this entire time.
To become better at editing, I need to become better at divorcing my art from myself as artist.
I need to see what I create as nothing but how well it communicates what I want it to say and who I want to say it to.
If my goals have changed between drafts, it's only reasonable that the words and structure would need to as well.\footnote{that's really what they call meta writing wow. I'm honestly really proud of that sentence}
Daily Reflection:
\begin{itemize}
\item Hobbies:
\begin{itemize}
\item Did I embroider today? It's still at work, so no, not so much.
\item Did I play guitar today? A little, I plan to do a little more after posting this because it turns out the place I used to play open mic at shifted the time earlier, so now I can actually go again.\footnote{the mic starting at 9 is difficult when my bed time is like 9:20}
\item Did I practice touch typing today? I will once I finish this musing!
\end{itemize}
\item Reading
\begin{itemize}
\item Have I made progress on my Currently Reading Shelf? Not in the way that I meant the goal, but I have been listening to an audio book today.
\item Did I read the book on craft? I believe in my ability to do this tomorrow.
\item Have I read the library books? I also believe in my ability to do this tomorrow.
\end{itemize}
\item Writing
\begin{itemize}
\item Did I write a sonnet? Need to do that now. I'm not super happy with it, for all that there was a nice line or two.
\item Did I revise a sonnet? No, but I did make plans to revise on Monday, so excited to see my poem get kindly ripped to shreds.
\item Did I blog? I mused no less! Look at this abomination.
\item Did I write ahead on Jeb? I'm tired right now. I'm going to break my no Jeb on Sundays rule because I don't want the panic of needing to write a chapter on Monday morning, and I don't have the space to do it right now.
\item Letter to friends? Got one from a friend! It was really sweet.
\item Paper? Nope.
\end{itemize}
\item Wellness
\begin{itemize}
\item How well did I pray? Not great.
\item Did I clean my space? A tad.
\item Did I spend my time well? The parts of today that were spent with others, yes. The rest of the day, not so much. I suppose that the time I've spent on this musing is time well spent.
\item Did I stretch? A little!
\item Did I exercise? I went to the gym with a friend, and that was really nice.
\item Water? Not enough, but I did try, which is worth at least a little
\end{itemize}
\end{itemize}


\section{Draft 2.1 Started a sentence that went too long}
If we take a moment to imagine writing not as creating, exercising some small sliver of the Divine in what we do, and instead consider it like the early Irish considered song: something found.
\section{Draft 2}
I find it kind of funny that what really seems to separate the mediocre writers from the good writers from the great writers is not writing.
Having listened to early versions of Piano Man, it is nothing like the final song that was released.
Terry Pratchett famously rewrote his books over and over until each word was what he wanted.
Editing is what separates the bad from the good writers.

It's obvious once stated, of course.
No one creates perfection constantly without needing to toss away words that don't fit, or redirect a thought that's strayed a little too far.
However, it does feel strange that there is an entire art, editing, which is so vital to proper writing.

I think of the different activities that I do a lot.\footnote{unsurprisingly, since I, you know, do them, and generally think a lot}
It's a bit of an issue, because there isn't really a way that editing is like anything except itself.

In music, I do occasionally delete lines that I've written.
Of course, that is also editing.
Playing music, however, I never unplay a note.

In reading, I can never forget having read a word.
In fact, doing so would be actively harmful towards my ability to understand a text.

In cooking, ingredients don't get separated once combined.

Maybe that's why I have such trouble with editing.
Probably because I analogize my different crafts so much, it's become easier and easier for me to do so.

Working off of a recipe is like reading sheet music.
Improvising a song is like trying a crochet project off of vague feelings.
Writing something is like cooking a meal.
Even when constrained, I have some freedom in what I choose to do.

What is editing like?

Editing is like trimming the fat off of a roast before cooking, so that you don't have to deal with silver skin.
But, you don't add more meat\footnote{or, at least, I don't add more meat. If you use meat glue, that's between you and the mirror}.

Editing is like practicing a difficult section of music until it flows.
But, you don't remove difficult notes\footnote{or, at least, you shouldn't.} when they don't feel right, you play until they do.

Editing is like revising a pattern.
Sure, that's true, but that's like saying the sky is like the air.

Of course, at this point, I kind of feel like I'm in a shifting goalposts meets no true scotsman land.
If it feels like editing in a skill, I lump it into the craft of editing.
Otherwise, whatever it is clearly is not editing.

This doesn't even get into the whole issue of what the difference is between revising and editing and redrafting.
\section{Draft 1}
One of the hardest parts of writing is editing.
It, like so much of life, requires balance and precision.
Encouraging the voice in your head to edit too much or too quickly, and I\footnote{there's probably something wrong with the sudden person shift, but I am fine with it} find that it becomes nearly impossible to get anything down onto the page.
On the other hand, I do know that it is an essential skill to develop.
For all that my ability to write words well on the first attempt continues to grow, it has not made the writing I put out as much better.

A great analogy comes from music.\footnote{I was reminded of the fact that I have not touched my instruments in far too long today, and so that might be why it's on my mind right now}
Some might think that I'm going to say that writing is like writing music and revising is like revising music.
That would be fun, but no.

Writing, generating new content, is like sight reading.
It's one of the most impressive things to do in front of someone, and in many respects, it is crucial to development as an artist.
To become better at sight reading, the first thing to do is just try sight reading.

This gets its own analogy.\footnote{which I should get rid of in the future drafts? maybe? This musing can ramble on and I'm actually ok with that}
I've seen a lot of lifting advice on the internet, describing exactly how best to optimize every second of one's life around the gym.
However, one piece of advice has stuck with me.

If you're just starting lifting, as long as what you do is not actively wrong, it's going to be more or less as effective as any other workout.
Whether one should squat or leg press is a valid question, but if you do not have a history of lifting, doing either will help a lot.
Only once plateauing does it really matter exactly how many reps and of what weight for how many sets on what workout you do.

Music is a similar way.
For most people, the best way to get better at sight reading is to sight read.
Honestly, for most people, the best way to get better at music is just to make music.
I've blogged about that \href{how-i-practice.html}{before}, though it has been almost five years.
My relationship with practice has changed a fair amount, which is probably fair, given that I no longer study music as an explicit part of my studies.

Anyways, I know without a doubt that I am past that point as a musician generally, for all that I am not past that point on every instrument.\footnote{ope I'm also probably near that point as a lifter, but less so. Hmm, actually I'm probably not right now. For all that I know about what weight I can do, it's not enough of a routine for me to feel like I'm plateauing yet}
At this point, the way that I grow as a musician is by working on skills, and working intentionally.
If we tie this directly to sight reading, I'm at the point in my musical career where I would benefit a lot from just practicing scales and common chord progressions, since that's the basis of most compositions, and therefore makes it easier to sight read.

An, returning to the real point of this musing, right now I know that, while continuing to generate new words will still help me incrementally, right now I need to really focus on my craft.
Now, there are a few ways that I'm trying to do so.

First, I'm trying to remember sensory cues as I write my fiction.
I know that, among all of the many ways that my writing is lacking, that is the one that I am most prone to.

Second, I'm writing more poetry again.
That, at a base level, forces me to remember how to construct a narrative in a limited number of words.
It also makes me incredibly aware of the specific cadence of the words that I use, as well as their ability to rhyme.
Though rhyming may not be incredibly necessary to my prose writing, cadence certainly is.

Third, as you might have noticed, I'm editing my writing more.
I think that I might be starting to go too far, if my FFF from \href{flash-fiction-230.html}{yesterday} is any indication.
I tried to start the story at least a dozen times, giving up within a few sentences.

Then again, a voice reminds me right now, none of the first dozen ways that I wanted to start a story were where I ended up.
I took a few cute phrases or images from the first sentences into the next one.
I am still worried that I'm getting to the point of writer's block again, my internal editor shutting down any prose I try to write.

I'm reminded of some advice from the creator of NaNoWriMo.
They recommend hiding your internal editor for the month, accepting that whatever you write may or may not be absolute garbage.
I manage to do that, maybe too well.

However, there is a time for planting, and a time for harvesting.
There is also, crucially in nearly every gardening book I've read\footnote{Yes, the number is far greater than one, no I don't want to explain why, no it wasn't just a phase (I think that it was like one a year for a while)}, a time for weeding, and a time to decide what you want to plant.
As a writer, finding time to write is not the biggest struggle for me.
It's important to me, and I've gotten better about prioritizing it and not feeling guilty for doing so.

As a writer, finding opportunities to share my writing hasn't been particularly difficult.
I have my web serial, which lots of people seem to enjoy, and I have a few friends who read my blog and sonnets.
That's most of the writing I do, and it's nice that it finds a home.

The time to decide what to write isn't even something I really struggle with.
For all that I often enough have trouble coming up with a musing idea, that's never a problem that lasts more than like 4 minutes of trying to write something.
At a step higher, I know that I need to write my web serial, my blog, and my sonnets.

And, while I could strain the metaphor further, talking about soil pH and making sure that you plant proper plants where they will receive ideal sunlight, to say nothing about seasonal considerations\footnote{to name just a few}, I will instead keep the metaphor relatively plain.
Part of weeding, at least the way it exists in my mind, is also pruning.
A fruit tree will grow more fruit, somehow, when its branches are effectively pruned.
In theory, the same is true of writing.

Deleting words can make the sentence clearer, and can make what you are trying to say clearer.
The book on writing well\footnote{hah, get it} that I'm reading does focus on that a lot.
The point of writing is connection.
If the words you wrote do not serve to convey what you are trying to say, then they must go, just like the beautiful rosebud that needs to be snipped because it's out of alignment.

So, why am I concerned?
Well, if editing is like pruning\footnote{yes, I've changed the metaphor, deal with it}, then right now I'm worried that I might start pruning too far.
The work might start to falter, unable to get the nutrients it needs.\footnote{I've lost control of the metaphor. Whoops}\end{document}