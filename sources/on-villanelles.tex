\documentclass[12pt]{article}[titlepage]
\newcommand{\say}[1]{``#1''}
\newcommand{\nsay}[1]{`#1'}
\usepackage{endnotes}
\newcommand{\1}{\={a}}
\newcommand{\2}{\={e}}
\newcommand{\3}{\={\i}}
\newcommand{\4}{\=o}
\newcommand{\5}{\=u}
\newcommand{\6}{\={A}}
\newcommand{\B}{\backslash{}}
\renewcommand{\,}{\textsuperscript{,}}
\usepackage{setspace}
\usepackage{tipa}
\usepackage{hyperref}
\begin{document}
\doublespacing
\section{\href{on-villanelles.html}{On Villanelles}}
First Published: 2022 January 22

Prereading note: I've decided I'm going to try to shoot for 500\footnote{I said 1000 initially but that's so many} words a post from now on so that I know what 1000 words feels like.

\section{Draft 1}
Today as I pulled up my text editor to work on the blog post, I realized that I was about to hit writer's block.
That is, I could not think of anything to write today.
Thankfully, old Jonathan knew this day might one day arise, and helpfully \href{restarting.html}{gave me some ideas.}\footnote{side-note: I do really think of my past, present, and future selves as distinct entities, which probably says something about me/influences the way I act but}
One of them was to write about my newfound enjoyment of villanelles.

I've written before\footnote{though I'm not looking for where right now} about how, among my many goals for self-enrichment, I have had a goal to write more poetry.
When I began this blog, probably due to my location, I focused on sonnets.
For whatever reason, sonnets don't really speak to my soul the way other forms do.
I just cannot connect to pentameter.

That being said, one thing that I loved about sonnets is their constraint.
I found that having the very restrictive rhyme and rhythm helped me not only to write them, which was the first-level goal of my doing so, but also to write better outside of poetry, because I became more aware of cadence.
When I wanted to restart writing poetry again, therefore, I thought it might be helpful to find another constrained pattern for writing.

I don't know if you've ever tried searching the internet for a list of poetic forms and how they work, but I could not find a good list.
So, I used my brain and thought about who in my life might have a list of poetic forms easily at hand.
And so, I messaged one of my close friends, who was an English major.\footnote{who I also met in London which is pretty fun}
She pointed me to the villanelle, which is constrained in a very odd way.

A villanelle is a nineteen line poem, comprised of six stanzas.
Five of the stanzas contain three lines, and the sixth contains four.
The second lines of each of the stanzas rhyme with each other, which is fairly normal.
The odd constraint comes with the interplay of the first stanza and the last line\footnote{s} of each remaining stanza.
The first line of the first stanza is the third line of the second, fourth, and sixth stanza of the villanelle.
The last line of the first stanza is the last line of the third, fifth, and sixth stanza.
And, the first line of each of the remaining stanzas must also rhyme with these lines.

As a result, you technically only need to write thirteen lines to fill a villanelle.
But, the repetition of the lines means that you need to think about how each line can serve multiple story elements.
The most famous example\footnote{that I'm aware of} of \href{https://poets.org/poem/do-not-go-gentle-good-night}{\say{Do Not Go Gentle into the Good Night}} by Dylan Thomas.

Writing them has been really illuminating for me, because it has really forced me to improve a skill that I previously had left fairly undeveloped: revising and editing a poem.
Villanelles almost require revision, because I at least will often realize that the rhyme I chose is too constrictive, or the line isn't evocative enough.
That being said, the cadence of a villanelle is incredibly moving to me, so I hope that I'll keep writing them.

Not counting this text:
545 words outside footnotes, 58 with footnotes.
\end{document}