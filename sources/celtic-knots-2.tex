\documentclass[12pt]{article}[titlepage]
\newcommand{\say}[1]{``#1''}
\newcommand{\nsay}[1]{`#1'}
\usepackage{endnotes}
\newcommand{\B}{\backslash{}}
\renewcommand{\,}{\textsuperscript{,}}
\usepackage{setspace}
\usepackage{tipa}
\usepackage{hyperref}
\begin{document}
\doublespacing
\section{\href{celtic-knots-2.html}{On Celtic Knots (Again)}}
First Published: 2022 April 12

\section{Draft 1}
As rapidly happens for me, I've run out of ideas to blog about.
I've been drawing a lot of celtic knots again lately, though, so I suppose that's something that I can talk about.

Something I've always wished and hoped for is the ability to freehand celtic knotwork.
As some of my readers might have seen on my Instagram page, every one I make starts with a grid and then a grid of circles representing the holes in the design.

I don't remember whether I started drawing them like this to practice freehand, or if the two just grew together, but I've started drawing celtic knots as single lines, rather than the two dimensional ribbons that I used to draw.
It's absolutely helped me with freehanding, though I'm still nowhere near as good as I would like to be.

As I tried reproducing a cross that I am particularly proud of today, I think I might have found a new solution, though.
I made all of the beams going a single direction first, and only then went through and added the crossing pieces.\footnote{if none of this makes sense, I apologize.
I procrastinated my writing a little too much this week, and I'm finally paying the price.}

It worked much better than I expected, though I was still fairly reliant on the grid of my paper.
Then again, practicing on a lightly gridded piece of paper is probably my best bet for learning how to quickly find the right sizing.
\end{document}