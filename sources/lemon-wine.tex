\documentclass[12pt]{article}[titlepage]
\newcommand{\say}[1]{``#1''}
\newcommand{\nsay}[1]{`#1'}
\usepackage{endnotes}
\newcommand{\1}{\={a}}
\newcommand{\2}{\={e}}
\newcommand{\3}{\={\i}}
\newcommand{\4}{\=o}
\newcommand{\5}{\=u}
\newcommand{\6}{\={A}}
\newcommand{\B}{\backslash{}}
\renewcommand{\,}{\textsuperscript{,}}
\usepackage{setspace}
\usepackage{tipa}
\usepackage{hyperref}
\begin{document}
\doublespacing
\section{\href{lemon-wine.html}{Lemon Wine}}
First Published: 2022 July 1


\section{Draft 1}
Yesterday I went through the past two months of my life, reflecting a little on what I did.
One thing I did which I'm very happy about is that I made my yearly recipe of lemon wine.

Lemon wine is a recipe I have adapted\footnote{for loose uses of the word} from \href{skeeterpee.com}{here}.
It's functionally just lemonade that you add yeast to, but I really enjoy it, and it seems like my peers did as well.

Briefly, I cooked boiled 8 pounds of sugar in water with a heavy splash of lemon juice to hopefully break the dimerized sugar into monomers.
From there, I combined it with two quarts of lemon juice with yeast nutrient and water to make 5ish gallons, and added EC-81 yeast.
I then measured its density and aerated it 1-2 times a day for the next four days, until the density readings showed it was approximately halfway through fermentation.
At that point I added another quart of lemon juice and sealed the container

I then left it alone for 10ish days, at which point it was finished fermenting.
I then transferred it off the lees\footnote{which I used to make soup, which was delicious} and let it sit for about a month.
At that point I backsweetened it and bottled it up.

It's always really fun to share with friends because homemade goods feel better to share than storebought, at least for me.
\end{document}