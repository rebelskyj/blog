\documentclass[12pt]{article}[titlepage]
\newcommand{\say}[1]{``#1''}
\newcommand{\nsay}[1]{`#1'}
\usepackage{endnotes}
\newcommand{\1}{\={a}}
\newcommand{\2}{\={e}}
\newcommand{\3}{\={\i}}
\newcommand{\4}{\=o}
\newcommand{\5}{\=u}
\newcommand{\6}{\={A}}
\newcommand{\B}{\backslash{}}
\renewcommand{\,}{\textsuperscript{,}}
\usepackage{setspace}
\usepackage{hyperref}
\begin{document}
\doublespacing

\section{\href{disclaimer.html}{Disclaimer}}
Pre-reading note: I'm writing this post\footnote{and hopefully future posts} as I tend to write my papers.
That is, a very rough sketch, a more polished draft, and so on until I have the quality I want\footnote{or have run out of time} in my writing.
Since I'm not too embarrassed of my (lack of) writing ability, I thought I'd post my drafts here as well\footnote{no, this isn't just a way to increase the word count}.
I'll leave the final draft on the top, with the lower drafts in order from newest to oldest underneath.

\section{Draft 4}
Those of you following this blog closely may note that this is the eighth blog post I've made since starting the blog.
Starting one week ago today, I began writing daily.
While I don't feel the need for a disclaimer at this exact moment\footnote{as nothing I've said seems too objectionable}, I think it wise that I craft one sooner than later

You may ask why I feel this need\footnote{I know this is unlikely, but it makes a convenient segue}.
To answer, I would reply that disclaimers are a good way to distance oneself from their work.
I have in the past\footnote{and will likely in the future continue to have} expressed views and beliefs I personally don't ascribe to because I believe that they are not being given their fair share of time.
Even though playing the Devil's Advocate hasn't come back to haunt me yet, I know that people tend to believe what you write is what you believe\footnote{rightly}, so I want to lessen the probability of misunderstandings.
Additionally, my views often change over time.
Since I'm not sure if I'll believe everything I write in the future, it seems smart to say that my thoughts may change now rather than later.

Now that I've explained, why I'm writing my disclaimer, I should probably write it.
I could copy my disclaimer from my inspiration's blog site: \say{The opinions stated herein are those of \textbf{me} and do not necessarily reflect those of \textbf{employer}, the Rebelsky family, \textbf{organizations}, or even most other sentient beings.}\footnote{taken from: \href{http://www.cs.grinnell.edu/~rebelsky/musings/}{SamR's Assorted Musings and Rants: Front Door}, Bolding mine}
However, much of that disclaimer isn't relevant to me, and it doesn't quite express the situation I find myself in.
So, this is my version: \say{The views and opinions expressed on this site (and linked sites) do not necessarily reflect those of the author's current, past, or future employers, the Rebelsky family, rebelsky.com, or even the author.
Readers are advised to read at their own discretion.}

\section{Draft 3}
People following this blog closely may note that this is the eighth blog post I've made here.
That is, starting one week ago today, I began writing daily.
By now, I think I may need a disclaimer.
You may ask why\footnote{I know this is unlikely, but it makes a convenient segue}.
To answer, I would reply that disclaimers are a good way to distance oneself from their work.
I have in the past\footnote{and will likely in the future continue to have} expressed views and beliefs I personally don't ascribe to because I believe that they are not being given their fair share of time.
Even though playing the Devil's Advocate hasn't come back to haunt me yet, I know that people tend to believe what you write is what you believe\footnote{rightly}, so I want to lessen the probability of misunderstandings.
Additionally, my views often change over time.
Since I'm not sure if I'll believe everything I write in the future, it seems smart to say that my thoughts may change now rather than later.

Now that I've explained, why I'm writing my disclaimer, I should probably write it.
I could copy my disclaimer from my inspiration's blog site: \say{The opinions stated herein are those of \textbf{me} and do not necessarily reflect those of \textbf{employer}, the Rebelsky family, \textbf{organizations}, or even most other sentient beings.}\footnote{taken from: \href{http://www.cs.grinnell.edu/~rebelsky/musings/}{SamR's Assorted Musings and Rants: Front Door}, Bolding mine}
However, much of that disclaimer isn't relevant to me, and it doesn't quite express the situation I find myself in.
So, this is my version: \say{The views and opinions expressed on this site (and linked sites) do not necessarily reflect those of the author's current, past, or future employers, the Rebelsky family, rebelsky.com, or even the author.
Readers are advised to read at their own discretion.}

\section{Draft 2}
I've been writing this blog for a week now, so I feel like I should probably write a disclaimer sooner rather than later.
You may ask why.
To that I would reply with the following:

Disclaimers are a good way to distance oneself from their work.
Although I work for no-one, it seems important to me that I distance myself from some of my words.
Those of you who know me may know that I say things I don't believe because I believe that they are supportable enough views that they deserve their time.
Even though that hasn't come back to bite me yet, I know that people tend to believe what you write is what you believe\footnote{rightly}, so I want to be aware.
Additionally, my views\footnote{like most people in the process of learning} change over time.
Since I'm not sure if I'll believe everything I write in the future, it seems smart to say that my thoughts may change now rather than later.

Now comes the writing of the disclaimer.
I could copy my disclaimer from my inspiration's blog site: \say{The opinions stated herein are those of \textbf{me} and do not necessarily reflect those of \textbf{employer}, the Rebelsky family, \textbf{organizations}, or even most other sentient beings.}\footnote{taken from: \href{http://www.cs.grinnell.edu/~rebelsky/musings/}{SamR's Assorted Musings and Rants: Front Door}, Bolding mine}
However, that feels intellectually dishonest.
So, with some slight revisions: \say{The views and opinions expressed on this site (and linked sites) do not necessarily reflect those of the Rebelsky family, the domain, rebelsky.com, or even the author.
Readers are advised to read at their own discretion.}

\section{Draft 1}
I've been writing this blog for a week now, so I feel like a disclaimer is probably a good thing to get out of the way.
I could copy from my inspiration's blog site: \say{The opinions stated herein are those of \textbf{me} and do not necessarily reflect those of \textbf{employer}, the Rebelsky family, \textbf{organizations}, or even most other sentient beings.}\footnote{taken from: \href{http://www.cs.grinnell.edu/~rebelsky/musings/}{SamR's Assorted Musings and Rants: Front Door}}
However, that feels intellectually dishonest.
Let's see, I could say: \say{The views and opinions expressed here do not necessarily reflect those of the Rebelsky family, GitHub, rebelsky.com, most sentient beings, or even the author.
Any and all offense taken is deeply regretted.}
That seems pretty good.

Why a disclaimer though?
I feel like all of the cool writers use disclaimers as a way to distance those they work for from their works.
Even though I work for no one right now, I still know that people tend to believe what you write is what you believe\footnote{rightly}.
Since I'm not sure if I'll believe everything I write in the future, it seems smart to get that idea out of the way quickly, as well as avoiding indemnifying my family.

\end{document}