\documentclass[12pt]{article}[titlepage]
\newcommand{\say}[1]{``#1''}
\newcommand{\nsay}[1]{`#1'}
\usepackage{endnotes}
\newcommand{\1}{\={a}}
\newcommand{\2}{\={e}}
\newcommand{\3}{\={\i}}
\newcommand{\4}{\=o}
\newcommand{\5}{\=u}
\newcommand{\6}{\={A}}
\newcommand{\B}{\backslash{}}
\renewcommand{\,}{\textsuperscript{,}}
\usepackage{setspace}
\usepackage{tipa}
\usepackage{hyperref}
\begin{document}
\doublespacing
\section{\href{talk-planning-1.html}{Talk Planning}}
First Published: 2022 August 8
\section{Draft 1}
Somehow I immediately stopped writing this blog again for a week, and I'm sorry for that.
Anyways, I have a talk in a little over two months that I'd really like to go well, so now feels like as good of a time as any to start planning it.

My general plan for the talk is to try connecting astronomy\footnote{because the talk is being hosted by my University's Astronomy Department's Outreach Team} and music/tuning theory generally.\footnote{because I like that topic and am always sad that others don't know about it.}
In theory, I should be able to tie the two together fairly easily.
In the early days of Western Music and Astronomy, the two concepts were seen as pretty much the same.
Or, if not the same, at least both very related.

So, my goal here is to start fleshing out the ways that I could construct the talk.
Currently, my plan is to go through a history of the way different philosophers thought of the two concepts, and then move from there into the way they might have experienced music.
From there, I'll just hard pivot to tuning theory.

The general chronology, as far as I can find on Wikipedia, goes:
\begin{itemize}
\item Pythagoras, who allegedly ties the two together. Of course, we have no surviving writings from him\footnote{i think}, so I'll look at Pliny the Elder's Natural History, which apparently recounts the story.
\item Plato, who apparently discusses how we are formed to experience astronomy and music in the same ways with different senses in his Republic. Ideally I'd find more about his views on either, but we'll see.
\item Aristotle, who apparently writes about it in his book \say{On the Heavens}, which sure feels like the sort of book I'd expect to see takes like this in.
\item From there we skip a while and move to Boethius. Boethius wrote \say{De Musica}, which again, feels like it will have good takes.
\item Kepler comes next. He apparently really wanted the universe to be musical. Even better, he's still respected in astronomy stuff today.
\item Kinda ends there, except for the whole orbital resonance thing, which I don't know if I want to get into.
\end{itemize}

So that's 6 different works/authors I need to read in order to create a cogent narrative.
Thankfully, I can kind of just do the whole musical section by memory, since I already know much of what I need there.\footnote{The issue with tuning is that 1:2 (octave) and 2:3 (fifth) never line up, so you can't have both in tune (that gets a major asterisk but)}
Anyways, I should really start reading them and thinking of an actual talk title.
\end{document}