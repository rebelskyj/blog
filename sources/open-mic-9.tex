\documentclass[12pt]{article}[titlepage]
\newcommand{\say}[1]{``#1''}
\newcommand{\nsay}[1]{`#1'}
\usepackage{endnotes}
\newcommand{\1}{\={a}}
\newcommand{\2}{\={e}}
\newcommand{\3}{\={\i}}
\newcommand{\4}{\=o}
\newcommand{\5}{\=u}
\newcommand{\6}{\={A}}
\newcommand{\B}{\backslash{}}
\renewcommand{\,}{\textsuperscript{,}}
\usepackage{setspace}
\usepackage{tipa}
\usepackage{hyperref}
\begin{document}
\doublespacing
\section{\href{open-mic-9.html}{Open Mic}}
First Published: 2023 December11

\section{Draft 2}
I meant to write a simple accounting of my first time playing at an open mic \href{open-mic-8.html}{in a few months.}
My muse, however, wanted to talk about the way that music feels.
There is something really special in instruments that are not fixed pitch.

I was chatting with one of the astronomers today about the fact that they had recently acquired a piano.
They were amazed at how fun it is to just play, especially when you can do so with friends or family.
I don't disagree, for all that I'm now realizing the very thing that makes piano so instantly rewarding is also what makes learning guitar so much deeper to me.

On a piano, there is a sense of removal from the note you are making.
You press a button which activates a hammer which hits a string.
There is no way to make the string slightly higher or lower with your strike.
Each note is tuned to an exact specification\footnote{or left to detune, as it may be}.
Knowing one chord shows you the hand position for most other chords, and the same is true for scales and notes.
It is easy to make any sound appear, which is part of why the ceiling feels so far away from what the average person does.
Not even the ceiling, honestly.

This could spiral off into a whole discussion about how the easier it is to make the correct sound on an instrument, the more that mastery is defined in the ability to do far more.
However, I don't want to talk about piano tonight.
Instead, let us look at guitar by contrast.

Guitar chords are generally broken into around two categories.\footnote{I'm ignoring power chords which tend to be subsets of barre chords, and I know that Jazz and other traditions might have their own way of breaking it down.
For the kind of music I make and the people I talk to, though, the categories are fairly valid.}
There are open chords, where some strings ring freely while others are fretted individually by specific fingers, and there are barre chords, where a simple chord shape is moved higher up the neck by virtue of pressing the index finger to effectively raise the pitch of the instrument by half tones.

There are around five open chord shapes that tend to be used: C, G, D, A and E.\footnote{wow look at that nice circle of fifths}
D, A, and E can all be major or minor, while C and G can only really be major.

When barring,\footnote{I don't think that it's barreing, and spell check seems to agree with me} by contrast, most of the time only E and A shapes are used.
I still find that most barre chords sound more or less the same, as they have the same quality of all strings being fretted.

The five-ish open chords, by contrast, each have their own unique sound profile.\footnote{this is, of course, ignoring the fact that I know of at least two variations for C and G major that have different sounds, for all that they do not change anything about the explicit character of the chord. As an example, a variation of the G chord has the B string fretted to the third fret, turning it into a D. This means that the only third in the chord is in the upper A string, which means that a treble focused strum turns into a power chord.}
At tonight's open mic, one of the songs I did was in E minor, and really spent most of its time there.
There's something really magical about the E minor chord, for all that I don't really use it in that way normally.

I've noticed that with many instruments, the lowest note that they can reasonably play is a little more powerful than the rest.
On cello, when playing the open C string, the entire instrument seems to vibrate.
On guitar, every string shares some resonance with the low E string.
If you pluck and silence it, the upper two strings will continue sounding, since they lie exactly on its harmonic series.

E minor is also the triad that requires the fewest alterations from the way that the guitar is tuned.
As someone who's started to explore DADGAD and other open tunings, there is something really inspiring about a droning note and open guitar strings.
With E minor, you get both the benefit of many open strings and the benefit of having the lowest note sounding.

Just by altering which strings you emphasize on strokes, the sound can go from dark and brooding to an almost angry beat, like something calling warriors to arms.
I hope that I remember this as I start writing music again, because I think that there's absolutely a place in my album for a driven beat song.

Daily Reflection:
\begin{itemize}
\item Hobbies:
\begin{itemize}
\item Did I embroider today? Somehow my needle ended up on top of my coat, which might have been a sign that I was supposed to do some embroidery today. I ignored that, and forgot.
\item Did I play guitar today? I even played in public! It was so fun, and I need to do it again.
\item Did I practice touch typing today? Oof, C remains my white whale. Given that I've finished this musing well before my candles have burned down, though, I think that I might spend some time working on it.
\end{itemize}
\item Reading
\begin{itemize}
\item Have I made progress on my Currently Reading Shelf? I've got a lot of driving tomorrow, so plenty of time to do the audiobooks then.
\item Did I read the book on craft? I might have time tomorrow.
\item Have I read the library books? Having too many goals is a bad idea for me.
\end{itemize}
\item Writing
\begin{itemize}
\item Did I write a sonnet? Will do now.
\item Did I revise a sonnet? Ended up cancelling the writer's session, because the other member didn't have time o write a sonnet, too busy working on a paper.
\item Did I blog? I like it.
\item Did I write ahead on Jeb? I caught up to where I needed to be. Tomorrow I get to write a fun chapter again, though, which is nice.
\item Letter to friends? See the goal of reading library books.
\item Paper? I realized that the approach I was using is fundamentally flawed when I extend the problem like I'm currently trying to do.
\end{itemize}
\item Wellness
\begin{itemize}
\item How well did I pray? Not good.
\item Did I clean my space? Infinitesimally.
\item Did I spend my time well? Eh, I think that I could have spent it better, but I certainly spent it better than I have been recently.
\item Did I stretch? Still no.
\item Did I exercise? Shoot!
\item Water? Completely spaced drinking water for much of today. Remembered to drink a fair amount before the open mic though.
\end{itemize}
\end{itemize}

\section{Draft 1}
Well, time always passes.
Somehow, it's been \href{open-mic-8.html}{just over 4 months} since the last time that I went to an open mic.
Some might be asking why, a mere ten days before the longest night of the year, I chose to play there again.

This past Saturday, I was at the bar for an unrelated reason, and the waiter mentioned that the open mics had moved to be much earlier.
One reason that I had stopped going was the fact that I generally try to get to bed somewhere around 9 most nights, and the open mic did not start before then.
As a result, I was forced to choose between playing in public and getting sleep.
Nowadays, though, the open mic has been moved to start closer to seven or eight, depending on the week.
That in mind, I showed up very early, had a few tacos, and then played a set.

I wish that I could say the entire set went flawlessly.
Or, failing that, I wish that I could say something went so terribly wrong that I can tell stories about it for ages to come.
Alas, neither happened.
My guitar wasn't getting picked up, for whatever reason\footnote{I think that the battery might have died or that I somehow put it in wrong, will troubleshoot soon}, and so I had a brief moment of panic.
Thankfully, the man who runs the open mics lent me his guitar, and I was off to go.

I started with an original.\footnote{I still haven't given most of my songs names, but I think Ashes will be what I call it today. I'm sure that I've given it other names, and until I record and publish it, it will continue to get more names as needed}
Ashes went really well, and I didn't miss the intro to the chorus like I tend to.
Still, applause was somewhat muted for it.

Since it's been a while since my last open mic\footnote{which a lot of people did comment on, which was kind of nice. It's nice to be recognized and remembered}, I went with a great standby and did Maid on the Shore.
I really started to get into my stride during that song, and I saw at least a few people tapping along to the song, which is always nice.
The emcee\footnote{I love that master of ceremonies got initialized to m c got extended to emcee} said that I had time for one more song, and I knew just what to do.
It's nearly Christmas, so I did one of my favorite Advent/ Christmas songs, The Angel Gabriel.\footnote{an apparently less well known one, which is strange.}
It went well, though the fact that I was playing it on a different guitar than normal meant that it did not go quite as well as normal.
Still, it was really fun, and I got even more applause for my final song.
A few people complimented me after my set, which is always nice.

I forgot how much more fun guitar is with fresh strings.
I just put them on yesterday, and it's amazing how much better the tone is, and how much better my fingers seem to move.
More than that, though, it is as though the guitar pulls more sound out of my voice.

In general, I have noticed that I'm starting to like my singing voice more lately.
I think that a lot of it is because of an off handed comment that my choir director made.
He said that in one of the songs we did, we should really lean into the baritone growl.
I had forgotten about that vocal register, and I do really think that it's where my tone is best, at least from the inside.
Given that the open mic plays the amp slightly back at the performer, though, I do think that it also sounds at least a little better from the outside.

I forgot how much I love the E minor chord on the guitar.
There's something really nice about being able to just drone on the low string, bass sound droning as your own voice melds into the tone.
The sound becomes almost rhythmic, starting to sound less like a pitch and more like a pitched percussion.
Then, just when the note feels like the only sound that your guitar could make, you change chords.

The A minor chord that follows suddenly feels electric.
The energy rises, and you barre\footnote{I hate that it's spelled like this} up to B minor.
As you reach the peak of the line, the rhythmic low E drone comes back, and you've returned to E minor.
All that to say, I really love the setting of The Angel Gabriel I found.
It was originally in the key of A minor, but I transposed it down a fifth to E minor.

I'm more and more understanding what a visiting composer once told me.
He said that every interval has a particular emotion attached to it.
While I don't know if I'll ever get there, or even get to the point of specific keys having feelings on fixed pitch instruments like harp or piano, I'm starting to get there on guitar.

E minor, as mentioned, feels resonant and droning.
The fact that only two notes are fretted, and both are fretted to octaves of notes that are already sounded probably helps a lot.

E major, by contrast, feels aggressive and intense.
It's the sound of beating drums, striving and preparing for a fight.

G major feels open.
It reminds me of the open prairie as a lone voice sings about the stark beauty of the land.

C major is hopefully happy.
It is the sound of a young protest singer, one who still believes in his country.

A minor is not melancholy, for all that the word almost fits.
It is not quite bittersweet.
It's the ache of absence, the pain that comes only because of a joy that has been taken.

D major uplifts.
It reminds the listener that there's something they need to be moving towards.\footnote{why yes, the music I play does tend to have a single sharp in its key signature, how did you know?}

All this to say, I think that the music I write needs to start focusing on E minor for the more intense portions, because there's an intensity to the low open E that I just cannot match with the other chords.
\end{document}