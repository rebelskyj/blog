\documentclass[12pt]{article}  
\newcommand{\say}[1]{``#1''}  
\newcommand{\nsay}[1]{`#1'}  
\usepackage{endnotes}  
\newcommand{\B}{\backslash{}}  
\renewcommand{\,}{\textsuperscript{,}}  
\usepackage{setspace}   
\usepackage{tipa}  
\usepackage{hyperref}  
\begin{document}  
\doublespacing  
\section{\href{paper.html}{On Paper}}  
First Published: 2025 June 10

\section{Draft 2: 10 June 2025}

I think about most things more than most people I know.  
I don't know why this is, but part of becoming the best version of myself\footnote{best here really just meaning like most internally consistent} is not denying realities, and instead actively acknowledging them.  
So, what is my relationship to paper?

I love paper.  
There's something so inherently wonderful about the feel of my pen gliding across the fibers.  
I love the way that different qualities and kinds of paper feel differently in my hands.  
I like the way that a bright white page seems so much dimmer when filled with ink.  
I love the tactile feeling of fully embodying my writing: I try to write from my shoulder, and so half of my body is in active focus as I stroke across the page.

I like paper for more than that, though.  
When I type, my thoughts come out in an ordered progression.  
Line breaks happen by the sentence, and double line breaks by the paragraph.  
Paragraphs, visible on the page, rarely last more than a few sentences.  
In analog, the page fills with my writing.

Even outside of writing, though, I love paper for what I can do with it.  
I'm sure that it's no surprise to know that I had an origami phase in high school.  
Even as I have always enjoyed the feeling of nice paper, I've also hated the idea of being limited by the quality of my equipment.  
And, perhaps, I've felt guilty \say{wasting} high quality supplies when lower quality supplies would suffice.  
So, throughout high school, I must have made hundreds if not thousands of boxes and flowers.

Much as I love paper, I do also find that I feel very differently about fresh and used paper.

I love having loose paper around me, because I like the freedom it embraces.  
It can be anything, and any number of things.  
When bound in a journal, the pages share a narrative thread, even if I do not intend for that to be the case.

\section{Draft 1: 10 June 2025}

I don't have a ton of energy or mental space today, so my goal is an easy win with an easy folly.  
I've been thinking a lot lately about the relationship I have with paper.

For as long as I can remember, at least through early high school, I've loved having loose paper with me.  
Then, I did much more origami and paper folding generally.\footnote{one of my HS teachers really recently showed me the box I made her more than a decade ago now, which she still uses.  
It's wild how much the actions we take echo into infinity.}  
For a while, I was also into doodling and drawing of different sorts.

I'm not entirely sure why I stopped doodling and drawing, and I don't really know how I feel about it.  
Even just a few weeks ago, I was doing penmanship exercises, and that was really fun.

Anyways, recently I was out with friends and excitedly showing off my then-current\footnote{arguably still current, even if I don't use it as much} plan for organizing my life: a number of report binders\footnote{three hole folders with transparent front covers}.  
Someone commented that it seemed incredibly disorganized and chaotic.  
Another friend commented that my default is otherwise to just put loose pages in my backpack.

I didn't think that this was true; I very rarely, in my mind, at least, just have random sheets of paper floating in my backpack.  
Looking for affirmation of this identity, I queried my group mates and friends over the next few days.  
To my utter shock, nearly all agreed with the statement that I was someone who kept loose paper in a backpack.  
I do, I realize, at least now.

It's inconvenient when I need to write something to hand someone or just like draw something and not have paper for it.  
Books, being the sacred objects they are\footnote{even and especially journals}, feel horrible to tear into.  
The benefit of the report binders is that I can take pages in and out.  
However, even that takes time, and sometimes I just want the sheet in front of me.  
Also, I want clean pages in the binders.  
I have a lot of things that I've used one side for something but not the other.  
I like being able to use all of a sheet of paper, and especially when all that I was attempting to do is read a page, it feels wrong to then throw it away.

So, loose paper is great.

What was the point of this folly?  
I think that's a great question.

\section{Daily Reflection}

\begin{enumerate}

\item Did you journal by hand, and do you feel like the stormy questions in your mind got on the page?

Kind of? I wrote two sentences because I did not really feel like writing this morning/that my mind is empty any more.

\item Did you do your best to sit in still silence?

Kind of? Yesterday I lay in bed for an hour after waking up. Other than that, though, almost not at all.

\item Are you making sure that each task is given your full attention, not just because the task deserves it, but because you deserve the luxury of doing a single thing at a time?

Generally! I did try assigning transitions while listening to an audiobook yesterday because data assignment is boring.  
I also tried multitasking while playing a game and listening to an audiobook. It did not go well, probably because the game required more reading than I was able to give.

\item Are you focusing on your posture and breath?

Eh, kind of!

\item What in your body is holding tension right now? How can you fix it? When will you fix it?

As evidenced by the fact that my legs always want to twist, it's my hips.  
Other than that, I think that my shoulders are.  
I have an interview today at 1000, so I might try stretching before/after it depending on how timing goes.

\item Comments on sleep?

I've been needing more the past few days, but I think that might be more due to below than anything real.

\item How's eating going? In particular, how are you doing with eating plants and unprocessed food?

Yesterday I had a handful of rice (then cooked), two sausages, a slice of cake, and four romaine hearts.  
The day before I had a slice of pizza, a breakfast sandwich, and a hot dog.  
I don't think that either of these is the correct amount of calories, and so going forward I do want to really focus on getting through the number of calories I know that I need.  
Might have to redownload a calorie tracker?

Also water, not doing anywhere near enough, which is also not a thing that I'm happy about.  
Then again, that's also part of life, I think.

\item Are you neglecting any of your familial obligations? If so, how can you rectify this?

Nope! I listened to last week's album, wrote a reflection about it, and chose this week's album.  
Other than that, I offered to come home to help with meal prep, but was rejected, and have brother call tonight.\

\item Cleaning: what is the biggest priority you have right now, and what is the next action item for it?

I bought far too many lemons, which means that my priority needs to be turning them into oleo saccharum and citrum.  
That's also the action item.  
I also just bought yeast, which means that I should pitch it sooner than later.  
There we go.

\item Thesis: current task. What's preventing you from finishing it? How will you remove that obstacle?

I need to finish the apparatus chapter, and I need to write the introduction.  
I also need to be working on the data analysis as it comes in.  
So far I haven't had data which converged to extant assignments, but which still produced something potentially meaningful??

Anyways obstacles are just apathy.  
I will not do the data analysis today because it will take as long as I give it, and I don't want to give it that much time when I also need to be writing.  
I said yesterday that I thought hand writing would be the right move.  
I still think that is true, so will do that.\footnote{going to quickly make a note to myself of my morning plan not on this document}

\item Thesis: next task. What will you need to be able to do it?

The task today is apparatus, which I hate. Task for next up is the remainder of the introduction, which I will also do.

\item What's the next job you're applying to?\footnote{note that this might be a \say{things we don't post} but}

Have interview! Cannot focus on two things at once.  
Next Tuesday, however, I also have an application due, and so will apply to that hopefully.

\item Are you intentionally trying to spend time with others?

Yeah! This weekend was filled with interactions and yesterday I worked in the office.

\item Are you doing your absolute best to ensure that you and those you interact with view the interactions in the same light? Are you sure?

No, and I feel kind of guilty about it.  
But, when people just make assumptions about me that I wish were true, it is really hard for me to correct them.

\item Are you keeping up on this daily set of reflection questions?

Didn't do it yesterday, am doing today!

\item Are you keeping up on writing the follies? If not, what's in the way?

As with the above.  
What is in the way? I think partially that I told myself I'd do it yesterday in the evening and knew even as I said that it was a lie.  
Outside of that, the general apathy and low energy I've had the past few days\footnote{hmmm I guess there is a benefit to daily journaling, which is that I can tell this is becoming a downward trend} means that it's hard to do anything, writing follies included.

\item How are the long form follies coming? Do you feel like they're weighing you down right now?

They aren't! I wrote one on Sunday, and that was fun.  
Outside of that, I think that I didn't say all that I wanted to say.

\item Are you writing poetry? When, and what were your takeaways from the previous day's writing?

I wrote some on Sunday.  
It mostly focused on belonging and sense of self, which is always lighthearted and never emotionally laden.

\item Are you making music? If not, what is in the way?

Sang prayer on Friday and Saturday, sang Mass and a concert Sunday.

\item Web novel?

Nope!

\end{enumerate}

\end{document}