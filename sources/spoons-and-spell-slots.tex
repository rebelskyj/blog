\documentclass[12pt]{article}[titlepage]
\newcommand{\say}[1]{``#1''}
\newcommand{\nsay}[1]{`#1'}
\usepackage{endnotes}
\newcommand{\1}{\={a}}
\newcommand{\2}{\={e}}
\newcommand{\3}{\={\i}}
\newcommand{\4}{\=o}
\newcommand{\5}{\=u}
\newcommand{\6}{\={A}}
\newcommand{\B}{\backslash{}}
\renewcommand{\,}{\textsuperscript{,}}
\usepackage{setspace}
\usepackage{tipa}
\usepackage{hyperref}
\begin{document}
\doublespacing
\section{\href{spoons-and-spell-slots.html}{On Spoons and Spell Slots}}
First Published: 2022 January 28

\section{Draft 1}
There's a concept in mental health that's been making the rounds in the internet.
The concept is that of \say{spoons}.
More or less, the idea is\footnote{as far as I can tell} that many activities in the life of someone with a neurological condition take mental effort, which can be quantified as spoons.
Some activities take one spoon, others two, etc.
If you have no more spoons, then you cannot do the activity.
It's a metaphor for executive dysfunction.

I personally prefer a modification of that which involves the tabletop roleplaying game\footnote{TTRPG} metaphor of spell slots.
This explanation is nicer for me because it fits better with my lived experience.
The concept is similar to above, but with levels.
As an example, sending an email is a fairly low-level spell slot in terms of executive function.
Calling someone about a misplaced order is a fairly high-level spell slot in terms of effort and functioning.
However, if I run out of the low-level slots, a high level slot can be used for the task.

Anyways, the two paragraphs of explanation are mostly to say that I've been running out of whichever metaphorical storage I have a lot lately.
I'm not totally sure why, but I have some ideas.
I'm really hopeful that I can get back into having the space and spoons to not run out constantly.
It's really sucky to me that yesterday I didn't even have the space in my mind to write this blog.

Writing this blog is something that I'm really becoming excited about as a retrospective.
Even if I don't think there's anything good about the words, the fact that I'm writing at least five hundred words a day here is really helping me to think about how I could write my exams for chemistry or\footnote{looking forward} my thesis when that happens.
Even without that, it fits with \href{creative-hobbies.html}{my goal of having more creative hobbies}.

All that to say, it really makes me sad that I was unable to write a post yesterday, even though this is functionally a write-only blog. 
I'm pretty sure no one reads this\footnote{which I say way too often}, but it's still something that I sometimes look back on, if only the titles each day as I write a new one.
I feel like it might be helpful for me to have a few concepts I can try to have more spoons in the upcoming days and weeks, so I'm going to brainstorm below.

I could try scheduling my day more.
I find that when I don't pay attention, I waste my hours away mindlessly scrolling Youtube.

In a similar vein, I could just find a time limiter and use it on Youtube, so that I can't just waste my hours on there.
Sometimes removing distractions is enough for me, though since sometimes it's really an absence of energy, dissociating without Youtube isn't really much different.

I could try sleeping more.
Generally I find that when I sleep I do better mentally.

I could try eating better.
I'm eating pretty well right now, but I wonder if having more snacks/food available might help, because I often get home and am completely out of energy.

Anyways, if I try any of these and they work particularly well or badly I'll be sure to report back.

Words: 546 15
\end{document}