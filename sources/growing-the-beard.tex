\documentclass[12pt]{article}[titlepage]
\newcommand{\say}[1]{``#1''}
\newcommand{\nsay}[1]{`#1'}
\usepackage{endnotes}
\newcommand{\1}{\={a}}
\newcommand{\2}{\={e}}
\newcommand{\3}{\={\i}}
\newcommand{\4}{\=o}
\newcommand{\5}{\=u}
\newcommand{\6}{\={A}}
\newcommand{\B}{\backslash{}}
\renewcommand{\,}{\textsuperscript{,}}
\usepackage{setspace}
\usepackage{tipa}
\usepackage{hyperref}
\begin{document}
\doublespacing
\section{\href{growing-the-beard.html}{Growing the Beard}}
\section{Draft 1}
In popular media, there's an expression:\footnote{well, many expressions. This is one of them} \say{growing the beard.}
It tends to mean that a show\footnote{or other medium} has begun to become more serious as it progresses.

While abroad, I've begun growing the beard myself.
Now, this isn't to say that I'm not still the joyful, carefree child that I was before I left.
I'm growing a literal beard.

The reasons for this are varied.
First, I enjoy having a beard, since it means that shaving takes far less time.\footnote{after shaving for the past 7 years (I think 7, maybe 8 by now)}
If I only shave my neck,\footnote{because fashion says not to have hair on your neck} I can shave and be done in less than a minute.
Even when also doing my cheeks,\footnote{because apparently that's another area} it still takes less than a few minutes.
When I have to shave my whole face, it takes a fair amount more time.

Partially, it's that I feel a much greater need to use shaving cream when doing my whole face.
My neck doesn't need it anymore,\footnote{although it does feel nice when I use it} so if I only need to trim my neck, I can save all the effort there.

Partially, it's that when I shave my face, I have much higher standards.
With my neck and cheeks, I'm satisfied with \say{less hair than before, looks vaguely like I care.}
But, when I'm shaving my whole face, I want it to be smooth.
The last time I counted, it took,\footnote{for each 1-3 inch section} 5 strokes with the grain, 5 strokes perpendicular to the grain, and 15 strokes against the grain, all on my very close cutting razor.
That's a lot of time, and it means that I inevitably cut myself at least once.

So, other than convenience, there's the fact that I like the way I look with a beard.
I continually get told I look far younger when clean shaven, so this way I don't.
Tying into that reason, it's also a way of mapping my time.

Every one of my friends\footnote{note: if you haven't and consider me a friend, I also have confirmation bias here, so don't worry, we're still friends} has at one point or another expressed to me that my beard has an ideal length.
That generally comes in the form of \say{your beard is the wrong length,} but.
So, while in London, I'm also taking a daily photo of my beard growth to document its progress.
I take a photo in a different place everyday, as a way to motivate me to remember where I am.

Next, and most practically, we've passed the summer solstice.
Days are getting shorter.
Nights, therefore, are getting longer.
In theory it's getting colder.\footnote{it was 20 degrees (F) cooler here than in Iowa when I left, so I just accept that temperature isn't real}
Having a beard means that I'll keep warm as the days get colder.

So, the reasons are: convenience, appearance, documentation, and warmth.
I think that's all the reasons I'm growing it out, but I also tend to make up reasons as asked, so it's possible that more have existed in the past, or will in the future.
\end{document}