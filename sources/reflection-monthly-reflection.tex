\documentclass[12pt]{article}[titlepage]
\newcommand{\say}[1]{``#1''}
\newcommand{\nsay}[1]{`#1'}
\usepackage{endnotes}
\newcommand{\B}{\backslash{}}
\renewcommand{\,}{\textsuperscript{,}}
\usepackage{setspace}
\usepackage{tipa}
\usepackage{hyperref}
\begin{document}
\doublespacing
\section{\href{reflection-monthly-reflection.html}{Reflection on Monthly Reflections}}
First Published: 2023 June 7

\section{Draft 1}
On thing that I've realized is less than ideal in my monthly reflections is that they are in many respects write-only memory.\footnote{that is, I write the data and never read it.}
That's sort of an issue, especially since I tend to have forward facing goals in each reflection.
One way I think I might be able to address the issues is by looking at each of my goals in each day's blog post.\footnote{this does also have the secondary benefit of increasing my wordcount, which is still far too low.}

My goals this month were broken into finite and infinite tasks, which may have been better framed as timed\footnote{read: done on a certain day} and untimed.\footnote{though looking at them, this does not actually work}
I don't think that reflecting on the finite tasks will be particularly helpful, so I'm only going to focus on the growth ones in my daily reflections.
I'm going to take the rest of this post to expound on the one line goals that I have.

I haven't invited anyone over to my home yet, and it's starting to become messy again.
I should really spend some time this afternoon getting my home cleaner.
In general, I would really love to have my default vision of my home as something that is clean, rather than one that is not.

I have been blogging fairly often, though I suppose I did miss a few days already.
Of the goals that I wanted to write about, I think that I've gone through all of them except for piping interactions and book reviews.
I'd like to blog about handwriting again, since that's coming up in my life.

Exercising three times a week has been not happening.
I've gone for one run since the month started.
I think making time to go to the gym after lunch might be a good idea today!\footnote{though maybe not.}
At the implicit\footnote{original wording was tacit, but I'm not totally sure what the connotations for tacit are} recommendation of a friend, I've also been considering doing personal training, if only because it would be nice to know that I'm approaching things in somewhat of the right direction for my goals.

Sleeping enough and getting up at six have remained fairly at odds.
The past few days, while I've generally woken up at six, I then immediately changed my alarm to seven thirty\footnote{seeing times written out really bothers me for some reason. I'm still going to do it, but it's good to notice.} and returned to sleep.

Making time for prayer has been a mixed bag.
On Sunday, I certainly did.
Yesterday, I remembered to pray a rosary.
In general, though, I do find that I waste my time on pointless distractions, and I would rather spend that time in prayer.\footnote{at least I think I would. The fact that I don't sort of implies that I am not.}

I'm currently\footnote{as of 957} about a quarter of a chapter ahead on my book, which is slightly behind where I would like to be if I'm trying to evenly get to five chapters ahead.
Getting one full chapter ahead by the end of the day would be fantastic.

Writing a song is going mixedly.
I realized that every song I write turns into some variation of a love song, or at least an ode to someone.
I tried to go through the list of songs that I like to see what common themes in them are.
In what should have been unsurprising, most of them are love-adjacent songs.
The rest of them generally have some sort of folk or union vibe, which I can't personally relate to.
Still, I'm working on a\footnote{nother} new song that I started last night, and I do have hopes of actually finishing it.

Writing letters to friends has not happened at all.
I'm planning to write one today.
It's interesting though, I never realize how bad my writer's block is until I try to write a letter.

That does bring me to an interesting piece of introspection, which is that I find that tasks that I don't want to do often feel nebulous until I make concrete what the issue is.
Here, I know that the issue is just writing the letter, so I'm going to take a break from writing this musing to write the letter.\footnote{one interesting piece of diaries in general (wow look at that callback to the genesis of this blog) is that there's no good way for the subjective time of the writer to be reflected to the reader.}
That took me all of about fifteen minutes, at least half of which was dealing with pencil lead.
Still, it's nice that I've written a letter to a friend!\footnote{and counting the words in the letter, I'm going to absolutely count the one hundred and fifty four words that I wrote towards my daily goal of three thousand, seven hundred, and forty (wow that's way too many)}

So, I think that I'll just replace the items within the list below this each day with quick updates on each task:

\begin{itemize}
\item Progress towards friends over\footnote{is the new goal because otherwise it's a finite task}
\item Blog-ress\footnote{Current ideas: book reviews, handwriting, song progress, piping interactions}
\item Exercise progress
\item Sleep and rise at six
\item Pray and make time for it
\item Backlog building
\item Songsmithing
\item Letters to Friends
\end{itemize}

Hopefully this will help me to actually accomplish my monthly goals!

(716/197)
\end{document}