\documentclass[12pt]{article}[titlepage]
\newcommand{\say}[1]{``#1''}
\newcommand{\nsay}[1]{`#1'}
\usepackage{endnotes}
\newcommand{\B}{\backslash{}}
\renewcommand{\,}{\textsuperscript{,}}
\usepackage{setspace}
\usepackage{tipa}
\usepackage{hyperref}
\begin{document}
\doublespacing
\section{\href{planning-2025.html}{Planning 2025}}

First Published: 

\section{Draft Two: 7 January 2025}

N.B. In the interest of having single day drafts, this is the first draft up until reflection of July 2023.  


I generally title posts like this a reflection, for a few reasons.  
Mostly, I want to use it as a way of seeing what goals I've had, kept, lost, and picked up.  
However, this musing is meant to serve a different role.

Up to this point, I've mostly been treating what's changed in my life as though it's no more a change than any other sudden shift, like studying abroad or graduating.  
I don't know if even now I've come to terms with all of the ways that my life is forever fundamentally altered for having lost my mother.  
Goals that relied, in any way, shape, or form, on her direct motivation\footnote{or even indirect} need new reasons.  
So, even as I look back at my past to see what my goals once were, my focus today is almost entirely on the future.  
I am willing to discard any goal I once had that no longer serves me.\footnote{this is an aspirational statement, not necessarily indicative of any real fact.}

Time to go through every reflection I've had.\footnote{at least those which I was smart enough to title \say{reflection-}}

Back in \href{reflection-january-19}{January of 2019}, I had the goal of writing a sonnet every day.  
I do think that forcing myself to do something creative, in both senses of the word, and analog before bed time is ultimately good for me.  
Much as I think this, however, I don't love sonnet form.  
I'm tempted to try common verse, because I do really enjoy alternating tetrameter and triameter.  
However, unlike a sonnet, common verse does not have a set number of stanzas or lines.  
Worth considering, at least.

\href{reflection-january-22}{Three years later}\footnote{and horrifyingly, almost three years ago}, I wanted to blog daily\footnote{which was true in the prior reflection, and is likely going to be true in every future one, so I'm just going to stop mentioning it}.  
I was working through the Bible in a Year that year, and had the goal of keeping up on it.  
I also wanted to work on working out and writing music.

Apparently \href{reflection-february-22}{that February} was the only February I reflected on.  
Interesting that I lose focus in the springtime.  
More or less, the same as January, though with working on my book\footnote{now my web serial} added in.

\href{reflection-march-2022}{March} has also only been reflected on once, and I misdid the naming of the url.  
As before, with practicing guitar added in.

In \href{reflection-april-2022}{April}, I wanted to finish a draft of a paper I needed to write.

\href{reflection-june-2022}{June}\footnote{nope I did not miss a month, no clue what you're talking about} brought the idea of listing the exciting things that had happened to me in the past month.  
That's probably something that's worth bringing back?  
It also added the goals of accordion practice and writing short stories.

\href{reflection-july-2022}{In July} I wanted to be able to run a little over five miles continuously, stretch, work on an exam, and revise my book.\footnote{assume that goals are generally transferred between each month unless otherwise noted}

Poems, journaling,\footnote{ah, right, the spell checker I use doesn't like journal to be a gerund} and curricular development were my \href{reflection-august-2022}{August goals} of note.

\href{reflection-september-2022}{September} had me hoping to work on my 24 for 24 list.\footnote{Which, realistically, is probably a good thing for me to bring back this year}

I wanted to do NaNoWriMo and fill out 24 for 24 in \href{reflection-october-24}{October}.

NOTE: At this point in the musing, I realized that I've been looking at the goals moving forward, so generally transpose all the goals forward one time unit.

I had the insane goal in \href{reflection-november-24}{November or December} of that year of writing another 50K word book.

On the start of \href{reflection-2022}{2023}, I wanted to add daily rosaries to my list, along with working through the Catechism in a Year.

\href{reflection-january-23}{February} brought no new goals.

In \href{reflection-april-23}{May} of that year, I split my goals into professional, personal, and growth.  
In general, they were single things that I needed to accomplish, and I should not forget the value of static goals.  
I also wanted to get back into bagpiping, which is also a thing I miss.  
Wildly, I hadn't written any public outreach talks at that point.  
I know that it's something I really started in graduate school, but it's so weird to me that I only really started to do it about 20 months ago.

\href{reflection-may-23}{June} had me counting my monthly words, and split goals into finite\footnote{static from above} and growth.  
I wanted to get up at 6am daily, get ahead on my book, and write letters to friends.  
I also wanted to invite friends over to my home, because I was happy with its state.\footnote{what a wild concept, honestly}

That same month, I also reflected \href{reflection-monthly-reflection}{on the reflections themselves.}  
In that post, I introduced my concept of actually tracking my goals.\footnote{For those reading, yes, it took me more than five years from starting a blog to realize that I could (and realistically, should) track the goals I have}

In \href{reflection-june-23}{July}, I wanted to be able to have friends over, which necessitated a clean home.  
That's really it for changes.

\href{reflection-july-23}{August} brought the goal of finishing a talk and nothing else.

\href{reflection-august-23}{September} had me reflecting forwards\footnote{projecting??} on what excited me about the coming month, which seems like a good goal.  
I also started work on my album, which I have yet to finish.

\href{reflection-september-23}{In October} I gave small snippets of explanation for why each goal was set.\footnote{Ok so that's something that's been extant before, but}

\href{reflection-october-23}{November} had me commenting that the month had passed me by for reasons I should have remembered. I don't know if I recall them right now. It's distinctly possible that October of 2023 was when I found out about my mother's illness.  
I was apparently doing well enough at prayer then that I wanted to do better than a rushed rosary.

\href{reflection-2023}{At the dawn of last year}, I kept up my tradition of five things that I was excited for in the coming year. Let's see how I did:

\begin{itemize}  
\item Publishing my first first author paper -> Did not do, likely because I lost most of the summer and fall  
\item Giving an invited lecture at a university -> I did this, but also like that was kind of already set up.  
\item I'm excited to grow in my relationships -> I would like to think that I've done so, though it certainly fell apart as the year progressed.  
\item I'm once again also excited to watch my little brother graduate -> It was fantastic, though bittersweet for it just being my brothers and me there.  
\item I'm excited to record an album -> I did not finish any of the songs, though I did share a decent recording of one of them with my mom.  
\end{itemize}

I may as well also list five good things that happened to me last year that I did not list:

\begin{itemize}  
\item Getting a new underclassman on my project. The new student has been fantastic and really revitalized me  
\item Learning to embroider  
\item Figured out where I want to end up in a career  
\item Got my family into a dumb marble racing show.  
\item Ta'd introductory astronomy  
\end{itemize}

I did also want to swim a mile, learn a polka, read through all my books. I did none of those, though I was generally ok at finishing the rest of my goals.

Nothing really new \href{reflection-2024a}{in July}, though that came with my acknowledgment that I more or less had given up on blogging.

At the \href{reflection-2024b}{end of last year}, I made a list of more or less every goal I had at the time.  
I then spent some time sorting them into different categories, like finite and infinite, goals for the near and far future, etc.

So, what have I learned from going through all my reflections?

In general, I always want to blog more, stretch more, and exercise more.  
Most of the time I want to pray more.  
I have also gone through different phases for how I reflect.

Going forwards, I think that I would like to continue monthly reflections, and I would like to do a few forward looking and backwards looking experiences.  
Looking through the goals from my last reflection, there are some changes and some similarities from there.  
So, what does this mean for me?

In 2025, I am excited to:

\begin{itemize}  
\item Have a timeline for finishing my Ph.D.\footnote{initially: Finish my album. Realizing that I've been nominally working on this for 18 months really just goes to show me how much I need to finish it.}  
\item Once again, continue to grow in my relationships.\footnote{initially: Restart my web serial and find a way to make it a healthy and growth-inspiring habit, but I realize that it was a goal, not an experience}  
\item Write a song for another friend's wedding.\item{initially: Mature in my faith. It's been a rough past few months, but I think that it will ultimately end up as a net benefit to me.} It's been great to be able to help my friends experience the best day of their lives\footnote{or so I hear that's what the wedding is} and enhance it with the meager\footnote{meager just feels like such a British word that I cannot help but spell it meagre} talents for music I possess.  
\item Feed myself better. I've gotten better and better about it, and these past few months in particular have been really good to me in this regard, if nothing else.  
\item Find a way to live with my grief. So far it's been alternating repression and collapse, though it's definitely been more of a damped wave than an amplified or static one.  
\end{itemize}  


Goals for this year\footnote{taken liberally from my 2024B reflection}:

\begin{itemize}  
\item Make significant progress on writing my thesis. Regardless of the timeline, it's always better to have written than not to have written.  
\item Come to terms with my grief. As mentioned above, I feel like it's a necessary thing to do.  
\item Find a sustainable way to feed myself. As mentioned above, I'm making progress on it, but it's certainly not fully there.  
\item Find a way to write my web novel sustainably. I miss writing it, and I really miss interacting with my friends and family about the new chapters, to say nothing of the joy I get seeing random internet strangers praise me.  
\item Stop wasting time. That's not to say stop taking breaks, but there are many things that I spend time on that I don't value, and that don't help me grow into the person I want to be. It's a journey for sure, but I'd at least like to feel like more of my days are being spent in service of growth.  
\item Get back into letter writing. This ties in really well with growing in my relationships, but also I feel like I have not been doing enough handwriting these days, and certainly not enough reflecting.  
\item Get back into journalling. As with blogging, it's a use of time that I'm always grateful for.  
\item Learn to draw. I had that going for a little bit last year, but then the holidays came and destroyed my schedule.  
\item Finish the album.  
\item Get back into classical\footnote{I don't really know how to describe it, maybe ensemble?} composition.  
\item Find a way to bring fitness into a routine. I've generally been good about at least doing a bit of stretching in the morning, but I know that I could improve on that.  
\item Blog daily. This goal is absolutely my white whale, but it's still my goal  
\item Grow interpersonally.  
\item Become more comfortable with myself.  
\item Art. I don't know in what ways I want to be doing art, but I know that I want non-musical art to be a part of my life.\footnote{why non-musical? see the other musical goals and also below.}  
\item Music. In general, I want to start building proficiency in my main instruments. Well, I do want to gain proficiency in every instrument, but that's not plausible, if only because I don't have every instrument. These days, that's singing, guitar, and accordion.  
\item Clean my life.  
\item Learn to draw.  
\item Do 26 for 26.\footnote{I know that I'm starting very late, but there's every chance that I've done novel things since my last birthday that I have noted somewhere. If not, well, that might be a sign that this summer will have to be filled with growth.}  
\end{itemize}

Looking more short term, what are my goals for the rest of the month?

\begin{itemize}  
\item Music:  
\begin{itemize}  
\item I don't really think that this month is calling me to ensemble or classical composition.  
\item Work on the album for at least an hour each Saturday, and find another half hour sometime in the week.  
\item Practice guitar and accordion. I'd like to say twice daily, but at least once for both.  
\end{itemize}  
\item Art:  
\begin{itemize}  
\item Continue working on the embroidery project. It brings me peace to do it.  
\item Work on art at least 3x a week. I have the drawing space all set up, now it's just a matter of doing it.  
\end{itemize}  
\item Exercise:  
\begin{itemize}  
\item Continue/improve at stretching in the morning and evening.  
\item Start going to the group fitness classes.  
\item I'm going to put food in here, so do my blog post or at least a personal posting about my macro needs  
\end{itemize}  
\item Writing:  
\begin{itemize}  
\item Write and send at least a letter a week. Probably do that on Saturdays.  
\item Journal every morning, at least five minutes.  
\item Write some poetry every night.  
\item Start writing my web novel again.  
\item Blog daily.  
\end{itemize}  
\item Grow in prayer  
\end{itemize}

That's a fair number of goals.  
How can we make this a daily and weekly check?

Daily:  
\begin{itemize}  
\item Practiced guitar?  
\item Practiced accordion?  
\item Twice daily stretching?  
\item Journal?  
\item Poetry?  
\item Blog?  
\item Net cleaner home?  
\end{itemize}  
Weekly:  
\begin{itemize}  
\item Embroidered?  
\item Drew on Monday, Saturday, Sunday?  
\item Group Fitness?  
\item Letter?  
\item Web Novel?  
\end{itemize}

Ok, that doesn't actually seem so bad. Saturdays seem like they'll start to be filled with activities which lead me to growth, which is always really nice. In general I do tend to find that I spend too much time wasting away on Saturdays, since I tend not to have anything in particular scheduled to do.  

\section{Draft One: 6 January 2025}  
I generally title posts like this a reflection, for a few reasons.  
Mostly, I want to use it as a way of seeing what goals I've had, kept, lost, and picked up.  
However, this musing is meant to serve a different role.

Up to this point, I've mostly been treating what's changed in my life as though it's no more a change than any other sudden shift, like studying abroad or graduating.  
I don't know if even now I've come to terms with all of the ways that my life is forever fundamentally altered for having lost my mother.  
Goals that relied, in any way, shape, or form, on her direct motivation\footnote{or even indirect} need new reasons.  
So, even as I look back at my past to see what my goals once were, my focus today is almost entirely on the future.  
I am willing to discard any goal I once had that no longer serves me.\footnote{this is an aspirational statement, not necessarily indicative of any real fact.}

Time to go through every reflection I've had.\footnote{at least those which I was smart enough to title \say{reflection-}}

Back in \href{reflection-january-19}{January of 2019}, I had the goal of writing a sonnet every day.  
I do think that forcing myself to do something creative, in both senses of the word, and analog before bed time is ultimately good for me.  
Much as I think this, however, I don't love sonnet form.  
I'm tempted to try common verse, because I do really enjoy alternating tetrameter and triameter.  
However, unlike a sonnet, common verse does not have a set number of stanzas or lines.  
Worth considering, at least.

\href{reflection-january-22}{Three years later}\footnote{and horrifyingly, almost three years ago}, I wanted to blog daily\footnote{which was true in the prior reflection, and is likely going to be true in every future one, so I'm just going to stop mentioning it}.  
I was working through the Bible in a Year that year, and had the goal of keeping up on it.  
I also wanted to work on working out and writing music.

Apparently \href{reflection-february-22}{that February} was the only February I reflected on.  
Interesting that I lose focus in the springtime.  
More or less, the same as January, though with working on my book\footnote{now my web serial} added in.

\href{reflection-march-2022}{March} has also only been reflected on once, and I misdid the naming of the url.  
As before, with practicing guitar added in.

In \href{reflection-april-2022}{April}, I wanted to finish a draft of a paper I needed to write.

\href{reflection-june-2022}{June}\footnote{nope I did not miss a month, no clue what you're talking about} brought the idea of listing the exciting things that had happened to me in the past month.  
That's probably something that's worth bringing back?  
It also added the goals of accordion practice and writing short stories.

\href{reflection-july-2022}{In July} I wanted to be able to run a little over five miles continuously, stretch, work on an exam, and revise my book.\footnote{assume that goals are generally transferred between each month unless otherwise noted}

Poems, journaling,\footnote{ah, right, the spell checker I use doesn't like journal to be a gerund} and curricular development were my \href{reflection-august-2022}{August goals} of note.

\href{reflection-september-2022}{September} had me hoping to work on my 24 for 24 list.\footnote{Which, realistically, is probably a good thing for me to bring back this year}

I wanted to do NaNoWriMo and fill out 24 for 24 in \href{reflection-october-24}{October}.

NOTE: At this point in the musing, I realized that I've been looking at the goals moving forward, so generally transpose all the goals forward one time unit.

I had the insane goal in \href{reflection-november-24}{November or December} of that year of writing another 50K word book.

On the start of \href{reflection-2022}{2023}, I wanted to add daily rosaries to my list, along with working through the Catechism in a Year.

\href{reflection-january-23}{February} brought no new goals.

In \href{reflection-april-23}{May} of that year, I split my goals into professional, personal, and growth.  
In general, they were single things that I needed to accomplish, and I should not forget the value of static goals.  
I also wanted to get back into bagpiping, which is also a thing I miss.  
Wildly, I hadn't written any public outreach talks at that point.  
I know that it's something I really started in graduate school, but it's so weird to me that I only really started to do it about 20 months ago.

\href{reflection-may-23}{June} had me counting my monthly words, and split goals into finite\footnote{static from above} and growth.  
I wanted to get up at 6am daily, get ahead on my book, and write letters to friends.  
I also wanted to invite friends over to my home, because I was happy with its state.\footnote{what a wild concept, honestly}

That same month, I also reflected \href{reflection-monthly-reflection}{on the reflections themselves.}  
In that post, I introduced my concept of actually tracking my goals.\footnote{For those reading, yes, it took me more than five years from starting a blog to realize that I could (and realistically, should) track the goals I have}

In \href{reflection-june-23}{July}, I wanted to be able to have friends over, which necessitated a clean home.  
That's really it for changes.

\href{reflection-july-23}{August} brought the goal of finishing a talk and nothing else.  


\end{document}