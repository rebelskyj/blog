\documentclass[12pt]{article}[titlepage]
\newcommand{\say}[1]{``#1''}
\newcommand{\nsay}[1]{`#1'}
\usepackage{endnotes}
\newcommand{\1}{\={a}}
\newcommand{\2}{\={e}}
\newcommand{\3}{\={\i}}
\newcommand{\4}{\=o}
\newcommand{\5}{\=u}
\newcommand{\6}{\={A}}
\newcommand{\B}{\backslash{}}
\renewcommand{\,}{\textsuperscript{,}}
\usepackage{setspace}
\usepackage{tipa}
\usepackage{hyperref}
\begin{document}
\doublespacing
\section{\href{dance-nation.html}{Dance Nation Review}}
Today I had the wonderful opportunity to see Clare Barron�s \textit{Dance Nation} at the Almeida Theatre.
The show follows a U13 dance squad as it prepares for national qualifying tournaments.
What follows is a review.
\section{Draft 3}
The show speaks about the sexualization and sexuality of young women.
As I share no part of that experience, my review will focus on other aspects of the show.

As many of you know, I was a techie\footnote{person in theatre who isn't an actor} in high school.
Perhaps because of that, or perhaps because something about the medium of live theatre really allows the technical aspects to shine forth, I tend to focus a lot more on the technical theatre aspect than the acting.
My reviews\footnote{because I plan on doing this for every show they take us to} will also likely focus on these.

When we are first allowed on stage, we see a curtain in the murky realm between translucence and transparancy.
We can barely make out a mirrored wall behind us, and lights running up the seams.
When the show opens, the curtain drops.
I was struck by the paneling of the stage.
Mirrors made up the entire back wall, with lights strung between each.
Throughout the show, the use of the lights as a way to demarcate different areas of the stage and different characters was masterful.
However, the panels do not remain mirrors.
The first piece of the show that struck me was the paneling of the stage.

Throughout the show, the panels turn and twist about, becoming anything from doors, to black glitter decoration, to a wolf's head, to a bathroom stall.
The use of the panels as entries and exits, as a way to bridge and break scenes, and, at the end, as a way to recognize the disjointed nature of both the show and life is fantastic.

The lighting of the show is also incredibly helpful for setting and moving the scenes along, as well as making the minimal set feel larger.
When Luke is on the car ride home with his mother, the moon in the upper right of the stage tells us of the long day he's had, and connects us to the other dancers, who each have their own memories of the night.
As the drive continues, orange lights flow by their faces, reminding the audience of their movement.
When Zuzu has a breakdown, the lights become harsher and harsher until her realization comes through.

Continuing to mess with the set, the show did a wonderful job of playing with set as prop and prop as set.
Actors tended to carry in the set pieces, or pieces that they worked with remained a part of the show.
Always in the wings of the stage were the detritus of a dance studio, reminding that it's never far away, especially in the minds of the dancers.
As the show progresses, the stage becomes more and more cluttered, just as the depth of the characters' experiences do.

A final note about the set, as means to transition into the acting, all along what is meant to be the ceiling of the room are trophies from dance competitions past.
In my experience as an athlete, leaving mementos of victory where they are always visible is meant one of two ways.
Either the coach wants to say \say{you're a part of a program with past success, and with hard work you can be a part of that too,} or \say{winning is all that matters. The teams that aren't up here don't belong in our memories because they were a failure as a team and as people.}\footnote{no that isn't something I've been directly told, just the overall message I received when discussing past parts of the program}
This show chose to take the second approach, and took it to the comical extreme that some coaches take it.
Not only does the coach point out the one year with no trophy by saying no one remembers them, he then mentions the next year, where a dancer was recruited to become a professional.
One of the panels turns and shows us a shrine dedicated to the dancer and all of the dancers chant her name.
This is the first of many times throughout the show that the coach and dancers talk about how winning\footnote{becoming noticed and recruited} is the only important aspect of dance.
To quote a common sports saying, \say{winning isn't everything, it's the only thing.}
What the show quietly points out, however, is that there is a constant culling of winners.
No one the rest of the star's team.
Slowly but surely, in memory and in action, the weak are culled.
Only the strong remain.

In the first scene, a dancer gets injured and is never heard from again.
Near the end of the show, another dancer makes a mistake and quits as a result. 
These actions, along with monologues, showcase a dangerous aspect of not only children's athletics, but also of society.
Weakness and failure are not opportunities to grow, but rather signs of being inherently inability.
And, since the weak get culled, any mistake means you no longer belong.

All in all though, it was a very enjoyable show.
\section{Draft 2}
The show speaks about the sexualization and sexuality of young women.
As I never went through that, my review will focus on other aspects of the show.

As many of you know, I was a techie\footnote{person in theatre who isn't an actor} in high school.
Perhaps because of that, or perhaps because something about the medium of live theatre really allows the technical aspects to shine forth, I tend to focus a lot more on the technical theatre aspect than the acting.
My reviews\footnote{because I plan on doing this for every show they take us to} will also likely focus on these.

When we are first allowed on stage, we see a curtain in the murky realm between translucence and transparancy.
We can barely make out a mirrored wall behind us, and lights running up the seams.
When the show opens with the curtain dropping, I was struck by the panels of the stage.
There was a string of lights between each mirror.
Throughout the show, the use of the lights as a way to demarcate different areas of the stage and different characters was masterful.
However, the panels do not remain mirrors.
The first piece of the show that struck me was the paneling of the stage.

Throughout the show, they turn and twist about, becoming anything from doors, to black glitter decoration, to a wolf's head, to a bathroom stall.
The use of the panels as entries and exits, as a way to bridge and break scenes, and, at the end, as a way to recognize the disjointed nature of both the show and life is fantastic.

The lighting of the show is also incredibly helpful for setting and moving the scenes along.
When Luke is on the car ride home with his mother, the moon in the upper right of the stage tells us of the long day he's had.
As the drive continues, orange lights flow by their faces, showing the audience how they're moving through time.
When Zuzu has a breakdown, the lights become harsher and harsher until her realization comes through.

The show did a wonderful job of playing with set as prop and prop as set.
Actors tended to carry in the set pieces, or pieces that they worked with remained a part of the show.
Always in the wings of the stage were the detritus of a dance studio, reminding that it's never far away, especially in the minds of the dancers.
As the show progresses, the stage becomes more and more cluttered.

A final note about the set, as means to transition into the acting, all along what is meant to be the ceiling of the room, there are trophies from dance competitions.
As an athlete, leaving trophies where they are always visible has in my experience been meant in one of two ways.
It either is left to say:\say{we have a history of hard work and success. Keep that alive,} or \say{these trophies represent the years we remember and care about. The years without trophies are failures, and the athletes are too.}
This show chose to take the second approach, and took it to an almost comical extreme.
Not only does the coach point out the one year with no trophy by saying no one remembers them, he then mentions the next year, where a dancer was recruited to become a professional.
One of the panels turns and shows us a shrine dedicated to the dancer.
This is the first of many times throughout the show that the coach and dancers talk about how winning\footnote{becoming noticed and recruited} is the only important thing.
The show quietly points out that there is a constant culling by no one remembering the rest of the team.
The show systematically culls the weak from its ranks.
Only the strong remain.

In the first scene, a dancer gets injured and is never heard from again.
Near the end of the show, a dancer makes a mistake and quits as a result. 
These actions, along with monologues, showcase a dangerous aspect of children's athletics, but also of society.
Weakness and failure are not growth experiences, but rather signs of being inherently weak.
And, since the weak get culled, any mistake means you no longer belong.

All in all though, it was a very enjoyable show.
\section{Draft 1}
The first piece of the show that struck me was the paneling of the stage.
When the curtain drops to begin the show, the back wall is composed of mirror panels.
Throughout the show, the panels are turned and become anything from doors, to black glitter decoration, to a wolf's head, to a bathroom stall.
Throughout the show, the use of the panels as entries and exits, as a way to bridge and break scenes, and, at the end, as a way to recognize the disjointed nature of the show and life is fantastic.

The lighting of the show is also incredibly helpful for setting and moving the scenes along.
When Luke is on the car ride home with his mother, the moon in the upper right of the stage tells us of the long day he's had.
As the drive continues, orange lights flow by their faces, showing the audience how they're moving through time.

In terms of the show itself, the show is set in a dance academy featuring the parts of youth athletics I was so incredibly fortunate to miss out on.
These preteens are being told that their future success in life depends wholly on how well they perform at a single show.
The coach\footnote{and to some extent the other dancers} reinforce the idea that it really doesn't matter whether you're having fun.\footnote{in my opinion the entire reason to do childhood activities}
All that matters is being the best and winning.
Weakness isn't tolerated, and the weak are culled out.

From the first scene, where a dancer gets injured and is never heard from again, through to the end, where a dancer makes a mistake and, as a result, quits dance forever, the idea of mistakes as being reflective of character flaws, and that character flaws (and by extension, those who hold them) should be hidden.

All along what is meant to be the ceiling of the room, there are trophies from dance competitions.
As an athlete, leaving trophies where they are always visible can be meant in one of two ways.
It can either be: \say{we have a history of success, and y'all are going to be a part of it,} or \say{we only care about you if you win.}
As you might guess, this show chose the second approach, and took it to an almost comical extreme.
Not only does the coach point out the one year with no trophy by saying no one remembers them, he then mentions the next year, where a dancer was recruited to become a professional.
He fully tells both the dancers and the crowd that the only important thing about dance is being the best.
Even the rest of the famous dancer's group is forgotten.
Only the strong remain.

Returning to painting the scene, the show did a wonderful job of playing with set as prop and prop as set.
Actors tended to carry in the set pieces, or pieces that they worked with remained a part of the show.
Always in the wings of the stage were the detritus of a dance studio, reminding that it's never far away.

All in all, it was a very enjoyable show.
\end{document}