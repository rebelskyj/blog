\documentclass[12pt]{article}[titlepage]
\newcommand{\say}[1]{``#1''}
\newcommand{\nsay}[1]{`#1'}
\usepackage{endnotes}
\newcommand{\1}{\={a}}
\newcommand{\2}{\={e}}
\newcommand{\3}{\={\i}}
\newcommand{\4}{\=o}
\newcommand{\5}{\=u}
\newcommand{\6}{\={A}}
\newcommand{\B}{\backslash{}}
\renewcommand{\,}{\textsuperscript{,}}
\usepackage{setspace}
\usepackage{tipa}
\usepackage{hyperref}
\begin{document}
\doublespacing
\section{\href{reflections-on-readings-32-ordinary-b.html}{Reflections on Today's Gospel}}
First Published: 2018 November 11

Mark 12:44-44 \say{
A poor widow also came and put in two small coins worth a few cents.
Calling his disciples to himself, he said to them, \nsay{Amen, I say to you, this poor widow put in more than all the other contributors to the treasury.
For they have all contributed from their surplus wealth, but she, from her poverty, has contributed all she had, her whole livelihood.}}
\section{Draft 1}
Today's Gospel gives the message that I've always had the hardest part reconciling with the teachings we see in daily religious life.
The Lord says that the widow's gift, though small, means the most, as she has the least to give.

As a Catholic, I am taught that faith is a call to action.
The sentiment that \say{faith without action is dead} is not an uncommon one.
So, why is it that we see good deeds done by the faithless as anything but the miracles that they are?

When a faithful person performs a good deed, it shouldn't be remarkable.
They are simply doing what their soul tells them to do.
When they make the choice to not do good, it takes an effort.
But, someone who doesn't believe doesn't have that call.
For those without faith, if we believe that faith is what calls you to action, there is no call to do good.

And yet, the faithless still do good.
To me, that's the most beautiful part of life
\end{document}