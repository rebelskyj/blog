\documentclass[12pt]{article}[titlepage]
\newcommand{\say}[1]{``#1''}
\newcommand{\nsay}[1]{`#1'}
\usepackage{endnotes}
\newcommand{\1}{\={a}}
\newcommand{\2}{\={e}}
\newcommand{\3}{\={\i}}
\newcommand{\4}{\=o}
\newcommand{\5}{\=u}
\newcommand{\6}{\={A}}
\newcommand{\B}{\backslash{}}
\renewcommand{\,}{\textsuperscript{,}}
\usepackage{setspace}
\usepackage{tipa}
\usepackage{hyperref}
\begin{document}
\doublespacing
\section{\href{talk-planning-3.html}{Talk Planning, Introduction}}
First Published: 2022 September 6
\section{Draft 1}
Pre-reading note: It's been nearly a month since my last post about this.
There's about a month until the talk.
I need to do better at prep.
Also, very rambly.
Be forewarned

On to the actual post:

My talk is titled\footnote{tentatively} \say{Understanding the \textit{Musica Universalis}, the Harmony of the Spheres}.
What is the \textit{Musica Universalis}?
As a quick anglicisation, we end up with the Universal Music, or the Music of the Universe.

This concept is also referred to as the harmony of the spheres.
The spheres\footnote{I need to fact check this section} refer to the Greek and general ancient belief of the multiple levels of the universe.

There's the most obvious sphere, earth.
Philosophers have known for millenia that the earth is round.

Above that is the sky, and above the sky is the rest of space.
Most translations will say the heavens, and I see no reason not to.
Anyways, before I get too distracted on this tangent, time to move on to what I'd planned to talk\footnote{write} about.

In a modern context, especially one outside of academic music, harmony is generally understood as a vertical concept.
That is, harmony refers to the immediate sounds we hear.
(If I have access to a piano like I think I will, I'll start to demonstrate).
As an example, think of a chord.
For those of you unfamiliar, this\footnote{sound is produced} is a C Major chord.
Without getting into what that means, I hope that all of you can hear\footnote{see?} that there is more than one note in that sound.\footnote{sound in the chord?}

There is another way to think of harmony, though, which is to think of the harmony as the notes being used over time.
That is, just like this (block CM) is a chord, so is this (arpeggiate).
Now that we understand what harmony is in music, how does that relate to the planets and universe?

As some of you probably already know, sounds as we hear them come from waves in the air.
In general, most of the musical sounds we hear come from stacked integer multiples of a single base frequency, known as the harmonic series.
The harmonic series shows up all over the place in modern physics, especially in quantum systems, where positions and energies are quantized to integer multiples of some value.
This is also where the connection between music as we think of it today and the music of the universe comes in.

Ancient and pre-modern philosophers saw the regular orbits and rotations of heavenly bodies and thought that they must be moving to some primal music, far slower than our ears can hear.
It's not hard to see why.
Take the most fundamental cycle we can observe: the sun's rising and falling.
Every day we see the sun make a full lap around the earth.
Every six months we have exactly 12 hours of sunshine.
Every year the cycle of equinoxes and solstices are repeated.

Next we could look at the moon.
The moon goes from dark to light every 28 days.

As we continue to add celestial bodies, it's not hard to see how this harmony keeps filling in more and more.\footnote{I should really see what they refer to in specific}

That's about as much time as I think I can justify as an introduction before I get into the actual meat of the talk, going through historical views and such.
Then again, maybe I should tie that in like this and spend more time by just introducing the development of views.

Ideas include:
\begin{itemize}
\item Setting sin generators at different frequencies of orbits to show how planets make a nice chord
\item Above but with other phenomena, esp. as early thinkers had it.
\item Make sure to hit on how when we say 2:3 in music we can mean a wide range, not exact to mathematical meanings.
\item Maybe have a drifting fifth for above
\item Making sure to talk about how the concept of the universe as a beautiful thing is inherent to early thought.
\end{itemize}

\end{document}