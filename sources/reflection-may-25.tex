\documentclass[12pt]{article}  
\newcommand{\say}[1]{``#1''}  
\newcommand{\nsay}[1]{`#1'}  
\usepackage{endnotes}  
\newcommand{\B}{\backslash{}}  
\renewcommand{\,}{\textsuperscript{,}}  
\usepackage{setspace}  
\usepackage{tipa}  
\usepackage{hyperref}  
\begin{document}  
\doublespacing  
\section{\href{reflection-may-25.html}{Monthly Reflection}}  
First Published: 2025 June 1

\section{Draft 1}  
As I said \href{reflection-april-25}{last month}, wow what a whirlwind.

Last month was great, and I'm really excited for this coming month as well.  
Let's see what we were excited for!

\begin{itemize}  
\item Today I TA for the last time, I think ever.

It was a really weird feeling, I'm not going to lie. It didn't really sink in for me for at least another few days.

\item I have a one week intensive writing camp for my dissertation, which will be fun.

Interestingly, I had a footnote that said I also wanted to figure out what the difference between a thesis and a dissertation was.  
I failed on that regard.  
\href{dissertation-camp}{I really enjoyed} my time at camp!

\item Giving the opening talk for the set of summer science outreach and communication program that I'm part of. That is, the university partners with state parks to give talks about space over the summer (I know I've talked about this before). I'm giving the first of the year (even though it isn't in a state park, technically).

The talk at the preserve ended up being cancelled, but luckily, I was still able to do the first two talks of the season. It was really fun!

\item Being able to dedicate entire days to a single project, rather than having to always split my attention.

As it turns out, I still have two tasks, and at any given day I feel myself pulled in infinite directions.

\item I'm going on a pilgrimage this weekend! That should be really fun!

It was really fun! I don't think that I mused about that at all, so that's a potential future choice!

\end{itemize}

Five quantized\goals from last month:  
\begin{itemize}  
\item Finish solid drafts of at least three chapters of my thesis.

Oof! I did not get anything close to this. I think that if I add the various sections together I end up with something resembling three chapters, though.  
Very few citations, however.

\item Write at least one in depth exploration of some theological topic I've been wrestling with.

I don't think that I did this. I was thinking about it again today, so might make more of an effort to get some words on the page tomorrow.

\item Do more than 50 greater than 5 minute stretching periods. Right now I have a computer alarm set for the last five minutes of every work hour reminding me to do so, so that should help.

I did 32 in total, which is actually far more than I thought I would have. I did also make a point of doing more stretching, even when I did not explicitly log it.

\item Write and post at least 20 follies, including this one.

Five is so close to 20. This month I can hopefully do better!  
\item Apply to five jobs.

I think that I ended up with four full applications last month.

\end{itemize}

Five amorphous ideals from last month:  
\begin{itemize}  
\item Prayer and mindfulness. Figure out what difference, if any, there is between them to me and work to do both more.

I don't really know that I made any progress towards figuring out the difference between them, and regardless, I did not really do either.

\item I've started tracking my things to do and whatnot with sticky notes. Either continue with that, or figure out a better way to track my time and, importantly, track myself and keep myself accountable and productive. I guess this is kind of two, since productivity and organization are not inherently linked, but they sure feel like they are to me.

I ended up going for other things, but I still don't know if I've found an ideal system. I'll see if my three block schedule works out tomorrow, even if I'm not too optimistic.

\item Self care. Make sure that I am eating, sleeping, moving, drinking water, and the other things which I need to be the healthiest version of myself. This relates a lot to the top goal, but I guess here I'm saying physical self care rather than spiritual and mental.

I generally did better last month than this month!

\item Reading. I want to get through more of the backlog. Yesterday I went to a bar and read the last few chapters of the science book I'd been working through. In general set aside more blocks of time to simply disconnect, sit, and read a book. This will help inspire me, give the creative part of myself a rest, and help me to be better at knowing what all is known and thought.

I didn't really get through most of the books, but I did absolutely make progress in the sense that I now feel comfortable giving up on a book if it doesn't have the material that I want.

\item Romance. Make efforts towards finding a life partner and generally try to be more open to more forms of love, not just the friendships I have and deeply treasure.

I'm going to say that I did as well with this goal as I could have

\end{itemize}

I like to think of five things about the previous month that I hadn't known would happen that were also positive:

\begin{itemize}

\item Watched the first US pope be elected. That was such a strange situation

\item Wrote multiple almost entirely distinct resumes for submission to different jobs.

\item Had two fantastic choir concerts

\item Tried out a new sleep schedule that's so far treating me really well

\item Reconnected with an old friend!\footnote{was initially listed as \say{Listened to some new albums and did a semi-critical listen of them} but I think that this new one is better}

\end{itemize}

Alright, let's look forward to this coming month in light of the previous.  
In general, I did not do as well with my goals as I wanted.  
The month ended up feeling much rougher than I had expected, which I think is fair, but hopefully will not be true again next month.  
Something I adapted from the dissertation camp and gave as advice to someone else is to make three levels of to-do lists: the bare minimum needed, the amount you think you can do, and the amount you want to do.  
Before I get through those, let's think about five things that I am looking forward to this month:

\begin{itemize}

\item Seeing a close family friend's wedding

\item Visiting a friend\footnote{a different one than the one getting married, though this friend is also getting married soon}

\item Presenting my research at a conference

\item Singing in a few more concert-like events.

\item It happened today, but going to a friend's\footnote{since I'm friends with both parents, not entirely sure how to possessive that word} baby shower! It was so nice to celebrate with them.

\end{itemize}

Ok so I think that it's important for me to be realistic with what I can do, and also realistic with what I want to do.  
The daily reflections from last month are generally decent, but I want to make sure that they still serve me.

At a bare minimum, this month I absolutely need to:

\begin{itemize}

\item Finish the presentation that I'm giving on Friday and then again at the end of the month

\item Continue applying to next steps

\item Make progress on my thesis

\end{itemize}

That's only three things, and most of them are vague enough to be incredibly doable.  
The presentation might end up not being as good as I would like, and that's the nature of life sometimes.

What do I think that I can reasonably do this month?

\begin{itemize}

\item Make an actual list of everything that needs to happen before I can submit the paper

\item Do all of those things

\item Finish the presentation and make it good

\item Finish the chapter of my thesis on the program

\item Rewrite the chapter of my thesis on the apparatus I used\footnote{I do so hate when people mis-pluralize the word}

\item Keep up on my web novel

\item Keep up on this site

\item Do music

\item Take care of myself.

\end{itemize}

ope. Survive needs to go in the top category, but in the interest of respecting the spirit\footnote{oof I sound like a politician} of the site, I'm putting it here instead.

And, what would I do in the ideal month of June?  
Or, at least, what is the most that I think is what I can do if all goes well

\begin{itemize}

\item Finish the Paper

\item Write a full outline\footnote{as in, to the sub-paragraph level} of the rest of my thesis.\footnote{which is then functionally the same thing as just straight up writing the thesis, because then it's just adding words to the page}

\item Writing a folly a day

\item Getting a backlog of chapters ready for the web novel

\item Finish the presentations that I'm going to be giving in July.\footnote{ok I guess also write them}

\item Get the video series done

\item Record the new song I wrote

\item Write a few more songs

\item Record a few more songs

\item Jam with other people

\item Find my next steps

\item Once again become comfortable with silence

\item Spend the time I need to confront the faith issue that's been boiling in the background for a few months now.

\item Learn shorthand and start writing it with my left hand

\end{itemize}

My priorities last month were:

\begin{itemize}

\item Sleep, food, family, movement, and spirituality.

\item Thesis work and love

\item Things adjacent to those and applying to jobs

\item Cleaning, other obligations

\item Keeping up on hobbies

\item Fixing/improving my penmanship

\end{itemize}

Does that still feel good?  
I think so, but I think that the questions are going to need to change at least a little.  
I'm going to try to spitball some, and we'll see what all shakes out as ok.\footnote{I'm debating whether or not to save al the ones I use or to edit as I go}

\begin{enumerate}

\item Family, Health, Cleaning\footnote{added as I started to iterate}

\item Thesis, Future Career, Love

\item This site, music

\item Other

\end{enumerate}

That seems reasonable enough.  
I think that the family question remains simply as am I keeping up with the obligations.  
What are better health questions though?  
Let's iterate:

\begin{itemize}

\item Eating appropriately?

I don't know if I need to get through the three meals a day, and I think that part of the goal for this also needs to include what I'm eating.  
I've gotten back into the habit of consuming food, so now I can maybe get back into eating well.  
I'd like to try at least vaguely tracking to make sure that I'm hitting a few important bits (protein, vegetable)

\item Sleeping appropriately?

Since I think that the biphasic schedule is working for me, it will be interesting to see if my current hypothesis\footnote{I'm actually spending too long in bed and need to schedule less time for sleep} bears out.  
Other than that, I think that moving the bed and spending more time lying on the ground has been good for me.

\item Attempting to sit in silence?

Honestly, I don't quite know what to do with this one.  
I think that it's really a few things masking as a single issue.  
The first is that I worry that I can't slow down or stop.  
I can barely let myself do a single task at a time\footnote{e.g. I was listening to an audiobook at 3.5x speed and actively practicing penmanship at the same time}, let alone intentionally do nothing.  
When I do just sit without other things, though, I find my mind going all over the place.  
As I know, when I don't physically journal, I suffer for that, so morning journal might need to be a must do each day.  
Otherwise, I think that just setting a timer for five minutes to start, where I will simply try to meditate, might be the best thing for me.  
Once I can make it through five, we can try ten and so on.

\item Cleaning:

This past weekend I went home and helped clean the house. I know that with marginal amounts of effort, I can make significant progress cleaning. I just need to be more comfortable with throwing things away and actively deciding what things belong where.  
This month I'd like to get more cleaning done

\end{itemize}

Let's see how this feels:

\begin{enumerate}

\item Must Do:

\begin{itemize}

\item Attempt to sit still for five minutes

\item Hand write in the journal until the thoughts slow down

\item Clean the home, at least for 10 minutes, or as long as you procrastinate

\item How's the sleep?

\item How's protein?

\item How's minimally processed eating going?

\item How's not multitasking going?\footnote{oof I think that I might actually have to just not let myself do two things at once. The only exception I'll allow is music while doing other tasks, and even that I'm not entirely sure of} The fact that this actively made me anxious while typing is probably a good sign that I need to be doing it.

\item Stretching? In particular: shoulders, hips, neck?

\item Familial Obligations?

\end{itemize}

\end{enumerate}

Great, on to the second set of priorities.

Thesis:

I think that what I should have here is whatever the next task I have is.  
Also, obviously, the current task.  
If I do not make the appropriate progress on those, then I can at least track that.  
It does mean that I might need to spend some time tomorrow morning figuring out exactly what I want to get done when.\footnote{though the current answer is getting the data I collected analyzed (needed to make graphs), and outlining/writing my apparatus chapter. I think that I really might need to actively force myself to only write the things in the form of outline, because otherwise I just freewrite and it ends up being a bunch of drafts that have different content, rather than the same content but more polished.}
At this point, I know what I need to put in each section, and just need to actually do that.

Career:

At the bare minimum, I think that I can submit a job application a day, assuming that I don't need to actively redo my entire resume.  
Making a new version of the resume can count for that, though.

Love:

This one is hard, because by its nature I cannot do it alone.  
However, there are absolutely things that I can do.  
At the bare minimum, I would like to be seeing people more often than I am, and regardless of if that's in platonic or romantic settings, it seems like a good goal.  
Also, though, clarifying intentions is always good, and I would like to do that more as well.  
I should also do more social activities, because I forgot how isolated I feel until I went to dissertation camp and had a group for the week.

\begin{itemize}

\item Current Thesis Task?

\item Next Thesis Task?

\item Job app?

\item Each week, pick one social activity and go to it\footnote{hmmm other than the weeks that I'm out of town, I guess...} ope no, as the footnote suggests, that's not workable.  
Do not spend more than four days in a row without having a meet up or social event with others, excluding when with family.

\item Clarify intentions

\end{itemize}

This site is relatively easy to consider: I just have to ask myself two questions:

Am I writing it?

Am I working through the big questions I have for myself?  
Hmm this might actually be part of the first bit: am I taking care of the big mental loads in my head?  
Eh, if it comes up in journalling then it comes up in the journal.

Music can mean a lot of things, but I think that primarily I want to focus on:

\begin{itemize}

\item Finish the new song

\item Practice guitar so that the wedding piece feels natural

\item Get back into writing four part string settings, because they're nice work background music and good practice

\item Listen to the book on being a better talker and use the info in that to make sure that I'm using my voice appropriately.\footnote{is it strange that I'm listening to a book where a lot of the advice comes from talking to music directors, but is focused on professional environs, to work on my music? probably.}

\end{itemize}

With this being said, though, I don't really want to force myself to write the music daily.  
Writing poetry daily might be a good thing to do, though. I think that I can fill a page with poetry every single day, and so from here on out, let's try to do that.  
Poetry has a way of connecting to journaling, I find at least.\footnote{yes, I did intentionally spell the gerund form of journal differently both times I used it}

With this in mind, the new reflection:

\begin{enumerate}

\item Did you journal by hand, and do you feel like the stormy questions in your mind got on the page?

\item Did you do your best to sit in still silence?

\item Are you making sure that each task is given your full attention, not just because the task deserves it, but because you deserve the luxury of doing a single thing at a time?

\item Are you focusing on your posture and breath?

\item What in your body is holding tension right now? How can you fix it? When will you fix it?

\item Comments on sleep?

\item How's eating going? In particular, how are you doing with eating plants and unprocessed food?

\item Are you neglecting any of your familial obligations? If so, how can you rectify this?

\item Cleaning: what is the biggest priority you have right now, and what is the next action item for it?

\item Thesis: current task. What's preventing you from finishing it? How will you remove that obstacle?

\item Thesis: next task. What will you need to be able to do it?

\item What's the next job you're applying to?\footnote{note that this might be a \say{things we don't post} but}

\item Are you intentionally trying to spend time with others?

\item Are you doing your absolute best to ensure that you and those you interact with view the interactions in the same light? Are you sure?

\item Are you keeping up on this daily set of reflection questions?

\item Are you keeping up on writing the follies? If not, what's in the way?

\item How are the long form follies coming? Do you feel like they're weighing you down right now?

\item Are you writing poetry? When, and what were your takeaways from the previous day's writing?

\item Are you making music? If not, what is in the way?

\end{enumerate}

These are slightly more freeform than last month's and my thought is both that I want to spend more time intentionally reflecting on each every day, because I want to center myself, and because I think that looking at the broader scale is more important.  
This month's reflection and planning came out a little different than in months and years past, but I don't know if it's really for the worse.  
Let's do the reflection, then head to finish the night.

\begin{enumerate}

\item Did you journal by hand, and do you feel like the stormy questions in your mind got on the page?

I did a little bit of journaling in the margins of the music I was writing.  
It was almost entirely focused on what I wanted to do today.  
I got through nearly everything on the list.

\item Did you do your best to sit in still silence?

Nope! I've been reading through most of the night, though I have at least been generally limiting myself to one form of distraction at a time.

\item Are you making sure that each task is given your full attention, not just because the task deserves it, but because you deserve the luxury of doing a single thing at a time?

Not at all.  
I intentionally sought out videos for cleaning, and was reading during stretches.

\item Are you focusing on your posture and breath?

Yes. I like the feeling of sitting with my shoulders held straight. It really does so much to make me feel better.

\item What in your body is holding tension right now? How can you fix it? When will you fix it?

I think that it's my shoulders right now. I'm going to do slow shoulder rolls, and do so as soon as I finish this reflection.

\item Comments on sleep?

Not so much. It's been going well so I'm going to try to pull off fifteen minutes on the back end.

\item How's eating going? In particular, how are you doing with eating plants and unprocessed food?

Not great. I had a romaine heart and an apple today, but that's basically it in terms of not highly processed food.  
I want to pack a lunch for tomorrow, I think?

\item Are you neglecting any of your familial obligations? If so, how can you rectify this?

I didn't listen to the album as much as I had wanted to last week.  
This week I will make more of a point to do so.  
Since it was explicitly to be coupled with movement, I will do so.

\item Cleaning: what is the biggest priority you have right now, and what is the next action item for it?

My biggest priority right now is clearing the walkway. Next action item is going to be grabbing a trash bag and throwing out anything that I do not want.  
Everything else will be tossed further back in the home.

\item Thesis: current task. What's preventing you from finishing it? How will you remove that obstacle?

I just submitted a section, so the task I have next is running the many jobs. I don't want to put them on the cluster because it seems to keep breaking, and so will do them on our lab computer instead. Tomorrow morning I will go into the office and start the jobs.

\item Thesis: next task. What will you need to be able to do it?

I need to finish the presentation for the conference\footnote{and also Friday of this coming week}. I mostly just need to figure out what the framing is, and that's basically all doing it.

\item What's the next job you're applying to?\footnote{note that this might be a \say{things we don't post} but}

I've got a bunch of jobs saved in LinkedIn, more than a few of which are through the federal job portal, so I should make my account there.

\item Are you intentionally trying to spend time with others?

Somewhat! I saw friends at the shower today.

\item Are you doing your absolute best to ensure that you and those you interact with view the interactions in the same light? Are you sure?

Right now yes.

\item Are you keeping up on this daily set of reflection questions?

Look at this!

\item Are you keeping up on writing the follies? If not, what's in the way?

Wow, writing the folly today.

\item How are the long form follies coming? Do you feel like they're weighing you down right now?

I haven't started it. I really feel like about half the mental storm I face right now comes down to the faith folly I'm going to write.

\item Are you writing poetry? When, and what were your takeaways from the previous day's writing?

Nope! Ok so after this post finishes I'm going to head over to the writing corner and write a page of poetry.

\item Are you making music? If not, what is in the way?

I sang in choir today.  
I'm otherwise too tired to play guitar right now, which is unfair because I still have at least 20 minutes of wakefulness scheduled.

\item Web Novel?

Multiple people asked me about it this past weekend, so that's really a sign that I need to get back on the train.  
As much as I enjoy the short stories set in the world, I would also really like to get back into the main content.

\end{enumerate}

And with that, I'm officially ready to face the coming month.  
The only way out is through, the only way through is forward.

\section{Daily Reflection}

\begin{enumerate}

\item Top Priorities:

\begin{itemize}

\item Sleep:

\begin{itemize}

\item Keeping sleep time sacred?

Generally! I'm trying a siesta style sleep schedule, and it works well, but unfortunately I cannot always disappear for 2 hours starting at 2pm.

\item Good sleep hygiene?

Yeah! I am actually really only going to my bed to sleep which is a pretty new experience for me, honestly.

\item Sleeping enough?

I think so? I generally feel well rested and wake up a few minutes before my alarms and make it through the day!

\item How well rested do I feel?

Generally well rested, and as mentioned above, I do not struggle to get out of bed in the morning or afternoon.  
Also, I've been drinking less caffeine, so a part of me is legitimately\footnote{and pleasantly} surprised that I remain awake.  
Then again, I tend to find that caffeine makes me calm and alert, which ends up meaning focused.

\end{itemize}

\item Feed myself:

\begin{itemize}

\item Did I eat breakfast?

It's been a few days, but I think that most days I have been.

\item Did I eat a second meal?

I think that I've been averaging a breakfast and a dinner most days. Yesterday I for sure had lunch, and today I ate a second meal at like noon.  
That may or may not end up being dinner, depending on how I feel.

\item Did I eat dinner?

Generally!

\item Water?

Nowhere near enough. I'm actually going to go ahead and fill the bottle right now because I was actively thirsty much of today.  
I'm also going to take that time to take a few minutes and clean.

\end{itemize}

\item Family:

\begin{itemize}

\item Am I neglecting any familial obligations?

Nope! I went home because it was requested, helped clean, and am ready for the brother call.   
Ope wait need to finish the weekly album shoot.

\end{itemize}

\item Movement:

\begin{itemize}

\item Am I stretching at least 5 minutes per hour of computer time?

Nope. Yesterday I did no stretching and thus far today the same is true.

\item Am I generally making efforts to be limber?

I notice when I don't stretch now, so very much so a thing I want to do more, and try to make time for.

\end{itemize}

\item Spirituality:

\begin{itemize}

\item Time for prayer?

Not so much, especially because silence remains hard.

\item Prayer?

Not a lot, but I'm trying more.

\item Time for sacred silence?

Not so much, see the time for prayer.

\item Deep breaths?

Yeah! I started listening to a new audiobook which is about exercises to better connect body and speech.  
One of the things he mentioned off hand was what it means to take a full breath, and I hadn't realized how few I've been taking lately.

\end{itemize}

\end{itemize}

\item Secondary Priorities:

\begin{itemize}

\item Thesis/ Ph.D. work:

\begin{itemize}

\item Keeping up on the writing deadlines?

I think so! I was hoping to finish revisions on the paper by this weekend, but the program broke for some reason.   
If I were 10 percent more motivated, I would go into the office tonight and start the jobs running.  
However, I am not that motivated right now, and I don't think that delaying 12 hours will markedly affect anything.

\item Reading the necessary things?

Nothing remains necessary.

\item Making graphs?

\begin{itemize}

\item Visual depiction of Latin Hypercube

\item Visual depiction of Loomis-Wood Diagrams

\item Visual depiction of Spectral Stacking

\item Visual depiction of how the fitness of the spectral stacks is really reliant on the graphs being the right height

\item Plots from the actual results of the runs, to make sure that it worked out.\footnote{SSC, AAT, if any vib states were good, what happened to the computations, etc}.

Turns out most jobs crashed.  
So, it's now time to rerun the jobs, which is kind of frustrating.

\end{itemize}

\begin{itemize}

\item Visual depiction of Grid Search

\item Visual depiction of random search

\item I guess that the stuff for intro to quantum video counts here.

\end{itemize}

\item Organizing citations?

Nope, but haven't really been in a situation where it's feeling needful.

\end{itemize}

\item Love:

\begin{itemize}

\item Taking risks?

Kind of? The absolute least risky ones possible, but that's still not nothing.

\item Making efforts?

Yeah! I have actively been making efforts, and I'm proud of myself for that.

\item Showing affection?

Generally!

\item Being honest?

Mostly! Luckily I'm not being grilled about anything.

\item Being open?

See above.

\item Being appropriately vulnerable?

I think so! I do still default to not vulnerable enough, and I still don't really know if some people know my mom died, which makes it hard when I haven't seen them in a few months.\footnote{this sentence was getting too long for my tastes so stopped it at that clause because it's when it started to feel that way. It was an awkward place to stop.}  
They ask how I've been, and sometimes allude to not having seen me since July.  
I kind of just go \say{at the grind!}

\end{itemize}

\end{itemize}

\item Adjacent to Primary and Secondary:

\begin{itemize}

\item Typing Practice?

Not at all. I did just get a new keyboard, so I am excited to try that out tomorrow.

\item Applying to jobs?

I officially got rejected from the first one this weekend.  
I will hopefully finish the applications I saved this past week, though.

\item Reading the things I think could be good?

I started the new audiobook, so I'm excited to see how that goes!

\item Making manim videos?

Nope!

\end{itemize}

\item Cleaning?

\begin{itemize}

\item Office

Haven't been there in a bit.

\item Home

I went to home home and helped there!

\item Car

Not really, but hasn't needed a ton.

\item Computer

Nope! Actively downloaded new games, in fact.

\item Other as needed

\end{itemize}

\item External Obligations:

\begin{itemize}

\item Guitar for wedding?

Oop.

\item Travel plans?

Just planned some travel for July 4!

\item Other requested talks?

I still have more than a month!

\item Talks for conferences?

That is the main goal for this week!

\end{itemize}

\item Tertiary Goals:\footnote{mmmm off by N numbering}

\begin{itemize}

\item Blogging?

Oop.

\item Reading?

Yeah! I think, at least. Oh hm, I guess maybe not. What did I do yesterday and today? Wow time like sand is constantly falling through my fingers.

\item Web Noveling?

I posted a chapter on Friday, which is the usual goal.

\item Guitar?

Nope, at least this weekend.

\item Other hobbies?

Listened to some albums at friend recommendations this weekend! They were fun and very different than what I normally listen to these days.

\end{itemize}

\item Quaternary Goals:

\begin{itemize}

\item Letter writing

eh.

\item Handwriting/penmanship

I decided i might want to start holding the pen like artists do? If only because ink sometimes gets on the pen top and I'm tired of having inky hands.

\item Picking a new signature

Nope

\end{itemize}

\end{enumerate}

\end{document}