\documentclass[12pt]{article}[titlepage]
\newcommand{\say}[1]{``#1''}
\newcommand{\nsay}[1]{`#1'}
\usepackage{endnotes}
\newcommand{\B}{\backslash{}}
\renewcommand{\,}{\textsuperscript{,}}
\newcommand{\foott}[2]{\endnotemark[#1]\endnotetext[#1]{#2}}
\usepackage{setspace}
\usepackage{tipa}
\usepackage{hyperref}
\usepackage{nested}
\begin{document}
\doublespacing
\section{\href{science-mysticism-revisited.html}{A Conclusion to Science as Mysticism}}
First Published: 2024 January 6

\section{Draft 7: 6 January 2024}
\endnoteversion[d]

The line between passion and obsession is the point where pleasure turns to pain.
Amateur means unpracticed and unskilled, not because love means that you cannot improve, but because love is fundamentally healthy.
To become virtuosic requires moving well beyond the point of love, growing obsessed with the very smallest minutiae in a subject.
A mystic is more like a great musician or a groundbreaking scientist than any of their followers are like them.

Obsession and passion are also the difference between knowledge and truth.\foott{1}{wow I'm actually really loving this so far}
Knowledge is simple.
Knowledge is facts and figures.
When we read and respond to the writings of the great thinkers, or when we play the notation that some great musician laid down, we are confronted with knowledge.
When we grasp in the darkness, gathering and creating, we work with knowledge.

Knowledge is comfortable.
It is practicing scales for hours on end, until the instrument feels like an extension of your own hands.
It is cutting pounds upon pounds of carrots identically, until it is easier to julienne one than do anything else.
Why do we spend our lives working with on these small minutiae?
We work with them because we must.

The Prophet Jeremiah puts it in a striking way.
\say{I say I will not mention him, I will no longer speak in his name. But then it is as if fire is burning in my heart, imprisoned in my bones; I grow weary holding back, I cannot!}\foott{2}{Jeremiah 20:9}
That is to say, truth is not something you get to choose.
Truth grasps you and forces you to share what you have learned.
It is the difference between shining a light in a dark room and a bolt of lightning illuminating the entire space.
One is active, and the other is passive.

It is here that the real difference between knowledge and truth comes.
Knowledge is something we seek, something we can create or discover.\foott{3}{depending on your belief structure and what specific piece of knowledge we're constructing. I think that as I've grown older I more and more believe that there are very few absolutes. Someone can create the most optimal implementation of an algorithm in a language on a machine, but there is an optimal way to sort in general that we have to find\endnotemark[4]}\endnotetext[4]{Or maybe we've already found it, I guess. I'm not particularly up to date on anything computer related. I doubt that we've found a proof for something generally optimized, but I refuse to look it up because I want to stay on track}
Calculus today is knowledge.
Nearly every college student for generations has been expected to understand it.
Literacy today is knowledge.

Calculus in the time of Zeno, however, would have been truth.
Without calculus, Zeno was able to formulate any number of real and unsolvable paradoxes.
When Newton and Leibniz discovered calculus, they did something that no one before had been able to do.\foott{5}{I think that I remember reading that some non European also discovered calculus at some point, but that does kind of add to the point that I'm trying to build to, which is that truth only becomes knowledge once spread. I only realize now that my goal was to do that}

When a mystic sees a vision of the world, they feel compelled to share it.
The reasons for their compulsions are as varied as the mysteries they reveal.
However, if they do not share it, the truth dies with them.
A scientific invention, unless shared, does no good to the world.

Truth becomes knowledge when it moves from the individual reception to the broader person.
The mystic, the poet, the scientist, and the musician all are given a revelation.
That revelation spreads, and the magic is gone.

\endnotes
\section{Draft 6: 6 January 2024}

The difference between science and Science is the difference between music and Music or religion and Religion.
At its core, it is the difference between truth and Truth.
It is the difference between passion and obsession.
It is, in short, the difference between knowledge and revelation.

Knowledge is simple.
It is facts and figures.
We can express it in theorems and proofs.
Calculus today is knowledge.
Almost anyone can learn calculus today.
Calculus was once revelation.

On the other side of experiential knowledge, the realm of virtuosity is constantly pushing forward.
In one generation, we have Jimmy Hendrix, who is frequently called the best rock musician ever.
Now that the music has been transcribed, however, anyone can learn and play it.

I feel like this is getting away from me again.
\section{Draft 5: 6 January 2024}

One of the hardest questions you can ask a musician is what makes something music.
Like basically any definition, you cannot draw a boundary without cutting out something that is music or including something that is not.
Everyone has their own definitions of music as well, and one person's music is another's noise.

\section{Draft 4: 6 January 2024}
\endnoteversion[c]

In the modern world, knowledge can be broken into any number of binaries.
Most commonly, though, I see truths\foott{1}{truth and knowledge will be interchangeable in the rest of this musing} as broken into subjective and the objective.
We have the objective truths we find in science, where the speed of light or the boiling point of water is constant.
On the other side of this coin, we have knowledge which is only true for the individual.
Think of the way that a song may remind one person of heartbreak and another of their first love.
The objective and subjective may be a helpful distinction in many regards, but there are so many places that it is lacking.

No this is bad

\endnotes
\section{Draft 3: 6 January 2024}

The core question behind any inquiry is what it means to gather new knowledge.
Are we shining lights into a dark room, illuminating what is already there?
Or, are we forcing a wave function to collapse, creating truth from the realm of possibilities?
Whether knowledge is created or discovered, however, both of these approaches presume that new knowledge comes from an active source.

To be sure, there are any number of places where we do, in fact, gather knowledge in an active form.
As a scientist, much of the work that I and every other scientist do is bean counting.
As a musician, almost all of what I do is \href{practicing-scales.html}{practicing scales} and other rote learning.
Artists need to practice drawing lines over and over in order to be able to transfer their thoughts onto the page.

However, when we speak of art or music or science, we are not speaking of these repetitive practices.
While we need these small details in order to fully illuminate our space, but that is not what we care about.
What we care about it the new rooms, rather than the note taking.

No no no this is bad.

I need to plot out what I want to say.
What do I want to say here?

Conclusion needs to be \say{knowledge is passively received}.
How do I get there, though?

Ok I can talk about the fact that we think of the two sides of learning as like science and art.
The two sides are instead knowledge received and knowledge taken.
Knowledge received is fundamentally deeper.
Knowledge received can be shared and given away.
That's great.
Maybe see something there?
Like ideas once learned are able to be shared and seem almost obvious.

See if that works for the next draft?
\section{Draft 2: 4 January 2024}
\endnoteversion[b]
Mysticism and science are interrelated through a weird network.
Wait wait, I have an idea.

When mystics share their findings, we get religions and new philosophies.
As the information spreads, it becomes less explosive and therefore more up to questioning.
You cannot question a mystic about what they saw.
What you can do, however, is ask someone three hundred years later about the consequences of what the interpretations of their experience has caused.
That can be formalized into philosophy.

Mathematics, as we all know, is a form of philosophy.

On the other end, a scientist has a sudden inspiration for an experiment.
Because science requires rigor, that idea is then formalized out into a proof or an experiment that's actually run.
When we continue to try to understand what's happening, we get to mathematics?\endnotemark[1]\endnotetext[1]{Ugh that's so dumb I don't quite know what I'm trying to get at, but this certainly isn't it}

Let us try again.

I already have the framing, which is important.
What I've struggled with, however, is the who cares, the takeaway.
What does it mean or matter if there is an overlap between the two?

Looking at yesterday's drafting, I think that I was going to try the framing of knowledge being discovered rather than created.
Then there's the question I have to ask myself, which is how to connect the two.
Even though I do think that the Catholic Church is True, I don't know if I want to explicitly make the musing Catholic.
I don't know why, but I have this gut instinct that it's something like how I feel like it's intellectually lazy to just go \say{ipso facto, Catholics are right}.

WAIT!

As I was writing there at the end yesterday I had the connection to eldritch knowledge.
Knowledge spreads like a fire.
Lies spread like a fire.

Flames illuminate, sending light further into the darkness, heating and illuminating more space for fire to spread.\endnotemark[2]\endnotetext[2]{hmm, the metaphor is starting to lose its coherence, and I am getting far too into the poetic, rather than illuminating my goal.}

A common thread in science fiction is the idea of forbidden knowledge.
More than that, though, nearly every society has had some variation on restriction of knowledge.
No, I don't think that leaning into the forbidden aspect is where I want to go.

Many cultures have a version of a Promethean myth.
Humanity was gifted fire.

No, I continue to circle this damned\endnotemark[3]\endnotetext[3]{interesting how I only feel as though I could put profanity in this blog when I fall into a poetic form} conclusion, unable to reach it.
I feel as though the conclusion is staring at me from behind a stained glass window, just distorted enough that I cannot see more than its vaguest outlines.
There's a fun joke there here where I can do the whole \say{I would have a conclusion to share, but I haven't gotten the revelation yet.}

What are things that we can take away?
Truth versus a truth?

There's the comment that the friend made to me, which went something like \say{we're always told there's only one right answer in math and science, but that's not true.}
It is true, though, but there are a lot of questions that are subtly different with vastly different answers.
I tried to get to that for why and how.

In rubber ducking a friend,\endnotemark[4]\endnotetext[4]{where you basically just talk to something inanimate to find an answer. In this case, I do it to a person, who often has great insights of their own} they had the idea of being able to refuse knowledge, but not revelation.\endnotemark[5]\endnotetext[5]{who then also required me to cite, because plagiarism is bad}
We also came to the idea that the coins are not science and humanities, but revelation and busywork.
I think that something about refusing knowledge is an idea for a takeaway.
Another idea of a takeaway is that, if knowledge is something we experience, then great\endnotemark[6]\endnotetext[6]{or really any, but I feel like starting with great is a safer line of inquiry} poets are not separated from great scientists by anything except for what revelation they were given.

Oh wait.
Wait.

Wait.

Ok so if knowledge is something that we discover, rather than create, then how do we discover?
Discover is the wrong word, I think.
There we go.
We don't discover truth, truth is revealed to us.\endnotemark[7]\endnotetext[7]{Passive voice intentional here, because the point is that this type of knowledge is not obtained but received}
This feels really good, but I suppose we'll see what rubber duck\endnotemark[8]\endnotetext[8]{there is an idea that we could have recurring people in my life show up as the same nickname over and over, but idk if we'll do that} says.
Update: approved!
Great, so now we just have to write the musing.

\endnotes
\section{Draft 1: 3 January 2024}
\endnoteversion[a] 
I recently mused about the fact that there is a shocking overlap between mysticism and science.
Or, at least, I mused about the way that I consider the two fundamentally related.
In a beautiful example of metacommentary\endnotemark[1]\endnotetext[1]{I really do need to reduce my usage of meta, for all that it remains an accurate representation of what I mean. It's commentary without being explicit? Ok so there's probably a better way of phrasing, but I don't want to put the work in on it right now}, the next day's posting was about the fact that I cannot find a way to write conclusions.
I haven't been satisfied with the way that the science and mysticism post turned out, and it's been playing in the back of my mind on some level since.

Today, I met a friend for coffee, and we ended up talking a little bit about the theory of mathematics.
One of the big questions in my view of theory of mathematics is whether mathematics is discovered or created.
I then connected it quickly to an alleged early Irish musical claim, which is that all the Irish airs were given to humans by the Tuatha.

I don't know if I can quite put the dots together right now, but there's no place to try like here.
So, I've always been one of the sort who believes that mathematics are discovered, rather than created.
There's the idea of shining a light into a dark room.
We steadily see more of reality, even if mathematics is an orthogonal reality to the real real world.\endnotemark[2]\endnotetext[2]{I'm sure that there's another way to describe this, and should revise it in a future revision, but for now I'm just trying to get ideas down onto the page}

Ok so let's see if we can't do something with that.
Mathematics is\endnotemark[3]\endnotetext[3]{are? I never know how plural the word mathematics is}, in many regards, the intersection of science and religion, being as it is one of the purest forms of philosophy.
No, that isn't quite what I wanted to say.

The friend mentioned something about how artists are famous for working under altered states, often with drugs.
Mathematicians, similarly, often describe their great discoveries as though they have been themselves under a trance like state.
However, just as musicians can share their songs with us even after they've come down from a high, the truths revealed to us by science are equally shareable.
There's something about the whole \say{forbidden eldritch knowledge} that I've always thought about, especially 
\endnotes
\end{document}