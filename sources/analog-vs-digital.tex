\hypertarget{analog-vs-digital}{%
\section{\href{analog-vs-digital.html}{Analog Vs. Digital}}\label{analog-vs.-digital}}

As I mentioned yesterday,
I'm keeping a digital and analog journal of my day. This is inspired by
a course I'm taking on London diaries, partially my own past experience,
and partially a conversation I had with the professor of the course.

The course on diaries includes an interactive component. That is, in
addition to reading London diaries, we will be crafting our own. Now,
like many children of the 21st century, I love the idea of keeping
everything journaled online. This way I can make notes on my phone as I
walk, have them automatically update on my computer, where I can expand
with more time. If I wake up in the middle of the night with an idea, I
don't need to wake up a roommate with a lamp. However, the course
requires an analog diary, so I needs must also keep a handwritten
journal.

This doesn't mean that each journal will be totally the same.
Conversely, they will not be totally different. I plan to keep both
journals mostly in sync, and keep only specific things out of each. For
instance, problems I might have with the professor I won't be putting in
the journal that will be graded. Conversely, experiences I don't much
care about that I write for a grade won't be recorded digitally.

In speaking to the professor about the idea of digital data, the point
of Beowulf was raised. However, as I think about it, I agree with the
professor, for the opposite reason. My professor's point was that the
Beowulf manuscript survived because it was hand written, and if a
similar digital file would exist, it needs constant maintenance to keep
operable.

I, however, feel the opposite, at least with regards to a journal. When
I write in my physical journal, there are no other copies of the
information. If I don't show it to others, no one may ever know what
I've written. I can burn the journal and remove all traces of it. Not so
with a digital record. Even if I delete this WordPress site, there are
still the memories left in the minds of those who read it \footnote{whether
  y'all want it or not\\} and potential digital caches and metadata
associated with it. My ``private'' journal is only synced through the
internet and a company or three, so there are almost certainly digital
records of it. So while I agree with my professor about the permanence
of digital and analog writings, I don't quite know if I agree with the
logic, especially on the level of skill and importance that my writings
will \footnote{probably and hopefully, respectively} lack.
