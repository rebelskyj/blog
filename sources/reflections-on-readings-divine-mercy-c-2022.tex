\documentclass[12pt]{article}[titlepage]
\newcommand{\say}[1]{``#1''}
\newcommand{\nsay}[1]{`#1'}
\usepackage{endnotes}
\newcommand{\1}{\={a}}
\newcommand{\2}{\={e}}
\newcommand{\3}{\={\i}}
\newcommand{\4}{\=o}
\newcommand{\5}{\=u}
\newcommand{\6}{\={A}}
\newcommand{\B}{\backslash{}}
\renewcommand{\,}{\textsuperscript{,}}
\usepackage{setspace}
\usepackage{tipa}
\usepackage{hyperref}
\begin{document}
\doublespacing
\section{\href{reflections-on-readings-divine-mercy-c-2022.html}{Reflections on Today's Gospel}}
First Published: 2022 April 24

Acts 5:12 \say{Many signs and wonders were done among the people
at the hands of the apostles.}

\section{Draft 1}
Today is Divine Mercy Sunday, the Second Sunday of Easter, and the Eighth and final day of the Octave of Easter.
What do each of these mean?

Divine Mercy Sunday was instituted in 2000 by Pope Saint John Paul II.
It stems from a devotion that Saint Faustina claimed as part of a private revelation with our Lord and Savior.
According to her records, He desired the Sunday following Easter to be in honor of Divine Mercy, and JPII made that happen.
As of 2002, there are indulgences, both plenary\footnote{total} and partial\footnote{partial} associated with the day.

To earn the indulgence, you must\footnote{as always for an indulgence (I think)} attend Confession, take Eucharist, and pray for the intentions of the Holy Father.\footnote{the Pope}
As with all plenary indulgences, you must be \say{in a spirit that is completely detached from the affection for a sin, even a venial sin}\footnote{https://www.vatican.va/roman_curia/tribunals/apost_penit/documents/rc_trib_appen_doc_20020629_decree-ii_en.html}.
You then must pray in the presence of the Blessed Sacrament the Our Father and the Creed, along with a devout prayer to Jesus.
Or, you can take part in a devotion for Divine Mercy.
For the partial indulgence, you just need to pray to the merciful Lord \say{at least with a contrite heart},

Anyways, moving from what Divine Mercy Sunday is, the Second Sunday in Easter is more or less what it claims to be.
Like Lent's 40\footnote{not counting Sundays, so why they're called Sundays of Lent is confusing to me} days, Easter Season is 50.
It begins on a Sunday, so each Sunday increments by one until we reach Pentecost.

The final day of the Octave of Easter is similar.
Octave means something to do with eight, depending on the context.
Liturgically, it means an eight day period.
Apparently the \say{General Norms of the Liturgical Calendar} assign that the first eight days of Easter and Christmas should continue festivities throughout.
I had no idea, but now do.

On to the readings.
The first reading shows the fact that the Church is alive, even though Christ is no longer in substance and accident present among us.
The apostles continue the mission of caring for the poor and helpless.

In the Gospel, we get one of my favorite passages from growing up.
St. Thomas did not believe that the Lord had visited the disciples, and famously claimed \say{Unless I see the mark of the nails in his hands and put my finger into the nailmarks and put my hand into his side, I will not believe.}\footnote{John 20:25B}
The Lord did not punish him for this, but instead showed Thomas the wounds on his hand and side.
Rather than condemn him for his disbelief, the Lord instead calls those of us who were not able to witness it blessed for our belief.

Growing up, I was an inquisitive child.
Thomas' willingness to question the claims of the other Apostles to me always read as something beautiful.
Within the Tradition of the Church, we do not hold to beliefs simply because we used to hold to them.
Rather, we use Revelation to inform what we believe.
Questions are not punished, but rewarded with answers.

That may be a little unclear, let me try again.
Thomas did not believe the other disciples.
Rather than calling him foolish, or condemning him for his disbelief, the Lord answered Thomas' question.
The consequence of a question asked in good faith is an answer in the Catholic Tradition, and that's vital.

Equally vital in this reading, though, is the institution of a Sacrament.
The disciples are given the power to bind and loose the sins of the world, a power that has been passed down to the priests today.
There's always something beautiful in the Lord's institution of Sacraments.
If I am remembering correctly, the Holy Spirit is always invoked, and there is always a metaphor of breath.
Here, \say{he breathed on them and said to them \nsay{Receive the holy Spirit. Whose sins you forgive are forgiven them, and whose sins you retain are retained.}}\footnote{John 20:22B-23}
I don't know what about the breath and invocation are striking to me right now, but I will think about it for the week.
\end{document}