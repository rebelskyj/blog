\documentclass[12pt]{article}[titlepage]
\newcommand{\say}[1]{``#1''}
\newcommand{\nsay}[1]{`#1'}
\usepackage{endnotes}
\newcommand{\1}{\={a}}
\newcommand{\2}{\={e}}
\newcommand{\3}{\={\i}}
\newcommand{\4}{\=o}
\newcommand{\5}{\=u}
\newcommand{\6}{\={A}}
\newcommand{\B}{\backslash{}}
\renewcommand{\,}{\textsuperscript{,}}
\usepackage{setspace}
\usepackage{tipa}
\usepackage{hyperref}
\begin{document}
\doublespacing
\section{\href{concert.html}{Concert!}}
First Published: 2018 October 30
\section{Draft 2}
A few days ago, I had the wonderful opportunity to see Resonet at the Brighton Early Music Festival.
Now, I didn't know it at the time, but, despite the fact that England is smaller than Iowa, it takes longer to get from point a to point b.
And, I also was unaware that Brighton was on the southern edge of the country.
All this is to say that what I thought would be a .5-1 hour trip ended up being an almost 5 hour round trip to get to the concert.

The show was medieval music, featuring Resonet together with Brighton Early Music Choir.\footnote{wow I wish Grinnell had an early music choir}
The choir was nice, but not extremely notable, which is to be expected from community choir.
Resonet's director played a citola, and did so with equal parts grace and style.
That is, at times he played with flourishes and showmanship, and at others, very subdued and calm.
Their percussionist seemed to have the perfect backdrop for every piece, from a gong to a dulcimer.
Although I was promised a hurdy gurdy that never appeared, the instrumentalist on recorder and bagpipes was stunning.\footnote{I'm still annoyed that there wasn't a hurdy gurdy}
The final instrumentalist played recorder and shawm and did so wonderfully.

As for the singers, the baritone had the voice made for singing love songs.
The soprano was clear and pure, and the countertenor slotted perfectly between the two.

But, as is becoming the case more and more, much of my enjoyment of the show did not come from the scripted portions.
I met a wonderful older woman\footnote{which should be unsurprising to anyone familiar with early music, as older people are the overwhelming majority of attendees} who I chatted with for a good half of an hour or so.
The interactions I can make like that are part of why I love music and cities.

I also got a chance to speak with the conductor about his citola, the piper about his pipes, and the dulcimist\footnote{this is the term I choose to use, regardless of correctness} about his dulcimer.
It was all great fun, and well worth being out too late.

All in all, it was a great time, and I forgot how nice it is to watch live music in a small\footnote{i.e. not multiple tiers of people} space.

\section{Draft 1 (28 October 2018)}
Today I had the wonderful opportunity to see Resonet at the Brighton Early Music Festival.
Those of you with a grasp of English geography may know that Brighton is 2 hours by public transit from London.
I did not, but do now.

The show was medieval music, featuring a group and the Brighton Choir.
The choir was nice.
The group's director played a citola, and did so with equal parts grace and style.
Their percussionist seemed to have the perfect backdrop for every piece, including a dulcimer.
Although I was promised a hurdy gurdy that never appeared, the instrumentalist on recorder and bagpipes was stunning.
The final instrumentalist played recorder and shawm and did so wonderfully.

As for the singers, the baritone had the voice made for singing love songs.
The soprano was clear and pure, and the countertenor slotted perfectly between the two.

But, as is becoming the case more and more, much of my enjoyment of the show did not come from the scripted portions.
I met a wonderful older woman\footnote{which should be unsurprising to anyone familiar with early music, as older people are the overwhelming majority of attendees} who I chatted with for a good half of an hour or so.
The interactions I can make like that are part of why I love music and cities.

I also got a chance to speak with the conductor about his citola, the piper about his pipes, and the dulcimist\footnote{this is the term I choose to use, regardless of societal norms} about his dulcimer.
It was all great fun, and well worth being out too late.\footnote{though I am writing this on the trip back, and may feel differently when I arrive}


\end{document}