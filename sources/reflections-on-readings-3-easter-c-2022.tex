\documentclass[12pt]{article}[titlepage]
\newcommand{\say}[1]{``#1''}
\newcommand{\nsay}[1]{`#1'}
\usepackage{endnotes}
\newcommand{\1}{\={a}}
\newcommand{\2}{\={e}}
\newcommand{\3}{\={\i}}
\newcommand{\4}{\=o}
\newcommand{\5}{\=u}
\newcommand{\6}{\={A}}
\newcommand{\B}{\backslash{}}
\renewcommand{\,}{\textsuperscript{,}}
\usepackage{setspace}
\usepackage{tipa}
\usepackage{hyperref}
\begin{document}
\doublespacing
\section{\href{reflections-on-readings-3-easter-c-2022.html}{Reflections on Today's Gospel}}
First Published: 2022 May 1

John 21:19 \say{He said to him the third time, \nsay{Simon, son of John, do you love me?} Peter was distressed that he had said to him a third time, \nsay{Do you love me?} and he said to him, \nsay{Lord, you know everything; you know that I love you.} Jesus said to him, \nsay{Feed my sheep.}}

\section{Draft 1}
On this Third Sunday in Easter, we continue to see the disciples in the aftermath of the Lord's resurrection.
I find it interesting how the different readings coincide.
The first speaks of the disciples preaching the Glory of our Lord on earth, the Church Militant, some might say.
The second has John seeing the Church Triumphant glorifying Him.
The Gospel does not directly speak on the Church Penitent, but it does show Peter's penance.

Three times Peter denied the Lord, so three times he was asked to affirm his love.
The fishing scene feels interesting to me as well.
In the beginning of His ministry on earth, he gathers many of the disciples by finding them as they search for fish.
Here, they do not recognize the voice of the Almighty when he tells them to cast out.

Anyways, my reflection today is short and somewhat disjointed.
I apologize that my postings are becoming more and more consistently like that.
\end{document}