\documentclass[12pt]{article}[titlepage]
\newcommand{\say}[1]{``#1''}
\newcommand{\nsay}[1]{`#1'}
\usepackage{endnotes}
\newcommand{\1}{\={a}}
\newcommand{\2}{\={e}}
\newcommand{\3}{\={\i}}
\newcommand{\4}{\=o}
\newcommand{\5}{\=u}
\newcommand{\6}{\={A}}
\newcommand{\B}{\backslash{}}
\renewcommand{\,}{\textsuperscript{,}}
\usepackage{setspace}
\usepackage{tipa}
\usepackage{hyperref}
\begin{document}
\doublespacing
\section{\href{book-review-atomic-habits.html}{Book Review of Atomic Habits}}
First Published: 2022 October 20
\section{Draft 1}
By recommendation,\footnote{read:advice} I listened to James Clear's \textit{Atomic Habits} today on a long drive.\footnote{the recommendation was to read the book at some point, I chose to interpret as listen during my 8 hour drive}
It's a book which uses anecdote and scientific evidence to help build the habits we want in our life.
Like every other self-help book I read by pseudo-scientific men who started out blogging,\footnote{a not-insignificant number of books} it uses the example of compounding interest, and how small percent differences become massive over time.
Unlike others, he really emphasizes how you're completely unlikely to see any of the gains until after you've fully established a habit, which is nice.

Another good part of the book is that he makes a point of the fact that habits don't form out of nowhere.
Whether it's a net good for our overall life or not, any habit we start is started, on some level, at least, because it's improving our immediate environment.

The advice seems good and practical, which is somewhat uncommon in these sorts of books.
There's also good mention of survivorship bias, which is often lacking in the genre.
Overall, it's a really enjoyable read, and doesn't suffer from the same page-length padding common to the genre.
It's a quick read that I'd recommend!
\end{document}