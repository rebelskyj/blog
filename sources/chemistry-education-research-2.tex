\documentclass[12pt]{article}  
\newcommand{\say}[1]{``#1''}  
\newcommand{\nsay}[1]{`#1'}  
\usepackage{endnotes}  
\newcommand{\B}{\backslash{}}  
\renewcommand{\,}{\textsuperscript{,}}  
\usepackage{setspace}   
\usepackage{tipa}  
\usepackage{hyperref}  
\begin{document}  
\doublespacing  
\section{\href{chemistry-education-research-2.html}{Reflection on Chemistry Education Readings Week Two}}  
First Published: 2025 July 21

\section{Draft 1}

This week's readings were more generally about education, and I read them away from computer, so thoughts are a little more organized.  
First chapter was about the two major modes of behavioral learning: classical\footnote{think pavlov} which focuses on the stimuli causing action, and operant\footnote{skinner} which focuses on the rewards or punishment for a behavior.  
The second chapter focused on Piaget, and wow he's even cooler than I thought.

My main takeaway from this week's readings is that the authors are way too concerned with neoplatonism.  
That is, there was a major trend in Greek philosophy, especially that which we still have, of going \say{this is how the world should be, and places where reality differ from the model mean that reality is the problem not the model.}  
We inherit a lot of that, especially in context of tuning theory.\footnote{ex: best tuned sounding octaves are marginally larger than a \say{perfect} 2:1 ratio}  
In this case, though, I think that it's particularly dumb.

Plato had this big idea of \say{people can't learn things, they must already know them}, and the authors keep saying that Piaget's theories fail to account for that issue.  
However, it's only an issue if you make it one.  
Like very demonstrably we see things learn behavior and knowledge.  
It requires ignoring reality to construct the belief to make that happen.

I realize that this also comes into play with the whole modern philosophy \say{solipsism is internally consistent} that Descartes started.  
Rather than beginning with observable reality, he started with the Greek thing of \say{I know that I am}.  
The sense of I is not a universal thing.

Anyways, I generally found it interesting to see the developmental categories Piaget sorted us into, and it's wild to me to remember that children can't abstract at first.  
Not my favorite set of readings, but I do also know that I have a fairly strong implicit background in child psychology, because my grandmother did literally write books on the subject which were still used in coursework after she died.  
I also find myself consistently becoming less sympathetic to the philosophy of science treatments that I see in many sources.

These authors are unwilling to say that things are observable by proxy.  
If I follow their argument, they would argue that we cannot observe energy levels in molecules, because we cannot directly see them.  
We can only see transitions between them.  
I think direct observation is a weirdly high bar, especially since then we get the question of what it means to observe.

Is a photon hitting my eye an observation?

Is a photon hitting a detector one?

What's the line between an observation and a proxy one? Most of the time what we measure is electrical current from photomultiplier tubes, not direct photons.  
Arguably under their interpretation, that means that any experiment with a PMT doesn't actually observe things.  
Since they're willing to go to solipsism, how can one even trust their senses?

\section{Daily Reflection}

\begin{enumerate}

\item Did you journal today?

Yes!

\item What ink/pen did you use, and what were your thoughts on it?

Pilot preppy platinum with private reserve electric dc blue.

I still find that my handwriting is messier like this.

\item How's prayer?

Pretty minimal, but I did have mindful time when waiting for a prescription to be filled, so that was nice.

\item How's focus?

Eh. I got two and a half great hours in and then struggled to restart

\item How's sleep?

I took a three hour nap that felt absolutely essential which is not great.  
Still, I do feel at least a little better rested now!

\item How many meals, and how balanced?

For breakfast I had an iced red eye, for lunch I had pasta in butter sauce with spinach and peas.  
It was nice.  
For dinner I had tacos, rice, and beans\footnote{or should it be tacos and rice and beans?} at a bar where I did an open mic.

\item How's the posture?

Fine?

\item How's the breath?

Eh, could be better.

\item How's the movement?

I had wanted to walk to open mic, but then lost track of time\footnote{read: nap felt great}

\item How's the physical flexibility?

Not fantastic, but a stretch  did help me to refocus during the day.

\item How's keeping up with the family obligations?

Good! I've now listened through the album three(?) times.

\item How's the thesis?

Decent, I finished the text of the apparatus chapter today, even if I didn't manage to find/make the figures. That's a tomorrow task, I guess.\footnote{I hate this chapter, and I'm not entirely sure why.}

\item How's the poetry?

eh, wrote a bit ysterday night, have ideas for tongiht.

\item How're the interpersonal relationships?

Good! Saw friends, got lunch with a friend, and am jamming tomorrow with some.

\item How's the music?

Went to an open mic! Did Iowa (Traveling III) by Dar Williams, Inconsolable by Katie Gavin, and Maid on the Shore by Stan Rogers.

It was my first performance in a while and so mistakes were made.  
Friends in the crowd claimed not to be able to notice them, though, so that was nice of them.

\item How's the other writing?

Bullet journal is weird for me because like, where do I take real notes? I guess in a notebook.

\item How's the cleaning?

eh, minimal today, that's something I don't have energy for right now though.

\item How's ordering the life?

I don't know if bullet journal really helped that much, since I think that the barrier right now is entirely motivation, rather than scheduling or tracking.

\item Water?

Not as much!

\end{enumerate}

Current Pen List\footnote{for my own posterity, mostly}

\begin{itemize}

\item Hongdian Black with Blade Nib: Missing but Private Reserve Chocolat  
\item Hongdian Black with Fude Nib: Private Reserve Neon Yellow (since 5/19)  
\item Shark: Purple of some sort, unsure which right now (I think since late june)  
\item Kaweko: apparently private reserve vampire red, but I did just finish it off. Tomorrow I'll refill with something else.  
\item Pilot Preppy: Private Reserve Electric DC Blue I think (I think since late june)  
\item Conklin: Montverde Fire Opal (since 7/17)  
\item Sheaffer: Private Reserve Spearmint (since 7/15)

\end{itemize}

\end{document}

  