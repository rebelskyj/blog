\documentclass[12pt]{article}[titlepage]
\newcommand{\say}[1]{``#1''}
\newcommand{\nsay}[1]{`#1'}
\usepackage{endnotes}
\newcommand{\1}{\={a}}
\newcommand{\2}{\={e}}
\newcommand{\3}{\={\i}}
\newcommand{\4}{\=o}
\newcommand{\5}{\=u}
\newcommand{\6}{\={A}}
\newcommand{\B}{\backslash{}}
\renewcommand{\,}{\textsuperscript{,}}
\usepackage{setspace}
\usepackage{tipa}
\usepackage{hyperref}
\begin{document}
\doublespacing
\section{\href{playing-the-ukulele.html}{Playing the Ukulele}}
\section{Draft 1}
As you many have gathered from my \href{j.rebelsky.com/pierogi-recipe.html}{prior post}, I play the ukulele.\footnote{I’ve already mentioned the different thoughts surrounding the naming of what you do to instruments}
There are two reasons I thought it important to bring with me on my trip to London.
First, it’s the smallest instrument I know\footnote{even for very generous definitions of know} that can play chords and harmony, which is nice when I want to sing along with a backing.

The other reason is that I have minimal difficulty playing melodies on it.
Whether I’m plucking out an old familiar melody to decide how to accompany it, playing along to my singing\footntoe{usually badly} new melodies, or picking out new melodies, I can do them all with relative ease.
Part of this is that the instrument is tuned to four of five pentatonic notes\footnote{for the normal pentatonic scale starting at C} as its four strings.
This also means when I want to play something with the dominant as the low note, it’s fairly easy to do, as the pentatonic scale a fourth below C still uses 4 of the 5 notes.
The only difference is F instead of E.

All in all, the ukulele is a fun and easy instrument.
I would highly recommend anyone to learn it.
\end{document}
