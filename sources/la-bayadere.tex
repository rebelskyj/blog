\documentclass[12pt]{article}[titlepage]
\newcommand{\say}[1]{``#1''}
\newcommand{\nsay}[1]{`#1'}
\usepackage{endnotes}
\newcommand{\1}{\={a}}
\newcommand{\2}{\={e}}
\newcommand{\3}{\={\i}}
\newcommand{\4}{\=o}
\newcommand{\5}{\=u}
\newcommand{\6}{\={A}}
\newcommand{\B}{\backslash{}}
\renewcommand{\,}{\textsuperscript{,}}
\usepackage{setspace}
\usepackage{tipa}
\usepackage{hyperref}
\begin{document}
\doublespacing
\section{\href{la-bayadere.html}{La Bayadere Review}}
First Published: 2018 November 6
\section{Draft 1}
Tonight I had the wonderful opportunity to see \textit{La Bayadere} at the Royal Opera House.\footnote{I'm cultured now}
It was odd.

Apparently in ballet, unlike theatre, you're supposed to read what's going to happen ahead of time.
I did not.
I was confused. 

I did like the second act, though.
There were a couple of dances with the whole chorus that were super fun and made nice lines and shapes.

Overall, while not a fan of opera, at least I have another piece in the music to theatrical story spectrum.
It begins with symphony, then sound painting, then ballet, opera, musical, then non musical theatre.
Yay.
\end{document}