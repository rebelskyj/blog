\documentclass[12pt]{article}[titlepage]
\newcommand{\say}[1]{``#1''}
\newcommand{\nsay}[1]{`#1'}
\usepackage{endnotes}
\newcommand{\1}{\={a}}
\newcommand{\2}{\={e}}
\newcommand{\3}{\={\i}}
\newcommand{\4}{\=o}
\newcommand{\5}{\=u}
\newcommand{\6}{\={A}}
\newcommand{\B}{\backslash{}}
\renewcommand{\,}{\textsuperscript{,}}
\usepackage{setspace}
\usepackage{tipa}
\usepackage{hyperref}
\begin{document}
\doublespacing
\section{\href{reflections-on-readings-31-ordinary-c-22.html}{Reflections on Today's Gospel}}
First Published: 2022 October 31

Luke 19:6 \say{And he came down quickly and received him with joy.}

\section{Draft 1}
I appreciate how last week we heard about the tax collector who knew he was a sinful man, and this week we hear about Zacchaeus, a tax collector, who is called to the faith.
Something that stuck out to me in particular is his line \say{if I have extorted anything from anyone I shall repay it four times over.}
In the homily I listened to, the priest mentioned that he absolutely extorted, which is likely where the other half of his wealth would go.

I, on the other hand, took Zacchaeus at his word.
One recurring message these past weeks is the contrast between perceived holiness and actual holiness.
\say{Of course Zacchaeus extorted}, we and the people at the time would say.
And yet, what if he didn't?

Too often I find myself judging, making wild extrapolations from minimal data.
To me, at least, yesterday's Gospel called us to reflect on why we assume the worst from those who we barely know.
\end{document}