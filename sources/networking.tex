\documentclass[12pt]{article}  
\newcommand{\say}[1]{``#1''}  
\newcommand{\nsay}[1]{`#1'}  
\usepackage{endnotes}  
\newcommand{\B}{\backslash{}}  
\renewcommand{\,}{\textsuperscript{,}}  
\usepackage{setspace}   
\usepackage{tipa}  
\usepackage{hyperref}  
\begin{document}  
\doublespacing  
\section{\href{networking.html}{On Networking}}  
First Published: 2025 June 7\footnote{because I forgot to post}

\section{Draft 1: 4 June 2025}

I don't think that I've reflected here about networking, which is kind of strange.  
I go to a lot of events where I end up interacting with others and trying to make an impression on them, and did so even in the first iteration of this site.  
I guess I probably used to reflect more on the event itself rather than the metanarrative, though.

Last night I was part of a small choir that sang at a reception at the governor's home.  
It was really fun, and I found out while warming up that we were also guests at the event.\footnote{which made my joke of \say{wow they're having us through the front door like actual guests} all the funnier}  
That meant that, in the two hour event, other than the ten or so minutes of a speech and us singing, my job was to mingle.

It was a great time in general, and I think that part of it is that the location and event were meant for mingling.  
People there were generally expecting conversation with strangers, and I was doing my best to not stand in a corner awkwardly.  
I chatted with one person about a potential job and otherwise just met a lot of great people from the surrounding area.

In general, I find that networking is such a hard thing to do, though.  
Part of it is that I am terrible with faces and mediocre with names; when I meet someone at an event, I may not recognize them a few minutes later.  
I don't remember the names well enough afterwards to do the b-school thing of finding an email and writing them.  
However, I have always looked at least fairly distinct.  
This works in my favor, because it means others are more able to remember me.  
Also, because networking events tend to be filled with people who are trying to network, they often remember me and are able to do the \say{following up with our conversation} messages.  
But, I then get to the part that truly makes me struggle with networking: the mercenary nature.

As far as I have ever been able to tell, one of the major points of networking is meeting people who you can help in their careers and who will help you in yours.  
I don't like seeing people as professional objects, and so that makes me uncomfortable.  
Also, like I like talking to people.  
The goal of a networking event for me is meeting interesting people who do interesting things; it is also nice to be able to go \say{A, meet B. You two do the same thing}, though.

Why am I writing about networking now?  
I think that it's primarily because I am hitting the point in my professional life where networking has to start taking a much larger role.  
I am actively on the job\footnote{and relationship} market, and the way to fix that is to find people with positions to fill

This is a much shorter post than usual, in large part because I don't really think that I have much to say.  
It's nice to talk to people, I wish there were more spaces to do it, and I love being able to see the web of interacting humanity that's so common in those sorts of spaces.

\section{Daily Reflection}

\begin{enumerate}

\item Did you journal by hand, and do you feel like the stormy questions in your mind got on the page?

I did, and I think so!

\item Did you do your best to sit in still silence?

Not really, I didn't really have a place where it felt appropriate yesterday.

\item Are you making sure that each task is given your full attention, not just because the task deserves it, but because you deserve the luxury of doing a single thing at a time?

I think generally! I did multitask a little last night while talking with a friend, but that was almost exclusively mindless work\footnote{assembling a chair, etc}.

\item Are you focusing on your posture and breath?

Not as much as I should be, but generally somewhat. I find that right now my shoulders seem to have three positions: rounded forward\footnote{the default}, slightly rounded forward or neutral, not sure which\footnote{which I can hold unthinkingly}, and what looks like straight\footnote{which takes constant effort to maintain}

\item What in your body is holding tension right now? How can you fix it? When will you fix it?

Neck, shoulders, hips I think. I can stretch my neck and shoulders, and I can generally make the time I need to do a full stretch today.\footnote{which requires cleaning my home, and so I should also figure out where things go and how much time I have in the day/where it's going. I have appointments at 1100, 1330, 1600 (the normal burger) and 1900 (board games with a friend who is about to leave forever). This feels like exactly the pace that absolutely destroys me, because the intermediate times never feel like enough to get anything done, but also don't feel like a short enough time period to not do anything. Maybe cleaning would occupy a good portion of the time? I guess if I do my nap after burgers instead of before, then I have the remaining time to work. It's also possible that breaking the sleep into two naps of shorter durations would work for me}

\item Comments on sleep?

I went to bed about an hour past when I meant to last night.\footnote{the joys of talking with friends and recovering from networking}  
I then reset my alarm to give me about an extra hour of sleep this morning, and generally feel ok.

\item How's eating going? In particular, how are you doing with eating plants and unprocessed food?

I had the lemon cake yesterday, and then ate a sandwich for lunch and had dinner-adjacent food at the reception I attended.\footnote{meaning lots of fruit, some mousse cups and mini cheesecakes, one crab cake (which was tragically sauced), a bunch of what I can only describe as fried cheese and ham pinwheels, and some peanut chicken skewers}  
In general the food at the reception seemed about normal processed, and this morning I had a day old pastry, which had peach and I think lemon curd. I'm assuming pretty heavily processed, all things considered.  
I have a lunch packed: sandwich.

\item Are you neglecting any of your familial obligations? If so, how can you rectify this?

Nope! I even started to listen to the album of the week.

\item Cleaning: what is the biggest priority you have right now, and what is the next action item for it?

It's a toss up between making space for the chairs I now own and making space for brewing. Since I will be traveling through most of the rest of the month, though, it probably makes the most sense to do make space for chairs. A part of me is wondering if the ideal might be starting as far away the entrance to my home as possible and slowly working my way forward.  
I can then make the piles of recycling, trash, gifts, things I want to keep but somewhere else, and things which are in their proper place.  
That might work better for me, because that way as things get closer to clean the mess gets moved further from my bed\footnote{bad, interferes with sleep} and closer to the entrance to my apartment.\footnote{inconvenient, reminds me to take it out of the home.}  
I'll see how that works out for me.

\item Thesis: current task. What's preventing you from finishing it? How will you remove that obstacle?

Yesterday I finished the plan for writing and made an outline of the apparatus chapter.  
I don't really want to write that right now, and I also have a presentation I must give on Friday.  
The presentation feels like a higher priority, and right now what's stopping me from finishing is the fact that I don't know what I want it to say.

I can fix that by making a better draft of the presentation and testing it out to see if it's good.

I'm also continuing to monitor the computational jobs I have running. 71 down, 829 remaining.  
At this rate I might be done by the end of the week!

\item Thesis: next task. What will you need to be able to do it?

If the jobs are fully monitored, I need to do the data analysis.  
I think that the plan will be \say{this is where the fit I have predicts line locations, this is where it was assigned, this is what the error was in each of the inputs}.  
I guess that I can make that table now.  
How do I want to do that, though?

I have fifteen interlacing windows, and two variants on each of those.  
Maybe just have a plot for that one? Like amount of data covered and how well it fit?  
Hmm If they're al overlapping, it feels like there should be a good way to visualize that.

This also ties into the presentation, which means that I get to count it as a thing to do.

I also have the apparatus draft due ASAP and plotting out the next chapter (Introduction) because that's due as written for next week.

\item What's the next job you're applying to?\footnote{note that this might be a \say{things we don't post} but}

I don't entirely know, honestly. I signed up for a phone interview yesterday, and honestly feel like it's ok for me to take the day off from the job search.  
Last night I did so much networking that my professional battery is completely drained.

\item Are you intentionally trying to spend time with others?

Yeah! I went to a nice reception last night and I wrote with a friend today.  
I'm going to a friend's house for games tonight.

\item Are you doing your absolute best to ensure that you and those you interact with view the interactions in the same light? Are you sure?

Yeah! I had a great call with a college friend last night, and we clarified a miscommunication we had almost a decade ago that I hadn't realized was coloring my interactions since.  
That was really great!  
Other than that, I have not had a lot of places for miscommunications to occur.

\item Are you keeping up on this daily set of reflection questions?

I am!

\item Are you keeping up on writing the follies? If not, what's in the way?

Yeah! I think that I might want to find a different time for them, though, because the morning is a great time for work and getting the thousands of words in my head on this page, while helpful, are also words that are not being put in my dissertation.

\item How are the long form follies coming? Do you feel like they're weighing you down right now?

They aren't. On the call last night with the friend I mentioned that I really need to write it, because I am pretty sure that I might be having a crisis of faith.  
It's nice that I can schedule that, though.

\item Are you writing poetry? When, and what were your takeaways from the previous day's writing?

I did not do it last night because the phone call was 3 great hours.

\item Are you making music? If not, what is in the way?

I sang last night! Other than that, I feel like I ran out of time.

\item Web novel?

Nope. That's on my due list for the week, though.

\end{enumerate}

\end{document}

  