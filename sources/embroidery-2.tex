\documentclass[12pt]{article}[titlepage]
\newcommand{\say}[1]{``#1''}
\newcommand{\nsay}[1]{`#1'}
\usepackage{endnotes}
\newcommand{\1}{\={a}}
\newcommand{\2}{\={e}}
\newcommand{\3}{\={\i}}
\newcommand{\4}{\=o}
\newcommand{\5}{\=u}
\newcommand{\6}{\={A}}
\newcommand{\B}{\backslash{}}
\renewcommand{\,}{\textsuperscript{,}}
\usepackage{setspace}
\usepackage{tipa}
\usepackage{hyperref}
\begin{document}
\doublespacing
\section{\href{embroidery-2.html}{On Embroidery Continued}}
First Published: 2023 November 11

\section{Draft 2}
Earlier this month, I \href{embroidery.html}{wrote a post} where, among other things, I discussed the fact that I'm now learning to embroider.
Specifically, I'm trying to learn counted thread embroidery, in a style somewhat inspired by Bargello.
I've finally finished the small squares of vertical and horizontal stripes that I had been using as a way to figure out what the stitches looked like with different numbers of threads.
I now have to make a different set of decisions when working on future projects.

On the one hand, more strands makes a prettier looking design.\footnote{to me, at least.}
On the other, it is a massive pain to thread a needle, especially when I start threading it with large numbers of threads.
Along with the friends from the first embroidery adventure, I went to a local tea shop, where we all worked on our own threadcraft and drank tea and chatted.

I, as mentioned, finished the test swatches and immediately started work on a test swatch for how to use color.
The cross stitching friend continued cross stitching,\footnote{which is, as it turns out, a form of counted thread embroidery} and the other two friends both practiced new skills.
One kept working on a woven bookmark that they started near the beginning of the month, and the other started a free form\footnote{that feels like the better term than traditional} embroidery project from a kit.
It was really nice to spend time with friends, and it was equally nice to work on something creative\footnote{in the sense of creating} without having to connect it to anything profit oriented.\footnote{as opposed to my writing, which I've recently learned also helps me to write for the work that I do}
There's something really fulfilling about being able to go from a half formed idea to something real and tangible in just a few minutes.
The fact that embroidery, at least how I've been doing it, has a really nice texture is just an added bonus.\footnote{it's soft and well oriented. What's not to love??}

I think that in the future I might start looking at higher hole density fabric, though, because as mentioned, threading enough threads onto the needle to make the fabric as thick as I want is a bit of a pain.
The smaller the separation between holes, the smaller the thread needs to be.
I've also learned that there's a real and legitimate benefit to doing even numbers of threads, which boils down to only needing to worry about ends at the beginning of each thread, because you can loop and double them over.\footnote{I don't know if I'm explaining it well, but I think the concept should be easy to grasp}


Daily Reflection:
\begin{itemize}
\item Did I write 1700 words for NaNoWriMo?
I did! I also plotted my plot plan in a daily manner, which should help.
\item Did I write a chapter of Jeb?
Arguably, since I finished a chapter and started another.
In reality, I think I wrote just under my new standard chapter word count, but I know exactly where the next chapter is going, so that's probably fine.
\item Did I blog? As mentioned, I said I would revise this. Interestingly, when revising, I think I do far less self talk.\footnote{which tends to come in asides like this (wow (though I did initially type ow, which feels equally fair) that's really meta)}
I think that the post is much better for being reduced in length by around half.
\item Did I stretch? I forgot to do that! I'll do it now and then finish this revision.
I feel a little better post stretching, which is pretty normal, I suppose.
\item Am I doing better at prayer than a rushed and thoughtless rosary? No, but I would like to be.
I should spend some time tonight, especially because it's still early, disconnecting and praying.
\item Am I doing a good job writing letters to friends?
I tried today, which I'm giving myself credit for.
I then realized my pen was basically out of ink and didn't want to risk half the letter coming out a different color, so opted against writing anything today.
\end{itemize}

\section{Draft 1}
I had an idea earlier today to try writing this post in two drafts.
There were a few reasons for that.
First, as you may have noticed, these posts have been getting longer and ramblier the longer the month has gone on.
There are a number of reasons for that, to be certain, but the net result is still a post that gets longer, even when the content may not deserve it.
Second, this 'blog\footnote{hey look I remembered to do it that time} was initially created to journal, at least in part, the way that my writing changes as I go through drafts.
Though there are still remnants of that process\footnote{see the fact that every post starts with Draft 1 and has a first posted date}, that part of the idea has more or less fallen by the wayside, for a number of reasons.\footnote{ooh I do love lists within lists. Still, I'll refrain from doing so here, if only because I'm already almost two hundred words into explaining how I was going to write today's musing, which is about embroidery, without even mentioning the skill once.}

And so, I thought it could be a fun idea to see what differences come from writing and rewriting the same post.
Unfortunately, as I tend to do, I lost track of time.
It's now late enough that I don't really feel like writing two posts.\footnote{truthfully, I barely feel like writing one.}
Still, I know that I have a tendency to overinflate the difficulty of tasks before I start on them, and there's a chance that the same is true now.
If so, then that just means that I need to write this post, and I may have the energy needed to rewrite the draft.
We'll see what happens.\footnote{oof, three hundred words of filler before I even get to the embroidery.
If I do draft 2, this is all absolutely being cut.
That does raise the interesting question, though, of what the point of my drafts is (are?).
If the goal is to tell the same story with each draft, then I should repeat this content.
If, instead, the goal is that each draft attempts to muse most effectively, then I probably shouldn't.
Eh, we'll see what I feel like doing to that bridge when it's time to cross it, and not before.}

Anyways, onto the meat of the post.
I've written \href{embroidery.html}{before} about how I'm learning embroidery.
Today, I finally finished with my blocking to see how many strands of thread I should use and started on another project.\footnote{ok, to be fair, the new project is also a test swatch, but it's a test swatch with a much more intricate pattern, which is like a real project.}
I went to a local tea shop with some friends\footnote{I'm realizing now that the group of us who went to the Embroidery Guild and who went to tea today were identical} and we all worked on our own crafts.

I did my counted thread embroidery, the friend of mine who cross stitches cross stitched, another friend is learning how to weave bookmarks and worked on that, and the final friend\footnote{not that I only have three friends, just only four of us did all the embroidery events.} worked on learning how to do traditional\footnote{honestly I have no clue what the relative age or prestige of different embroidery styles is. He was working on a free piece of fabric doing non bit-wise (is that the term? if not, I feel like it should be) stitching.} embroidery with a starter kit he got.
It was really fun to spend time with friends, and it was also really fun to spend some time working on a craft for its own sake, with full knowledge that the skill will likely never benefit me in any professional sense.\footnote{that may sound sarcastic, but I do truly mean it.
Ever since I realized that my habit of writing a lot of fiction also transfers to writing academic text faster, I feel bad, as though I have corrupted that joy somehow by making it useful.
It's a similar feeling I get when friends tell me that I should set up a way for my readers to financially compensate me.
It isn't that I'm opposed to making money off the work that I do, it's that a part of me feels like it will stop being a hobby and start being a job, and that will change my desire to write/what I write.
I don't know how valid that fear is, but it's one that I've had for a while, ever since I read an article talking about how paying children to solve puzzles makes them solve fewer puzzles than just telling them to do it because it's a fun activity.
Once again, I feel like I'm getting off the point of the post, though.
It will be fun to actually do a word count versus footnote count again for this post, because I'm beyond certain that there are more footnotes than in text words for this post.}
There's something really fulfilling about having a rough idea for a pattern, trying it out, seeing where you can improve, and then trying again in a matter of minutes.
It's even nicer when the failed attempt is still pretty and has a nice texture.\footnote{like wow I love the texture of embroidery floss that's been stitched.
It's soft but also incredibly aligned and ordered, in a way that's a little hard to describe.}
I think I might want to start looking at getting higher number aida\footnote{I'm sure that this has to be an acronym of some sort, but I couldn't tell you anything about what it stands for}, because it's a massive pain to thread six or more distinct strands of embroidery floss onto a needle to sew.
The larger the number, the tighter the grid.\footnote{I'm positive that the number is holes per inch or holes per cm or some other metric like that, for all that the specifics don't interest me right now (right now absolutely the key phrase, because I know that I'm going to geek out about everything related to embroidery someday soon, I just don't want that day to be today)}
Right now I think I'm using 20 or 24, which I saw a recommendation online for six strands when embroidering.
In either case, I've also learned that doing even numbers is nicer, because then you can just cut a length of floss twice as long as you want to use and then double it with the needle.
There are a fair number of benefits to doing that, most of which revolve around the fact that you don't have to worry about the floss falling out of the needle as you sew.\footnote{I can't help but feel like sew is the wrong verb here.
Maybe it is the right one, though, since I guess it does describe the action taking place? I guess that's something I should look up before I do this musing again}

Ok so wow this post absolutely needs to be cleaned up.
Even though it's been dark for hours,\footnote{wow I love the winter (this one is sarcastic)} it's still pretty early, so I should have time to revise.

Daily Reflection:
\begin{itemize}
\item Did I write 1700 words for NaNoWriMo?
I did! I also kind of plotted out what my goals are for writing each day plotwise for the rest of the month, which is nice.
It made it a little easier to get to the wordcount today, since I had a real goal in mind.
\item Did I write a chapter of Jeb?
I was halfway through a chapter at the end of last night, which I did not end up finishing.
I finished that chapter and started on another one, though, which is like writing a chapter.
One chapter was finished.
\item Did I blog? I did! And I think I'm even going to revise it wow.
\item Did I stretch? I should stretch. Before revising, I think that I'll do that.
\item Am I doing better at prayer than a rushed and thoughtless rosary? I don't think that my rosary last night was particularly rushed, and I did pray an Angelus today, though that one was very rushed.
\item Am I doing a good job writing letters to friends?
So I did legitimately mean to do a good job writing letters to friends.
But, when I went to write letters between embroidery projects, I found that my pen was almost completely empty of ink.
That was a shame, but one that I should have remembered, since we're like two weeks into the month, and that's usually when I run out of ink.
\end{itemize}

\end{document}