\documentclass[12pt]{article}  
\newcommand{\say}[1]{``#1''}  
\newcommand{\nsay}[1]{`#1'}  
\usepackage{endnotes}  
\newcommand{\B}{\backslash{}}  
\renewcommand{\,}{\textsuperscript{,}}  
\usepackage{setspace}   
\usepackage{tipa}  
\usepackage{hyperref}  
\begin{document}  
\doublespacing  
\section{\href{mindfulness.html}{On Mindfulness}}  
First Published: 2025 September 16

\section{Draft 4: 16 September 2025}
One of those I hold dearest and nearest to my heart recently posted a consideration\footnote{much as I have chosen follies and my father chose musings} about wanting to be less connected to telephones and occupy time more mindfully in general.
I too, wish that I was less attached to screens and more present in my day to day.
Even had my dearest friend not posted the consideration, though, I think that i might have come to this folly on my own.
As I have been mentioning, I'm on a train, and have been now for more than a full calendar day.
Something about the nature of riding a train feels almost designed to foster mindful usage of time.

In part, I think it is because of the fundamentally intentional nature of riding a train.
Unlike cars, who nominally follow lanes, or planes which do so solely out of agreement, trains can only run on a single path.
There do exist depots to change lines, but they are few and far between.
The route I ride is not one which is being chosen by the train conductor; it was chosen by the backs of countless men whose labor built the rails through the empire.

Even more than that, though, the train is filled with windows.
Each moment is a new vista.
Each blink misses some ineffable piece of a beautiful reality.

Travel by train is also not perfectly smooth.
As the train rides along, there are gentle hills and vales, bumps and turns.
Each one is an invitation to remember that time, like the train passes by.

Finally, travel by train is fundamentally opposed to the modern optimization and multitasking model.
Trains move around the speed of a car, though slightly slower usually.
That is, they move far slower than a plane.
Also unlike a plane, the trains I have been on do not have any WiFi connections.
I've been in dead zones which last hours on end, and so there is no way for me to ignore my present with the allure of the online.

These past few weeks, I have found myself living minimally.
That is, I have been making few decisions, and generally letting the flow of life move me along.
I play games because they occupy my time.
Partially as a response to my reflection last night, I came into this morning wanting to be more mindful of the passage of time.
I think I can call it a success, given that I've written more than two drafts of a musing, bound a book, read eleven chapters of my web novel, three chapters of \textit{Musicking}, and meditated for forty minutes.\footnote{not quite forty, but}
It is not yet 1500, meaning that I still have more than 9 hours with which to continue to use the day.

Why, though, have I been living minimally?

Computer and phone screens are optimized and designed to draw me in.
The quick rush of scrolling or watching a screen go boom are engaging to base parts of myself, and that's something that I cannot control.
This is not to let myself off, however.

What is to let myself off is the realization that it's been two weeks since I finished the paperwork for my doctorate.
Fourteen days to recenter after decades of schooling feels reasonable to me.
However, fourteen days feels like enough, at least today.
I know that I am most happy with my memories in days and weeks where I pushed myself.
Never do I look back on days where I was gentle with myself and be happy for it; the most I can do is accept that it was likely necessary to rest.

So, in the interest of being more mindful, I look at my daily reflection.
In general, the one I've struggled most with is the idea of not looking at screens.
For whatever reason, I can often read for far longer on a phone or computer than I can with physical books.
I can also write for far longer on a computer, though that is far less surprising; I both have a faster output (matching my thoughts) and can do so for longer without pain (both because less motion and ones I'm more familiar with).
However, there are obviously gradations to what it means to use a screen.

Writing these follies, for instance, feels a reasonable use of the screen.
I can and often do gaze out the window at the passing scenery as the words appear on the page.\footnote{thanks touch typing practice. I should like to get back into you}
Also, it is something mentally engaging and that brings me closer to the sort of person I wish to be.
Reading on my phone or computer can likely have similar benefits, but perhaps because I cannot look away from the screen while consuming, I find it far less acceptable towards the goal.

In general, I want to become the sort of person that I want to become.
That feels like a tautology, partially because it is, but also because I don't entirely know what person I want to become.
There are arguments in philosophy of belief about what it means to believe something, and I more and more find myself agreeing with the idea that belief and action are intrinsically linked.
That is, you can, in fact, tell what someone believes by how they act.
I'd like to be someone that believes time is precious\footnote{not the right word, I think? Or maybe i just put too much capitalism thought into the word}, and that means I need to act like it.
I want to be someone that creates, and that means creating.
I want to be someone with thoughtful and well-derived opinions; that means I need to read broadly and deeply.

While I cannot choose to believe immediately, I can take the paths which branch towards a better version of me.
Rather than blindly walking a rut in life, I can choose to look up, seeing the rushing river\footnote{the train has been going along the Colorado River for hours now} that marks many other winding paths.
Mindful life is a goal, not a destination.
Each day, I can try to believe in mindful life more.

Post Script\\
It occurs to me that I also don't want this to fall into the trap of the capitalist \say{every moment must be optimized}.
Treating time as precious means living actively, not always filling the day.
Sitting and looking at the way the stones break the tide is just as precious a use of time as generating something for others to consume.
I need to remind myself of this often, especially in context of what it means to have spent time well.

Time being something one can spend, after all, fundamentally puts time as a form of capital.

\section{Draft 3.1: 16 September 2025}
\subsection{Meta-commentary}
It occurs to me that the initial goal of drafts was to see the way that a thought developed.
Right now, I am treating each draft more like a small essay of its own.
In the past, I would title those with fractions of draft numbers.
Of course, most of the time I would do so in fragmentary ways, rather than these (relatively) complete drafts.
Still, I think that there are often ideas expressed within different drafts that all belong in the \say{final} version of a post.
With that in mind, let's try generating a full draft of what we have
\subsection{Real Text}
For this partial draft, I think that I want to mostly free write about what I think is important in each draft, starting at the beginning.\footnote{ooh, counter to the way I post!}

Draft 1:

I mention the fact that this folly is inspired by a friend's quest towards mindfulness and better texting.
I want to live better, including an idea I don't develop about how I can read for longer on phone than on paper.
I know (novel) that I can also write for far longer on computer than by hand, though that may be equally speed of typing as ease of writing; my hands do not cramp when I type.

I point out that I am generally happiest when I drove myself like a task master.
I reflected on the fact that productive is not the word I want, though it is at least close to the idea I want.

I reflect on the mindful nature of trains, where scenery is momentary.

Finally, I talk about the changes I wanted to make towards myself today.

Or, really finally, I end with the note from my brother, who is exploring meditation.

Draft 2:

I comment on how passive I am experiencing life, and again run into the fact that my words paint a picture of me as capital.
Overstimulation becoming stimulation, and the way that double media consumption means that the odds of both ending together are minimal.
Found a typo, but discussed a way to maybe be better at playing the game by stopping when it stops being fun.
(new) it did work.

I reflected on being scared of boredom, but I don't actually think that's resonant with me.
I think that I'm just bored of being bored.
Might be worthwhile to try just turning everything off for an hour and staring out the window? (novel)

I went through to see how much media I wanted to consume.
Perhaps because I did so, I then consumed the media more quickly.

Finally, Draft 3:

I reflect on the difference between intentionality and mindfulness, especially in relation to trains, which cannot move as freely as other forms of transport.
I also point out that trains are slower and more divorced from the broader world.\footnote{a folly about the sacred nature of trains would be interesting, and is something to consider in the post on trains}

I point out the comfort, and the way that I am becoming more connected with physical, rather than digital, time.

Finally, I end with the note that I am generally mentally engaged. 
These past weeks are probably better seen as recovery from the effort of my dissertation, rather than my new norm, when it comes to how I learn and grow.

Now to try sitting mindfully for an hour looking out the window, and then to write Draft 4.

Welp. Made it 36 minutes, but then the train came to a stop. Hard to stop thinking and not sleep.
\section{Draft 3: 16 September 2025}
This next draft\footnote{init. reflection, changed before starting Draft 4} may belong better in my folly \href{trains.html}{On Trains}, but it will lie here instead.

I am not entirely sure where the line between intentionality and mindfulness lies.
Certainly they describe two different modes of operating in the universe, as they are two different words.\footnote{I am a firm beleiver that every word is uniquely meaning bound}
I wonder if it may be a causal relationship.
By living with intention, life becomes mindful.
Or is it that when approaching life with mindfulness, intentionality comes out.

I suppose in setting the causal relationship, I can start to tease a difference out.
Intentionality requires action, while mindfulness does not.
I can intentionally sit and do nothing, but the absence of doing becomes the action.
Mindfully sitting, however, may result in doing nothing.

It is perhaps unsurprising that I find myself playing with the border between these two terms.
Trains are, in many regards, the most intentional of forms of transit.
They can only move along perfectly prescribed paths, and only those trains which are constructed with rails at the exact same distance are able to ride.
Contrast this with cars and motorcycles, which though nominally bound, can still at least in theory break from the street.
Abstracted further, boats follow what they call sea lanes, but they are not fixed nor mandated.
At the final level, planes fly in three dimensional space.
Even though constrained to go in set patterns, there is nothing in the form itself which requires the movement.

And yet, despite the fact that planes are theoretically freer than cars, modern travel makes cars the freest option.
I suppose that if I were able to fly my own plane I might feel differently, but.
If I were to do this trip via plane, there would have been a similar number of locations that I could go to, with similarly constrained scheduling.
Of course, planes are faster than trains, so what is a daily or sub-daily arrival can happen hourly.

If I were to have driven this, there would have been functionally no constraints on the way I traveled.
Far from being intentional about when I reach and leave each location, I would be able to even choose where I travel on a whim.

It is perhaps this freedom in cars that makes driving feel so much less intentional to me.
Even though there is almost always a best route, rare are the roads which do not at least allow for a diversion every few hours.

It is perhaps the speed which keeps me from feeling intentional on a plane.

On a train, however, the touch of the maker is everywhere.
I pass by the same route that has been tread countless times by countless others.
Unlike in a car, I know that my train follows the identical route; there is nowhere else its rails can take it.
And, the train seems almost designed for intentional travel.
In today's fast-paced world of planes, the train travels no faster than a car.
In a world where even planes are beginning to get high-speed WiFi, all of the trains I have driven on either removed their WiFi or never had it to begin with.
Hours at a time see me unable to reach cellular service.

Unlike in a car, however, this inability to connect to the outside world does not leave me feeling a sense of fear.
I am not in control of the motion, and the system is designed to make it hard or impossible to collide with another train.

Too, the train has a viewing deck.
Even outside of this area with chairs facing towards the windows, the seating area has windows larger than in my home.
My view of nature is just obstructed enough to remind me that I am enlosed in a climate protected shell.
The scenery passes me by, and because trains were laid with linear intention, we pass through the mountains, rather than looping around them as cars are oft to.
Because trains do not turn well, the track does not whip me about.

And so, even though I find myself fighting to break the habits which call for me to live mindlessly, the mere act of riding a train pushes me towards a mindful state.
While I can wander up and down the aisles, there are still only four cars.
I can stream non-local media when there is service, but it is so variable as to be difficult.
I can waste my time away\footnote{whittle? spend? waste just feels wrong} by playing games, and yet even still the not-infrequent bumps mean that I am never able to completely escape the reality of motion, unlike on a plane or while stationary.
The passage of time flows by me, and though I am untethered from hours and minutes, I am much more tied to the time of day.
I watched the sunrise from dawn through nearly noon.
I watched the sunfire sky fade into night.

Of course, this may equally be that I am simply finally recovering from the burnout of writing my thesis.
I often found\footnote{oof, I wrote find, but that is no longer true, now is it?} that I would only make it a few days or a week into any given break before I would be done with rest and once again yearn for motion and action.
Or, at the very least, I become able to sit with myself again.
I want to play the games right now, but I think this is a legitimate want, rather than simply a question of filling my hours.
I'm going to try asking myself after each round whether I actually enjoy the gameplay loop still.
If the answer is no, then I will set the game down and find another activity.

\section{Draft 2: 16 September 2025}
One of my dearest friends recently posted a consideration\footnote{inspired by my questioning of what to call these writings, I think} on the theme of mindfulness.
It made me think about the fact that, despite my own goals to experience this month more mindfully, I have really been fairly passively spending my time.\footnote{hate the fact that my metaphor is so steeped in capitalism.}
I am on the way back to the middle from the western portion of the trip, which makes this as good a time as any other, if not a better one, to think about how to better use my time.\footnote{experience my time? work with my time? hmm}

So, why do I passively take in time?

I do enjoy playing the games that I play, at least at first.
However, much like the joke of \say{oh no, there's milk left over after I finished my cookies... oh no there are cookies left over after I finished my milk}, I never seem to end the games in sync with whatever media I'm consuming.
There's a form of the hedonic treadmill: neither listening to the podcast nor playing the game feels stimulating enough on their own, and so I tie them together.
Perhaps in the same way that I consume media\footnote{as fast as I can}, I feel that it's less unproductive to consume more media in the same time period.
Of course, there's nothing about this phase of my life that requires production.

I suppose one answer to the wasting time is to reflect after each game whether or not I'm still actively having fun.
If not, I can stop the game.
I do know that decision fatigue is a real thing, however.
Then again, I'm kept in this train for the rest of the day.
Assuming that decision fatigue is reset at sleep, there's minimal that I can choose other than being mindless.\footnote{is mindless the antonym for mindful? great question}

Why else do I passively pass through the stream of time?\footnote{better? Nah too poetic}

Boredom is scary.
It don't think that I'm yet in a great mental place in regards to the death of my mother and the death of the part of me that is a student.
When I stop to let my mind go where it will, I worry that I'll get stuck somewhere darker than I can escape.
Writing it out, though, I hate that option.
Fear will not define me unless I let it, and I refuse to let it right now.

Other than fear of boredom, ease of getting trapped in loops, and a vague feeling of wanting to consume, is there anything else that drives me not to be mindful?

Lack of inspiration maybe; there's nothing I have with me that I particularly want to give my time to.
I suppose that I have yet to do the philosophy books, and those would likely work really well.
I enjoy book binding, at least in theory\footnote{for some reason the thread has been a huge pain for me lately. Maybe I just need to use more strands}, and I said that I wanted to learn philosophy.
That's still true enough, and so maybe when I finish here, I'll once again work on reading the web novel, and then bind the book.
For now, I'm comfortable goign through the media that I have until I've caught up\footnote{if that's the right word} on all the things I said I would do.
Then again, 160 items is a lot.
I'm curious how much time they'll take.

At single speed: 86 hours, 15 minutes if I include the audio drama, 56 hours, 45 minutes if not.
At 3x speed, that means about 28 hours 45 minutes or 19 hours.
By the end of the trip I'll have absoluitley gone though the first of those, and likely the second as well.
However, it's enough that I don't think that I need to pace myself lest I run out of content.\footnote{also ignoring the whole \say{all the content creators are still creating content}}

For now, let's see what book binding feels like while listening to some of the videos.
I'll come back and reflect on that then. maybe then also read some of the books I'm binding.
Current time as of sitting down to bind intro to phenomen and musicking: 758.
16 folios for musicking later: 806.
One hour later, the other book is folded, and musicking is finished.
Listened to about 1.3 hours of 3x speed, which feels about right?

Now to go read!\footnote{because wow this window seat is boiling hot}


\section{Draft 1: 15 September 2025}
One of my friends is considering ways to live more mindfully and to also be better at texting.
Despite the fact that these two goals feel almost intrinsically opposed, I cannot say that I don't understand.
I think that I would benefit from being worse at texting, but both of us share the sentiment that we would be better served by using phones less.
So, why do I want to live mindfully?

At a gut level, I think that it's probably generally better to live life experientially than passively.
When I'm on the phone or media-maxing\footnote{3x podcast and bright flashy game}, I don't have time to sit with myself.
Even when just on the train, I find myself getting bored after just a few minutes.
For some reason, I can read for far longer on my phone than on my computer.

I'm not entirely sure what the solution is.
On the one hand, most of the platforms I use to contact people also have space for me to mindlessly scroll.
I also enjoy being up to date on the trends of my generation.
On the other, I do really love the concept of living authentically.
I've talked a lot with some people about how I may live too much for the future.
That is, I rarely look back at myself and wish that I had been gentler on myself.\footnote{my thesis is a shocking counter example}
Rather, I find that I am most happy with my life when I was pushing myself.

I'm just over a week into my journey across the country.
I'm just over two weeks into my journey as a Doctor.

In that time, I have been far less productive than I have desired.
Productive is not exactly the word I'm looking for, however.
Much as I would have liked to do more posts or get back into my web novel, I also wanted to be on my devices less and be more present.
I wanted to bind the books I've packed and read them.

Maybe the word I'm looking for is actually mindful.
I have not been as mindful of my time and place as I've wanted.

Something about riding on the train is fundamentally geared towards mindfulness.
The scenery is constantly passing by, and so each sight out the window is different from every view before and after.
I've seen countless stunning vistas that passed away before I was able to get my phone out to photograph them.\footnote{I've been asked to take photos of my travel. Also, while preparing my thesis, I realized that I don't take photos with people as often as I might like.}
Whenever I look up from a screen or book or nap, I find my breath almost taken away by the beauty of the land.\footnote{with the obvious exception of  at night. For whatever reason, Amtrak uses blue lights at night, and so my night vision is effectively nonexistent.
Unlike on highways and freeways and interstates, there are few lights along the path a train travels.}

So, as I get ready to end my day\footnote{it may be 2000 right now, but I do find that there's something really nice about just kind of lying down and vaguely letting my thoughts flow}, I think on the ways that I can be better going forward.
There's a philosophy channel I follow who talks a fair amount about belief.
Apparently, there exists a fairly large strain in the philosophy of belief that claims belief is fundamentally tied to action; if I do not act a way, I do not truly believe.
From that, being better to me means equally setting better beliefs and acting in accordance with those stated beliefs.
Looking at my daily reflection, I think that there is something to be said for the goals I set.
I will try to go stretch and pray now, and then I'll lie down and see how I feel in a few minutes.
Maybe I'll be up again for some composition, maybe I'll want to read my web novel.
Maybe I'll want to use my screen and need to remind myself that while the base parts of me want to spend all day staring at media, the rest of me does not.
As my brother said, \say{it's about making the media in your mind.}


\section{Daily Reflection: 16 September 2025}

\begin{itemize}

\item Did you journal by hand today?

I did! After a lovely breakfast with an old Virginian.\footnote{or should I call ppl from Virginia virgins?}

\item Did you do a folly?

Yesterday! I also wrote a draft of today's yesterday, which is kind of fun.

\item Did you in some way, shape, or form advance the web novel?

Yesterday after posting the folly I read nine chapters.
This morning I have read 11 chapters.
That means that I just have like 250 to go.

\item Did you work on music, whether education or creation?

No. I listened to an audio book, so not even music in that sense.
I guess that I did talk a fair amount about music in some chapters of the web novel!

\item Did you work on book binding?

The novel is now fully bound! Other than that, though, no.
Yet another activity I can do.

\item Did you work on another hobby?

I played video games, and listened to podcasts and audiobooks.
I think that counts.

\item Did you stretch? Really?

Yes! Before bed I stretched and then this morning after breakfast I stretched again.

\item Prayer?

A little bit.
Still hard for me to know what to pray.

\item Meditation?

Yeah! Or at the very least, sitting without explicit stimulus and trying to have a quiet mind.

\item Reading?

Eh.

\item Minimizing screen time?

Eh.
\end{itemize}

Current Pen List\footnote{for my own posterity, mostly}

\begin{itemize}  
\item Hongdian Black with Fude Nib: Diplomat Caribbean (8/30ish)  
\item Jinhao Shark: Diplomat Caribbean (8/30ish)  
\item Pilot Preppy: Private Reserve Electric DC Blue I think (I think since late june. I think)  
\item Sheaffer: Private Reserve Spearmint (since 7/15) (I Think)
\end{itemize}

\end{document}