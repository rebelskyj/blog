\documentclass[12pt]{article}[titlepage]
\newcommand{\say}[1]{``#1''}
\newcommand{\nsay}[1]{`#1'}
\usepackage{endnotes}
\newcommand{\1}{\={a}}
\newcommand{\2}{\={e}}
\newcommand{\3}{\={\i}}
\newcommand{\4}{\=o}
\newcommand{\5}{\=u}
\newcommand{\6}{\={A}}
\newcommand{\B}{\backslash{}}
\renewcommand{\,}{\textsuperscript{,}}
\usepackage{setspace}
\usepackage{tipa}
\usepackage{hyperref}
\begin{document}
\doublespacing
\section{\href{euchre-tournament.html}{Euchre Tournament}}
First Published: 2023 September 13


\section{Draft 1}
As I mentioned \href{talk-planning-eclipse-1}{yesterday}, I have done a number of things during my downtime from blogging.
I'd say one of the most fun of them was going to a friend's euchre tournament last Saturday.\footnote{how was that only a few days ago? It feels like so much time has passed since then}

I think that I knew how to play euchre before the tournament, though I'm not entirely sure if I ever actually did.
For those who don't already know what euchre is, it's a trick taking game.
As far as I've ever played it, it has four players.\footnote{and I'm almost certain that there are variants for fewer, though I haven't noticed any for more (or fewer, I suppose)}
The deck is comprised of a standard deck of cards, though with only cards 9 and higher.

Each hand is played in the same way.
The dealer passes out five cards to each player, then turns up the top card of the remainder.
If someone wants that to be the trump suit, they tell the dealer to pick it up, at which point it becomes trump, and the dealer gets to replace any card in their hand with it.\footnote{they do not reveal to the rest of the players what card they replaced.}
From there, the person to the dealer's left leads with the first trick.
If no one tells the dealer to pick it up, the card is discarded and everyone once more goes around and can choose to pick trump.
If no one does the second time through, there

Points are scored at the end of each hand.
If your team\footnote{you and the person sitting opposite you} took a majority of the tricks\footnote{three or more}, you get a point.
You get another point for taking all five tricks, and another if the other team called trump.
I'm told there's also a benefit if you choose to go alone, though that was banned at the tournament I went to.

For those who've never played a trick taking game, the general rules are that high cards win a trick.
Each trick's suit is the suit that the first player led with\footnote{which is the benefit of leading}.
However, the trump suit always beats any other suit.

In euchre, you are mandated to play in suit if you can, including if you have trump in your hand.
That's an interesting variation, and one that I'll have to consider more as I keep playing it.
I'm sure that it does things to the strategy that I did not immediately think about.

Euchre also introduces an interesting variation on trump, where the Jack of trump suit becomes the highest value card.
The second highest value card is the jack of the other same colored suit.\footnote{so if trump is spades, the Jack of Clubs, etc}\footnote{oh, that does also mean that the jack is now a part of the trump suit, not the other suit, interestingly enough}
After that, trump goes down ace, king, queen, 10, 9.
It's kind of fun that jack goes from being a middling card to the best one if and only if the suit is trump.
I'm not sure how that impacts when you should have the dealer pick up the card, but I am certain that it will.

Ok so, explanation of euchre out of the way, let's talk about the tournament.
Of the twelve participants, three had played before, and I vaguely knew the rules of trick taking games.
I ended up on a team with one of my group mates, and we got absolutely destroyed in the first game.
I primarily blame luck for that, as it's hard to win a game when you don't have any trump cards in your hand.

The winning teams all rotated, and the second game began.
Just as time was running out, we were down one point.
I was dealer, and I gave myself\footnote{somehow} the ace, queen, and jack of spades, along with the jack of clubs.
I also had the nine of clubs, but that doesn't matter as much.
What's most important is that the card that declared suit was the king of spades.

Obviously, no one else wanted spades to be trump\footnote{because I held basically all of it, so they didn't.}
When the choice came to me, I, obviously, took it, and immediately declared that we had taken all five tricks as time was called for the game.
That lucky streak remained in our next game, and we ended up winning 2/3 games, which put us in second place.\footnote{behind one of the two teams with absolutely no experience with the game, which is kind of funny.}
It was a really fun time, and then the hosts took us to their back yard for a fire.

The fire was incredible, especially because they had some fancy fire pit with holes in the bottom, which means that the fire both never suffocated and also burned incredibly hot and fast.
The fact that the wood was incredibly dry probably helped that, but it was mesmerizing to watch the flames.

\begin{itemize}
\item I'm watching a blender tutorial after this, which is progress, if only meta.
\item I'm going to clean after the blender tutorial! That is some fighting against entropy.
\item Blogging streak is returning, which is great and fun and exciting and happy.
\item I actually did stretch, which is wild to me. It felt nice, but wow is my entire body stiff. I should stretch again tonight.
\item No exercise today, but I did walk, which is good.
\item I went to bed a little late last night, but set the alarm for much later.
\item I woke up before my alarm and everything!
\item I'm not sure that I'm not sick, so I stayed out of the chapel on purpose today.
\item If I write more Jeb today, which I plan to, I will be ahead, which is nice.
\item Book remains totally plotted, though I have been considering dropping the total number of chapters from 500 to 400 or even 300.
At a rate of three chapters per day, there are a little under two and a half years left in the book.
I would rather it be done before then, which either means I need to have fewer chapters or release them faster.\footnote{or both I suppose}
Every hundred chapters I drop is about two thirds of a year less, which is weird to think about.

In actual reasoning, though, I'm not sure if the third, fourth, and fifth sentences describing the book actually need and deserve their own total arcs.
\item No poetry, for all that I should.
\item Oh gosh, that's something I keep forgetting to do.
I will work on my album after cleaning and tutorial.
\item I just now wrote 5 things I like about myself.
\item I just now wrote three things I'm excited for.
\item As above, just wrote ten more things I'm grateful for.
\item I think that in general I've been doing pretty well cultivating joy.
I certainly am living more in the present, which is nice.
\end{itemize}

\end{document}