\documentclass[12pt]{article}[titlepage]
\newcommand{\say}[1]{``#1''}
\newcommand{\nsay}[1]{`#1'}
\usepackage{endnotes}
\newcommand{\1}{\={a}}
\newcommand{\2}{\={e}}
\newcommand{\3}{\={\i}}
\newcommand{\4}{\=o}
\newcommand{\5}{\=u}
\newcommand{\6}{\={A}}
\newcommand{\B}{\backslash{}}
\renewcommand{\,}{\textsuperscript{,}}
\usepackage{setspace}
\usepackage{tipa}
\usepackage{hyperref}
\begin{document}
\doublespacing
\section{\href{reflections-on-readings-32-ordinary-a-23.html}{Reflections on Today's Gospel}}
First Published: 2023 November 12

\section{Draft 2}
In the first draft of this post,\footnote{readable below} I thought a lot about what the Gospel message was, and then remembered that other, better read, and smarter people than I have asked the same question before.
So, after I reflected on the readings, I read through some reflections from Doctors of the Church and other Saints.
In this draft, I think I'd like to go through the three readings in order, because that feels like a better way to construct the narrative.

We begin with the First Reading.\footnote{that feels obvious in retrospect, but it feels important for segue reasons}
The reading today comes from Wisdom\footnote{also known in some circles as Wisdom of Solomon}, and more or less exhorts the reader to seek wisdom.
In reading commentaries, I was reminded of how much of the faith has been so effectively handed down through the centuries.
These days, it feels obvious to say that seeking wisdom means seeking the Holy Spirit, but that was not always a settled question.\footnote{and, to be fair, in some belief systems it still isn't settled}

The Second Reading is also pretty straightforward.
St. Paul explains that we should mourn with the hope and knowledge that those we love are asleep in Christ, ready to be raised up on the last day.

With both of those readings priming us, we are taken to the Gospel.
The Gospel passage today concerns the parable of the ten virgins waiting for the bridegroom.
As you might expect, there are a number of interpretations of nearly every part of the reading, and what, exactly, they symbolize.
What is not in contention, though, is the meaning of lamps and oils.

Lamps represent a belief in the Almighty, and the oil represents the good works we do.
Works without faith are meaningless, as oil without a lamp is fairly useless.
St. Augustine points out that, at some point, we are unable to create oil ourselves.
That is, while we can press olives to make oil, we cannot cause an olive to grow.
In such a way, any good we do is only through He who is Goodness itself.
But, just like a lamp without oil cannot shine, so to is faith without works dead.
It's interesting to me how clearly that was seen in the Early Church, given the controversies that arose a few centuries later.

The fact that all ten virgins fell asleep is seen as a euphemism for the fact that all die.
When the bridegroom, Christ, returns, not all will be ready.
Rather than explicitly punishing, as he does in other parables, though, he simply ignores the faithless.

Two parts of the Gospel that I did not immediately think of as speaking to any truths were the fact that there were ten, and the fact that the ten were virgins.
Most of the commentaries I read, however, made a big deal out of both points.
A common refrain was that not every virgin ended up being invited to the feast.
That is, we are not saved by an absence of sinful action.
Instead, that is the bare minimum.
We still require the Almighty's grace to be able to love truly.

In connection to the First Reading, the Gospel divides the virgins into the foolish and the wise.
Wisdom, as the commentaries said, is knowledge of the three Divine Persons.
It's said that to know G-d is to love Him, and I think that's an appropriate sentiment for this reading.

Daily Reflection:
\begin{itemize}
\item Did I write 1700 words for NaNoWriMo?
I did! I found that it was really helpful to center myself and take some time before writing to decide what I wanted to write today.
\item Did I write a chapter of Jeb?
Despite the fact that I wrote three hundred or so words of Jeb today, I think that from now on I'm going to take Sundays off.
There are a few reasons for this, most of which boil down to the fact that we are called to dedicate a day each week to the Lord.
Writing my web serial, for all that I enjoy it, is not a particularly prayerful action to me.
The priest today warned of spiritual sloth in his homily, and that stuck with me.
\item Did I blog? Two drafts! As yesterday, the second draft is much tighter than the first.
\item Did I stretch? Not yet. I will before bed, though.
\item Am I doing better at prayer than a rushed and thoughtless rosary? Honestly, I think that writing this was somewhat prayerful, if only because it forced me to reflect on what I heard at mass a little more.
There's a voice in my head that suggests that doing this reflection before Mass could be even more beneficial, but I also spent a few hours on it, so that may not be as doable.
\item Am I doing a good job writing letters to friends?
I wrote a letter, which is a good job in my books.
\end{itemize}


\section{Draft 1}
As with yesterday, I think I'm going to do this musing in two drafts.
I found that it was much easier to write the first draft knowing that I would be able to revise anything I said, and I felt like I was able to explore much better.
With that in mind, let's see where my mind takes me.

Today's readings, as is apparently always true at the end of a liturgical year, concern death and the afterlife.
For once, the second reading actually connects really well to the Gospel, in that both are incredibly oriented towards the end times and the Christian message of awaiting the next life.
The second reading definitely explains where the concept of the rapture comes from, at least to me.
If I read the line about the faithful being carried to heaven, I could absolutely see where people would think that is what happens, especially if I come from a tradition which rejects Tradition.

Anyways, one thing that the priest mentioned today's homily was a question I hadn't thought about when I listened to the readings today.
Why didn't the women waiting with oil share it with the ones who did not have enough to keep their lamps lit?
He brushed past the question, but it's been sticking with me since he mentioned it.

Of course, the answers I come to need to work both in the context of the parable and in the context of what the parable is implying.
I'm going to address only in the context of the parable for now, throwing out plausible ideas without exploring them, and then explore them in the context of the parable, and only then explore them in the context of the Gospel message.
I feel like there's a benefit in that approach, which is that it lets me get past the initial impulses I have much more quickly.

Ideas for what the maidens\footnote{I knew there was a word instead of virgins. I don't remember which verbiage (Idk if that's an appropriate usage for the term, but I like it, so will keep it in this draft at least) the translation we used at Mass had, but I'm willing to bet there's at least a few bibles with Imprimaturs or Imprimi Potests that have either word} who did not share their oil were thinking, in no particular order, and assuming nothing about the maidens' intent:
\begin{itemize}
\item They might not have enough oil if they share it
\item It's on the ones who forgot the oil, not them
\item Eliminate competition
\item Don't have spare money for oil
\item Were themselves running low once they had topped off their lamps
\item Different lamps?
\item The other maidens did not ask them to share?\footnote{as you might be able to tell, at this point I don't have the Gospel in front of me, so I can't say for certain exactly what is and isn't a valid reading of the text.
I'm pretty sure that the maidens whose light went out asked for oil, but I'm not completely sure, so this interpretation gets to stay}
\item Teach them a lesson about being prepared?
\end{itemize}

That's really as many ideas as I can think of right now.
I should read some commentaries to see what theologians have said, and I might spend some time doing that right now.\footnote{hmm or should I wait to do that until after I've exhausted my thoughts of how the metaphor works within the context of the parable?
Or, should I wait even longer and do it after I've connected to the Gospel message?
I think that I should do it at least after the exploration of the message within the story, so let's go through those now.}

Ok, let's see how each interpretation stands up to textual scrutiny, for all that I'm not going to read the passage in any sort of explicit context.\footnote{of course, I know that there's the context that this comes from Christ's sayings and in Matthew in particular, which means that it's targeted towards the Jewish people.
I also carry with me a lifetime of exposure to Catholic and general Christian ideology, which shapes how I view the world.}

\begin{itemize}
\item They might not have enough oil if they share it. That's supported, because they explicitly say \say{No, for there may not be enough for us and you. Go instead to the merchants and buy some for yourselves}.\footnote{Mt 25: (I don't know specific verse because my wifi won't let me access the bible right now. All I have is the email with today's readings, which tells me that it's somewhere between 1 and 13.)}
\item It's on the ones who forgot the oil, not them. I feel like the fact that they are explicitly called wise and foolish implies that this reading is at least somewhat supported.
However, the fact that they say no, not because of that, but because they do not have enough extra oil, casts that interpretation into a bit of doubt.
\item Eliminate competition. This feels too mean spirited, especially since there are no insults exchanged.
\item Don't have spare money for oil. This feels orthogonal to the overall message.
There's nothing to support the idea that the foolish would not have repaid the wise for the oil they burned when the night was over, but there is also nothing to support the idea that they were.
\item Were themselves running low once they had topped off their lamps. This is somehow a different interpretation than the first, which I suppose may be that if they topped up their lamps and had no extra oil, they would have had to dump the oil out? That seems like something that could be difficult to do.
Without being a scholar who knows things like how lamps work, though, I can't judge the validity of that take.
\item Different lamps? There's nothing to suggest that their lamps are anything but standardized.\footnote{there is, of course, the voice in my head which screams at me that the concept of standardized anything is incredibly anachronistic, but I've got a lot of practice ignoring those voices.}
Spoiler for the next section, but that also doesn't seem to mesh with the idea that the lamps are a stand-in for readiness for heaven.
I suppose there's a way to make that metaphor work, and I will try it, but I don't feel good about it.
\item The other maidens did not ask them to share? There is absolutely an interpretation that works where the fact that they demanded, rather than asked, \say{
The foolish ones said to the wise, \nsay{Give us some of your oil,
for our lamps are going out.}}\footnote{Matthew 25: also unknown in this draft}. That being said, I don't know if I like that interpretation.
The response from the wise maidens does not mention their demanding rather than asking, and I don't know whether that works with the broader message.
Still, it's among the better ideas I've had so far.
\item Teach them a lesson about being prepared? This one has some merit.
If we consider the fact that these virgins might serve this role for a number of bridegrooms, it would be in the foolishs' best interest to learn to keep oil.
If this were the first wedding they attended for, then it would be important that they do not make the mistake in the future.
If it is not their first wedding, then they should know better.
Ok, this interpretation also gets ranked pretty well.
\end{itemize}

So, after considering the way the metaphors could work, let's rank them.
I'm going to use a fairly absolute scale, calling them each plausible, probably, improbable, or wrong, completely based on my interpretation of the text of the parable as meant for itself.

\begin{itemize}
\item They might not have enough oil if they share it. Probable.
\item It's on the ones who forgot the oil, not them. Plausible.
\item Eliminate competition. Wrong.
\item Don't have spare money for oil. Improbable.
\item Were themselves running low once they had topped off their lamps. Plausible.
\item Different lamps? Wrong.
\item The other maidens did not ask them to share? Improbable.
\item Teach them a lesson about being prepared? Plausible.
\end{itemize}

Cool, now let's look at what each of these mean in the broader Gospel message, where the virgins awaiting the bridegroom are Christians awaiting the second coming.\footnote{I don't know for certain that this is the correct interpretation, but it's what I'm going to run with.}
Using that message, let's try reading each of the proposed ideas\footnote{even the wrong ones}.
What does the idea mean, in context of the parable's intended meaning?

\begin{itemize}
\item They might not have enough oil if they share it.
I feel like there are a number of ways to interpret oil and fire in this situation.
If we say oil is the willingness to wait for the coming of Christ, then it's a little hard to see how you could share it at all.
I could see an interpretation where there's an idea that you only have so much time and energy in a given day, or in your life as a whole.
At some point, helping others to find Christ could come in conflict with your own call to holiness, though I am struggling to come up with a particular example.
I suppose if you know that singing in the choir actively makes you disconnect from the Mass, then it might be better for you not to sing in the choir?
Maybe mass is a bad example, but I feel more comfortable with the claim that helping others may not always be best for your immortal salvation
\item It's on the ones who forgot the oil, not them.
I feel like there's not a good way to interpret this in the Gospel message.
We are constantly told that we are to love our neighbors as ourselves and to practice radical love.
I guess there's something in the like idea of how those in heaven must not be sad about those in hell, because definitionally there is not sorrow in heaven?
I know I read some non Catholic complaining about Aquinas's conclusions re: heaven and hell, but I don't remember enough about that or the Saint's own words to be confident in that take.
There is something to be said in the fact that earthly relationships are goods only in so far as they orient us towards the most vital relationship, that of us with the Lord.
If we use that lens, where helping others at the expense of paying what is due to the Lord is wrong, I can see the interpretation having some validity.
That is, those who forgot the oil are asking you to risk your salvation for their behalf.
The Church, so far as I know, does not say that you should ever put your own path to heaven in jeopardy for the sake of another.
Christ famously told someone to let the dead bury the dead,\footnote{for all that I've seen interpretations suggesting that his father was not yet dead and he was awaiting an inheritance} and that you should leave your mother and father and despise everything for the sake of the kingdom.
I think that this idea is worth sitting with a little longer, even if I think it's mostly good in how it connects to the first idea, rather than as a message in and of itself.
\item Eliminate competition.
There's really no way to sell this idea in context of the Catholic mission.
We are called to bring the entirety of Creation with us to eternity with the Father.
I think that's enough said on that.
\item Don't have spare money for oil.
Ok let's see how far we can strain a metaphor.
Oil is, on some level, connected with Faith and works.
I think implicitly, I've been thinking of the lamps as faith and the oil as works.
Faith without works, as we know, is dead.
Likewise, works without faith, while useful, do not lead to salvation on their own.
Even if that interpretation is wrong, I think it's a nice one.
It does lead to the whole \say{you are saved by performing some number of good deeds,} which is explicitly wrong, though does have some elements of working in the metaphor for sharing.
After all, Christ calls out people who point to a splinter in their friend's eyes to first take out the log in their own eye.
There's probably something to this, but I don't know if I'm at a point where I can make that work as a metaphor with my own level of spiritual development.
\item Were themselves running low once they had topped off their lamps.
Once again, unless the metaphor is that you have to figure out how to pour oil out of a lamp, I don't think this metaphor works very differently from the first example.
In the latter case, I guess there's something to be said for you can't retroactively pray on someone's behalf.
\item Different lamps? 
One of the Saints\footnote{saintesses?} said something about how we are not all called to be saints in the same way, even though we are all called to be saints.\footnote{I am absolutely mangling something profound and beautiful}
I guess there's also the quote from St. Paul about there being one Spirit but many manifestations, or whichever section discussed how a body requires more than just a finger or a toe.\footnote{that's probably the same section, as I think about it.
Probably worth having a bible with me if I'm going to keep referencing it in these reflections, which is probably a good thing for me to do}
\item The other maidens did not ask them to share?
I don't know about this interpretation.
Certainly, in the modern day, we are called to respect religious diversity to some extent.
I guess I don't know whether that extends to the point of not offering the faith to those who don't ask.
In the context of knowing that the maidens demanded, if not asked, it becomes a little more difficult.
Maybe there's something to be said in the fact that we cannot make others believe or even act as though they believe, much as we might wish to.
I don't know if there's anything there, but there might be.
The latter idea at least feels somewhat resonant.
\item Teach them a lesson about being prepared?
Hmm, I did immediately shy away from this take.
Then again, I know that I generally shy away from anything that suggests fraternal correction.
I guess there's absolutely a take to be made that you should encourage people to reform their lives up to and including at the moment of their death.
How that intersects with the Second Coming is a little more difficult, but I suppose that the metaphor still holds? Maybe?
The fact that this is so clearly about Christ's return makes the metaphor hard to connect.
After all, we know a posteri\footnote{I think that's the right Latin} that\footnote{to within any standard rounding error, and little t traditions of immortals that Christ raised during his earthly ministry aside} nobody was alive for both his First and Second Comings.
We don't have multiple chances to get it right.\footnote{I suppose an argument could be made that if we treat the bridegroom coming as the day that each of us individually dies, then there's the whole every day we get the chance to serve the Lord better, but that feels like a bit of a stretch}
\end{itemize}

Ok, so having now reflected a lot on the Gospel, let's look a bit at the other readings.\footnote{I feel like this reflection is lacking right now, but I can't quite think of why.
Ope, wait, that's wrong. I know two ways in which it's lacking. 1: I didn't read any commentaries, and 2: I didn't ever connect the parable to the Gospel message explicitly/pick an interpretation I like.
Let's do that instead of moving on}
Or actually, thanks to the footnotes, I know that there's still more to say about the Gospel.

Reading the commentary annotated bible that I have ready access to\footnote{the Catena app}, there are a few things that stand out:
\begin{itemize}
\item Hilary of Poitiers points out the connection to the Decalogue\footnote{which I don' quite understand}. He does expound on what the lamps and oil represent.
To Bishop and Doctor Hilary of Poiteirs, the oil i the fruit of good works and the vessels are the bodies of the faithful.
It's an interesting reflection, and one that I find kind of compelling.
He ties the sellers of oil to the poor, though I don't quite understand how.
\item Augustine says that the number five relates to the five senses.\footnote{I'm beginning to realize I did absolutely no numerology, but that was probably relevant, since they didn't just say a number, they specified five.}
In Augustine's view, the term virgin is meant to refer to the fact that they have refrained from sin through any of their senses.
However, absence of active sin is not enough to earn heaven.
He ties the lamps to good works as well, quoting \say{Let your works shine before men, that they may see your good works and glorify your Father who is in heaven,} as well as \say{let your loins be girded and your lamps burning.}
Oil to him, represents the fact that we have all good through the Lord, who is Good.
We cannot make olives to turn into oil, and likewise, we cannot create goodness.
Instead, we can reflect and receive the goodness that the Lord gives us.

Augustine also speaks about the bridegroom's delay.
To him, the delay is a form of test, showing that their love continues to preserve and glow to the end.
He speaks to the wise and foolish both sleeping by referring to sleep in the same way that St. Paul does in the second reading, where it acts as a euphemism for death.

In the rebuke, he sees that they are wise not because they belong anything, but because the wisdom of the Lord is with them.
He again connects the oil to works, saying that the oil of the foolish is praise and adulation from man.
\item John Chrysostom connects this parable to the one preceding it, about the faithful servant.
He points to the difference, where rather than punish, he simply refrains from help.
He also points to the poor as being the ones who sell, which still doesn't make sense to me.
There's I think something about how we should not take advantage of the poor, but I don't quite get the connection, which does admittedly say more about me than the great thinkers of the Church.
St. John Chrysostom points to the fact that they are virgins as a way of showing how worthless everything is without works.\footnote{hey nice, I overlapped slightly}
Even though virginity is good, it is only good insomuch as it is accompanied by works of mercy.
Oil once again denotes good works, while the lamps denote gifts of virginity.
That's an interesting framing, and one that I had not considered.
\item Cornelius a Lapide\footnote{a seventeenth century Jesuit, as it turns out,} says similarly that the virgins represent the Church. He connects it somehow to the fact that there was a king who fed the poor, and when told that was beneath his dignity, pointed to Christ saying what we do to the least we do to Him.
Ooh, he points to ten as being a symbol of totality.
Apparently Jerome and Hilary take the virgins to be all of humanity, the wise as Christians and the foolish as everyone else.
He also points to the virgins sleeping as dying.
Apparently there's an old tradition of not leaving the Easter Vigil before midnight because it's believed that A: the plague of firstborns happened at midnight, and B: Christ would come again at midnight.
This commentary seems much less directly relating to the parable and much more focused on the Gospel as a whole, which makes interpreting it hard.
\end{itemize}

Ok so that was interesting and informative.
I should absolutely spend more time reading commentary from Doctors of the Church, because much of what they said just instantly resonated within me.
Let's see what takeaways I have from my notes.

\begin{itemize}
\item Numerology apparently is something I need to consider.
\item the oil absolutely relates to good works, though how exactly, there's some disagreement
\item the fact that they are called virgins is probably meaningful outside of the Jewish tradition.
\item sleep means die
\end{itemize}

Alright! That feels like a good place to end my thoughts on the Gospel for this draft.
In revising, I'm certain that I'll have to tie all of this together in a less rambly fashion\footnote{under 3500 words shouldn't be too hard. In fact, I feel like I'd be hard pressed to be as or more rambly}, but for now I think I should move on.

Let's look at the first reading.
Oh gosh, it's all about wisdom.
Since the Gospel is all about how there are wise and foolish virgins, there is absolutely something related in those.
Still, my thoughts at the time of reading it were fairly simple.
Wisdom comes to those who seek it\footnote{her?} and is something you grow in, not something you innately possess.

Rather than spending tons of time\footnote{can you tell that I'm getting tired of typing?} trying to come up with my own interpretations, let's see what the commentaries have to say.

\begin{itemize}
\item Pseudo-Augustine\footnote{wow what a fun name. ah it's not a person, but writings incorrectly attributed to Augustine} says that we can love better by seeking wisdom.
That seems reasonable.
\item Augustine ties the fact that we cannot know the Father without knowing the Son, and so too with not knowing.
It's an interesting idea to tie to the Gospel, and I'll need to digest it at least a little.
\item St. Cyril ties wisdom directly to the Holy Spirit, which makes a lot of sense to me.
It's also probably worth digesting that in context as well.
\end{itemize}

Though I suppose that there's some value in reflecting on the Psalm, I don't know if I really feel like I want to.
It seems a little ambitious to try to interpret the Pslams when I didn't even think to investigate the listed numbers in the reading today.
The second reading feels like it should be pretty straightforward, but I'm not sure if it will be.
My immediate interpretation is just that death is not the end, which is a pretty easy Catholic take.
It connects pretty clearly to the Gospel, since the early Church writers connect sleep to death in the parable.

Interesting points from the commentaries include:

\begin{itemize}
\item Tertullian points that Christ will come back bodily, which I suppose I've never really thought about that much.
I guess that I do really see the second coming more in an intellectual than any emotive sense.
\item John Chrysostom points out that Paul is directing this letter to the mourning as a way to get hope.
\item Augustine says that Paul saying we should not be sad as the heathens are is not saying we should not be sad.
Instead, like most priests say at funeral masses in my experience, we are to mourn the loss of relationship, rather than the person's cessation.
\item Ooh, Carsarius of Arles ties this to the Gospel.
\item Ambrose of Milan points out that we are raised in the proper order and to the proper state.
\item Bede points out that even Moses was buried. By connecting that to the Transfiguration of the Lord, Bede points out that we already have further hope of being raised\footnote{I think, I'm bad at reading comprehension}
\end{itemize}

Anyways, this feels like a good place to finish this first draft.
I have some ideas floating around for structuring the second one, but we'll see how I end up feeling when it comes time to write it.

Daily Reflection:
\begin{itemize}
\item Did I write 1700 words for NaNoWriMo?
I did! Having the notes for where the plot should go and then taking a few minutes to expound on them really did a lot to help me figure out what to write.
\item Did I write a chapter of Jeb?
There's a voice in my head that tells me I should take Sundays off from writing the book.
I don't know if I disagree with it, and honestly, I think that I generally agree with it.
That in mind, I'm going to say the 400 words I wrote for Jeb today are enough.\footnote{we'll ignore the fact that my daily blog post is shaping up to be well over five thousand words as a reason to not do Jeb}
\item Did I blog? Did I ever! I think that this is almost certainly my longest post, and I haven't even gotten to the second draft.
\item Did I stretch? As always, not yet, but I feel like I should.
Probably sometime between now and when I revise this post would be the ideal time.
\item Am I doing better at prayer than a rushed and thoughtless rosary? I don't think last night's rosary was all that rushed, and I tried extra hard to be mindful of my prayer before, during, and after Mass today.
\item Am I doing a good job writing letters to friends?
With my shiny new ink, I wrote a letter! I would like to write another, and probably should.\footnote{should in the I think I would be happier if I did, not in the I'll judge myself if I don't way (there's context but I feel like it's pretty obvious)}
\end{itemize}

\end{document}