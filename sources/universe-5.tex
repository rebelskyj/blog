\documentclass[12pt]{article}[titlepage]
\newcommand{\say}[1]{``#1''}
\newcommand{\nsay}[1]{`#1'}
\usepackage{endnotes}
\newcommand{\1}{\={a}}
\newcommand{\2}{\={e}}
\newcommand{\3}{\={\i}}
\newcommand{\4}{\=o}
\newcommand{\5}{\=u}
\newcommand{\6}{\={A}}
\newcommand{\B}{\backslash{}}
\renewcommand{\,}{\textsuperscript{,}}
\usepackage{setspace}
\usepackage{tipa}
\usepackage{hyperref}
\begin{document}
\doublespacing
\section{\href{universe-5.html}{Universe in the Park Part 5}}
First Published: 2023 September 4
\section{Draft 1}
Well, first off, sorry for anyone who tried to read \href{universe-4}{my last post}.
I apparently completely forgot how to do formatting.
I don't have the spoons\footnote{i wrote a blog about this concept at some point, but I'm not going to look for it now (funnily enough)} to fix it right now, but I will try my hardest to remember to do so in the future.

Unlike my last post, this reflection is only on a single Universe in the Park\footnote{ok so I did use the explicit name of the program. Good to know (re: yesterday's footnotes}, rather than five.
I think I said this yesterday, but it was not only my final talk of the summer, it was also the final talk of this program for the summer.
It had good attendance, for all that the crowd wasn't as great as I would've loved.
Most of the first few questions were just polite ways of asking \say{right, but like, what's the point?}
Thankfully, there were three children there.\footnote{ok, realistically, there were probably much more than three children, but only three are relevant because they did what children should do}
I can always rely on children to ask many and interesting\footnote{if sometimes really difficult and philosophical} questions.

Some highlights from last night include:
\begin{itemize}
\item Do the stars we see at night move?\footnote{yes, and also we move which makes them look like they're moving, both over the course of a night and over the course of a year and over the course of millennia}
\item Are there rocky planets other than the moon?\footnote{yes? technically the moon isn't a planet, though I do think (I didn't say this, of course, correcting children needlessly in their questions is mean) it is large enough to be considered one if it wasn't, you know, orbiting the earth instead of the sun directly}
\item Two versions of \say{what color are planets?}
\end{itemize}

As always, after the formal talk portion of the night, I tried to set up my telescope.
It was looking cloudy beforehand, so I did not have any of the prep work done, but I thankfully was able to make quick work of it.\footnote{shocking how setting up a telescope repeatedly will make you faster at it.}
The moon was large and beautiful\footnote{and incredibly red, which I said was due to dust in the atmosphere, since it was at the horizon.
Even if that wasn't true, it feels emotionally true}, so I started pointing at it.
Someone told me that Saturn was apparently visible, so I hesitantly pointed the telescope there next.

Despite the fact that Saturn was partially hidden behind a cloud, I think it was the coolest thing I've pointed at this summer.
The moon is great, but it mostly just looks like a higher definition image of the moon.
Most stars just look like brighter dots on the sky.
The occasional double is cool, but it's just two close by dots.
Saturn, on the other hand, had resolvable disks!\footnote{it was absolutely tiny on the scope, but people could still see them.
It was wild to me}

After that, I used my trusty star finding app to point me to a nice double given what part of the sky I could see.\footnote{we were behind the nature center, and so big dipper (Mizar I want to say is the double in Ursa Major) was invisible. Sadly, that does mean that I was unable to make my dumb bear joke (if you want to know, just ask)}
There was a nice double, and people seemed very\footnote{reasonably} excited to see it.
After that, there were just a few stragglers, so I showed them Polaris\footnote{at their request} and then not so subtly suggested that they should go home.

It did not feel like a good talk, but I got tons of compliments afterwards.
In retrospect, I was in a really bad head space, and while I don't think it negatively affected my talk too terribly much, I think it absolutely crushed my idea of how well the talk went.
It was my tenth of the season and my I think twelfth overall.\footnote{thirteenth? I should check through my notes.}
There are, as it turns out, sixty eight state parks in this state.
Given the overlap and the fact that my first talk of the summer wasn't at a state park\footnote{and it doesn't show up anywhere in the parks and etc booklet I was given}, I have slim hopes of making it to all of them for a talk before I leave graduate school.
Still, I can try my hardest I suppose.

The park as a whole was pretty beautiful.
It was on one of the nice lakes\footnote{some might even call them great}, and it had dunes.
I didn't get in until lateish, and so I didn't want to go swimming.\footnote{that's a lie, of course. I always want to go swimming.
I just didn't want the consequences of swimming, and that was enough to stop me}
I then left early in the morning because I had a fairly packed schedule that day.

Anyways, I had a lovely season of these talks, and I'm grateful for the chance to have done them.\footnote{as it turns out, the program is so grateful to me that I've now been placed on the steering committee.
I guess that's what they say happens.
Rewards for a job well done and all}

\begin{itemize}
\item I made a little progress, in that I learned how to record an animation in blender.
\item I did not fight in the least against the entropy in my home.
\item I suppose I'm better, in that I did it again today.
\item I stretched a little while lifting, but only in so far as a low weight lift is a stretch.
\item I went lifting with a friend!
\item I did not prioritize sleep. Instead, I went to a friend's party.
\item I woke like three hours earlier than I'd hoped. That is not a positive.
\item I am very tired, and so only spent five or so.
It was still a great use of time, and I will make sure to do more tomorrow, especially since I can do it on the walk to work instead of on the way home from the gym
\item I wrote about four hundred words, which is a little less than I need if I write that much every day.
That being said, I also have three days to write that chapter, so.
\item I plotted out\footnote{loosely. There are one sentence descriptions for about half the chapters, so I make sure that i hit every plot point I want to.
There are also thirty chapter gaps where I'll just write where the muse takes me} the rest of the book.
\item I did not write poetry.
\item I jammed with a friend and shared the new song with him, so sure.
I also realized that my installation of Musescore had broken, so I fixed that.
\item I wrote five things I like about myself.
\item I wrote three things I was excited to have done today.\footnote{because accidentally did at end of day not beginning}
\item I wrote ten things I'm grateful for.
\item I did not generally cultivate joy.
\end{itemize}

\end{document}