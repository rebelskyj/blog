\documentclass[12pt]{article}[titlepage]
\newcommand{\say}[1]{``#1''}
\newcommand{\nsay}[1]{`#1'}
\usepackage{endnotes}
\newcommand{\1}{\={a}}
\newcommand{\2}{\={e}}
\newcommand{\3}{\={\i}}
\newcommand{\4}{\=o}
\newcommand{\5}{\=u}
\newcommand{\6}{\={A}}
\newcommand{\B}{\backslash{}}
\renewcommand{\,}{\textsuperscript{,}}
\usepackage{setspace}
\usepackage{tipa}
\usepackage{hyperref}
\begin{document}
\doublespacing
\section{\href{ms-holmes-ms-watson.html}{Ms. Holmes and Ms. Watson Review}}
First Published: 2023 August 24
\section{Draft 1}
A few weeks ago now\footnote{because wow it turns out that one reason I might stop posting is because I become suddenly busy} I did something that I haven't done in a while.\footnote{though, given the infinite number of potential permutations of activities I can do, that's not the most meaningful statement.}
I went to go see a play.
The university I attend\footnote{which is such an awkward construction} was putting on a production of Ms. Holmes and Ms. Watson.


Now, if you're anything like me, then your first instinct will be to complain that it's \say{Ms.} Watson, not \say{Dr.} Watson.\footnote{of course, that's presupposing a lot about the (allegedly extant) readership of my blog, I suppose}
I had that complaint going in, especially since the play was described as a feminist retelling.\footnote{not to stereotype, but as one might expect by the gender swapping in the title.}
I had\footnote{have?} recently subscribed to a SubStack\footnote{newsletter app, more or less} called \say{Letters from Watson,} which just sends out serialized versions of the Sherlock Holmes short stories.
It's really interesting reading them, especially as someone who reads\footnote{and writes} a fair amount of modern serialized fiction.
The genre conventions have certainly shifted in the intervening years.
However, since this post is a review of the show, rather than the fiction, I digress.


When I arrived, the play promised many of the common staples of modern small work plays.
It said that there would be forced audience participation\footnote{though it phrased it differently,} and that it was a loving but disrespectful\footnote{again, me paraphrasing, I don't remember the actual verbiage} retelling of Arthur Conan Doyle's work.
It was fine.


I'm not sure whether I was just hungry and so not as into the play as normal, but the play lacked a lot of flow to me.
When I go to a play that I really enjoy, at least for a moment I forget that what I am watching is a play.
At no point during the show did I forget that I was sitting in a crowd, watching a performance.
I suppose the fourth wall breaks did add to that feeling.\footnote{this is not the place for me to go into a rant about the way that I dislike modern (popular) media's tendency (tendencies? media is technically inherently plural) to avoid any sort of genuine emotional appeal by hiding it behind sarcasm and smarm.
There will be a place for it sometime soon, but not today}
Still, it was an enjoyable show, and I would happily see it again.


\begin{itemize}
\item One of my research group mates is also planning to learn how to use Blender, so now I have an accountability buddy for it.
I also think it might be worth finding some set of online tutorials and starting them sooner than later.

\item It is beyond hot, which meant that I have been trying to minimize moving as much as possible. 

As a direct result, that means that I was not making those big strides.

\item I'm now four for four\footnote{which is shockingly hard for me to type. I had to erase four four four at least once (and then again when trying to type what my mistake was, I put for in the wrong place as well} posts, for all that I did not remember to post the actual post from yesterday until this morning.
I suppose that there's not much of a way for future readers to see that, though if you were on the post last night, I apologize.

\item As a direct result of thinking about math and coding for my entire day yesterday, I did not write too much. According to my logs, I wrote a little under four hundred words of Jeb today. I need to write more, since I'm currently typing tomorrow's chapter.
I'm not sure where I'm trying to go with the story at this exact moment, which I think is part of the issue.
My nominal plans for the book say that the next 85 chapters will cover a few\footnote{though indeterminate} amount of years in the character's life.
I suppose that sitting and blocking out what I want to happen could be a good idea.

2.3/4+
\item Last night I left my laptop at work.
Probably as a direct result, I wrote an entire song!\footnote{two verses, a bridge, and a chorus.
It's probably only like two or three minutes long, but that's fine.}

\item I did not journal. That's probably fine.

\item I forgot to stretch this morning.
Or, equally fairly, it was very hot and I wanted to not do anything.

\item I fell asleep at a normal time last night, which was nice. I also woke up before my goal of 6 am today, which was absolutely fantastic.

\item I prayed a rosary last night before bed and then listened along to a rosary podcast that a friend recommended on my commute to work today.

\end{itemize}


\end{document}