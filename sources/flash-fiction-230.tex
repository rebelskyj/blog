\documentclass[12pt]{article}[titlepage]
\newcommand{\say}[1]{``#1''}
\newcommand{\nsay}[1]{`#1'}
\usepackage{endnotes}
\newcommand{\1}{\={a}}
\newcommand{\2}{\={e}}
\newcommand{\3}{\={\i}}
\newcommand{\4}{\=o}
\newcommand{\5}{\=u}
\newcommand{\6}{\={A}}
\newcommand{\B}{\backslash{}}
\renewcommand{\,}{\textsuperscript{,}}
\usepackage{setspace}
\usepackage{tipa}
\usepackage{hyperref}
\begin{document}
\doublespacing
\section{\href{flash-fiction-230.html}{Flash Fiction Friday}}
First Published: 2023 December 8

\section{Draft 1}
Another Friday means another FFF!
This week's theme is \say{Fool me once,} and I have so many different ways to interpret that prompt.
Most of the ways that I can consider this phrase being used are in a negative context.

Even the positive contexts tend to still have an edge of lying to them, as you might expect from the fact that there's absolutely an implication of misleading.
As someone who enjoys a good verbal puzzle, though, there should be a way to make this work.
I think that I'm going to try for fiction again, and this time I'm going to write drafts of it until I run out of time this morning writing session.\footnote{I reserve the right to work on the project more than this amount, but I will absolutely spend at least the next forty minutes working}

Alright, time to do a retrospective on the whole project.
It was really interesting the way that my mind took the question.
At first, I kept wanting to make the prose explicitly poetic, which was really not the goal.

I iterated through a number of ways to consider tricking as a positive.
At first, I took it as like the simple sleights of hand that you can do to amuse a child.
Somehow, though, none of those felt right.
Maybe it's because they were first person, and maybe it's because I just didn't like them.

I left small reflections to myself after each attempt failed.
One still sticks out to me, \say{I'm thinking something about how joy is an illusion I choose to believe in.}
That really resonated as what I wanted to write about today.

From there, it was a bunch of iterations to pick the main character\footnote{an unnamed he}, the premise\footnote{discussed later}, and the tense.
I ended up choosing present tense, which I think gives the fiction something of a dreamy feel.
There's something really strange about writing in the present tense.
Normally, everything happens in the past, and the narrative records that.
In present tense, though, each action is passing by as you read it.\footnote{and they call it meta writing}

An unnamed and undescribed\footnote{Shoot! I need to work on physical cues. Hold on, going to revise the story one more time} man is sad and walking down the street.
He encounters a stranger, who decides that they know each other.
Eventually,\footnote{like 50 words later. This fiction is not long} the lie is revealed, and the two continue speaking.
As the story ends, each of the physical descriptors from the opening lines are repeated, but the emotional connections have changed.

Rather than trudging\footnote{which wow what a fantastic word} through slush and grey ice, he walks through fallen snow.
I think that I like it, for all that it's still a little more of a vignette than I might want.
It's only 360 or so words, so I do technically have plenty of space if I want to change it, but I don't think that I want to add anything more.
I've rewritten the ending, and now it feels fine.

Honestly, I think that one issue I have is with flash fiction going up to a thousand words.
This musing is currently about five hundred.
A story that has twice as many words as this feels like it's a markedly different project, for all that I should probably try to get back into writing longer flash fictions.
I've now posted this story, so there's nothing I can really do to change it.
It feels good to put my writing out, for all that I have now been falling behind on Jeb.

Daily Reflection:
\begin{itemize}
\item Hobbies:
\begin{itemize}
\item Did I embroider today? Yesterday I finished a project, which was nice. Today I took the day off, and I just realized I forgot the supplies at work, so will probably take the weekend off as well.
\item Did I play guitar today? I did! I even retuned it back to standard tuning.
\item Did I practice touch typing today? I did a single lesson. It's nice that I'm slowly learning, for all that it still takes a fair amount of effort to use the right finger al the time, and I still need to work on it.
\end{itemize}
\item Reading
\begin{itemize}
\item Have I made progress on my Currently Reading Shelf? I gave up on one book and started the next one, which is a good choice.
\item Did I read the book on craft? It's laying\footnote{lying? I always forget which is which} open next to me, but I don't think I'll get to it today.
\item Have I read the library books? Not for a few days, no.
\end{itemize}
\item Writing
\begin{itemize}
\item Did I write a sonnet? Wrote one yesterday, it was a lot.\footnote{in that it was incredibly emotionally charged and I felt a deep connection to it}
\item Did I revise a sonnet? Revised the one from yesterday more than a few times.
\item Did I blog? So yesterday's blog does exist. It was a very emotionally charged piece, and as I've said since the first musing, for all that I treat this in theory as a diary, I do recognize very much that there are limits to what I can feel comfortable posting. I shared the thoughts with a friend, who agreed.
\item Did I write ahead on Jeb? I was advised to take from Christmas to New Years off from posting, which is probably healthy for em.
\item Letter to friends? Nope, turns out a letter got returned though.
\item Paper? Not as much as I'd like.
\end{itemize}
\item Wellness
\begin{itemize}
\item How well did I pray? Much better, honestly. I made it through a rosary last night, made time this morning to go to a bit of Adoration, and made it to Mass early enough this evening\footnote{Happy Solemnity of the Immaculate Conception by the way} that I had plenty of time for prayer.
\item Did I clean my space? A little.
\item Did I spend my time well? Today flew past me so incredibly quickly.
\item Did I stretch? Nope.
\item Did I exercise? About to.
\item Water? Finally remembering to, at least a little, which is nice.
\end{itemize}
\end{itemize}
\end{document}