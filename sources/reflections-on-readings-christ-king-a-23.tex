\documentclass[12pt]{article}[titlepage]
\newcommand{\say}[1]{``#1''}
\newcommand{\nsay}[1]{`#1'}
\usepackage{endnotes}
\newcommand{\1}{\={a}}
\newcommand{\2}{\={e}}
\newcommand{\3}{\={\i}}
\newcommand{\4}{\=o}
\newcommand{\5}{\=u}
\newcommand{\6}{\={A}}
\newcommand{\B}{\backslash{}}
\renewcommand{\,}{\textsuperscript{,}}
\usepackage{setspace}
\usepackage{tipa}
\usepackage{hyperref}
\begin{document}
\doublespacing
\section{\href{reflections-on-readings-christ-king-a-23.html}{Reflections on Today's Gospel}}
First Published: 2023 November 26

\section{Draft 1}
Wow! The end of another liturgical year.
Despite the fact that the Church more or less created the calendar we use\footnote{it isn't the Gregorian Calendar for no reason. One of the Popes Gregory (should it be Pope Gregories? I never quite know what to do with anglicized titles) created it, or at least mandated its usage. It happened after 1050, which we know because the Orthodox Church doesn't recognize it (honestly, the fact that I can use random facts to start dating things relative to each other is kind of fun)} and absolutely created the liturgical calendar we use, the two do not line up.
There is probably some legitimate reason for that.
Off the top of my head, the fact that Christmas is not on January 1st, despite arguably being the most important winter day in the Modern Church\footnote{ok, so pre Vatican 2, the Solemnity of Mary, Mother of G-d (January 1st) was celebrated as the Feast of the Circumcision (which I have always loved as a title, for all that it icks a lot of modern listeners out (for good reason)), and that was (as best as I understand Second Temple practice) the day that Christ would have been officially named and brought into the Church. It isn't seen that way anymore, and that is probably not a fight the Vatican wanted to start}, is already enough of a reason to know that the two would diverge.
It does go further, though.
The modern liturgical calendar begins with Advent, which begins on the first Sunday of November.\footnote{I'm sure there's edge cases, because there always are with simple rules.
However, the edge cases were already dealt with, because we can, with total certainty, predict what day will occur when arbitrarily far into the future. There's probably something to add about a post I saw recently talking about how modern society, having now given up belief in the Almighty, is almost required to fall into nihilism. Having just now watched a video from a science channel I really like exactly on how the universe will be mostly dead in a fraction of a fraction of the time that it will take to totally die, I can't say that I have any real argument against that logic. I want to think about it later, but I'm not sure that now is the perfect time.}

I can probably think of some benefits of doing it this way.
One obvious one is that it gives us, at minimum, two days a year that we can call the first of the year.
For all that each day is special and perfect in its own way, there's something intrinsically human about the desire to treat continuity as discrete.
Hearing that it's the end of the year spurs us to consider the next year and to think about how we might change for the better.\footnote{Or at least it does for me.
The fact that New Year's Resolutions are a whole thing implies that it's at least a strong cultural cue, even if it's not innate}

Regardless of how important the fact that each year has ended in a new beginning, and will continue to do so until time itself ends, the fact that a year ends is equally important.
It reminds us of other endings, most pressing of which is the ending of our life.

The Feast of Christ the King is the last Sunday of the Liturgical Year.
Obviously,\footnote{well, obviously given the segue I used to get us here} it concerns the end of time and the end of our own lives.
It contains some readings that I've always struggled with, though differently as I've grown older.\footnote{the fact that I feel like most weeks I start the reflection with a comment similar to this about struggling probably implies something.
Whether it's dissatisfaction with the way the world is, lack of depth of understanding, or disagreement with the Church, I haven't quite figured out.
As it is, I think it's good for me not to be complacent or comfortable.
Growth only happens when you're neither}

The first reading comes from Ezekiel.
In it, we are told that we will be separated into rams and goats.
Of course, I know that this does not mean that we are predestined, as is a common enough heresy in the modern age.
The fact remains, though, that we will be judged.
We are reminded constantly that only in accepting our own limitations and weaknesses can we truly be strong.
The reading emphasizes this, saying: \say{The lost I will seek out, the strayed I will bring back, the injured I will bind up, the sick I will heal, but the sleek and the strong I will destroy, shepherding them rightly.}\footnote{Ezekiel 34:16}

The second reading comes, as they tend to, from Paul.
Here we are reminded that all will be raised on the last day, beginning with Christ.
For all that it's very easy and settled theology now, I'm more and more appreciating the fact that he makes incredibly bold and outlandish claims by the standards of how the world is otherwise known to work.

Finally, we get to the Gospel.
This Gospel passage is quite possibly the easiest passage to understand what we are called to do, for all that it is also one of the hardest to accomplish.
In no uncertain terms, we are told that every time we see someone hungry, thirsty, a stranger, ill, or naked\footnote{which I think almost entirely overlaps the corporal works of mercy, may discuss that in full text} we see Christ.
We can either choose to ignore Him, or we can choose to help Him.

Of course, there is clearly more to the Gospel today than just that message.
If there wasn't, then there would be no way that the Church could justify letting the hungry starve, the prisoner live neglected, the ill die on the streets, or immigrants be torn from their families.
Where we are able to draw the distinction, I do not know.

However, as I write today's musing, I am in my heated home, wearing warm clothes and a soft blanket.
I know that there are far too many people right now who are not inside, instead huddling for warmth in the bitter winter air.
Knowing that each of them is as Christ to me makes me feel like I need to do more.
When I look at what's happening in the world right now, it seems as though no one has heard this Gospel.

A common objection to giving money to the poor is that they might waste it on something frivolous.
As a person who believes very strongly in medicine, the idea that someone with a mental illness might self medicate with alcohol doesn't really bother me.
After all, I have plenty of friends who use different substances to self medicate, and I don't judge them for it.
Why would I judge someone living an objectively harder life?

I'm late enough in writing this musing that I don't really have the energy to see what the great thinkers of the Church have said about this passage, but I would be curious.
Maybe I'll look another day.

Daily Reflection:
\begin{itemize}
\item Did I write 1700 words for NaNoWriMo? It's far too late for me to start, but start I must. Yesterday I was in a similar boat, which is why the musing didn't come out.
\item Did I write a chapter of Jeb? I said I wasn't going to work on it on Sundays, and so I shan't.
\item Did I blog? I think this counts, for all that I chose actively not to yesterday.
\item Did I stretch? Nope! I'm going lifting early tomorrow morning, though, so maybe that will count.
\item Am I doing better at prayer than a rushed and thoughtless rosary? I'll try to do a rosary tonight, but very well may fall asleep during it.
\item Am I doing a good job writing letters to friends? I met with friends yesterday and called a friend for like 2 hours today. Even though it's not a letter, it serves much of the same purpose, which is staying in touch with the people I care about and love.
I also got to see some other friends during catechesis tonight, which is always nice.
\end{itemize}
\end{document}