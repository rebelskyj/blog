\documentclass[12pt]{article}  
\newcommand{\say}[1]{``#1''}  
\newcommand{\nsay}[1]{`#1'}  
\usepackage{endnotes}  
\newcommand{\B}{\backslash{}}  
\renewcommand{\,}{\textsuperscript{,}}  
\usepackage{setspace}   
\usepackage{tipa}  
\usepackage{hyperref}  
\begin{document}  
\doublespacing  
\section{\href{bagels-25.html}{On Another Year of Thanksgiving Bagels}}  
First Published: 2025 November 28

\section{Draft 1: 28 November 2025}

Another Thanksgiving has come and gone.  
For the first time I can remember, we did Thanksgiving not at home.  
Except, that's not true, because I do also have a firm memory of going to Thanksgiving at a friend's house sometime when I was the same height as the chef.\footnote{who's about my current height}  
It's possible that I've gotten that holiday mix in my memories with some other seasonal celebration, it's possible it was one of the years near COVID being in the public consciousness, and it's possible that I'm remembering back to high school, which is further than I can remember.\footnote{for the purpose of this post, at least}

This year for Thanksgiving, my family decided that we would not be making our own dinner.  
I'm generally not opposed to the concept, because very little about Thanksgiving that's important to me is also related to cooking.  
Part of me is a little sad that I don't get to enjoy slop\footnote{my family's affectionate name for the way we consume leftovers, which is to just put some turkey, dressing (or stuffing, if you're a strange kind of pedant), green bean casserole, gravy, and potato (or whatever other ingredients are calling out at the moment) into a pan and stir around over heat until at a temperature desired}, but that's a fairly minimal part of me.  
Not having to make the meal\footnote{which for whatever reason was really my job the past few years}, and especially not needing to clean up the kitchen after making a large meal is really nice.

It was also so incredibly nice to not have to worry about timing when it came to making bagels.

For whatever reason, this year I decided that I'd like to make exactly a gross of bagels.  
Unlike previous years, I didn't have to finish cooking before some arbitrary deadline, which meant that I was finally able to do the bagel making in what, in my mind, at least, felt more photographable.  
So, I slept in a little\footnote{read: didn't wake up for first rise until 5 am, as opposed to my historic (three??) am}, and then broke the large dough into twelve smaller balls, each of which was then shaped into a dozen small balls, with mixins as appropriate.  
Once I had a beautiful\footnote{and impossible for me to count, apparently. I thought I had twenty four cranberry bagels and actually had 26. Not at all sure why I was completely unable to count} 144 balls of dough, I punched a hole in each and started to boil and bake them.  
Rather than trying to shove as many pre-boils into the pot at once as I could, my brother pointed out that doing them by the dozen wouldn't actually end up costing any more time.

So, 144 bagels went into the oven and came out delicious and brown.

Unlike every year before, I also\footnote{cruelly, if you are to ask my brother} did not let people start eating bagels until I had a photograph of the entire batch finished.  
I've never before seen all the bagels at any stage, in large part because of the historic process.

Back in the before times, if I was making a large number of bagels, I would separate the dough that was getting whatever form of mix-ins, shape balls and bagels, and then immediately boil and put into the oven.  
The overall timing ended up meaning that I was more or less constantly moving between putting bagels in water, shaping them, putting them in the oven, taking them out of the oven, and mixing new flavors into the base dough.  
It's definitely a faster process, probably saving more than an hour of overall time.  
However, it's also much more stressful of a process, and it's significantly less photogenic.

While making my variant of a poolish\footnote{preferment?} the night before, my brother commented that it's \say{terrifying} the way that I actively avoid using measurements.  
Although this is not untrue, I feel like, especially for bread, it's an unfair expectation.  
The general considerations that come into making a good bread are the relative moisture content and the protein content of the flour.

Flour can vary by such a large margin in its protein content, and I'm honestly not sure how much I trust the companies when they make claims about the protein content of their flour.  
After all, a difference of one to two percent is around what I remember the FDA regulations control, and is often also the difference between bread flour and AP flour.

Even more importantly, though, the relative water content of the flour can vary by such large amounts, primarily from the ambient humidity.\footnote{I think}

If measuring by weight, as so many people recommend for some reason, then the higher moisture flour will mislead about how much water needs to be added.  
After all, if 100 grams of dry flour is instead reading as 110 from the moisture it absorbed, then a mass or weight based measure will imply not just that the ten g of water need to be added, but also the water to make up the 10g.  
If using a standard hydration of 100 percent\footnote{a relatively normal one, as I understand it}, that means that the hydration will, instead of being 110 g water and flour, as desired, 120 g water and 100 g flour, a twenty percent difference of hydration!

So, with that information as justification, I made my bagels the way I make most of my breads: by heart.

This year, rather than using potato starch,\footnote{as is the norm, now that we've moved from using potato water (the water from boiling potato)} I decided to try adding some dark rye flour to the dough.  
In general, I love the way that rye tastes, and I adore the depth of flavor it adds to a basic flour loaf.  
Given the response from people who ate this year's batch, I'm not alone in the judgement.

My brother did comment on the fact that the raw dough had some strange looking flecks in it, which were quickly explained away as rye.

So, despite not measuring any of the ingredients with anything but my heart, I think that the final dough had about twenty pounds of flour, six ounces of gluten, and a pound of dark rye.  
How much water?

fantastic question.

Once the dough was able to rise, I divided it carefully into what appeared to be a dozen evenly sized balls.  
As luck would have it, they were not evenly sized, but that's the nature of life sometimes.  
Each of those balls\footnote{being totally honest, usually in combination with another ball or two} was then divided into a dozen balls, which I tried to have be about evenly sized.  
Once the 144 balls were shaped and\footnote{as needed} flavored, I punched a hole in each and then started to boil.

I did find it somewhat humorous that I was so carefully breaking the dough into even balls, despite not measuring what went into the dough.  
My father commented on the fact that I tore off chunks of dough at a time, rather than subdividing the ball of dough in quarters and then thirds.\footnote{I think he suggested thirds and quarters, but the effect is the same}  
Realistically, I generally have tried that approach before.  
For whatever reason, it ends up working out much worse for me.

I think it's because I internally decide that if I'm going to be subdividing multiple times, I might as well just go for prime factorization.  
I'd divide the dough in twain\footnote{mmmm old word} and then half again, then split each of those into thirds.  
When considering how much correction I tend to need to do, I do find that I need less when I'm just grabbing hunks of dough off, especially by the end of the process.  
After shaping about fifty dough balls, after all, one tends to get a pretty good impression of what size to make them.

Ok so that's enough about bagels for now.

On to the rest of Thanksgiving.

Upon finishing the bagels at the objectively ridiculous time of 1147, I went and had a nice nap.  
From there, we went to two different Thanksgiving parties, both of which were fun and lovely in different ways.

The first was with a close family friend's family, who had us and two or three other families over.  
We played what they call \say{the bag game}, which they claim is a game of luck, even though I know better.  
I'm still not sure I loved the method for sorting between rounds, but it was as good a method as any other.

In the end, I won what was called the best gift\footnote{chocolate, naturally,} which did also come with two shooters.\footnote{I'm pretty sure everyone there is of age}  
I was told I had to do both shooters at once, but was then judged for doing so.\footnote{to be fair, jager and fireball apple honestly go pretty well together}

We then ate a nice meal, and then the brother father and I transported to the second party.\footnote{obviously I didn't drive}

At the second event, we saw what I imagine are my father and brother's students.\footnote{given that all four of them said as much}  
The food was delicious there, but I was fading incredibly quickly, and so took the fall to say I wanted to go home.\footnote{people are shocked to hear that the chef, a man who now spends his weekends providing free (delicious and nutritious) meals to local people in need called me a wimp for wanting to go to sleep. It comes from a place of love}  
From here, we're now asleep, awake, and once more giving out bagels.

At party number one, people were shocked to learn that it is not my father who's the bagelmeister in the home.  
I think that there are likely two explanations.  
First, my father made them bagels recently.  
Second, before I was born, my father apparently made bagels often.

I'm not sure which reason is the main one for the people's surprise that they came from me, but regardless.

Reviews from the bagels were generally positive.  
My first comment was \say{dang, it's really hard to complain and go \nsay{my bread is too fluffy and light and delicious}}.  
My family agreed, commenting that there was not enough chew to the dough.  
There are apparently two things that can lead to a chewier dough\footnote{according to a quick search}, more water and more kneading.  
I do always fear that I underknead my bagels, but they were higher hydration than I normally make them, so I don't think that the water is a real answer.  
Also, I did much more kneading of the small balls than I normally do, so I don't know that I think that it's that either.  
Who can say.

From the non-my family people, I got much better reviews.  
Many said it was the best bagel that they'd ever had, which may be true but is likely just a kind thing to say.  
Given the excitement in people's eyes when we told them they were welcome to come by today to take more bagels, I think that at least some of the joy was real.

Some, almost two thousand words into a story of crafting bagels, might wonder what I made.

I made seven dozen un-adulterated dough bagels.  
Of these, three dozen then were covered in everything bagel seasoning.\footnote{an intense amount, if my father is to be believed}  
Because I once saw a post about how some place's bagels were much better because they had seasonings on both sides, I seasoned both sides of the bagels.  
The thirty six everything bagels did use almost half a container of Costco everything bagel seasoning, so it's possible I did more than was expected.  
Then again, thirty six bagels is a lot of bagels!

After the thirty six everything, I made four dozen plains.\footnote{seven dozen, for those keeping score at home}

From there, two dozen rosemary and sage bagels, where I mixed the two ingredients into the bagel dough.  
It wasn't the most even mixing I've ever done, but honestly they may be the best bagels I have had in ages.  
Something about the Thanksgiving spices in a warm circlet of bread is just unbeatable.\footnote{nine dozen}

Two dozen cranberry bagels, which didn't go super well.  
I don't think that I realized how much craisins can rehydrate, and I didn't put as much effort as I often did in the past to incorporate the dried fruits.  
Still, cranberry bagel isn't a hard sell.\footnote{eleven dozen}

Those who know me might know that I recently had a fairly intense falling out with a friend over the concept of a jalepeno bagel.  
I said and still say that they feel like an abomination.  
Unfortunately, I am too much of a pushover.  
Just as if the friend was there that day, my brother looked with pleading eyes as he asked if I would make jalepeno cheddar bagels.  
And so, despite feeling like I was committing a crime through every stage of making them, I finished the day with a dozen of those abominations.  
They were well received, which almost hurts more.

Of the 144 bagels, I no longer know how many we have left.  
What's important, though, is that I got to share joy with others.

When looking to see what I've written about bagels in the past, I learned that I've apparently only made \href{bagels-23.html}{one post about bagels} before.  
It's likely that I commented on their output in a future posting, but I love the brevity in that one.  
I've also reflected about morning baking \href{baking-reflection.html}{before}.

At the end of that post, I mention that my recipe of adding flour and water until an appropriate amount of dough exists was not helpful.  
That's fair.

OH!

For the people in my life who comment on the salting I \footnote{allegedly don't} do in my doughs, the bagels were salted, and no one commented on the lack of salt.  
Tasting them, there's perhaps less salt than some might put in the dough, but I often find that people love very salted toppings for bagels.

And now to the sad part of the musing.

It's hard to think about Thanksgiving without thinking about my mom.  
She and I were the ones who did most of the Thanksgiving cooking for the past few years.  
It's her side of the family where our mashed potatoes\footnote{a common food in the home, strange for being peeled red potatoes} and butter mushrooms\footnote{a Thanksgiving exclusive. Apparently not a normal food, but mushrooms simmered in butter for hours on end. Makes a delicious compound butter and a, perhaps unsurprisingly, buttery mushroom} arise.

More than that, though, I remembered yesterday how she was always the one who remembered the little things that mean so much.  
I've never been a huge fan of cream cheese, but I've generally always liked flavored cream cheese.\footnote{berry flavors, to be clear}  
Every Thanksgiving, she was sure to stock at least a few flavors of berry cream cheeses.  
Yesterday, when I asked, I learned that we had not picked any up.  
Of course, that's also on me; I was going through lists of ingredients for Thanksgiving bagels and never once mentioned it.  
But, there's something really nice in not needing to ask for something because someone who knows and loves you remembers it without asking.

It's just one of the little pains from the life.

At the first Thanksgiving party, someone I had never met before and I ended up talking about something that led to another person at the party commenting on my mom.  
It's always so lovely to hear the memories that others associate with her.  
It's a little strange that she is remembered so identically in the minds of so many people, if only because I feel like normally people are crystallized into a single role in others' minds.  
The fact that more than a year out, though, people still comment on how much some medical thing my mom did meant to them is lovely.

I've been thinking about a post for a little while now about many things, but primarily the idea of greatness and how it is or isn't somewhat of a curse.  
In thinking about that post, I realized that mostly, I'm trying to justify to myself whether or not I need to strive for greatness.  
It's hard, seeing the incredible impacts my parents have and had on so many people, to not think that I need to do the same.  
Every time I try to tell myself that their impacts are probably exaggerated, though, I am faced with people who actively dismantle that thought.  
My mother was a great woman.  
My father is a great man.  
I will think more on greatness in that post.

Thanksgiving bagels were fantastic.  
It's lovely having a family tradition, and I hope that someday soon I'm able to share that tradition with my own children.  
It would be somewhat beautiful if I also had to stop it because taking care of children is hard, only for my own son to bring it back as a teenager.

Current Pen List\footnote{for my own posterity, mostly}

\begin{itemize}  
\item Hongdian Black with Fude Nib: Empty  
\item Jinhao Shark: Diplomat Sepia Black. 10/6  
\item Pilot Preppy: Diamine Bilberry. 10/6  
\item Shaeffer (blue): Empty  
\item Diplomat: Laban Zeus Purple 11/23  
\item Kaweko: Stipela Sepia. 10/6  
\item Monteverde: empty  
\item Shaeffer Calligraphy: missing

\end{itemize}

\end{document}