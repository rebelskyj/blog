\documentclass[12pt]{article}[titlepage]
\newcommand{\say}[1]{``#1''}
\newcommand{\nsay}[1]{`#1'}
\usepackage{endnotes}
\newcommand{\1}{\={a}}
\newcommand{\2}{\={e}}
\newcommand{\3}{\={\i}}
\newcommand{\4}{\=o}
\newcommand{\5}{\=u}
\newcommand{\6}{\={A}}
\newcommand{\B}{\backslash{}}
\renewcommand{\,}{\textsuperscript{,}}
\usepackage{setspace}
\usepackage{tipa}
\usepackage{hyperref}
\begin{document}
\doublespacing
\section{\href{writing-5.html}{On Writing Again}}
First Published: 2023 June 26

\section{Draft 1}
Huh I hadn't even realized that my last post on writing was \href{writing-4.html}{less than two months ago}.
It was a very short post, more so\footnote{which I will always think should be spelled moreso} about my lack of writing for the day than anything else.
Today, I want to talk about a few writing things I've learned or rediscovered lately.

First, I found a new workflow for writing my book.
I am easily motivated by numbers going up\footnote{as should be obvious, given the genre I write in}, so I had taken to writing my\footnote{uncompiled \LaTeX because I am a monster} chapters in an online word counter, which updated the character counts basically any time that I stopped typing for half a second.
Someone at \href{conference.html}{the conference I went to} was horrified by that workflow, so I spent a few days using TeXShop's built in word counter, which was slightly less nice.
Today, though, an online source recommended StimuWrite, which is basically just weaponizing the dopamine bursts of using the internet as a way to make you write more efficiently.
The features I love most about it are the pleasant\footnote{which I always think should be spelled pleasent for some reason} moving backgrounds it lets you set, and the fact that there's a fun little progress bar that tells you how far through your goal you are.
It does mean that my workflow now looks like writing the goal, copying it into my document, changing the word goal, and deleting the text in StimuWrite, but that's the nature of the beast I suppose.

I don't think that I'm particularly more efficient with this tool, though its explicit lack of a spell checker does mean that I rewrite words far less often while actually typing.
I think that it does certainly help me to keep writing when I'm halfway through a sprint, though.
Looking down at the bar that is sixty percent filled\footnote{yes I am aware that sixty is more than half, but I like to round and it's my 'blog so I'll do as I please} and watching it slowly tick up as I write a few more words is really motivating to me for some obvious reason.

But, not all is roses and sunshine in the realm of my writing.
I realized that I haven't really written any sonnets in ages.
I know that sonnets are like (insert metaphor here).
The first time I write a sonnet, even just after taking a break\footnote{read: of multiple years}, it's very difficult.
If I keep it up for a few weeks, though, it becomes almost as easy as just transcribing a sonnet.

Back in the realm of sunlight and flowers, I wrote a letter to a friend today!
I apparently fit nineteen words on almost every line\footnote{known because I'm absolutely counting these words in my daily word goals}, which is a strange number, especially since it was reproduced on more than half of the lines.

Once more leaving the realm of daytime and flora,\footnote{which I would like to alliterate, so optionally, read as daylight and daisies} I would like to journal more.
In part, I think it makes me more thoughtful.
In part, a book I'm reading recommends it.
In part, someone I look up to a lot does it daily.
Between those three parts, it seems like something I might benefit from doing, especially since I already have topics in mind.

Looking back at that last paragraph, I'm reminded of why I started this blog.
It was to be a digital diary, in part fulfilling course requirements.
However, the reflecting I do here is far different than the reflecting I do in my private journal.
I don't want to say that it's sanitized.\footnote{but it absolutely is, I have no other way to end that sentence}

Anyways, I'm glad to have at least a little bit of time to start returning to normalcy before my life goes crazy again.\footnote{oh that's next Wednesday. Crud. I legitimately need to be two chapters ahead by next Wednesday, because I will be unable to write from the fifth to the seventh. Being ahead enough for the following week would be ideal, though unlikely.}
I need to write up a character for a DnD one shot I'll be playing with friends this weekend.\footnote{my first ever paladin!}

\begin{itemize}
\item I quantum cleaned.\footnote{read: I did the smallest notable amount} In particular, I dug out the blanket that I've been nominally working on since 2019.
\item I blogged today and even yesterday. Wow, look at me, setting up a streak.
\item The air quality was bad today, so I did an online pilates class. That was really fun, though absolutely killer. I also learned that my entire body is far less flexible than I remember.\footnote{which is already not particularly flexible}
\item I did, in fact, sleep in today. Tomorrow I will regain my historical sleeping goal, though, I hope at least.
\item After this, I'll do a rosary as I get ready for bed.
\item I did finish today's chapter last night, and I wrote tomorrow's chapter today. If I can just write four more extra chapters by Friday, I get to my goal.\footnote{back of the envelope math suggests I need two chapters for the actual week and I want five extra, so that means seven chapters by Friday (or Sunday, since the number of chapters I post over the weekend is none [but that's not the goal, now is it?]). That's basically two a days for the next four days, which somehow feels both very doable and impossibly difficult, as all goals should.}
\item None music today.
\item I've addressed another four letters, and I even wrote\footnote{as mentioned} and posted one today! That means that I'm officially one third through this monthly goal.
\end{itemize}

732/235
\end{document}