\documentclass[12pt]{article}[titlepage]
\newcommand{\say}[1]{``#1''}
\newcommand{\nsay}[1]{`#1'}
\usepackage{endnotes}
\newcommand{\B}{\backslash{}}
\renewcommand{\,}{\textsuperscript{,}}
\usepackage{setspace}
\usepackage{tipa}
\usepackage{hyperref}
\begin{document}
\doublespacing
\section{\href{schedules-3.html}{On Schedules}}
First Published: 2023 December 27

\section{Draft 3}
I've mused \href{schedules.html}{a} \href{meta-schedules.html}{few} \href{schedules-2.html}{times} about schedules on this blog.
I have also mused more than a few times about writer's block or general struggles to find a prompt for the day.\footnote{if you just iterate writers-block-n, you'll find a bunch of them}
Finally, I've started realizing that I want to try writing better essays, and that the most effective way to do that is by spending more chronological time with the essay as a work in progress.
There's an old expression about birds and stones that feels relevant here.

To write better musings, I need to be able to spend more time on a musing.
The amount of mental energy that it takes to come up with a prompt and start exploring it is such that I cannot really dedicate myself to writing two completely disparate and unique musings at once.\footnote{assuming that I want to keep up with the rest of my work, which I do. Also I know that nothing I write is really unique or disparate, but they feel like it to me}
Additionally, I do better with deadlines.
There's an expression I once heard in a class\footnote{I think it was composition, but it may have been art}, which goes something like \say{a work of art isn't finished until it's wrested from the author's grasp}.
Or, in slightly more real terms, as Jim Butcher once said, \say{I don't have a muse, I have a mortgage}.

All that to say, I think that it is important for me to schedule not just a day where I post more in depth musings, but also a more scheduled way of musing overall so that I have the mental space to write it.\footnote{them?}
Having spoken to my family, they still think that it is too ambitious to write a blog post every day, keep up with my research and work, record an album, and keep up with my web novel.\footnote{ok yes, when I write it all out it does feel like a lot. Then again, I find that I thrive when I am doing a lot. As long as I recognize that this is my lowest priority overall (even if the album tends to be the lowest priority on any given day), that should be fine}
However, I think that I have the space to be able to do so, at least for now, as long as I make some changes.

Moving forwards, I will reduce the mental load on myself by scheduling each day's activity.
Sundays, as you might expect, will continue to be dedicated to Reflections on the Readings.\footnote{Musings on the Mass is also probably a title that I could work with, since I love alliteration}
Fridays, as well, will remain focused on Flash Fiction Friday musings.\footnote{something that I'm very willing to not do, if I need}
Saturdays, I have decided, will be for long form writing.

Now, this is three of seven days.
Even that is more planning than I have been typically doing\footnote{not really, but I kind of forget about Friday and Sunday most of the time.}, but that's not enough.
Also, as mentioned, my family is worried about my ability to remain focused on research.
To help with that, I will be dedicating the Monday Musing of each week to Thesis Work.\footnote{oh Thursday Thesis Work would have been some fun alliteration, as would Tuesday Thesis Time, but alas}
My thought there is that starting each week with writing explicitly focused on my thesis will help me to remain focused on my life goal.\footnote{ok, so honestly, it is not my sole life goal, but at this point in my career and life, it is absolutely one of the highest priorities}

On Wednesdays I will muse about the progress on my album.
Wednesdays tend to be one of my freest nights, which gives me the space to do some last minute writing or recording if I feel behind.
Additionally, since it will often be a short reflection, that will give me more time for writing.

Thursdays will be dedicated to quick recaps of what I have been reading or writing.
Partially, this is because Thursdays tend to be busy nights for me, and often what I read and write does not change week to week.
However, I do also want to be more intentional about what I'm reading and writing, and I think that needing to reflect on them weekly will help me with that.

The astute among\footnote{I keep wondering what the difference between among and amongst is. I remember seeing part of it is formal and informal or new versus old. I know that there is also something about British and American English. Looking it up, Garner apparently believes that amongst is needlessly pretentious in American English, being as it is, an Archaism. Alack, nigh, and wot are all also in that list, which is kind of funny and interesting. Cambridge, on the other hand, suggests amongst is just the more formal version. I guess that I never really write formally enough that amongst is viable, except maybe in my thesis writing.} readers may note that this only describes six days of the week.
The week, however,\footnote{contrary to the Beatles' claim (wow that was almost funny enough to put in the main text of this musing)} is seven days long.
I know that many musings that I might be interested in pursuing do not deserve or warrant\footnote{ooh the concept of a topic deserving anything is interesting. I need to figure out how I feel about anthropomorphizing and animizing inanimates like ideas (wow my spell checker doesn't believe any of those are words), warrant is also an interesting choice, because I would not have expected practicing scales to become a musing on love. However, most of the time I feel like I have a good grasp} a full week of consideration\footnote{or honestly, multiple weeks, I could always start two musings one week and then give the voices in me two weeks to finish a final draft (or n and n, for any positive integer)}, and I want the freedom to be able to write about whatever I want at least once a week.
I think that I know myself well enough to know that I do best when I have some freedom, however constrained.
Giving myself one day a week to let musings die will probably be a good valve for my stress to escape.

Daily Reflection:
\begin{itemize}
\item Hobbies:
\begin{itemize}
\item Did I embroider today? I went through another needle's worth of embroidery! I think that I only need two or three more total strands, given that I've filled in about a quarter of the grid.
\item Did I play guitar today? I also took today off, which is probably less great given that I want to start writing for an album.
\item Did I practice touch typing today? A few lessons! I'm kind of unsure how it prioritizes which letters go when, but I think that it might be lowest wpm letter\footnote{how it decides wpm on a letter isn't exactly clear to me} going through?
\end{itemize}
\item Reading
\begin{itemize}
\item Have I made progress on my Currently Reading Shelf? I'm enjoying the book once more, which is really nice.
\item Did I read the book on craft? I brought it with me, but it is currently dark, so I don't think that I'll be reading it for at least a little longer (if at all)
\end{itemize}
\item Writing
\begin{itemize}
\item Did I blog? Three drafts may have been the right number for this musing.
\item Did I write ahead on Jeb? I should probably start Jeb again tomorrow.
\item Letter to friends? Met with two and set up a meeting with a third!
\item Paper? I will try to work on it tomorrow as well.\footnote{assuming I can bully my brother into helping me}
\end{itemize}
\item Wellness
\begin{itemize}
\item How well did I pray? oof.
\item Did I spend my time well? Eh! I wrote this which I'm pretty proud of, and I spent a lot of time with family.
\item Did I stretch? Should probably delete this goal. Won't but will continue to feel bad.
\item Did I exercise? Nope
\item Water? Eh. not a lot
\end{itemize}
\end{itemize}


\section{Draft 2}
I've mused \href{schedules.html}{a} \href{meta-schedules.html}{few} \href{schedules-2.html}{times} about schedules on this blog.
Additionally, I have mused more than a few times about writer's block or general struggles to find a prompt for the day.\footnote{if you just iterate writers-block-n, you'll find a bunch of them}
There's an old expression about birds and stones that feels relevant here.

As I've grown older, I've realized that schedules generally make my life better.
I not only prefer to know where I'm going and what I'm doing when, but also find that it makes me more productive.
As I've grown older, though, I've also gotten more and more freedom to set my own schedules.

Because I know that I need schedules, I have begun to find ways to govern myself.
At least once every few months, I try some strategy to keep me productive through every minute of the day.\footnote{ok, not actually, but to reduce the amount of time that I waste, at least, and often to become more productive.}
Without fail, these plans fail.
However, I've gotten better at following the advice of some random blogger somewhere, who said that it's far far healthier to reframe that.
That is, rather than saying that I failed to keep up my tracking activities to the minute, I could instead say that for the week I did track my time, I was more productive.
Things can serve you for short periods of time without becoming permanent fixtures.\footnote{for all that I'd rather have my life be at a 7 in terms of progress at all times, I should accept that I want average to be a 5 and that means that definitionally my median (or mean, I suppose, but hopefully my mood is level enough that there's a single monomial distribution of days) has to be a 5. Whether that means reframing the amount of progress I consider for 5, or if that means decreasing the amount of progress I feel like I need to make is a conversation for another day, another time (I don't know where that quote comes from, but \say{another day, another time} is something that feels like a brain worm I once had. The fact that the melody around it in my head is something only fragmented and without any timbre (as most of my conjured voices tend to be (not like that (well a little like that sometimes), in the sense that I try to think about what music I write should sound like, not that I have voices in my head)), it's entirely possible that it's just something I'm making up wholesale. Apparently there's a concert based on a Cohen brother's movie with that title (the concert not the movie), so that's interesting. I also feel like there's a voice in my head trying to make it a quote from Rent, but I don't think that's accurate)}

However, there are kinds of schedules that I do find I can keep myself on, at least in general.\footnote{there are periods of my life where I cannot keep myself doing anything, but those are hopefully going to be fewer and farer (farrer? I don't think that English has a comparative for far. Oh wait, duh further or farther. Time to quickly read \href{https://dictionary.cambridge.org/grammar/british-grammar/farther-farthest-or-further-furthest}{cambridge's take on it}. For those not interested in reading that, the tl;dr (too long, didn't read) is that further is more common, farther is more commonly used to denote distance from a speaker, and further is the only word which can be used to denote greater or at a higher level) between}
In general, things which occur monthly or weekly are much easier for me to keep up on.
Daily activities that don't need to be done in a schedule can also be easier to keep up on, which isn't really relevant here.
The fact that I can do things with a weekly schedule is the way that I can kill the birds with a stone.

There is even a third bird that my suggestion will help ricochet to.

Sorry, I have written about 500 words in footnotes for this draft, and I find that I've completely lost the thread of this musing. Time to try again (if you didn't read the footnotes, they're a time).

\section{Draft 1}
I've mused a few times before\footnote{remember to hyperlink next draft} about schedules.
As I come to another day where none of the prompts that I have feel particularly good, I'm going to jump the gun on 2024's New Year's Resolutions and start talking about my schedule.
I'd like to start writing one really good essay a week.\footnote{really good may be an overstatement, but you know what I mean}

I think that one way I can get through that is by scheduling other blog posts to be more mundane, and therefore require less energy, with the goal and expectation that I then use the extra time and mental space to write parts of a longer and better thought out musing.
So, what are things that I want to keep up on every week?

Maybe I should take a step back and ask what things I want to prioritize in the next year.

Nope, that's a bad idea because most of what I want to prioritize are habits.

There are seven days each week.
If Sundays are reserved for Reflections on the Gospel or Readings generally\footnote{which they are}, and I want to spend Saturdays on a more in depth writing, what do I do with the other five days?

I want to do more composition, so one day a week could be reserved for my progress getting through species counterpoint.
Actually, let's take a moment and decide whether I actually want to keep learning species counterpoint.
The choir I'm in has implied that it would be willing\footnote{ok, the director has implied that he would} to let me compose some music, but I don't really know if that's something that fits into my life.
Right now writing is really becoming a huge part of it, and I think that I would like that to continue.

One day a week could be reserved for me to reflect on the writing I've done over the past week.
I think that monthly and daily checkins might be the wrong level for me to be motivated to outwrite on my web novel.

Since I want to make progress on the album, one day a week could be reserved for talking about it. 
Right now, I need to write and record and all the other parts of songs, so there's bound to be stuff to talk about re: difficulties, fun parts, etc.
That brings us to four.

If I continue doing open mics semi regularly, talking about them could be fun.
When I consider that open mics tend to run opposite a DND campaign that I'm in, we get Mondays taken care of.
Then again, I don't want to have to force myself to do those activities or muse about them.

Still, four days a week of musings is better than one.
One day a week could be a brief review of everything that I've read.\footnote{and feel comfortable admitting that I've read. For all that I make fun of one of my friends (hi) about it, there is absolutely content I consume that I would rather not be associated with}
That actually sounds really fun, especially because I can also jot down my notes somewhere more cogent.
On weeks that I feel like reviewing a book, it can go there, and on other weeks I can just discuss the reading that I've been doing.

One day a week devoted to working on my thesis would probably not go amiss.
Since I want it to be fairly long, I do need to get a lot of writing done.
I do worry about the fact that I would like those to also be edited, and the musing format tends to be more informal than I'd like.
Still, I am sure that I could find at least one thing to quickly talk about each week, especially since I want to work on making a lot of my research more approachable to the general person.

(took a break to play some backgammon, time to start musing again)

Right now we have:
\begin{itemize}
\item Sunday: Reflections on Readings
\item Saturday: Better Musing
\item A Day: What I've been Reading
\item A Day: Thesis Work (very nominal, can be as vague as needed)
\item A Day: Progress on the Album
\item A Day: Progress on writing generally
\end{itemize}

Oh shoot! I also have flash fiction fridays! That takes up my whole weekend, so then we should probably get rid of one of the seven so that I have time to have one day free?
There are absolutely topics I want to discuss that I don't think deserve a full length essay.\footnote{though, maybe that's just because I haven't written them.
I didn't expect practicing scales to be a discussion of love}

so:
\begin{itemize}
\item Sunday: Reflections on Readings

Rationale: I'm obligated to attend Mass on Sunday, it's the day that I've been told I should set aside for the Lord, and it's a good habit to make me more prayerful.
\item Monday: Thesis Work.

Rationale: Starting the week with active productivity will make me better at work generally and should help me prime the week to come.
\item Tuesday: Free day.

Rationale: I filled the other six days and I want one day where I'm free to write whatever is on my mind/ generally have a day to discuss what's going on.
\item Wednesday: Progress on the Album.

Rationale: Wednesdays are my freest evenings, for all that I teach a discussion section until 5pm those days.
As a result, when I'm not as far along as I want to be, I can use the rest of the night to write or record
\item Thursday What I'm reading/ writing.

Rationale: I think that it makes most sense to alternate these or maybe do both each week.
Of the things that I want to do, these are the freest, and so I feel like deserve to get crammed together.
\item Friday: Flash Fiction Friday.

Rationale: the event is on Fridays, and I think that it will continue to do me good to think about it.
\item Saturday: Long form Musing.

Rationale: Saturdays should be days of rest, and that way I have the whole week to look forward to the writing.
\end{itemize}

For all that this feels very ordered on the page, a retrospective note should say that I wrote it in order: Sunday, Monday, Friday, Saturday, Wednesday, Thursday, Tuesday.

1000 words later, let's try to make this coherent?
\end{document}