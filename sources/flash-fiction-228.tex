\documentclass[12pt]{article}[titlepage]
\newcommand{\say}[1]{``#1''}
\newcommand{\nsay}[1]{`#1'}
\usepackage{endnotes}
\newcommand{\1}{\={a}}
\newcommand{\2}{\={e}}
\newcommand{\3}{\={\i}}
\newcommand{\4}{\=o}
\newcommand{\5}{\=u}
\newcommand{\6}{\={A}}
\newcommand{\B}{\backslash{}}
\renewcommand{\,}{\textsuperscript{,}}
\usepackage{setspace}
\usepackage{tipa}
\usepackage{hyperref}
\begin{document}
\doublespacing
\section{\href{flash-fiction-228.html}{Flash Fiction Friday}}
First Published: 2023 November 24

\section{Draft 1}
Prereading note: this is being written as a way to generate my fiction for the week.
As a result, the first draft (potentially the only draft I post) will be very rambly and disjointed.
I will pick up and discard ideas like small pebbles or shells on a wave-swept beach, sifting to find the one that speaks to me.

This is my third week in a row musing about what I want to write for Flash Fiction Friday.\footnote{a friend pointed out that FFFXYZ for X,Y,Z as single digit integers means that you end up with a hex code in the yellow.
That's kind of fun, especially since that means every Flash Fiction Friday is technically a shade of yellow}
This week's theme is \say{A form of distraction.}

Alright, let's start iterating.

Form of distraction could mean kind of distraction.
Form can also be like the shape of an object, such as form over function.
What shape does distraction take?

Ok that's maybe too far into the weeds, but we'll put a pin in that and maybe come back.

What do I want to do with a form of distraction?

The obvious meaning\footnote{to me writing right now} is something we do to redirect our attention away from something we don't want to deal with.
Form here modifies distraction in the sense of a shape that distraction can be forced into.

If we instead make distraction the agent, and we the object\footnote{man I love the rare linguistics classes I took to give me the words to discuss some of the metacognition I need}, we end up with a form of distraction being a way that distraction takes over us.

I haven't been writing as much these past few days as I would normally like, so I suppose that it's only natural that my mind goes to writing as the actual center of this story.
A form of distraction here could refer to the ways that my attention are pulled from the writing and craft I've been dedicated to.
Ok I also want to try using the writing advice that I read about recently.
Since I don't want to have to keep going to the \href{writing-7.html}{musing about it} and searching through the rambled prose, the rules are:

\begin{itemize}
\item No think verbs
\item No remembering, only flashbacks
\item Characters do not belong alone.
\end{itemize}

If I stick to those three rules, there's also an implication that I'm in either first person or third person omniscient, since third limited, in my experience, at least, explicitly calls out the thoughts that a character has.
Maybe that's just because the writing I read is bad according to this method, but I don't think so.\footnote{I mean, a lot of the writing I read is unquestionably unskilled, at the very least.
However, it seems as though the writing I read that is better, at least nominally, still uses think.
I guess it's like the analogy of vibrato and ketchup (early musicians tend to say that vibrato is like ketchup. It's fine and good, but if you put it on everything, everything will taste like ketchup).}

Ok so I think that I want to have something about the form of distraction in writing.
I need to have characters, plural, but I don't know if I really want that.
The two desires are clashing.
I'm going to take a minute\footnote{the colloquial meaning of the word, not the definitional one based on the decay of cesium} and try to form the fiction, seeing what I produce.

Quick note: even before starting to write, i realize that I don't really want dialogue in this fiction.
I write a lot of fiction in everything else that I do, and I think that this site being a place where I minimize dialogue in fiction might be a good idea for me.
Maybe not.
Ultimately, though, I'm writing this for fun, and it's completely uncoupled from all the other writing I do, even more than the rest of it is fairly atomized.
I'm going to write the first draft as free association and see what comes out.

Two lines in, I'm realizing that I want this to be poetry?
I think that there could be something fun with a mixture between poetry and prose.
Maybe the first half is poetry\footnote{sonnet form right now, so getting at least through the first stanza, probably two stanzas would be best} and then it shifts to prose, reflecting a distraction at a meta level.
Is that too edgy for the sake of being different?
Maybe! Bu we'll get to find out, and that's really the nature of doing short writing.
At worst, it doesn't work and I spent a few minutes doing some writing that didn't end up working how I thought it might.

Ok, so I don't really hate what I've written.
I think that it's a really strong start, for all that I ended where I think that I want the actual story to start.
I guess the question is how blatant I want to be.
The writer I cite for writing advice says that starting with the declarative kills the suspense of the other writing.
I think that I'll try leaving the reveal to the end, for all that I know that I'll need to put the reveal somewhere that makes it better.
What I've drafted so far, though, is probably good enough for me to start working on the other two and a half writing assignments I have for the day.\footnote{NaNo, Jeb, and that's about it. I guess this counts as the third, but I've (clearly) already started on it. I guess the question is if I should go straight through NaNo like I've done basically every day this month, or if I should take some breaks in the middle.
Last night, despite the fact that I said I was not going to finish, I somehow did, which was really great and cool.}

Having now finished one of the writing assignments,\footnote{NaNo, the only one that I actually feel obligated to do in a given day} it is now time, I suppose to try revising the FFF.
Honestly, though, given that I'm spending time with my family, I don't know what my motivation level really is.
We're all doing something else while watching a movie,.
Then again, I should keep writing, if only because I want to increase my word count for the day.
Maybe I'll try my second assignment for the day.\footnote{Hmm, I wonder if looking at jeb as an assignment, rather than a hobby, makes me want to do it less.
Hmmm, that's probably something that's worth interrogating myself about}

Well, I think that I honestly am pretty happy with the way this latest draft came out.
I'm not sure if I quite want to post it, if only because I feel like there's merit in writing things solely for my own sake.
Now, I suppose, the question is whether I want to rewrite this draft or start writing the final thing I have to write today.

Let's make a quick pro contra for each.
Pros of redrafting this: it's easy, and I think that it could be fun to reflect about what I wrote in a slightly more coherent manner.
Cons of redrafting this: redrafting is really just code for rewriting wholesale, and I don't know if I really want to do that.
Pros of writing Jeb: I get further ahead of the writing that I give to my readers, and I have less stress about doing it in the future.
Cons of writing Jeb: I have to figure out where I want the plot\footnote{admittedly minimal as it is} to go.

I think that it might be worth, if nothing else, trying to figure out where I want to take the book in the next few chapters.
For all that it feels like I kind of have too many words to fill, the number of chapters I have left to tell my story is rapidly shrinking.

Honestly, drafting the book did a lot to help me start to feel ready for writing.
I only got halfway through plotting the rest of this book, but I realized that there's actually a very limited number of things that I need\footnote{need is such a weird word in fiction.
This book does not need to exist} to have happen in the rest of this arc, but I started working backwards.
Unfortunately, that means that I still have the rest of the intermediate content to do, including the chapters I need to write right now.\footnote{I know that some authors do not write strictly chronologically, which doesn't really make sense to me. I don't really know how to think about the end of the book without thinking about each step before it. Learning to write better prose might help with that.}
Anyways, good job with writing today me.

Daily Reflection:
\begin{itemize}
\item Did I write 1700 words for NaNoWriMo? I did write the content. It's weird for me to realize that I really only have six days to finish the book. I'm beginning to understand why so many authors put all the action in the last five to ten percent of the novel.
\item Did I write a chapter of Jeb? I did not, but I did start to plot it out, which is probably better for me in the mid to long term. A family member just caught up on it, though, which is pretty exciting!
\item Did I blog? Wow look at this completely incomprehensible post. It's still a post though.
\item Did I stretch? I have been so lazy today. In fairness, I think that the day after a major holiday is almost always a lazy day for my family. It's nice that we go from active rest, where we do things to make us all feel special, to passive rest, where we just spend time with each other.
\item Am I doing better at prayer than a rushed and thoughtless rosary? Despite the rosy description above, no.
\item Am I doing a good job writing letters to friends? I am not. I'm meeting with friends tomorrow, though, which is a similar vibe.
\end{itemize}

\end{document}