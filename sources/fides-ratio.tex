\documentclass[12pt]{article}[titlepage]
\newcommand{\say}[1]{``#1''}
\newcommand{\nsay}[1]{`#1'}
\usepackage{endnotes}
\newcommand{\1}{\={a}}
\newcommand{\2}{\={e}}
\newcommand{\3}{\={\i}}
\newcommand{\4}{\=o}
\newcommand{\5}{\=u}
\newcommand{\6}{\={A}}
\newcommand{\B}{\backslash{}}
\renewcommand{\,}{\textsuperscript{,}}
\usepackage{setspace}
\usepackage{tipa}
\usepackage{hyperref}
\begin{document}
\doublespacing
\section{\href{fides-ratio.html}{On Faith and Reason}}
First Published: 2023 December 30

\section{Draft 2: 30 December}
N.B. In the interest of showing chronology of my thoughts, any rough drafts will end where I end them on a given day and start as a new section or draft afterwards.

The phrase Fides et Ratio means many things to many different people.
To those who are not Catholic and speak\footnote{read, most likely, if we're being honest} no Latin, the phrase likely means nothing.
To those who know Latin, it translates perfectly to Faith and Reason.\footnote{insofar as anything can actually be translated. There is (are) so much literature about how translation is inherently imperfect.
However, given that this document was written by a Pole who my Latin professor actively criticized the Latin of, probably not relevant here, given that the thoughts were likely in Polish, translated to Latin, and then translated into English (I should look up how encyclicals get translated, actually, it seems like an interesting experiment)}
For those who are familiar with their Papal Encyclicals\footnote{a venn diagram which is not a circle with Catholic, much as I might wish that all Catholics were within the set of people who are familiar with them}, it is the opening phrase\footnote{the incipit, if you want to sound fancy} to Pope Sain John Paul II's writing on the way that faith and reason\footnote{science, often} overlap.
It is also the name of any number of catholic and catholic adjacent groups who focus on that same overlap.

If I'm being honest with myself, I think that Fides et Ratio, and the worldview it professes, are what makes Catholicism so fundamentally different from most other forms of Christian thought.\footnote{I'd say religious thought, but wow is it hard to define religion I learn more and more, especially because the least credulous kosher Jew and the most credulous lottery ticket buyer probably end up grouped opposite of where most would feel like they should be at a gut level}
It is not that Faith and Science occupy different spheres and never overlap.
That story, while often told to young children, especially those with questions, is only true in the way that most stories we tell children are.\footnote{which is to say, it helps explain a concept at the level they are developmentally ready for. I'm very pro lies to children, if that wasn't clear. I just think it's important for us to move past them at some point}

Faith and reason both give us glimpses into the truth.
As such, JP2 writes, they cannot contradict each other.
If there is an apparent contradiction, one of the two must be wrong.

For me, a great place to point to for this is the concept of the death penalty.
It was long taught as licit for nations to have capital punishment, with one of the primary justifications being that it reduces crime rates.
As more and more modern research has shown that not to be the case, however, the Church changed her stance on the issue.\footnote{now, there's the whole \say{infallibility} which states that the Church can never be wrong. However, there are so many layers of nuance to both my statement and the meaning of infallibility that I'm not interested in getting into now.}

On the other end of the spectrum, we have what is allegedly taught as one of the great sticking points of modern Catholicism for atheists: transubstantiation.
That is, the bread and wine that the priest holds at the altar does, in fact, truly become the body and blood of our Lord and Savior Jesus Christ.
However, absent other miracles\footnote{because transubstantiation is, itself a miracle, as far as I've ever known}, it is still indistinguishable to mortal eyes.
A chemical analysis would not show that they have changed.

There are two ways of\footnote{somewhat legitimately. There are infinite ways of arguing anything, depending on how honest and non-sequitored you're willing to be (less and more for more arguments, respectively)} disputing this.
First is the way that I feel least comfortable with, which leans close to the God of the Gaps heresy.
That is, just because the Host remains unchanged by any technique we currently have available does not mean that it is unchanged.
New equipment or forms of measurement could be\footnote{theoretically, and for the example here. I, being steeped in the scientific tradition, can no longer really think of a technique that this would be true for} developed which would account for the difference.

The other is a much easier argument from a theoretical basis.
The second argument is, however, far more difficult to convince someone of.
It is to claim that there are truths that cannot be measured via scientific inquiry.

Of course, we know that this is true.
Science cannot prove its own validity.
Even outside of that, though, philosophers\footnote{mathematicians are philosophers. On this hill I remain willing to die} since Godel\footnote{I may never learn how to do diacritics} have proven that there is no way to construct a set of truths that can prove every truth.\footnote{If this isn't what the incompleteness theorem shows, I apologize for my misunderstanding}

Before I drift too far into my \say{multiple methods of inquiry are good, and what do we mean when we say science anyways} rant, let's pull back to faith and reason.
The two should be thought of as supporting each other, rather than existing independently or in opposition.
For instance, because I know that the Lord created the universe to follow rules,\footnote{which I know because He is definitionally All Good and would not make something misleading} I can perform scientific inquiry.
More than that, though, because the universe we were created can be studied from within itself, we are capable of showing when things are, in fact miracles.

This is a fine line to walk, of course.
Just because something cannot be explained by modern science does not mean that no science is capable of explaining it.

On the other hand, the Lord helps those who need it.
I am fully willing to believe that in days before antibiotics and immunizations, far more miraculous healings were performed.
This is not the God of the Gaps, whose power diminishes as we learn more.
This is a G-d who loves us so dearly that he gives us what we could never hope to have on our own accord.

So that is one way that I can use faith to increase my ability to do science.\footnote{kind of? I at least see that as faith means I know reproducible means I know I can do science}
How can we do the reverse?

Truthfully, that is where the spark for this musing came.
Yesterday I was talking with a friend who knows that I was raised Catholic.\footnote{they did not know my current relationship to the Church, which is fair}
The friend asked me if there was any way that I connected the work I do to the faith that I was raised in.

It's a difficult question to answer.\footnote{obviously, given that we're 1600 words in and I haven't really touched the topic at all}

If faith and reason can both complement\footnote{I think that this is the right word. compliment means to say nice things to, while complement is more usually used for completing} each other, then the system needs to work both ways.
As someone who does science and reads reason, how can that enrich my faith life?
There are the obvious places like reading great works of speculative theology or works like the Summa.\footnote{Summa Theologica, Doctor of the Church St. Thomas Aquinas's book for Dominicans to learn how to teach catechesis. There's a musing somewhere about the fact that most of the best documents for learning your faith are not meant to be read by the laity.}
However, that feels like a cop out.

Today is a Saturday musing, and those are supposed to be deeper delving than my usual musings.
So, let's take some time and stew for how my science makes my faith grow.
There's the fact that it forces me to remember that there is beauty on every level?
That's true but an easy answer.

There's the stock answer that I give people, which is that you cannot hope to understand a sculptor without understanding his sculpture, which feels on the right track, if still a little too pithy.
Can I take that a little deeper?

Let's see, I study the bible\footnote{in as much as you can call what I do studying} because it helps me to understand G-d, since it's His divine revelation.
The universe, however, is just as much a creation of the Almighty, and it is sustained constantly through his active choice to keep it.
In studying the universe, we get glimpses into something deeper?
Is that true?

Sometimes I find that questions are best answered by approaching from another angle.
When learning about the vastness of space, it is common to fall into a state of existentialism.
The universe is just so vast that it defies words and explanations.\footnote{in the same way that a quintillion is technically a number but I have no real way of visualizing what it is}

When I learn about the vastness of the universe, though, I do not feel smaller.
Instead, I understand that the Lord created such a large space for us.
Why, exactly, he did it is a mystery, but I guess that I can take some stabs myself.

The earth is so big that it feels almost infinite.
The Odyssey, for instance, takes place completely within the Mediterranean Sea.
Moses and his people traveled for forty years inside of a small middle eastern desert.
I can absolutely see how there could be a motivation to treat the concept of infinity with less grandeur if the universe was as small as the earth.

There's also the concept of harmony of the spheres.
I know that science has more or less entirely moved past\footnote{passed? no you move past I'm pretty sure} the concept, but science has also seemingly given up on animism.
Animism means a lot of things to a lot of people.\footnote{as most philisophical concepts, to be fair}
In general, though, when I think about animism I think about the idea that the inanimate, isn't.\footnote{that sentence may or may not be a great sentence, but I refuse to rewrite it}
One of the biggest flexes that modern scientists try to have over prescientific thought is that we now know that lightning doesn't come because the gods are angry.

On the other hand, every explanation of more or less any scientific concept does rely on the idea of animism.
We say that atoms form bonds.
In part, this is because we are taught to write in the active voice, which requires an agent and therefore agency.
However, when we speak in that way, we end up defining our thoughts.

I knot that there's more for me to say, but I can no longer find the words.
As much as I would like to keep going or post this next week, I think that I need to throw in the towel.

Daily Reflection:
\begin{itemize}
\item Hobbies:
\begin{itemize}
\item Did I embroider today? It has been another day without embroidery, but maybe tomorrow would be a better day. I hope it will, at least.
\item Did I play guitar today? Shoot! I spent the day with my family, but that's about all that I can say for most everything.
\item Did I practice touch typing today? I dd a few lessons, and that is what matters. I don't think that I made any progress.
\end{itemize}
\item Reading
\begin{itemize}
\item Have I made progress on my Currently Reading Shelf? A bit! I talked about it with the person I'm reading it with, which is fun and exciting.
That's always great, and I'll probably make more progress reading after I post this.
\item Did I read the book on craft? Drat. If I want to finish the book by the end of the month, I really need to do a bunch of reading tomorrow.
\end{itemize}
\item Writing
\begin{itemize}
\item Did I blog? I even blogged yesterday, even though I didn't post it. I didn't like the post, and I wasn't proud of it, so I didn't post it.
I think that, for all that I keep planning to write the posts every day, I will give myself grace to not post them when they are bad.
\item Did I write ahead on Jeb? I really need to start writing. Even though I am not posting Monday, I still hope to find time to write it before Monday.
Unfortunately, that's abut all that I can say.
\item Letter to friends? I wrote to a few friends, and that was fun! That's about all that I can say in terms of that, though.
\item Paper? I did a little, and I started thinking about appendices that I might need to add to my thesis. I think that there's something to be said about needing to justify a lot of what I did, and even more to be said about explaining what happens.
\end{itemize}
\item Wellness
\begin{itemize}
\item How well did I pray? Not at all, really. I need to do better, but it's hard.
\item Did I spend my time well? I spent it more or less napping and spending time with my family. There's nothing wrong with that, but I wish that I didn't need more sleep.
\item Did I stretch? I was so sedentary, which I'm not happy about.
\item Did I exercise? See just above.
\item Water? Rip
\end{itemize}
\end{itemize}

\section{Draft 1: 29 December}
Fides et Ratio are the opening words to an encyclical by Pope St. JPII.
It is also the name of various faith and science groups that I have friends who take part in.
After a conversation with a friend today, I find that I am thinking a lot about it as a concept.

I've mused before about the way I find mysticism and science to be intrinsically linked, but this is something different.
Mysticism, after all, is a tradition that nearly every faith has.
As a scientist, I do believe that there is objective truth.
As a Catholic, I believe the same.

As far as I have ever been taught, the two domains should never intersect.
Or, as JP2 put it, truth cannot contradict truth.
That's probably the better way to frame the differences.
It isn't that science and faith cannot answer the same questions, it's that they should come to the same answer.

For instance, we have the Big Bang.
It's the most famous example that I can think of where science and faith absolutely agree.
We believe that the universe was created ex nihilo\footnote{from or out of, depending on context, nothing}, and that seems to be what the Big Bang suggests.
Because the universe was created out of nothing, it makes sense to me that there would be no way to predict what came before the Big Bang, much as scientists might try to speculate.\footnote{I'm not going to get into the whole \say{it was also created by a Catholic priest}, because truth is truth, and it shouldn't matter who discovered it (outside of the whole \say{know the biases of the author so that you can be aware of blind spots they may have or framing that they may be using})}

But, there's far more to it than just that.
I am getting a Ph. D. 
On some level, I do believe that this involves dedicating my life, at least in part\footnote{can you partially dedicate something? Great question}, to science and teaching.
Even if the rest of the world and field no longer thinks that Doctor means teacher, I still do.

But, of course, I also find that I am continuing to grow in my faith.
On some level, everything that I do should and needs to be oriented towards the salvation of the world.
As someone whose favorite research questions do not heal the sick, feed the hungry, clothe the naked, or house the homeless, that means I have to find another way that the research points to salvation.

As I reconnect with old friends and meet new people, a common question I am asked is what my favorite part of my research is.
Of course, they rarely actually mean the research, and tend to mean my favorite part of being a graduate student.\footnote{I hope. If not, I feel really bad for not answering their question}
\end{document}