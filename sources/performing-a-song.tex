\documentclass[12pt]{article}[titlepage]
\newcommand{\say}[1]{``#1''}
\newcommand{\nsay}[1]{`#1'}
\usepackage{endnotes}
\newcommand{\1}{\={a}}
\newcommand{\2}{\={e}}
\newcommand{\3}{\={\i}}
\newcommand{\4}{\=o}
\newcommand{\5}{\=u}
\newcommand{\6}{\={A}}
\newcommand{\B}{\backslash{}}
\renewcommand{\,}{\textsuperscript{,}}
\usepackage{setspace}
\usepackage{tipa}
\usepackage{hyperref}
\begin{document}
\doublespacing
\section{\href{performing-a-song.html}{Performing a Song}}
First Published: 2022 September 26
\section{Draft 1}
As I mentioned \href{sharing-a-song.html}{about a month ago}, I wrote and shared a song with some friends.
Last Monday, I performed the song for the first time in public.
It was a good time, even though it didn't go exactly how I'd hoped.

That is, the tempo got a little messed up between me and the other band members, and the mic placement wasn't quite right.
Still it was very rewarding to share something I wrote, as terrifying as it is.
After the show, I realized that the concern I had about being booed was completely unrealistic.
Even if the song had been horrible\footnote{which I've been assured it isn't}, the vibes at the open mic are very supportive generally.

The song was reviewed as incredibly sad, which is a little funny to me as I think about it more.
Certainly the lyrics are sad, and I performed it in a sad way\footnote{the tessitura is mostly in the low end of my range, so it comes out more spoken}, but when played, even at tempo, on an accordion or piano, it's very clear that the piece is just in a standard major key.
My friends told me it was enjoyable, and that's nice.

One of them sent me a message after the show saying that it was brave of me to play the song, which was nice.
Anyways, I'm excited to keep playing with this band in the future.\footnote{like tonight!}
\end{document}