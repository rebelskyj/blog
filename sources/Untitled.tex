\documentclass[12pt]{article}[titlepage]
\newcommand{\say}[1]{``#1''}
\newcommand{\nsay}[1]{`#1'}
\usepackage{endnotes}
\newcommand{\1}{\={a}}
\newcommand{\2}{\={e}}
\newcommand{\3}{\={\i}}
\newcommand{\4}{\=o}
\newcommand{\5}{\=u}
\newcommand{\6}{\={A}}
\newcommand{\B}{\backslash{}}
\renewcommand{\,}{\textsuperscript{,}}
\usepackage{setspace}
\usepackage{tipa}
\usepackage{hyperref}
\begin{document}
\doublespacing
\section{\href{.html}{Final Essay for Diary}}


\section{Draft 0}
Journalled Diaries
What's the difference between a journal and a diary?
That question lies at the heart of what separates either form or process.

Merriam Webster defines both words.
But, just as we don't consult a dictionary every time we use a word, so too should we avoid doing so here.
What's important is the meaning behind the word, rather than the meaning of the word.

A diary, as \href{defining-diary.html}{discussed previously}, is a work that "focuses on expressing the reality of a contemporaneous account of the author's passage through time."
But, what's a journal?

A journal seems to be either simply the bound pages that a diary is written in, not a useful definition, or a logbook.
So, we can again define journal, through the same way we defined diary.
As with diary, journals are writings \textit{realis}, concerned with reality.
Unlike diary, we cannot determine whether a journal is concerned about a single person, \textit{solus}.
That is, it's not important whether a journal focuses on a single person or not.

Certainly we can think of journals accounting the war, or a single person's life.
Since all of these seem valid, maybe the dichotomy of \textit{solus} and \textit{multis} is not as helpful in defining a journal.
However, to see whether we can keep any of the previous work, we'll continue down again.
The next cleft was whether a writing is presented as being by the subject, which, as we mentioned above, isn't helpful.

Next, there's the division of whether the writing is temporally focused.
Again, since journals do focus on the passage through time, they should be seen as writings \textit{tempus}.
Then, the final division is whether the writings are done, \say{after the bulk of the narrative has occured,} or \say{as the narrative progresses.}
Since journals, like diaries, are written as the diary progresses, it makes sense that they, too would be writings \textit{iam}.

But, what else fits into that category?

Johnson in his dictionary takes the words as synonymous.
Britain calls calendar diary.
Journal as periodical, newspaper.

\section{Draft -1}
Title:Redefining Diaries
Imagine the diary as a beautifully carved rendition of a heart, a biography as a bust, and a fantasy story as a dragon, all carved from marble.
If each literary form is a sculpture carved from marble, then they must have come from blocks.
And, just as an artist can chip away chunks of rock to find the sculpture within, we too can chip away pieces of literature, as a way to find what is at the core of a genre.
Just as a fine sculptor can carefully take a single strike along a fault, knocking off a much larger piece, we too can use singular examples to determine what is at the core of a diary.

To begin, we assume that all writing is a diary.
Of course, if we want our diary work to be a heart, the block in front of us certainly isn't one.
Although arguments could be made as to how a cube is a representation of a heart, and any removal from the cube removes some piece of that, the goal of this sculpture is a representative heart.

So, we begin by finding something we know is not a diary.
For instance, we can begin with Henry Wadsworth Longfellow's \textit{Snow-flakes}.
Clearly, this evocative image of a winter's day is not a diary, but why?

I see this as not belonging for two reasons.
First, it's atemporal.
There's no sense of a movement through time as the poem progresses.
Second, it doesn't have (nevermind I'm just keeping that as the only reason)
So, with a stroke of a hammer and chisel, we've removed the entire set of writing that does not have a movement through time.

Next, we could look at J.K. Rowling's \textit{Harry Potter}, which I also hope is not controversial to describe as not a diary.
Why though?
I believe that it's not a diary because its main goal is to tell a story, rather than to record events.
Although there are many cases where that's less clear, there are very clearly times when pieces are written either to describe reality, or to tell a story.
Another stroke of the hammer comes down, and the fault line breaks again.
Here, however, it's slightly more ragged, as there are more individual pieces that need to be broken down.

Next, the sculptor might realize that he was making what could either be a journal or a diary.


\end{document}