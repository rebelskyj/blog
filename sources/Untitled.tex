\documentclass[12pt]{article}[titlepage]
\newcommand{\say}[1]{``#1''}
\newcommand{\nsay}[1]{`#1'}
\usepackage{endnotes}
\newcommand{\1}{\={a}}
\newcommand{\2}{\={e}}
\newcommand{\3}{\={\i}}
\newcommand{\4}{\=o}
\newcommand{\5}{\=u}
\newcommand{\6}{\={A}}
\newcommand{\B}{\backslash{}}
\renewcommand{\,}{\textsuperscript{,}}
\usepackage{setspace}
\usepackage{tipa}
\usepackage{hyperref}
\begin{document}
\doublespacing
\section{\href{theatre-final.html}{Theatre Final}}
First Published: 2018 December 
\section{Draft 0: 9 December}
Shakespeare, like all great writers for the stage, uses his staging notes as a way to inform his own views of characters and plot.
The clearest example of this is the fact that Shakespeare doesn't believe that Macbeth is truly the rightful ruler of Scotland.
To truly justify this claim, I will also look at two other Shakespearian tragedies: \textit{Antony and Cleopatra} and \textit{Othello}.

\textit{Antony and Cleopatra}, a show about a group of rightful monarchs exercising their rights, has a variety of flourishes.
Throughout the five acts of the show, Shakespeare calls for a grand total of nine flourishes.
\textit{Othello}, which has no royalty, has no flourishes.
\textit{Macbeth} itself has five flourishes.
However, these flourishes are never given to Macbeth, only his predecessor and successor.

In Act 1, Scene 4, Duncan receives two flourishes.
The first happens as the scene begins, and he enters the stage.
The second occurs as Duncan leaves.
Through these two flourishes, Shakespeare clearly states that Duncan is the rightful authority in the realm, as he calls for flourishes in his other shows at these similar moments.

Shakespeare does not dispute Macbeth's force of arms or leadership on the battlefield.
When Macbeth is first seen, in Act 1 Scene 3, Shakespeare calls for a \say{drum within,} as Macbeth enters.
The witches comment, \say{a drum! Macbeth doth come,} which echoes Iago in \textit{Othello}.

In Act 2, Scene 1 of \textit{Othello}, Iago proclaims \say{The Moor! I know his trumpet} after Shakespeare calls for a \say{trumpet within.}
By doing this, Shakespeare has his two antagonists comment on the marching symbol of their protagonist.
\section{Draft -1: 9 December}
Antony and Cleo:
2.5 Cleo wants music
2.6 Pompey and Menas enter w instruments
2.7 Music plays
2.7 sennet sounded, egyptian bacchanals, 
1.1, 2.2x2, 2.6, 2.7 w drums, 4.4 \say{trumpets flourish}, 4.6, 5.2x2 flourishes
\href{http://www.lieder.net/lieder/get_text.html?TextId=18775}{Come thou monarch}  is a shakespeare original 

Argument: Shakespeare wants the audience to know that Macbeth isn't a rightful king, and makes it clear through the use of flourishes.
Othello, which doesn't concern any royalty, doesn't have any flourishes.
Antony and Cleopatra, on the other hand, has lots of flourishes.
1.1: Antony and Cleo enter,
2.2: Antony announcing he'll fight before being greeted by caesar,
2.2: Caesar, Antony, Lepidus exit,
2.6: Pompey and Menas enter w drum and trumpet, Caesar, Antony, Lepidus, Enobarbus, Mecaenas enter from other door,
2.7: flourish bc they're saying goodbye before exiting,
4.4: Captains and soldiers enter,
4.6: Caesar, Agripa, barbus enter,
5.2: Caesar, Gallos, Proculeius, Mecaenas, Seleucus, etc enter,
5.2: Caesar exits.

(2k words needed)
\section{Draft -2: 8 December}
I can't believe I deleted my draft again.
Oh well, I only had the first draft, so it's fine.
I'm comparing use of music in 3 Shakespeare tragedies: Macbeth, Othello, and Antony and Cleopatra.

Macbeth:
Flourishes called for five times: first time Duncan enters and leaves stage (1.4), when malcolm enters after macbeth dies (5.8), when macbeth head is brought out, and when he leaves to end the show.
Hautboy (1.7) to signal the horrible thing happening at castle Macbeth.
Hecate speaks in rhyme.
Song: come away, come away.
\href{https://doi.org/10.1093/mq/XLVII.1.22}{this guy} claims that come away come away (all of hecate really) doesn't count as Shakespeare, so we'll ignore it.
4.1 has more hautboys, during the scene with witches and ghosts.
Idk if that's included in the not real bit, so I'll assume it is since I don't remember those lines from the performance.

1.3 calls for a drum- witches prepare for macbeth
5.2 calls for drum and colours- marching to kill mac
5.4 again drum and colours, same thing
5.5 drum and colours as macbeth preps for war
5.6 as macduff enters to kill beth
5.8, with flourish is drum and colour

Othello:
\href{https://www.shakespeare.org.uk/explore-shakespeare/blogs/shakespeares-drinking-songs/}{canakin clink} (2.3) was likely a popular drinking song. \href{http://www.lieder.net/lieder/get_text.html?TextId=18061}{claims} original.
\href{http://www.lieder.net/lieder/get_text.html?TextId=31027} based off of folk song (2.3)
\href{http://www.lieder.net/lieder/get_text.html?TextId=14898}{poor} \href{https://www.shakespeare.org.uk/explore-shakespeare/blogs/shakespeares-saddest-song/}{soul} not Shakespeare original, but version by him 4.3
Iago noticing othello trumpet (2.1) \say{The Moor! I know his trumpet.}
3.1 cassio brings musicians, they get thrown away
5.2 emilia and willow song


\end{document}