\documentclass[12pt]{article}[titlepage]
\newcommand{\say}[1]{``#1''}
\newcommand{\nsay}[1]{`#1'}
\usepackage{endnotes}
\newcommand{\1}{\={a}}
\newcommand{\2}{\={e}}
\newcommand{\3}{\={\i}}
\newcommand{\4}{\=o}
\newcommand{\5}{\=u}
\newcommand{\6}{\={A}}
\newcommand{\B}{\backslash{}}
\renewcommand{\,}{\textsuperscript{,}}
\usepackage{setspace}
\usepackage{tipa}
\usepackage{hyperref}
\begin{document}
\doublespacing
\section{\href{open-mic-6.html}{Open Mic}}
First Published: 2023 June 6

\section{Draft 1}
Oh wow, it's been a few months since the last time I blogged about going to open mics.
I think it's been about that long since I went to an open mic, which is sadder.
It's hard for me to balance going to the open mic, which I enjoy, and going to sleep early, which I enjoy \href{do-versus-done.html}{having done}.\footnote{wow look at that callback. I'm truly a master of the genre.}
Still, when a friend said that they missed my presence at open mics\footnote{not in those exact words, but that is the interpretation I'm choosing to live with}, I decided it was time to dust\footnote{metaphorically. I had literally played it at Mass that night and was practicing at least a few times a week before that} off my old\footnote{it's not really. I got it around a year ago, since I wanted an acoustic guitar that A: was in tune and B: had an amp hookup.} guitar.

It was hard for me to pick what I would play.
For one, while I have been playing guitar, that has mostly meant doing some picking patterns, some scales, and a few songs that I don't love singing along to in public.\footnote{mostly because I still don't know Fisherman's Wharf (the song I play the most) well enough to want to share it with others}
My staple open mic songs are all fairly stale, which also didn't seem like an ideal situation.
So, like all great folk musicians,\footnote{I'm not one, and it's almost certainly overly reductive to say they all did this. I do know that Stan Rogers was encouraged to do this, though, so I assume others were as well} I retuned my guitar to DADGAD.\footnote{I.e. bottom, top, and second from the top strings all tuned down one step.}
I'd been working on a Great Big Sea song\footnote{Boston and St. John} that's in Celtic tuning\footnote{DADGAD}, and I felt like I probably knew a Stan Roger's song\footnote{Harris and the Mare} well enough to do them.

As it turns out, I knew the Great Big Sea song well enough, and I stumbled through the end of the Stan Rogers'\footnote{I'm not sure how to add the genitive to Rogers} piece.
Also I learned a fun fact!
My throat feeling clogged before open mics is most likely a consequence of having brassicas\footnote{which I knew do that} and not being nervous, as I had historically believed!

237/171
\end{document}