\documentclass[12pt]{article}[titlepage]
\newcommand{\say}[1]{``#1''}
\newcommand{\nsay}[1]{`#1'}
\usepackage{endnotes}
\newcommand{\1}{\={a}}
\newcommand{\2}{\={e}}
\newcommand{\3}{\={\i}}
\newcommand{\4}{\=o}
\newcommand{\5}{\=u}
\newcommand{\6}{\={A}}
\newcommand{\B}{\backslash{}}
\renewcommand{\,}{\textsuperscript{,}}
\usepackage{setspace}
\usepackage{tipa}
\usepackage{hyperref}
\begin{document}
\doublespacing
\section{\href{writers-slump.tex}{On Writer's Slumps}}
First Published: 2024 July 2
\section{Draft 1}
I've written \href{writers-block.html}{at} \href{writers-block-2.html}{least} \href{writers-block-3.html}{a} \href{writers-block-4.html}{few} \href{writers-block-5.html}{times} about writer's block.
Today, however, a friend asked me to write about being in a writer's slump.
Despite not being within the friend's own lived experience, I do feel like I have at least a vague idea of what was being referred to.

Writer's slump, as I see it, at least, is at least apparently different than writer's block.
In writer's block, the motivation to write is there, even if the ability feels as though it is not.
In a writer's slump, however, the motivation to write has vanished.

Most of the time, what manifests as writer's block is itself a form of writer's slump, at least for me.
That is, most of the time that I feel as though I cannot think of anything to write, I am really just in a position where I don't want to write, for whatever reason.
As such, I hope that my methods for dealing with writer's block can be at least a little helpful for an acute writer's slump.
Given that the friend is in more of a chronic writer's slump, however, I don't know if the advice is as helpful.

Of course, the first question I find myself asking when a hobby that once brought me joy no longer does is whether I have an interest in anything.
My desire to do any particular hobby often ebbs and flows due to life situations and my own experience with the hobby.
Sometimes it isn't the season of my life to be doing any given hobby, and I find that being intentional about setting it aside\footnote{often literally. If I have a physical project that works, I will move it a few feet away into a place where, although still visible (because anything I cannot see doesn't exist), I have actively chosen not to engage with it} does a lot to inspire me to pick the hobby back up when the season is correct again.
If, for whatever reason,\footnote{cough cough, burnout, other things that aren't going on the blog} nothing seems like a hobby I want, then I address that instead.

However, that answer doesn't always work.
As someone who has made a commitment to publishing a large amount of content, for example, I do not always have the luxury of setting it aside.
More than that, sometimes writing in particular is something that needs to be done completely independent of my motivation.
So, how do I keep up with a hobby when I don't want to, especially one like writing?

One of the best things I've found is simply journaling a little bit, especially since my writing is usually done on the computer.
Writing the date and then simply listing what I want to write and why often helps inspire me to writing.
When that fails to inspire me, I often find that typing a motivational phrase\footnote{I'm partial to \say{the only way out is through, the only way through is forward, and the only way forward is going} myself, but that is because the phrase has taken an almost ritual meaning to me} helps me to remember why I'm writing.
That leads to the best piece of advice that I have, which is to figure out why I want to write whatever I claim to want to write.
If the goal is simply to write for the sake of writing, then I often swap between writing vaguely related to my thesis, the book I'm writing, inspiration for other books I'd like to write, and this blog depending on which feels best.
When the goal is to write something in particular, however, I try to find a motivation that remains resonant to myself.

If all this fails, of course, I rely on external motivation in the forms of bribes and threats.

\end{document}