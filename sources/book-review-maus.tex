\documentclass[12pt]{article}[titlepage]
\newcommand{\say}[1]{``#1''}
\newcommand{\nsay}[1]{`#1'}
\usepackage{endnotes}
\newcommand{\1}{\={a}}
\newcommand{\2}{\={e}}
\newcommand{\3}{\={\i}}
\newcommand{\4}{\=o}
\newcommand{\5}{\=u}
\newcommand{\6}{\={A}}
\newcommand{\B}{\backslash{}}
\renewcommand{\,}{\textsuperscript{,}}
\usepackage{setspace}
\usepackage{tipa}
\usepackage{hyperref}
\begin{document}
\doublespacing
\section{\href{book-review-maus.html}{Book Review of Maus}}
First Published: 2023 July 29
\section{Draft 1}
Maus is a book that I feel like I've always known existed.
Despite that, I do not believe that I read it before this year.
I can be honest with myself and admit that a large part of the reason I ended up picking it up is the fact that schools are replacing it in their curricula.

For those who don't know, Maus is a graphic novel about a man's experience interviewing his father about surviving the Shoah.\footnote{honestly, this word choice is something that I'm spending a fair amount of consideration on. Holocaust is the most common word used in the American world, and I'm nearly positive that's the term Spiegelman used within the book.
But, within the Jewish circles I've seen, Shoah is almost universally the phrase used.}
The Jewish people are represented by mice,\footnote{apparently in part because Jews were (are? [sadly enough present tense is still needed]) often depicted in Nazi propaganda as rats} Germans are represented by cats,\footnote{because cat and mouse} and Poles are represented by pigs.\footnote{other nationalities get other animals, but they're rarely relevant to the story.}\footnote{as it turns out, the choice to represent Poles as pigs is incredibly controversial.}
Some see the choice of animal characters as a way to make the horrors seem more abstract, in a way that humans would not.
I can't find anything explicit about why he chose to use animals in the novel, and so I'll just accept it for the literary device that it is.

Being semi-autobiographical and written in two parts separated by a vast gulf of time, the story drives you to keep turning the page.
The first book sets the framing as a man interviewing his father\footnote{who he has a difficult relationship with} about his experiences.
The first book covers from before the war until his father is sent to Auschwitz.
There is an element of dark humor in the way that dramatic moments within his father's younger life\footnote{such as being forced to flee his home in the middle of the night} are interjected by the reminder that the book is telling the story of collecting the story.\footnote{with conversations about how tea is made poorly or the like}
After revealing something horrifying about the war, the book will cut back to Art's father complaining about something minor in the present of the book.

Around two months after reading the collected version, the opening of the second book sticks with me the most.
It is a panel with a plethora of labeled numbers.
A few jump out immediately: the number of Jews who died in the Shoah, the time since the last book was published, the time since his father died, and the time since his daughter was born.
Because the book is now being written also as a memorial to his father, a third timeline is now explicitly introduced to the story.

Book two of Maus is the story of someone realizing that they have lost a relationship with their now dead father while listening to recordings of their interviews about his time in the Shoah.
Where the first book emphasized that Art's father had some ridiculous complaints, the second highlights how much his father was trying to manufacture reasons to see his son.
Interjected between scenes of the abject horror of a concentration camp, the framing of a strained relationship of father and son becomes all the more heartbreaking.
More, scenes where his father scrimps to save every last piece of money or food become far more reasonable when he discusses how that tendency is one reason\footnote{along with luck} that he was able to survive the horrors he did.

The book is a fairly quick read.
If you haven't read it, you should.
The fact that it is being banned or even restricted is horrifying to me.

I cannot think of anything else to say about the book, if only because I lack the words to express the way that it made me feel.
Most books that I read do little to affect me mentally or emotionally.
Those that do tend to tug on a single chord of my heart string or line of thought that I can identify and follow and recognize.
Maus hit me and continues to hit me like a tidal wave.

\footnote{I won't be doing the reflection here because it feels somehow off to treat Maus like any other book I've read.}
\end{document}