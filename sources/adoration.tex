\documentclass[12pt]{article}[titlepage]
\newcommand{\say}[1]{``#1''}
\newcommand{\nsay}[1]{`#1'}
\usepackage{endnotes}
\newcommand{\1}{\={a}}
\newcommand{\2}{\={e}}
\newcommand{\3}{\={\i}}
\newcommand{\4}{\=o}
\newcommand{\5}{\=u}
\newcommand{\6}{\={A}}
\newcommand{\B}{\backslash{}}
\renewcommand{\,}{\textsuperscript{,}}
\usepackage{setspace}
\usepackage{tipa}
\usepackage{hyperref}
\begin{document}
\doublespacing
\section{\href{adoration.html}{On Adoration}}
First Published: 2023 December 2

\section{Draft 2}
As far as I can tell, I've never mused about adoration on this blog.
Matching the times in my life that I've blogged to the times in my life that I have gone to adoration, I suppose that there isn't a lot of overlap.
Last night, however,\footnote{as I mentioned in the daily reflection portion of last night's blog post} I went to my parish's monthly young adult adoration event.
 
It was a really nice experience.
I feel like every time that I just sit in a church and pray for an hour, my life suddenly seems much better.
Modern psychology supports that idea, calling it meditation, and I don't want to discount the value of just unplugging and focusing on where I am in the moment for a little bit.
However, I've also gotten to the point in my life where I really do want to start exploring the spiritual side of my faith more.
There's more to prayer than simple meditation, and there's something special about praying in front of Christ truly present.
 
Last night's meditation was about the value of hope.
For all that the priest demurred and said that anything insightful or thoughtful he said came from one of the great thinkers he cited in his talk, it was still a fantastic reflection and meditation.
Hope is something that I've been struggling with lately.
The priest connected hope to prayer, reminding us that to pray more is to become better at praying, and to pray better.
I cannot remember exactly how he connected them, but I remember that it made sense at the time.
One line that I do really strongly remember is him pointing out that when we pray together, as in Mass or even in that moment, it is not just a bunch of people individually praying.
It is that, of course, but it is also so much more profound, for all that I cannot remember the exact words he used that resonated with me so strongly.
 During adoration, I found that my mind slowed down for the first time in what felt like months.
It was not the muffled silence of my thoughts being drowned out by earthly distractions.
Nor was I struggling to piece together fragments of a shattered idea from behind a murky window.
Instead, I simply was there, and I was at rest.
 
I was advised to read Psalm 118 last night, and it is an absolutely beautiful Psalm.\footnote{as, I assume, all of them are when given the chance.
This one just has so many of the lines that I know and remember from my childhood.}
I think that helped me to focus my prayers in a more uplifting and positive direction, rather than the negative spirals I've tended to start falling into.
There's so much going wrong in the world right now that it can be hard to remain hopeful.
 
Of course, as the apologetics book I'm reading points out, hope and optimism are not synonyms.
They can be as linked or unlinked as any other two traits.
I've been trying to keep that in mind. 
 
\begin{itemize}
\item Hobbies:
\begin{itemize}
\item Did I embroider today? I decided that I'll take today off from embroidery, especially since the initial goal was just doing it more often. 
\item Did I play guitar today? Yes! I played a quick few scales and did my normal warm-up etude.\footnote{composed by myself, and I've realized it completely avoids one of the strings (learned when I played it on a guitar missing a string)} 
\item Did I practice touch typing today? I did a few lessons. For some reason, I keep struggling with my letter t. I'm allegedly working on learning y, but have to keep going back to try t again. 
\end{itemize}
\item Reading
\begin{itemize}
\item Have I made progress on my Currently Reading Shelf? Yes! I finished one of the books and then pulled back up the oldest book on my currently reading shelf\footnote{started early 2022, which is a shame} 
\item Did I read the book on craft? I read a bit more of writing well and have started to get to the point that I'm starting to feel a disconnect between what they're saying and what I've found to be true.
They say you write like you read, and the writing I want to do is the pulp fiction that I read, but that has a very different style than what they recommend.
Still, I suppose that it's worthwhile to try, if only because learning how to write in a formal register cannot help but aid me as I work on my thesis, even if it doesn't necessarily help with the fun writing I do. 
\item Have I read the library books? I did not do this yesterday, which is a bit of a shame. Once I post my musing, though, I will absolutely do so. 
\end{itemize}
\item Writing
\begin{itemize}
\item Did I write a sonnet? Woo! Just cranked one out. It was spooky and winter themed, which is always nice. 
\item Did I revise a sonnet? I need to learn how to do this. 
\item Did I blog? Look at this, revised and everything. 
\item Did I write ahead on Jeb? Now that NaNo is over, I'm not sure if I'm going to keep up on the no Jeb on Sundays rule.
Mostly, the issue is that right now I feel very tired, and I've only written about 400 words. 
\item Letter to friends? Nope! I might have time tomorrow, though\footnote{I say that every day} 
\item Paper? I downloaded the latest set of computations,\footnote{that's a word that bothers me. We don't say computators, so why is it computations and not computions? Interestingly, people have asked the question, so it might be worth reading up on it} and will probably make time to look at them sometime soon. 
\end{itemize}
\item Wellness
\begin{itemize}
\item How well did I pray? Badly. I got home a little late last night, and so didn't make it through a rosary. Slept through most of the afternoon which also didn't help. 
\item Did I clean my space? Only for a few minutes, but I did my best. I think that I'm slowly getting ahead of entropy, which is my only real goal. 
\item Did I spend my time well? I spent a little more time on instagram and youtube than I wanted, but by and large I think so.
I woke up at my standard weekday time and volunteered at a concession stand, where I did some reading and chatted with a friend.
After that, it was home to eat, sleep, and then go lift. 
\item Did I stretch? I did! After lifting. I'm very tight. 
\item Did I exercise? I did a long leg workout with a friend that did not go well in the slightest. 
\item Water? Absolutely not enough, as evidenced by the lift going so poorly. Still, I have hope that I will feel better if I drink more for the rest of the night.
Update: I've continued to drink more water since coming home and do, in fact, feel better. 
\end{itemize} 
\end{itemize}

\section{Draft 1} 
As far as I can tell, I've never mused about adoration on this blog before.
Thinking about the times of my life that I've written this blog, though, I suppose it makes sense.
I don't think that I went to adoration a single time while abroad, and it wasn't something I sought out while in undergraduate either.\footnote{for all that I don't really want/need to defend myself, it also was far less emphasized at my home parish than it is here}
It was mostly something reserved for the occasional youth retreat.
 
As an older and more cynical person, I do respect how well the youth retreats I went to were scheduled.
They encouraged us to just the right level of sleep deprivation and exhaustion where the final Mass would always feel that much more impactful.
There's a trend in modern Christianity to say that the emotional experience you feel in a situation like that is the only valid religious experience.
That, of course, is wrong.
 
Definitionally, not every moment can be a highlight.
At some point, our minds and bodies adapt to the situation we are in, treating the average as a new baseline.
This is the same issue that creates addictions and overdoses, and can be just as harmful to our spiritual lives.\footnote{whoop, riffing a little too hard right now, tone it back and bring it back}
 
As a result, there's another trend in modern Christianity\footnote{yes yes, there's no new heresies, only old ones in a new day. It's still a modern trend, so I'm treating it as one.} that says that the emotional experience doesn't matter at all.
People in this camp, I've found at least, tend to be very dismissive of praise and worship music and most of the new Catholic hymnody of the early post Vatican II era.\footnote{I do think it's funny that I will uncritically refer to the entire 300 years around Trent as a single period and refer to the less than 60 year old time since Vatican II as multiple eras}
Their argument is that the music is too sentimental and sappy.
 
That's also clearly wrong.
We are creatures made with body and soul.
As a music major, the majority of what I learned about Trent\footnote{the famous council, not any person with that name or any of the other places, times, or events that could reference} is from a musical perspective.
 
One of the major controversies in the early Protestant era\footnote{calling it a revolution implies that I think they were in the right. Rebellion feels too dismissive, and so we use a nice little era. After reading a book talking about how much they dislike the postmodern (in the academic literary theory sense, not any of the thousand other versions) idea of how words define reality, I find that I'm becoming slightly more aware of the subtle tones of words I use in text.
Of course, I do know that I already tend to pay more attention to the variations of semantic meaning than my peers, so this may not be entirely beneficial.
Where was I? Right, Prots hate music} was a belief that the robust and intense polyphony and rich instrumentation of Mass settings of the era were bad.
They felt the same about most of sacred art.
 
Interestingly, both sides framed most of their arguments on the average illiterate peasant, rather than the people actually writing the arguments.
Luther himself said that the hymns he wrote should be used for laborers.\footnote{I don't really want to get into the fundamental classism inherent and required in Protestant ideology, but I guess I might have to, at least a little}
The Catholic position was\footnote{and is, for all that many seem to forget it} that G-d is Beauty\footnote{a claim I wanted to verify, and found that Pope St. JP2 wrote a book with that title}.
To see beautiful things cannot help but bring us closer, in some way, to the source of all beauty Himself.
 
The early Protestant position is less unified, and more or less says the same thing that all iconoclasts say.
Earthly beauty is bad, because it makes us focus on earth, rather than G-d.\footnote{I think? I've never really been able to understand it too well, honestly.}
Because of the intense push back, the Catholic Church did strongly consider restricting or banning polyphony.
In the end, though, the Holy Spirit won\footnote{saying that about council decisions I like, while true, always feels a little blasphemous. Probably worth making sure that I don't only think that about the things I agree with}, and polyphony was maintained in this Church.
 
Now, I'm sure that you're all wondering whether the reason I have a whole diversion on how interesting the Church's relationship with art, especially music, and especially early music, is is because I have a degree focused on early music.\footnote{oof that sentence hurts}
In part, yes.
In part, though, it's the same tension that we have today.
 
G-d is Beauty, Love, Truth, and Goodness.
While the reverse is not true, and something beautiful is not inherently divine, there is a part of us that is always reaching out for the Divine.
Especially in situations like the Mass, taking advantage of the fact that we can, in fact, use emotion to make people holier is a good thing that I think Churches should do.
Of course, as with all things, it needs to be a careful balance.
The music can never become the goal in itself, which is where the iconoclasts are right.
However, denying the part of our divinely created and inspired souls and bodies that seeks music and beauty is just as wrong.
 
Tying this back to the actual post, I don't think that there's anything wrong with the planners for youth retreats putting the most spiritually important moments at the places that we were most primed to receive them.
That's just good catechesis, leading people and helping them to be as open to the Truth as possible.
But, how does this all relate to adoration?
 
Adoration, in case any of my readers don't know, in this context refers to the Catholic practice of praying in front of the Blessed Sacrament.\footnote{most commonly the bread that has been transubstantiated into the Body (and Blood? I think? the metaphysics are a little unclear to me) of Christ}
Often, the Host is placed into an ornate golden frame on the altar.
Some songs written by Thomas Aquinas are commonly sung at the beginning and end of the time, and the middle portion is usually reserved for silent prayer.
In the diocese I'm currently at, most Parishes seem to try to schedule opportunities for Confession during adoration hours.
 
I'm realizing now that I haven't been sure whether I want to write about adoration generally or my experience at adoration last night.
I've been trying to split the difference, which is probably the worst version of each.
I think that right now I want to reflect solely on last night's reflection, so on to draft two I suppose.\footnote{there is an open question about whether that counts as draft two when I'm more or less throwing out all of this content.
I couldn't say for certain, but since I'm still planning to post this reflection, we'll see} 

\end{document}