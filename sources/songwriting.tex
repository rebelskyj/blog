\documentclass[12pt]{article}[titlepage]
\newcommand{\say}[1]{``#1''}
\newcommand{\nsay}[1]{`#1'}
\usepackage{endnotes}
\newcommand{\1}{\={a}}
\newcommand{\2}{\={e}}
\newcommand{\3}{\={\i}}
\newcommand{\4}{\=o}
\newcommand{\5}{\=u}
\newcommand{\6}{\={A}}
\newcommand{\B}{\backslash{}}
\renewcommand{\,}{\textsuperscript{,}}
\usepackage{setspace}
\usepackage{tipa}
\usepackage{hyperref}
\begin{document}
\doublespacing
\section{\href{songwriting.html}{On Songwriting}}
First Published: 2023 November 27
\section{Draft 1}
Five years ago today \href{writing-a-song.html}{I mused about writing a song}.\footnote{I made it all the way to getting ready to post this before I realized that I did not, in fact, write that post 5 years ago.
I wrote it 5 years and one months ago.
Whoopsie! This one is certainly better than reflecting on bat out of hell: the musical five years later (probably, but we will never know, I suppose)}
Today, I updated my \href{twenty-five.html}{list of goals for the year}, and saw again that I wanted to record an album.
While I've written probably an album's worth of music, a lot of it is either choral or instrumental in nature.

Some of the choral and instrumental songs\footnote{I used to be a pedant about how song only referred to sung music.
I don't know if I still am, but a part of me still thinks that it's, at the very least, a reasonable way to distinguish kinds of music.
Given that the piece I wrote was a chamber ensemble for two clarinets, an oboe, and a piano, it was very song like in its nature.
Of course, as a person who now claims that his main area of study was historical music, all music has an element of being sung by its nature.
The way that we think about music changing over time is a fascinating topic, and one that I would study if I had infinite time.
(more and more I'm realizing that I don't have enough time to learn everything I want to learn, and will have to pick and choose what I am interested in)}
would work on an album, especially since a few of my pieces\footnote{a much less controversial word} have been recorded.
However, as a person who loves the rock opera\footnote{honestly, that might be my favorite genre, for all that I treat it, like most genre titles, exceedingly loosely} format, I kind of want there to be a narrative string tying my album together.

Is that demanding far too much of someone who is a Ph. D. candidate\footnote{should that be capitalized? I never know} and has a rich and varied life outside of composing?\footnote{I sometimes feel guilty describing my life as rich and varied, but the number of people who disagree vehemently with me when I try to claim it as anything else keep me from doing it too often}
Quite possibly.
Then again, I find that I, like most people, do best when I feel pushed into my yellow zone.\footnote{a concept for describing mental wellness. Green zone is comfort, yellow is growth, but is uncomfortable. Red is dangerous. Many people struggle to tell the difference between the different zones, but that's a topic for another musing}
I also do want to write an album, for reasons that I don't know if I've explored fully.

I think part of it is that I feel like my musical education is a little too academic.
For all that I know a lot about music, and can produce competent music quickly, it kind of feels like a standard I use to judge an artist is whether they've created an album.
Is that a fair judgement?
Probably not.

Is that a prejudice I should investigate inside of myself to see if it really actually resonates with the deeper parts of me?
Probably!

Will I?
*shrug*

Accepting that elitism/populism\footnote{hard for me to classify the feeling without investigating more. I'm sure there's elements both of \say{if you haven't dedicated the effort} (elitism) along with the stated one of \say{you know so much but do nothing with it} (populist, for all that I'm not sure if that's entirely the right term for what I want. I feel like I see it as contrasting elitism often enough that I'm ok with keeping it, at least for now)}
is not a good enough reason to record an album, why else do I want to?

I've had friends ask to listen to my music on Spotify.
I have so many moral objections to Spotify as a company, but I don't know if they neatly overlap with my putting music on the platform.
Certainly if my goal is letting my friends listen to the music I've written, it's by far the past of least resistance. 
I know that more than a few of them refuse to install MP3s\footnote{which remains absolutely insane to me. Then again, I feel like the way that I interact with digital media is fundamentally different than a lot of my peers.
The fact that I'm an active writer and musician might have something to do with that.
I could get into another rant about how important I think creation is as a part of people's lives, but I've made that somewhere else on this blog, and I don't want to rehash that in the middle of another musing},
and so putting stuff on Spotify would make that accessible to them.

Doing things to please others is not a sufficient condition, nor is it even a necessary condition for doing something, for all that it's nice.\footnote{a dear friend who is pursuing a Ph. D. in Mathematics introduced me to the idea of necessary and sufficient conditions. Something can be necessary but not sufficient, such as eating to stay alive. Of course, much is neither. I'm sure that there's something in the category of sufficient but not necessary, for all that I'm not willing to put forth the effort to figure out an example}
Especially since I know I have a tendency towards people pleasing\footnote{for all that there are so many people who would laugh riotously at hearing that claim}, it's worth seeing if that's a good reason.

Let's take a step back.
Rather than thinking about why I, personally, want to write an album, let's see if I can't find some reasons that a person somewhere could conceivably want to write an album.\footnote{last night a friend said that he really enjoyed the way that one of my Gospel reflections had a list that I explored each item in depth. It feels like it could be a generally good way of approaching things?}

\begin{itemize}
\item As mentioned above: feeling that they need to in order to comfortably call themselves a musician\footnote{that might be a better way of framing what I feel?}
\item Also as mentioned above, pressure from friends.\footnote{I use pressure much more liberally than most people do, I've realized.
I can either change my usage to match the rest of the world or try to be influential enough that the world's usage begins to match mine.
Given this musing, I'm certain that you can figure out which I'm planning on}
\item Monetary reasons.\footnote{as I look up reasons to write an album, this appears to be a lot of reasons coupled into one. I'm still leaving it as a single bullet point, for reasons that will be clear later}
\item They feel like the world needs to hear their voice
\item Demonstrate musical proficiency
\item It seems fun?\footnote{I feel like this always belongs as a reason on any list. Fun can, of course, be replaced with your choice of nearly any positive word, like useful, informative, etc. (yes I'm well aware that I treat fun and informative as basically the same motivation. Yes I know that's strange. No I don't care)}
\item Hold yourself accountable.\footnote{that feels like a good reason. Goals are great}
\end{itemize}

Let's investigate how they relate to me:
\begin{itemize}
\item Feelings of inferiority: they apply to me, but I don't know if that's a healthy thing. I can get over it either by deciding that I don't need to write an album to call myself a musician or by writing an album. Probably worth thinking more on whether I should move past that thought generally, for all that making an album will never make me less of a musician.\footnote{unless people think of it as selling out, but given that my expected profits are approximately 0 for the music, I cannot imagine that it will really be relevant to me}
\item Pressure: I think that this is a driving force to me. I'm almost certain that my friends were just being polite when they expressed interest in me recording an album.
Nonetheless, I still want them to be able to listen to what I've written.\footnote{or at least recorded/arranged. I'm more and more leaning towards throwing at least a single traditional or public domain (the difference between the two is murky, for all that I acknowledge it's academically important. As someone whose main goal is not paying royalties, they serve the same purpose)}
\item Monetary reasons: as demonstrated by my lack of monetization of the book I'm writing, despite explicit requests to set up a way for people to pay me, this is not a motivation for me.
Honestly, I am lucky enough that I can afford to treat art as a thing I do entirely separate from the monetary compensation it could produce.
I don't think artists necessarily make better work when they don't have a financial incentive, but I think that I, at least, do.
\item I don't really think that the world needs to hear my musical voice.
I'm getting a Ph. D. in part because I do think that I have something to add to the world.
I'm writing a book, in some small part, at least, for the same reason.
I don't really feel like this idea resonates with me at all, which is interesting.
The monetary one at least struck me as wrong.
This just strikes me as irrelevant
\item Demonstration of proficiency: I can see how this seems similar to the first reason, but I can see a difference.\footnote{I don't plan to explore the difference. If you cannot see a difference, please feel free to message me through any platform we share (also hi! Thanks for reading. I appreciate you)}
I think that there's something to this.
I do love type two fun in retrospect.\footnote{I've seen references to fun as either type one, which is fun because you're doing it, and type two, which is fun because you have done it. I know I've mused about that before. Also, type two fun is explicitly supposed to be fun primarily in retrospect}
That's at least one reason to record an album, I suppose.
It would be cool to be able to point to it as something that I've written and composed, if only because it's a cool flex.
\item Fun/other positive terms: yeah ok, as discussed above, this is a motivation.
I think that I would learn a lot about myself from writing an album.
I think that I would learn a lot more in recording one.
I think that I would learn a ton more in mastering and mixing and releasing.
I think that I've gotten a thick enough skin from my web serial to be comfortable putting out my music for the wider world.
\item Accountability: I think there's also something about this to me.
When the band I play in\footnote{first time typing this said \say{when I was in a band} but that feels too pessimistic. I have hope that we will return to being a band sometime soon} was more active, that was a reason to compose.
Now that I don't even go to open mics\footnote{in large part because I want to sleep earlier than they allow}, I don't have much reason to do music.
Having an explicit goal would, hopefully, at least, keep me accountable.
\end{itemize}

Ok, so I for sure still want to write an album.
This past month\footnote{I know that November isn't over, but it's basically over, and I need to start planning for the next month} has been really focused on my ability to output large quantities of prose.
I think that it is an important skill to develop, but I think that producing better prose is also important.
In an ideal world, I think that I, at least\footnote{I don't know how universal this experience is, and I don't want to extrapolate too terribly much} would benefit from alternating between phases where work on getting more words out faster and getting the words which come out to be better faster.
The better my first drafts are, the less that I will need to revise them, which means the less time that I end up needing to spend on a project, which means that I can get through more of the infinite projects I want to write.

For all that I've already decided that next month I'll really explore the craft of prose more, I think that it could also be healthy and fun to get back into the practice of poetry.
I don't think that jumping right back into composing songs would be my best bet, because that's such a jump to go from nothing to full songs.
Instead, I think that a month of intentional poetry practice could be really helpful and healthy.
I probably want to do some sort of metered form, but I have a few days to decide what, exactly, I want that form to look like.\footnote{I also have a friend I've recently learned got a degree in poetry, so may ask that friend (my avoidance of any identifying information is not constant, I'm realizing. I wonder why that is) for any advice on reading for poetry.
Crud. I should also make an effort to read good poetry if I'm going to try to write better poetry.
Eh, I can't help but imagine that better prose leads to better poetry and vice versa.
Much like this blog and the books I write help each other and my academic writing, I imagine any words on a page make me better at some part of the process.}
So that's December sorted.

I have vague memories of the goals I set for myself at the start of the year.\footnote{I'm going to take a minute and read through them. If you want to \href{reflection-2022.html}{click here}}
Hmm, it appears as though they were more implicit than explicit.
Oh well! 

I remember that one goal was to do species counterpoint every day.
I still would like to do that.
Given that I've been able to find the time to write at least 1800 words\footnote{the fewest words I've written this month, which is an admittedly kind of small number I suppose} a day every day today, I can certainly make the time to do a quick little counterpoint exercise.
If I keep up the poetry into January, it's not inconceivable that the process of daily composing music and writing poetry will cause me to explicitly create song.
If not, however, I always have the option of making that a goal for February.

From there, I would have like 6 months to record the album, which means really that I need to dedicate a week or two to doing nothing but it.\footnote{or, more realistically, an afternoon a week a few months}
That really does feel somewhat realistic.
What else did I plan to do this past year?
Even though it's not the end of the year, I still have time to make the resolutions work.

I have yet to take an improv class.
I have one more chance to make it happen.\footnote{the class I was going to take is a monthly thing. I've fallen out of touch with the friend who I was going to do it with, sadly}

Huh, otherwise, I think I've generally done ok with keeping up on the goals.
Next year I do want to be better at getting the sleep I need.
Very rarely are activities worth losing a night of sleep over, and I more and more realize that every day.

Well, while this post went on a long and rambly course, I guess I did reaffirm the fact that I want to make an album.
I just need to write songs, compose music, and record and master it.
I don't need it to be perfect, just serviceable.

Daily Reflection:
\begin{itemize}
\item Did I write 1700 words for NaNoWriMo? I did! Yesterday too. I'm so close to the finish line, which is cool.
I still don't really like the book, which is ok. I will finish it, and then I never have to look at it again.
\item Did I write a chapter of Jeb? I finished plotting out the rest of the arc, and I sent the arc to the beta reader.
Beta approved it, so now I just need to write it.\footnote{I've been working on this blog post for around an hour, if not longer, so I don't really have the time left before sleep to do it now}
Edit: I realized that, due to the frankly nonsensical way that I structured my daily word goals, every extra word I write today drops my goals for tomorrow by at least a word.
That's probably worth doing, if only because I'm tired of playing catch-up, and setting lower goals is one way to do that.
\item Did I blog? I have! Look at this monstrosity. Gaze upon the madness.\footnote{this is one of those musings that both requires a revision and that I don't know how I could revise.
The path is so meandering that more or less I would just need to cut out almost all the content to keep it on theme with the title \say{On songwriting} (which I did pick ahead of time this time.
The way that I pick titles is probably something that's worth musing on another time, but not now.)}
\item Did I stretch? I went lifting far too early today.
It was fun, and it absolutely set me up for success today.
However, I know without a doubt that I will regret not stretching if I do not.
I'll do it after this post.
\item Am I doing better at prayer than a rushed and thoughtless rosary? I made it through a decade yesterday. I have higher hopes for today.
\item Am I doing a good job writing letters to friends? No! I wrote the number of letters that I had hoped to this month, though, which is nice to realize.
\end{itemize}\end{document}
